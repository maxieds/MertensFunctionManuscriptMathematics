\documentclass[11pt,reqno]{amsart} 

\usepackage{amsthm,amsfonts,amscd,amsmath}
\usepackage[hidelinks]{hyperref} 
\usepackage{url}
\usepackage{xcolor}
\hypersetup{
    colorlinks,
    linkcolor={red!50!black},
    citecolor={blue!50!black},
    urlcolor={blue!80!black}
}

\usepackage[normalem]{ulem}
\usepackage{graphicx} 
\usepackage{datetime} 
\usepackage{cancel}
\usepackage{caption,subcaption}
\captionsetup{format=hang,labelfont={bf},textfont={small,it}} 
\numberwithin{figure}{section}
\numberwithin{table}{section}

\usepackage{stmaryrd} 
\usepackage{framed} 

\let\citep\cite

\newcommand{\undersetbrace}[2]{\underset{\displaystyle{#1}}{\underbrace{#2}}}

\newcommand{\gkpSI}[2]{\ensuremath{\genfrac{\lbrack}{\rbrack}{0pt}{}{#1}{#2}}} 
\newcommand{\gkpSII}[2]{\ensuremath{\genfrac{\lbrace}{\rbrace}{0pt}{}{#1}{#2}}}

\newcommand{\cf}{\textit{cf.\ }} 
\newcommand{\Iverson}[1]{\ensuremath{\left[#1\right]_{\delta}}} 

\newcommand{\floor}[1]{\left\lfloor #1 \right\rfloor} 
\newcommand{\ceiling}[1]{\left\lceil #1 \right\rceil} 
\newcommand{\e}[1]{e\left(#1\right)} 

\usepackage{upgreek}

\renewcommand{\chi}{\upchi}

\DeclareMathOperator{\DGF}{DGF} 
\DeclareMathOperator{\ds}{ds} 
\DeclareMathOperator{\Id}{Id}
\DeclareMathOperator{\fg}{fg}
\DeclareMathOperator{\Div}{div}
\DeclareMathOperator{\rpp}{rpp}
\DeclareMathOperator{\sgn}{sgn}

\renewcommand{\Re}{\operatorname{Re}}
\renewcommand{\Im}{\operatorname{Im}}

\title[Sign Smoothing Convolutions]{
       Sign Smoothing Convolutions of the Dirichlet Inverses of Arithmetic Functions 
} 
\author[H. Mousavi and M. D. Schmidt]{
         Hamed Mousavi \\ 
         Maxie Dion Schmidt 
} 

\email{\href{mailto:hmousavi6@gatech.edu}{hmousavi6@gatech.edu} \\ 
       \href{mailto:mschmidt34@gatech.edu}{mschmidt34@gatech.edu}
}
\address{Georgia Institute of Technology \\ 
         School of Mathematics \\ 
         117 Skiles Building \\ 
         686 Cherry Street NW \\ 
         Atlanta, GA 30332 \\ 
         USA
} 

\date{\today} 

\keywords{Arithmetic functions; Dirichlet inverse; Dirichlet convolution; 
          Dirichlet series; sign changes of arithmetic function; 
          smoothing transformations and sums; discrete convolution. }
\subjclass[2010]{11A25; 11N64; 11N56. }


\allowdisplaybreaks 

\theoremstyle{plain} 
\newtheorem{theorem}{Theorem}
\newtheorem{conjecture}[theorem]{Conjecture}
\newtheorem{claim}[theorem]{Claim}
\newtheorem{prop}[theorem]{Proposition}
\newtheorem{lemma}[theorem]{Lemma}
\newtheorem{cor}[theorem]{Corollary}
\numberwithin{theorem}{section}

\theoremstyle{definition} 
\newtheorem{example}[theorem]{Example}
\newtheorem{remark}[theorem]{Remark}
\newtheorem{definition}[theorem]{Definition}
\newtheorem{notation}[theorem]{Notation}
\newtheorem{question}[theorem]{Question}
\newtheorem{heuristic}[theorem]{Heuristic}

\renewcommand{\arraystretch}{1.25} 

\setlength{\textwidth}{7in} 
\setlength{\evensidemargin}{-0.25in} 
\setlength{\oddsidemargin}{-0.25in} 

\begin{document} 

\begin{abstract} 
Sign changes in sums of arithmetic functions and their inverses are a subtle topic 
with room to grow new results. Suppose that $S_f(x) := \sum_{n \leq x} f(n)$ is the 
summatory function of some arithmetic function $f$ such that $f(1) \neq 1$. There are known 
lower bounds on the limiting growth of $V(S_f, Y)$ -- the number of sign changes of $S_f(y)$ 
on the interval $y \in (0, Y]$ as $Y \rightarrow \infty$. We observe a partition theoretic 
smoothing discrete convolution of the local oscillatory properties of sums of the 
Dirichlet inverse of $f$, $S_{f^{-1}}(x)$, which leads to a sequence of convolution sums which 
are eventually constant in sign. We investigate exponential function sign smoothing convolutions 
and then generalize our results to prove more optimal sign smoothing weight functions for any 
fixed Dirichlet invertible $f$. We give applications a plenty of these sign smoothing convolution 
sums. 
\end{abstract}

\maketitle

\section{Introduction} 

\subsection{Dirichlet convolutions and Dirichlet inverse functions} 

The sign changes of an arithmetic function $f$ are often considered in applications 
where we must estimate the growth of sums depending on $f$. For any fixed $f$, we 
define its summatory function for all positive integers $x \geq 1$ by 
\[
S_f(x) := \sum_{n \leq x} f(n). 
\]
Given any two arithmetic functions $f$ and $g$, we define their 
\emph{Dirichlet convolution}, $f \ast g$, to be the divisor sum 
\[
(f \ast g)(n) := \sum_{d|n} f(d) g\left(\frac{n}{d}\right), \forall n \geq 1. 
\] 
The multiplicative inverse with respect to Dirichlet convolution is defined by 
$\varepsilon(n) \equiv \delta_{n,1}$ so that $f \ast \varepsilon = \varepsilon \ast f = f$ for 
any arithmetic $f$. If $f(1) \neq 1$, then it is Dirichlet invertible. That is, there is another 
arithmetic function $f^{-1}(n)$ such that $f \ast f^{-1} = f^{-1} \ast f = \varepsilon$. 
Moreover, the function $f^{-1}$ is unique when it exists and satisfies the recursive formula 
\[
f^{-1}(n) = \begin{cases} 
     \frac{1}{f(1)}, & \text{if $n = 1$; } \\ 
     -\frac{1}{f(1)} \times \sum\limits_{\substack{d|n \\ d>1}} 
     f(d) f^{-1}\left(\frac{n}{d}\right), & \text{if $n \geq 2$. } 
     \end{cases} 
\]
We find that the signedness of $f^{-1}$ is dictated, or prescribed, by the local 
sign change patterns of $f$. 
Because such locally unpredictable signage is crucial to the understanding of many 
classical problems and applications, we state the next proposition to 
clarify the situation for special classes of nicely behaved invertible arithmetic functions 
$f \geq 0$. 

\begin{prop}[The Sequence of Signs of the Dirichlet Inverse] 
\label{prop_SeqOfSignsOfDirInv_SpCases_stmt_intro_v1} 
Suppose that $f(1) := c_f \neq 0$ and that $f(n) \geq 0$ for all $n \geq 2$. Then 
\begin{itemize} 

\item[1.] If $f$ is completely multiplicative then $\sgn(f^{-1}(n)) = \mu(n)$; 
\item[2.] If $f$ is multiplicative and $c_f \geq 1$, then 
          $\sgn(f^{-1}(n)) = \lambda_{\ast}(n) = (-1)^{\omega(n)}$; 
\item[3.] If $f$ is additive and $c_f \geq 1$, then 
          $\sgn((f+1)^{-1}(n)) = \lambda(n) = (-1)^{\Omega(n)}$. 

\end{itemize} 
In the previous equations we denote by $\omega(n) := \sum_{p|n} 1$ and 
$\Omega(n) := \sum_{p^{\alpha} || n} \alpha$ the 
strongly and completely additive functions that count the number of distinct prime factors 
of $n$ with and without counting multiplicity, respectively. 
\end{prop} 

Other formulas for $f^{-1}$ when $f$ is any Dirichlet invertible arithmetic function provide limited 
insight into distributions of the signs of the inverse function over $n \geq 1$. 
A partition theoretic motivation for expressing the Dirichlet inverse of any $f$ such that 
$f(1) \neq 0$ provides that for $n > 1$: 
\begin{equation} 
\label{eqn_fInvDirInv_PartitionsFormula} 
f^{-1}(n) = \sum_{k=1}^{\Omega(n)} (-1)^k \left\{ 
     \sum_{{\lambda_1+2\lambda_2+\cdots+k\lambda_k=n} \atop {\lambda_1, \lambda_2, \ldots, \lambda_k | n}} 
     \frac{(\lambda_1+\lambda_2+\cdots+\lambda_k)!}{1! 2! \cdots k!} 
     f(\lambda_1) f(\lambda_2)^2 \cdots f(\lambda_k)^k\right\}. 
\end{equation} 
Let the $m$-fold convolution of an arithmetic function $g$ with itself (i.e., convolve 
$g$ with itself $m$ times in a row at $n$) be denoted by $[g]_{\ast_m}$. Then 
Mousavi and Schmidt proved that \cite{MOUSAVI-SCHMIDT-2019} 
\begin{equation} 
\label{eqn_} 
f^{-1}(n) = \frac{\varepsilon(n)}{f(1)} + 
     \sum_{j=0}^{\left\lfloor \frac{\Omega(n)}{2} \right\rfloor} \left( 
     [f-f(1)\cdot\varepsilon]_{\ast_{2j+1}}(n) - f(1) \times 
     [f-f(1)\cdot\varepsilon]_{\ast_{2j}}(n)
     \right) \frac{1}{f(1)^{2j+1}}. 
\end{equation} 
Since we can expand $m$-fold convolutions of a sum of $k$ arithmetic functions using the 
multinomial (with $k$ terms at $m$) theorem as 
\[
[f_1+f_2+\cdots+f_k]_{\ast_m} = \sum_{\substack{i_1+i_2+\cdots+i_k = m \\ i_1,i_2,\ldots,i_k \geq 0}} 
     \binom{m}{i_1,i_2,\ldots,i_k} [f_1]_{\ast_{i_1}} [f_2]_{\ast_{i_2}} \cdots 
     [f_k]_{\ast_{i_k}}, 
\]
this last formula for $f^{-1}$ leads to another set of insights we can apply in expressing the 
sign of $f^{-1}(n)$. 

\subsection{Local sign changes of an arithmetic function} 

The \emph{Dirichlet generating function} (or DGF) of 
an arithmetic function $f$ is defined as follows for all 
$s := \sigma+\imath t \in \mathbb{C}$ such that the 
following sequence converges: 
\[
D_f(s) := \sum_{n \geq 1} \frac{f(n)}{n^s} = 
     \sum_{n \geq 1} \frac{f(n) \left[\cos(t \log n) + \imath \sin(t \log n)\right]}{n^{\sigma}}, 
     \Re(s) > \sigma_{c,f}. 
\]
Provided that $f$ is multiplicative, we have the \emph{Euler product} representation of the DGF 
of $f$ given by the prime-indexed product
\[
D_f(s) = \prod_{p \in \mathbb{P}} \left(1 + \sum_{r=1}^{\infty} \frac{f(p^r)}{p^{rs}}\right), 
     \Re(s) > \sigma_{c,f}. 
\]
The DGF of $f$ is related to the Mellin transform of its summatory function 
$S_f(x) := \sum_{n \leq x} f(n)$ by 
\[
D_f(s) = s \cdot \int_1^{\infty} \frac{S_f(x)}{x^{s+1}} dx, \Re(s) > \sigma_{a,f}, 
\]
where $\sigma_{a,f}$ is the abscissa of absolute convergence of the DGF $D_f(s)$. 
An inversion formula for recovering the coefficients $f(n)$ of $n^{-s}$ 
from a given DGF is stated as follows \cite[\S 11.11]{APOSTOL-ANUMT}: 
\[
f(x) = \lim_{T \rightarrow \infty} \frac{1}{2T} \int_{-T}^{T} x^{\sigma+\imath t} 
     D_f(\sigma + \imath t) dt, \forall x \in \mathbb{Z}^{+}; 
     \forall \sigma > \sigma_{a,f}. 
\]
One heuristic that recurs in applications is that if an 
invertible $f \geq 0$ is non-negative for all $n \geq 1$, then the corresponding 
sequence of sign changes for its inverse $f^{-1}$ is oscillatory and typically 
hard to predict. The same is true for invertible non-negative integer matrices: 
the corresponding inverse matrices as a general measure tend to display 
semi-random, highly oscillatory, and variably signed behavior. 
\nocite{DIRINVFUNC-GROWTH-PROPS}

We have several existing results that characterize the expected number of sign changes of 
well enough behaved $f$ on increasingly large intervals of consecutive integers. 
Let $V(f, Y)$ denote the number of sign changes of $f$ on the interval $(0, Y]$ for 
real $Y > 0$: 
\[
V(f, Y) := \sup \left\{N: \exists \{x_i\}_{i=1}^N, 0<x_1<\cdots<x_N \leq Y, 
     f(x_i) \neq 0, \sgn(f(x_i)) \neq \sgn(f(x_{i+1})), \forall 1 \leq i < N\right\}. 
\]
It is nown that the analytic properties, poles, and zeros of the 
DGF $D_f(s)$ of $f$ provide key insights into the 
sign changes of these functions \cite{OSCPROPS-ARITHFUNCSI}. 
For example, if the DGF of $f$ is analytic on some half-plane, subject to certain 
restrictions, then Landau showed in 1905 that 
the summatory function of $f$, $S_f(x)$, changes signs infinitely often as we let 
$x$ tend to infinity. We also have the next 
theorem that extends Landau's and which provides a more precise minimal 
statement concerning the frequency of the sign changes of $S_f(x)$. 

\begin{theorem}[P\'olya] 
Suppose that $S_f(x)$ is real-valued for all $x \geq x_0$, and define the function 
$\hat{F}_f(s)$ by the Mellin transform at $-s$ as 
\[
\hat{F}_f(s) := \int_{x_0}^{\infty} \frac{S_f(x)}{x^{s+1}} dx. 
\]
Suppose that $\hat{F}_f(s)$ is analytic for all $\Re(s) > \theta$, but is not analytic 
in any half-plane $\Re(s) > \theta - \varepsilon$ for $\varepsilon > 0$. 
Furthermore, suppose that $\hat{F}_f(s)$ is meromorphic in some half-plane 
$\Re(s) > \theta - c_0$ for some $c_0 > 0$. Let 
\[
\gamma_f := \begin{cases} 
     \inf \{|t|: \hat{F}_f(s) \text{\ is not analytic at\ } s = \theta+\imath t\}, & \text{\rm
     if $f$ is not analytic at $\Re(s) = \theta$; } \\ 
     \infty, & \text{\rm otherwise.}
     \end{cases}. 
 \]
Then 
 \[
 \limsup_{Y \rightarrow \infty} \left\{\frac{V(S_f, Y)}{\log Y}\right\} \geq \frac{\gamma_f}{\pi}. 
\]
\end{theorem} 

\subsection{Motivation for a new approach to determining the signs of $f^{-1}(n)$ and $S_f(x)$} 

We have experimentally observed an interesting new trend of so-called 
\emph{sign smoothing transformations} of certain signed integer sequences under invertible 
discrete convolution based encodings by special partition functions. 
Let the partition functions $p_1(n)$ and $p_2(n)$ be defined for integers $n \geq 0$ 
as the coefficients of the following generating functions: 
\begin{align} 
p_1(n) & := [q^n] \prod_{m \geq 1} (1+q^m) \\ 
\notag 
       & \phantom{:} = [q^n]\left(1+q+q^2+2q^3+2q^4+3q^5+4q^6+5q^7+6q^8+8q^9+10q^{10}+ 
       12q^{11}+15q^{12}+\cdots\right) \\ 
\notag 
p_2(n) & := [q^n] \prod_{m \geq 1} (1+q^m)^{-1} \\ 
\notag 
       & \phantom{:} = [q^n]\left(1-q-q^3+q^4-q^5+q^6-q^7+2q^8-2q^9 + 
       2q^{10}-2q^{11}+3q^{12} + \cdots \right)
\end{align} 
By convention, we denote $\widehat{p}_2(n) = |p_2(n)| = (-1)^n p_2(n)$. We can apply the 
circle method to the $q$-series generating functions of these sequences to obtain 
limiting asymptotic formulas for the partition numbers of the form 
the following limiting asymptotics for these two functions \cite{ANALYTIC-COMB}: 
%\footnote{ 
%     An application of the circle method is successful because we can write the product 
%     generating function for these sequences as a ratio of the Dedekind eta function, which 
%     is modular. Namely, the sequence $p_1(n)$ is seen to enumerate the number of partitions of 
%     $n$ into odd parts, so that 
%     \[
%     \sum_{n \geq 0} p_1(n) q^n = \prod_{k \geq 1} (1-q^{2k+1})^{-1} = e^{\imath\pi\tau/6} 
%          \frac{\eta(2\tau)}{\eta(\tau)}. 
%     \]
%} 
\begin{align} 
\label{eqn_PFuncsP1P2n_asymptotics_stmt_v1} 
p_1(n) & \sim \frac{3^{3/4}}{12 \cdot n^{3/4}} \exp\left(\pi\sqrt{\frac{n}{3}}\right) \\ 
\notag
\widehat{p}_2(n) & \sim \frac{1}{2 \cdot 24^{1/4} n^{3/4}} \exp\left(\pi\sqrt{\frac{n}{6}}\right). 
\end{align} 
We define two invertible transformations, or encodings, that are respective inverses of one another 
on any fixed arithmetic $f$ as the discrete 
convolution sums given by 
\begin{align} 
\label{eqn_TfEncodingsSpecialCasePFunc_examples_v1} 
s_1[f](n) & := \sum_{j=1}^{n} f(j) p_1(n-j) \\ 
\notag 
s_2[f](n) & := \sum_{j=1}^{n} f(j) p_2(n-j). 
\end{align} 
We say that a arithmetic sequence $\{f(n)\}_{n \geq 1}$ has 
\emph{property $\mathcal{P}_1$} at $N$ if the sign of $s_1[f](n)$ is constant for all $n \geq N$. 
Likewise, we say that the sequence has \emph{property $\mathcal{P}_2$} at $N$ if the sign of 
$s_2[f](n)$ alternates for all $n \geq N$. We define 
\begin{align*} 
M_{f,1} & := \sup \left\{n \geq 1: f \text{\ does not have property\ } \mathcal{P}_1 \text{\ at\ } n\right\} \\ 
M_{f,2} & := \sup \left\{n \geq 1: f \text{\ does not have property\ } \mathcal{P}_2 \text{\ at\ } n\right\}. 
\end{align*} 
The characteristic limiting behavior we observe in the 
encoding transformations in \eqref{eqn_TfEncodingsSpecialCasePFunc_examples_v1} 
is typified by the next conjectured properties based on empirical observation and 
computational data sets. 

\begin{conjecture}[Sign Smoothing Convolution Operators] 
\label{conj_MainTheorem_Stmt_v1} 
For any arithmetic function $f$ which is non-vanishing on the positive integers, 
both $M_{f,1}$ and $M_{f,2}$ are finite. Moreover, provided that $f(n) \ll p_1(n)$, 
we have $\lVert s_1[f](n)\rVert \rightarrow +\infty$ as $n \rightarrow \infty$. If 
$f(n) \ll p_2(n)$, then 
$$\lVert s_2[f](n) \rVert \xrightarrow{n \rightarrow \infty} +\infty.$$ 
\end{conjecture} 

The truth of 
Conjecture \ref{conj_MainTheorem_Stmt_v1} 
implies that if an arithmetic function $f$ satisfies $\mathcal{P}_1$ at some finite $N \geq 1$, then 
for all sufficiently large $n \geq N$, the signed magnitude of the real part of the transformation 
$s_1[f](n)$ tends to one side of the real line or the other. A similar obervation is made for 
arithmetic functions $f$ satisfying property $\mathcal{P}_2$ at some finite $N \in \mathbb{N}$. 

\subsection{Generalizations} 

Given the essentially exponential nature of the limiting asymptotics in 
\eqref{eqn_PFuncsP1P2n_asymptotics_stmt_v1}, we can reconcile the expected behavior 
from Conjecture \ref{conj_MainTheorem_Stmt_v1} with a more general phenomenon that 
characterizes a large class of more general \emph{exponential sign smoothing operators}. 
Let $\zeta_m := \exp(2\pi\imath / m)$ denote the primitive $m^{th}$ root of unity. 
We define the next two classes of invertible transformations in terms of the 
real parameters $m,k$ for any integers $n \geq 1$: 
\begin{align} 
\label{eqn_sfn_tfn_SignSmoothingByExpFuncSumsFuncs_intro_v1}
s_{m,k}[f](n) & := \sum_{j=1}^{n} f(j) \zeta_m^{n-j} \exp\left(\pi \sqrt{k(n-j)}\right), 
     \forall n \geq 1; m \in \mathbb{Z}^{+}; k \in (0, \infty); \\ 
\notag 
t_{m,k}[f](n) & := \sum_{j|n} f(j) \zeta_m^{n-j} \exp\left(\pi \sqrt{k(n-j)}\right), 
     \forall n \geq 1; m \in \mathbb{Z}^{+}; k \in (0, \infty).
\end{align} 
We say that a arithmetic sequence $\{f(n)\}_{n \geq 1}$ has 
\emph{property $\mathcal{P}_{1,m,k}$} at $N$ if the sign of $\Re\left\{s_{m,k}[f](n) \cdot \zeta_m^{-n}\right\}$ 
is constant for all $n \geq N$. Similarly, we say that $f$ has 
\emph{property $\mathcal{P}_{2,m,k}$} at $N$ if the sign of $\Im\left\{s_{m,k}[f](n) \cdot \zeta_m^{-n}\right\}$ 
is constant for all $n \geq N$. 
We define for $i := 1,2$ 
\begin{align*} 
M_{i,m,k}(f) & := \sup \left\{n \geq 1: f \text{\ does not have property\ } \mathcal{P}_{i,m,k} \text{\ at\ } n\right\}. 
\end{align*} 
The above definition similarly shows that if an arithmetic function $f$ 
satisfies $\mathcal{P}_{m,k}$ at some finite $N \geq 1$, then 
for all sufficiently large $n \geq N$, the signed magnitude of the real part of the transformation 
$s_{m,k}[f](n)$ tends to one side of the real line or the other. 
The definition does not provide the limiting signage of the 
transformation sequence even if the function $f$ satisfies property $\mathcal{P}_{m,k}$. 
We prove the following three main theorems in the next section of the article. 

\begin{theorem}[A Sign Smoothing Convolution Operator by Exponential Function Scaling] 
\label{theorem_MainTheorem_Stmt_v1} 
For any Dirichlet invertible arithmetic function $f$ which is non-vanishing on the positive integer, 
any $m \in \mathbb{Z}^{+}$, and any $k \in (0, \infty)$ such that 
$f(n) \ll \exp(\pi\sqrt{kn})$, $M_{1,m,k}(f^{-1})$ and $M_{2,m,k}(f^{-1})$ are finite. Moreover, 
\[
\lim_{n \rightarrow \infty} \left\lvert \Re\left\{\frac{s_{m,k}[f^{-1}](n)}{\zeta_m^n}\right\} \right\rvert = +\infty, 
     \quad\text{ and }\quad 
\lim_{n \rightarrow \infty} \left\lvert \Im\left\{\frac{s_{m,k}[f^{-1}](n)}{\zeta_m^n}\right\} \right\rvert = +\infty
\]
\end{theorem} 

\begin{theorem} 
\label{theorem_Intro_InitBounds} 
For any Dirichlet invertible $f$ which is non-vanishing on the positive integers, 
any $m \in \mathbb{Z}^{+}$, and any $k \in (0, \infty)$, we have that 
\begin{align}
\tag{A} 
M_{1,m,k}(f^{-1}) & = TODO_{1,m,k}; \\ 
\tag{B} 
M_{2,m,k}(f^{-1}) & = TODO_{2,m,k}; \\ 
\tag{C} 
\limsup_{n \rightarrow \infty} \left(\sgn\left\{\Re\left[s_{m,k}[f^{-1}](n) \cdot \zeta_m^{-n}\right]\right\}\right) & = 
     TODO_{m,k}; \\ 
\tag{D} 
\limsup_{n \rightarrow \infty} \left(\sgn\left\{\Im\left[s_{m,k}[f^{-1}](n) \cdot \zeta_m^{-n}\right]\right\}\right) & = 
     TODO_{m,k}.
\end{align} 
\end{theorem}

\begin{theorem}[A Sign Smoothing Convolution Operator by Exponential Function Scaling] 
\label{theorem_MainTheorem_Stmt_v2} 
For any Dirichlet invertible arithmetic function $f$ which is non-vanishing on the positive integers, 
any $m \in \mathbb{Z}^{+}$, and any $k \in (0, \infty)$ such that 
$f(n) \ll \exp(\pi\sqrt{kn})$, the sign of 
$\Re\left[t_{m,k}[f^{-1}](n) \cdot \zeta_m^{-n}\right]$ is eventually constant. 
That is, there exists a finite $N_f \geq 1$ such that for all $n > N_f$, 
\[
\sgn\left\{\Re\left[t_{m,k}[f^{-1}](n) \zeta_m^{-n}\right]\right\} - 
     \sgn\left\{\Re\left[t_{m,k}[f^{-1}](n-1) \zeta_m^{1-n}\right]\right\} = 0. 
\]
\end{theorem} 

\section{Proofs of the theorems and key consequences} 

\subsection{Proofs of the main theorems stated in the introduction} 

\subsection{Some immediate corollaries} 

We can relate the sequence of $s_{1,k}[f](x)$ to the summatory function $S_f(x)$ of $f$ closely within 
some bounded error term. In particular, it is not difficult to prove that 
\[
S_f(x) = \frac{s_{1,k}[f](x)}{\exp(\pi\sqrt{kx})} + O\left(\sum_{j=1}^{x-1} S_f(j) e^{-\pi\sqrt{kx}}\right). 
\]
So provided that the growth rates of $S_f(x)$ are sub-polynomial (significantly sub-exponential), we have a good 
approximation to $S_f(x)$ when $x \gg 1$ is large. This leads us to the following corollary: 

\begin{cor}[A Sign Bias for General Summatory Functions]
For limiting large $x \gg 1$, the summatory function $S_f(x)$ has a sign bias towards 
\[
\limsup_{x \rightarrow \infty} \left(\sgn\left\{s_{1,k}[f](x)\right\}\right). 
\]
\end{cor}
\begin{proof}
\textbf{TODO ... } This is easy given the eventually constant sign theorems ... 
\end{proof} 

\subsection{Local optimality of the convolution weight functions} 

\begin{remark}[A Necessary Condition for Sign Smoothing Convolutions] 
We call $s_f[E]$ a \emph{sign smoothing convolution} if the sign of $s_f[E](x)$ is 
constant for all sufficiently large $x \gg 1$. 
Notice from the proof of the previous result that we have shown
\[
\mathcal{M}[s_f[E]](-s) = \frac{(1-s) D_f(s)}{s} \Gamma(s) \sin(\pi s) \left[ 
     E(0) + \int_1^{\infty} \frac{E(x)-E(x-1)}{x^{s+1}} dx 
     \right]. 
\]
Suppose that this Mellin transformation is analytic for $\Re(s) > \theta_0$, but is not 
analytic in any larger half-plane $\Re(s) > \theta_0-\varepsilon$ for $\varepsilon>0$. 
If $s_f[E](x)$ \emph{does not change sign} 
infinitely often as $x \rightarrow \infty$, then it must 
be the case that $\mathcal{M}[s_f[E]](-s)$ is not analytic at $s := \theta_0$ 
\citep[\cf Landau's Theorem]{OSCPROPS-ARITHFUNCSI}. 
\end{remark} 

\begin{prop}[Growth of Sign Smoothing Convolutions]
For any arithmetic function $f$ and positive (strictly increasing) functions $E$ on the non-negative reals, 
we define 
\[
s_f[E](n) := \sum_{j=1}^{n} f(j) E(n-j), \forall n \in \mathbb{Z}^{+}. 
\] 
We can assert conditions on the rate of growth of these sums given by the functions $\phi$ such 
that $$s_f[E](x) = \sum_{k \geq 0} \phi(k) \frac{(-x)^{k}}{k!}.$$ In particular, we have that 
for any $k > \sigma_{a,f}$, these functions are given by 
\[
\phi(k) = \frac{(1-k) D_f(k)}{k} \Gamma(k) \sin(\pi k) \left[ 
     E(0) + \int_1^{\infty} \frac{E(x)-E(x-1)}{x^{k+1}} dx 
     \right]. 
\] 
%Clearly, $\phi(0) = 0$ and for all $k \in [1, \floor{\sigma_{a,f}}] \cap \mathbb{Z}$, if we can 
%analytically continue $D_f(s)$ at $s = k$, then we can obtain formulas for these initial values of 
%$\phi$. 
\end{prop} 
\begin{proof} 
\textbf{TODO ... } Apply Ramanujan's Master Theorem for Mellin transforms ... 
\end{proof} 

\section{Applications} 

\subsection{Improving bounds on sign changes of arithmetic functions on short intervals} 

\subsection{Partition theoretic applications (smoothness, or roundness, etc.)} 

\section{Generalizations of the sign smoothing convolutions} 

For any $\mathcal{A}_n \subseteq \{1,2,\ldots,n\}$, and monotone increasing function 
$E: \mathbb{N} \rightarrow \mathbb{R}$ such that $f(n) \ll E(n)$, that 
\begin{align*} 
\sum_{j \in \mathcal{A}_n} f(j) E(n-j) & = \sum_{j \in \mathcal{A}_n} \left[ 
     1-2(V(f, j)-V(f, j-1))\right] |f|(j) E(n-j) \\ 
     & = \sum_{j \in \mathcal{A}_n} |f|(j) E(n-j) - 2 V(f, n) |f|(n) E(0) \\ 
     & \phantom{=\quad\ } + 
     2 \sum_{\substack{j \in \mathcal{A}_n \\ j < n}} V(f, j) \left[ 
     |f|(j+1) E(n-1-j) - |f|(j) E(n-j) 
     \right]. 
\end{align*} 
The definitions of $\mathcal{A}_n$ corresponded to 
$\{1 \leq d \leq n: d|n\}$ for the sums $t_{m,k}[f](n)$ and to 
$\{1,2,\ldots,n\}$ for the sums $s_{m,k}[f](n)$ 
we defined in \eqref{eqn_sfn_tfn_SignSmoothingByExpFuncSumsFuncs_intro_v1} of the 
previous subsection. 

We now arrive at a natural question to consider 
about sign smoothing convolutions with the previous eventually constant sign properties 
where the weight function (previously of 
exponentials to a square root power of the input) is in some sense ``optimal'' for the 
Dirichlet invertible function $f$ and its inverse $f^{-1}$. 
That is, what is the most natural choice of the sign smoothing weight function we use in our convolved 
sums that results in 
\begin{itemize} 
\item[1.] Optimal growth rates of the convolved sums (or otherwise nice properties to 
work with); and 
\item[2.] The resulting sequence of convolved sums is constant in sign for the maximal number of 
natural numbers possible? 
\end{itemize} 
Consider the following questions in particular: 

\begin{question}[A Natural Sign Smoothing Function for $f$]
Given that $f(n) = O(U_f(n))$ and $V(f, Y) = O(V_f(Y))$ -- and these upper bounds are \emph{tight} in so much as 
$\forall \varepsilon, \delta>0$, $\exists N, Y \geq 1$ such that $f(N) + \varepsilon > U_f(N)$ and 
$V(f, Y) + \delta > V_f(Y)$ -- 
what is the optimal (minimal) choice of 
monotone increasing functions $E_f, E_f^{\ast}$ such that the signs of 
\[
t_f[E^{\ast}](n) := \sum_{j|n} f(j) E_f^{\ast}(n/j), s_f[E](n) := \sum_{j=1}^{n} f(j) E_f(n-j), 
\]
are eventually constant? Can we choose the best possible functions $E_f, E_f^{\ast}$ so that the signs of the 
corresponding sequences above are constant for all $n \geq N_f$ with $N_f \geq 1$ minimal over the natural 
numbers? That is, given some function space $\mathcal{E}_f$, what is 
\[
E_f := \operatorname{argmin}\limits_{E \in \mathcal{E}_f} \left\{ 
     \min_{N \in \mathbb{Z}^{+}} \left[s_f[E](n+1) - s_f[E](n) = 0, \forall n \geq N
     \right]\right\}, 
\]
and is this function unique? 
\end{question} 

\subsection{Remarks about smoothing by partition functions with sub-exponential growth} 

Why do the functions $E_k(n) := \exp(\pi \sqrt{kn})$ seem to do universally so well for all functions $f$ 
we have tried (what is special about these forms)? Why do some partition function sequences work better than 
others (or not at all) with respect to preserving the property of sign smoothing of any arithmetic $f$? 

\section{Conclusions} 

\subsection{Summary} 

\subsection{Open questions and possible generalizations} 

\subsection{Working topics list} 

We also should consider the roles of the following properties: 
\begin{itemize} 

\item The role of local osciallations of an arithmetic function? 
\item Sign changes of $\nabla[f](n)$? 

\end{itemize} 

\subsection*{Acknowledgements} 


\begin{thebibliography}{10} 

\bibitem{APOSTOL-ANUMT} 
T. M. Apostol, \textit{Introduction to analytic number theory}, Springer, 1976. 

\bibitem{DIRINVFUNC-GROWTH-PROPS} 
F. Baustian and V. Bobkov, On asymptotic behavior of Dirichlet inverse, {\em ArXiv:math.NT/1903.12445} (2019). 

\bibitem{OSCPROPS-ARITHFUNCSI} 
J. Kaczorowski and J. Pintz, Oscillatory properties of arithmetical 
  functions I, \emph{Acta Math. Hung.} {\bf 48}, pp. 173--185, 
  1986. 

\bibitem{MULTFUNCS-INSHORTINTS} 
K. Matom\"aki and M. Radziwitt, Multiplicative functions in 
  short intervals, \emph{Annals of Math.}, Vol. 183, No. 3, 
  pp. 1015--1056 (2016). 

\bibitem{MOUSAVI-SCHMIDT-2019} 
H. Mousavi and M. D. Schmidt, Factorization theorems for relatively prime 
  divisor sums, GCD sums, and generalized {R}amanujan sums, 
  \emph{Ramanujan J.}, to appear (2019). 

\bibitem{NG-MERTENS} 
N. Ng, The distribution of the summatory function of the M\"obius function, 
  \emph{ArXiv:math.NT/0310381} (2008). 

\bibitem{SUMMABILITY-DIRCVLS} 
S. L. Segal, Summability by Dirichlet convolutions, 
  \emph{Proc. Camb. Phil. Soc.} \textbf{63}, 393 (1967). 

\end{thebibliography} 


\end{document} 
