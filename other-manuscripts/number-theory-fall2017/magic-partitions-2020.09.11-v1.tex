\documentclass[11pt,reqno]{amsart} 

\usepackage{amsthm,amsfonts,amscd,amsmath}
\usepackage[hidelinks]{hyperref} 
\usepackage{url}
\usepackage{xcolor}
\hypersetup{
    colorlinks,
    linkcolor={red!50!black},
    citecolor={blue!50!black},
    urlcolor={blue!80!black}
}

\usepackage[normalem]{ulem}
\usepackage{graphicx} 
\usepackage{datetime} 
\usepackage{cancel}
\usepackage{caption,subcaption}
\captionsetup{format=hang,labelfont={bf},textfont={small,it}} 
\numberwithin{figure}{section}
\numberwithin{table}{section}

\usepackage{stmaryrd} 
\usepackage{framed} 

\let\citep\cite

\newcommand{\undersetbrace}[2]{\underset{\displaystyle{#1}}{\underbrace{#2}}}

\newcommand{\gkpSI}[2]{\ensuremath{\genfrac{\lbrack}{\rbrack}{0pt}{}{#1}{#2}}} 
\newcommand{\gkpSII}[2]{\ensuremath{\genfrac{\lbrace}{\rbrace}{0pt}{}{#1}{#2}}}

\newcommand{\cf}{\textit{cf.\ }} 
\newcommand{\Iverson}[1]{\ensuremath{\left[#1\right]_{\delta}}} 

\newcommand{\floor}[1]{\left\lfloor #1 \right\rfloor} 
\newcommand{\ceiling}[1]{\left\lceil #1 \right\rceil} 
\newcommand{\e}[1]{e\left(#1\right)} 

\usepackage{upgreek}

\renewcommand{\chi}{\upchi}

\DeclareMathOperator{\DGF}{DGF} 
\DeclareMathOperator{\ds}{ds} 
\DeclareMathOperator{\Id}{Id}
\DeclareMathOperator{\fg}{fg}
\DeclareMathOperator{\Div}{div}
\DeclareMathOperator{\rpp}{rpp}
\DeclareMathOperator{\sgn}{sgn}

\renewcommand{\Re}{\operatorname{Re}}
\renewcommand{\Im}{\operatorname{Im}}

\title[Sign Smoothing Convolutions]{
       Sign Smoothing Convolutions of the Dirichlet Inverses of Arithmetic Functions 
} 
\author[H. Mousavi and M. D. Schmidt]{
         Hamed Mousavi \\ 
         Maxie D. Schmidt 
} 

\email{\href{mailto:hmousavi6@gatech.edu}{hmousavi6@gatech.edu} \\ 
     \href{mailto:maxieds@gmail.com}{maxieds@gmail.com} \\ 
       \href{mailto:mschmidt34@gatech.edu}{mschmidt34@gatech.edu}
}
\address{Georgia Institute of Technology \\ 
         School of Mathematics \\ 
         117 Skiles Building \\ 
         686 Cherry Street NW \\ 
         Atlanta, GA 30332 \\ 
         USA
} 

\date{\today} 

\keywords{Arithmetic functions; Dirichlet inverse; Dirichlet convolution; 
          Dirichlet series; sign changes of arithmetic function; 
          smoothing transformations and sums; discrete convolution. }
\subjclass[2010]{11A25; 11N64; 11N56. }


\allowdisplaybreaks 

\theoremstyle{plain} 
\newtheorem{theorem}{Theorem}
\newtheorem{conjecture}[theorem]{Conjecture}
\newtheorem{claim}[theorem]{Claim}
\newtheorem{prop}[theorem]{Proposition}
\newtheorem{lemma}[theorem]{Lemma}
\newtheorem{cor}[theorem]{Corollary}
\numberwithin{theorem}{section}

\theoremstyle{definition} 
\newtheorem{example}[theorem]{Example}
\newtheorem{remark}[theorem]{Remark}
\newtheorem{definition}[theorem]{Definition}
\newtheorem{notation}[theorem]{Notation}
\newtheorem{question}[theorem]{Question}
\newtheorem{heuristic}[theorem]{Heuristic}

\renewcommand{\arraystretch}{1.25} 

\setlength{\textwidth}{7in} 
\setlength{\evensidemargin}{-0.25in} 
\setlength{\oddsidemargin}{-0.25in} 

\begin{document} 

\begin{abstract} 
Sign changes in sums of arithmetic functions and their inverses are a subtle topic 
with room to grow new results. Suppose that $S_f(x) := \sum_{n \leq x} f(n)$ is the 
summatory function of some arithmetic function $f$ such that $f(1) \neq 1$. There are known 
lower bounds on the limiting growth of $V(S_f, Y)$ -- the number of sign changes of $S_f(y)$ 
on the interval $y \in (0, Y]$ as $Y \rightarrow \infty$. We observe a partition theoretic 
smoothing discrete convolution of the local oscillatory properties of sums of the 
Dirichlet inverse of $f$, $S_{f^{-1}}(x)$, which leads to a sequence of convolution sums which 
are eventually constant in sign. We investigate exponential function sign smoothing convolutions 
and then generalize our results to prove more optimal sign smoothing weight functions for any 
fixed Dirichlet invertible $f$. We give applications a plenty of these sign smoothing convolution 
sums. 
\end{abstract}

\maketitle

\section{Introduction} 

\subsection{Dirichlet convolutions and Dirichlet inverse functions} 

The sign changes of an arithmetic function $f$ are often considered in applications 
where we must estimate the growth of sums depending on $f$. For any fixed $f$, we 
define its summatory function for all positive integers $x \geq 1$ by 
\[
S_f(x) := \sum_{n \leq x} f(n). 
\]
Given any two arithmetic functions $f$ and $g$, we define their 
\emph{Dirichlet convolution}, $f \ast g$, to be the divisor sum 
\[
(f \ast g)(n) := \sum_{d|n} f(d) g\left(\frac{n}{d}\right), \forall n \geq 1. 
\] 
The multiplicative inverse with respect to Dirichlet convolution is defined by 
$\varepsilon(n) \equiv \delta_{n,1}$ so that $f \ast \varepsilon = \varepsilon \ast f = f$ for 
any arithmetic $f$. If $f(1) \neq 1$, then it is Dirichlet invertible. That is, there is another 
arithmetic function $f^{-1}(n)$ such that $f \ast f^{-1} = f^{-1} \ast f = \varepsilon$. 
We find that the signedness of $f^{-1}$ is dictated, or prescribed, by the local 
sign change patterns of $f$. 
Because such locally unpredictable signage is crucial to the understanding of many 
classical problems and applications, we state the next proposition to 
clarify the situation for a class of nicely behaved invertible arithmetic $f \geq 0$. 

\begin{prop}[The Sequence of Signs of the Dirichlet Inverse] 
\label{prop_SeqOfSignsOfDirInv_SpCases_stmt_intro_v1} 
Suppose that $f(1) := c_f \neq 0$ and that $f(n) \geq 0$ for all $n \geq 2$. Then 
\begin{itemize} 

\item[1.] If $f$ is completely multiplicative then $\sgn(f^{-1}(n)) = \mu(n)$; 
\item[2.] If $c_f = 1$, then $\sgn(f^{-1}(n)) = \lambda(n) = (-1)^{\Omega(n)}$; 
\item[3.] If $c_f < 1$, then $\sgn(f^{-1}(n)) = -1$. 

\end{itemize} 
\end{prop} 

The class of non-negative invertible $f$ covered by the formulas above tend to have 
nicer, more manageable summatory functions as well. For example, consider that 
whenever $f(n) \geq 0$ for all $n \geq 1$, we have that its summatory function 
satsfies predictable properties under Riemann differentiation and integration 
\cite[\cf \S 13]{APOSTOL-ANUMT}. 

\begin{definition}[Dirichlet Generating Functions, or DGFs, of an Arithmetic Function] 
Just as the generating function of many combinatorially relevant sequences encodes 
information about the form and properties of the underlying sequence, we have another 
notion of a generating function, the \emph{Dirichlet generating function}, or DGF, of 
an arithmetic function $f$ which is defined for all 
$s := \sigma+\imath t \in \mathbb{C}$ such that the 
following sequence converges: 
\[
D_f(s) := \sum_{n \geq 1} \frac{f(n)}{n^s} = 
     \sum_{n \geq 1} \frac{f(n) \left[\cos(t \log n) + \imath \sin(t \log n)\right]}{n^s}, 
     \Re(s) > \sigma_{c,f}. 
\]
Provided that $f$ is multiplicative, we have the \emph{Euler product} representation of the DGF 
of $f$ given by the prime-indexed product
\[
D_f(s) = \prod_{p \in \mathbb{P}} \left(1 + \sum_{r=1}^{\infty} \frac{f(p^r)}{p^{rs}}\right), 
     \Re(s) > \sigma_{c,f}. 
\]
The DGF of $f$ is related to the Mellin transform of its summatory function at 
$-s$ by 
\[
D_f(s) = s \cdot \int_1^{\infty} \frac{S_f(x)}{x^{s+1}} dx, \Re(s) > \sigma_{a,f}, 
\]
where $\sigma_{a,f}$ is the abscissa of absolute convergence of the DGF $D_f(s)$. 
The DGF of $f$ is an analytic function of $s$ for all $\Re(s) > \sigma_{a,f}$. 
The DGF of the Dirichlet inverse of $f$ is given by the reciprocal of the 
DGF of $f$. An inversion formula for recovering the coefficients $f(n)$ of $n^{-s}$ 
from a given DGF is stated as follows \cite[\S 11.11]{APOSTOL-ANUMT}: 
\[
f(x) = \lim_{T \rightarrow \infty} \frac{1}{2T} \int_{-T}^{T} x^{\sigma+\imath t} 
     D_f(\sigma + \imath t) dt, \forall x \in \mathbb{Z}^{+}; 
     \forall \sigma > \sigma_{a,f}. 
\]
\end{definition} 

\subsection{Known results on local sign changes of an arithmetic function} 

\begin{heuristic}[Duality of the Behavior and Local Oscillations of Dirichlet Inverse Functions]
One heuristic that recurs in applications is that if an 
invertible $f \geq 0$ is non-negative for all $n \geq 1$, then the corresponding 
sequence of sign changes for its inverse $f^{-1}$ is oscillatory and typically 
hard to predict. The same is true for invertible non-negative integer matrices: 
the corresponding inverse matrices as a general measure tend to display 
semi-random, highly oscillatory, and variably signed behavior. So it stands to reason 
that we may expect the initial invertible function $f$ to determine the behavior and other 
structural properties of $f^{-1}$. In general, the local growth rates of $f$ dictates by 
upper bound on the unsigned magnitude of $f^{-1}$ the maximum growth rates of the inverse 
functions \cite{DIRINVFUNC-GROWTH-PROPS}. 
\end{heuristic} 

\subsubsection{Sign changes of $f$ on increasingly large intervals} 

Let $V(f, Y)$ denote the number of sign changes of $f$ on the interval $(0, Y]$ for 
real $Y > 0$. 
More precisely, we have that 
\[
V(f, Y) := \sup \left\{N: \exists \{x_i\}_{i=1}^N, 0<x_1<\cdots<x_N \leq Y, 
     f(x_i) \neq 0, \sgn(f(x_i)) \neq \sgn(f(x_{i+1})), \forall 1 \leq i < N\right\}. 
\]
It is known that the analytic properties, poles, and zeros of the 
Dirichlet generating function, $D_f(s)$, or DGF, of $f$ provide key insights into the 
sign changes of these functions \cite{OSCPROPS-ARITHFUNCSI}. 
For example, if the DGF of $f$ is analytic on some half-plane, subject to certain 
restrictions, then Landau showed in 1905 that 
the summatory function of $f$, $S_f(x)$, changes signs infinitely often as we let 
$x$ tend to infinity. In particular, we 
obtain the following theorem of P\'olya, which extends Landau's result to provide a 
statement concerning the frequency of the sign changes of $S_f(x)$: 

\begin{theorem}[P\'olya on Sign Changes of the Summatory Function of an Arithmetic Function] 
Suppose that $S_f(x)$ is real-valued for all $x \geq x_0$, and define the function 
$\hat{F}_f(s)$ by the Mellin transform at $-s$ as 
\[
\hat{F}_f(s) := \int_{x_0}^{\infty} \frac{S_f(x)}{x^{s+1}} dx. 
\]
Suppose that $\hat{F}_f(s)$ is analytic for all $\Re(s) > \theta$, but is not analytic 
in any half-plane $\Re(s) > \theta - \varepsilon$ for $\varepsilon > 0$. 
Furthermore, suppose that $\hat{F}_f(s)$ is meromorphic in some half-plane 
$\Re(s) > \theta - c_0$ for some $c_0 > 0$. Let 
\[
\gamma_f := \begin{cases} 
     \inf \{|t|: \hat{F}_f(s) \text{\ is not analytic at\ } s = \theta+\imath t\}, & \text{\rm
     if $f$ is not analytic at $\Re(s) = \theta$; } \\ 
     \infty, & \text{\rm otherwise.}
     \end{cases}. 
 \]
 Then 
 \[
 \limsup_{Y \rightarrow \infty} \left\{\frac{V(S_f, Y)}{\log Y}\right\} \geq \frac{\gamma_f}{\pi}. 
\]
\end{theorem} 

\subsection{A smoothing sum of exponentials that extracts signedness from a sequence} 

Let $\zeta_m := \exp(2\pi\imath / m)$ denote the primitive $m^{th}$ root of unity. 
Our idea and new construction is to ``encode'' $f$ in such a way that its signedness is expressed in a 
predictable way. To this end, we define two invertible transformations on a fixed $f$ formed by a 
discrete convolution with the special partition number sequences defined above: 
\begin{align} 
\label{eqn_sfn_tfn_SignSmoothingByExpFuncSumsFuncs_intro_v1}
s_{m,k}[f](n) & := \sum_{j=1}^{n} f(j) \zeta_m^{n-j} \exp\left(\pi \sqrt{k(n-j)}\right), 
     \forall n \geq 1; m \in \mathbb{Z}^{+}; k \in (0, \infty); \\ 
\notag 
t_{m,k}[f](n) & := \sum_{j|n} f(j) \zeta_m^{n-j} \exp\left(\pi \sqrt{k(n-j)}\right), 
     \forall n \geq 1; m \in \mathbb{Z}^{+}; k \in (0, \infty).
\end{align} 
The transformation in the previous equation is invertible by convolution with another signed 
sequence of exponential functions. 
The characteristic limiting behavior we observe in these 
transformations is typified by the next definitions. 

\begin{definition}
We say that a arithmetic sequence $\{f(n)\}_{n \geq 1}$ has 
\emph{property $\mathcal{P}_{1,m,k}$} at $N$ if the sign of $\Re\left\{s_{m,k}[f](n) \cdot \zeta_m^{-n}\right\}$ 
is constant for all $n \geq N$. Similarly, we say that $f$ has 
\emph{property $\mathcal{P}_{2,m,k}$} at $N$ if the sign of $\Im\left\{s_{m,k}[f](n) \cdot \zeta_m^{-n}\right\}$ 
is constant for all $n \geq N$. 
We define for $i := 1,2$ 
\begin{align*} 
M_{i,m,k}(f) & := \sup \left\{n \geq 1: f \text{\ does not have property\ } \mathcal{P}_{i,m,k} \text{\ at\ } n\right\}. 
\end{align*} 
\end{definition} 

The above definition shows that if an arithmetic function $f$ satisfies $\mathcal{P}_{m,k}$ at some finite $N \geq 1$, then 
for all sufficiently large $n \geq N$, the signed magnitude of the real part of the transformation 
$s_{m,k}[f](n)$ tends to one side of the real line or the other. The definition does not provide the limiting signage of the 
transformation sequence even if the function $f$ satisfies property $\mathcal{P}_{m,k}$. 

\begin{theorem}[A Sign Smoothing Convolution Operator by Exponential Function Scaling] 
\label{theorem_MainTheorem_Stmt_v1} 
For any Dirichlet invertible arithmetic function $f$ which is non-vanishing on the positive integer, 
any $m \in \mathbb{Z}^{+}$, and any $k \in (0, \infty)$ such that 
$f(n) \ll \exp(\pi\sqrt{kn})$, $M_{1,m,k}(f^{-1})$ and $M_{2,m,k}(f^{-1})$ are finite. Moreover, 
\[
\lim_{n \rightarrow \infty} \left\lvert \Re\left\{\frac{s_{m,k}[f^{-1}](n)}{\zeta_m^n}\right\} \right\rvert = +\infty, 
     \quad\text{ and }\quad 
\lim_{n \rightarrow \infty} \left\lvert \Im\left\{\frac{s_{m,k}[f^{-1}](n)}{\zeta_m^n}\right\} \right\rvert = +\infty
\]
\end{theorem} 

\begin{theorem} 
\label{theorem_Intro_InitBounds} 
For any Dirichlet invertible $f$ which is non-vanishing on the positive integers, 
any $m \in \mathbb{Z}^{+}$, and any $k \in (0, \infty)$, we have that 
\begin{align}
\tag{A.1} 
M_{1,m,k}(f^{-1}) & = TODO_{1,m,k}; \\ 
\tag{A.2} 
M_{2,m,k}(f^{-1}) & = TODO_{2,m,k}; \\ 
\tag{B} 
\limsup_{n \rightarrow \infty} \left(\sgn\left\{\Re\left[s_{m,k}[f^{-1}](n) \cdot \zeta_m^{-n}\right]\right\}\right) & = 
     TODO_{m,k}; \\ 
\tag{C} 
\limsup_{n \rightarrow \infty} \left(\sgn\left\{\Im\left[s_{m,k}[f^{-1}](n) \cdot \zeta_m^{-n}\right]\right\}\right) & = 
     TODO_{m,k}.
\end{align} 
\end{theorem}

\begin{theorem}[A Sign Smoothing Convolution Operator by Exponential Function Scaling] 
\label{theorem_MainTheorem_Stmt_v2} 
For any Dirichlet invertible arithmetic function $f$ which is non-vanishing on the positive integers, 
any $m \in \mathbb{Z}^{+}$, and any $k \in (0, \infty)$ such that 
$f(n) \ll \exp(\pi\sqrt{kn})$, the sign of 
$\Re\left[t_{m,k}[f^{-1}](n) \cdot \zeta_m^{-n}\right]$ is eventually constant. 
That is, there exists a finite $N_f \geq 1$ such that for all $n > N_f$, 
\[
\sgn\left\{\Re\left[t_{m,k}[f^{-1}](n) \zeta_m^{-n}\right]\right\} - 
     \sgn\left\{\Re\left[t_{m,k}[f^{-1}](n-1) \zeta_m^{1-n}\right]\right\} = 0. 
\]
\end{theorem} 

\subsection{Generalizations of the sign smoothing convolutions} 

For any $\mathcal{A}_n \subseteq \{1,2,\ldots,n\}$, and monotone increasing function 
$E: \mathbb{N} \rightarrow \mathbb{R}$ such that $f(n) \ll E(n)$, that 
\begin{align*} 
\sum_{j \in \mathcal{A}_n} f(j) E(n-j) & = \sum_{j \in \mathcal{A}_n} \left[ 
     1-2(V(f, j)-V(f, j-1))\right] |f|(j) E(n-j) \\ 
     & = \sum_{j \in \mathcal{A}_n} |f|(j) E(n-j) - 2 V(f, n) |f|(n) E(0) \\ 
     & \phantom{=\quad\ } + 
     2 \sum_{\substack{j \in \mathcal{A}_n \\ j < n}} V(f, j) \left[ 
     |f|(j+1) E(n-1-j) - |f|(j) E(n-j) 
     \right]. 
\end{align*} 
The definitions of $\mathcal{A}_n$ corresponded to 
$\{1 \leq d \leq n: d|n\}$ for the sums $t_{m,k}[f](n)$ and to 
$\{1,2,\ldots,n\}$ for the sums $s_{m,k}[f](n)$ 
we defined in \eqref{eqn_sfn_tfn_SignSmoothingByExpFuncSumsFuncs_intro_v1} of the 
previous subsection. 

We now arrive at a natural question to consider 
about sign smoothing convolutions with the previous eventually constant sign properties 
where the weight function (previously of 
exponentials to a square root power of the input) is in some sense ``optimal'' for the 
Dirichlet invertible function $f$ and its inverse $f^{-1}$. 
That is, what is the most natural choice of the sign smoothing weight function we use in our convolved 
sums that results in 
\begin{itemize} 
\item[1.] Optimal growth rates of the convolved sums (or otherwise nice properties to 
work with); and 
\item[2.] The resulting sequence of convolved sums is constant in sign for the maximal number of 
natural numbers possible? 
\end{itemize} 
Consider the following questions in particular: 

\begin{question}[A Natural Sign Smoothing Function for $f$]
Given that $f(n) = O(U_f(n))$ and $V(f, Y) = O(V_f(Y))$ -- and these upper bounds are \emph{tight} in so much as 
$\forall \varepsilon, \delta>0$, $\exists N, Y \geq 1$ such that $f(N) + \varepsilon > U_f(N)$ and 
$V(f, Y) + \delta > V_f(Y)$ -- 
what is the optimal (minimal) choice of 
monotone increasing functions $E_f, E_f^{\ast}$ such that the signs of 
\[
t_f[E^{\ast}](n) := \sum_{j|n} f(j) E_f^{\ast}(n/j), s_f[E](n) := \sum_{j=1}^{n} f(j) E_f(n-j), 
\]
are eventually constant? Can we choose the best possible functions $E_f, E_f^{\ast}$ so that the signs of the 
corresponding sequences above are constant for all $n \geq N_f$ with $N_f \geq 1$ minimal over the natural 
numbers? That is, given some function space $\mathcal{E}_f$, what is 
\[
E_f := \operatorname{argmin}\limits_{E \in \mathcal{E}_f} \left\{ 
     \min_{N \in \mathbb{Z}^{+}} \left[s_f[E](n+1) - s_f[E](n) = 0, \forall n \geq N
     \right]\right\}, 
\]
and is this function unique? 
\end{question} 

\subsubsection{Remarks about smoothing by partition functions with sub-exponential growth} 

Why do the functions $E_k(n) := \exp(\pi \sqrt{kn})$ seem to do universally so well for all functions $f$ 
we have tried (what is special about these forms)? Why do some partition function sequences work better than 
others (or not at all) with respect to preserving the property of sign smoothing of any arithmetic $f$? 

\subsubsection{Remarks about the significance of establishing the 
               existence of optimal sign smoothing convolutions} 

\section{Proofs of the theorems and key results} 


\subsection{Some immediate corollaries} 

We can relate the sequence of $s_{1,k}[f](x)$ to the summatory function $S_f(x)$ of $f$ closely within 
some bounded error term. In particular, it is not difficult to prove that 
\[
S_f(x) = \frac{s_{1,k}[f](x)}{\exp(\pi\sqrt{kx})} + O\left(\sum_{j=1}^{x-1} S_f(j) e^{-\pi\sqrt{kx}}\right). 
\]
So provided that the growth rates of $S_f(x)$ are sub-polynomial (significantly sub-exponential), we have a good 
approximation to $S_f(x)$ when $x \gg 1$ is large. This leads us to the following corollary: 

\begin{cor}[A Sign Bias for General Summatory Functions]
For limiting large $x \gg 1$, the summatory function $S_f(x)$ has a sign bias towards 
\[
\limsup_{x \rightarrow \infty} \left(\sgn\left\{s_{1,k}[f](x)\right\}\right). 
\]
\end{cor}
\begin{proof}
This is easy given the eventually constant sign theorems ... 
\end{proof} 

\begin{example}[A Sign Bias for the Mertens Function]
The \emph{Mertens function}, $M(x) := \sum_{n \leq x} \mu(n)$, defines the summatory function of the 
M\"obius function. Since $M(x) = o(x)$ for all $x \geq 1$, we have that 
\[
M(x) = s_{1,k}[\mu](x) \cdot \exp(-\pi\sqrt{kx}) + O\left(x^2 e^{-\pi\sqrt{kx}}\right), 
\]
where $\mu$ is the Dirichlet inverse of $f(n) \equiv 1$. 
Now by Theorem \ref{theorem_Intro_InitBounds}, we have that the 
sign of $s_{1,k}[\mu](x)$ is constant for all 
$x \geq TODO$. Then by computation, we can see that the sign of $s_{1,1}[\mu](x)$ eventually 
tends towards $-1$ (TODO). Thus we have an observation of a long suspected bias of the values of the 
Mertens function on the positive integers towards negative values!
\end{example} 

\begin{example}[Sign Biases for Weighted Mertens Functions]
For real $\alpha \geq 0$, define the weighted Mertens function by 
\[
M_{\alpha}(x) := \sum_{n \leq x} \frac{\mu(n)}{n^{\alpha}}, \forall x \in \mathbb{Z}^{+}. 
\]
The function $M_{\alpha}(x)$ is the summatory function of the Dirichlet inverse of the function 
$f(n) = \operatorname{Id}_{-\alpha}(n)$. Then (TODO) ... 
\end{example} 

\begin{example}[Verifying Known Sign Biases for $L_{\alpha}(x)$]
The weighted summatory function of the Liouville lambda function is defined by 
\[
L_{\alpha}(x) := \sum_{n \leq x} \frac{\lambda(n)}{n^{\alpha}}, 
     \forall x \in \mathbb{Z}^{+}; \alpha \geq 0
\]
It is known that for all real $\alpha \geq 1$, there is a negative sign bias for $L_{\alpha}(x)$ 
\cite{TODO}. 
Since TODO ... 
\end{example} 



\section{Local optimality of the convolution weight functions} 


\begin{remark}[A Necessary Condition for Sign Smoothing Convolutions] 
We call $s_f[E]$ a \emph{sign smoothing convolution} if the sign of $s_f[E](x)$ is 
constant for all sufficiently large $x \gg 1$. 
Notice from the proof of the previous result that we have shown
\[
\mathcal{M}[s_f[E]](-s) = \frac{(1-s) D_f(s)}{s} \Gamma(s) \sin(\pi s) \left[ 
     E(0) + \int_1^{\infty} \frac{E(x)-E(x-1)}{x^{s+1}} dx 
     \right]. 
\]
Suppose that this Mellin transformation is analytic for $\Re(s) > \theta_0$, but is not 
analytic in any larger half-plane $\Re(s) > \theta_0-\varepsilon$ for $\varepsilon>0$. 
If $s_f[E](x)$ \emph{does not change sign} 
infinitely often as $x \rightarrow \infty$, then it must 
be the case that $\mathcal{M}[s_f[E]](-s)$ is not analytic at $s := \theta_0$ 
\citep[\cf Landau's Theorem]{OSCPROPS-ARITHFUNCSI}. 
\end{remark} 

\begin{prop}[Growth of Sign Smoothing Convolutions]
For any arithmetic function $f$ and positive (strictly increasing) functions $E$ on the non-negative reals, 
we define 
\[
s_f[E](n) := \sum_{j=1}^{n} f(j) E(n-j), \forall n \in \mathbb{Z}^{+}. 
\] 
We can assert conditions on the rate of growth of these sums given by the functions $\phi$ such 
that $$s_f[E](x) = \sum_{k \geq 0} \phi(k) \frac{(-x)^{k}}{k!}.$$ In particular, we have that 
for any $k > \sigma_{a,f}$, these functions are given by 
\[
\phi(k) = \frac{(1-k) D_f(k)}{k} \Gamma(k) \sin(\pi k) \left[ 
     E(0) + \int_1^{\infty} \frac{E(x)-E(x-1)}{x^{k+1}} dx 
     \right]. 
\] 
%Clearly, $\phi(0) = 0$ and for all $k \in [1, \floor{\sigma_{a,f}}] \cap \mathbb{Z}$, if we can 
%analytically continue $D_f(s)$ at $s = k$, then we can obtain formulas for these initial values of 
%$\phi$. 
\end{prop} 
\begin{proof} 
Apply Ramanujan's Master Theorem for Mellin transforms ... 
\end{proof} 




\section{Applications} 

\subsection{Computations of the signs of the Dirichlet inverse of any arithmetic function $f$} 

A partition theoretic motivation for expressing the Dirichlet inverse of any $f$ such that 
$f(1) \neq 0$ provides that for $n > 1$: 
\begin{equation} 
\label{eqn_fInvDirInv_PartitionsFormula} 
f^{-1}(n) = \sum_{k=1}^{\Omega(n)} (-1)^k \left\{ 
     \sum_{{\lambda_1+2\lambda_2+\cdots+k\lambda_k=n} \atop {\lambda_1, \lambda_2, \ldots, \lambda_k | n}} 
     \frac{(\lambda_1+\lambda_2+\cdots+\lambda_k)!}{1! 2! \cdots k!} 
     f(\lambda_1) f(\lambda_2)^2 \cdots f(\lambda_k)^k\right\}. 
\end{equation} 
Let the $m$-fold convolution of an arithmetic function $g$ with itself (i.e., convolve 
$g$ with itself $m$ times in a row at $n$) be denoted by $[g]_{\ast_m}$. Then 
Mousavi and Schmidt proved that \cite{MOUSAVI-SCHMIDT-2019} 
\begin{equation} 
\label{eqn_} 
f^{-1}(n) = \frac{\varepsilon(n)}{f(1)} + 
     \sum_{j=0}^{\left\lfloor \frac{\Omega(n)}{2} \right\rfloor} \left( 
     [f-f(1)\cdot\varepsilon]_{\ast_{2j+1}}(n) - f(1) \times 
     [f-f(1)\cdot\varepsilon]_{\ast_{2j}}(n)
     \right) \frac{1}{f(1)^{2j+1}}. 
\end{equation} 
Since we can expand $m$-fold convolutions of a sum of $k$ arithmetic functions using the 
multinomial (with $k$ terms at $m$) theorem as 
\[
[f_1+f_2+\cdots+f_k]_{\ast_m} = \sum_{\substack{i_1+i_2+\cdots+i_k = m \\ i_1,i_2,\ldots,i_k \geq 0}} 
     \binom{m}{i_1,i_2,\ldots,i_k} [f_1]_{\ast_{i_1}} [f_2]_{\ast_{i_2}} \cdots 
     [f_k]_{\ast_{i_k}}, 
\]
this last formula for $f^{-1}$ leads to another set of insights we can apply in expressing the 
sign of $f^{-1}(n)$. Namely, we give the next proof of 
Proposition \ref{prop_SeqOfSignsOfDirInv_SpCases_stmt_intro_v1} which 
we stated in the introduction section on Dirichlet convolutions to identify some key 
properties of ``nicely behaved'' Dirichlet inverse functions. 

\begin{proof}[Proof of Proposition \ref{prop_SeqOfSignsOfDirInv_SpCases_stmt_intro_v1}]
\end{proof} 

\subsection{New bounds on summatory functions of a Dirichlet inverse} 

We begin by stating a theorem proved by Schmidt in her dissertation research: 

\begin{theorem}[Schmidt, 2019] 
\label{theorem_Schmidt2019} 
Let $f,g$ be arithmetic functions such that $f(1) \neq 0$. We define the summatory 
function of the Dirichlet convolution of $f$ and $g$ by the next expansions. 
\begin{align*} 
\pi_{f,g}(x) & := \sum_{n \leq x} \sum_{d|n} f(d) g(n/d) \\ 
     & \phantom{:} = \sum_{d=1}^{x} f(d) S_g\left(\floor{\frac{x}{d}}\right) \\ 
     & \phantom{:} = \sum_{k=1}^{x} \left(S_f\left(\floor{\frac{x}{k}}\right) - 
     S_f\left(\floor{\frac{x}{k+1}}\right)\right) S_g(k). 
\end{align*} 
Then the summatory function of $g$ has a representation by a convolved sum involving 
the summatory function of $f^{-1}$ and $\pi_{f,g}(x)$ of the following form: 
\begin{align*} 
S_g(x) & = \sum_{k=1}^{x} \left(\sum_{j=\floor{\frac{x}{k+1}}+1}^{\floor{\frac{x}{k}}} f^{-1}(j)\right) 
     \pi_{f,g}(k) \\ 
     & = \sum_{k=1}^{x} \left(S_{f^{-1}}\left(\floor{\frac{x}{k}}\right) - 
     S_{f^{-1}}\left(\floor{\frac{x}{k+1}}\right)\right) \pi_{f,g}(k), 
     \forall x \in \mathbb{Z}^{+}. 
\end{align*} 
\end{theorem} 

\begin{example}[A New Expression for the Mertens Function Involving Euler's Totient] 
\label{example_Mx_ExOfThm} 
Given that we have the standard convolution identity $\phi(n) = (\operatorname{Id}_1 \ast \mu)(n)$, 
where the Dirichlet inverse of the identity function is expressed exactly as 
$\operatorname{Id}_1^{-1}(n) = n \cdot \mu(n)$, we can use the theorem to write a new 
summation identity for the \emph{Mertens function}, $M(x) := \sum_{n \leq x} \mu(n)$, in terms of 
familiar arithmetic functions. 
In particular, we obtain that 
\[
M(x) = \sum_{k=1}^{x} \left(\sum_{j=\floor{\frac{x}{k+1}}+1}^{\floor{\frac{x}{k}}} j \cdot \mu(j)\right) 
     \Phi(k),
\]
where 
$$\Phi(x) := \sum_{n \leq x} \phi(n) \sim \frac{3x^2}{\pi^2} + O\left(x (\log x)^{2/3} (\log\log x)^{4/3}\right),$$ 
is the summatory function of the totient function. 
The inner difference of summatory function terms (over $j \cdot \mu(j)$) in the above equation 
can be summed by parts to obtain that 
\[
\sum_{j=\floor{\frac{x}{k+1}}+1}^{\floor{\frac{x}{k}}} j \cdot \mu(j) = 
     \floor{\frac{x}{k}} M\left(\floor{\frac{x}{k}}\right) - 
     \floor{\frac{x}{k+1}} M\left(\floor{\frac{x}{k+1}}\right) - 
     \sum_{n=\floor{\frac{x}{k+1}}+1}^{\floor{\frac{x}{k}}} M(n). 
\]
Other similar examples can be used, and in particular any divisor sum identity resulting from 
M\"obius inversion, to obtain new formulas for $M(x)$ using 
Theorem \ref{theorem_Schmidt2019}. 
\end{example} 

\begin{remark}[General Procedure for Applying the Theorem]
The general procedure for using the theorem to express a useful formula for the summatory 
function of some $f$ is to write $h := g \ast f$ where $S_h(x)$ is ``nice'' and $g$ is Dirichlet 
invertible. Then we have that 
\begin{equation}
\label{eqn_Sfx_hEQgAstfCvl_ident} 
S_f(x) = \sum_{k=1}^{x} \left[S_{g^{-1}}\left(\floor{\frac{x}{k}}\right) - 
     S_{g^{-1}}\left(\floor{\frac{x}{k+1}}\right)\right] S_h(k). 
\end{equation} 
Notice that this method is not dissimilar to applying the so-called 
\emph{convolution method} to approximate a sum of arithmetic functions 
\cite{TODO-CVLMETHOD-CROOT}. 
Given that we have the broadly applicable result in Theorem \ref{theorem_Schmidt2019}, 
we now turn to the question of obtaining better estimates of the summatory functions, 
$S_{f^{-1}}(x)$, for any Dirichlet invertible $f$. 
\end{remark} 

\begin{theorem}[Summatory Functions of the Dirichlet Inverse of an Arithmetic Function] 
Suppose that $f$ is a Dirichlet invertible arithmetic function. 
Let the shifted summatory functions be defined as 
$$S_f^{\ast}(x) := \sum_{2 \leq n \leq x} f(n) = S_f(x) - f(1).$$ 
For natural numbers $m \geq 1$, let the sequence $T_f^{[m]}(x)$ be defined recursively by 
\[
T_f^{[m]}(x) = \begin{cases} 
     S_f^{\ast}(x), & m = 1; \\ 
     \sum\limits_{k=2}^{x} f(k) T_f^{[m-1]}\left(\floor{\frac{x}{k}}\right), & m \geq 2. 
     \end{cases} 
\]
The functions $T_f^{[m]}(x)$ are also given explicitly by the nested summation formula 
\[
T_f^{[m]}(x) = \sum_{2 \leq k_1 \leq x} \sum_{2 \leq k_2 \leq \frac{x}{k_1}} \cdots 
     \sum_{2 \leq k_m \leq \frac{x}{k_1k_2 \cdots k_{m-1}}} f(k_1) f(k_2) \cdots f(k_m). 
\]
Using this notation, we have that 
\[
S_{f^{-1}}(x) = \frac{1}{f(1)} + \sum_{m=1}^{\log_2(x)} \frac{(-1)^{m} T_f^{[m]}(x)}{f(1)^{m+1}}. 
\]
\end{theorem} 
\begin{proof} 
\end{proof} 

\begin{cor}[Summatory Functions of Non-Negative Arithmetic Functions] 
\label{cor_SummatoryFuncs_NonNegf} 
Suppose that $f$ is a Dirichlet invertible arithmetic function such that $f(n) \geq 0$ for all 
natural numbers $n \geq 1$. Then the expression for the summatory functions defined in 
\eqref{eqn_Sfx_hEQgAstfCvl_ident} can be split (much like a signed measure) 
into positive and negative components as 
\begin{align*} 
S_f^{+}(x) & = \sum_{k=1}^{x} \left(\sum_{m=1}^{\log(x/k)} \frac{T_g^{[2m]}\left(\frac{x}{k}\right)}{ 
     g(1)^{2m+1}}\right) h(k) \\ 
S_f^{-}(x) & = \sum_{k=1}^{x} \left(\sum_{m=1}^{\log(x/k)} \frac{T_g^{[2m+1]}\left(\frac{x}{k}\right)}{ 
     g(1)^{2m+2}}\right) h(k), 
\end{align*} 
where $S_f(x) = S_f^{+}(x) - S_f^{-}(x)$. Moreover, the signed component functions each satisfy 
$S_f^{+}(x), S_f^{-1}(x) \geq 0$ for all $x \geq 1$. 

Furthermore, the nagative bias of the negatively signed terms in the expansion of $S_f(x)$ only 
contribute weight summing over $1 \leq k \leq x$ for $k \in [1, x/2]$. In particular, we can write 
\[
S_f(x) = \sum_{k=1}^{\frac{x}{2}} \sum_{m \geq 1} \left(T_g^{[2m]}\left(\frac{x}{k}\right) - \int_2^{x/k} 
     \frac{g(t)}{g(1)} T_g^{[2m]}\left(\frac{x}{kt}\right) dt\right) \frac{h(k)}{g(1)^{2m+1}} + 
     \sum_{\frac{x}{2} < k \leq x} \sum_{m \geq 1} T_g^{[2m]}\left(\frac{x}{k}\right) \frac{h(k)}{g(1)^{2m+1}}. 
\]
If we make the additional assumptions that $h \geq 0$ and $g(1) \equiv 1$, then we can use the above 
formula to quantify an expression for the negative bias of the summatory function $S_f(x)$. 
\end{cor} 
\begin{proof} 
\end{proof} 

\begin{question}[Applications to Signed Measures]
In \cite{NG-MERTENS}, Ng gives a probabilistic argument which establishes the existence of a 
limiting measure $\nu$ on $\mathbb{R}$. This measure is defined such that for all 
bounded Lipschitz continuous functions $f$ on the reals, we have the existence of $\nu$ 
satisfying 
\[
\lim_{Y \rightarrow \infty} \frac{1}{Y} \int_0^{Y} f\left(e^{-y/2} M(e^y)\right) dy = 
\int_{-\infty}^{\infty} f(x) d\nu(x), 
\]
where $M(x)$ is the Mertens function. Given that we know how to split the summatory functions 
into signed positive and negative parts via Corollary \ref{cor_SummatoryFuncs_NonNegf}, it seems 
natural to ask whether we can make similar sense out of the signed components of any summatory 
function expressed in this way? 
\end{question} 

\begin{example}[Returning to the Mertens Function Application of the Theorem]
In the example of the Mertens function sums started in Example \ref{example_Mx_ExOfThm}, we have our 
function $g(n) \equiv n, \forall n \geq 1$ in the form of 
\eqref{eqn_Sfx_hEQgAstfCvl_ident}. Then we can compute the functions $T_g^{[m]}(x)$ inductively by the 
iterated integrals 
\[
T_g^{[m]}(x) = \int_2^{x} \int_2^{\frac{x}{t_0}} \cdots \int_2^{\frac{x}{t_0t_1 \cdots t_{m-2}}} 
     t_0 t_1 \cdots t_{m-1} dt_{m-1} \cdots dt_1 dt_0. 
\]
We find that for sequences of constants $\hat{C}_0(m), C_i(m)$, these functions are given by 
\[
T_g^{[m]}(x) = \hat{C}_0(m) + x^2 \times \sum_{i=0}^{m} C_i(m) \log^{i}(x). 
\]
The coefficient sequences, which satisfy $\hat{C}_0(m), C_i(m) \geq 0$ for all $m \geq 1$ and 
$0 \leq i \leq m$, can be computed recursively as 
\begin{align*} 
\hat{C}_0(m+1) & = -2 \hat{C}_0(m) \\ 
C_0(m+1) & = \frac{\hat{C}_0(m)}{2} + \sum_{i=0}^m \frac{C_i(m)}{i+1} (-\log(2))^{i+1} \\ 
C_{m+1}(m+1) & = \frac{C_m(m)}{m+1} \\ 
C_i(m+1) & = \sum_{j=i-1}^{m} \binom{j+1}{i} \frac{C_j(m)}{j+1} (-\log(2))^{j+1-i}, \forall 1 \leq i \leq m. 
\end{align*} 
Thus if we let 
\[
U_g^{[u]}(x) := \sum_{m=1}^{u} (-1)^m T_g^{[m]}(x) = \sum_{m=1}^u (-1)^{m} \hat{C}_0(m) + 
     \sum_{i=0}^u \sum_{m=i}^{u} x^2 (-1)^m C_i(m) \log^{i}(x), 
\]
then we obtain a formula for $M(x)$ given by 
\begin{align*} 
M(x) & \approx \sum_{k|x} \Phi\left(\frac{x}{k}\right) \left[U_g^{[\log k]}(k) - U_g^{[\log(k-1)]}(k)\right] \\ 
     & = \sum_{k|x} \Phi\left(\frac{x}{k}\right) \left[(-1)^m T_g^{[m]}(k)\right]\Biggr|_{m=\floor{\log(k-1)}}. 
\end{align*} 
TODO ... 
\end{example} 

\subsection{Improving bounds on sign changes of arithmetic functions on short intervals} 

\subsection{Partition theoretic applications (smoothness, or roundness, etc.)} 

\section{Open questions and possible generalizations} 


\section{Numerical data: Tables of examples for special Dirichlet inverse functions} 

\subsection{Examples of the theorems} 

\subsection{Examples of sign smoothing, but non-optimal lower bound on the number of natural numbers 
               it takes to arrive at a constant sign in the convolved sequence} 
               
\subsection{Counter examples of ``natural'' choices of sign smoothing convolution functions} 

The $f$ by $f^{-1}$ sums, examples where this seems to work, and examples where it totally fails ... 

\section{Concluding thoughts and remarks} 

\subsection{Working topics list} 

We also should consider the roles of the following properties: 
\begin{itemize} 

\item The role of local osciallations of an arithmetic function? 
\item Sign changes of $\nabla[f](n)$? 
\item An exponential sum formula for $M_j(x)$ ... 

\end{itemize} 

\subsection*{Acknowledgements} 


\begin{thebibliography}{10} 

\bibitem{APOSTOL-ANUMT} 
T. M. Apostol, \textit{Introduction to analytic number theory}, Springer, 1976. 

\bibitem{DIRINVFUNC-GROWTH-PROPS} 
F. Baustian and V. Bobkov, On asymptotic behavior of Dirichlet inverse, {\em ArXiv:math.NT/1903.12445} (2019). 

\bibitem{OSCPROPS-ARITHFUNCSI} 
J. Kaczorowski and J. Pintz, Oscillatory properties of arithmetical 
  functions I, \emph{Acta Math. Hung.} {\bf 48}, pp. 173--185, 
  1986. 

\bibitem{MULTFUNCS-INSHORTINTS} 
K. Matom\"aki and M. Radziwitt, Multiplicative functions in 
  short intervals, \emph{Annals of Math.}, Vol. 183, No. 3, 
  pp. 1015--1056 (2016). 

\bibitem{MOUSAVI-SCHMIDT-2019} 
H. Mousavi and M. D. Schmidt, Factorization theorems for relatively prime 
  divisor sums, GCD sums, and generalized {R}amanujan sums, 
  \emph{Ramanujan J.}, to appear (2019). 

\bibitem{NG-MERTENS} 
N. Ng, The distribution of the summatory function of the M\"obius function, 
  \emph{ArXiv:math.NT/0310381} (2008). 

\bibitem{SUMMABILITY-DIRCVLS} 
S. L. Segal, Summability by Dirichlet convolutions, 
  \emph{Proc. Camb. Phil. Soc.} \textbf{63}, 393 (1967). 

\end{thebibliography} 


\end{document} 
