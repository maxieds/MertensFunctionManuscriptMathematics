%\documentclass[draft,12pt,reqno]{article} 
\documentclass[10pt,reqno,letterpaper]{article} 

%\usepackage[showframe,verbose=true,margin=1in,nohead,letterpaper]{geometry}
\usepackage[verbose=true,margin=1in,nohead,letterpaper]{geometry}

\usepackage{amsmath} 
\usepackage{amsfonts} 
\usepackage{amsthm}
\usepackage[hidelinks]{hyperref}
\usepackage[usenames,dvipsnames]{xcolor} 
\hypersetup{
    colorlinks,
    linkcolor={Gray!50!Black}, 
    citecolor={Gray!50!Black},
    urlcolor={Gray!50!Black}
}
\usepackage{amssymb} 
\usepackage{amscd} 
\usepackage{url} 
\usepackage{datetime}
\usepackage{tcolorbox}

\theoremstyle{plain} 
\newtheorem{theorem}{Theorem}
\newtheorem{conjecture}[theorem]{Conjecture}
\newtheorem{claim}[theorem]{Claim}
\newtheorem{prop}[theorem]{Proposition}
\newtheorem{lemma}[theorem]{Lemma}
\newtheorem{cor}[theorem]{Corollary}
\numberwithin{theorem}{section}

\theoremstyle{definition} 
\newtheorem{example}[theorem]{Example}
\newtheorem{remark}[theorem]{Remark}
\newtheorem{definition}[theorem]{Definition}
\newtheorem{notation}[theorem]{Notation}
\newtheorem{question}[theorem]{Question}
\newtheorem{discussion}[theorem]{Discussion}
\newtheorem{facts}[theorem]{Facts}
\newtheorem{summary}[theorem]{Summary}
\newtheorem{heuristic}[theorem]{Heuristic}
\newtheorem{ansatz}[theorem]{Ansatz}

%\usepackage[T1]{fontenc} 
\usepackage{MnSymbol}

%% : Customize the beleagueredBibliography and Citations for BibTeX: 
%\usepackage{natbib} 
%\bibpunct{[}{]}{;}{}{,}{,~} 

%% : Page Formatting and Related Configuration Options: 
%\setlength{\headsep}{0.1in} 
%\setlength{\headheight}{0in} 
%\setlength{\topskip}{0in} 
%\setlength{\topmargin}{0in} 
%\setlength{\topsep}{0in} 
%\setlength{\partopsep}{0in} 
%\setlength{\pdfpagewidth}{8.5in} 
%\setlength{\pdfpageheight}{11in} 
%\setlength{\textwidth}{6.5in}
%\setlength{\textheight}{9in} 
%\setlength{\oddsidemargin}{-0.0in}
%\setlength{\evensidemargin}{-0.0in}
%\setlength{\topmargin}{-0.2in}

%% Define and setup special stylized section number headings:
\usepackage{tikz}
\usetikzlibrary{arrows}
\usepackage{adjustbox}

\newcommand{\CustomSectionNumberHeading}[1]{
     \vspace*{-1mm}
     \trimbox{1.8mm 1.6mm 5.8mm 2mm}{% LHS-SPC FROM-BOT RHS-SPC
          \begin{tikzpicture}
               \draw[fill,color=black] (-2mm,-2mm) rectangle (2mm,2mm);
               \draw[color=white] (0mm,0mm) node { \textbf{\large{#1}} };
          \end{tikzpicture}
     }
     \hspace*{-1.5mm}
}

\usepackage[compact]{titlesec} 
%\titlespacing{\section}{0pt}{-0.9mm}{-0.9mm} 
%\titlespacing{\subsection}{0pt}{-0.9mm}{-0.9mm} 
%\def\TitleSpacingAdjust{-0.895mm}
\def\TitleSpacingAdjust{-0.5mm}
\titlespacing{\section}{0pt}{\TitleSpacingAdjust}{\TitleSpacingAdjust} 
\titlespacing{\subsection}{0pt}{\TitleSpacingAdjust}{\TitleSpacingAdjust} 
%\titleformat{\section}{\bfseries\large}{\CustomSectionNumberHeading{\thesection}}{10pt}{}
\titleformat{\section}{\bfseries\large}{\thesection}{10pt}{}
%\titleformat{\subsection}{\bfseries\normalsize}{
%     \hspace*{-1mm}\scalebox{1.35}{$\blacktriangleright$}\hspace*{3.5mm}\thesubsection\ \ --\hspace*{-1.5mm}}{10pt}{} 
%\titleformat{\subsection}{\bfseries\normalsize}{\thesubsection\ \ --\hspace*{-1.5mm}}{10pt}{}
\titleformat{\subsection}{\bfseries\normalsize}{\thesubsection}{10pt}{}

\setlength{\parindent}{0.2in} 
\setlength{\parskip}{0.55mm} 
\setlength{\parsep}{0in} 

\pagestyle{empty}

%\setlength{\abovedisplayskip}{0in} 
%\setlength{\abovedisplayshortskip}{0in} 
%\setlength{\belowdisplayskip}{0in} 
%\setlength{\belowdisplayshortskip}{0in} 

%% : Table Spacing and Formatting: 
\setlength{\abovecaptionskip}{0in} 
\setlength{\belowcaptionskip}{0in} 

%% : Change Various Labelings and Counter Defines: 
\renewcommand{\thesection}{\arabic{section}} 
\renewcommand{\thefootnote}{\fnsymbol{footnote}} 
\addtocounter{footnote}{1}

\newcommand{\ie}[0]{i.e.\ } 
\newcommand{\cf}[0]{\emph{cf.}\ } 
\newcommand{\etc}[0]{etc.\ } 
\newcommand{\PhD}[0]{Ph.D.\ } 

%% : Concrete mathematics Book Recurrence Notation (Special Case Triangles): 
\newcommand{\gkpSI}[2]{\ensuremath{\genfrac{\lbrack}{\rbrack}{0pt}{}{#1}{#2}}} 
\newcommand{\gkpSII}[2]{\ensuremath{\genfrac{\lbrace}{\rbrace}{0pt}{}{#1}{#2}}} 
\newcommand{\gkpEI}[2]{\ensuremath{\genfrac{\langle}{\rangle}{0pt}{}{#1}{#2}}} 
\newcommand{\gkpEII}[2]{\ensuremath{\left\langle\genfrac{\langle}{\rangle}{
            0pt}{}{#1}{#2}\right\rangle}} 

%% : Other math Mode Macros Used by the Article: 
\newcommand{\Iverson}[1]{\ensuremath{\left[#1\right]_{\delta}}} 
\newcommand{\Floor}[2]{\ensuremath{\left\lfloor \frac{#1}{#2} \right\rfloor}}

\usepackage{enumitem}
\usepackage{textcomp}  % Required for encoding \textbigcircle
\usepackage{scalerel}  % Required for emoji \scalerel
\def\🌐{\scalerel*{\includegraphics{cv-resume/earth-globe-americas-1f30e.png}}{\textrm{\textbigcircle}}}
\def\OneFFourCFour{\scalerel*{\includegraphics{cv-resume/paper-1F4C4.png}}{\textrm{$\qed$}}}

%% Set the default font to Helvetica:
\usepackage{helvet}
\renewcommand{\familydefault}{\sfdefault}

\newcounter{completeBibitemIncrementCtr}
\setcounter{completeBibitemIncrementCtr}{0}


\let\oldthebibliography\thebibliography
\let\endoldthebibliography\endthebibliography
\renewenvironment{thebibliography}[1]{
     \renewcommand{\refname}{} 
     \bibliographystyle{plain}
     \begin{oldthebibliography}{#1}
     \setlength{\itemsep}{-0.6mm}
     \setlength{\parskip}{0em}
     \setlength{\topsep}{0pt}
     \setlength{\partopsep}{0pt}
     \setcounter{enumiv}{\value{completeBibitemIncrementCtr}}
     \normalsize%\footnotesize 
}
{
     \setcounter{completeBibitemIncrementCtr}{\value{enumiv}}
     \end{oldthebibliography}
}

\newcommand{\seqnum}[1]{\href{http://oeis.org/#1}{\color{Gray!50!Black}{\underline{#1}}}}

%% Misc spacing (lines and space between equations and text): 
\linespread{1.0}
\AtBeginDocument{
     \setlength{\abovedisplayskip}{0.25pt} 
     \setlength{\abovedisplayshortskip}{0.25pt} 
     \setlength{\belowdisplayskip}{0.25pt} 
     \setlength{\belowdisplayshortskip}{0.25pt} 
     \setlength{\jot}{0pt} 
}
\allowdisplaybreaks

%% Start Content:
\begin{document}

\newcommand{\GRFPEssayTitle}{NSF Postdoctoral Fellowship Project Description}

\section{My background and direction as a professional researcher}

My journey from an undergraduate at the University of Illinois (UIUC) to now, 
where I will graduate with my Ph.D. in mathematics from the 
Georgia Institute of Technology (GA Tech) in 2022, 
is long and was difficult for me. 
One constant that has always been with me over this time period has been my 
enthusiasm and drive to work on mathematics. 
I have always had an excellent support system over the years 
by keeping my immediate family around me. 
In 2005--2006, when I first assumed academic leave from UIUC as a freshmen in computer science, 
Hurricane Katrina struck New Orleans and the Gulf Coast. 
As a result of that historic natural disaster and the 
structural engineering catastrophe that ensued, my grandmother's house 
was submerged under water for weeks. I traveled with my mom by car from the Midwest to try to 
recover what remained of her life and belongings. I ended up 
receiving a fateful challenge and inspiration from another family member. 

My late uncle talked with me about what I was interested in studying, and then used 
his home office inkjet printer to assemble a stack of six to ten printouts of top ranked general 
mathematics journal websites with their submission guidelines for authors. 
I took him literally when he issued me the kind directive to ``\emph{[here] go work on this}'' as a challenge 
to publish my own original research in mathematics. 
I started with the material from my high school copy of the graduate 
\emph{Concrete Mathematics} textbook \cite{GKP} and gradually 
read every survey and book on the prime numbers that I could find online 
(\cf \cite{PRIMEREC}). 
The effort and emphasis on personal work and 
discovery paid dividends after I returned 
to UIUC as a double major in mathematics and computer science. 
I received the Barry M. Goldwater scholarship in 2010. 
My first peer-reviewed publication was also accepted into a respected journal that year. 

\section{Research objectives, methodology and significance of proposed work} 
\label{Section_SummaryOfPriorWork} 

\subsection{Diversity of my past accomplishments and publications}

My research combines number theory, combinatorics and software development. 
More broadly, I have interests in studying combinatorial and analytic number theory, 
in applied cryptography and embedded computer hardware, and in software engineering. 
I have been funded as a graduate research assistant for the last three years or so developing 
open source software in applied mathematical biology at GA Tech. 
My active peer-reviewed publication list is diverse with now over twenty-one entries, as is my 
public profile of open source software projects, each of which reflect the breadth and 
depth of my combined research areas. 
%\footnote{
%     My active list of manuscripts is tracked at \url{https://arxiv.org/a/schmidt_m_2.html} 
%     with major publications at \url{http://orcid.org/0000-0002-3170-5535}. 
%     My open source software contributions are mostly indexed at 
%     \url{https://github.com/maxieds}. 
%}. 
Reviewers can find the detailed bibliography sections starting on page 
\pageref{page_Section_BibliographyB} for complete reference information. 

\subsection{Recent work in analytic number theory}
\label{subSection_PriorWork_MertensFunctionMx} 

\vskip -0.5mm
\noindent
{\normalsize \textbf{\emph{History of the Mertens function (or partial sums of the M\"obius function).}}} 
The function $\mu(n)$ is deeply connected to the 
distribution of the prime numbers and the Riemann Hypothesis (RH) 
through the Euler product representation of the Riemann zeta 
function where it is convergent \cite{APOSTOLANUMT}. 
The Mertens function, $M(x) := \sum_{n \leq x} \mu(n)$, forms partial sums 
of the classical M\"obius function for $x \geq 1$ \cite[\seqnum{A008683}; \seqnum{A002321}]{OEIS}.
We interpret the probability that any integer $n \geq 2$ has an even (or odd) number of prime 
factors so that we can view the function $\lambda(n) := (-1)^{\Omega(n)}$ as a 
randomized ``\emph{digital coin flip}'' of sorts 
\cite[\seqnum{A008836}; \seqnum{A002819}]{OEIS}. Since $\mu(n) \neq 0$ if and only if 
$n$ is squarefree where $\mu(n) = \lambda(n)$ whenever 
$n$ is squarefree, this description facilitates modeling the 
partial sums, $M(x)$, as a $\{-1,0,+1\}$-weighted random walk on the integers. 
The value of the Mertens function indicates the position of a particle starting at zero 
after $x$ steps whose directions are simulated in this quasi-randomized way according to 
the values of $\mu(n)$, or equivalently, by the distribution of the number of distinct prime factors of 
any $n \geq 2$. 
%An inverse Mellin transform applied to the DGF of $\mu(n)$, given by $\zeta(s)^{-1}$, 
%implies that 
%\[
%M(x) = \lim_{T \rightarrow \infty}\ \frac{1}{2\pi\imath} \times \int_{T-\imath\infty}^{T+\imath\infty} 
%     \frac{x^s}{s \zeta(s)} ds.
%\]
The sequence of slow growing oscillatory values of $M(x)$ begin 
as in the following figure (\cf plots to the left; legend on the right): \\[0.3ex] 
%\[
%\{M(x)\}_{x \geq 1} = \{1, 0, -1, -1, -2, -1, -2, -2, -2, -1, -2, -2, -3, -2, -1, -1, -2, -2, 
%     -3, -3, -2, -1, -2, -2, -2, \ldots \}. 
%\] 
\begin{minipage}{0.53\textwidth}
     \includegraphics[height=1.6in]{Mathematica/MertensPlots-v3.png}
\end{minipage} \hfill
\begin{minipage}{0.425\textwidth}
\begin{tcolorbox}[height=1.6in,width=\linewidth,halign=justify,boxsep=0mm,arc=2.25mm,colframe=gray!49]
\vskip -0.65mm
\em\small 
A comparison of the plots of $M(x)$, $L(x) := \sum_{n \leq x} \lambda(n)$, and 
three simulated random walks for $1 \leq x \leq 1500$. 
We define a sequence of i.i.d. random variables, $\{B_k(p)\}_{k \geq 1}$, such that 
$\mathbb{P}[B_k(p) = +1] = 1 - \mathbb{P}[B_k(p) = -1] = p$ for $p \in (0, 1)$. 
The two random walk variants we consider are defined by the sums 
$\operatorname{ORW}(p, x) := \sum_{k=1}^{x} B_k(p)$ and 
$\operatorname{MRW}(p, x) := \sum_{k=1}^{x} B_k(p) \Iverson{\mu(k) \neq 0}$ for $x \geq 1$. 
\end{tcolorbox}
\end{minipage} \\[0.21ex]
An inverse Mellin transform leads to the exact expression of $M(x)$ for any $x > 0$ 
given by the statement of the following theorem of Titchmarsh \cite{TITCHMARSH}. 
It relates the Mertens function 
to the non-trivial zeros of the Riemann zeta function for any $x > 0$. 
That is, assuming the RH, there is an infinite sequence 
$\{T_k\}_{k \geq 1}$ satisfying $k \leq T_k \leq k+1$ for each $k \geq 1$ 
such that for any $x > 0$ 
\[
M(x) = \lim_{k \rightarrow \infty} 
     \sum_{\substack{\rho: \zeta(\rho) = 0 \\ |\Im(\rho)| < T_k}} 
     \frac{x^{\rho}}{\rho \zeta^{\prime}(\rho)} - 2 + 
     \sum_{n \geq 1} \frac{(-1)^{n-1}}{n (2n)! \zeta(2n+1)} 
     \left(\frac{2\pi}{x}\right)^{2n} + 
     \frac{\mu(x)}{2} \Iverson{x \in \mathbb{Z}^{+}}. 
\]
The RH is equivalent to showing that 
$M(x) = O\left(x^{\frac{1}{2}+\epsilon}\right)$ for any 
$0 < \epsilon < \frac{1}{2}$. 
There is a rich history to the original statement of the Mertens conjecture which 
asserted that $|M(x)| < C \sqrt{x}$ for some absolute constant $C > 0$.
The conjecture was first verified by F.~Mertens himself for $C = 1$ and all $x < 10^{4}$ 
without the benefit of modern computation \cite{HAVIL-GAMMA}. 
Since its beginnings in 1897, the Mertens conjecture has been disproved by computational methods using 
non-trivial simple zeta function zeros with comparatively small imaginary parts in the famous paper by 
A.~M.~Odlyzko and H.~J.~J.~t\'{e} Riele \cite{ODLYZ-TRIELE}. 
More recent attempts 
at bounding $M(x)$ naturally consider determining the rates at which the function 
$M(x) x^{-\frac{1}{2}}$ grows with or without bound along infinite 
subsequences in the limit supremum and infimum senses 
\cite{ORDER-MERTENSFN}. 
Non-computational breakthrough progress on 
bounding or determining significant properties 
underlying the distribution of the Mertens function has been slow with the 
exception of two major papers over the last twenty years or so 
\cite{NG-MERTENS,SOUND-MERTENS-ANNALS}. 

\vskip -0.5mm
\noindent
{\normalsize \textbf{\emph{My recent work with the Mertens function.}}} 
My preprint manuscript \cite{MDS-MANU-JNT2021} is accepted 
into the \emph{Journal of Number Theory} this year.  
I rigorously prove a more combinatorial 
and additively structured new characterization of $M(x)$ by 
considering an auxiliary signed sequence, $g^{-1}(n)$, and its corresponding 
partial sums, $G^{-1}(x)$. 
These comparatively regular sequences have not yet been studied in connection with the Mertens function. 
%The integer sequence we consider is precisely defined as the Dirichlet inverse function 
We precisely define the Dirichlet inverse function 
$g^{-1}(n) := (\omega+1)^{-1}(n)$ 
for all $n \geq 1$ where $\omega(n)$ is the strongly additive function that 
counts the number of \emph{distinct} prime factors of $n$ 
\cite[\seqnum{A001221}; \seqnum{A001222}]{OEIS}. 
The first few values of $g^{-1}(n)$ are given by 
\cite[\seqnum{A341444}]{OEIS}. 
%\[
%\{g^{-1}(n)\}_{n \geq 1} = \{1, -2, -2, 2, -2, 5, -2, -2, 2, 5, -2, -7, -2, 5, 5, 2, -2, -7, -2,
%     -7, 5, 5, -2, 9, 2, 5, -2, -7, -2, \ldots \}. 
%\] 
An exact formula for $g^{-1}(n)$ is 
\[
g^{-1}(n) = \lambda(n) \times \sum_{d|n} \mu^2\left(\frac{n}{d}\right) C_{\Omega(d)}(d), n \geq 1,  
\]
where \cite[\seqnum{A008480}]{OEIS} 
\[
C_{\Omega(n)}(n) = \begin{cases}
     1, & \text{if $n = 1$; } \\ 
     (\Omega(n))! \times \prod\limits_{p^{\alpha}||n} \frac{1}{\alpha!}, & \text{if $n \geq 2$. }
     \end{cases}
\]
%We have the following noteworthy properties of the inverse sequence:
%\begin{itemize} 
%\item[\textbf{(A)}] For all $n \geq 1$, $\operatorname{sgn}(g^{-1}(n)) = \lambda(n)$; 
%\item[\textbf{(B)}] For all squarefree integers $n \geq 1$, we have that 
%     \[
%     |g^{-1}(n)| = \sum_{m=0}^{\omega(n)} \binom{\omega(n)}{m} \times m!; 
%     \]
%\item[\textbf{(C)}] 
%     For large $x$ and $n \leq x$, we associate a natural combinatorial 
%     significance to the magnitude of the distinct values of 
%     $g^{-1}(n)$ that depends directly on the exponent patterns in the 
%     prime factorizations of the integers in $\{2,3,\ldots,x\}$ viewed as multisets.
%     Suppose that $n_1, n_2 \geq 2$ are such that their factorizations into distinct primes are 
%     given by $n_1 = p_1^{\alpha_1} \cdots p_r^{\alpha_r}$ and $n_2 = q_1^{\beta_1} \cdots q_r^{\beta_r}$. 
%     If $\{\alpha_1, \ldots, \alpha_r\} \equiv \{\beta_1, \ldots, \beta_r\}$ as multisets of prime exponents, 
%     then $g^{-1}(n_1) = g^{-1}(n_2)$.
%\end{itemize} 
For large $x$ and $n \leq x$, we can then associate a natural combinatorial 
significance to the magnitude of the distinct values of 
$g^{-1}(n)$ that depends directly on the exponent patterns in the 
prime factorizations of the integers in $\{2,3,\ldots,x\}$ viewed as multisets.
Proof methods adapted in the spirit of Montgomery and Vaughan's work \cite{MV} 
allow us to uniquely reconcile the property of strong additivity with signed sums of 
multiplicative functions through our new characterization of 
$M(x)$ by the partial sums of $g^{-1}(n)$. 
%We first prove that 
%uniformly for all $1 \leq k \leq \frac{3}{2} \log\log x$ there is an absolute constant 
%$A_0 > 0$ such that 
%\begin{equation}
%\label{eqn_cor_SummatoryFuncsOfUnsignedSeqs_v2}
%\sum_{\substack{n \leq x \\ \Omega(n) = k}} C_{\Omega(n)}(n) \sim 
%     A_0\sqrt{2\pi} x \times \widehat{G}\left(\frac{k-1}{\log\log x}\right) 
%     \frac{(\log\log x)^{k-1/2}}{(k-1)!}, 
%     \text{\ as\ } x \rightarrow \infty, 
%\end{equation}
%where $\widehat{G}(z) := \frac{\zeta(2)^{-z}}{\Gamma(1+z) (1+P(2)z)}$ for 
%$0 \leq |z| < P(2)^{-1} \approx 2.21118$. 
%Using the result in \eqref{eqn_cor_SummatoryFuncsOfUnsignedSeqs_v2}, 
We prove 
that for an absolute constant $B_0 > 0$ the average order 
of the unsigned sequence satisfies 
\begin{align*}
%\frac{1}{n} \times \sum_{k \leq n} C_{\Omega(k)}(k) & = 
%     B_0(\log n) \sqrt{\log\log n} \left(1 + O\left(\frac{1}{\log\log n}\right)\right), 
%     \mathrm{\ as\ } n \rightarrow \infty \\ 
\frac{1}{n} \times \sum_{k \leq n} |g^{-1}(k)| & = 
     \frac{6B_0 }{\pi^2} (\log n)^2\sqrt{\log\log n} (1+o(1)), 
     %\left(1 + O\left(\frac{1}{\log\log n}\right)\right), 
     \mathrm{\ as\ } n \rightarrow \infty. 
\end{align*}
There is an analog to the famous Erd\H{o}s-Kac theorem 
\cite{BILLINGSLY-CLT-PRIMEDIVFUNC,ERDOS-KAC-REF} 
characterizing the 
distribution of the distinct values of $\omega(n)$ for all $n \leq x$ at large $x$. 
Let $\sigma_x(C) := \sqrt{\log\log x}$ and 
$\mu_x(C) := \log\log x - \log\left(\frac{\sqrt{2\pi}A_0}{\zeta(2)(1+P(2))}\right)$ where 
$P(s)$ is the \emph{prime zeta function} \cite{FROBERG-1968} and $A_0 > 0$ is an absolute constant. 
Our theorem states that for any fixed $Y > 0$, 
we have uniformly for any $-Y \leq y \leq Y$ as $x \rightarrow \infty$ that 
\begin{align} 
\label{eqn_ginvAbsValueSeq_DistByShiftedStandardNormalCDF_v1} 
\frac{1}{x} \#\left\{3 \leq n \leq x: 
     \frac{|g^{-1}(n)|}{(\log n) \sqrt{\log\log n}} - 
     \frac{6}{\pi^2 n (\log n) \sqrt{\log\log n}} 
     \times \sum_{k \leq n} |g^{-1}(k)| \leq y\right\} 
     & = 
     \Phi\left(\frac{\frac{\pi^2 y}{6}-\mu_x(C)}{\sigma_x(C)}\right) + 
     o(1). 
\end{align}
The function $\Phi(z) = \frac{1}{\sqrt{2\pi}} \times \int_{-\infty}^z e^{-\frac{t^2}{2}} dt$ 
is the CDF of the standard normal distribution. 
%Moreover, if $x$ is sufficiently large and 
%if we pick any integer $n \in [2, x]$ uniformly at random, then 
%the following statement also holds as $x \rightarrow \infty$: 
%\begin{align*} 
%\mathbb{P}\left(|g^{-1}(n)| - \frac{6}{\pi^2 n} \times \sum_{k \leq n} |g^{-1}(k)| \leq 
%     \frac{6}{\pi^2} (\log n) \sqrt{\log\log n} 
%     \left(\alpha \sigma_x(C) + \mu_x(C)\right)
%     \right) & = 
%     \Phi\left(\alpha\right) + o(1), \text{ for } \alpha \in \mathbb{R}. 
%\end{align*} 
Let the partial sums of $g^{-1}(n)$ be defined as follows 
\cite[\seqnum{A341472}]{OEIS}: 
$G^{-1}(x) := \sum_{n \leq x} g^{-1}(n) = \sum_{n \leq x} \lambda(n) |g^{-1}(n)|$. 
We prove for all $x \geq 1$ the characterizations that 
%both characterizations of $M(x)$ by $G^{-1}(x)$ that 
\begin{align}
%M(x) & = \sum_{n \leq x} g^{-1}(n) \left[\pi\left(\Floor{x}{n}\right) + 1\right], \\ 
\notag 
M(x) & = G^{-1}(x) + G^{-1}\left(\Floor{x}{2}\right) + 
     \sum_{1 \leq k < \frac{x}{2}} G^{-1}(k) \left[ 
     \pi\left(\Floor{x}{k}\right) - \pi\left(\Floor{x}{k+1}\right) 
     \right], \\ 
\label{eqn_Mx_GInvx_SumOverPrimesFlooredFormula_v3} 
%M(x) & = G^{-1}(x) + \sum_{\substack{p \leq x \\ p\mathrm{\ prime}}} G^{-1}\left(\Floor{x}{p}\right).
M(x) & = G^{-1}(x) + \sum_{\substack{p \leq x}} G^{-1}\left(\Floor{x}{p}\right), 
\end{align}
where $\pi(x) := \sum_{p \leq x} 1$ is the classical prime counting function \cite[\seqnum{A000720}]{OEIS}. 
My approach to $M(x)$ using the $\lambda$-sign-weighted distribution in 
\eqref{eqn_ginvAbsValueSeq_DistByShiftedStandardNormalCDF_v1} and the 
summatory function $G^{-1}(x)$ via 
\eqref{eqn_Mx_GInvx_SumOverPrimesFlooredFormula_v3} 
offers new hope and methodology to characterizing the asymptotics of this classical function. 
Indeed, these auxiliary sequences have a decidedly more additive nature with clearer structure and 
more insightful properties on inspection. 
There is a direct link between 
the summatory function $L(x)$ of the completely multiplicative function 
$\lambda(n)$ and the distribution of 
$|g^{-1}(n)|$. Hence, we see another non-classical 
concrete link between the distributions of $L(x)$ and $M(x)$. 
The new results in this article then 
connect the distributions of $L(x)$, an explicitly identified 
probability distribution, and $M(x)$ as $x \rightarrow \infty$.
Formalizing the properties of the distribution of 
$L(x)$ is viewed as a problem that is equally as difficult 
as understanding the distribution of $M(x)$ well along infinite 
subsequences of large integers 
(\cf \cite{HUMPHRIES-JNT-2013}). 

\vskip -0.5mm
\noindent
{\normalsize \textbf{\emph{Future related work, open problems and conjectures.}}} 
I plan on extending the work for this manuscript in a few directions 
as a postdoctoral fellow. 
The following is a list of open conjectures about the asymptotic behavior of $M(x)$ at 
large $x$ and along infinite subsequences I wish to make progress towards proving: 
\begin{itemize}[leftmargin=0.475in,itemsep=-1.4mm,topsep=0pt]

\item[\textbf{(M-A)}] We have that 
     \[
     \liminf_{x \rightarrow \infty} \frac{M(x)}{\sqrt{x}} = -\infty; 
     \quad\text{ and }\quad 
     \limsup_{x \rightarrow \infty} \frac{M(x)}{\sqrt{x}} = +\infty;
     \]
\item[\textbf{(M-B)}] There is an infinite sequence of integers $\{x_n\}_{n \geq 1}$ 
                      along which $M(x_n) = 0$ for all $n \geq 1$. That is, the Mertens 
                      function takes on arbitrarily small absolute values infinitely often. 
                      We have also conjectured that $M(x)$ 
                      changes sign infinitely often through the conjectures in \textbf{(M-A)}. 
\item[\textbf{(M-C)}] Let $\mathcal{M}_{-}(x) := \#\{n \leq x: M(n) < 0\}$ denote 
                      the set of positive integers less than $x$ for which the Mertens function 
                      is negative. As $x \rightarrow \infty$, the values of $M(x)$ assume a 
                      pronounced negative bias insomuch as the density 
                      \[
                      \frac{\mathcal{M}_{-}(x)}{x} > \frac{3}{5}(1+o(1)), \text{ as } x \rightarrow \infty. 
                      \]
                      We remark on how interesting this property is by noting that the asymptotic 
                      densities of the respective sets of squarefree integers for which 
                      $\mu(n) = \pm 1$ are proven to be equal: 
                      \[
                      \lim_{x \rightarrow \infty} \frac{1}{x} \times \#\{n \leq x: \mu(n) = +1\} = 
                      \lim_{x \rightarrow \infty} \frac{1}{x} \times \#\{n \leq x: \mu(n) = -1\} = 
                      \frac{3}{\pi^2} \approx 0.303964. 
                 \]

\end{itemize}
I have started work to generalize the new discoveries for the Mertens function case to 
characterize partial sums of arithmetic functions. 
The aim is to develop a standardized methodology by which we can associate the distribution of another  
strongly additive function in analog to the role $\omega(n)$ plays with $M(x)$ given any 
multiplicative function $f(n)$ or any $h(n)$ whose Dirichlet series, 
$\sum_{n \geq 1} h(n) n^{-s}$ convergent for 
$\Re(s) > \sigma_{a}(h) \geq \sigma_{c}(h)$, satisfies 
certain properties and is hence suitably well enough behaved. 

\subsection{Formal generating function methods in number theory and combinatorial analysis}

\vskip -0.5mm
\noindent
{\normalsize \textbf{\emph{Background and significant research experience.}}} 
My peer-reviewed research profile in mathematics since 2010 includes nearly two 
dozen articles written on the 
following topics: generalized Stirling numbers, polynomials and factorial functions, 
continued fraction representations for sequence generating functions (OGFs) leading to 
new congruences and properties modulo any $h \geq 2$, 
and varied methods of transformations of formal integer sequence OGFs.  
I have published in top journals in general mathematics and within my subject 
areas including articles in the \emph{Journal of Number Theory}, 
the \emph{American Mathematical Monthly}, \emph{Acta Artithmetica}, and the 
\emph{Ramanujan Journal}. 
%(see bibliography starting on page 
%\pageref{page_BibliographyB_subSection_MDSPubs}). 
I will become a co-author with 
Prof.~Ernie Croot at GA Tech in the near future on a preprint to be submitted to the 
\emph{Annals of Mathematics} on the $p$-divisibility of the 
central binomial coefficients. 
In 2019, I published a survey article on 
generating function transformation methods 
in a special issue of the journal \emph{Axioms} focused on 
mathematical analysis and its applications. 

\vskip -0.5mm
\noindent
{\normalsize \textbf{\emph{My doctoral thesis work at Georgia Tech.}}} 
My collaboration and coauthored work with M.~Merca from 2016--2018 
led to several theorems that characterize 
so-termed matrix-based \emph{factorization theorems} for Lambert series generating functions (LGFs) 
that provide unexpected new identities for multiplicative functions. 
This forms the basis for the work in my doctoral thesis 
at GA Tech under supervision of Prof.~Michael Lacey and Prof.~Josephine Yu. 
The work with Merca grew out of a common interest in matrix-based formulas from my 
\emph{Acta Arithmetica} article \cite{MDS-NO-7} to express the 
LGFs that formally enumerate the divisor sums $\sum_{d|n} f(d)$ that are characteristic of many 
multiplicative functions: 
\begin{equation}
\label{eqn_LGFFactThm_v1} 
\sum_{n \geq 1} \frac{f(n)q^n}{1-q^n} = \frac{1}{(q; q)_{\infty}} \times \sum_{n \geq 1} \left( 
     \sum_{k=1}^{n} s_{n,k} f(k)\right) q^n = 
     \sum_{m \geq 1} \left(\sum_{d|m} f(d)\right) q^m, |q| < 1. 
\end{equation}
In \eqref{eqn_LGFFactThm_v1}, the lower triangular sequence 
$s_{n,k} = [q^n](q; q)_{\infty} \times \frac{q^k}{1-q^k} = s_o(n, k) - s_e(n, k)$ where 
the functions $s_e(n, k)$ and $s_o(n, k)$ denote the number of $k$'s in all 
partitions of $n$ into an even (respectively odd) number of distinct parts. The matrix $s_{n,k}$ is 
invertible with $s_{n,k}^{-1} = \sum_{d|n} p(n-k) \mu\left(\frac{n}{d}\right)$. 
Hence, there is a concrete link connecting both the structures underneath 
multiplicative functions and the distinctively more additive theory of partitions that is 
apparent from the LGF factorization theorem results and its corollaries given in print. 

\vskip -0.5mm
\noindent
{\normalsize \textbf{\emph{Natural generalizations.}}}
A natural question for future work on these topics is why (restricted) partition functions 
are such a natural fit in connecting the generating functions 
that enumerate multiplicative functions within the context of 
\eqref{eqn_LGFFactThm_v1}? 
Suppose that we consider more general special convolution-type sums of the 
following form: $(f \boxdot_{\mathcal{D}} g)(n) := \sum_{k=1}^{n} f(k) g(n+1-k) \mathcal{D}(n, k)$. 
An open problem is to find the corresponding ``canonically best'' pairing of sequences to expand a matrix-based 
factorization theorem in analog to \eqref{eqn_LGFFactThm_v1}. 
In the last, more expository, section of my doctoral dissertation, I consider quantitative metrics in the 
form of sequence cross-correlation statistics 
for the qualitative observation of ``good (or most revealing, or interesting) fit`` 
witnessed between multiplicative and partition theoretic functions in the LGF case above. 
In particular, if the lower triangular kernel function 
$\mathcal{D}(n, k)$ is invertible, then we define 
\begin{align}
\label{eqn_CrossCorrelationStatFormula_for_fixed_CDFuncs} 
\operatorname{Corr}(n; \mathcal{C}, \mathcal{D}) & := 
     \frac{1}{n} \times \frac{\sum\limits_{k=1}^{n} |c_k(\mathcal{C}) \mathcal{D}^{-1}(n, k)|}{ 
     \sqrt{\left(\sum\limits_{k=1}^{n} c_k(\mathcal{C})^2\right) \left( 
     \sum\limits_{k=1}^{n} \mathcal{D}^{-1}(n, k)^2\right)}}
\end{align}
\vskip -0.5mm
\noindent
{\normalsize \textbf{\emph{Open questions.}}} 
Given a fixed kernel $\mathcal{D}$, what is the optimal OGF $\mathcal{C}(q)$ with 
integer (or rational) coefficients and where 
$\mathcal{C}(0) := 1$ that generates the sequence $\{c_n(\mathcal{C})\}_{n \geq 0}$ corresponding to 
a minimal possible correlation statistic, $\operatorname{Corr}(n; \mathcal{C}, \mathcal{D})$, defined in 
\eqref{eqn_CrossCorrelationStatFormula_for_fixed_CDFuncs}? 
Preliminary theorems I prove in my thesis combined with 
numerical results suggest that, indeed, in the LGF case, the optimal 
correlation statistic defined above corresponds to taking the factorizing 
OGF as the \emph{infinite $q$-Pochhammer symbol}, 
$\mathcal{C}(q) := (q; q)_{\infty} = \prod_{i \geq 1} (1-q^i)$ 
\cite{ANDREWS-THEORY-PARTITIONS,HARDYWRIGHT}, so that we obtain the 
observed expansions involving partition functions. 

\vskip -0.5mm
\noindent
{\normalsize \textbf{\emph{Ties to existing work and modern literature in number theory.}}} 
The ongoing study of this type of cross-correlation statistic in future work is an active and 
fruitful method for understanding complicated sums of these types 
that I will continue to pursue in my postdoctoral work. 
There is already a vast body of modern work in number theory that motivates 
semi-standardized ways to quantify relationships between functions and sequences we study via 
correlation based statistics. There is historically relevant literature about using 
statistical analysis to motivate studying the structure underneath number theoretic objects. 
For example, the non-trivial zeros of the Riemann zeta function have been related and 
bounded via pair correlation formulas. 
Results in analytic number theory that make sense of the distribution of the 
non-trivial zeros of $\zeta(s)$ originated in the work of Montgomery. 
Subsequent follow-up work that collectively builds on Montgomery’s contributions in the 
context of $L$-functions, Gaussian unitary ensemble (GUE), 
random matrix theory applications and associated correlation statistics is 
famously due to Hejal, Rudnick, Sarnak and Odlyzko 
\cite{WILLIAMS-BARRETT-SURVEY-2016}. 

\section{Broader impacts of proposed activities, future goals and career trajectory}

\subsection{Long-term career goals, research objectives and methodology} 

My long-term professional goal is to become a top-tier research mathematician. In making this happen, 
I can advance and make progress on important topics of merit and broad interest in my areas of study. 
I strive to continue to contribute high quality open source software, educationally 
literate publicly available source code advancing STEM areas, and to grow myself as a 
professional software developer. 
I seek to find interesting intersections and interplay between these fields while making 
a difference to others through my engagement working on this new research. 
A career goal of mine is to never stagnate by always keeping things 
fresh, open and interesting. I will continue to dare to take on the uncommon risk of enjoying 
challenging research problems that I encounter within new fields and 
sub-branches of mathematics. 
I will pursue stimulating research problems 
with cross-disciplinary applications in mathematics, computer science and OSS. I intend to 
publish the results of my research on these topics in peer-reviewed journals, present 
the results through talks at professional conferences, and make the research broadly 
available for education, teaching, and other purposes in venues such as the web. As 
woman working within my areas of study, I will continue to promote new learning 
and research mentorship opportunities to encourage diversity and help to bridge the gender 
gap for the under-represented talented women in these fields.
I will make these plans a reality by leading seminars at PSU targeted at broader outreach by getting 
women and minority students at or below the graduate level involved in fulfilling extracurricular roles 
presenting their research and mentoring other students in math at PSU.

\subsection{Active work on open source software projects}

I am an active user of the \emph{open source software} (OSS) based 
platforms Linux and OpenBSD since my 
teenage years. I was exposed to the availability, education, superior documentation and 
freely available source to high-quality production software that runs the backbone of all large 
computer networks early on by teaching myself how to use these systems. This experience provides me 
insight, grounded philosophy and a great passionate love for OSS. I have gone to significant efforts to 
donate time to and develop publicly available OSS. This work contributing to OSS is 
both as an extension of my formal studies as 
a graduate student and to facilitate growth of my skill set as a professional software engineer. 
I actively develop and maintain 
over a dozen public cross-platform OSS projects on \emph{GitHub} 
written in the C/C++, Java, Python and assembly languages, among others.
I will actively hold a seminar focused on teaching all types of students and educators how to use and 
write software using OSS in mathematics as a postdoc at PSU. 

My responsibilities working with 
Prof.~Christine Heitsch as the \emph{gtDMMB} mathematical biology group's graduate RA and 
software engineer on record at GA Tech have allowed me to gain experience through 
hundreds of hours back-porting and extending OSS. This work that enables research and experimentation 
in mathematical and computational biology. 
In 2021, we co-authored an application note publication in \emph{BioInformatics} 
focused on the \emph{RNAStructViz} graphical tool for visualizing RNA secondary structures 
\cite{MDS-NO-21}. 
We are planning to publish other OSS I have developed with her group online this year providing 
Python bindings, or a Python accessible scripting interface, to the historic \emph{GTFold} software 
\cite{GTFOLD-PUB-REFERENCE} 
that provides computationally efficient predictions of RNA secondary structures. 
Another subset of my applied research interests in OSS since relocating to GA Tech in 2017 
is focused on security software (utilities and libraries) on contactless NFC smartcard and wireless RFID 
hardware technology. 
%For example, refer to the DESFire embedded firmware related projects referenced in the 
%\emph{Project Summary} statement and the relevant subsection of the bibliography on 
%page \pageref{page_BibliographyB_SubSection_DESFireProjectRefs}. 
%The freely available software I have developed from scratch includes an 
%Android phone based real-time NFC data logger (the \emph{CMLD} project)
%with over $500$ international users. 

\subsection{Teaching profile, philosophy and influences} 

I aim to interact with students by an active learning approach 
that includes combining new technologies into the classroom. 
I also seek to promote a flexible, friendly and open learning environment for 
students coming from diverse backgrounds to engage with me as the instructor from within.
While working on computational geometry projects funded by 
Prof.~Jayadev Athreya at the University of Washington from 2016--2017, 
I was offered an unforgettable opportunity to take part in mentoring advanced undergraduates 
in mathematics by teaching a self-created junior-level topics course. The course outline 
focused on getting students hands-on experience with experimental mathematics, 
gap distributions and spatial statistics by visualizing substitution tilings of the 
plane in the Python programming language while practicing 
standardized agile software development methodologies. 
My enthusiasm for teaching students 
has grown as I have developed more techniques to overcome shyness in large groups. 
I was promoted to be the first head TA of \emph{Integral Calculus} in the Fall of 2018 at GA Tech. 
I accepted the opportunity to teach a section of integral calculus 
as the instructor of record over the summer of 2021. 
This is an exciting learning experience for me to administer a large course 
and to expand my CV with more expansive requisite professional experience in academia. 

\section{How the NSF fellowship at PSU will help me in my career development} 

I have elected sponsorship for 
the postdoctoral fellowship from Prof.~R.~C.~Vaughan at Penn State University (PSU). 
One of the primary benefits and foremost reasons I chose to pursue PSU as my host 
institution is that I will get to work with world class professors and mathematicians 
at the top of their respective fields. 
The modern analytic number theory textbook by H.~Montgomery and Prof.~Vaughan 
is a key inspiration and influence on my work with the Mertens function, $M(x)$, summarized in 
Section \ref{subSection_PriorWork_MertensFunctionMx}. 
Once I began chatting and communicating mathematics with Prof.~Vaughan online this year in 2021, 
I felt an immediate intellectual connection and personal interest formed by 
talking with him about my work. 
I have so far learned new techniques in analytic number theory from him and have quickly 
acquired the benefit of several historical reference points with respect to $M(x)$. 
Another PSU faculty that I will have available to support me is 
George~E.~Andrews. Prof.~Andrews is a former president of the AMS, 
a prolific expert that has championed the theory of partitions throughout his career, and 
an admirable cataloger of the great collective notebooks of the nineteenth century 
mathematical genius, S.~Ramanujan. 
More tenure track faculty at PSU whose research interests are compatible with mine 
that I have been in contact with include A.~J.~Yee and A.~Malik. 
The combined faculty expertise, academic mentorship and professional development 
at PSU is thus an excellent fit, and so a 
recipe for success, for me to continue the proposed future research 
from Section \ref{Section_SummaryOfPriorWork}. 
My background and experience in symbolic computation and 
experimental mathematics will add breadth to the faculty at 
PSU that sets my goals and career trajectory apart from other 
distinguished applicants.


\newpage 
\renewcommand{\thesection}{B}
\section{Bibliography and software contributions} 
\label{page_Section_BibliographyB} 

%\vskip 0.1in
\subsection{Citations to external works}

\vskip -4mm
\begin{thebibliography}{10}

\bibitem{ANDREWS-THEORY-PARTITIONS}
Andrews, G.~E. \emph{The Theory of Partitions (Encyclopedia of Mathematics and its Applications)}. 
Cambridge University Press, 1984.

\bibitem{APOSTOLANUMT}
Apostol, T.~M.
\newblock {\em Introduction to Analytic Number Theory}.
\newblock Springer--Verlag, 1976.

\bibitem{WILLIAMS-BARRETT-SURVEY-2016} 
Barrett O., Firk F.W.K., Miller S.J., Turnage-Butterbaugh C. 
\emph{From Quantum Systems to L-Functions: Pair Correlation Statistics and Beyond}. 
In: Nash, Jr. J., Rassias M. (eds) Open Problems in Mathematics. Springer, Cham. (2016) 
%\url{https://doi.org/10.1007/978-3-319-32162-2_2}

\bibitem{BILLINGSLY-CLT-PRIMEDIVFUNC}
Billingsley, P.
\newblock On the central limit theorem for the prime divisor function.
\newblock {\em Amer. Math. Monthly}, 76(2):132--139, 1969.

\bibitem{ERDOS-KAC-REF}
Erd{\H{o}}s, P. and Kac, M.
\newblock The Gaussian errors in the theory of additive arithmetic functions.
\newblock {\em American Journal of Mathematics}, 62(1):738--742, 1940.

\bibitem{ACOMB} 
Flajolet, P. and Sedgewick, R. Analytic Combinatorics. Cambridge University Press, 2009.

\bibitem{FROBERG-1968}
C.~E. Fr{\"{o}}berg.
\newblock On the prime zeta function.
\newblock {\em BIT Numerical Mathematics}, 8:87--202, 1968.

\bibitem{GKP}
Graham, R.~L., Knuth, D.~E. and Patashnik, O. 
\emph{Concrete Mathematics: A Foundation for Computer Science}. Addison-Wesley, 1994.

\bibitem{HARDYWRIGHT}
Hardy, G.~H. and Wright E.~M., editors.
\newblock {\em An Introduction to the Theory of Numbers}.
\newblock Oxford University Press, 2008.

\bibitem{HAVIL-GAMMA} 
Havil, J.  \emph{Gamma: Exploring Euler's constant}. Princeton University Press, 2003. 

\bibitem{HUMPHRIES-JNT-2013}
Humphries, P.
\newblock The distribution of weighted sums of the {L}iouville function and
  {P}\'{o}lya's conjecture.
\newblock {\em J. Number Theory}, 133:545--582, 2013.

\bibitem{IWANIEC-KOWALSKI}
Iwaniec, H. and Kowalski, E.
\newblock {\em Analytic Number Theory}, volume~53.
\newblock AMS Colloquium Publications, 2004.

\bibitem{ORDER-MERTENSFN}
Kotnik, T. and ~van~de Lune, J.
\newblock On the order of the {M}ertens function.
\newblock {\em Exp. Math.}, 2004.

\bibitem{MV}
Montgomery, H.~L. and Vaughan, R.~C.
\newblock {\em Multiplicative Number Theory: I. Classical Theory}.
\newblock Cambridge, 2006.

\bibitem{NG-MERTENS}
Ng, N.
\newblock The distribution of the summatory function of the {M}{\'{o}}bius
  function.
\newblock {\em Proc. London Math. Soc.}, 89(3):361--389, 2004.

\bibitem{ODLYZ-TRIELE}
Odlyzko, A.~M. and te~Riele, H.~J.~J. 
\newblock Disproof of the {M}ertens conjecture.
\newblock {\em J. Reine Angew. Math.}, 1985.

\bibitem{PRIMEREC}
Ribenboim, P.
\newblock {\em The new book of prime number records}.
\newblock Springer, 1996.

\bibitem{OEIS}
Sloane, N.~J.~A.
\newblock The {O}nline {E}ncyclopedia of {I}nteger {S}equences, 2021.
%\newblock \url{http://oeis.org}

\bibitem{SOUND-MERTENS-ANNALS}
Soundararajan, K.
\newblock Partial sums of the {M}{\"{o}}bius function.
\newblock {\em J. Reine Angew. Math.}, 2009(631):141--152, 2009.

\bibitem{GTFOLD-PUB-REFERENCE} 
Swenson, M.~S., Anderson, J., Ash, A., Gaurav, P., Sukos, Z., Bader, D.~A., 
Harvey, S.~C., and Heitsch, C.~E. 
\emph{GTfold: Enabling parallel RNA secondary structure prediction on multi-core desktops}. 
BMC Research Notes. 5(1): 341, 2012. 
%\url{https://github.com/gtDMMB/gtfold} 

\bibitem{TITCHMARSH}
Titchmarsh, E.~C.
\newblock {\em The theory of the {R}iemann zeta function}.
\newblock Oxford University Press, second edition, 1986.

\end{thebibliography}

\vskip -1.5mm
\hrule\medskip
\subsection{Peer-reviewed publications of the PI} 
\label{page_BibliographyB_subSection_MDSPubs} 

\vskip -4mm
\begin{thebibliography}{10}

\bibitem{MDS-NO-10}
Merca, M. and Schmidt, M.~D. \emph{A partition identity related to Stanley's theorem}. 
Amer. Math. Monthly 125 {\bf 10}: 929--933 (2018). 
%\url{https://doi.org/10.1080/00029890.2018.1521232}

\bibitem{MDS-NO-16}
Merca, M. and Schmidt, M.~D. \emph{Factorization theorems for generalized Lambert series and applications}. 
Ramanujan J. {\bf 51}: 391--419 (2020). 
%\url{https://doi.org/10.1007/s11139-018-0095-7}

\bibitem{MDS-NO-12}
Merca, M. and Schmidt, M.~D. \emph{Generating special arithmetic functions by Lambert series factorizations}. 
Contrib. Discrete Math. 14 {\bf (1)}: 31--45 (2019). 

\bibitem{MDS-NO-9}
Merca, M. and Schmidt, M.~D. \emph{The partition function $p(n)$ in terms of the classical M\"{o}bius function}. 
Ramanujan J. {\bf 49}: 87--96 (2019). 

\bibitem{MDS-NO-18}
Mousavi, H. and Schmidt, M.~D. \emph{Factorization theorems for relatively prime divisor sums, 
                                     GCD sums and generalized Ramanujan sums}. 
Ramanujan J. {\bf 54}: 309--341 (2021). 
%\url{http://doi.org/10.1007/s11139-020-00323-5} 

\bibitem{MDS-NO-2}
Schmidt, M.~D. \emph{A computer algebra package for polynomial sequence recognition}. 
Illinois IDEALS (2014). 
%\url{https://www.ideals.illinois.edu/handle/2142/49378}

\bibitem{MDS-NO-17}
Schmidt, M.~D. \emph{A short note on integral transformations and 
                     conversion formulas for sequence generating functions}. 
Axioms Special Issue on Mathematical Analysis and Applications II 8 {\bf 2}, 62 (2019). 
%\url{https://doi.org/10.3390/axioms8020062} 

\bibitem{MDS-NO-13}
Schmidt, M.~D. \emph{Combinatorial identities for generalized Stirling numbers 
                     expanding $f$-factorial functions and the $f$-harmonic numbers}. 
J. Integer Seq. 21 {\bf 18.2.7} (2018). 

\bibitem{MDS-NO-19}
Schmidt, M.~D. \emph{Combinatorial sums and identities involving generalized divisor functions with bounded divisors}. 
Integers 20 {\bf A85} (2020). 

\bibitem{MDS-NO-6}
Schmidt, M.~D. \emph{Continued fractions and $q$-series generating functions for the 
                     generalized sum-of-divisors functions}. 
J. Number Theory 180: 579--605 (2017). 
%\url{https://doi.org/10.1016/j.jnt.2017.05.023}

\bibitem{MDS-NO-8}
Schmidt, M.~D. \emph{Continued Fractions for Square Series Generating Functions}. 
Ramanujan J. {\bf 46}: 795--820 (2018). 
%\url{https://doi.org/10.1007/s11139-017-9971-9}

\bibitem{MDS-NO-5}
Schmidt, M.~D. \emph{Generating function transformations related to 
                     polylogarithm functions and the $k$-order harmonic numbers}. 
Online J. Anal. Comb. 12 {\bf 2} (2017). 

\bibitem{MDS-NO-20}
Schmidt, M.~D. \emph{Exact formulas for the generalized sum-of-divisors functions}. 
Integers 21 {\bf A19} (2021). 

\bibitem{MDS-NO-1}
Schmidt, M.~D. \emph{Generalized $j$-factorial functions, polynomials, and applications}. 
J. Integer Seq. 13 {\bf 10.6.7}  (2010). 

\bibitem{MDS-NO-11}
Schmidt, M.~D. \emph{Jacobi-type continued fractions and congruences for 
                     binomial coefficients modulo integers $h \geq 2$}. 
Integers 18 {\bf A46} (2018). 

\bibitem{MDS-NO-3}
Schmidt, M.~D. \emph{Jacobi-type continued fractions for the ordinary generating functions of 
                     generalized factorial functions}. 
J. Integer Seq. 20 {\bf 17.3.4} (2017). 

\bibitem{MDS-NO-14}
Schmidt, M.~D. \emph{New congruences and finite difference equations for generalized factorial functions}. 
Integers 18 {\bf A78} (2018). 

\bibitem{MDS-NO-7}
Schmidt, M.~D. \emph{New recurrence relations and matrix equations for 
                     arithmetic functions generated by Lambert series}. 
Acta Arith. 181 (2017): 355-367. 
%\url{http://doi.org/10.4064/aa170217-4-8} 

\bibitem{MDS-NO-21}
Schmidt, M.~D., Kirkpatrick, A., and Heitsch, C. \emph{RNAStructViz: graphical base pairing analysis}. 
Bioinformatics {\bf 197} (2021). 
%\url{https://doi.org/10.1101/2021.01.20.427505}

\bibitem{MDS-NO-4}
Schmidt, M.~D. \emph{Square series generating function transformations}. 
J. Inequal. Spec. Funct. 8 {\bf 2} (2017). 

\bibitem{MDS-NO-15}
Schmidt, M.~D. \emph{Zeta series generating function transformations related to 
                     generalized Stirling numbers and partial sums of the Hurwitz zeta function}. 
Online J. Anal. Comb. 13 {\bf 158}. (2018). 

\end{thebibliography}

\vskip -1.5mm
\hrule\medskip
\subsection{Additional manuscripts by the PI} 

\vskip -4mm
\begin{thebibliography}{10}

\bibitem{MDS-MANU-LSERIES} 
Schmidt, M.~D. \emph{A catalog of interesting and useful Lambert series identities}. 
Preprint (2020). 
arXiv/2004.02976.
%\url{https://arxiv.org/abs/2004.02976} 

\bibitem{MDS-MSTHESIS-EXT}
Schmidt, M.~D. \emph{A computer algebra package for polynomial sequence recognition}. 
Preprint (2016). 
arXiv/1609.07301.

\bibitem{MDS-MANU-FACTTHMS-EXOTIC} 
Schmidt, M.~D. \emph{Factorization theorems for Hadamard products and 
                     higher-order derivatives of Lambert series generating functions}. 
Preprint (2017). 
arXiv/1712.00608.
%\url{https://arxiv.org/abs/1712.00608} 

\bibitem{MDS-MANU-JNT2021} 
Schmidt, M.~D. \emph{New characterizations of partial sums of the M\"{o}bius function}. 
Preprint (2021). 
arXiv/2102.05842.
%\url{https://arxiv.org/abs/2102.05842}

\bibitem{MDS-MANU-MERCA-FACTORPAIRS} 
Merca, M. and Schmidt, M.~D. \emph{New factor pairs for factorizations of Lambert series generating functions}. 
Preprint (2017). 
arXiv/1706.02359.
%\url{https://arxiv.org/abs/1706.02359}

\bibitem{MDS-MANU-PCGAPDISTS-TILINGS} 
Schmidt, M.~D. \emph{Pair correlation and gap distributions for substitution tilings and 
                     generalized Ulam sets in the plane}. 
Preprint (2017). 
arXiv/1707.05509.
%\url{https://arxiv.org/abs/1707.05509}

\end{thebibliography}

\vskip -1.5mm
\hrule\medskip
\subsection{DESFire security software project references (described in the Project Summary)} 
\label{page_BibliographyB_SubSection_DESFireProjectRefs} 

    \smallskip%\vskip 0.15in 
    {\normalsize%\small%\footnotesize 
        \begin{itemize}[leftmargin=0cm,itemsep=-1.3mm]

             \item[]%[$\blacktriangleright$] 
                  \emph{\textbf{Chameleon Mini crypto mod firmware extension}}: 
                  A modification of the stock Chameleon Mini firmware sources to enable 
                  cryptographically secure and integrity checked binary data uploads onto the device. \\ 
                  \🌐 \url{github/maxieds/ChameleonCryptoModFirmware} 
             
             \item[]%[$\blacktriangleright$] 
                  \emph{\textbf{Chameleon Mini Live Debugger (CMLD)}}: 
                  An interactive NFC logging interface for 
                  Android OS phones that interfaces to Chameleon Mini hardware over USB. 
                  Over 500 active users on the Google \emph{Play Store}. \\ 
                  \🌐 \url{github/maxieds/ChameleonMiniLiveDebugger} 

             \item[]%[$\blacktriangleright$] 
                  \emph{\textbf{DESFire emulation support for the Chameleon Mini}}: 
                  The Chameleon Mini is a hardware tool for NFC debugging, card emulation, security testing, 
                  reconnaissance, and general purpose low-level data logging for contactless RFID cards 
                  like university IDs. 
                  This work enables embedded emulation support for the complex and proprietary 
                  Mifare DESFire type NFC tags on recent Chameleon Mini devices. \\ 
                  %Includes extensions of hardware accelerated cryptographic routines for AES/2K-DES/3DES. \\ 
                  \🌐 \url{github/emsec/ChameleonMini} \\ 
                  \🌐 \url{github/maxieds/ChameleonMiniDESFireStack} 

             \item[]%[$\blacktriangleright$] 
                  \emph{\textbf{Preprint manuscript}}: 
                  Schmidt, M.~D. \emph{A recent open source embedded implementation of the DESFire specification 
                  designed for on-the-fly logging with NFC based systems}. Preprint (2021). 

        \end{itemize}
    }

\hrule\medskip
\subsection{STEM supportive and educational software}

     \smallskip%\vskip 0.15in
     {\normalsize%\small%\footnotesize 
        \begin{itemize}[leftmargin=0cm,itemsep=-1.3mm] 
        
             \item[]%[$\blacktriangleright$] 
                  \emph{\textbf{GTFold Python}}: 
                  Python bindings and library to modernize and extend for the historical set of \emph{GTFold} 
                  command line utilities for use with Python. It is a scientific computing project to facilitate 
                  experimentation with RNA structures in computational biology. 
                  The source code will be released publicly on GitHub in late 2021. 

             \item[]%[$\blacktriangleright$] 
                  \emph{\textbf{Mathematically-oriented Unix fortune utility mod}}: 
                  A math-related add-on package providing terminal-based text to be displayed on the 
                  command line in the form of 
                  Unix fortune cookie wisdom. It features a custom \emph{Concrete Math} book style 
                  upper case $\Sigma$ summation ASCII-art graphic. \\ 
                  \🌐 \url{github/maxieds/math-fortune-mod} 

             \item[]%[$\blacktriangleright$] 
                  \emph{\textbf{Mertens function manuscript computational supplement}}: 
                  Facilitates computations with and exploration of the Mertens function, $M(x)$, 
                  in both \emph{SageMath} and \emph{Mathematica}. 
                  Software and supporting documentation written to accompany the publication of 
                  \cite{MDS-MANU-JNT2021}. \\ 
                  \🌐 \url{github/maxieds/MertensFunctionComputations}

             \item[]%[$\blacktriangleright$] 
                  \emph{\textbf{OptiKey ``Big Hacker'' keyboard extensions}}: 
                  Open source code and documentation that makes typing programming languages on-screen for users 
                  with disabilities more accessible. These extensible ``Big Hacker'' encoded 
                  keyboards are designed to simplify on-screen entry 
                  of programming languages. This task otherwise requires scrolling through a cell-phone-style 
                  nested set of keyboard screens to enter a single line of code in C++, Perl or Python. %\\ 
                  %\🌐 \url{https://github.com/OptiKey/OptiKey/wiki/Creating-and-Using-Dynamic-Keyboards} 

             \item[]%[$\blacktriangleright$] 
                  \emph{\textbf{Partitions into parts package}}: 
                  An extendable and expository Mathematica demo 
                  package for computing the number of partitions of a positive integer $n$ 
                  into parts of the form $pt+a$ for $p$ prime and $0 \leq a < p$. \\ 
                  \🌐 \url{github/maxieds/PartitionsIntoParts} 

             \item[]%[$\blacktriangleright$] 
                  \emph{\textbf{Prairie Learn contributor}}: 
                  Prairie Learn is an open source \emph{learning management system} (or LMS) 
                  that is a viable option to replace usage of the popular \textit{Canvas} LMS at many 
                  universities. It is actively developed at UBC and UIUC and is used on a 
                  private server form at UC Berkeley. 
                  I have so far contributed code to enable custom function names, symbolic constants, 
                  custom-defined operator symbols, and documentation available 
                  for use with \texttt{sympy} Python library parsing of internal 
                  \texttt{pl-symbolic-input} elements. This pull request enables crucial 
                  parsing for questions in calculus, mathematics and physics by enabling custom function names and 
                  symbolic constants. \\ 
                  \🌐 \url{github/PrairieLearn/PrairieLearn} 

             \item[]%[$\blacktriangleright$] 
                  \emph{\textbf{RNAStructViz}}: 
                  A cross-platform GUI-based application to visualize and compare RNA secondary structures 
                  that commonly arise in mathematical biology applications. 
                  See the application note in \cite{MDS-NO-21}. \\ 
                  %Supports standardized text-based 
                  %input formats for sequences and is able to export summary results and statistics to 
                  %common formats such as PNG, SVG and CSV. \\ 
                  \🌐 \url{github/gtDMMB/RNAStructViz/wiki} 

            \item[]%[$\blacktriangleright$] 
                 \emph{\textbf{Sage and Mathematica special sequence formula recognition packages}}: 
                  UIUC MS thesis software in both Mathematica (original) and Sage (extended). Designed to 
                  recognize formulas for sequences 
                  involving special combinatorial primitives and functions. \\ 
                  \🌐 \url{github/maxieds/GuessPolynomialSequences} \\ 
                  \🌐 \url{github/maxieds/sage-guess} 
             
             \item[]%[$\blacktriangleright$] 
                  \emph{\textbf{WXML tilings Python library}}: 
                  I was offered an unforgettable opportunity by Jayadev Athreya over 2016--2017 to take part in 
                  mentoring advanced undergraduates in mathematics. 
                  %by teaching a self-created topics course remotely 
                  %with the University of Washington. 
                  The course outline focused on getting students hands-on 
                  experience with experimental mathematics methodology, gap distributions and spatial statistics and 
                  visualizing substitution tilings of the plane in the Python programming language. \\ 
                  \🌐 \url{github/maxieds/WXMLTilingsHOWTO} 

        \end{itemize}
     } 

\hrule\medskip
\subsection{Other significant open source software}

     \smallskip%\vskip 0.15in
     {\normalsize%\small%\footnotesize 

        \begin{itemize}[leftmargin=0cm,itemsep=-1.3mm]

             \item[]%[$\blacktriangleright$] 
                  \emph{\textbf{Android file picker light library}}: 
                  A file and directory chooser widget library for Android OS that focuses on presenting an easy to 
                  configure lightweight UI. Designed from the top down to work with newer Android 10 and 11 
                  (API 29+) platforms in the future. \\ 
                  \🌐 \url{github/maxieds/AndroidFilePickerLight}

             \item[]%[$\blacktriangleright$] 
                  \emph{\textbf{Homebrew live streamer}}: 
                  A customizable, roll-your-own solution for live A/V recording to an Android phone device. 
                  It is also used 
                  with live media streaming to Facebook and YouTube for a transparent, open source non-proprietary 
                  application to perform the media streaming. 
                  The application was written to covertly record a private memento of a 
                  special three hour Smashing Pumpkins concert in Atlanta from 2018. \\ 
                  \🌐 \url{github/maxieds/HomeBrewLiveStreamer} 

             \item[]%[$\blacktriangleright$] 
                  \emph{\textbf{Mifare classic tool library}}: 
                  A Java and Android OS library wrapper around the functionality of the popular 
                  \emph{Mifare Classic Tool} (MCT) application for Android phones. \\ 
                  \🌐 \url{github/maxieds/MifareClassicToolLibrary} \\ 
                  \🌐 \url{github/maxieds/ChameleonMiniUSBInterface} 

          \end{itemize}

     }
                  
\end{document}
