%\documentclass[draft,12pt,reqno]{article} 
\documentclass[10pt,reqno,letterpaper]{article} 

%\usepackage[showframe,verbose=true,margin=1in,nohead,letterpaper]{geometry}
\usepackage[verbose=true,margin=1in,nohead,letterpaper]{geometry}

\usepackage{amsmath} 
\usepackage{amsfonts} 
\usepackage{amsthm}
\usepackage[hidelinks]{hyperref}
\usepackage[usenames,dvipsnames]{xcolor} 
\hypersetup{
    colorlinks,
    linkcolor={Gray!50!Black}, 
    citecolor={Gray!50!Black},
    urlcolor={Gray!50!Black}
}
\usepackage{amssymb} 
\usepackage{amscd} 
\usepackage{url} 
\usepackage{datetime}
\usepackage{tcolorbox}

\theoremstyle{plain} 
\newtheorem{theorem}{Theorem}
\newtheorem{conjecture}[theorem]{Conjecture}
\newtheorem{claim}[theorem]{Claim}
\newtheorem{prop}[theorem]{Proposition}
\newtheorem{lemma}[theorem]{Lemma}
\newtheorem{cor}[theorem]{Corollary}
\numberwithin{theorem}{section}

\theoremstyle{definition} 
\newtheorem{example}[theorem]{Example}
\newtheorem{remark}[theorem]{Remark}
\newtheorem{definition}[theorem]{Definition}
\newtheorem{notation}[theorem]{Notation}
\newtheorem{question}[theorem]{Question}
\newtheorem{discussion}[theorem]{Discussion}
\newtheorem{facts}[theorem]{Facts}
\newtheorem{summary}[theorem]{Summary}
\newtheorem{heuristic}[theorem]{Heuristic}
\newtheorem{ansatz}[theorem]{Ansatz}

%\usepackage[T1]{fontenc} 
\usepackage{MnSymbol}

%% : Customize the beleagueredBibliography and Citations for BibTeX: 
%\usepackage{natbib} 
%\bibpunct{[}{]}{;}{}{,}{,~} 

%% : Page Formatting and Related Configuration Options: 
\setlength{\parindent}{0in} 
\setlength{\parskip}{0.0325in} 
\setlength{\parsep}{0in} 
%\setlength{\headsep}{0.1in} 
%\setlength{\headheight}{0in} 
%\setlength{\topskip}{0in} 
%\setlength{\topmargin}{0in} 
%\setlength{\topsep}{0in} 
%\setlength{\partopsep}{0in} 
%\setlength{\pdfpagewidth}{8.5in} 
%\setlength{\pdfpageheight}{11in} 
%\setlength{\textwidth}{6.5in}
%\setlength{\textheight}{9in} 
%\setlength{\oddsidemargin}{-0.0in}
%\setlength{\evensidemargin}{-0.0in}
%\setlength{\topmargin}{-0.2in}

%% Define and setup special stylized section number headings:
\usepackage{tikz}
\usetikzlibrary{arrows}
\usepackage{adjustbox}

\newcommand{\CustomSectionNumberHeading}[1]{
     \vspace*{-1mm}
     \trimbox{1.8mm 1.6mm 5.8mm 2mm}{% LHS-SPC FROM-BOT RHS-SPC
          \begin{tikzpicture}
               \draw[fill,color=black] (-2mm,-2mm) rectangle (2mm,2mm);
               \draw[color=white] (0mm,0mm) node { \textbf{\large{#1}} };
          \end{tikzpicture}
     }
     \hspace*{-1.5mm}
}

\usepackage[compact]{titlesec} 
%\titlespacing{\section}{0pt}{-0.9mm}{-0.9mm} 
%\titlespacing{\subsection}{0pt}{-0.9mm}{-0.9mm} 
%\def\TitleSpacingAdjust{-0.895mm}
\def\TitleSpacingAdjust{-0.5mm}
\titlespacing{\section}{0pt}{\TitleSpacingAdjust}{\TitleSpacingAdjust} 
\titlespacing{\subsection}{0pt}{\TitleSpacingAdjust}{\TitleSpacingAdjust} 
%\titleformat{\section}{\bfseries\large}{\CustomSectionNumberHeading{\thesection}}{10pt}{}
\titleformat{\section}{\bfseries\large}{\thesection}{10pt}{}
%\titleformat{\subsection}{\bfseries\normalsize}{
%     \hspace*{-1mm}\scalebox{1.35}{$\blacktriangleright$}\hspace*{3.5mm}\thesubsection\ \ --\hspace*{-1.5mm}}{10pt}{} 
%\titleformat{\subsection}{\bfseries\normalsize}{\thesubsection\ \ --\hspace*{-1.5mm}}{10pt}{}
\titleformat{\subsection}{\bfseries\normalsize}{\thesubsection}{10pt}{} 

\usepackage{lastpage,fancyhdr}
\renewcommand{\headrulewidth}{0pt}
\setlength{\footskip}{12pt}
\pagestyle{fancy}
\fancyhf{}
\cfoot{\hrule\hrule\smallskip\textbf{Page\ \thepage\ of \pageref*{LastPage}}}

%% : Table Spacing and Formatting: 
\setlength{\abovecaptionskip}{0in} 
\setlength{\belowcaptionskip}{0in} 

%% : Change Various Labelings and Counter Defines: 
\renewcommand{\thesection}{\arabic{section}} 
\renewcommand{\thefootnote}{\fnsymbol{footnote}} 
\addtocounter{footnote}{1}

\newcommand{\ie}[0]{i.e.\ } 
\newcommand{\cf}[0]{\emph{cf.}\ } 
\newcommand{\etc}[0]{etc.\ } 
\newcommand{\PhD}[0]{Ph.D.\ } 

%% : Concrete mathematics Book Recurrence Notation (Special Case Triangles): 
\newcommand{\gkpSI}[2]{\ensuremath{\genfrac{\lbrack}{\rbrack}{0pt}{}{#1}{#2}}} 
\newcommand{\gkpSII}[2]{\ensuremath{\genfrac{\lbrace}{\rbrace}{0pt}{}{#1}{#2}}} 
\newcommand{\gkpEI}[2]{\ensuremath{\genfrac{\langle}{\rangle}{0pt}{}{#1}{#2}}} 
\newcommand{\gkpEII}[2]{\ensuremath{\left\langle\genfrac{\langle}{\rangle}{
            0pt}{}{#1}{#2}\right\rangle}} 

%% : Other math Mode Macros Used by the Article: 
\newcommand{\Iverson}[1]{\ensuremath{\left[#1\right]_{\delta}}} 
\newcommand{\Floor}[2]{\ensuremath{\left\lfloor \frac{#1}{#2} \right\rfloor}}

\usepackage{enumitem}
\usepackage{textcomp}  % Required for encoding \textbigcircle
\usepackage{scalerel}  % Required for emoji \scalerel
\def\🌐{\scalerel*{\includegraphics{../cv-resume/earth-globe-americas-1f30e.png}}{\textrm{\textbigcircle}}}
\def\OneFFourCFour{\scalerel*{\includegraphics{../cv-resume/paper-1F4C4.png}}{\textrm{$\qed$}}}

%% Set the default font to Helvetica:
\usepackage{helvet}
\renewcommand{\familydefault}{\sfdefault}

\newcounter{completeBibitemIncrementCtr}
\setcounter{completeBibitemIncrementCtr}{0}


\let\oldthebibliography\thebibliography
\let\endoldthebibliography\endthebibliography
\renewenvironment{thebibliography}[1]{
     \renewcommand{\refname}{} 
     \bibliographystyle{plain}
     \begin{oldthebibliography}{#1}
     \setlength{\itemsep}{0em}
     \setlength{\parskip}{0em}
     \setlength{\topsep}{0pt}
     \setlength{\partopsep}{0pt}
     \setcounter{enumiv}{\value{completeBibitemIncrementCtr}}
     \footnotesize 
}
{
     \setcounter{completeBibitemIncrementCtr}{\value{enumiv}}
     \end{oldthebibliography}
}

\newcommand{\seqnum}[1]{\href{http://oeis.org/#1}{\color{Gray!50!Black}{\underline{#1}}}}

%% Misc spacing (lines and space between equations and text): 
\linespread{1.0}
\AtBeginDocument{
     \setlength{\abovedisplayskip}{0.25pt} 
     \setlength{\abovedisplayshortskip}{0.25pt} 
     \setlength{\belowdisplayskip}{0.25pt} 
     \setlength{\belowdisplayshortskip}{0.25pt} 
     \setlength{\jot}{0pt} 
}
\allowdisplaybreaks

%% Start Content:
\begin{document}

\noindent 
{\bfseries\Large Maxie Dion Schmidt} \\[0.5ex]
{\bfseries\Large Teaching Statement} \\ 
\hrule\medskip

I aim to interact with students by an active learning approach 
that includes combining new technologies into the classroom. 
I also seek to promote a flexible, friendly and open learning environment for 
students coming from diverse backgrounds to engage with me as the instructor from within.
While working on computational geometry projects funded by 
Prof.~Jayadev Athreya at the University of Washington from 2016--2017, 
I was offered an unforgettable opportunity to take part in mentoring advanced undergraduates 
in mathematics by teaching a self-created junior-level topics course. The course outline 
focused on getting students hands-on experience with experimental mathematics, 
gap distributions and spatial statistics by visualizing substitution tilings of the 
plane in the Python programming language while practicing 
standardized agile software development methodologies. 
My enthusiasm for teaching students 
has grown as I have developed more techniques to overcome shyness in large groups. 
I was promoted to be the first head TA of \emph{Integral Calculus} in the Fall of 2018 at GA Tech. 
I accepted the opportunity to teach a section of integral calculus 
as the instructor of record over the summer of 2021. 
This is an exciting learning experience for me to administer a large course 
and to expand my CV with more expansive requisite professional experience in academia. 

I feel that I will be able to more broadly contribute 
to initiatives in mathematics and other STEM fields as a 
postdoctoral fellow. 
As woman working within my areas of study, I will continue to promote new learning 
and research mentorship opportunities to encourage diversity and help to bridge the gender 
gap for the under-represented talented women in these fields.
I also reflect positively on having benefited over the years from my undergraduate research experiences 
with Prof.~Bruce Reznick and 
mentoring experiences with Prof.~Bruce Berndt who was influential in 
developing my mathematical writing style.
Personally having been considered seriously as an 
undergraduate pursuing original research was encouraging to me through 
my mentors that has been mission critical to my academic success and development over the years. 
I plan to pass on the wisdom I have learned
by talking with these experienced career mathematicians to young motivated researchers as my 
own professional career moves forward. This especially applies to offering advice and 
influencing talented women in mathematics to develop their own early original 
research work as undergraduates and onward. 

I am an active user of the \emph{open source software} (OSS) based 
platforms Linux and OpenBSD since my 
teenage years. I was exposed to the availability, education, superior documentation and 
freely available source to high-quality production software that runs the backbone of all large 
computer networks early on by teaching myself how to use these systems. This experience provides me 
insight, grounded philosophy and a great passionate love for OSS. I have gone to significant efforts to 
donate time to developing publicly available OSS. 
This work on OSS is both as an extension of my formal studies as 
a graduate student and is utilized to grow my skill set as a professional software engineer. 
I actively develop and maintain 
over a dozen public cross-platform OSS projects on \emph{GitHub} 
written in the C/C++, Java, Python and assembly languages, among others. 
I strive to contribute high quality open source software, educationally 
literate publicly available source code advancing STEM areas, and to grow myself as a 
professional software developer. 
I will continue to pursue stimulating research problems 
with cross-disciplinary applications in mathematics, computer science and OSS. I intend to 
publish the results of my research on these topics in peer-reviewed journals, present 
the results through talks at professional conferences, and make the research broadly 
available for education, teaching, and other purposes in venues such as the web.

\end{document}
