%\documentclass[draft,12pt,reqno]{article} 
\documentclass[10pt,reqno,letterpaper]{article} 

%\usepackage[showframe,verbose=true,margin=1in,nohead,letterpaper]{geometry}
\usepackage[verbose=true,margin=1in,nohead,letterpaper]{geometry}

\usepackage{amsmath} 
\usepackage{amsfonts} 
\usepackage{amsthm} 
\usepackage[usenames,dvipsnames]{color} 
\usepackage{amssymb} 
\usepackage{amscd} 
\usepackage{url} 
\usepackage{datetime}
\usepackage{enumitem}

\theoremstyle{plain} 
\newtheorem{theorem}{Theorem}
\newtheorem{conjecture}[theorem]{Conjecture}
\newtheorem{claim}[theorem]{Claim}
\newtheorem{prop}[theorem]{Proposition}
\newtheorem{lemma}[theorem]{Lemma}
\newtheorem{cor}[theorem]{Corollary}
\numberwithin{theorem}{section}

\theoremstyle{definition} 
\newtheorem{example}[theorem]{Example}
\newtheorem{remark}[theorem]{Remark}
\newtheorem{definition}[theorem]{Definition}
\newtheorem{notation}[theorem]{Notation}
\newtheorem{question}[theorem]{Question}
\newtheorem{discussion}[theorem]{Discussion}
\newtheorem{facts}[theorem]{Facts}
\newtheorem{summary}[theorem]{Summary}
\newtheorem{heuristic}[theorem]{Heuristic}
\newtheorem{ansatz}[theorem]{Ansatz}

%\usepackage[T1]{fontenc} 
\usepackage{MnSymbol}

%% : Customize the beleagueredBibliography and Citations for BibTeX: 
\usepackage{natbib} 
\bibpunct{[}{]}{;}{}{,}{,~} 

%% : Page Formatting and Related Configuration Options: 
\setlength{\parindent}{0in} 
\setlength{\parskip}{0.0325in} 
\setlength{\parsep}{0in} 
%\setlength{\headsep}{0.1in} 
%\setlength{\headheight}{0in} 
%\setlength{\topskip}{0in} 
%\setlength{\topmargin}{0in} 
%\setlength{\topsep}{0in} 
%\setlength{\partopsep}{0in} 
%\setlength{\pdfpagewidth}{8.5in} 
%\setlength{\pdfpageheight}{11in} 
%\setlength{\textwidth}{6.5in}
%\setlength{\textheight}{9in} 
%\setlength{\oddsidemargin}{-0.0in}
%\setlength{\evensidemargin}{-0.0in}
%\setlength{\topmargin}{-0.2in}

%\usepackage[compact]{titlesec} 
%\titlespacing{\section}{0pt}{-0.75mm}{-0.75mm} 
%\titlespacing{\subsection}{0pt}{-0.75mm}{-0.75mm} 
%\titlespacing{\subsubsection}{0pt}{*0}{*0} 
%\titleformat{\subsection}{\normalsize\bfseries}{\thesubsection}{}{} 
%\titleformat{\subsubsection}{\small\bfseries}{\thesubsubsection}{}{} 

%% Define and setup special stylized section number headings:
\usepackage{tikz}
\usetikzlibrary{arrows}
\usepackage{adjustbox}

\newcommand{\CustomSectionNumberHeading}[1]{
     \vspace*{-1mm}
     \trimbox{1.8mm 1.6mm 5.8mm 2mm}{% LHS-SPC FROM-BOT RHS-SPC
          \begin{tikzpicture}
               \draw[fill,color=black] (-2mm,-2mm) rectangle (2mm,2mm);
               \draw[color=white] (0mm,0mm) node { \textbf{\large{#1}} };
          \end{tikzpicture}
     }
     \hspace*{-1.5mm}
}

\usepackage[compact]{titlesec} 
%\titlespacing{\section}{0pt}{-0.9mm}{-0.9mm} 
%\titlespacing{\subsection}{0pt}{-0.9mm}{-0.9mm} 
%\def\TitleSpacingAdjust{-0.895mm}
\def\TitleSpacingAdjust{-0.5mm}
\titlespacing{\section}{0pt}{\TitleSpacingAdjust}{\TitleSpacingAdjust} 
\titlespacing{\subsection}{0pt}{\TitleSpacingAdjust}{\TitleSpacingAdjust} 
%\titleformat{\section}{\bfseries\large}{\CustomSectionNumberHeading{\thesection}}{10pt}{}
\titleformat{\section}{\bfseries\large}{\thesection}{10pt}{}
%\titleformat{\subsection}{\bfseries\normalsize}{
%     \hspace*{-1mm}\scalebox{1.35}{$\blacktriangleright$}\hspace*{3.5mm}\thesubsection\ \ --\hspace*{-1.5mm}}{10pt}{} 
%\titleformat{\subsection}{\bfseries\normalsize}{\thesubsection\ \ --\hspace*{-1.5mm}}{10pt}{}
\titleformat{\subsection}{\bfseries\normalsize}{\thesubsection}{10pt}{}

\setlength{\parindent}{0.2in} 
\setlength{\parskip}{0.8mm} 
\setlength{\parsep}{0in} 

\pagestyle{empty}

\setlength{\abovedisplayskip}{0in} 
\setlength{\abovedisplayshortskip}{0in} 
\setlength{\belowdisplayskip}{0in} 
\setlength{\belowdisplayshortskip}{0in} 
\setlength{\jot}{0pt} 

%% : Table Spacing and Formatting: 
\setlength{\abovecaptionskip}{0in} 
\setlength{\belowcaptionskip}{0in} 

%% : Change Various Labelings and Counter Defines: 
\renewcommand{\thesection}{\arabic{section}} 
\renewcommand{\thefootnote}{\fnsymbol{footnote}} 
\addtocounter{footnote}{1}

\newcommand{\ie}[0]{i.e.\ } 
\newcommand{\cf}[0]{\emph{cf.}\ } 
\newcommand{\etc}[0]{etc.\ } 
\newcommand{\PhD}[0]{Ph.D.\ } 

%% : Concrete mathematics Book Recurrence Notation (Special Case Triangles): 
\newcommand{\gkpSI}[2]{\ensuremath{\genfrac{\lbrack}{\rbrack}{0pt}{}{#1}{#2}}} 
\newcommand{\gkpSII}[2]{\ensuremath{\genfrac{\lbrace}{\rbrace}{0pt}{}{#1}{#2}}} 
\newcommand{\gkpEI}[2]{\ensuremath{\genfrac{\langle}{\rangle}{0pt}{}{#1}{#2}}} 
\newcommand{\gkpEII}[2]{\ensuremath{\left\langle\genfrac{\langle}{\rangle}{
            0pt}{}{#1}{#2}\right\rangle}} 

%% : Other math Mode Macros Used by the Article: 
\newcommand{\Iverson}[1]{\ensuremath{\left[#1\right]_{\delta}}} 
\newcommand{\Floor}[2]{\ensuremath{\left\lfloor \frac{#1}{#2} \right\rfloor}}

\usepackage{textcomp}  % Required for encoding \textbigcircle
\usepackage{scalerel}  % Required for emoji \scalerel
\def\🌐{\scalerel*{\includegraphics{cv-resume/earth-globe-americas-1f30e.png}}{\textrm{\textbigcircle}}}
\def\OneFFourCFour{\scalerel*{\includegraphics{cv-resume/paper-1F4C4.png}}{\textrm{$\qed$}}}

%% Set the default font to Helvetica:
\usepackage{helvet}
\renewcommand{\familydefault}{\sfdefault}

%% Misc spacing (lines and space between equations and text): 
\linespread{1.0}
\AtBeginDocument{
     \setlength{\abovedisplayskip}{0.25pt} 
     \setlength{\abovedisplayshortskip}{0.25pt} 
     \setlength{\belowdisplayskip}{0.25pt} 
     \setlength{\belowdisplayshortskip}{0.25pt} 
     \setlength{\jot}{0pt} 
}
\allowdisplaybreaks

%% Start Content:
\begin{document}

\newcommand{\GRFPEssayTitle}{NSF Postdoctoral Fellowship Project Summary}

\section{Overview}

The primary research interests of the PI are in combinatorial and 
analytic number theory with an emphasis on 
integer sequences, generating function methods, continued fractions, 
software development, and experimental mathematics. 
The proposed host institution is Penn State University (PSU) under the direction of 
supporting mathematician Prof.~R.~C.~Vaughan. 
The expertise of the PSU faculty in both analytic and combinatorial number theory 
with a focus on partitions 
will provide the support needed to continue creative work in the areas in 
which the applicant (the PI; henceforth, MDS) already has $21+$ publications and manuscripts. 
The breadth of the tenure track PSU math faculty also offers unique 
prospects to get involved in new work with innovative mathematicians. 

\section{Intellectual Merit}

The work of MDS on the Mertens function, $M(x)$, over the past few years explores 
new unconventional connections between strong additivity and signed sums of 
multiplicative functions. Her research on this function connects $M(x)$ with a key 
unsigned auxiliary sequence and the characterization of its distribution with an 
explicit non-centrally normal probability distribution. A favorite quote of hers 
due to her NSF sponsor 
%, whose work is a significant influence, 
is reproduced as follows: 
``\emph{It is evident that the primes are randomly distributed but, 
      unfortunately, we do not know [yet] what 'random' means}.''
%She will present her recent work on these new characterizations of $M(x)$ at her invited talk 
%in the work in early career number theory special session of the \emph{AMS JMM 2022} in Seattle this coming year. 
The recent work of hers on cross-correlation statistics provides new ways to express deep 
connections of number theoretic objects, such as the distribution of the primes and 
divisor sums of multiplicative functions, to other branches of mathematics such as partitions. 
Her plans as a postdoc are to continue to publishing work in number theory 
and combinatorial analysis. This includes generalizing the new connections connecting signed sums of 
sequences besides $\mu(n)$ 
to additivity with explicit probability distributions and 
work on the open questions in her doctoral thesis research. 
The project description precisely motivates her proposed work 
%coordinated with the NSF 
at PSU and addresses significant open problems. 

\section{Broader Impacts}

%\subsection*{A personal tribute to RBG: DESFire firmware project and wireless security software} 
%\subsection*{Open source STEM related software, DESFire firmware project and wireless security software}
\label{subSection_TributeToRBG_DESFire} 

From 2020--2021, MDS worked on a larger scale, energizing and experimental 
security project that was a welcome challenge and software outlet away 
from her pure math research over the pandemic. 
The project brings sophisticated new 
low-level embedded firmware extensions to a type of NFC penetration testing and 
wireless data sniffer-loggers 
(the Chameleon Mini) that interfaces with common NFC tags (e.g., most university student ID cards). 
This project afforded a heightened platform online 
over which to bring 
more awareness by users to important issues 
%local US and global socioeconomic issues and disparity that 
that have increasingly beleaguered the world this pandemic. 
In particular, the first beta testing versions became a reality 
the long weekend of the sad passing of inspirational 
US supreme court justice Ruth Bader Ginsberg on 
Rosh Hashanah, to whom MDS was able to pay tribute by working on this software. 
MDS is proud to have ``hacked'' for freedom with free software, as she likes to say, for days on end 
while cloaked in an over-sized \emph{Black Lives Matter} shirt. 
%It is immortalized in a photo snapshot posted to the development page for the project. 
%The net gain from the project is that the complex and proprietary \emph{DESFire} NFC tag
%specification is now emulated by the Chameleon Mini hardware appreciated globally by 
%wireless security researchers. 
The experience working on this project speaks to the power of 
open source software (OSS) and how MDS intends to use its reach as a postdoc. 

MDS also feels that she will be able to more broadly contribute 
to initiatives in mathematics and other STEM fields as a 
NSF postdoctoral fellow at PSU.
One crucial component of the proposed activities at PSU is to continue collaborating on 
OSS with a focus on advancing educational STEM initiatives.  
For example, \emph{Prairie Learn} is a LMS that is a viable option to replace \textit{Canvas} at many 
institutions. It is actively developed at a few leading universities. 
%It is an excellent demonstration of the type of open source software work she will continue to 
%develop as a postdoc. 
She has so far contributed code to enhance 
the \texttt{sympy} Python library parsing within the project. This enables 
parsing for questions posed to students in mathematics and physical sciences. 
MDS will actively hold a seminar focused on teaching all types of students and educators how to use and 
write software using OSS in mathematics as a postdoc at PSU. 

%\subsection*{Advancing opportunities for women and minorities in mathematics}

MDS reflects positively on having benefited over the years from her undergraduate research experiences 
with Prof.~Bruce Reznick and 
mentoring experiences with Prof.~Bruce Berndt who was influential in 
developing her mathematical writing style.
MDS reflects being taken seriously as an 
undergraduate pursuing original research was particularly encouraging through 
her mentors. She plans to pass on the wisdom she has learned
by talking with these experienced career mathematicians to young motivated researchers as her 
own professional career moves forward. This especially applies to offering advice and 
influencing talented women in mathematics to develop their own early original 
research work as undergraduates and onward. 
MDS will make these plans a reality by leading seminars at PSU targeted at broader outreach by getting 
women and minority students at or below the graduate level involved in fulfilling extracurricular roles 
presenting their research and mentoring other students in math at PSU. 

\end{document}
