%\documentclass[draft,12pt,reqno]{article} 
\documentclass[10pt,reqno,letterpaper]{article} 

%\usepackage[showframe,verbose=true,margin=1in,nohead,letterpaper]{geometry}
\usepackage[verbose=true,margin=1in,nohead,letterpaper]{geometry}

\usepackage{amsmath} 
\usepackage{amsfonts} 
\usepackage{amsthm} 
\usepackage[usenames,dvipsnames]{color} 
\usepackage{amssymb} 
\usepackage{amscd} 
\usepackage{url} 
\usepackage{datetime}
\usepackage{enumitem}

\theoremstyle{plain} 
\newtheorem{theorem}{Theorem}
\newtheorem{conjecture}[theorem]{Conjecture}
\newtheorem{claim}[theorem]{Claim}
\newtheorem{prop}[theorem]{Proposition}
\newtheorem{lemma}[theorem]{Lemma}
\newtheorem{cor}[theorem]{Corollary}
\numberwithin{theorem}{section}

\theoremstyle{definition} 
\newtheorem{example}[theorem]{Example}
\newtheorem{remark}[theorem]{Remark}
\newtheorem{definition}[theorem]{Definition}
\newtheorem{notation}[theorem]{Notation}
\newtheorem{question}[theorem]{Question}
\newtheorem{discussion}[theorem]{Discussion}
\newtheorem{facts}[theorem]{Facts}
\newtheorem{summary}[theorem]{Summary}
\newtheorem{heuristic}[theorem]{Heuristic}
\newtheorem{ansatz}[theorem]{Ansatz}

%\usepackage[T1]{fontenc} 
\usepackage{MnSymbol}

%% : Customize the beleagueredBibliography and Citations for BibTeX: 
\usepackage{natbib} 
\bibpunct{[}{]}{;}{}{,}{,~} 

%% : Page Formatting and Related Configuration Options: 
\setlength{\parindent}{0in} 
\setlength{\parskip}{0.0325in} 
\setlength{\parsep}{0in} 
%\setlength{\headsep}{0.1in} 
%\setlength{\headheight}{0in} 
%\setlength{\topskip}{0in} 
%\setlength{\topmargin}{0in} 
%\setlength{\topsep}{0in} 
%\setlength{\partopsep}{0in} 
%\setlength{\pdfpagewidth}{8.5in} 
%\setlength{\pdfpageheight}{11in} 
%\setlength{\textwidth}{6.5in}
%\setlength{\textheight}{9in} 
%\setlength{\oddsidemargin}{-0.0in}
%\setlength{\evensidemargin}{-0.0in}
%\setlength{\topmargin}{-0.2in}

%\usepackage[compact]{titlesec} 
%\titlespacing{\section}{0pt}{-0.75mm}{-0.75mm} 
%\titlespacing{\subsection}{0pt}{-0.75mm}{-0.75mm} 
%\titlespacing{\subsubsection}{0pt}{*0}{*0} 
%\titleformat{\subsection}{\normalsize\bfseries}{\thesubsection}{}{} 
%\titleformat{\subsubsection}{\small\bfseries}{\thesubsubsection}{}{} 

%% Define and setup special stylized section number headings:
\usepackage{tikz}
\usetikzlibrary{arrows}
\usepackage{adjustbox}

\newcommand{\CustomSectionNumberHeading}[1]{
     \vspace*{-1mm}
     \trimbox{1.8mm 1.6mm 5.8mm 2mm}{% LHS-SPC FROM-BOT RHS-SPC
          \begin{tikzpicture}
               \draw[fill,color=black] (-2mm,-2mm) rectangle (2mm,2mm);
               \draw[color=white] (0mm,0mm) node { \textbf{\large{#1}} };
          \end{tikzpicture}
     }
     \hspace*{-1.5mm}
}

\usepackage[compact]{titlesec} 
%\titlespacing{\section}{0pt}{-0.9mm}{-0.9mm} 
%\titlespacing{\subsection}{0pt}{-0.9mm}{-0.9mm} 
%\def\TitleSpacingAdjust{-0.895mm}
\def\TitleSpacingAdjust{-0.5mm}
\titlespacing{\section}{0pt}{\TitleSpacingAdjust}{\TitleSpacingAdjust} 
\titlespacing{\subsection}{0pt}{\TitleSpacingAdjust}{\TitleSpacingAdjust} 
%\titleformat{\section}{\bfseries\large}{\CustomSectionNumberHeading{\thesection}}{10pt}{}
\titleformat{\section}{\bfseries\large}{\thesection}{10pt}{}
%\titleformat{\subsection}{\bfseries\normalsize}{
%     \hspace*{-1mm}\scalebox{1.35}{$\blacktriangleright$}\hspace*{3.5mm}\thesubsection\ \ --\hspace*{-1.5mm}}{10pt}{} 
%\titleformat{\subsection}{\bfseries\normalsize}{\thesubsection\ \ --\hspace*{-1.5mm}}{10pt}{}
\titleformat{\subsection}{\bfseries\normalsize}{\thesubsection}{10pt}{}

\setlength{\parindent}{0.2in} 
\setlength{\parskip}{0.8mm} 
\setlength{\parsep}{0in} 

\pagestyle{empty}

\setlength{\abovedisplayskip}{0in} 
\setlength{\abovedisplayshortskip}{0in} 
\setlength{\belowdisplayskip}{0in} 
\setlength{\belowdisplayshortskip}{0in} 
\setlength{\jot}{0pt} 

%% : Table Spacing and Formatting: 
\setlength{\abovecaptionskip}{0in} 
\setlength{\belowcaptionskip}{0in} 

%% : Change Various Labelings and Counter Defines: 
\renewcommand{\thesection}{\arabic{section}} 
\renewcommand{\thefootnote}{\fnsymbol{footnote}} 
\addtocounter{footnote}{1}

\newcommand{\ie}[0]{i.e.\ } 
\newcommand{\cf}[0]{\emph{cf.}\ } 
\newcommand{\etc}[0]{etc.\ } 
\newcommand{\PhD}[0]{Ph.D.\ } 

%% : Concrete mathematics Book Recurrence Notation (Special Case Triangles): 
\newcommand{\gkpSI}[2]{\ensuremath{\genfrac{\lbrack}{\rbrack}{0pt}{}{#1}{#2}}} 
\newcommand{\gkpSII}[2]{\ensuremath{\genfrac{\lbrace}{\rbrace}{0pt}{}{#1}{#2}}} 
\newcommand{\gkpEI}[2]{\ensuremath{\genfrac{\langle}{\rangle}{0pt}{}{#1}{#2}}} 
\newcommand{\gkpEII}[2]{\ensuremath{\left\langle\genfrac{\langle}{\rangle}{
            0pt}{}{#1}{#2}\right\rangle}} 

%% : Other math Mode Macros Used by the Article: 
\newcommand{\Iverson}[1]{\ensuremath{\left[#1\right]_{\delta}}} 
\newcommand{\Floor}[2]{\ensuremath{\left\lfloor \frac{#1}{#2} \right\rfloor}}

\usepackage{textcomp}  % Required for encoding \textbigcircle
\usepackage{scalerel}  % Required for emoji \scalerel
\def\🌐{\scalerel*{\includegraphics{cv-resume/earth-globe-americas-1f30e.png}}{\textrm{\textbigcircle}}}
\def\OneFFourCFour{\scalerel*{\includegraphics{cv-resume/paper-1F4C4.png}}{\textrm{$\qed$}}}

%% Set the default font to Helvetica:
\usepackage{helvet}
\renewcommand{\familydefault}{\sfdefault}

%% Misc spacing (lines and space between equations and text): 
\linespread{1.0}
\AtBeginDocument{
     \setlength{\abovedisplayskip}{0.25pt} 
     \setlength{\abovedisplayshortskip}{0.25pt} 
     \setlength{\belowdisplayskip}{0.25pt} 
     \setlength{\belowdisplayshortskip}{0.25pt} 
     \setlength{\jot}{0pt} 
}
\allowdisplaybreaks

%% Start Content:
\begin{document}

\newcommand{\GRFPEssayTitle}{NSF Postdoctoral Fellowship Project Summary}

\noindent
{\bf\LARGE{Data Management Plan}}
\smallskip\hrule\smallskip

\section{Plans for sharing the products of research}

The PI will pursue stimulating research problems 
with cross-disciplinary applications in mathematics, computer science and open source software (OSS). 
The PI strives to continue to contribute high quality open source software, educationally 
literate publicly available source code advancing STEM areas, and to grow herself as a 
professional and academic software developer. 
The PI intends to 
publish the results of her research on these topics in peer-reviewed journals, present 
the results through talks at professional conferences, and make the research broadly 
available for education, teaching, and other purposes in venues such as the web. 
This includes plans of the PI to widely disseminate her research and make it 
freely available to others via preprint manuscript servers such as \emph{arXiv} and 
on public open source software repositories including \emph{GitHub}. 

The PI plans to publish software 
contributions to support her work in pure mathematics including releasing files, scripts and 
other programs that were key in motivating new discoveries and conjectures in print publications 
online in open venues. 
Other STEM supportive software contributions made while an active recipient of the NSF fellowship award, 
including extensions of the projects referenced from the bibliography 
section of the \emph{Project Description} statement, will continue to be made widely available to the 
general public online. 
The PI's experience with open source software provides her with  
insight, grounded philosophy and a great passionate love for OSS. 
As such, unless pre-existing license agreements for derivative software sources prevent her plans, the PI 
intends to release all software she writes related to her work 
funded by the NSF fellowship as OSS under a permissive 
license such as the \emph{GNU Public License} (GPL-V3) or a 
\emph{Creative Commons} (CC) license variant. 

\section{Plans for data management} 

The bulk of the work proposed for the fellowship project is to conduct research in pure mathematics. 
For work of this type, the PI does not require the collection of data from external sources nor subjects. 
That is to say that while she may use experimental mathematics to motivate and 
explore new topics in her work, saving the results on her personal computers in the process, there is no 
need for a rigorously structured data management methodology for her work in these areas. 
Likewise, the PI does not plan to develop software funded by the NSF that relies on privileged, 
confidential, government classified, nor otherwise sensitive data sets. 
The PI similarly has no plans to create software that relies on the interplay of personally 
identifiable information of human subjects nor participants in the development process of these works, 
such as general users and pre-release beta testers. 
Hence, the PI requires no extensive plans nor statements of protocol for data management in the 
software type work within her proposed project. 
The PI will of course strictly adhere to any guidelines or mandated requirements by her sponsoring 
institution, Penn State University, in handling her own private information and 
maintaining the computer systems she uses to conduct her research.  

\end{document}
