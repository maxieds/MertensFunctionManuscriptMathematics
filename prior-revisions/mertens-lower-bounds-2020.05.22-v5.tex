\documentclass[11pt,reqno,a4letter]{article} 

\usepackage{amsthm,amsfonts,amscd,amsmath}
\usepackage[hidelinks]{hyperref} 
\usepackage{url}
\usepackage[usenames,dvipsnames]{xcolor}
\hypersetup{
    colorlinks,
    linkcolor={green!63!darkgray},
    citecolor={blue!70!white},
    urlcolor={blue!80!white}
}

\usepackage[normalem]{ulem}
\usepackage{graphicx} 
\usepackage{datetime} 
\usepackage{cancel}
\usepackage{subcaption}
\captionsetup{format=hang,labelfont={bf},textfont={small,it}} 
\numberwithin{figure}{section}
\numberwithin{table}{section}

\usepackage{stmaryrd,tikzsymbols,mathabx} 
\usepackage{framed} 
\usepackage{ulem}
\usepackage[T1]{fontenc}
\usepackage{pbsi}


\usepackage{enumitem}
\setlist[itemize]{leftmargin=0.65in}

\usepackage{rotating,adjustbox}

\usepackage{diagbox}
\newcommand{\trianglenk}[2]{$\diagbox{#1}{#2}$}
\newcommand{\trianglenkII}[2]{\diagbox{#1}{#2}}

\let\citep\cite

\newcommand{\undersetbrace}[2]{\underset{\displaystyle{#1}}{\underbrace{#2}}}

\newcommand{\gkpSI}[2]{\ensuremath{\genfrac{\lbrack}{\rbrack}{0pt}{}{#1}{#2}}} 
\newcommand{\gkpSII}[2]{\ensuremath{\genfrac{\lbrace}{\rbrace}{0pt}{}{#1}{#2}}}
\newcommand{\cf}{\textit{cf.\ }} 
\newcommand{\Iverson}[1]{\ensuremath{\left[#1\right]_{\delta}}} 
\newcommand{\floor}[1]{\left\lfloor #1 \right\rfloor} 
\newcommand{\ceiling}[1]{\left\lceil #1 \right\rceil} 
\newcommand{\e}[1]{e\left(#1\right)} 
\newcommand{\seqnum}[1]{\href{http://oeis.org/#1}{\color{ProcessBlue}{\underline{#1}}}}

\usepackage{upgreek,dsfont,amssymb}
\renewcommand{\chi}{\upchi}
\newcommand{\ChiFunc}[1]{\ensuremath{\chi_{\{#1\}}}}
\newcommand{\OneFunc}[1]{\ensuremath{\mathds{1}_{#1}}}

\usepackage{ifthen}
\newcommand{\Hn}[2]{
     \ifthenelse{\equal{#2}{1}}{H_{#1}}{H_{#1}^{\left(#2\right)}}
}

\newcommand{\Floor}[2]{\ensuremath{\left\lfloor \frac{#1}{#2} \right\rfloor}}
\newcommand{\Ceiling}[2]{\ensuremath{\left\lceil \frac{#1}{#2} \right\rceil}}

\DeclareMathOperator{\DGF}{DGF} 
\DeclareMathOperator{\ds}{ds} 
\DeclareMathOperator{\Id}{Id}
\DeclareMathOperator{\fg}{fg}
\DeclareMathOperator{\Div}{div}
\DeclareMathOperator{\rpp}{rpp}
\DeclareMathOperator{\logll}{\ell\ell}

\title{
       \LARGE{
       Lower bounds on the Mertens function $M(x)$ along infinite subsequences for large 
       $x \gg 5.43591 \times 10^{3313040}$ 
       } 
       %\\ 
       %\large{\it New unique lower bounds on $M(x) / \sqrt{x}$ along an asymptotically 
       %   huge infinite subsequence of reals} 
}
\author{{\Large Maxie Dion Schmidt} \\ 
        %{\normalsize \href{mailto:maxieds@gmail.com}{maxieds@gmail.com}} \\[0.1cm] 
        {\small Georgia Institute of Technology} \\[0.025cm] 
        {\small School of Mathematics} \\[0.025cm] 
        {\small Department of Number Theory} 
} 

\date{\small\underline{Last Revised:} \today\ \ -- \ \ Compiled with \LaTeX2e} 

\theoremstyle{plain} 
\newtheorem{theorem}{Theorem}
\newtheorem{conjecture}[theorem]{Conjecture}
\newtheorem{claim}[theorem]{Claim}
\newtheorem{prop}[theorem]{Proposition}
\newtheorem{lemma}[theorem]{Lemma}
\newtheorem{cor}[theorem]{Corollary}
\numberwithin{theorem}{section}

\theoremstyle{definition} 
\newtheorem{example}[theorem]{Example}
\newtheorem{remark}[theorem]{Remark}
\newtheorem{definition}[theorem]{Definition}
\newtheorem{notation}[theorem]{Notation}
\newtheorem{question}[theorem]{Question}
\newtheorem{discussion}[theorem]{Discussion}
\newtheorem{facts}[theorem]{Facts}
\newtheorem{summary}[theorem]{Summary}
\newtheorem{heuristic}[theorem]{Heuristic}

\renewcommand{\arraystretch}{1.25} 

\setlength{\textheight}{9in}
\setlength{\topmargin}{-.1in}
\setlength{\textwidth}{7.5in} 
\setlength{\evensidemargin}{-0.25in} 
\setlength{\oddsidemargin}{-0.25in} 

\usepackage{geometry}
%\newgeometry{top=0.65in, bottom=18mm, left=15mm, right=15mm, outer=2in, heightrounded, marginparwidth=1.5in, marginparsep=0.15in}
\newgeometry{top=0.65in, bottom=18mm, left=15mm, right=15mm, heightrounded, marginparwidth=0in, marginparsep=0.15in}

\usepackage{fancyhdr}
\pagestyle{empty}
\pagestyle{fancy}
\fancyhead[RO,RE]{M. D. Schmidt -- Prepared for Myself to Review -- on \today} 
\fancyhead[LO,LE]{}
\fancyheadoffset{0.005\textwidth} 

\setlength{\parindent}{0in}
\setlength{\parskip}{2cm} 

\renewcommand{\thefootnote}{\Alph{footnote}}
\makeatletter
\@addtoreset{footnote}{section}
\makeatother

%\usepackage{marginnote,todonotes}
%\colorlet{NBRefColor}{RoyalBlue!73} 
%\newcommand{\NBRef}[1]{
%     \todo[linecolor=green!85!white,backgroundcolor=orange!50!white,bordercolor=blue!30!black,textcolor=cyan!15!black,shadow,size=\small,fancyline]{
%     \color{NBRefColor}{\textbf{#1}
%     }
%     }
%}
\newcommand{\NBRef}[1]{}  

\newcommand{\SuccSim}[0]{\overset{_{\scriptsize{\blacktriangle}}}{\succsim}} 
\newcommand{\PrecSim}[0]{\overset{_{\scriptsize{\blacktriangle}}}{\precsim}} 

\input{glossaries-bibtex/PreambleGlossaries-Mertens}

\usepackage{tikz}
\usetikzlibrary{shapes,arrows}

\usepackage{enumitem} 

\allowdisplaybreaks 

\begin{document} 

\maketitle

\begin{abstract} 
The Mertens function, $M(x) = \sum_{n \leq x} \mu(n)$, is classically 
defined to be the summatory function of the M\"obius function 
$\mu(n)$. The 
Mertens conjecture stating that $|M(x)| < C \cdot \sqrt{x}$ with $C > 0$ for all 
$x \geq 1$ has a well-known disproof due to Odlyzko and t\'{e} Riele given in the early 1980's by computation of 
non-trivial zeta function zeros in conjunction with integral formulas expressing $M(x)$. 
It is conjectured and widely believed that $M(x) / \sqrt{x}$ changes sign infinitely often and grows 
unbounded in the direction of both $\pm \infty$ along subsequences of integers $x \geq 1$. 
Our proof this property of $M(x)/\sqrt{x}$, e.g., showing that 
$$\limsup_{x \rightarrow \infty} \frac{|M(x)|}{\sqrt{x}} = +\infty,$$ is not based on 
standard estimates of $M(x)$ we find from Mellin inversion, which are intimately 
(and stubbornly) tied to the 
intricate distribution of the non-trivial zeros of the Riemann zeta function. 
There is a distinct stylistic 
flavor and new element of combinatorial analysis 
peppered in with the standard methods from analytic, additive and elementary number theory. 
This stylistic tendency distinguishes 
our methods from other proofs of established conditional and unconditional 
upper, rather than lower, bounds on $M(x)$. 

\bigskip 
\noindent
\textbf{Keywords and Phrases:} {\it M\"obius function sums; Mertens function; summatory function; 
                                    arithmetic functions; 
                                    Dirichlet inverse; Liouville lambda function; prime omega functions; 
                                    prime counting functions; Dirichlet series and DGFs; 
                                    asymptotic lower bounds; Mertens conjecture. } \\ 
% 11-XX			Number theory
%    11A25  	Arithmetic functions; related numbers; inversion formulas
%    11Y70  	Values of arithmetic functions; tables
%    11-04  	Software, source code, etc. for problems pertaining to number theory
% 11Nxx		Multiplicative number theory
%    11N05  	Distribution of primes
%    11N37  	Asymptotic results on arithmetic functions
%    11N56  	Rate of growth of arithmetic functions
%    11N60  	Distribution functions associated with additive and positive multiplicative functions
%    11N64  	Other results on the distribution of values or the characterization of arithmetic functions
\textbf{Primary Math Subject Classifications (2010):} {\it 11N37; 11A25; 11N60; and 11N64. } 
\end{abstract} 

\bigskip\hrule\bigskip

\newpage
%\section{Reference on abbreviations, special notation and other conventions} 
\label{Appendix_Glossary_NotationConvs}
     \vskip 0in
     \printglossary[type={symbols},
                    title={Reference on special notation and other conventions},
                    style={glossstyleSymbol},
                    nogroupskip=true]


%\newpage
%\setcounter{tocdepth}{2}
%\renewcommand{\contentsname}{Listing of major sections and topics} 
%\tableofcontents 

\newpage
\section{Preface: Explanations of unconventional notions and preconceptions of asymptotics and 
         notation for asymptotic relations} 

We exphasize that the next itemized careful explanation of the subtle distinctions to our usage of 
what we consider to be traditional notation for asymptotic relations are key to 
understanding our choices of upper and lower bound expressions given throughout the article. 
Thus, to avoid any confusion that may linger as we begin to state our new results and bounds on the 
functions we work with in this article, we preface the article starting with this section detailing 
our precise definitions, meanings and assumptions on the uses of certain symbols, operators, and 
relations. The interpretation of this notation forms the core of how we choose 
to convey the growth rates of arithmetic functions on their domain of $x$ within this article 
when $x$ is taken to be very large as $x \rightarrow \infty$ 
\cite[\cf \S 2]{NISTHB} \cite{ACOMB-BOOK}. 

\subsection{Average order, similarity and approximation of asymptotic growth rates of quantities} 

\subsubsection{Similarity and average order (expectation)} 

We say that two functions $A(x), B(x)$ satisfy the relation $A \sim B$ if 
\[
\lim_{x \rightarrow \infty} \frac{A(x)}{B(x)} = 1. 
\] 
It is sometimes standard to express the \emph{average order} of an arithmetic function f as 
$f \sim h$, even when the values of $f(n)$ may actually non-monotonically 
oscillate in magnitude infinitely often. What the notation $f \sim h$ means when expressing 
the average order of $f$ is that 
$\frac{1}{x} \cdot \sum_{n \leq x} f(n) \sim h(x)$. 

For example, in the acceptably classic language of \cite{HARDYWRIGHT} we would normally write that 
$\Omega(n) \sim \log\log n$, even though technically, 
$1 \leq \Omega(n) \leq \frac{\log n}{\log 2}$. 
To be absolutely clear about notation, we intentionally do not re-use the $\sim$ relation by 
instead writing $\mathbb{E}[f(x)] = h(x)$ (as in expectation of $f$) 
to denote that $f$ has a limiting average order growing at the 
rate of $h$. 

A related conception of $f$ having a so-called \emph{normal order} of $g$ holds whenever 
$$f(n) = (1+o(1)) g(n), \mathrm{a.e.}$$
            
\subsubsection{Approximation} 
     
We choose to adpot the convention to write that $f(x) \approx g(x)$ if $|f(x) - g(x)| = O(1)$. 
That is, we write $f(x) \approx g(x)$ to denote that $f$ is approximately equal to $g$ at $x$ modulo at most a
small constant difference between the functions. 

The formula we prefer for the Abel summation variant of summation by parts 
to express finite sums of a product of two functions is stated as follows 
\cite[\cf \S 4.3]{APOSTOLANUMT} \footnote{
     Compare to the exact formula for \emph{summation by parts} of any arithmetic functions, $u_n,v_n$, 
     stated as in \cite[\S 2.10(ii)]{NISTHB} for $U_j := u_1+u_2+\cdots+u_j$ when $j \geq 1$: 
     \[
     \sum_{j=1}^{n-1} u_j \cdot v_j = U_{n-1} v_n + \sum_{j=1}^{n-1} U_j \left(v_j - v_{j+1}\right), n \geq 2. 
     \]
}: 
 
\begin{prop}[Abel Summation Integral Formula] 
\label{prop_AbelSummationFormula} 
Suppose that $t > 0$ is real-valued, and that $A(t) \sim \sum_{n \leq t} a(n)$ for some weighting 
arithmetic function $a(n)$ with $A(t)$ continuously differentiable on $(0, \infty)$. Furthermore, suppose that 
$b(n) \sim f(n)$ with $f$ a differentiable function of $n \geq 0$ -- that is, $f^{\prime}(t)$ exists and is smooth for all 
$t \in (0, \infty)$. 
Then for $0 \leq y < x$, where we typcially assume that the bounds of summation satisfy 
$x, y \in \mathbb{Z}^{+}$, we have that 
\[
\sum_{y < n \leq x} a(n) b(n) \sim A(x)b(x) - A(y)b(y) - \int_y^{x} A(t) f^{\prime}(t) dt. 
\] 
\end{prop}

\subsubsection{Vinogradov's notation for asymptotics} 

We use the conventional relations $f(x) \gg g(x)$ and $h(x) \ll r(x)$ to symbolically express that we should expect 
$f$ to be ``substantially`` larger than $g$, and $h$ to be ``significantly'' smaller than $r$, in asymptotic order 
(e.g., rate of growth when $x$ is large). In practice, we adopt a somewhat looser definition of these symbols which 
allows $f \gg g$ and $h \ll r$ provided that there are constants $C, D > 0$ such that whenever $x$ is sufficiently 
large we have that $f(x) \geq C \cdot g(x)$ and $h(x) \leq D \cdot r(x)$. This notation is sometimes called 
\emph{Vinogradov's asymptotic notation}. 

Another way of expressing our precise meaning of these relations is by writing 
$$f \gg g \iff g = O(f),$$ and $$h \ll r \iff r = \Omega(h),$$ using Knuth's well-trodden 
style of big-$O$ (and Landau notation) and big-$\Omega$ (Hardy-Littlewood notation)
symbols from the language of theoretical computer science and in the analysis of algorithms. 

\subsection{An unconventional pair of asymptotic relations employed to drop 
            lower-order terms in upper and lower bounds on arithmetic functions} 
         
We define two new defintions of relations for expressing limiting asymptotic bounds on functions by 
adapting notation for existing operators for clarity of the way we use them here.       
Namely, we say that $h(x) \SuccSim r(x)$ if $h \gg r$ as $x \rightarrow \infty$, and define 
the relation $\PrecSim$ similarly as 
$h(x) \PrecSim r(x)$ if $h \ll r$ as $x \rightarrow \infty$. 
This usage of the notation of $\SuccSim,\PrecSim$ intentionally breaks with the usual conventions for the use of 
the relations $\succsim,\precsim$. 
Our intentional distinct usage of these new relations is intended to 
simplify notation for limiting upper and lower bounds that are valid as $x \rightarrow \infty$. 

The use of the new (modified) notation for $\SuccSim$ is intended to capture both that we are conveying a 
lower (upper) bound for the 
function, and crucially that this lower bound is valid only when $x$ is very large, i.e., in some sense that the lower bound 
holds in the same sense as the relation $\sim$ for equality. 
This is a subtle distinction that comes into play when we use it later to state lower bounds 
within our new results. 

An intentionally mock example motivating this usage of these 
relations thats clarifies the point of this new use of notation appears below: 

\begin{example}
Suppose that exactly for all $x \geq 1$ we have 
\[
f(x) \geq -(\log\log\log x)^2 + 3 \times 10^{1000000} \cdot (\log\log\log x)^{1.999999999} + E(x), 
\]
where $E(x) = o\left((\log\log\log x)^2\right)$ and there is a complicated expression for $E(x)$ that requires 
more than $100000$ ascii characters to typeset accurately, e.g., is too exceedingly complicated to write down and 
include as a component of our expression for the terms in the primary bound. 
Then naturally, we prefer to work with only the expression
for the asymptotically dominant main term in the lower bounds on $f(x)$ stated above. 

Note that in this case the main term contribution 
does not dominate the bound until $x$ is very large, so that replacing the right-hand-side expression with just this 
term yields an invalid inequality except for in limiting cases. In this instance, we prefer to write 
\[
f(x) \SuccSim -(\log\log\log x)^2, \mathrm{\ as \ } x \rightarrow \infty, 
\]
or more conventionally applying this notation only to unsigned functions by writing 
\[
|f(x)| \SuccSim (\log\log\log x)^2, \mathrm{\ as \ } x \rightarrow \infty. 
\]
It is problematic to only write a lower bound expression for $f(x)$ that states 
\[
f(x) \geq -(\log\log\log x)^2. 
\]
The problem with the bound as stated in the previous equation is that 
there is a substantial (however, asymptotically negligible) initial range of $x \geq 1$ where the given lower bound is 
invalid. 
\end{example} 

\begin{remark}[Emphasizing the rationale of the use of the new notation]
We emphasize that our new uses of the traditional symbols as asymptotic 
relations are defined to simplify our results by dropping expressions involving more precise, exact terms 
that are nonetheless asymptotically insignificant. Instead, we choose to express upper and lower bounds in $x$ 
that form accurate statements in limiting cases as $x \rightarrow \infty$. 
This convention allows us to 
write out simplified bounds that still capture the most simple 
essence of the upper or lower bound as we choose to view it in this article without regard to technical corner cases 
of initial intervals of $x$ where the bound is invalid. 
\end{remark} 

\subsection{Asymptotic expansions and uniformity} 

Because a subset of the results we cite that are proved in the references 
provide statements of 
asymptotic bounds that hold \emph{uniformly} for $x$ large depending on parameters, 
we need to briefly make precise what our preconceptions are of this terminology. 
We introduce the notation for asymptotic expansions of a function $f: \mathbb{R} \rightarrow \mathbb{R}$ from 
\cite[\S 2.1(iii)]{NISTHB}. 

\subsubsection{Ordinary asymptotic expansions of a function} 

Let $\sum_{n} a_n x^{-n}$ denote a formal power series expansion in $x$ where we 
ignore any necessary conditions to guarantee convergence of the series. For each integer $n \geq 1$, suppose that 
\[
f(x) = \sum_{s=0}^{n-1} a_s x^{-s} + O(x^{-n}), 
\]
as $|x| \rightarrow \infty$ where this limiting bound holds for $x \in \mathbb{X}$ in some unbounded set 
$\mathbb{X} \subseteq \mathbb{R}, \mathbb{C}$. 
When such a bound holds, we say that $\sum_s a_s x^{-s}$ is a \emph{Poincar\'{e} asymptotic expansion}, 
or just \emph{asymptotic series expansion}, of $f(x)$ as $x \rightarrow \infty$ along the fixed set $\mathbb{X}$. 
The condition in the previous equation is equivalent to writing 
\[
f(x) \sim a_0 + a_1 x^{-1} + a_2 x^{-2} + \cdots; x \in \mathbb{X}, \mathrm{\ for \ } |x| \rightarrow \infty. 
\]
The prior two characterizations of an asymptotic expansion for $f$ are also equivalent to the 
statement that 
\[
x^n \left(f(x) - \sum_{s=0}^{n-1} a_s x^{-s}\right) \xrightarrow{x \rightarrow \infty} a_n. 
\] 

\subsubsection{Uniform asymptotic expansions of a function} 

Let the set $\mathbb{X}$ from the definition in the last subsection correspond to a 
closed sector of the form 
$$\mathbb{X} := \{x \in \mathbb{C}: \alpha \leq \operatorname{arg}(x) \leq \beta\}.$$ 
Then we say that the asymptotic property 
\[
f(x) = \sum_{s=0}^{n-1} a_s x^{-s} + O(x^{-n}), 
\]
from before holds \emph{uniformly} with respect to $\operatorname{arg}(x) \in [\alpha, \beta]$ as 
$|x| \rightarrow \infty$. 

Another useful, important notion of uniform asymptotic bounds is taken with respect to some parameter $u$ 
(or set of parameters, respectively) that ranges over the point set (point sets, respectively) 
$u \in \mathbb{U}$. In this case, if we have that the $u$-parameterized expressions 
\[
\left\lvert x^n\left(f(u, x) - \sum_{s=0}^{n-1} a_s(u) x^{-s}\right) \right\rvert, 
\]
are bounded for all integers $n \geq 1$ for $x \in \mathbb{X}$ as $|x| \rightarrow \infty$, then we say that 
the asymptotic expansion of $f$ holds \emph{uniformly} for $u \in \mathbb{U}$. 
Now the function $f \equiv f(u, x)$ and the 
asymptotic series coefficients $a_s(u)$ may have an implicit dependence on the parameter $u$. 
If the previous boundedness condition holds for all positive integers $n$, we write that 
\[
f(u, x) \sim \sum_{s=0}^{\infty} a_s(u) x^{-s}; x \in \mathbb{X}, \mathrm{\ as \ } |x| \rightarrow \infty, 
\]
and say that this asymptotic expansion, or bound, holds \emph{uniformly with respect to $u \in \mathbb{U}$}. 
For $u$ taken outside of $\mathbb{U}$, the stated bound may fail to be valid even for $x \in \mathbb{X}$ as 
$|x| \rightarrow \infty$. 

\newpage
\section{An introduction to the Mertens function} 
\label{subSection_MertensMxClassical_Intro} 

Suppose that $n \geq 1$ is a natural number with factorization into 
distinct primes given by 
$n = p_1^{\alpha_1} p_2^{\alpha_2} \cdots p_k^{\alpha_k}$. 
We define the \emph{M\"oebius function} to be the signed indicator function 
of the squarefree integers: 
\[
\mu(n) = \begin{cases} 
     1, & \text{if $n = 1$; } \\ 
     (-1)^k, & \text{if $\alpha_i = 1$, $\forall 1 \leq i \leq k$; } \\ 
     0, & \text{otherwise.} 
     \end{cases} 
\]
There are many known variants and special properties of the M\"oebius function 
and its generalizations \cite[\cf \S 2]{HANDBOOKNT-2004}. For our 
purposes we seek to explore the properties and asymptotics of weighted 
summatory functions over $\mu(n)$. 
The Mertens summatory function, or \emph{Mertens function}, is defined as 
\cite[\seqnum{A002321}]{OEIS} 
\begin{align*} 
M(x) & = \sum_{n \leq x} \mu(n),\ x \geq 1, \\ 
     & \longmapsto \{1, 0, -1, -1, -2, -1, -2, -2, -2, -1, -2, -2, -3, -2, 
     -1, -1, -2, -2, -3, -3, -2, -1, -2, -2, \ldots\}
\end{align*} 
A related function which counts the 
number of \emph{squarefree} integers than $x$ sums the average order of the M\"obius function as 
\cite[\seqnum{A013928}]{OEIS} 
\[ 
Q(n) = \sum_{n \leq x} \mu^2(n) \sim \frac{6x}{\pi^2} + O\left(\sqrt{x}\right). 
\] 
It is known that the asymptotic density of the positively versus negatively 
weighted sets of squarefree numbers are in fact equal as $x \rightarrow \infty$: 
\[
\mu_{+}(x) = \frac{\#\{1 \leq n \leq x: \mu(n) = +1\}}{Q(x)} \sim 
     \mu_{-}(x) = \frac{\#\{1 \leq n \leq x: \mu(n) = -1\}}{Q(x)} 
     \xrightarrow[x \rightarrow \infty]{} \frac{3}{\pi^2}. 
\]
While this limiting law suggests an even bias for the Mertens function, 
in practice $M(x)$ has an apparent unproven negative bias in its values. Moroever, the actual 
local oscillations between the approximate densities of the sets 
$\mu_{\pm}(x)$ lend an unpredictable nature to the function and characterize the function's 
oscillatory sawtooth shaped plot when viewed over the positive integers. 

\subsection{Properties} 

The conventional approach to evaluating the behavior of $M(x)$ for large 
$x \rightarrow \infty$ results from a formulation of this summatory 
function as a predictable exact sum involving $x$ and the non-trivial 
zeros of the Riemann zeta function for all real $x > 0$. 
This formula is expressed given the inverse Mellin transformation 
over the reciprocal zeta function. In particular, 
we notice that since 
\[
\frac{1}{\zeta(s)} = \int_1^{\infty} \frac{s \cdot M(x)}{x^{s+1}} dx, 
\]
we obtain that 
\[
M(x) = \lim_{T \rightarrow \infty}\ \frac{1}{2\pi\imath} \int_{T-\imath\infty}^{T+\imath\infty} 
     \frac{x^s}{s \cdot \zeta(s)} ds. 
\] 
This representation along with the standard Euler product 
representation for the reciprocal zeta function leads us to the 
exact expression for $M(x)$ when $x > 0$ given by the next theorem. 

\begin{theorem}[Analytic Formula for $M(x)$] 
\label{theorem_MxMellinTransformInvFormula} 
Assuming the Riemann Hypothesis (RH), we can show that there exists an infinite sequence 
$\{T_k\}_{k \geq 1}$ satisfying $k \leq T_k \leq k+1$ for each $k$ 
such that for any real $x > 0$ 
\[
M(x) = \lim_{k \rightarrow \infty} 
     \sum_{\substack{\rho: \zeta(\rho) = 0 \\ |\Im(\rho)| < T_k}} 
     \frac{x^{\rho}}{\rho \cdot \zeta^{\prime}(\rho)} - 2 + 
     \sum_{n \geq 1} \frac{(-1)^{n-1}}{n \cdot (2n)! \zeta(2n+1)} 
     \left(\frac{2\pi}{x}\right)^{2n} + 
     \frac{\mu(x)}{2} \Iverson{x \in \mathbb{Z}^{+}}. 
\] 
\end{theorem} 

A historical unconditional bound on the Mertens function due to Walfisz (1963) 
states that there is an absolute constant $C > 0$ such that 
$$M(x) \ll x \cdot \exp\left(-C \cdot \log^{3/5}(x) 
  (\log\log x)^{-3/5}\right).$$ 
Under the assumption of the RH, Soundararajan proved new updated estimates in 2009 
bounding $M(x)$ for large $x$ in the following forms \cite{SOUND-MERTENS-ANNALS}: 
\begin{align*} 
M(x) & \ll \sqrt{x} \cdot \exp\left(\log^{1/2}(x) (\log\log x)^{14}\right), \\ 
M(x) & = O\left(\sqrt{x} \cdot \exp\left( 
     \log^{1/2}(x) (\log\log x)^{5/2+\epsilon}\right)\right),\ 
     \forall \epsilon > 0. 
\end{align*} 
To date, 
considerably less has been conjectured about explicit lower bounds on $|M(x)|$ along 
subsequences. 

\subsection{Conjectures} 

The RH is equivalent to showing that 
$M(x) = O\left(x^{1/2+\varepsilon}\right)$ for any 
$0 < \varepsilon < \frac{1}{2}$. 
It is still unresolved whether 
\[ 
\limsup_{x\rightarrow\infty} |M(x)| / \sqrt{x} = \infty, 
\] 
although computational evidence suggests that this is a likely conjecture 
\cite{ORDER-MERTENSFN,HURST-2017}. 
There is a rich history to the original statement of the \emph{Mertens conjecture} which 
states that 
\[ 
|M(x)| < c \cdot x^{1/2},\ \text{ some absolute constant $c > 0$. }
\] 
Mertens conjecture was first verified by Mertens for $c = 1$ and $x < 10000$. 
Since its beginnings in 1897, the conjecture has been disproven by computation 
of low-lying zeta function zeros in a famous paper by 
Odlyzko and t\'{e} Riele from the early 1980's. 
Since the truth of the Mertens conjecture would have implied the RH, more recent attempts 
at bounding $M(x)$ favor determining the rate at which the function 
$M(x) / \sqrt{x}$ grows without bound towards both $\pm \infty$ along infinite 
subsequences. 

One of the most famous still unanswered questions about the Mertens 
function concerns whether $|M(x)| / \sqrt{x}$ is in actuality unbounded on the 
natural numbers. A precise statement of this 
problem is to produce an affirmative answer whether 
$\limsup_{x \rightarrow \infty} M(x) / \sqrt{x} = +\infty$ and 
$\liminf_{x \rightarrow \infty} M(x) / \sqrt{x} = -\infty$, or 
equivalently whether there is an infinite sequence of natural numbers 
$\{x_1, x_2, x_3, \ldots\}$ such that the magnitude of 
$M(x_i) x_i^{-1/2}$ grows without bound along the subsequence. 
Currently, an exact rigorous 
proof that $M(x) / \sqrt{x}$ is unbounded still remains elusive, though there is suggestive probabilistic 
evidence of this property established by Ng in 2008. 
We cite that prior to this point it is known that \cite[\cf \S 4.1]{PRIMEREC} 
\[
\limsup_{x\rightarrow\infty} \frac{M(x)}{\sqrt{x}} > 1.060\ \qquad (\text{now } 1.826054), 
\] 
and 
\[ 
\liminf_{x\rightarrow\infty} \frac{M(x)}{\sqrt{x}} < -1.009\ \qquad (\text{now } -1.837625), 
\] 
although based on work by Odlyzyko and t\'{e} Riele it seems probable that 
each of these limits should be $\pm \infty$, respectively 
\cite{ODLYZ-TRIELE,MREVISITED,ORDER-MERTENSFN,HURST-2017}. 

Extensive computational evidence has produced 
a conjecture due to Gonek (among attempts on limiting bounds by others) that in fact the limiting behavior of 
$M(x)$ satisfies 
that $$\limsup_{x \rightarrow \infty} \frac{|M(x)|}{\sqrt{x} 
(\log\log x)^{5/4}} = O(1).$$ 
While it seems to be widely believed that $|M(x)| / \sqrt{x}$ tends to $+\infty$ at a logarithmic rate 
along subsequences, infinitely tending factors such as the $(\log\log x)^{\frac{5}{4}}$ in Gonek's conjecture 
do not appear to readily fall out of work on bounds for $M(x)$ by existing methods. 

\newpage
\section{A summary outline: Listing the core logical steps and critical components to the proof} 

\subsection{Step-by-step overview} 

We offer another brief step-by-step summary overview of the critical components 
to our proof outlined in the next section of the introduction below 
that are proved piece-by-piece in the following sections of the article. 
This outline is provided to help 
the reader see our logic and proof methodology as easily and quickly as possible. 
As the proof methodology is new and relies on non-standard elements compared to more 
traditional methods of bounding $M(x)$, we hope that this sketch of the logical components 
to our new argument makes the article easier to parse. 
\begin{itemize} 

\item[\textbf{(1)}] We prove a matrix inversion formula relating the summatory 
           functions of an arithmetic function $f$ and its Dirichlet inverse $f^{-1}$ (for $f(1) \neq 0$). 
           See 
           Theorem \ref{theorem_SummatoryFuncsOfDirCvls} in 
           Section \ref{Section_PrelimProofs_Config}.  
\item[\textbf{(2)}] This crucial step provides us with an exact formula for $M(x)$ in terms of $\pi(x)$, the seemingly 
           unconnected prime counting function, and the 
           Dirichlet inverse of the shifted additive function $g(n) := \omega(n)+1$. This 
           formula is stated in \eqref{eqn_Mx_gInvnPixk_formula}.  
\item[\textbf{(3)}] We tighten an updated result from \cite[\S 7]{MV} providing summatory functions that indicate the parity of 
           $\lambda(n)$ using elementary arguments and offer more combinatorially flavored expansions of Dirichlet series 
           (see Theorem \ref{theorem_GFs_SymmFuncs_SumsOfRecipOfPowsOfPrimes}). 
           We use this result to sum $\sum_{n \leq x} \lambda(n) f(n)$ for particular non-negative arithmetic 
           functions $f$ when $x$ is large. 
\item[\textbf{(4)}] We then turn to the average order 
           asymptotics of the quasi-periodic $g^{-1}(n)$, estimating this inverse function's 
           limiting asymptotics for large $n \leq x$ as $x \rightarrow \infty$ in 
           Section \ref{Section_InvFunc_PreciseExpsAndAsymptotics}. 
           We eventually use these estimates to prove a substantially unique new lower bound formula 
           for the summatory function $G^{-1}(x) := \sum_{n \leq x} g^{-1}(n)$ along prescribed asymptotically large 
           infinite subsequences (see Theorem \ref{theorem_gInv_GeneralAsymptoticsForms}). 
\item[\textbf{(5)}] We spend some interim time in Section \ref{Section_ProofOfValidityOfAverageOrderLowerBounds} 
           carefully working out a rigorous justification for why the limiting lower bounds we obtain from average 
           order case analysis of certain arithmetic function approximations 
           we define are sufficient to prove the limit supremum corollary below 
           (our primary new significant result established the article). 
\item[\textbf{(6)}] When we return to step \textbf{(2)} 
           with our new lower bounds at hand, and bootstrap, we find ``magic'' in the form of 
           showing the unboundedness of $\frac{|M(x)|}{\sqrt{x}}$ 
           along a very large increasing infinite subsequence 
           of positive natural numbers. What we recover is a quick, and rigorous, proof of 
           Corollary \ref{cor_ThePipeDreamResult_v1} given in 
           Section \ref{subSection_TheCoreResultProof}. 
           
\end{itemize} 

\subsection{Diagramatic flowchart of the proof logic with references to results} 

\subsubsection*{Flowchart schematic diagram: } 

The next flowchart diagramed below shows how the seemingly disparate components of the proof are organized. 
It also indicates how the separate initial ``lands'' of material and corresponding sets of requisite results 
forming the connected components to steps $\mathcal{A}$, $\mathcal{B}$ and $\mathcal{C}$ (as viewed below) 
combine to form the next core stages of the proof. 

\tikzstyle{CoreComponent} = [diamond, draw, fill=blue!35, text width=4.5em, text badly centered, 
                             node distance=3cm, inner sep=0.1cm]
\tikzstyle{SubComponent} = [rectangle, draw, fill=blue!19, text width=4.5em, text centered, 
                            rounded corners, minimum height=4em, node distance=3cm]
\tikzstyle{MainResultComponent} = [ellipse, draw, fill=green, text width=4.5em, text centered, 
                            rounded corners, minimum height=4em, node distance=3cm]
\tikzstyle{ComponentConnectionLine} = [draw, -latex]

\begin{center}
\fbox{
\begin{tikzpicture}[node distance = 2cm, auto]
%% : == Nodes: 
\node[CoreComponent] (A)  {Step $\mathcal{A}$}; 
\node[SubComponent, right of=A]  (A2) {A.2}; 
\node[right of=A2] (CenterDiagram) {            };
\node[CoreComponent, right of=CenterDiagram] (B)  {Step $\mathcal{B}$}; 
\node[SubComponent, right of=B]  (B2) {B.2}; 
\node[SubComponent, below of=B2]  (B3) {B.3}; 
\node[MainResultComponent, below of=CenterDiagram] (C)  {Step $\mathcal{D}$}; 
\node[MainResultComponent, below of=C] (D)  {Step $\mathcal{E}$}; 
\node[CoreComponent, left of=A, below of=C] (AvgOrderProofs) {Step $\mathcal{C}$};
%% : == Arrows:
\path[ComponentConnectionLine, dashed, style={<->}] (A) -- (A2);
\path[ComponentConnectionLine, dashed, style={<->}] (A2) -- (C);
\path[ComponentConnectionLine] (AvgOrderProofs) -- (D);
\path[ComponentConnectionLine, dashed, style={<->}] (B) -- (B2);
\path[ComponentConnectionLine, dashed, style={<->}] (B2) -- (B3);
\path[ComponentConnectionLine, dashed, style={<->}] (B3) -- (B);
\path[ComponentConnectionLine] (A) -- (C);
\path[ComponentConnectionLine] (B) -- (C);
\path[ComponentConnectionLine] (C) -- (D);
\end{tikzpicture} 
}
\end{center}

\subsubsection*{Key to the diagram stages: } 
\begin{itemize}[noitemsep,topsep=0pt]

\item[\textbf{Step A:}] \textit{Citations and re-statements of existing theorems proved elsewhere}: 
     E.g., statements of non-trivial theorems and key results we need that are proved in the references. 
     \begin{itemize}[noitemsep,topsep=0pt] 
     \item[\textbf{A.A}] Key results and constructions: 
          \begin{itemize}[noitemsep,topsep=0pt]
          \item[--] \small{Theorem \ref{theorem_HatPi_ExtInTermsOfGz}} 
          \item[--] \small{Theorem \ref{theorem_MV_Thm7.20-init_stmt}} 
          \item[--] \small{Corollary \ref{theorem_MV_Thm7.20}} 
          \item[--] \small{The results, lemmas, and facts cited in Section \ref{subSection_OtherFactsAndResults}}
          \end{itemize} 
     \item[\textbf{A.2}] Lower bounds on the Abel summation based formula for $G^{-1}(x)$: 
          \begin{itemize}[noitemsep,topsep=0pt]
          \item[--] \small{Theorem \ref{theorem_GFs_SymmFuncs_SumsOfRecipOfPowsOfPrimes} 
                    (on page \pageref{proofOf_theorem_GFs_SymmFuncs_SumsOfRecipOfPowsOfPrimes})} 
          \item[--] \small{Proposition \ref{cor_PartialSumsOfReciprocalsOfPrimePowers}} 
          \item[--] \small{Theorem \ref{theorem_gInv_GeneralAsymptoticsForms}} 
          \item[--] \small{Lemma \ref{lemma_CLT_and_AbelSummation}} 
          \item[--] \small{Lemma \ref{lemma_lowerBoundsOnLambdaFuncParitySummFuncs}} 
          \end{itemize} 
     \end{itemize} 
\item[\textbf{Step B:}] \textit{Constructions of an exact formula for $M(x)$}: The exact formula we prove 
     uses special arithmetic functions with particularly ``nice'' properties and bounds. This choice of 
     the expression from Theorem \ref{theorem_SummatoryFuncsOfDirCvls} 
     is key to how far we have traveled along the new approaches in this article. 
     In particular, the additivity of $\omega(n)$ and the easily integrable logarithmically weighted bound on 
     $\pi(x)$ for large $x$ are indispensible components to why this proof works well. 
     \begin{itemize}[noitemsep,topsep=0pt] 
     \item[\textbf{B.B}] Key results and constructions: 
          \begin{itemize}[noitemsep,topsep=0pt]
          \item[--] \small{Corollary \ref{cor_Mx_gInvnPixk_formula}} (follows from 
                    Theorem \ref{theorem_SummatoryFuncsOfDirCvls} 
                    proved on page \pageref{proofOf_theorem_SummatoryFuncsOfDirCvls}) 
          \item[--] \small{Conjecture \ref{lemma_gInv_MxExample} (to a lesser expository only extent)} 
          \item[--] \small{Proposition \ref{prop_SignageDirInvsOfPosBddArithmeticFuncs_v1}} 
          \end{itemize} 
     \item[\textbf{B.2}] Asymptotics for the component functions $g^{-1}(n)$ and $G^{-1}(x)$: 
          \begin{itemize}[noitemsep,topsep=0pt]
          \item[--] \small{Theorem \ref{theorem_Ckn_GeneralAsymptoticsForms} 
                    (on page \pageref{proofOf_theorem_Ckn_GeneralAsymptoticsForms})} 
          \item[--] \small{Lemma \ref{lemma_AnExactFormulaFor_gInvByMobiusInv_v1}} 
          \end{itemize} 
     \item[\textbf{B.3}] Simplifying formulas for $g^{-1}(n)$ and $G^{-1}(x)$: 
          \begin{itemize}[noitemsep,topsep=0pt]
          \item[--] \small{Corollary \ref{cor_ASemiForm_ForGInvx_v1}} 
          \end{itemize} 
     \end{itemize} 
\item[\textbf{Step C:}] \textit{A justification for why lower bounds holding on average suffice}: 
     \begin{itemize}[noitemsep,topsep=0pt]
     \item[--] \small{Theorem \ref{theorem_CondAvgOrderGInvxSummatoryFunc_v1} 
               (proved on page \pageref{proofOf_theorem_CondAvgOrderGInvxSummatoryFunc_v1})} 
     \end{itemize} 
\item[\textbf{Step D:}] \textit{Re-writing the exact formula for $M(x)$}: Key interpretations used in 
     formulating the lower bounds based on the re-phrased integral formula. 
     \begin{itemize}[noitemsep,topsep=0pt]
     \item[--] \small{Proposition \ref{prop_Mx_SBP_IntegralFormula}} 
     \item[--] \small{Theorem \ref{theorem_gInv_GeneralAsymptoticsForms}} 
     \end{itemize} 
\item[\textbf{Step E:}] \textit{The Holy Grail}: A big leap twoards proving that 
     $\frac{|M(x)|}{\sqrt{x}}$ is unbounded in the limit supremum sense. 
     \begin{itemize}[noitemsep,topsep=0pt]
     \item[--] \small{Corollary \ref{cor_ThePipeDreamResult_v1} (on page \pageref{proofOf_cor_ThePipeDreamResult_v1})} 
     \end{itemize} 

\end{itemize} 

\newpage 
\section{An introduction to our new methodology: A concrete approach to bounding $M(x)$ from below} 

\subsection{Summing series over Dirichlet convolutions} 

\begin{theorem}[Summatory functions of Dirichlet convolutions] 
\label{theorem_SummatoryFuncsOfDirCvls} 
Let $f,g: \mathbb{Z}^{+} \rightarrow \mathbb{C}$ be any arithmetic functions such that $f(1) \neq 0$. 
Suppose that $F(x) := \sum_{n \leq x} f(n)$ and $H(x) := \sum_{n \leq x} h(n)$ denote the summatory 
functions of $f,g$, respectively, and that $F^{-1}(x)$ denotes the summatory function of the 
Dirichlet inverse $f^{-1}(n)$ of $f$. Then, letting the counting function $\pi_{f \ast h}(x)$ be defined 
as in the first equation below, we have the following equivalent expressions for the 
summatory function of $f \ast h$ for integers $x \geq 1$: 
\begin{align*} 
\pi_{f \ast h}(x) & = \sum_{n \leq x} \sum_{d|n} f(d) h(n/d) \\ 
     & = \sum_{d \leq x} f(d) H\left(\Floor{x}{d}\right) \\ 
     & = \sum_{k=1}^{x} H(k) \left[F\left(\Floor{x}{k}\right) - 
     F\left(\Floor{x}{k+1}\right)\right]. 
\end{align*} 
Moreover, we can invert the linear system determining the coefficients of $H(k)$ for $1 \leq k \leq x$ 
naturally to express $H(x)$ as a linear combination of the original left-hand-side 
summatory function as follows:
\begin{align*} 
H(x) & = \sum_{j=1}^{x} \pi_{f \ast h}(j) \left[F^{-1}\left(\Floor{x}{j}\right) - 
     F^{-1}\left(\Floor{x}{j+1}\right)\right] \\ 
     & = \sum_{n=1}^{x} f^{-1}(n) \pi_{f \ast h}\left(\Floor{x}{n}\right). 
\end{align*} 
\end{theorem} 

\begin{cor}[Convolutions Arising From M\"obius Inversion] 
\label{cor_CvlGAstMu} 
Suppose that $g$ is an arithmetic function with $g(1) \neq 0$. Define the summatory function of 
the convolution of $g$ with $\mu$ by $\widetilde{G}(x) := \sum_{n \leq x} (g \ast \mu)(n)$. 
Then the Mertens function equals 
\[
M(x) = \sum_{k=1}^{x} \left(\sum_{j=\floor{\frac{x}{k+1}}+1}^{\floor{\frac{x}{k}}} g^{-1}(j)\right) 
     \widetilde{G}(k), \forall x \geq 1. 
\]
\end{cor} 

\begin{cor}[A motivating special case] 
\label{cor_Mx_gInvnPixk_formula} 
We have exactly that for all $x \geq 1$ 
\begin{equation} 
\label{eqn_Mx_gInvnPixk_formula} 
M(x) = \sum_{k=1}^{x} (\omega+1)^{-1}(k) \left[\pi\left(\Floor{x}{k}\right) + 1\right]. 
\end{equation} 
\end{cor} 

\subsection{Elaborating on the construction behind the motivating special case formula for $M(x)$} 
\label{example_InvertingARecRelForMx_Intro}

We can compute the first few terms for the
Dirichlet inverse sequence of the arithmetic function 
$g(n) := \omega(n) + 1$ from 
Corollary \ref{cor_Mx_gInvnPixk_formula} 
numerically for the first few sequence values as 
\[
\{g^{-1}(n)\}_{n \geq 1} = \{1, -2, -2, 2, -2, 5, -2, -2, 2, 5, -2, -7, -2, 5, 5, 2, -2, -7, -2, 
     -7, 5, 5, -2, 9, \ldots \}. 
\] 
The sign of these terms is given by $\operatorname{sgn}(g^{-1}(n)) = \frac{g^{-1}(n)}{|g^{-1}(n)|} = \lambda(n)$ 
(see Proposition \ref{prop_SignageDirInvsOfPosBddArithmeticFuncs_v1}). 
This useful property is inherited from the distinctly 
additive nature of the component function $\omega(n)$. 
We will still require substantially simpler asymptotic formulae for $g^{-1}(n)$ than what 
complications are suggested by inspection of the initial 
numerical calculations of this sequence. 

Consider first the following motivating conjecture: 
\NBRef{A01-2020-04-26}

\begin{conjecture}
\label{lemma_gInv_MxExample} 
Suppose that $n \geq 1$ is a squarefree integer. We have the following properties characterizing the 
Dirichlet inverse function $g^{-1}(n) = (\omega+1)^{-1}(n)$ over these integers: 
\begin{itemize} 

\item[(A)] $g^{-1}(1) = 1$; 
\item[(B)] $\operatorname{sgn}(g^{-1}(n)) = \mu(n) \equiv \lambda(n)$; 
\item[(C)] We can write the inverse function at squarefree $n$ as 
     \[
     g^{-1}(n) = \sum_{m=0}^{\omega(n)} \binom{\omega(n)}{m} \cdot m!. 
     \]
\end{itemize} 
We illustrate parts (B)--(C) of this conjecture clearly using 
Table \ref{table_conjecture_Mertens_ginvSeq_approx_values} given on 
page \pageref{table_conjecture_Mertens_ginvSeq_approx_values} of the appendix section. 
\end{conjecture} 

The realization that the beautiful and remarkably simple form of property (C) 
in Conjecture \ref{lemma_gInv_MxExample} holds for all squarefree $n \geq 1$ 
motivates our pursuit of formulas for the inverse functions $g^{-1}(n)$ based on the configuration of the 
exponents in the prime factorization of any $n \geq 2$. 
The summation methods we employ in Section \ref{Section_InvFunc_PreciseExpsAndAsymptotics} 
to weight sums of our arithmetic functions according to the sign of 
$\lambda(n)$ (or parity of $\Omega(n)$) is also 
reminiscent of the combinatorially motivated sieve methods in 
\cite[\S 17]{OPERADECRIBERO}. 

\begin{remark}[Comparison to formative methods for bounding $M(x)$]
Note that since the DGF of $\omega(n)$ is given by 
$D_{\omega}(s) = P(s) \zeta(s)$ where $P(s)$ is the \emph{prime zeta function}, we do have a 
Dirichlet series for the inverse functions to invert coefficient-wise using more classical 
contour integral methods\footnote{
E.g., using contour integration or the following integral formula for Dirichlet series 
inversion \cite[\S 11]{APOSTOLANUMT}: 
\[
f(n) = \lim_{T \rightarrow \infty} \frac{1}{2T} \int_{-T}^{T} 
     \frac{n^{\sigma+\imath t}}{\zeta(\sigma+\imath t)(P(\sigma+\imath t) + 1)}, \sigma > 1. 
\]
Fr\"oberg has also previously done some preliminary investigation as to the properties of the 
inversion to find the coefficients of $(1+P(s))^{-1}$ in \cite{FROBERG-1968}. 
}. 
However, the uniqueness to our new methods is that our new approach does not rely on typical constructions for 
bounding $M(x)$ based on estimates of the non-trivial zeros of the Riemann zeta function that have so far 
been employed to bound the Mertens function from above. 
That is, we will instead take a more combinatorial tack to investigating bounds on this inverse function 
sequence in the coming sections. By Corollary \ref{cor_Mx_gInvnPixk_formula}, 
once we have established bounds on this $g^{-1}(n)$ and its summatory function, we will be able to 
formulate new lower bounds (in the limit supremum sense) on $M(x)$. 
\end{remark} 

\subsection{Fixing an exact expression for $M(x)$ using additive functions} 

From this point on, we fix the notation for the Dirichlet invertible function $g(n) := \omega(n) + 1$ and denote its 
inverse with respect to Dirichlet convolution by $g^{-1}(n) = (\omega+1)^{-1}(n)$. 
For natural numbers $n \geq 1, k \geq 0$, let 
\begin{align*} 
C_k(n) := \begin{cases} 
     \varepsilon(n) = \delta_{n,1}, & \text{ if $k = 0$; } \\ 
     \sum\limits_{d|n} \omega(d) C_{k-1}(n/d), & \text{ if $k \geq 1$. } 
     \end{cases} 
\end{align*} 
We have limiting asymptotics on these functions in terms of $n$ and $k$ within a fixed range 
depending on $n$ given by the following theorem: 

\begin{theorem}[Asymptotics for the functions $C_k(n)$] 
\label{theorem_Ckn_GeneralAsymptoticsForms} 
For $k := 0$, we have by definition that $C_0(n) = \delta_{n,1}$. 
For all sufficiently large $n > 1$ and any fixed $1 \leq k \leq \Omega(n)$ 
taken independently of $n$, 
we obtain that the dominant asymptotic term for $C_k(n)$ is given uniformly by 
\[
\mathbb{E}[C_k(n)] \geq (\log\log n)^{2k-1}, \mathrm{\ as\ }n \rightarrow \infty. 
\]
\end{theorem} 

Since we have that 
\begin{equation} 
\label{eqn_AnExactFormulaFor_gInvByMobiusInv_v1} 
(g^{-1} \ast 1)(n) = \lambda(n) \cdot C_{\Omega(n)}(n), \forall n \geq 1, 
\end{equation} 
M\"{o}bius inversion provides us with an exact divisor sum based expression for $g^{-1}(n)$ 
(see Lemma \ref{lemma_AnExactFormulaFor_gInvByMobiusInv_v1}). 
Then we can prove (see Corollary \ref{cor_ASemiForm_ForGInvx_v1}) that we can obtain lower bounds on 
the magnitude of $g^{-1}(n)$ by approximating it by the simpler divisor sums 
\[
\lambda(n) \times \sum_{d|n} C_{\Omega(d)}(d). 
\]
Specifically, the last result in turn implies that 
\begin{equation} 
\label{eqn_GInvx_prelim_sum_formulas_intro_v1} 
|G^{-1}(x)| \SuccSim \left\lvert 
     \sum_{n \leq x} \lambda(n) \cdot C_{\Omega(n)}(n) \times 
     \sum_{d=1}^{\Floor{x}{n}} \lambda(d) \right\rvert. 
\end{equation} 
In light of the fact that (see Proposition \ref{prop_Mx_SBP_IntegralFormula}) 
\[
M(x) \sim G^{-1}(x) - \sum_{k=1}^{x/2} G^{-1}(k) \cdot \frac{x}{k^2 \log(x/k)}, 
\]
the formula in \eqref{eqn_GInvx_prelim_sum_formulas_intro_v1} implies that we can establish 
new finite \emph{lower bounds} on $M(x)$ along large infinite subsequences 
by appropriate estimates of the summatory function $G^{-1}(x)$. 
As explicit lower bounds for $M(x)$ along particular subsequences are not obvious, and are 
historically ellusive non-trivial features of the function to obtain as 
we expect sign changes of this function infinitely often, we find this approach to be an effective one. 

\subsection{Uniform asymptotics from enumerative counting DGFs in Mongomery and Vaughan} 

The precise formulations of the inverse function asymptotics 
proved in Section \ref{Section_InvFunc_PreciseExpsAndAsymptotics} depend on being able to form 
significant lower bounds on the summatory functions of an always positive arithmetic function 
weighted by $\lambda(n)$. 

Our inspiration for the new bounds found in the last sections of this article allows us to sum 
non-negative arithmetic functions weighted by the Liouville lambda function, 
$\lambda(n) = (-1)^{\Omega(n)}$. In particular, it uses a hybrid generating function and 
enumerative DGF method 
under which we are able to recover ``good enough'' asymptotics about the summatory functions that 
encapsulate the parity of $\lambda(n)$ through the summatory tally functions $\widehat{\pi}_k(x)$. 
The precise statement of the theorem that we transform to state these new bounds is re-stated next as 
Theorem \ref{theorem_HatPi_ExtInTermsOfGz}. 

\begin{theorem}[Montgomery and Vaughan, \S 7.4]
\label{theorem_HatPi_ExtInTermsOfGz} 
Recall that we have defined 
$$\widehat{\pi}_k(x) := \#\{n \leq x: \Omega(n)=k\}.$$ 
For $R < 2$ we have that 
\[
\widehat{\pi}_k(x) = \mathcal{G}\left(\frac{k-1}{\log\log x}\right) \frac{x}{\log x} 
     \frac{(\log\log x)^{k-1}}{(k-1)!} \left(1 + O_R\left(\frac{k}{(\log\log x)^2}\right)\right),  
\]
uniformly for $1 \leq k \leq R \log\log x$ where 
\[
\mathcal{G}(z) := \frac{F(1, z)}{\Gamma(z+1)} = \frac{1}{\Gamma(z+1)} \times 
     \prod_p \left(1-\frac{z}{p}\right)^{-1} \left(1-\frac{1}{p}\right)^z, z \geq 0. 
\]
\end{theorem} 

The next theorem, proved carefully in Section \ref{Section_MVCh7_GzBounds}, 
is the primary starting point for our new asymptotic lower bounds. 

\begin{theorem}[Generating functions of symmetric functions] 
\label{theorem_GFs_SymmFuncs_SumsOfRecipOfPowsOfPrimes} 
\label{cor_BoundsOnGz_FromMVBook_initial_stmt_v1} 
We obtain lower bounds of the following form on the function 
$\mathcal{G}(z)$ from Theorem \ref{theorem_HatPi_ExtInTermsOfGz} 
for $A_0 > 0$ an absolute constant, for 
$C_0(z)$ a strictly linear function only in $z$, and 
where we must take $0 \leq z \leq 1$, or equivalently $1 \leq k \leq \log\log x$ for $x$ large: 
\begin{align*} 
\mathcal{G}(z) \geq A_0 \cdot (1-z)^{3} \cdot C_0(z)^{z}. 
\end{align*} 
It suffices to take the components to the bound in the previous equation as 
\begin{align*}
A_0 & = \frac{2^{9/16} \exp\left(-\frac{55}{4} \log^2(2)\right)}{ 
     (3e\log 2)^3 \cdot \Gamma\left(\frac{5}{2}\right)} \approx 3.81296 \times 10^{-6} \\ 
C_0(z) & = \frac{4(1-z)}{3e \log 2}. 
\end{align*} 
In particular, with $0 \leq z \leq 1$ and 
$z \equiv z(k, x) = \frac{k-1}{\log\log x}$, by Theorem \ref{theorem_HatPi_ExtInTermsOfGz}, 
we have that 
\[
\widehat{\pi}_k(x) \SuccSim \frac{A_0 \cdot x}{\log x \cdot (\log\log x)^4 \cdot (k-1)!} \cdot 
     \left(\frac{4}{3e\log 2}\right)^{\frac{k}{\log\log x}}.
\]
\end{theorem} 

\subsection{Rigorous proofs jusifying that so-called average case lower bounds are 
            meaningful with respect to our problem} 
\label{subSection_Intro_RigorToTheAverageCaseEstimates} 

\begin{theorem} 
\label{theorem_CondAvgOrderGInvxSummatoryFunc_v1} 
Let the summatory function $G_E^{-1}(x)$ be defined for $x \geq 1$ by 
\begin{equation} 
\label{eqn_GEInvxSummatoryFuncDef_v1} 
G_E^{-1}(x) := \sum_{n \leq x} \lambda(n) \times \sum_{\substack{d|n \\ d > e^e}} 
     \mathbb{E}[C_{\Omega(d)}(d)]. 
\end{equation} 
If for some respectively minimally and maximally defined absolute constants 
$B, C \in (0, 1)$, we have that as $x \rightarrow \infty$
\[
B + o(1) \leq \frac{1}{x} \cdot \#\left\{n \leq x: |G^{-1}(n)| - |G_E^{-1}(n)| \leq 0\right\} \leq 
     C + o(1), 
\]
then there is some $\varepsilon \in (0, 1)$ (depending on $B,C$) with 
$0 < B - \varepsilon, C+\varepsilon < 1$ such that 
for all sufficiently large $x$ we have some 
$x_0 \in [(B - \varepsilon) x, (C + \varepsilon) x]$ such that 
\[
|G^{-1}(x_0)| \geq \left\lvert G_E^{-1}(x_0) \right\rvert. 
\]
\end{theorem} 
We prove Theorem \ref{theorem_CondAvgOrderGInvxSummatoryFunc_v1}, and 
rigorously justify that its hypotheses holds, in 
Section \ref{Section_ProofOfValidityOfAverageOrderLowerBounds}.  
This result combines to allow us to take lower bounds based on average case estimates of 
certain arithmetic functions we have defined to approximate $g^{-1}(n)$ and still recover 
an infinite subsequence along which we can witness the unboundedness in 
Corollary \ref{cor_ThePipeDreamResult_v1} stated below. 

The following observation is suggestive of the quasi-periodicity at play 
with the distinct values of $g^{-1}(n)$ distributed over $n \geq 2$: 

\begin{heuristic}[Symmetry in $g^{-1}(n)$ in the exponents in the prime factorization of $n$] 
Suppose that $n_1, n_2 \geq 2$ are such that their factorizations into distinct primes are 
given by $n_1 = p_1^{\alpha_1} \cdots p_r^{\alpha_r}$ and $n_2 = q_1^{\beta_1} \cdots q_r^{\beta_r}$. 
If $\{\alpha_1, \ldots, \alpha_r\} \equiv \{\beta_1, \ldots, \beta_r\}$ as multisets of prime exponents, 
then $g^{-1}(n_1) = g^{-1}(n_2)$. For example, $g^{-1}$ has the same values on the squarefree integers 
with exactly two, three, and so on prime factors 
(see Table \ref{table_conjecture_Mertens_ginvSeq_approx_values} starting on page 
\pageref{table_conjecture_Mertens_ginvSeq_approx_values}). 
\end{heuristic} 

There does not appear to be an easy, nor subtle 
direct recursion between the distinct $g^{-1}$ values, except through auxiliary function sequences. 
However, the distribution of distinct sets of prime exponents is fairly regular with 
$\omega(n)$ and $\Omega(n)$ playing a crucial role in the repitition of common values of 
$g^{-1}(n)$. The next remark makes clear what our intuiton ought suggest about the relation of 
the actual function values to the average case expectation of $g^{-1}(n)$ for $n \leq x$ when 
$x$ is large. 

\begin{remark}[Essential components of the proof]
Given that we have chosen to work with a representation for $M(x)$ that depends critically on 
the distribution of the values of the additive functions, $\omega(n)$ and $\Omega(n)$, there is 
substantial intuition involved \'{a} priori that suggests our sums over these functions ought 
behave regularly on average. Notably, we have an Erd\"os-Kac like theorem for each of 
$\omega(n)$ and $\Omega(n)$, which when the bounding parameter is set to $z := 0$, we provably 
can expect these sums involving the classically ``nice'' functions to tend towards their 
average case asymptotic nature infinitely often, and predictably near any large $x$ 
\cite[\S 1.7]{IWANIEC-KOWALSKI} (\cf Theorem \ref{theorem_MV_Thm7.20-init_stmt}). 
Thus the choice in stating \eqref{eqn_Mx_gInvnPixk_formula} as it depends on the 
cannonical additive function examples we have cited is 
\emph{absolutely essential} to the success of 
our proof making the ``magic'' happen out of the average case 
scenario we easily bound from below. 
\end{remark} 

\subsection{Cracking the classical unboundedness barrier} 

In Section \ref{Section_KeyApplications}, 
we provide the culmination of what we build up to in the proofs established in 
prior sections of the article. 
What we obtain at the conclusion of the section 
is the following important summary corollary that resolves the classical question of the 
unboundedness of the scaled function Mertens function 
$|M(x)| / \sqrt{x}$ in the limit supremum sense: 

\begin{cor}[Unboundedness of the the Mertens function scaled by $\sqrt{x}$] 
\label{cor_ThePipeDreamResult_v1} 
Let $u_0 := e^{2e^{e^{e}}}$ and define the infinite increasing subsequence, 
$\{x_{0,n}\}_{n \geq 1}$, by $x_{0,n} := e^{2e^{e^{e^{4n}}}}$. 
We have that along the increasing subsequence $\{x_y\}_{y \ggg 1}$, where 
$x_y \in \left(x_{0,y-1}, x_{0,y+1}\right)$ pointwise at each $y$, for all sufficiently large 
$y \gg \max\left(\ceiling{x_{0,1}}+1, u_0+2\right)$ the following bound holds: 
\begin{align*} 
\frac{|M(x_y)|}{\sqrt{x_y}} & \SuccSim 
     C_{\ell,1} \cdot (\log\log \sqrt{x_y})^{5/2} \cdot \frac{(\log\log\log \sqrt{x_y})^2}{ 
     (\log\log\log\log \sqrt{x_y})^{5/2}} + o(1), \mathrm{\ as\ } y \rightarrow \infty. 
\end{align*} 
In the previous equation, we adopt the notation for the 
absolute constant $C_{\ell,1} > 0$ defined more precisely by 
\[
C_{\ell,1} := 
     \frac{256 \cdot 2^{1/8}}{59049 \cdot \pi^2 e^8 \log^8(2)} 
     \exp\left(-\frac{55}{2} \log^2(2)\right) 
     \approx 5.51187 \times 10^{-12}. 
\]
\end{cor} 

This is all to say that in establishing the rigorous proof of 
Corollary \ref{cor_ThePipeDreamResult_v1} 
based on our new methods, we not only show that 
\[
\limsup_{x \rightarrow \infty} \frac{|M(x)|}{\sqrt{x}} = +\infty, 
\]
but also set a minimal rate (along a large infinite subsequence) at which this form of the 
scaled Mertens function grows without bound. 

For technical reasons found in the proof of 
Theorem \ref{theorem_gInv_GeneralAsymptoticsForms}, 
the new primary result used to show 
Corollary \ref{cor_ThePipeDreamResult_v1}, 
it is more difficult to exactly define a secondary subsequence, $\{\hat{x}_{0,n}\}_{n \geq 1}$, 
around which we can witness the unboundedness of $M(x) / \sqrt{x}$ towards $-\infty$. 
In this instance, we would require that 
$$\floor{\log\log\log n} \equiv \floor{\log\log\log\log n} \pmod{2}.$$ 

\newpage 
\section{Preliminary proofs of lemmas and new results} 
\label{Section_PrelimProofs_Config} 

The purpose of this section is to provide proofs and statements 
of elementary and otherwise well established facts and results. In particular, the proof of 
Theorem \ref{theorem_SummatoryFuncsOfDirCvls} allows us to easily justify the formula in 
\eqref{eqn_Mx_gInvnPixk_formula}. 
This formula is the crucial formulation that constiutes an exact expression for $M(x)$. 
The indispensible property inherent to the arithmetic functions, $\omega(n)$ and $g^{-1}(n)$, 
that are used to state the formula are strong additivity, which leads to the sign of the inverse function 
$g^{-1}(n)$ being given by $\lambda(n)$. Hence the summatory function of $g^{-1}(n)$ is 
intimately tied to the exact limiting 
distribution of the values of $\Omega(n)$. 

\subsection{Establishing the summatory function inversion identities} 

We will prove Theorem \ref{theorem_SummatoryFuncsOfDirCvls}, a crucial component to our new more combinatorial 
formulations used to bound $M(x)$ in later sections, using matrix methods before moving on. 
Related results on summations of Dirichlet convolutions appear in 
\cite[\S 2.14; \S 3.10; \S 3.12; \cf \S 4.9, p.\ 95]{APOSTOLANUMT}. 

\begin{proof}[Proof of Theorem \ref{theorem_SummatoryFuncsOfDirCvls}] 
\label{proofOf_theorem_SummatoryFuncsOfDirCvls} 
Let $h,g$ be arithmetic functions where $g(1) \neq 0$ 
necessarily has a Dirichlet inverse. Denote the summatory functions of $h$ and $g$, 
respectively, by $H(x) = \sum_{n \leq x} h(n)$ and $G(x) = \sum_{n \leq x} g(n)$. 
We define $\pi_{g \ast h}(x)$ to be the summatory function of the 
Dirichlet convolution of $g$ with $h$: $g \ast h$. 
Then we can easily see that the following expansions hold: 
\begin{align*} 
\pi_{g \ast h}(x) & := \sum_{n=1}^{x} \sum_{d|n} g(n) h(n/d) = \sum_{d=1}^x g(d) H\left(\floor{\frac{x}{d}}\right) \\ 
     & = \sum_{i=1}^x \left[G\left(\floor{\frac{x}{i}}\right) - G\left(\floor{\frac{x}{i+1}}\right)\right] H(i). 
\end{align*} 
We form the matrix of coefficients associated with this system for $H(x)$, and proceed to invert it to express an 
exact solution for this function over all $x \geq 1$. Let the ordinary (initial, non-inverse) matrix entries be denoted by 
\[
g_{x,j} := G\left(\floor{\frac{x}{j}}\right) - G\left(\floor{\frac{x}{j+1}}\right) \equiv G_{x,j} - G_{x,j+1}. 
\]
The matrix we must invert in this problem is lower triangular, with ones on its diagonals -- and hence is invertible. 
Moreover, if we let $\hat{G} := (G_{x,j})$, then this matrix is 
expressable by an invertible shift operation as 
\[
(g_{x,j}) = \hat{G} (I - U^{T}); \qquad U = (\delta_{i,j+1}), (I - U^T)^{-1} = (\Iverson{j \leq i}). 
\]
Here, $U$ is the $N \times N$ matrix whose $(i,j)^{th}$ entries are defined by 
$(U)_{i,j} = \delta_{i+1,j}$. 

It is a useful fact that if we take successive differences of floor functions, we get non-zero behavior at divisors: 
\[
G\left(\floor{\frac{x}{j}}\right) - G\left(\floor{\frac{x-1}{j}}\right) = 
     \begin{cases} 
     g\left(\frac{x}{j}\right), & \text{ if $j | x$; } \\ 
     0, & \text{ otherwise. } 
     \end{cases}
\]
We use this property to shift the matrix $\hat{G}$, and then invert the result, to obtain a matrix involving the 
Dirichlet inverse of $g$: 
\begin{align*} 
\left[(I-U^{T}) \hat{G}\right]^{-1} & = \left(g\left(\frac{x}{j}\right) \Iverson{j|x}\right)^{-1} = 
     \left(g^{-1}\left(\frac{x}{j}\right) \Iverson{j|x}\right). 
\end{align*} 
Now we can express the inverse of the target matrix $(g_{x,j})$ in terms of these Dirichlet inverse functions 
as follows: 
\begin{align*} 
(g_{x,j}) & = (I-U^{T})^{-1} \left(g\left(\frac{x}{j}\right) \Iverson{j|x}\right) (I-U^{T}) \\ 
(g_{x,j})^{-1} & = (I-U^{T})^{-1} \left(g^{-1}\left(\frac{x}{j}\right) \Iverson{j|x}\right) (I-U^{T}) \\ 
     & = \left(\sum_{k=1}^{\floor{\frac{x}{j}}} g^{-1}(k)\right) (I-U^{T}) \\ 
     & = \left(\sum_{k=1}^{\floor{\frac{x}{j}}} g^{-1}(k) - \sum_{k=1}^{\floor{\frac{x}{j+1}}} g^{-1}(k)\right). 
\end{align*} 
Thus the summatory function $H$ is exactly expressed by the inverse vector product of the form 
\begin{align*} 
H(x) & = \sum_{k=1}^x g_{x,k}^{-1} \cdot \pi_{g \ast h}(k) \\ 
     & = \sum_{k=1}^x \left(\sum_{j=\floor{\frac{x}{k+1}}+1}^{\floor{\frac{x}{k}}} g^{-1}(j)\right) \cdot \pi_{g \ast h}(k). 
     \qedhere
\end{align*} 
\end{proof} 

\subsection{Proving the crucial signedness property from the conjecture} 

Let $\chi_{\mathbb{P}}$ denote the characteristic function of the primes, 
$\varepsilon(n) = \delta_{n,1}$ be the multiplicative identity with respect to Dirichlet convolution, 
and denote by $\omega(n)$ the strongly additive function that counts the number of 
distinct prime factors of $n$. Then we can easily prove that 
\begin{equation}
\label{eqn_AntiqueDivisorSumIdent} 
\chi_{\mathbb{P}} + \varepsilon = (\omega + 1) \ast \mu. 
\end{equation} 
When combined with Corollary \ref{cor_CvlGAstMu}, an immediate consequence of 
Theorem \ref{theorem_SummatoryFuncsOfDirCvls}, 
this convolution identity yields the necessary convolution identity that yields the exact 
formula for $M(x)$ stated in \eqref{eqn_Mx_gInvnPixk_formula} of 
Corollary \ref{cor_Mx_gInvnPixk_formula}. 

The proof of the next proposition is essential to our argument given in later sections. 
We try to keep the argument brief while sketching all relevant details to rigorously justifying the key parts 
to the proof of our claim. 

\begin{prop}[The key signedness property of $g^{-1}(n)$]
\label{prop_SignageDirInvsOfPosBddArithmeticFuncs_v1} 
For the Dirichlet invertible function, $g(n) := \omega(n) + 1$ defined such that $g(1) = 1$, at any 
$n \geq 1$, we have that $\operatorname{sgn}(g^{-1}(n)) = \lambda(n)$. 
The notation for the operation given by 
$\operatorname{sgn}(h(n)) = \frac{h(n)}{|h(n)| + \Iverson{h(n) = 0}} \in \{0, \pm 1\}$ denotes the sign 
of the arithmetic function $h$ at $n$. 
\NBRef{A02-2020-04-26}
\end{prop} 
\begin{proof} 
Recall that $D_f(s) := \sum_{n \geq 1} f(n) n^{-s}$ denotes the Dirichlet generating function (DGF) of any 
arithmetic function $f(n)$ which is convergent for all $s \in \mathbb{C}$ satisfying $\Re(s) > \sigma_f$. 
In particular, recall that $D_1(s) = \zeta(s)$, $D_{\mu}(s) = 1 / \zeta(s)$ and $D_{\omega}(s) = P(s) \zeta(s)$. 
Then by \eqref{eqn_AntiqueDivisorSumIdent} and the known property that the DGF of $f^{-1}(n)$ is 
the reciprocal of the DGF of the original arithmetic function $f$, for all $\Re(s) > 1$ we have 
\begin{align} 
\label{eqn_DGF_of_gInvn} 
D_{(\omega+1)^{-1}}(s) = \frac{1}{(P(s)+1) \zeta(s)}. 
\end{align} 
It follows that $(\omega + 1)^{-1}(n) = (h^{-1} \ast \mu)(n)$ when we take 
$h := \chi_{\mathbb{P}} + 1$. 
We show that $\operatorname{sgn}(h^{-1}) = \lambda$. From this fact, it follows by inspection 
that $\operatorname{sgn}(h^{-1} \ast \mu) = \lambda$. The remainder of the proof fills in the 
precise details needed to make this intuition precise. 

By the standard recurrence relation we used to define the Dirichlet inverse function of any 
arithmetic function $h$ such that $h(1) = 1 \neq 0$, 
we have that 
\begin{equation} 
\label{eqn_proof_tag_hInvn_ExactRecFormula_v1}
h^{-1}(n) = \begin{cases} 
            1, & n = 1; \\ 
            -\sum\limits_{\substack{d|n \\ d>1}} h(d) h^{-1}(n/d), & n \geq 2. 
            \end{cases} 
\end{equation} 
For $n \geq 2$, the summands in \eqref{eqn_proof_tag_hInvn_ExactRecFormula_v1} 
can be simply indexed over the primes $p|n$. This observation yields that we can inductively 
expand these sums into nested divisor sums provided the depth of the sums does not exceed the 
capacity to index summations over the primes dividing $n$. Namely, notice that for $n \geq 2$ 
\begin{align*} 
h^{-1}(n) & = -\sum_{p|n} h^{-1}(n/p), && \text{\ if\ } \Omega(n) \geq 1 \\ 
     & = \sum_{p_1|n} \sum_{p_2|\frac{n}{p_1}} h^{-1}\left(\frac{n}{p_1p_2}\right), && \text{\ if\ } \Omega(n) \geq 2 \\ 
     & = -\sum_{p_1|n} \sum_{p_2|\frac{n}{p_1}} \sum_{p_3|\frac{n}{p_1p_2}} h^{-1}\left(\frac{n}{p_1p_2p_3}\right), 
     && \text{\ if\ } \Omega(n) \geq 3. 
\end{align*} 
Then by induction, again with $h^{-1}(1) = 1$, we obtain by expanding the 
nested divisor sums as above to their maximal depth as 
\[
h^{-1}(n) = \lambda(n) \times \sum_{p_1|n} \sum_{p_2|\frac{n}{p_2}} \times \cdots \times 
     \sum_{p_{\Omega(n)}|\frac{n}{p_1p_2 \cdots p_{\Omega(n)-1}}} 1, n \geq 2. 
\]
If for $n \geq 2$ we write the prime factorization of $n$ as 
$n = p_1^{\alpha_1} p_2^{\alpha_2} \cdots p_{\omega(n)}^{\alpha_{\omega(n)}}$ where the exponents $\alpha_i \geq 1$ are all 
non-zero for $1 \leq i \leq \omega(n)$, we can see that 
\begin{align*} 
h^{-1}(n) & \geq \lambda(n) \times 1 \cdot 2 \cdot 3 \cdots \omega(n) = \lambda(n) \times (\omega(n))!, && n \geq 2 \\ 
h^{-1}(n) & \leq \lambda(n) \times (\omega(n))!^{\max(\alpha_1, \alpha_2, \ldots, \alpha_{\omega(n)})}, && n \geq 2. 
\end{align*} 
In other words, what these bounds show is that for all $n \geq 1$ (with $\lambda(1) = 1$) the following property holds: 
\begin{equation} 
\label{eqn_proof_tag_SignedTimesPosConstantFormOf_hInvn_v2}
\operatorname{sgn}(h^{-1}(n)) = \lambda(n). 
\end{equation}
By \eqref{eqn_proof_tag_SignedTimesPosConstantFormOf_hInvn_v2}, we immediately have bounding constants 
$1 \leq C_{1,n}, C_{2,n} < +\infty$ that exist for each $n \geq 1$ so that 
\begin{equation} 
\label{eqn_proof_tag_hInvMunCvl_UpperLowerBounds_v3} 
C_{1,n} \cdot (\lambda \ast \mu)(n) \leq (h^{-1} \ast \mu)(n) \leq C_{2,n} \cdot (\lambda \ast \mu)(n). 
\end{equation} 
Since both $\lambda,\mu$ are multiplicative, the convolution $\lambda \ast \mu$ is multiplicative. 
We know that the values of 
any multiplicative function are uniquely determined by its action at prime powers. 
So we can compute that for any prime $p$ and non-negative integer exponents $\alpha \geq 1$ that 
\begin{align*} 
(\lambda \ast \mu)(p^{\alpha}) & = \sum_{i=0}^{\alpha} \lambda(p^{\alpha-i}) \mu(p^{i}) \\ 
     & = \lambda(p^{\alpha}) - \lambda(p^{\alpha-1}) \\ 
     & = 
     (-1)^{\Omega(p^{\alpha})} - (-1)^{\Omega(p^{\alpha-1})} = 
     (-1)^{\alpha} - (-1)^{\alpha-1} = 
     2 \lambda(p^{\alpha}). 
\end{align*} 
Then by the multiplicativity of $\lambda(n)$, the previous inequalities derived in 
\eqref{eqn_proof_tag_hInvMunCvl_UpperLowerBounds_v3} are re-stated in the form of 
\[
2 C_{1,n} \cdot \lambda(n) \leq h^{-1}(n) \leq 2 C_{2,n} \cdot \lambda(n). 
\] 
Since the absolute constants $C_{1,n}, C_{2,n}$ are always positive for all $n \geq 1$, 
we clearly recover the signedness of $g^{-1}(n)$ as $\lambda(n)$. 
\end{proof} 

\subsection{Other facts and listings of results we will need in our proofs} 
\label{subSection_OtherFactsAndResults} 

\begin{theorem}[Mertens theorem]
\label{theorem_Mertens_theorem} 
For all $x \geq 2$ we have that 
\[
P_1(x) := \sum_{p \leq x} \frac{1}{p} = \log\log x + B + o(1), 
\]
where 
$B \approx 0.2614972128476427837554$ 
is an explicitly defined absolute constant.
\end{theorem} 

\begin{cor}
\label{lemma_Gz_productTermV2} 
We have that for sufficiently large $x \gg 1$ 
\[
\prod_{p \leq x} \left(1 - \frac{1}{p}\right) = \frac{e^{-B}}{\log x}\left( 
     1 + o(1)\right). 
\]
Hence, for $1 < |z| < 2$ we obtain that 
\[
\prod_{p \leq x} \left(1 - \frac{1}{p}\right)^{z} = \frac{e^{-Bz}}{(\log x)^{z}} \left(1+o(1)\right)^{z}. 
\]
\end{cor} 

\begin{facts}[Exponential Integrals and Incomplete Gamma Functions] 
\label{facts_ExpIntIncGammaFuncs} 
\begin{subequations}
The following two variants of the \emph{exponential integral function} are defined by 
\cite[\S 8.19]{NISTHB} 
\begin{align*} 
\operatorname{Ei}(x) & := \int_{-x}^{\infty} \frac{e^{-t}}{t} dt, \\ 
E_1(z) & := \int_1^{\infty} \frac{e^{-tz}}{t} dt, \Re(z) \geq 0, 
\end{align*} 
where $\operatorname{Ei}(-kz) = -E_1(kz)$ for real $k > 0$. 
We have the following inequalities providing 
quasi-polynomial upper and lower bounds on $E_1(z)$: 
\begin{equation}
1-\frac{3}{4} z \leq E_1(z) - \gamma - \log z \leq 1-\frac{3}{4} z + \frac{11}{36} z^2. 
\end{equation}
A related function is the (upper) \emph{incomplete gamma function} defined by \cite[\S 8.4]{NISTHB} 
\[
\Gamma(s, x) = \int_{x}^{\infty} t^{s-1} e^{-t} dt, \Re(s) > 0. 
\]
We have the following properties of $\Gamma(s, x)$: 
\begin{align} 
\Gamma(s, x) & = (s-1)! \cdot e^{-x} \times \sum_{k=0}^{s-1} \frac{x^k}{k!}, s \in \mathbb{Z}^{+}, \\ 
\Gamma(s, x) & \sim x^{s-1} \cdot e^{-x}, |x| \rightarrow +\infty. 
\end{align}
\end{subequations}
\end{facts} 

\newpage 
\section{Summing arithmetic functions weighted by $\lambda(n)$} 
\label{Section_MVCh7_GzBounds} 

In this section, we re-state a couple of key results proved in \cite[\S 7.4]{MV} that we rely on 
to state and prove Corollary \ref{theorem_MV_Thm7.20} stated below. This corollary is important as it shows 
that (signed) summatory functions over $\widehat{\pi}(x)$ 
capture the dominant asymptotics of the full summatory function formed by taking $1 \leq k \leq \log_2(x)$ when 
we truncate and instead sum only up to the uniform bound of $1 \leq k \leq \log\log x$ guaranteed by applying 
Theorem \ref{theorem_HatPi_ExtInTermsOfGz}. 

We also prove 
Theorem \ref{theorem_GFs_SymmFuncs_SumsOfRecipOfPowsOfPrimes} in this section. 
This key theorem allows us to establish a global minimum we can attain on the function $\mathcal{G}(z)$ from 
Theorem \ref{theorem_HatPi_ExtInTermsOfGz} by truncating the formerly stated infinite 
range of the primes $p$ over which we take a component product in the definition of this function. 
This in turn implies the uniform lower bounds on $\widehat{\pi}_k(x)$ guaranteed by that theorem by 
a straightforward manipulation of inequalities. 

\subsection{Discussion: The enumerative DGF result from 
            Montgomery and Vaughan} 
\label{subSection_MVPrereqResultStmts} 

What the enumeratively-flavored result of Montgomery and Vaughan 
in Theorem \ref{theorem_HatPi_ExtInTermsOfGz} allows us to do is get a 
``good enough'' lower bound on sums of positive and asymptotically bounded arithmetic functions 
weighted by the Liouville lambda function, $\lambda(n) = (-1)^{\Omega(n)}$. 
For comparison, we already have known, more classical bounds due to Erd\"os (and earlier) that 
we can tightly bound \cite{ERDOS-PRIMEK-FUNC,MV} 
\[
\pi_k(x) = (1 + o(1)) \cdot \frac{x}{\log x} \frac{(\log\log x)^{k-1}}{(k-1)!}. 
\] 
We seek to approximate the right-hand-side of $\mathcal{G}(z)$ by only taking the products of the primes 
$p \leq u$, e.g., indexing the component products only over those primes 
$p \in \left\{2,3,5,\ldots,u\right\}$ for some minimal upper bound $u$ (with respect to $x$) 
as $x \rightarrow \infty$. 

We also state the following theorems reproduced from \cite[\S 7.4]{MV} that handle the relative 
scarcity of the distribution of the $\Omega(n)$ for $n \leq x$ such that 
$\Omega(n) > \log\log x$. 

\begin{theorem}[Bounds on exceptional values of $\Omega(n)$ for large $n$] 
\label{theorem_MV_Thm7.20-init_stmt} 
Let 
\begin{align*} 
A(x, r) & := \#\left\{n \leq x: \Omega(n) \leq r \cdot \log\log x\right\}, \\ 
B(x, r) & := \#\left\{n \leq x: \Omega(n) \geq r \cdot \log\log x\right\}. 
\end{align*} 
If $0 < r \leq 1$ and $x \geq 2$, then 
\[
A(x, r) \ll x (\log x)^{r-1 - r\log r}, \text{ \ as\ } x \rightarrow \infty. 
\]
If $1 \leq r \leq R < 2$ and $x \geq 2$, then 
\[
B(x, r) \ll_R x \cdot (\log x)^{r-1-r \log r}, \text{ \ as\ } x \rightarrow \infty. 
\]
\end{theorem} 

\begin{theorem}[Bounds on exceptional values of $\Omega(n)$ for large $n$, MV 7.21] 
\label{theorem_MV_Thm7.21-init_stmt} 
We have that uniformly 
\[
\#\left\{3 \leq n \leq x: \frac{\Omega(n) - \log\log n}{\sqrt{\log\log n}} \leq 0\right\} = 
     \frac{x}{2} + O\left(\frac{x}{\sqrt{\log\log x}}\right). 
\]
\end{theorem} 

\begin{remark} 
The proofs of Theorem \ref{theorem_MV_Thm7.20-init_stmt} and 
Theorem \ref{theorem_MV_Thm7.21-init_stmt} 
are found in Chapter 7 of Montgomery and Vaughan. 
The key interpretation we need is the result stated in the next corollary. 
In the previous theorem, the dependence on $R$, and the necessity of using the 
conditional relation $\ll_R$, serves to denote this $R$ as a 
bounding (maximally limiting) parameter on the 
input $r \in (1, R)$ to the functions $B(x, r)$. 
The precise way in which the bound 
stated in this cited theorem depends on this bounded, 
indeterminate paramater $R$ can be reviewed for reference in the proof 
algebra and relations cited in the reference \cite[\S 7]{MV}. 
The role of the parameter $R$ involved in stating the previous theorem 
is notably important as a scalar factor the upper bound on $k \leq R\log\log x$ in 
Theorem \ref{theorem_HatPi_ExtInTermsOfGz} up to which 
we obtain the valid uniform bounds in $x$ on the asymptotics for 
$\widehat{\pi}_k(x)$. 

We have a discrepancy to work out in so much as we 
can only form summatory functions over the $\widehat{\pi}_k(x)$ for 
$1 \leq k \leq R\log\log x$ using the desirable, or ``nice'', asymptotic formulas
guaranteed by Theorem \ref{theorem_HatPi_ExtInTermsOfGz}, even though we can actually 
have contributions from values distributed throughout the range $1 \leq \Omega(n) \leq \log_2(n)$. 
It is then crucial that we can show that the dominant growth of the asymptotic formulas we obtain 
for these summatory functions is captured by summing only over $k$ in the truncated range 
where the uniform formulas hold. In particular, we will require a proof 
that we can discard the terms in the full summatory function 
asymptotic formulas as negligible (up to at most a constant) 
for large $x$ when they happen to fall in the 
limiting exceptional range of $\Omega(n) > R\log\log x$ for $n \leq x$. 
\end{remark} 

\begin{cor} 
\label{theorem_MV_Thm7.20} 
Using the notation for $A(x, r)$ and $B(x, r)$ from 
Theorem \ref{theorem_MV_Thm7.20-init_stmt}, 
we have that for $\delta > 0$, 
\[
0 \leq \left\lvert \frac{B(x, 1+\delta)}{A(x, 1)} \right\rvert \ll 2, 
     \mathrm{\ as\ } \delta \rightarrow 0^{+}, x \rightarrow \infty. 
\]
\end{cor} 
\begin{proof} 
The lower bound stated above should be clear. To show that the asymptotic 
upper bound is correct, we compute using Theorem \ref{theorem_MV_Thm7.20-init_stmt} and 
Theorem \ref{theorem_MV_Thm7.21-init_stmt} that 
\begin{align*} 
\left\lvert \frac{B(x, 1+\delta)}{A(x, 1)} \right\rvert & \ll 
     \left\lvert \frac{x \cdot (\log x)^{\delta - \delta\log(1+\delta)}}{ 
     \widehat{\pi}_1(x) + \widehat{\pi}_2(x) + \frac{x}{2} + 
     O\left(\frac{x}{\sqrt{\log\log x}}\right)} \right\rvert \\ 
     & \sim 
     \left\lvert \frac{x \cdot (\log x)^{\delta - \delta\log(1+\delta)}}{ 
     \frac{x}{\log x} + \frac{x \cdot (\log\log x)}{\log x} + \frac{x}{2} + 
     O\left(\frac{x}{\sqrt{\log\log x}}\right)} \right\rvert \\ 
     & = 
     \left\lvert \frac{(\log x)^{1 + \delta - \delta\log(1+\delta)}}{ 
     1 + \log\log x + \frac{\log x}{2} + o(1)}\right\rvert \\ 
     & \xrightarrow{\delta \rightarrow 0^{+}} 
     \left\lvert \frac{(\log x)}{ 
     1 + \log\log x + \frac{\log x}{2} + o(1)} \right\rvert \\ 
     & \sim 2, 
\end{align*} 
as $x \rightarrow \infty$. Notice that since $\mathbb{E}[\Omega(n)] = \log\log n + B$ for $0 < B < 1$, the 
absolute constant from Mertens theorem, 
when we apply this result, the range $k > \log\log x$, denoted as above in the form of 
$k > (1 + \delta) \log\log x$, we can assume that $\delta \rightarrow 0^{+}$ as 
$x \rightarrow \infty$. 
\end{proof} 

We again emphasize that 
Corollary \ref{theorem_MV_Thm7.20} implies that for sums involving $\widehat{\pi}_k(x)$ indexed by $k$, 
we can capture the dominant asymptotic behavior of these sums by taking $k$ in the truncated range 
$1 \leq k \leq \log\log x$, e.g., with $0 \leq z \leq 1$ in Theorem \ref{theorem_HatPi_ExtInTermsOfGz}. 
This fact will be important when we prove 
Theorem \ref{theorem_gInv_GeneralAsymptoticsForms} in 
Section \ref{Section_KeyApplications} using a sign-weighted 
summatory function in Abel summation that depends on these functions 
(see Lemma \ref{lemma_CLT_and_AbelSummation}). 


\subsection{The key new results utilizing Theorem \ref{theorem_HatPi_ExtInTermsOfGz}} 
\label{subSection_PartialPrimeProducts_Proofs} 

We will require a handle on partial sums of integer powers of the reciprocal primes as 
functions of the integral exponent and the upper summation index $x$. 
The next corollary is not a triviality as it comes in handy when we take to the next task of 
proving the bound in Theorem \ref{theorem_GFs_SymmFuncs_SumsOfRecipOfPowsOfPrimes}. 
The statement of Proposition \ref{cor_PartialSumsOfReciprocalsOfPrimePowers} 
effectively provides a coarse rate in $x$ below which the reciprocal prime sums tend to 
absolute constants given by the prime zeta function, $P(s)$. We also require the finite-degree 
polynomial dependence of these bounds on $s$ to simplify the computations in the theorem below. 

\begin{prop} 
\label{cor_PartialSumsOfReciprocalsOfPrimePowers} 
For real $s \geq 1$, let 
\[
P_s(x) := \sum_{p \leq x} p^{-s}, x \gg 2. 
\]
When $s := 1$, we have the known bound in Mertens theorem 
(see Theorem \ref{theorem_Mertens_theorem}). For $s > 1$, we obtain that 
\[
P_s(x) \approx E_1((s-1) \log 2) - E_1((s-1) \log x) + o(1). 
\]
For integers $s \geq 2$ we have that 
\[
P_s(x) \leq \gamma_1(s, x) + o(1). 
\]
It suffices to take the bounding function in the previous equation as 
\begin{align*}
%\gamma_0(z, x) & = -s\log\left(\frac{\log x}{\log 2}\right) + \frac{3}{4}s(s-1) \log(x/2) - 
%     \frac{11}{36} s(s-1)^2 \log^2(x) \\ 
\gamma_1(s, x) & = -s\log\left(\frac{\log x}{\log 2}\right) + \frac{3}{4}s(s-1) \log(x/2) + 
     \frac{11}{36} s(s-1)^2 \log^2(2). 
\end{align*}
\end{prop} 
\NBRef{A05-2020-04-26} 
\begin{proof} 
Let $s > 1$ be real-valued. 
By Abel summation with the summatory function $A(x) = \pi(x) \sim \frac{x}{\log x}$ and where 
our target function $f(t) = t^{-s}$ with $f^{\prime}(t) = -s \cdot t^{-(s+1)}$, we obtain that 
\begin{align*} 
P_s(x) & = \frac{1}{x^s \cdot \log x} + s \cdot \int_2^{x} \frac{dt}{t^s \log t} \\ 
     & = E_1((s-1) \log 2) - E_1((s-1) \log x) + o(1), |x| \rightarrow \infty. 
\end{align*} 
Now using the inequalities in Facts \ref{facts_ExpIntIncGammaFuncs}, we obtain that the 
difference of the exponential integral functions is bounded above and below by 
\begin{align*} 
\frac{P_s(x)}{s} & \geq -\log\left(\frac{\log x}{\log 2}\right) + \frac{3}{4}(s-1) \log(x/2) - 
     \frac{11}{36} (s-1)^2 \log^2(x) \\ 
\frac{P_s(x)}{s} & \leq -\log\left(\frac{\log x}{\log 2}\right) + \frac{3}{4}(s-1) \log(x/2) + 
     \frac{11}{36} (s-1)^2 \log^2(2). 
\end{align*} 
This completes the proof of the bounds cited above in the statement of this lemma. 
\end{proof} 

\NBRef{A06-2020-04-26} 
\begin{proof}[Proof of Theorem \ref{theorem_GFs_SymmFuncs_SumsOfRecipOfPowsOfPrimes}] 
\label{proofOf_theorem_GFs_SymmFuncs_SumsOfRecipOfPowsOfPrimes} 
We have that for all integers $0 \leq k \leq m$
\begin{equation} 
\label{eqn_pf_tag_hSymmPolysGF} 
[z^k] \prod_{1 \leq i \leq m} (1-f(i) z)^{-1} = [z^k] \exp\left(\sum_{j \geq 1} 
     \left(\sum_{i=1}^m f(i)^j\right) \frac{z^j}{j}\right). 
\end{equation} 
In our case we have that $f(i)$ denotes the $i^{th}$ prime. 
Hence, summing over all $p \leq ux$ 
in place of $0 \leq k \leq m$ in the previous formula in tandem with 
Proposition \ref{cor_PartialSumsOfReciprocalsOfPrimePowers}, we obtain that the logarithm of the 
generating function in $z$ obtained when we sum over all $p \leq ux$ for some minimal parameter 
$u$ is given by 
\begin{align*} 
\log\left[\prod_{p \leq ux} \left(1-\frac{z}{p}\right)^{-1}\right] & \geq (B + \log\log (ux)) z + 
     \sum_{j \geq 2} \left[a(ux) + b(ux)(j-1) + c(ux) (j-1)^2\right] z^j \\ 
     & = (B + \log\log (ux)) z - a(ux) \left(1 + \frac{1}{z-1} + z\right) \\ 
     & \phantom{= (B + \log\log (ux)) z\ } + 
     b(ux) \left( 
     1 + \frac{2}{z-1} + \frac{1}{(z-1)^2}\right) \\ 
     & \phantom{= (B + \log\log (ux)) z\ } - 
     c(ux) \left( 
     1 + \frac{4}{z-1} + \frac{5}{(z-1)^2} + \frac{2}{(z-1)^3}\right) \\ 
     & =: \widehat{\mathcal{B}}(u, x; z). 
\end{align*} 
In the previous equations, the lower bounds formed by the functions $(a,b,c)$ 
evaluated at $ux$ are 
given by the corresponding upper bounds from 
Proposition \ref{cor_PartialSumsOfReciprocalsOfPrimePowers} 
due to the leading sign on the previous expansions as 
\begin{align*} 
(a_{\ell}, b_{\ell}, c_{\ell}) & := \left(-\log\left(\frac{\log (ux)}{\log 2}\right), 
     \frac{3}{4} \log\left(\frac{ux}{2}\right), \frac{11}{36} \log^2 2\right). 
\end{align*} 
Now we make a decision to set the uniform bound parameter to a middle ground value of 
$1 < R < 2$ at $R := \frac{3}{2}$ 
(practically, to be truncated and taken as though 
$R \equiv 1$ in sums by the restriction that $z \leq 1$) so that 
$$z \equiv z(k, x) = \frac{k}{\log\log x} \in (0, R),$$ for $x \gg 1$ very large. 
Thus $(z-1)^{-m} \in [(-1)^m, 2^m]$ for integers $m \geq 1$, and so we can obtain the 
lower bound stated below. Namely, these bounds on the signed reciprocals of $z-1$ 
lead to an effective bound of the following form: 
\begin{align*} 
\widehat{\mathcal{B}}(u, x; z) & \geq (B + \log\log (ux)) z - a(ux) \left(1 + \frac{1}{z-1} + z\right) \\ 
     & \phantom{= (B + \log\log (ux)) z\ } + 
     b(ux) \left( 
     1 + \frac{2}{z-1} + \frac{1}{(z-1)^2}\right) - 
     45 \cdot c(ux). 
\end{align*} 
Since the function $c(ux)$ is constant, we then also obtain the next bounds. 
\begin{align} 
\notag 
\frac{e^{-Bz}}{(\log (ux))^{z}} \times \exp\left(\widehat{\mathcal{B}}(u, x; z)\right) & \geq 
    \exp\left(-\frac{55}{4} \log^2(2)\right) \times \left(\frac{\log(ux)}{\log 2}\right)^{1 + \frac{1}{z-1} + z} \\ 
\notag 
    & \phantom{\geq \times} \times \left(\frac{ux}{2}\right)^{\frac{3}{4}\left(1 + \frac{2}{z-1} + \frac{1}{(z-1)^2}\right)} \\ 
\label{eqn_proof_simpl_v1} 
     & =: \widehat{\mathcal{C}}(u, x; z) 
\end{align} 
Now we need to determine which values of $u$ minimize the expression for the function defined 
in \eqref{eqn_proof_simpl_v1}. 
For this we will use a somewhat limited elementary method from 
introdutory calculus to determine a global minimum for the products. 
%\footnote{ 
%     The global minimum on these products is sufficient to obtain the results we prove in this 
%     article. It is expected that significant refinements, and improvements to the residual logarithmic 
%     factor we recover against the classical unboundedness conjectures, are possible by 
%     applying more sophisticated asymptotic techniques. In particular, these lower bounds have no dependence 
%     on $x$, a feature that will surely sharpen the end results substantially. 
%}. 
We can symbolically use \emph{Mathematica} to see that 
\[
\frac{\partial}{\partial u}\left[\widehat{\mathcal{C}}(u, x; z)\right] \Biggr\rvert_{u \mapsto u_0} = 0 \implies 
     u_0 \in \left\{\frac{1}{x}, \frac{1}{x} e^{-\frac{4}{3}(z-1)}\right\}. 
\]
When we substitute this outstanding parameter value of $u_0 =: \hat{u}_0 \mapsto \frac{1}{x} e^{-\frac{4}{3}(z-1)}$ 
into the next expression for the second derivative of the same function 
$\widehat{\mathcal{C}}(u, x; z)$ we obtain 
\begin{align*} 
\frac{\partial^2}{{\partial u}^2}\left[\widehat{\mathcal{C}}(u, x; z)\right] \Biggr\rvert_{u = \hat{u}_0} & = 
     \exp\left(-\frac{55}{4} \log^2(2)\right) x^2 2^{\frac{8 z^3-27 z^2+32 z-16}{4 (z-1)^2}} 
     3^{-z+\frac{1}{1-z}+1} e^{\frac{5 z^2-16 z+8}{3 (z-1)}} \times \\ 
     & \phantom{=\times} \times (1-z)^{z+\frac{1}{z-1}-2} z^2
     \log(2)^{\frac{z^2}{1-z}} > 0, 
\end{align*} 
provided that $z < 1$. 
The restriction to $0 \leq z < 1$ is equivalent to requiring that 
$1 \leq k \leq \log\log x$ in Theorem \ref{theorem_HatPi_ExtInTermsOfGz}. 
This restriction on $k$ to note 
leads to a minimum value on the partial product, or lower bound, at this $u = \hat{u}_0$ 
since the second derivative is positive at this critical value for $z$ within this range. 

After substitution of $u = \frac{1}{x} e^{-\frac{4}{3}(z-1)}$ into the expression for 
$\widehat{\mathcal{C}}(u, x; z)$ defined above, we have that 
\[
\widehat{\mathcal{C}}(u, x; z) \geq \exp\left(-\frac{55}{4} \log^2(2)\right) \cdot 2^{\frac{9}{16}} 
     \left(\frac{1-z}{3e\log 2}\right)^3 \times \left(\frac{4(1-z)}{3e\log 2}\right)^z. 
\]
Finally, since $z \equiv z(k, x) = \frac{k}{\log\log x}$ and $k \in [0, R\log\log x)$, we obtain that 
for small $k$ and $x \gg 1$ large $\Gamma(z+1) \approx 1$, and for $k$ towards the upper range of 
its interval that $\Gamma(z+1) \approx \Gamma(5/2) = \frac{3}{4} \sqrt{\pi}$. 
In total, what we get out of these formulas is stated up to accurate 
constant factor as in the theorem bounds. 
\end{proof} 

\newpage
\section{Precisely bounding the Dirichlet inverse functions, $g^{-1}(n)$, on average} 
\label{Section_InvFunc_PreciseExpsAndAsymptotics} 

This section is essential because we prove key results that allow us to 
bound the oscillatory 
Dirichlet inverse functions $g^{-1}(n)$ from the exact formula for $M(x)$ given in 
\eqref{eqn_Mx_gInvnPixk_formula}. 
Using summation by parts, we eventually show that this formula can be approximated with a clear 
dependency on the summatory functions $G^{-1}(x)$ of $g^{-1}(n)$ by the integral formula we later 
state and prove as Proposition \ref{prop_Mx_SBP_IntegralFormula}. 

The pages of tabular data given as Table \ref{table_conjecture_Mertens_ginvSeq_approx_values} 
given in the appendix section starting on 
page \pageref{table_conjecture_Mertens_ginvSeq_approx_values} are intended to 
provide clear insight into why we arrived at the convenient approximations to 
$g^{-1}(n)$ proved in this section. The table data offers 
numerical data formed by examining the approximate behavior 
at work here for the asymptotically 
small order cases of $1 \leq n \leq 350$ with \emph{Mathematica}. 

It happens that Conjecture \ref{lemma_gInv_MxExample} is not the most 
simple accurate way to express the limiting behavior of the 
Dirichlet inverse functions $g^{-1}(n)$ we can formulate, 
though it does capture an important characteristic that is true more globally than just at the 
squarefree $n \geq 1$. Namely, that these 
functions can be expressed via more simple formulas than inspection of the initial 
repetitive, quasi-periodic sequence properties in the table might otherwise suggest. 

With all of this in mind, we define the following sequence for integers $n \geq 1, k \geq 0$: 
\begin{align} 
\label{eqn_CknFuncDef_v2} 
C_k(n) := \begin{cases} 
     \varepsilon(n), & \text{ if $k = 0$; } \\ 
     \sum\limits_{d|n} \omega(d) C_{k-1}(n/d), & \text{ if $k \geq 1$. } 
     \end{cases} 
\end{align} 
The sequence of important semi-diagonals of these functions begins as 
\cite[\seqnum{A008480}]{OEIS} 
\[
\{\lambda(n) \cdot C_{\Omega(n)}(n) \}_{n \geq 1} \mapsto \{
     1, -1, -1, 1, -1, 2, -1, -1, 1, 2, -1, -3, -1, 2, 2, 1, -1, -3, -1, \
     -3, 2, 2, -1, 4, 1, 2, \ldots \}. 
\]
Notice that by expanding the recursively-based definition in \eqref{eqn_CknFuncDef_v2} 
out to its maximal depth by nested divisor sums, for fixed $n$, $C_k(n)$ is seen to 
only ever possibly be non-zero for $k \leq \Omega(n)$. 
This observation follows from the fact that 
a minimal condition on the forms of 
divisors $d > 1$ of $n$ requires that $d$ have at least a single prime factor. 
Thus, the effective range of $k$ for fixed $n$ is restricted by the 
conditions that $C_0(n) = \delta_{n,1}$ and $C_k(n) = 0$ $\forall k > \Omega(n)$. 
That is, for $n \geq 2$, the contributions from summations over $C_k(n)$ are only 
significant when $1 \leq k \leq \Omega(n)$. 

\begin{lemma}[An exact formula for $g^{-1}(n)$] 
\label{lemma_AnExactFormulaFor_gInvByMobiusInv_v1} 
For all $n \geq 1$, we have that 
\[
g^{-1}(n) = \sum_{d|n} \mu(n/d) \lambda(d) C_{\Omega(d)}(d). 
\]
\end{lemma}
\begin{proof} 
We first write out the standard recurrence relation for the Dirichlet inverse of 
$\omega+1$ as 
\begin{align*} 
g^{-1}(n) & = - \sum_{\substack{d|n \\ d>1}} (\omega(d) + 1) g^{-1}(n/d) && \implies \\ 
     (g^{-1} \ast 1)(n) & = -(\omega \ast g^{-1})(n). 
\end{align*} 
Now by repeatedly expanding the right-hand-side, and removing corner cases in the nested sums since 
$\omega(1) = 0$ by convention, we find that 
\[
(g^{-1} \ast 1)(n) = (-1)^{\Omega(n)} C_{\Omega(n)}(n) = \lambda(n) C_{\Omega(n)}(n). 
\]
The statement follows by M\"obius inversion applied to each side of the last equation. 
\end{proof} 

\begin{cor} 
\label{cor_AnExactFormulaFor_gInvByMobiusInv_nSqFree_v2} 
For all squarefree integers $n \geq 1$, we have that 
\[
g^{-1}(n) = \lambda(n) \times \sum_{d|n} C_{\Omega(d)}(d). 
\]
\end{cor} 
\begin{proof} 
Since $g^{-1}(1) = 1$, clearly the claim is true for $n = 1$. Suppose that $n \geq 2$ and that 
$n$ is squarefree. Then $n = p_1p_2 \cdots p_{\omega(n)}$ where $p_i$ is prime for all 
$1 \leq i \leq \omega(n)$. So we can transform the exact divisor sum guaranteed for all $n$ in 
Lemma \ref{lemma_AnExactFormulaFor_gInvByMobiusInv_v1} into the following: 
\begin{align*} 
g^{-1}(n) & = \sum_{i=1}^{\omega(n)} \sum_{\substack{d|n \\ \omega(d)=i}} (-1)^{\omega(n) - i} (-1)^{i} \cdot 
     C_{\Omega(d)}(d) + \mu(1) \lambda(n) C_1(1) \\ 
     & = \lambda(n) \left[\sum_{i=1}^{\omega(n)} \sum_{\substack{d|n \\ \omega(d)=i}} C_{\Omega(d)}(d) + 1\right] \\ 
     & = \lambda(n) \times \sum_{d|n} C_{\Omega(d)}(d). 
\end{align*} 
The signed computations in the first of the previous equations is justified by noting that $\lambda(n) = (-1)^{\omega(n)}$ 
whenever $n$ is squarefree, and that for $d$ squarefree with $\omega(d) = i$, $\Omega(d) = i$. 
\end{proof} 

\begin{example}[Special cases of the functions $C_k(n)$ for small $k$] 
\label{example_SpCase_Ckn} 
We cite the following special cases which should be easy enough to see on paper by 
explicit computation using \eqref{eqn_CknFuncDef_v2}: 
\NBRef{A07-2020-04-26} 
\begin{align*} 
C_0(n) & = \delta_{n,1} \\ 
C_1(n) & = \omega(n) \\ 
C_2(n) & = d(n) \times \sum_{p|n} \frac{\nu_p(n)}{\nu_p(n)+1} - \gcd\left(\Omega(n), \omega(n)\right). 
\end{align*} 
We have a recurrence relation between successive $C_k(n)$ values over $k$ of the form 
given in Lemma \ref{lemma_Ckn_recFormula_v1}. 
\end{example} 

\begin{lemma}[Recurrence relation between the $C_k(n)$] 
\label{lemma_Ckn_recFormula_v1} 
For $k \geq 2$, we have that 
\begin{equation}
\label{eqn_Ckn_recFormula_v1} 
C_k(n) = \sum_{p|n} \sum_{d\bigr\rvert\frac{n}{p^{\nu_p(n)}}} \sum_{i=0}^{\nu_p(n)-1} C_{k-1}\left(d \cdot p^i\right). 
\end{equation}
\end{lemma} 
\begin{proof} 
First, we re-write the formula for $C_k(n)$ given directly by the recursion implicit to its 
definition stated in \eqref{eqn_CknFuncDef_v2} for $k \geq 2$:
\begin{align} 
\notag 
C_k(n) & = \sum_{d|n} \omega(d) C_{k-1}(n/d) \\ 
\notag 
     & = \sum_{d|n} \sum_{p|d} C_{k-1}(n/d) \\ 
\label{eqn_proof_tag_Ckn_rec_divSumExp_v1} 
     & = \sum_{d|n} \sum_{p|\frac{n}{d}} C_{k-1}(d). 
\end{align} 
We wish to interchange the inner and outer divisor sums in the last equation. 
Since $p|\frac{n}{d}$ where $d$ ranges over the divisors of $n$, the exchanged outer sum should 
clearly be indexed over $p|n$. 

We claim that in fact $A_n = B_n$ where these two sets are defined for all $n \geq 2$ as follows: 
\begin{align*} 
A_n & := \left\{d: d|n, p \bigr\rvert\frac{n}{d}\right\} \\ 
B_n & := \left\{d \cdot p^i: 0 \leq i < \nu_p(n), p|n, d \biggr\rvert \frac{n}{p^{\nu_p(n)}} \right\}. 
\end{align*} 
The truth of this claim then establishes \eqref{eqn_Ckn_recFormula_v1} based on 
\eqref{eqn_proof_tag_Ckn_rec_divSumExp_v1}. We prove the claim by subset inclusion below. \\ 
\textit{Proof that $A_n \subseteq B_n$. } 
Suppose that $d|n$ and $p|\frac{n}{d}$ with $p$ prime. This implies that 
$\exists m \in \mathbb{Z}^{+}$ such that $dm=n$. The assumptions also imply that 
$\exists j \in \mathbb{Z}^{+}$ such that $pj = \frac{n}{d} = \frac{n}{\frac{n}{m}} = m$. 
So $p|m$ which implies that $p|n$. Since $d|n$ with $p|n$, clearly there is some $0 \leq i < \nu_p(n)$ 
such that $d = d_0p^i$ where $d_0 \bigr\rvert \frac{n}{p^{\nu_p(n)}}$. \\ 
\textit{Proof that $B_n \subseteq A_n$. } 
First suppose that $i = 0$, let $p|n$, and suppose that $dp^{i} = d$ satisfies 
$d \bigr\rvert \frac{n}{p^{\nu_p(n)}}$. We have to show that $d \in A_n$. Clearly, as $p|n$, $\nu_p(n) \geq 1$. 
So $d \bigr\rvert \frac{n}{p^{\nu_p(n)}}$ implies also that $d|n$. This assumption also shows that 
$\exists j \in \mathbb{Z}^{+}$ with $dj \cdot p^{\nu_p(n)} = n$ $\implies$ 
$p \cdot \left(j p^{\nu_p(n)-1}\right) = \frac{n}{d}$, or eqivalently that $p | \frac{n}{d}$ since 
$j p^{\nu_p(n)-1} \in \mathbb{Z}^{+}$. 
Now suppose that $1 \leq i < \nu_p(n)$, $p|n$, and that $d \bigr\rvert \frac{n}{p^{\nu_p(n)}}$. 
We have to show that $dp^{i} \in A_n$. Clearly, we have that $\nu_p(n) \geq 2$ in this case. 
The third assumption implies that $\exists j \in \mathbb{Z}^{+}$ such that $dj \cdot p^{\nu_p(n)} = n$, 
which in turn implies that $dp^{i} | n$. Moreover, since 
$j p^{\nu_p(n) - 1 - i} \in \mathbb{Z}^{+}$ here, 
we get that $p \bigr\rvert \frac{n}{dp^i}$. Thus, $B_n \subseteq A_n$. 
\end{proof} 

Theorem \ref{theorem_Ckn_GeneralAsymptoticsForms} from the introduction is proved next. 
The theorem makes precise what these formulas already 
suggest about the main terms of the growth rates of 
$C_k(n)$ as functions of $k,n$ for limiting cases of $n$ large for fixed $k$ which is bounded in $n$, 
but taken as an independent parameter. 

\NBRef{A08-2020-04-26} 
\begin{proof}[Proof of Theorem \ref{theorem_Ckn_GeneralAsymptoticsForms}] 
\label{proofOf_theorem_Ckn_GeneralAsymptoticsForms} 
We can see by Example \ref{example_SpCase_Ckn} that $C_1(n)$ 
satsfies the formula we must establish when $k := 1$ since $\mathbb{E}[\omega(n)] = \log\log n$. 
We prove our bounds by induction on $k$. 
In particular, suppose that $k \geq 2$ and let the inductive assumption for all $1 \leq m < k$ 
be that 
\[
\mathbb{E}[C_m(n)] \geq (\log\log n)^{2m-1}. 
\]
Now we have by the recursive formula in \eqref{eqn_Ckn_recFormula_v1} that 
\begin{align} 
\notag 
C_k(n) & = \sum_{p|n} \sum_{d\bigr\rvert\frac{n}{p^{\nu_p(n)}}} \sum_{i=1}^{\nu_p(n)} \log\log(dp^i)^{2k-3} \\ 
\label{eqn_proof_tag_Ckn_AsymptoticExp_v1} 
     & \sim \sum_{p|n} \sum_{d\bigr\rvert\frac{n}{p^{\nu_p(n)}}} \left[ 
     \int \log\log(dp^{\alpha})^{2k-3} d\alpha\right] \Biggr\rvert_{\alpha=\nu_p(n)}. 
\end{align} 
The inner integral in the previous equation can be evaluated using the 
limiting asymptotic expansions for the incomplete gamma function stated in 
Section \ref{subSection_OtherFactsAndResults}. 
In particular, for $p|n$ and $n \geq 2$ large, we let the parameters assume average order values of 
\[
\mathbb{E}[\nu_p(n)] = \log\log n, \mathbb{E}[p] = \frac{n}{\log n}. 
\]
Then we evaluate the integral from above as 
\begin{align*} 
\int (\log\log(dp^{\alpha}))^{2k-3} d\alpha & \sim 
     \alpha \times \log\log(dp^{\alpha})^{2k-3} + 
     \frac{\log d}{\log p} \times \log\log(dp^{\alpha})^{2k-3} \\ 
     & \SuccSim 
     \alpha \times \log\log(dp^{\alpha})^{2k-3}. 
\end{align*} 
We know that the average order of the number of primes $p|n$ is given by 
$\mathbb{E}[\omega(n)] = \log\log n$, so approximating $p$ as the cited function of $n$ initially 
allows us to take a factor of $\log\log n$ and remove the outer divisor sum in 
\eqref{eqn_proof_tag_Ckn_AsymptoticExp_v1}. So we obtain that 
\begin{align*} 
\mathbb{E}[C_k(n)] & \sim (\log\log n)^2 \mathbb{E}\left[\frac{d(n)}{(\nu_p(n) + 1)}\right] \times 
     \left[\log\log n + \log\log\log n + \frac{\log d}{\log n \cdot \log\log n} - 
     \frac{\log\log n}{\log n}\right]^{2k+3} \\ 
     & \sim (\log n) \cdot (\log\log n) \times \left[\log\log n + \log\log\log n + o(1)\right]^{2k-3} \\ 
     & \sim (\log n) \cdot (\log\log n)^{2k-2} \\ 
     & \geq (\log\log n)^{2k-1}, \mathrm{\ as\ } n \rightarrow \infty. 
\end{align*} 
In the previous equation, we have used that the average order of the divisor function, $d(n)$, 
is given by $\mathbb{E}[d(n)] = \log n$ \cite[\S 27.11]{NISTHB}. 
Thus the claim holds by mathematical induction. 
\end{proof} 

Note that in Section \ref{Section_ProofOfValidityOfAverageOrderLowerBounds} 
(proof of fact (A)) we show that when $k := \Omega(n)$ depends on $n$, then 
\[
\mathbb{E}[C_{\Omega(n)}(n)] \gg (\log n) (\log\log n)^{2\log\log n - 1} \geq \log n \cdot \log\log n.  
\]
\newpage 
\section{A rigorous justification for using so-called average case lower bounds to prove 
         Corollary \ref{cor_ThePipeDreamResult_v1}} 
\label{Section_ProofOfValidityOfAverageOrderLowerBounds} 

The point of proving the results in this section before moving onto the core results needed in 
the next section is to provide a rigorous justification for the intuition we sketched in 
Section \ref{subSection_Intro_RigorToTheAverageCaseEstimates} of the introduction. 
That is, we expect our arithmetic functions that are closely 
tied to the additive functions, $\omega(n)$ and $\Omega(n)$, to similarly behave regularly (and 
infinitely often) in accordance with their values being close to the average case for large $x$. 

What we have established so far, and will establish for $G^{-1}(x)$ in 
Section \ref{Section_KeyApplications}, are lower bound estimates that hold essentially 
\emph{on average}. 
This means that for limiting cases of $x$, we need to show that the expected value lower bounds 
are achieved in asymptotic order predictably often 
within some small window depending linearly on $x$ that we will determine precisely in this section. 

\begin{proof}[Proof of Theorem \ref{theorem_CondAvgOrderGInvxSummatoryFunc_v1}] 
\label{proofOf_theorem_CondAvgOrderGInvxSummatoryFunc_v1} 
The result is obtained simply by contradiction. Suppose that $x$ is so large that the inequalities in the 
hypothesis hold. Also, suppose that for all $x_0 \in [(B-\varepsilon) x, (C+\varepsilon) x]$, we have that 
\begin{equation} 
\label{eqn_proof_tag_IneqSetG0x0_DNHold_v1} 
|G^{-1}(x_0)| < |G_E^{-1}(x_0)|. 
\end{equation} 
We have supposed that the constants $B,C \in (0, 1)$ are the tightest possible bounds on the next set as 
$x \rightarrow \infty$. 
That is, for all sufficiently large $x$ if 
$$\mathcal{G}_0(x) := \frac{1}{x} \cdot \#\left\{n \leq x: |G^{-1}(x_0)| - |G_E^{-1}(x_0)| \leq 0\right\},$$ 
denotes the density at $x$, then there is no larger $B > 0$ such that $B+o(1) \leq \mathcal{G}_0(x)$ and 
no smaller $C < 1$ such that $\mathcal{G}_0(x) \leq C + o(1)$. 
Let $\varepsilon \in (0, 1)$ satisfy $0 < B - \varepsilon, C + \varepsilon < 1$. We need to show that such a 
concrete fixed $\varepsilon$ satisfying the conditions in the theorem exists (depending only on $B,C$). 

For $n \geq 1$, we have the disjoint set decomposition of the positive integers $\{n \leq x\}$ given by  
\[
\{n \leq x\} = \{1 \leq n < (B-\varepsilon) x\} \oplus \{(B-\varepsilon) x \leq n \leq (C + \varepsilon) x\} \oplus 
     \{(C+\varepsilon) x < n \leq x\}, 
\]
where the three disjoint sets, respectively denoted by $\mathcal{D}_i(x)$ for $i = 1,2,3$, yield that as 
$x \rightarrow \infty$, if \eqref{eqn_proof_tag_IneqSetG0x0_DNHold_v1} is true, then 
\begin{align*} 
\mathcal{G}_1(x) & := \frac{1}{x} \cdot \#\left\{n \in \mathcal{D}_1(x): |G^{-1}(x_0)| - |G_E^{-1}(x_0)| \leq 0\right\} 
     \in [(B-\varepsilon)^2, (B-\varepsilon) (C+\varepsilon)] \\ 
\mathcal{G}_2(x) & := \frac{1}{x} \cdot \#\left\{n \in \mathcal{D}_2(x): |G^{-1}(x_0)| - |G_E^{-1}(x_0)| \leq 0\right\} 
     = C - B \\ 
\mathcal{G}_3(x) & := \frac{1}{x} \cdot \#\left\{n \in \mathcal{D}_3(x): |G^{-1}(x_0)| - |G_E^{-1}(x_0)| \leq 0\right\} 
     \in [(B-\varepsilon)-(B-\varepsilon) (C+\varepsilon), (C+\varepsilon)-(C+\varepsilon)^2]. 
\end{align*} 
So we obtain that 
\[
(B-\varepsilon)^2 + (1+\varepsilon-B) (C+\varepsilon) + o(1) \leq \mathcal{G}_0(x) \leq 
     (C+\varepsilon)(2-C - \varepsilon) + (B-\varepsilon) (C + \varepsilon-1) + o(1). 
\]
We show that we can pick any $\varepsilon > 0$ that satisfies 
$B - \varepsilon < C < 1-\varepsilon$. In fact, given such a choice of this parameter, 
we have that 
\[
C + \varepsilon - \left[(C+\varepsilon)(2-C - \varepsilon) + (B-\varepsilon) (C + \varepsilon-1)\right] = 
     (C+\varepsilon-B)(C + \varepsilon-1) < 0. 
\]
This implies a contradiction to the maximality of our bound $C \in (0, 1)$. 
Then we must have that our contrary assumption on $x_0$ is invalid as $x \rightarrow \infty$. 
Indeed, there is such a fixed $\varepsilon > 0$ and such a $x_0 \in [(B-\varepsilon) x, (C+\varepsilon) x]$ 
such that $|G^{-1}(x_0)| \geq |G_E^{-1}(x_0)|$ for all sufficiently large $x$. 
\end{proof} 

\subsection{Proving the necessary hypothesis in Theorem \ref{theorem_CondAvgOrderGInvxSummatoryFunc_v1}} 

\begin{facts} 
To prove the hypothesis assumed by the conclusion of 
Theorem \ref{theorem_CondAvgOrderGInvxSummatoryFunc_v1}, 
we require the following two facts of our notation for average order: 
\begin{itemize} 
     \item[\textbf{(A)}] $\mathbb{E}[C_{\Omega(n)}(n)] \gg (\log n) \cdot (\log\log n)^{2\mathbb{E}[\Omega(n)]-1} \gg 
                          \log n \cdot \log\log n$; and 
     \item[\textbf{(B)}] $\mathbb{E}\left[\sum_{d|n} C_{\Omega(d)}(d)\right] \geq \sum_{d|n} \mathbb{E}[C_{\Omega(d)}(d)]$ 
     for all $n$ in some set $\mathcal{C}_E$ of such that 
     $\mathcal{C}_E$ has asymptotic density one. 
\end{itemize} 
\end{facts} 

\begin{proof}[Proof of fact (A)] 
We utilize Theorem \ref{theorem_MV_Thm7.20-init_stmt} 
to show each of the following as $x \rightarrow \infty$: 
\begin{align} 
\notag 
\frac{1}{x} \cdot \#\{n \leq x: \Omega(n) \geq (1+\delta) \log\log x\} & \ll (\log x)^{\delta - (1+\delta)\log(1+\delta)} 
     && = o(1), \forall \delta > 0, \delta \approx 0^{+} \\ 
\label{eqn_proof_tag_OmeganGeqLeqAsymptoticDensityCalcs_v1} 
\frac{1}{x} \cdot \#\{n \leq x: \Omega(n) \leq (1+\rho) \log\log x\} & \ll (\log x)^{\rho - (1+\rho)\log(1+\rho)} 
     && = o(1), \forall \rho > 0, \rho \approx 0^{+}.   
\end{align} 
Thus with our result for fixed $1 \leq k \leq \Omega(n)$ from 
Theorem \ref{theorem_Ckn_GeneralAsymptoticsForms}, 
we can conclude that 
\begin{align*} 
\mathbb{E}[C_{\Omega(n)}(n)] & \gg \frac{1}{n} \sum_{d \leq n} (\log\log d)^{2\Omega(d)-1} \\ 
     & \sim (\log n) \cdot (\log\log n)^{2\log\log n - 1} \\ 
     & \gg (\log\log n)^{2\log\log n - 1}, \mathrm{\ as\ } n \rightarrow \infty \\ 
\mathbb{E}[C_{\Omega(n)}(n)] & \gg \log n \cdot \log\log n
\end{align*} 
The results expanded in 
\eqref{eqn_proof_tag_OmeganGeqLeqAsymptoticDensityCalcs_v1} show that we can expect the 
asymptotic density of the $n \leq x$ where $\Omega(n) \not{\approx} \mathbb{E}[\Omega(n)]$ to be small, tending 
to zero as $n \rightarrow \infty$. 
The previous two implications follow by an expansion by the binomial series where 
\begin{align*} 
\frac{1}{t} \times \int_{e^e}^{n} (\log\log t)^{2\log\log t - 1} dt & \approx 
     \frac{1}{t} \times \int_{e^e}^{n} \frac{(1 + \log\log t)^{2\log\log t}}{\log\log t} dt \\ 
     & = \frac{1}{t} \times \int_{e^e}^{n} 
     \sum_{s \geq 0} \sum_{k=0}^{s} \gkpSI{s}{k} (2\log\log t)^{k} (-1)^{s-k} \times 
     \frac{(\log\log t)^{s-1}}{s!} dt. 
\end{align*} 
Then since for any fixed $m$ we have integrating by parts that 
\begin{align*} 
\frac{1}{t} \times \int (\log\log t)^m dt & = \frac{1}{t}\left( 
     t \cdot (\log t) (\log\log t)^m - (\log t) (\log\log t)^m\right) + C \\ 
     & \sim (\log t) (\log\log t)^m \\ 
     & \gg (\log\log t)^m, 
\end{align*} 
we obtain our lower bounds on the average order of $C_{\Omega(n)}(n)$. 
\end{proof} 

\begin{proof}[Proof of fact (B)] 
We begin by proving a subclaim. Let the set defined by 
\begin{align*} 
\mathcal{D}_{+} & := \left\{n \geq 1: d(n) > \log n\right\}, 
\end{align*} 
have corresponding asymptotic density
\begin{align*} 
\alpha_{+} & := \lim_{x \rightarrow \infty} \frac{1}{x} \cdot \#\{n \leq x: n \in \mathcal{D}_{+}\}, 
\end{align*} 
if it exists. 
We claim that the limit exists and that $\alpha_{+} = 0$. To prove this we first note the following 
precise classical 
result on the asymptotic tendencies of the summatory function of $d(n)$ \cite[\S 27.11]{NISTHB}: 
\begin{equation} 
\label{eqn_proof_tag_DnAsymptotics_v1} 
\sum_{n \leq x} d(n) = x\log x + (2\gamma-1) x + O(\sqrt{x}). 
\end{equation} 
Then we see that as $x \rightarrow \infty$, by Abel summation we have that when 
$\alpha_{+}^{\ast} := \max\left(\alpha_{+}, \frac{1}{x}\right)$ 
\begin{align*} 
\sum_{\substack{n \leq x \\ n \in \mathcal{D}_{+}}} \log n & \geq \sum_{n \leq \alpha_{+}^{\ast} x} \log n \\ 
     & = (\alpha_{+}^{\ast} x) \cdot \log(\alpha_{+}^{\ast} x) - \int_0^{\alpha_{+}^{\ast} x} dt \\ 
     & \sim (\alpha_{+}^{\ast} x) \log(\alpha_{+}^{\ast} x). 
\end{align*} 
Thus we can bound the summatory function of $d(n)$ for $n \leq x$ as $x \rightarrow \infty$ by 
\begin{align*} 
\sum_{n \leq x} d(n) & \geq \sum_{\substack{n \leq x \\ n \notin \mathcal{D}_{+}}} d(n) + 
     \sum_{\substack{n \leq x \\ n \notin \mathcal{D}_{+}}} \log n \\ 
     & \geq (1-\alpha_{+}) x \left(\log x + \log(1- \alpha_{+}) + 2\gamma-1\right) + 
     (\alpha_{+} x) \log x + O(\sqrt{x}), 
\end{align*} 
which can only be true if $\alpha_{+} = 0$ according to the well known formula 
in \eqref{eqn_proof_tag_DnAsymptotics_v1}. 
By a similar argument reversing inequalities, we can see that indeed $\alpha_{-} = 0$ 
corresponding to the reversed inequality set is true as well. 

We next bound the average order expectations we see in the upper bound of the 
inequality using the known identity 
\[
\sum_{n \leq x} \sum_{d|n} f(d) = \sum_{d \leq x} f(d) \Floor{x}{d}, 
\]
and summation by parts: 
\begin{align} 
\notag 
\mathbb{E}\left[\sum_{d|n} C_{\Omega(d)}(d)\right] & \sim \sum_{d \leq x} \frac{C_{\Omega(d)}(d)}{d} \\ 
\notag 
     & \sim \mathbb{E}[C_{\Omega(n)}(n)] + \sum_{d<n} \frac{\mathbb{E}[C_{\Omega(d)}(d)]}{d} \\ 
\label{eqn_proof_tag_ExpectedUpperBound_v1} 
     & \sim (\log n) \cdot (\log\log n)^{2\log\log n - 1} + \frac{(\log n)^2}{2} \cdot (\log\log n)^{2\log\log n - 1}. 
\end{align} 
The rightmost term in the previous equation follows by integration on the monotone non-decreasing 
summand (for all large enough $d > e^e$) as we justified using a close binomial series 
approximation in the proof of fact (A). 
The new part of the result in \eqref{eqn_proof_tag_ExpectedUpperBound_v1} 
is calculated according to the 
following indefinite integral formula when $m$ is fixed and where we approximate the result 
using the asymptotic formula for the incomplete gamma function from 
Section \ref{subSection_OtherFactsAndResults}:  
\[
\int \frac{(\log t) (\log\log t)^{m}}{t} dt = \frac{(\log t)^2}{2} \times (\log\log t)^m + C. 
\]
Next, we bound the lower portion of the expected inequality from above to agree at a middleground 
where the $n \in \mathcal{C}_E$ converges based on \eqref{eqn_proof_tag_ExpectedUpperBound_v1}: 
\begin{align} 
\label{eqn_proof_tag_ExpectedUpperBound_v2} 
\sum_{d|n} \mathbb{E}[C_{\Omega(d)}(d)] & \ll \sum_{d|n} (\log d) \cdot (\log\log d)^{2\log\log n - 1}. 
\end{align} 
Consider that for $n$ large, $\forall \floor{\sqrt{n}} \leq j < n$, we have that 
$\pi(j) = \pi(\floor{\sqrt{n}})$. 
This inequality implies that for some $p|n$, the smallest $d|n$ smaller than $n$ is of the form 
$d = \frac{n}{p} \leq \sqrt{n}$ for some prime divisor $p$ of $n$. Then we have from 
\eqref{eqn_proof_tag_ExpectedUpperBound_v2} that 
\begin{align} 
\label{eqn_proof_tag_ExpectedUpperBound_v3} 
\sum_{d|n} \mathbb{E}[C_{\Omega(d)}(d)] & \leq (\log n) \cdot (\log\log n)^{2\log\log n - 1} + 
     d(\floor{\sqrt{n}}) (\log n) (\log\log n)^{2\log\log n - 1}. 
\end{align} 
As we have shown above, the set of positive integers on which $d(n) > \log n$ has thin 
asymptotic density $\alpha_{+} = 0$. 
Now what this implies from \eqref{eqn_proof_tag_ExpectedUpperBound_v3} 
is that on an infinite set $\mathcal{A}_0 \subseteq \mathcal{C}_E$ of asymptotic density one, 
we have that 
\[
\sum_{d|n} \mathbb{E}[C_{\Omega(d)}(d)] \leq (\log n) \cdot (\log\log n)^{2\log\log n - 1} + 
     \frac{(\log n)^2}{2} (\log\log n)^{2\log\log n - 1}. 
\]
Thus by \eqref{eqn_proof_tag_ExpectedUpperBound_v1} and the previous equation, we have that 
\[
\mathbb{E}\left[\sum_{d|n} C_{\Omega(d)}(d)\right] \geq \sum_{d|n} \mathbb{E}[C_{\Omega(d)}(d)], 
     \forall n \in \mathcal{A}_0. 
\]
Since $\mathcal{A}_0$ has limiting density of one, and $\mathcal{A}_0 \subseteq \mathcal{C}_E$, i.e., 
the $\mathcal{C}_E$ has an asymptotic density at least as great as that of $\mathcal{A}_0$, 
we conclude that the limiting density of $\mathcal{C}_E$ is also one. 
That is, 
\[
\lim_{x \rightarrow \infty} \frac{1}{x} \cdot \#\{n \leq x: n \in \mathcal{C}_E\} = 1. 
     \qedhere 
\]
\end{proof}  

\begin{remark} 
\label{remark_AsymptoticDensitiesOfExceptionalSets_v1} 
What we have actually argued in less generality for the divisor function case above holds for 
the average order expectation of positive arithmetic functions in general. That is, 
suppose that $f(n) > 0$ for all $n \geq 1$, and that there exists a function $F$ such that 
\[
\sum_{n \leq x} f(n) \sim x \cdot F(x). 
\]
If $F(x) \not{\rightarrow} 0$ as $x \rightarrow \infty$, then the asymptotic density of the set of 
large order exeptional values of $f$ is zero: 
\[
\lim_{x \rightarrow \infty} \frac{1}{x} \cdot \#\{n \leq x: f(n) > F(n)\} = 0. 
\]
\end{remark} 

\begin{facts}[Bounds on the asymptotic densities of the sets over the parity of $\lambda(n) = \pm 1$] 
\label{lemma_AsymptoticDensitiesParityOmegan_v1} 
Let the asymptotic densities for the distinct parity of $\Omega(n)$ (sign of $\lambda(n)$) 
be denoted by 
\begin{align*} 
\lambda_{+} & := \lim_{x \rightarrow \infty} \frac{1}{x} \cdot \#\{n \leq x: \lambda(n) = +1\} \\ 
\lambda_{-} & := \lim_{x \rightarrow \infty} \frac{1}{x} \cdot \#\{n \leq x: \lambda(n) = -1\}. 
\end{align*} 
Then we can prove that both limits exist and $\lambda_{+} = \lambda_{-} = \frac{1}{2}$. 
\end{facts} 

\begin{lemma}[Bounded expectation of $g^{-1}(n)$] 
\label{lemma_BddExpectationOfgInvn} 
For all large enough $n$, we have that 
\[
|\mathbb{E}[g^{-1}(n)]| \leq 2\log(n) 2\gamma + o(1). 
\]
\end{lemma}
\begin{proof} 
Consider that by using Lemma \ref{lemma_AnExactFormulaFor_gInvByMobiusInv_v1} we have 
\begin{align*} 
\left\lvert \frac{1}{x} \sum_{n \leq x} g^{-1}(n) \right\rvert \\ 
     & \leq \left\lvert \frac{1}{x} \sum_{n \leq x} \sum_{d|n} \mu(n/d) \lambda(d) C_{\Omega(d)}(d) \right\rvert \\ 
     & = \left\lvert \frac{1}{x} \sum_{d \leq x} \lambda(d) C_{\Omega(d)}(d) M\left(\frac{x}{d}\right) \right\rvert \\ 
     & \leq \sum_{d \leq x} \frac{C_{\Omega(d)}(d)}{d} \leq \sum_{d \leq x} \frac{2}{d} \\ 
     & = 2\log(x) + 2\gamma + O\left(\frac{1}{x}\right). 
\end{align*} 
In the previous equations we used that $C_{\Omega(n)}(n) \geq 2$ for all $n \geq 1$. 
We have also used that $|M(x)| \leq x$ trivially for all $x \geq 1$. 
\end{proof} 

Notice that since for all $n \geq 2$ 
$|g^{-1}(n)|$ is minimal at prime powers with $g^{-1}(p^{\alpha}) = (-2)^{\alpha}$ 
for all primes $p$ and integers $\alpha \geq 1$, we also have that 
$\mathbb{E}|g^{-1}(n)| \geq 2$. Since $|\mathbb{E}[g^{-1}(n)]| \leq \mathbb{E}|g^{-1}(n)|$, 
by taking reciprocals we can invert the order of inequality above using 
Lemma \ref{lemma_BddExpectationOfgInvn} 
as we employ its statement in the proof that we can attain the 
necessary hypotheses when $x$ is large to reach the conclusion of 
Theorem \ref{theorem_CondAvgOrderGInvxSummatoryFunc_v1} proved below. 

\begin{prop} 
\label{prop_DensityOfGInvxPosAndBdd} 
Let the set where $G^{-1}(x)$ is non-negative be defined as 
\[
\mathcal{G}_{+} := \left\{n \leq x: G^{-1}(x) \geq 0\right\}. 
\]
We claim that for all large $x \rightarrow \infty$, the density of this set is 
positive: 
\[
0 < \frac{1}{x} \cdot \#\{n \leq x: n \in \mathcal{G}_{+}\} < 1. 
\]
Moreover, if a limiting asymptotic density for $\mathcal{G}_{+}$ exists, it does not 
tend to zero as $x \rightarrow \infty$: 
\[
\lim_{x \rightarrow \infty} \frac{1}{x} \cdot \#\{n \leq x: n \in \mathcal{G}_{+}\} \neq 0. 
\]
\end{prop} 

Note that the proposition above also implies that the corresponding set $\mathcal{G}_{-}$ over which 
$G^{-1}(x) < 0$ has positive density for all $x$ sufficiently large, and that this density does not 
tend to zero as $x \rightarrow \infty$. 
We will prove Proposition \ref{prop_DensityOfGInvxPosAndBdd} after we prove 
Proposition \ref{prop_Mx_SBP_IntegralFormula} in the next section. 

\begin{proof}[Proof of the hypothesis of Theorem \ref{theorem_CondAvgOrderGInvxSummatoryFunc_v1}]
Let $G_E^{-1}(x)$ be defined as in \eqref{eqn_GEInvxSummatoryFuncDef_v1} of the theorem. 
We need to find some absolute tight (tightest possible) constants 
$B, C \in (0, 1]$ such that as $x \rightarrow \infty$ 
\begin{equation} 
\label{eqn_proof_tag_ThmConstsBCHyp_defs_v1} 
B + o(1) \leq \frac{1}{x} \cdot \#\left\{n \leq x: |G^{-1}(n)| - |G_E^{-1}(n)| \leq 0\right\} \leq 
     C + o(1). 
\end{equation} 
Our proof of fact (B) above implies that on the set 
$n \in \mathcal{C}_E$, which has asymptotic density of one, we have 
\begin{subequations} 
\begin{equation} 
\label{eqn_proof_tag_IneqsHold_va} 
\mathbb{E}\left[\sum_{d|n} C_{\Omega(d)}(d)\right] - |g^{-1}(n)| \geq 
     \sum_{d|n} \mathbb{E}[C_{\Omega(d)}(d)] - |g^{-1}(n)|. 
\end{equation} 
Moreover, by Remark \ref{remark_AsymptoticDensitiesOfExceptionalSets_v1} and 
Lemma \ref{lemma_BddExpectationOfgInvn}, 
for $n$ within a set $\mathcal{S}_E$ of asymptotic density also one, 
\begin{align*} 
\mathbb{E}[C_{\Omega(d)}(d)] - |g^{-1}(n)| & \geq \mathbb{E}\left[ 
     \sum_{d|n} (\log d) (\log\log d)\right] - \mathbb{E}[|g^{-1}(n)|] \\ 
     & \SuccSim \frac{(\log n)^2}{2} (\log\log n) - 2(\log n) \\ 
     & \geq 0, \mathrm{\ as\ } n \rightarrow \infty, 
\end{align*} 
where we have used that 
\[
\int \frac{(\log t) (\log\log t)}{t} dt = \frac{(\log t)^2}{2} (\log\log t) - 
     \frac{(\log t)^2}{4} + C. 
\]
Now we aim to sum the functions $G^{-1}(x)$ and $G_E^{-1}(x)$ weighted by the same signs on the 
terms at each $n$ that respectively satisfy the following condition: 
\begin{equation} 
\label{eqn_proof_tag_IneqsHold_vb} 
\mathbb{E}\left[\sum_{d|n} C_{\Omega(d)}(d)\right] - |g^{-1}(n)| \geq 
     \sum_{d|n} \mathbb{E}[C_{\Omega(d)}(d)] - |g^{-1}(n)| \geq 0, \forall n \in \mathcal{S}_E. 
\end{equation} 
\end{subequations} 
Notice that the intersection of the sets $\mathcal{C}_E \cap \mathcal{S}_E$ also 
has asymptotic density of one. So \eqref{eqn_proof_tag_IneqsHold_vb} holds for all $n$ on a set 
of full asymptotic density one (e.g., almost everywhere on the integers). 

Since the sign of $g^{-1}(n)$ is $\lambda(n)$ as given by 
Proposition \ref{prop_SignageDirInvsOfPosBddArithmeticFuncs_v1}, 
on the set $\mathcal{S}_E$ defined as in \eqref{eqn_proof_tag_IneqsHold_vb}, we have that both 
\begin{align*} 
\sum_{\substack{n \leq x \\ \lambda(n) = +1}} g^{-1}(n) & \leq 
     \sum_{\substack{n \leq x \\ \lambda(n) = +1}} \sum_{d|n} \mathbb{E}[C_{\Omega(d)}(d)] \\ 
\sum_{\substack{n \leq x \\ \lambda(n) = -1}} g^{-1}(n) & \leq 
     -\sum_{\substack{n \leq x \\ \lambda(n) = -1}} \sum_{d|n} \mathbb{E}[C_{\Omega(d)}(d)]. 
\end{align*} 
Hence, on this set we have that 
\begin{equation} 
\label{eqn_proof_tag_IneqsHold_vc} 
G^{-1}(x) \leq \sum_{n \leq x} \lambda(n) \sum_{\substack{d|n \\ d > e^e}} \mathbb{E}[C_{\Omega(d)}(d)], 
     \forall x \in \mathcal{S}_E \cap \mathcal{C}_E. 
\end{equation} 
Now noticing that the right-hand-side of \eqref{eqn_proof_tag_IneqsHold_vc} corresponds to the definition of 
the function $G_E^{-1}(x)$, we observe that if $G^{-1}(x) \geq 0$ on this set, then also 
$G_E^{-1}(x) \geq 0$, and so for $x \in \mathcal{S}_E \cap \mathcal{C}_E$ such that this holds 
(almost everywhere) we have that $|G^{-1}(x)| - |G_E^{-1}(x)| \leq 0$. 

Finally, using Proposition \ref{prop_DensityOfGInvxPosAndBdd}, we can see that there are constants 
$B, C \in (0, 1)$ such that there is an infinite set with 
limiting asymptotic density bounded between these constants such that the condition 
$|G^{-1}(x)| - |G_E^{-1}(x)| \leq 0$ holds: 
\[
B + o(1) \leq \frac{1}{x} \cdot \#\left\{n \leq x: |G^{-1}(n)| - |G_E^{-1}(n)| \leq 0\right\} \leq C + o(1), 
     \mathrm{\ as\ } x \rightarrow \infty. 
\]
Hence, we have shown that the necessary conditions in hypotheses of 
Theorem \ref{theorem_CondAvgOrderGInvxSummatoryFunc_v1} can in fact be achieved for all 
sufficiently large $x \rightarrow \infty$. 
\end{proof} 

\newpage
\section{Establishing the lower bounds for $M(x)$ by cases along infinite subsequences} 
\label{Section_KeyApplications} 

\subsection{The culmination of what we have done so far} 

\begin{prop}
\label{prop_Mx_SBP_IntegralFormula} 
For all sufficiently large $x$, we have that 
\begin{align} 
\label{eqn_pf_tag_v2-restated_v2} 
M(x) & \approx G^{-1}(x) - x \cdot \int_1^{x/2} \frac{G^{-1}(t)}{t^2 \cdot \log(x/t)} dt, 
\end{align} 
where $G^{-1}(x) := \sum_{n \leq x} g^{-1}(n)$ is the summatory function of $g^{-1}(n)$. 
\end{prop} 
\begin{proof} 
We know by applying Corollary \ref{cor_Mx_gInvnPixk_formula} that 
\begin{align} 
\notag
M(x) & = \sum_{k=1}^{x} g^{-1}(k) (\pi(x/k)+1) \\ 
\label{eqn_proof_tag_MxFormulaInitSepTerms_v1} 
     & = G^{-1}(x) + \sum_{k=1}^{x} g^{-1}(k) \pi(x/k), 
\end{align} 
where we can drop the asymptotically unnecessary floored integer-valued arguments to $\pi(x)$ in place of 
its approximation by the monotone non-decreasing $\pi(x) \sim \frac{x}{\log x}$. 
Moreover, we can always 
bound $$\frac{Ax}{\log x} \leq \pi(x) \leq \frac{Bx}{\log x},$$ for suitably defined 
absolute constants, $A,B > 0$. 
Therefore the approximation obtained is valid for all $x > 1$ up to a small constant difference. 

What we now require to sum and simplify the right-hand-side summation from 
\eqref{eqn_proof_tag_MxFormulaInitSepTerms_v1} is an ordinary summation by parts argument. 
Namely, we obtain that for sufficiently large 
$x \geq 2$ \footnote{
     Since $\pi(1) = 0$, the actual range of summation corresponds to 
     $k \in \left[1, \frac{x}{2}\right]$. 
}
\begin{align*} 
\sum_{k=1}^{x} g^{-1}(k) \pi(x/k) & = G^{-1}(x) \pi(1) - \sum_{k=1}^{x-1} G^{-1}(k) \left[ 
     \pi\left(\frac{x}{k}\right) - \pi\left(\frac{x}{k+1}\right)\right] \\ 
     & = -\sum_{k=1}^{x/2} G^{-1}(k) \left[ 
     \pi\left(\frac{x}{k}\right) - \pi\left(\frac{x}{k+1}\right)\right] \\ 
     & \approx -\sum_{k=1}^{x/2} G^{-1}(k) \left[ 
     \frac{x}{k \cdot \log(x/k)} - \frac{x}{(k+1) \cdot \log(x/k)}\right] \\ 
     & \approx -\sum_{k=1}^{x/2} G^{-1}(k) \frac{x}{k^2 \cdot \log(x/k)}. 
\end{align*} 
Since for $x$ large enough the summand is monotonic as $k$ ranges in order over $k \in [1, x/2]$, and 
since the summands in the last equation are smooth functions of $k$ (and $x$), and also since $G^{-1}(x)$ is 
a summatory function with jumps at the positive integers, we can approximate 
$M(x)$ for any finite $x \geq 2$ by 
\[
M(x) \approx G^{-1}(x) - x \cdot \int_1^{x/2} \frac{G^{-1}(t)}{t^2 \cdot \log(x/t)} dt. 
\]
We will later only use unsigned lower bound approximations to this function in the next theorems so that 
the signedness of the summatory function term in the integral formula above 
as $x \rightarrow \infty$ is a moot point entirely. 
\end{proof} 

\begin{proof}[Proof of Proposition \ref{prop_DensityOfGInvxPosAndBdd}] 
Suppose to the contrary that 
\[
\lim_{x \rightarrow \infty} \frac{1}{x} \cdot \#\{n \leq x: n \in \mathcal{G}_{+}\} = 0, 
\]
i.e., that $G^{-1}(x) \leq 0$ almost everywhere for all integers $x$ sufficiently large. 
We will utilize \eqref{eqn_proof_tag_MxFormulaInitSepTerms_v1} 
from Proposition \ref{prop_Mx_SBP_IntegralFormula} to 
derive a contradiction under this assumption. 
In particular, assuming the above limiting density is zero, we have that 
\begin{equation} 
\label{eqn_proof_tag_MxAbsValueConsequence_v1} 
|M(x)| \approx \left\lvert x \cdot \int_1^{x/2} \frac{|G^{-1}(t)|}{t^2 \cdot \log(x/t)} - 
     |G^{-1}(x)| \right\rvert, \mathrm{\ a.e. }, \mathrm{\ as\ } x \rightarrow \infty. 
\end{equation} 
So since we expect by Lemma \ref{lemma_BddExpectationOfgInvn} 
(see Remark \ref{remark_AsymptoticDensitiesOfExceptionalSets_v1}) that 
almost everywhere on the suffficiently large integers $x$, we have 
\[
2x \leq |G^{-1}(x)| \leq x \log x. 
\]
Then from \eqref{eqn_proof_tag_MxAbsValueConsequence_v1}, we then obtain that 
\begin{align*} 
|M(x)| & \geq \left\lvert 2x \cdot \int_{e^e}^{x/2} \frac{dt}{t \cdot \log(x/t)} - x \log x 
     \right\rvert + C_M \\ 
     & = \left\lvert 2 \log\log(2) \cdot x + x \log x \right\rvert + C_M, 
\end{align*} 
for some bounded constant $C_M \in (-\infty, +\infty)$. 
Here, we have computed that 
\[
\int \frac{dt}{t \cdot \log(x/t)} = -\log\log(x/t) + C. 
\]
This inequality clearly violates the known (trivial) bound that $|M(x)| \leq x$ for all $x \geq 1$. 
A similarly phrased argument shows the corresponding result is true for the set $\mathcal{G}_{-}$. 
Thus, combined, these two consequences show that the limiting density of $\mathcal{G}_{+}$ is positive, 
and in particular, that it cannot tend to zero along infinitely many limiting cases 
as $x \rightarrow \infty$. 
\end{proof} 

\subsubsection{From the routine: Proofs of a few cut-and-dry lemmas} 
\label{subsubSection_RoutineProofsNeededForMainBoundOnGInvxFunc} 

\begin{cor} 
\label{cor_ASemiForm_ForGInvx_v1} 
We have that for sufficiently large $x$, as $x \rightarrow \infty$ that 
``on average'' \footnote{ 
     E.g., within a predictably bounded interval around each $x$ sufficiently large. 
     This distinction in the statement is necessary since our limiting lower bounds have 
     so far depended on average order estimates of certain sums and arithmetic functions 
     as $n \rightarrow \infty$. We will rely on the results proved in 
     Section \ref{Section_ProofOfValidityOfAverageOrderLowerBounds} to justify that these 
     lower bounds that hold on average can still be reconciled to prove 
     the key corollary in the next subsection using an infinitely tending subsequence 
     defined pointwise within intervals. 
} 
\begin{align*} 
\left\lvert G_E^{-1}(x) \right\rvert & \SuccSim 
     \left\lvert 
     \widehat{L}_0\left(\log\log x\right) \times \sum_{e^e \leq n \leq \log x} 
     \lambda(n) \cdot \log n \cdot \log\log n \right\rvert, 
\end{align*} 
where the function 
\[
\left\lvert \widehat{L}_0(x) \right\rvert \SuccSim 
     \sqrt{\frac{2}{\pi}} \cdot \frac{A_0}{3e \log 2} \cdot 
     \frac{x}{(\log\log x)^{\frac{5}{2} + \log\log x}}, 
\]
and such that $\operatorname{sgn}(\widehat{L}_0(x)) = (-1)^{\floor{\log\log x}}$ 
(as the function is defined inline below). 
\end{cor} 
\NBRef{A10-2020.04-26} 
\begin{proof} 
Using the definition in, on average we obtain that 
\eqref{eqn_GEInvxSummatoryFuncDef_v1}, we have that \footnote{ 
     For any arithmetic functions $f,h$, we have that \cite[\cf \S 3.10; \S 3.12]{APOSTOLANUMT} 
     \[
     \sum_{n \leq x} h(n) \times \sum_{d|n} f(d) = \sum_{d \leq x} f(d) \times \sum_{n=1}^{\Floor{x}{d}} h(dn). 
     \] 
}
\begin{align*} 
\left\lvert G_E^{-1}(x) \right\rvert & = 
     \left\lvert \sum_{n \leq x} \lambda(n) \sum_{d|n} \mathbb{E}[C_{\Omega(d)}(d)] \right\rvert \\ 
     & = \left\lvert \sum_{d \leq \log\log x} \mathbb{E}[C_{\Omega(d)}(d)] \times 
     \sum_{n=1}^{\Floor{x}{d}} \lambda(dn) \right\rvert \\ 
     & \SuccSim \left\lvert \sum_{e^e \leq d \leq \log\log x} \log d \cdot \log\log d \times 
     \sum_{n=1}^{\Floor{x}{d}} \lambda(dn) \right\rvert. 
\end{align*} 
Now we see that by complete additivity of $\Omega(n)$ 
(multiplicativity of $\lambda(n)$) that 
\begin{align*} 
\sum_{n=1}^{\Floor{x}{d}} \lambda(dn) & = \sum_{n=1}^{\Floor{x}{d}} \lambda(d) \lambda(n) 
     = \lambda(d) \sum_{n \leq \Floor{x}{d}} \lambda(n). 
\end{align*} 
Using the result proved in Section \ref{Section_MVCh7_GzBounds} as 
(see Theorem \ref{theorem_GFs_SymmFuncs_SumsOfRecipOfPowsOfPrimes} and 
Corollary \ref{theorem_MV_Thm7.20})
we can establish that \footnote{ 
     See the proof of Lemma \ref{lemma_CLT_and_AbelSummation} below for 
     a justification of the $\gg$ bound. 
}
\begin{align*} 
\left\lvert \sum_{n \leq x} \lambda(n) \right\rvert & \gg 
     \left\lvert \sum_{k \leq \log\log x} (-1)^k \cdot \widehat{\pi}_k(x) \right\rvert 
     =: \left\lvert \widehat{L}_0(x) \right\rvert. 
\end{align*} 
For large enough $x \rightarrow \infty$ and $e^e \leq d \leq \log\log x$, 
we can easily prove by bounding each function from above and below that 
\[
\log(x/d) \sim \log x, \log\log(x/d) \sim \log\log x.  
\] 
Then we have that 
$$\left\lvert \widehat{L}_0(\log\log x) \right\rvert \sim 
 \left\lvert \widehat{L}_0(\log\log (x/d)) \right\rvert,$$ 
for all large $x \rightarrow \infty$ whenever $e^e \leq d \leq \log\log x$. 

We note that the precise formula for the 
limiting lower bound stated above for $\widehat{L}_0(x)$ is computed by symbolic summation 
in \emph{Mathematica} using the new bounds on $\widehat{\pi}_k(x)$ guaranteed by 
Theorem \ref{theorem_GFs_SymmFuncs_SumsOfRecipOfPowsOfPrimes}. We also have from this formula that 
$\left\lvert \widehat{L}_0(\log x) \right\rvert \ll \left\lvert \widehat{L}_0(\log\log x) \right\rvert$. 
The inner summation in the lower bound stated for $|G_E^{-1}(x)|$ is correctly indexed only for 
$n \leq \log x$ as the definition of this summatory function depends on 
$\mathbb{E}[C_{\Omega(n)}(n)]$ for $n \leq x$ where the functions $C_k(n)$ are only non-zero for large 
$n$ when $k \leq \Omega(n) \ll \log x$ (e.g., the upper bound on $\Omega(n)$ 
is valid up to a constant factor). 
Finally, even though we have signed terms within an absolute value operation, we still 
recover the absolute lower bound on $|G_E^{-1}(x)|$ as stated above 
when we replace $\mathbb{E}[C_{\Omega(d)}(d)]$ by its lower bound $\log d \cdot \log\log d$ proved in 
Section \ref{Section_ProofOfValidityOfAverageOrderLowerBounds}. 
The proof of this fact (omitted here) follows along the same lines of the proof of 
Lemma \ref{lemma_lowerBoundsOnLambdaFuncParitySummFuncs} with few modifications. 
\end{proof} 

\begin{lemma} 
\label{lemma_CLT_and_AbelSummation} 
Suppose that $f_k(n)$ is a sequence of arithmetic functions 
such that $f_k(n) > 0$ for all $n \geq 1$, $f_0(n) = \delta_{n,1}$, and 
$f_{\Omega(n)}(n) \SuccSim \widehat{\tau}_{\ell}(n)$ as $n \rightarrow \infty$. We suppose that 
$\widehat{\tau}_{\ell}(t)$ is a continuously differentiable function of $t$ for all 
large enough $t \gg 1$ \footnote{ 
     We will require that $\widehat{\tau}_{\ell}(t) \in C^{1}(\mathbb{R})$ when we apply the 
     Abel summation formula in the proof of Theorem \ref{theorem_gInv_GeneralAsymptoticsForms}. 
     At this point, it is technically an unnecessary condition that is 
     vacously satisfied by assumption (by requirement) 
     and will importantly need to hold only when we specialize to the 
     actual functions employed to form our new bounds in the theorem below. 
}.  
We define the $\lambda$-sign-scaled summatory function of $f$ as follows: 
\[
F_{\lambda}(x) := \sum_{\substack{n \leq \log x}} 
     \lambda(n) \cdot f_{\Omega(n)}(n). 
\]
Let 
\[
A_{\Omega}^{(\ell)}(t) := \sum_{k=1}^{\floor{\log\log t}} (-1)^k \widehat{\pi}_k^{(\ell)}(t),  
\]
where $\widehat{\pi}_k(x) \geq \widehat{\pi}_k^{(\ell)}(x) \geq 0$ for 
$\widehat{\pi}_k^{(\ell)}(t)$ some a smooth monotone non-decreasing 
function of $t$ whenever $t$ sufficiently large. 
Then we have that on average 
\[
|F_{\lambda}(x)| \SuccSim \left\lvert 
     A_{\Omega}^{(\ell)}(\log x) \widehat{\tau}_{\ell}(\log x) - 
     \int_1^{\log x} 
     A_{\Omega}^{(\ell)}(t) \widehat{\tau}_{\ell}^{\prime}(t) dt 
     \right\rvert.  
\]
\end{lemma}
\begin{proof} 
We can form an accurate $C^{1}(\mathbb{R})$ approximation by the smoothness of 
$\widehat{\pi}_k^{(\ell)}(x)$ that allows us to apply the Abel summation formula using the summatory 
function $A_{\Omega}^{(\ell)}(t)$ for $t$ on any connected subinterval of $[1, \infty)$. 
The second stated formula for $F_{\lambda}(x)$ is valid by Abel summation whenever 
\[
0 \leq \left\lvert \frac{\displaystyle\sum\limits_{\log\log t < k \leq \frac{\log t}{\log 2}} 
     (-1)^k \widehat{\pi}_k(t)}{A_{\Omega}^{(\ell)}(t)}\right\rvert \leq 2, 
     \mathrm{\ as\ } t \rightarrow \infty, 
\]
What the last equation implies is that the asymptotically dominant terms indicating the parity of 
$\lambda(n)$ are captured up to a constant factor 
by the terms in the range over $k$ summed by 
$A_{\Omega}^{(\ell)}(t)$ for 
sufficiently large $t \rightarrow \infty$. 

In other words, taking the sum over the summands that defines $A_{\Omega}(x)$ only over the truncated range of 
$k \in [1, \log\log x]$ does not affect the limiting asymptotically 
dominant terms in the lower bound obtained from using this form of the summatory function in 
conjunction with the Abel summation formula. This property holds even when we should technically 
index over all $k \in [1, \log_2(x)]$ to obtain an exact formula for this function. 
Using the arguments in Montgomery and Vaughan \cite[\S 7; Thm.\ 7.20]{MV} (see 
Corollary \ref{theorem_MV_Thm7.20}), we can see that 
the assertion above holds in the limit as $t \rightarrow \infty$. 
\end{proof} 

The results in Corollary \ref{cor_ASemiForm_ForGInvx_v1} and in 
Lemma \ref{lemma_CLT_and_AbelSummation} combine to provide a key formula used in the 
proof of Theorem \ref{theorem_gInv_GeneralAsymptoticsForms} to bound $G^{-1}(x)$ from 
below in the average order sense. We require one more sanity check to our approximations 
used in that proof explored in the next subsection in the form of the next lemma. 
Observe that we now use the superscript and subscript notation of 
$(\ell)$ not to denote a formal parameter to 
the functions we define below, but instead to denote that these functions form \emph{lower bound} 
approximations to other forms of the functions without the scripted $(\ell)$. 

\begin{lemma} 
\label{lemma_lowerBoundsOnLambdaFuncParitySummFuncs} 
Suppose that $\widehat{\pi}_k(x) \geq \widehat{\pi}_k^{(\ell)}(x) \geq 0$ 
with $\widehat{\pi}_k^{(\ell)}(x)$ a monotone non-decreasing real-valued function 
for all sufficiently large $x$. 
Let 
\begin{align*} 
A_{\Omega}^{(\ell)}(x) & := \sum_{k \leq \log\log x} (-1)^k \widehat{\pi}_k^{(\ell)}(x) \\ 
A_{\Omega}(x) & := \sum_{k \leq \log\log x} (-1)^k \widehat{\pi}_k(x). 
\end{align*} 
Then for all sufficiently large $x$, we have that 
$$|A_{\Omega}(x)| \gg |A_{\Omega}^{(\ell)}(x)|.$$ 
\end{lemma} 
\begin{proof} 
Given an explicit smooth lower bounding function, $\widehat{\pi}_k^{(\ell)}(x)$, we define the 
similarly smooth and monotone residual terms in approximating $\widehat{\pi}_k(x)$ 
using the following notation: 
\[
\widehat{\pi}_k(x) = \widehat{\pi}_k^{(\ell)}(x) + \widehat{E}_k(x). 
\]
Then we can form the ordinary form of the summatory function $A_{\Omega}$ as 
\begin{align*} 
|A_{\Omega}(x)| & = \left\lvert \sum_{k \leq \frac{\log\log x}{2}} 
     \left[\widehat{\pi}_{2k}(x) - \widehat{\pi}_{2k-1}(x)\right] \right\rvert \\ 
     & \geq \left\lvert A_{\Omega}^{(\ell)}(x) - \sum_{k \leq \frac{\log\log x}{2}} \widehat{E}_{2k-1}(x) 
     \right\rvert \\ 
     & \geq 
     \left\lvert A_{\Omega}^{(\ell)}(x) \right\rvert - 
     \left\lvert \sum_{k \leq \frac{\log\log x}{2}} \widehat{E}_{2k-1}(x) 
     \right\rvert. 
\end{align*} 
If the latter sum, $$\operatorname{ES}(x) := \sum_{k \leq \frac{\log\log x}{2}} \widehat{E}_{2k-1}(x) \rightarrow \infty,$$ as 
$x \rightarrow \infty$, then we can always find some absolute (by monotonicity) $C_0 > 0$ such that 
$\operatorname{ES}(x) \leq C_0 \cdot A_{\Omega}(x)$. If on the other hand this sum becomes constant as 
$x \rightarrow +\infty$, then we also clearly have another absolute $C_1 > 0$ such that 
$|A_{\Omega}(x)| \geq C_1 \cdot |A_{\Omega}^{(\ell)}(x)|$. 
In either case, the claimed result holds for all large enough $x$. 
\end{proof} 

\subsubsection{A proof of the key bound from below on $G^{-1}(x)$} 

The following central theorem is the last key barrier required to prove 
Corollary \ref{proofOf_cor_ThePipeDreamResult_v1} 
in the subsection: 

\begin{theorem}[Asymptotics and bounds for the summatory functions $G^{-1}(x)$] 
\label{theorem_gInv_GeneralAsymptoticsForms}
We define a lower summatory function, $G_{\ell}^{-1}(x)$, 
to provide bounds on the magnitude of $G_E^{-1}(x)$ such that 
$$|G_{\ell}^{-1}(x)| \ll |G_E^{-1}(x)|,$$ 
for all sufficiently large $x \geq e^e$ as follows: 
\[
G_{\ell}^{-1}(x) := \sum_{n \leq x} \lambda(n) \times \sum_{\substack{d|n \\ d > e^e}} 
     \log d \cdot \log\log d. 
\]
Then we have the next asymptotic approximations for the lower summatory function where 
$C_{\ell,1}$ is the absolute constant defined by 
\[
C_{\ell,1} = \frac{8 A_0^2}{9 \pi e^2 \log^2(2)}  = 
     \frac{256 \cdot 2^{1/8}}{59049 \cdot \pi^2 e^8 \log^8(2)} 
     \exp\left(-\frac{55}{2} \log^2(2)\right) 
     \approx 5.51187 \times 10^{-12}.  
\]
That is, we obtain ``on average''  
\begin{align*} 
 & \left\lvert G_{\ell}^{-1}\left(x\right) \right\rvert
     \SuccSim 
     C_{\ell,1} \cdot (\log x) (\log\log x)^{5/2} \cdot \frac{(\log\log\log x)^2}{ 
     (\log\log\log\log x)^{5/2}}. 
\end{align*} 
\end{theorem} 
\NBRef{A10-2020.04-26} 
\begin{proof} 
Recall from our proof of Corollary \ref{cor_BoundsOnGz_FromMVBook_initial_stmt_v1} that 
a lower bound on the variant prime form counting function is given by 
\[
\widehat{\pi}_k(x) \SuccSim \frac{A_0 \cdot x}{\log x \cdot (\log\log x)^4 \cdot (k-1)!} \cdot 
     \left(\frac{4}{3e\log 2}\right)^{\frac{k}{\log\log x}}, \mathrm{\ as\ } x \rightarrow \infty. 
\]
So we can then form a lower summatory function indicating the parity of all 
$\Omega(n)$ for $n \leq x$ as 
\begin{align} 
\label{proof_thm_GInvFunc_v0} 
\left\lvert A_{\Omega}^{(\ell)}(t) \right\rvert & = 
     \left\lvert \sum_{k \leq \log\log t} (-1)^k \widehat{\pi}_k(t) \right\rvert \\ 
\notag
     & \SuccSim  
     \sqrt{\frac{2}{\pi}} \cdot \frac{A_0}{3e \log 2} \cdot 
     \frac{t}{(\log\log t)^{\frac{5}{2} + \log\log t}}, 
\end{align} 
where the actual sign on this function is given by 
$\operatorname{sgn}(A_{\Omega}^{(\ell)}(t)) = (-1)^{\floor{\log\log\log\log t}}$ 
(see Lemma \ref{lemma_lowerBoundsOnLambdaFuncParitySummFuncs}). 

Next, by Corollary \ref{theorem_Ckn_GeneralAsymptoticsForms} 
we recover from the main term approximation to $C_{\Omega(n)}(n)$ proved in 
Section \ref{Section_ProofOfValidityOfAverageOrderLowerBounds}, denoted here by 
$\widehat{\tau}_0(t) = \log t \cdot \log\log t$, that 
\begin{align*} 
\widehat{\tau}_0^{\prime}(t) & = \frac{d}{dt}\left[ 
     \log t \cdot \log\log t 
     \right] \SuccSim \frac{\log\log t}{t}. 
\end{align*} 
As in Lemma \ref{lemma_CLT_and_AbelSummation} and Corollary \ref{cor_ASemiForm_ForGInvx_v1}, 
we apply Abel summation to obtain that we have 
\begin{equation} 
\label{proof_thm_GInvFunc_v1} 
G_{\ell}^{-1}(x) = \widehat{L}_0(\log\log x) \left[
     \widehat{\tau}_0(\log x) A_{\Omega}^{(\ell)}(\log x) - 
     \widehat{\tau}_0(u_0) A_{\Omega}^{(\ell)}(u_0) - \int_{u_0}^{\log x} 
     \widehat{\tau}_0^{\prime}(t) A_{\Omega}^{(\ell)}(t) dt\right]. 
\end{equation} 
The inner integral term on the rightmost side of \eqref{proof_thm_GInvFunc_v1} 
is summed approximately in the form of 
\begin{align} 
\label{eqn_proof_thm_GInvFunc_v3_approx} 
\int_{u_0}^{\log x} \widehat{\tau}_0^{\prime}(t) A_{\Omega}^{(\ell)}(t) dt & \sim 
     \sum_{k=u_0+1}^{\frac{1}{2}\log\log\log x} \left( 
     I_{\ell}\left(e^{e^{2k+1}}\right) - 
     I_{\ell}\left(e^{e^{2k}}\right) 
     \right) e^{e^{2k}} \\ 
\notag 
     & \approx 
     C_0(u_0) + 
     (-1)^{\Floor{\log\log\log x}{2}} \times 
     \int_{\frac{\log\log\log x}{2}-1}^{\frac{\log\log\log x}{2}} 
     I_{\ell}\left(e^{e^{2k}}\right) 
     e^{e^{2k}} dk. 
\end{align} 
We define the integrand function, 
$I_{\ell}(t) := \widehat{\tau}_0^{\prime}(t) A_{\Omega}^{(\ell)}(t)$, 
from the previous equations with some limiting simplifications for the 
$k \in \left[\frac{\log\log\log\log x}{2}-1, \frac{\log\log\log\log x}{2}\right]$ as 
\begin{align} 
\label{eqn_proof_thm_GInvFunc_v3_approx} 
I_{\ell}\left(e^{e^{2k}}\right) e^{e^{2k}}& \SuccSim 
     \frac{A_0}{3e \sqrt{\pi} \log 2} \cdot \frac{\exp\left(e^{2k}\right)}{2^{2k} \cdot k^{2k+3/2}}. 
\end{align} 
So using the lower bound on the integrand in \eqref{eqn_proof_thm_GInvFunc_v3_approx}, 
we find that \footnote{ 
     We have invoked the simplifications that for sufficiently large $x$, 
     \[
     \exp\left(-\log\log\log x \cdot \log\log\log\log x\right) \SuccSim 
          \exp\left(-(\log\log\log x)^2\right) \SuccSim 
          (\log\log x)^2,  
     \]
     and 
     \[
     \exp\left(-\log\log\log\log x \cdot \log\log\log\log\log x\right) \SuccSim 
          \exp\left(-(\log\log\log\log x)^2\right) \SuccSim 
          (\log\log\log x)^2.  
     \]
} 
\begin{align} 
\label{eqn_proof_thm_GInvFunc_v4_approx} 
\widehat{L}_0(\log\log x) & \times \int_{\frac{\log\log\log x}{2}-1}^{\frac{\log\log\log x}{2}} 
     I_{\ell}\left(e^{e^{2k}}\right) 
     e^{e^{2k}} dk \\ 
\notag 
     & \SuccSim 
     2 C_{\ell,1} \cdot (\log x) (\log\log x)^{5/2} \cdot \frac{(\log\log\log x)^2}{ 
     (\log\log\log\log x)^{5/2}}. 
\end{align} 
It is clear from our prior computations of the growth of 
$A_{\Omega}^{(\ell)}(x)$ and $\widehat{\tau}_0(x)$ 
that the asymptotically dominant behavior of the lower bound for 
$|G_{\ell}^{-1}(x)|$ corresponds the integral term calculated in the last equation of 
\eqref{eqn_proof_thm_GInvFunc_v4_approx} up to a constant factor. 

To make this observation precise, consider the following expansion for the leading term in 
the Abel summation formula from \eqref{proof_thm_GInvFunc_v1} for comparison with 
\eqref{eqn_proof_thm_GInvFunc_v4_approx}: 
\begin{align*} 
\left\lvert \widehat{L}_0(\log\log x) \widehat{\tau}_0(\log x) A_{\Omega}^{\ell}(\log x) \right\rvert 
     & \SuccSim 
     C_{\ell,1} \cdot (\log x) (\log\log x)^{5/2} \cdot \frac{(\log\log\log x)^2}{ 
     (\log\log\log\log x)^{5/2}}. 
\end{align*} 
We have used the same simplifications noted in the footnote annotated as above in arriving at the 
limiting lower bound given in the previous equation. 
\end{proof} 

A good sign of the correctness of our proof given here is that up to a small rational non-zero 
constant factor distinction, the two terms summed to contribute asymptotic weight to 
\eqref{proof_thm_GInvFunc_v1} coincide, 
even though the main order terms match after computation for subtly different reasons in form. 
Such a coincidence of these terms, one of which forms a product and the other a 
scaled integral, over the same function triplet should be viewed as a rarity of a  
phenomenon we just happen to capture exactly by our computations in the proof above. 

\subsection{Proof of the unboundedness of the scaled Mertens function along infinite subsequences}
\label{subSection_TheCoreResultProof} 

What we will have shown in total concluding the proof of 
Corollary \ref{cor_ThePipeDreamResult_v1} below is the classically conjectured 
unboundedness property of $M(x)$ in the form of 
\[
\limsup_{x \rightarrow \infty} \frac{|M(x)|}{\sqrt{x}} = +\infty. 
\]
This statement comprises a better than previously known rate of the minimal asymptotic tendencies of 
$|M(x)| / \sqrt{x}$ towards unboundedness along an infinite subsequence. 
Note that this result is still a much weaker condition than the RH as stated, and moreover, 
we must emphasize that its construction is much differently 
motivated by the encouraging combinatorial structures and additive functions 
we have observed in our new formulas. 

Now we finally address the conclusion of our argument: 

\begin{proof}[Proof of Corollary \ref{cor_ThePipeDreamResult_v1}] 
\label{proofOf_cor_ThePipeDreamResult_v1} 
It suffices to take $u_0 = e^{2e^{e^{e}}} \approx 5.43591 \times 10^{3313040}$. 
Now, we break up the integral in 
Proposition \ref{prop_Mx_SBP_IntegralFormula} 
over $t \in [u_0, x/2]$ into two pieces: one that is easily bounded 
from $u_0 \leq t \leq \sqrt{x}$, 
and then another that will conveniently give us our slow-growing tendency towards 
infinity along the subsequence when evaluated using 
Theorem \ref{theorem_gInv_GeneralAsymptoticsForms}. 
In the next calculations, we assume that $x \mapsto x_y$ is taken along the 
subsequence defined inexplicitly within the intervals defined above. 
For sufficiently large $y$, this subsequence is guaranteed to exist by our proof of 
Theorem \ref{theorem_CondAvgOrderGInvxSummatoryFunc_v1} 
using the methods we used to establish it and the necessary hypotheses in 
Section \ref{Section_ProofOfValidityOfAverageOrderLowerBounds}. 

First, since $\pi(j) = \pi(\sqrt{x})$ for all $\sqrt{x} \leq j < x$, we can take the first chunk 
of the interval of integration and bound it using \eqref{eqn_pf_tag_v2-restated_v2} as 
\begin{align*} 
-\int_{u_0}^{\sqrt{x}} \frac{2\sqrt{x}}{t^2 \log(x)} |G_{\ell}^{-1}(t)| dt & \SuccSim 
     B_{\ell,2}(u_0) \times \frac{2}{\log(x)} \cdot \left(\min\limits_{u_0 \leq t \leq \sqrt{x}} 
     |G_{\ell}^{-1}(t)| 
     \right) = o\left(1\right), 
\end{align*} 
where $B_{\ell,2}(u_0)$ can be taken as an indefinite, but still absolute positive constant with respect to $u_0$. 
The maximum in the previous equation is clearly attained by taking $t := \sqrt{x}$, where the minimum extreme value 
occurs at the lower bound of integration at $u_0$. 

We next have to prove a related bound over the second portion of the interval from 
$\sqrt{x} \leq t \leq x/2$: 
\begin{align*} 
- & \int_{\sqrt{x}}^{x/2} \frac{2 x}{t^2 \log(x)} \cdot |G_{\ell}^{-1}(t)| dt \SuccSim 
     \frac{2\sqrt{x}}{\log x} \cdot \left(\min_{\sqrt{x} \leq t \leq x/2} |G_{\ell}^{-1}(t)|\right) \\ 
     & = C_{\ell,1} \cdot \sqrt{x} \cdot 
     (\log\log \sqrt{x})^{5/2} \cdot \frac{(\log\log\log \sqrt{x})^2}{ 
     (\log\log\log\log \sqrt{x})^{5/2}} + o(1). 
\end{align*} 
Finally, since $G_{\ell}^{-1}(x) = o(\sqrt{x})$, we obtain in total that as 
$x \rightarrow \infty$ along this infinite subsequence: 
\begin{align*} 
|M(x)| & \SuccSim 
     C_{\ell,1} \cdot \sqrt{x} \cdot 
     (\log\log \sqrt{x})^{5/2} \cdot \frac{(\log\log\log \sqrt{x})^2}{ 
     (\log\log\log\log \sqrt{x})^{5/2}} + o(1).  
\end{align*} 
The absolute constant $C_{\ell,1} > 0$ is defined in the statement of 
Theorem \ref{theorem_gInv_GeneralAsymptoticsForms} proved in the last subsection. 
\end{proof} 

\newpage 
\renewcommand{\refname}{References} 
\bibliography{glossaries-bibtex/thesis-references}{}
\bibliographystyle{plain}

\newpage
\setcounter{section}{0} 
\renewcommand{\thesection}{T.\arabic{section}} 

\newpage
\section{Table: Computations with a signed Dirichlet inverse function and its summatory function} 
\label{table_conjecture_Mertens_ginvSeq_approx_values}

\begin{table}[h!]

\centering

\tiny
\begin{equation*}
\boxed{
\begin{array}{|cc|c|ccc|c|c|ccc|c|ccc}
 n & \mathbf{Primes} & & \mathbf{Sqfree} & \mathbf{PPower} & \bar{\mathbb{S}} & & g^{-1}(n) & 
 \lambda(n) \operatorname{sgn}(g^{-1}(n)) & \lambda(n) g^{-1}(n) - \widehat{f}_1(n) & 
 \frac{\sum\limits_{d|n} C_{\Omega(d)}(d)}{|g^{-1}(n)|} & & G^{-1}(n) & G^{-1}_{+}(n) & G^{-1}_{-}(n) \\ \hline 
 1 & 1^1 & \text{--} & \text{Y} & \text{N} & \text{N} & \text{--} & 1 & 1 & 0 & 1.0000000 & \text{--} & 1 & 1 & 0 \\
 2 & 2^1 & \text{--} & \text{Y} & \text{Y} & \text{N} & \text{--} & -2 & 1 & 0 & 1.0000000 & \text{--} & -1 & 1 & -2 \\
 3 & 3^1 & \text{--} & \text{Y} & \text{Y} & \text{N} & \text{--} & -2 & 1 & 0 & 1.0000000 & \text{--} & -3 & 1 & -4 \\
 4 & 2^2 & \text{--} & \text{N} & \text{Y} & \text{N} & \text{--} & 2 & 1 & 0 & 1.5000000 & \text{--} & -1 & 3 & -4 \\
 5 & 5^1 & \text{--} & \text{Y} & \text{Y} & \text{N} & \text{--} & -2 & 1 & 0 & 1.0000000 & \text{--} & -3 & 3 & -6 \\
 6 & 2^1 3^1 & \text{--} & \text{Y} & \text{N} & \text{N} & \text{--} & 5 & 1 & 0 & 1.0000000 & \text{--} & 2 & 8 & -6 \\
 7 & 7^1 & \text{--} & \text{Y} & \text{Y} & \text{N} & \text{--} & -2 & 1 & 0 & 1.0000000 & \text{--} & 0 & 8 & -8 \\
 8 & 2^3 & \text{--} & \text{N} & \text{Y} & \text{N} & \text{--} & -2 & 1 & 0 & 2.0000000 & \text{--} & -2 & 8 & -10 \\
 9 & 3^2 & \text{--} & \text{N} & \text{Y} & \text{N} & \text{--} & 2 & 1 & 0 & 1.5000000 & \text{--} & 0 & 10 & -10 \\
 10 & 2^1 5^1 & \text{--} & \text{Y} & \text{N} & \text{N} & \text{--} & 5 & 1 & 0 & 1.0000000 & \text{--} & 5 & 15 & -10 \\
 11 & 11^1 & \text{--} & \text{Y} & \text{Y} & \text{N} & \text{--} & -2 & 1 & 0 & 1.0000000 & \text{--} & 3 & 15 & -12 \\
 12 & 2^2 3^1 & \text{--} & \text{N} & \text{N} & \text{Y} & \text{--} & -7 & 1 & 2 & 1.2857143 & \text{--} & -4 & 15 & -19 \\
 13 & 13^1 & \text{--} & \text{Y} & \text{Y} & \text{N} & \text{--} & -2 & 1 & 0 & 1.0000000 & \text{--} & -6 & 15 & -21 \\
 14 & 2^1 7^1 & \text{--} & \text{Y} & \text{N} & \text{N} & \text{--} & 5 & 1 & 0 & 1.0000000 & \text{--} & -1 & 20 & -21 \\
 15 & 3^1 5^1 & \text{--} & \text{Y} & \text{N} & \text{N} & \text{--} & 5 & 1 & 0 & 1.0000000 & \text{--} & 4 & 25 & -21 \\
 16 & 2^4 & \text{--} & \text{N} & \text{Y} & \text{N} & \text{--} & 2 & 1 & 0 & 2.5000000 & \text{--} & 6 & 27 & -21 \\
 17 & 17^1 & \text{--} & \text{Y} & \text{Y} & \text{N} & \text{--} & -2 & 1 & 0 & 1.0000000 & \text{--} & 4 & 27 & -23 \\
 18 & 2^1 3^2 & \text{--} & \text{N} & \text{N} & \text{Y} & \text{--} & -7 & 1 & 2 & 1.2857143 & \text{--} & -3 & 27 & -30 \\
 19 & 19^1 & \text{--} & \text{Y} & \text{Y} & \text{N} & \text{--} & -2 & 1 & 0 & 1.0000000 & \text{--} & -5 & 27 & -32 \\
 20 & 2^2 5^1 & \text{--} & \text{N} & \text{N} & \text{Y} & \text{--} & -7 & 1 & 2 & 1.2857143 & \text{--} & -12 & 27 & -39 \\
 21 & 3^1 7^1 & \text{--} & \text{Y} & \text{N} & \text{N} & \text{--} & 5 & 1 & 0 & 1.0000000 & \text{--} & -7 & 32 & -39 \\
 22 & 2^1 11^1 & \text{--} & \text{Y} & \text{N} & \text{N} & \text{--} & 5 & 1 & 0 & 1.0000000 & \text{--} & -2 & 37 & -39 \\
 23 & 23^1 & \text{--} & \text{Y} & \text{Y} & \text{N} & \text{--} & -2 & 1 & 0 & 1.0000000 & \text{--} & -4 & 37 & -41 \\
 24 & 2^3 3^1 & \text{--} & \text{N} & \text{N} & \text{Y} & \text{--} & 9 & 1 & 4 & 1.5555556 & \text{--} & 5 & 46 & -41 \\
 25 & 5^2 & \text{--} & \text{N} & \text{Y} & \text{N} & \text{--} & 2 & 1 & 0 & 1.5000000 & \text{--} & 7 & 48 & -41 \\
 26 & 2^1 13^1 & \text{--} & \text{Y} & \text{N} & \text{N} & \text{--} & 5 & 1 & 0 & 1.0000000 & \text{--} & 12 & 53 & -41 \\
 27 & 3^3 & \text{--} & \text{N} & \text{Y} & \text{N} & \text{--} & -2 & 1 & 0 & 2.0000000 & \text{--} & 10 & 53 & -43 \\
 28 & 2^2 7^1 & \text{--} & \text{N} & \text{N} & \text{Y} & \text{--} & -7 & 1 & 2 & 1.2857143 & \text{--} & 3 & 53 & -50 \\
 29 & 29^1 & \text{--} & \text{Y} & \text{Y} & \text{N} & \text{--} & -2 & 1 & 0 & 1.0000000 & \text{--} & 1 & 53 & -52 \\
 30 & 2^1 3^1 5^1 & \text{--} & \text{Y} & \text{N} & \text{N} & \text{--} & -16 & 1 & 0 & 1.0000000 & \text{--} & -15 & 53 & -68 \\
 31 & 31^1 & \text{--} & \text{Y} & \text{Y} & \text{N} & \text{--} & -2 & 1 & 0 & 1.0000000 & \text{--} & -17 & 53 & -70 \\
 32 & 2^5 & \text{--} & \text{N} & \text{Y} & \text{N} & \text{--} & -2 & 1 & 0 & 3.0000000 & \text{--} & -19 & 53 & -72 \\
 33 & 3^1 11^1 & \text{--} & \text{Y} & \text{N} & \text{N} & \text{--} & 5 & 1 & 0 & 1.0000000 & \text{--} & -14 & 58 & -72 \\
 34 & 2^1 17^1 & \text{--} & \text{Y} & \text{N} & \text{N} & \text{--} & 5 & 1 & 0 & 1.0000000 & \text{--} & -9 & 63 & -72 \\
 35 & 5^1 7^1 & \text{--} & \text{Y} & \text{N} & \text{N} & \text{--} & 5 & 1 & 0 & 1.0000000 & \text{--} & -4 & 68 & -72 \\
 36 & 2^2 3^2 & \text{--} & \text{N} & \text{N} & \text{Y} & \text{--} & 14 & 1 & 9 & 1.3571429 & \text{--} & 10 & 82 & -72 \\
 37 & 37^1 & \text{--} & \text{Y} & \text{Y} & \text{N} & \text{--} & -2 & 1 & 0 & 1.0000000 & \text{--} & 8 & 82 & -74 \\
 38 & 2^1 19^1 & \text{--} & \text{Y} & \text{N} & \text{N} & \text{--} & 5 & 1 & 0 & 1.0000000 & \text{--} & 13 & 87 & -74 \\
 39 & 3^1 13^1 & \text{--} & \text{Y} & \text{N} & \text{N} & \text{--} & 5 & 1 & 0 & 1.0000000 & \text{--} & 18 & 92 & -74 \\
 40 & 2^3 5^1 & \text{--} & \text{N} & \text{N} & \text{Y} & \text{--} & 9 & 1 & 4 & 1.5555556 & \text{--} & 27 & 101 & -74 \\
 41 & 41^1 & \text{--} & \text{Y} & \text{Y} & \text{N} & \text{--} & -2 & 1 & 0 & 1.0000000 & \text{--} & 25 & 101 & -76 \\
 42 & 2^1 3^1 7^1 & \text{--} & \text{Y} & \text{N} & \text{N} & \text{--} & -16 & 1 & 0 & 1.0000000 & \text{--} & 9 & 101 & -92 \\
 43 & 43^1 & \text{--} & \text{Y} & \text{Y} & \text{N} & \text{--} & -2 & 1 & 0 & 1.0000000 & \text{--} & 7 & 101 & -94 \\
 44 & 2^2 11^1 & \text{--} & \text{N} & \text{N} & \text{Y} & \text{--} & -7 & 1 & 2 & 1.2857143 & \text{--} & 0 & 101 & -101 \\
 45 & 3^2 5^1 & \text{--} & \text{N} & \text{N} & \text{Y} & \text{--} & -7 & 1 & 2 & 1.2857143 & \text{--} & -7 & 101 & -108 \\
 46 & 2^1 23^1 & \text{--} & \text{Y} & \text{N} & \text{N} & \text{--} & 5 & 1 & 0 & 1.0000000 & \text{--} & -2 & 106 & -108 \\
 47 & 47^1 & \text{--} & \text{Y} & \text{Y} & \text{N} & \text{--} & -2 & 1 & 0 & 1.0000000 & \text{--} & -4 & 106 & -110 \\
 48 & 2^4 3^1 & \text{--} & \text{N} & \text{N} & \text{Y} & \text{--} & -11 & 1 & 6 & 1.8181818 & \text{--} & -15 & 106 & -121 \\
\end{array}
}
\end{equation*}

\bigskip\hrule\smallskip 

\caption*{\textbf{\rm \bf Table \thesection:} 
          \textbf{Computations of $\mathbf{g^{-1}(n) \equiv (\omega+1)^{-1}(n)}$ 
          for small $\mathbf{1 \leq n \leq 350}$.} \\ 
          The column labeled \texttt{Primes} provides the prime factorization of each $n$ so that the values of 
          $\omega(n)$ and $\Omega(n)$ are easily extracted. The columns labeled, respectively, \texttt{Sqfree}, \texttt{PPower} and 
          $\bar{\mathbb{S}}$ list inclusion of $n$ in the sets of squarefree integers, prime powers, and the set $\bar{\mathbb{S}}$ 
          that denotes the positive integers $n$ which are neither squarefree nor prime powers. The next two columns provide the 
          explicit values of the inverse function $g^{-1}(n)$ and indicate that the sign of this function at $n$ is given by 
          $\lambda(n)$. \\[0.05cm] 
          The next column shows the small-ish magnitude differences between the unsigned 
          magnitude of $g^{-1}(n)$ and the summations $\widehat{f}_1(n) := \sum_{k \geq 0} \binom{\omega(n)}{k} \cdot k!$. 
          The following column in order shows the ratio of $\sum_{d|n} C_{\Omega(d)}(d) / |g^{-1}(n)|$. \\[0.05cm] 
          The last three 
          columns show the summatory function of $g^{-1}(n)$, $G^{-1}(x) := \sum_{n \leq x} g^{-1}(n)$, decomposed into its 
          respective positive and negative summatory function components: $G^{-1}(x) = G^{-1}_{+}(x) + G^{-1}_{-}(x)$ where 
          $G^{-1}_{+}(x) > 0$ and $G^{-1}_{-}(x) < 0$. 
          } 

\end{table}

\newpage
\begin{table}[h!]

\centering

\tiny
\begin{equation*}
\boxed{
\begin{array}{|cc|c|ccc|c|c|ccc|c|ccc}
 n & \mathbf{Primes} & & \mathbf{Sqfree} & \mathbf{PPower} & \bar{\mathbb{S}} & & g^{-1}(n) & 
 \lambda(n) \operatorname{sgn}(g^{-1}(n)) & \lambda(n) g^{-1}(n) - \widehat{f}_1(n) & 
 \frac{\sum\limits_{d|n} C_{\Omega(d)}(d)}{|g^{-1}(n)|} & & G^{-1}(n) & G^{-1}_{+}(n) & G^{-1}_{-}(n) \\ \hline 
 49 & 7^2 & \text{--} & \text{N} & \text{Y} & \text{N} & \text{--} & 2 & 1 & 0 & 1.5000000 & \text{--} & -13 & 108 & -121 \\
 50 & 2^1 5^2 & \text{--} & \text{N} & \text{N} & \text{Y} & \text{--} & -7 & 1 & 2 & 1.2857143 & \text{--} & -20 & 108 & -128 \\
 51 & 3^1 17^1 & \text{--} & \text{Y} & \text{N} & \text{N} & \text{--} & 5 & 1 & 0 & 1.0000000 & \text{--} & -15 & 113 & -128 \\
 52 & 2^2 13^1 & \text{--} & \text{N} & \text{N} & \text{Y} & \text{--} & -7 & 1 & 2 & 1.2857143 & \text{--} & -22 & 113 & -135 \\
 53 & 53^1 & \text{--} & \text{Y} & \text{Y} & \text{N} & \text{--} & -2 & 1 & 0 & 1.0000000 & \text{--} & -24 & 113 & -137 \\
 54 & 2^1 3^3 & \text{--} & \text{N} & \text{N} & \text{Y} & \text{--} & 9 & 1 & 4 & 1.5555556 & \text{--} & -15 & 122 & -137 \\
 55 & 5^1 11^1 & \text{--} & \text{Y} & \text{N} & \text{N} & \text{--} & 5 & 1 & 0 & 1.0000000 & \text{--} & -10 & 127 & -137 \\
 56 & 2^3 7^1 & \text{--} & \text{N} & \text{N} & \text{Y} & \text{--} & 9 & 1 & 4 & 1.5555556 & \text{--} & -1 & 136 & -137 \\
 57 & 3^1 19^1 & \text{--} & \text{Y} & \text{N} & \text{N} & \text{--} & 5 & 1 & 0 & 1.0000000 & \text{--} & 4 & 141 & -137 \\
 58 & 2^1 29^1 & \text{--} & \text{Y} & \text{N} & \text{N} & \text{--} & 5 & 1 & 0 & 1.0000000 & \text{--} & 9 & 146 & -137 \\
 59 & 59^1 & \text{--} & \text{Y} & \text{Y} & \text{N} & \text{--} & -2 & 1 & 0 & 1.0000000 & \text{--} & 7 & 146 & -139 \\
 60 & 2^2 3^1 5^1 & \text{--} & \text{N} & \text{N} & \text{Y} & \text{--} & 30 & 1 & 14 & 1.1666667 & \text{--} & 37 & 176 & -139 \\
 61 & 61^1 & \text{--} & \text{Y} & \text{Y} & \text{N} & \text{--} & -2 & 1 & 0 & 1.0000000 & \text{--} & 35 & 176 & -141 \\
 62 & 2^1 31^1 & \text{--} & \text{Y} & \text{N} & \text{N} & \text{--} & 5 & 1 & 0 & 1.0000000 & \text{--} & 40 & 181 & -141 \\
 63 & 3^2 7^1 & \text{--} & \text{N} & \text{N} & \text{Y} & \text{--} & -7 & 1 & 2 & 1.2857143 & \text{--} & 33 & 181 & -148 \\
 64 & 2^6 & \text{--} & \text{N} & \text{Y} & \text{N} & \text{--} & 2 & 1 & 0 & 3.5000000 & \text{--} & 35 & 183 & -148 \\
 65 & 5^1 13^1 & \text{--} & \text{Y} & \text{N} & \text{N} & \text{--} & 5 & 1 & 0 & 1.0000000 & \text{--} & 40 & 188 & -148 \\
 66 & 2^1 3^1 11^1 & \text{--} & \text{Y} & \text{N} & \text{N} & \text{--} & -16 & 1 & 0 & 1.0000000 & \text{--} & 24 & 188 & -164 \\
 67 & 67^1 & \text{--} & \text{Y} & \text{Y} & \text{N} & \text{--} & -2 & 1 & 0 & 1.0000000 & \text{--} & 22 & 188 & -166 \\
 68 & 2^2 17^1 & \text{--} & \text{N} & \text{N} & \text{Y} & \text{--} & -7 & 1 & 2 & 1.2857143 & \text{--} & 15 & 188 & -173 \\
 69 & 3^1 23^1 & \text{--} & \text{Y} & \text{N} & \text{N} & \text{--} & 5 & 1 & 0 & 1.0000000 & \text{--} & 20 & 193 & -173 \\
 70 & 2^1 5^1 7^1 & \text{--} & \text{Y} & \text{N} & \text{N} & \text{--} & -16 & 1 & 0 & 1.0000000 & \text{--} & 4 & 193 & -189 \\
 71 & 71^1 & \text{--} & \text{Y} & \text{Y} & \text{N} & \text{--} & -2 & 1 & 0 & 1.0000000 & \text{--} & 2 & 193 & -191 \\
 72 & 2^3 3^2 & \text{--} & \text{N} & \text{N} & \text{Y} & \text{--} & -23 & 1 & 18 & 1.4782609 & \text{--} & -21 & 193 & -214 \\
 73 & 73^1 & \text{--} & \text{Y} & \text{Y} & \text{N} & \text{--} & -2 & 1 & 0 & 1.0000000 & \text{--} & -23 & 193 & -216 \\
 74 & 2^1 37^1 & \text{--} & \text{Y} & \text{N} & \text{N} & \text{--} & 5 & 1 & 0 & 1.0000000 & \text{--} & -18 & 198 & -216 \\
 75 & 3^1 5^2 & \text{--} & \text{N} & \text{N} & \text{Y} & \text{--} & -7 & 1 & 2 & 1.2857143 & \text{--} & -25 & 198 & -223 \\
 76 & 2^2 19^1 & \text{--} & \text{N} & \text{N} & \text{Y} & \text{--} & -7 & 1 & 2 & 1.2857143 & \text{--} & -32 & 198 & -230 \\
 77 & 7^1 11^1 & \text{--} & \text{Y} & \text{N} & \text{N} & \text{--} & 5 & 1 & 0 & 1.0000000 & \text{--} & -27 & 203 & -230 \\
 78 & 2^1 3^1 13^1 & \text{--} & \text{Y} & \text{N} & \text{N} & \text{--} & -16 & 1 & 0 & 1.0000000 & \text{--} & -43 & 203 & -246 \\
 79 & 79^1 & \text{--} & \text{Y} & \text{Y} & \text{N} & \text{--} & -2 & 1 & 0 & 1.0000000 & \text{--} & -45 & 203 & -248 \\
 80 & 2^4 5^1 & \text{--} & \text{N} & \text{N} & \text{Y} & \text{--} & -11 & 1 & 6 & 1.8181818 & \text{--} & -56 & 203 & -259 \\
 81 & 3^4 & \text{--} & \text{N} & \text{Y} & \text{N} & \text{--} & 2 & 1 & 0 & 2.5000000 & \text{--} & -54 & 205 & -259 \\
 82 & 2^1 41^1 & \text{--} & \text{Y} & \text{N} & \text{N} & \text{--} & 5 & 1 & 0 & 1.0000000 & \text{--} & -49 & 210 & -259 \\
 83 & 83^1 & \text{--} & \text{Y} & \text{Y} & \text{N} & \text{--} & -2 & 1 & 0 & 1.0000000 & \text{--} & -51 & 210 & -261 \\
 84 & 2^2 3^1 7^1 & \text{--} & \text{N} & \text{N} & \text{Y} & \text{--} & 30 & 1 & 14 & 1.1666667 & \text{--} & -21 & 240 & -261 \\
 85 & 5^1 17^1 & \text{--} & \text{Y} & \text{N} & \text{N} & \text{--} & 5 & 1 & 0 & 1.0000000 & \text{--} & -16 & 245 & -261 \\
 86 & 2^1 43^1 & \text{--} & \text{Y} & \text{N} & \text{N} & \text{--} & 5 & 1 & 0 & 1.0000000 & \text{--} & -11 & 250 & -261 \\
 87 & 3^1 29^1 & \text{--} & \text{Y} & \text{N} & \text{N} & \text{--} & 5 & 1 & 0 & 1.0000000 & \text{--} & -6 & 255 & -261 \\
 88 & 2^3 11^1 & \text{--} & \text{N} & \text{N} & \text{Y} & \text{--} & 9 & 1 & 4 & 1.5555556 & \text{--} & 3 & 264 & -261 \\
 89 & 89^1 & \text{--} & \text{Y} & \text{Y} & \text{N} & \text{--} & -2 & 1 & 0 & 1.0000000 & \text{--} & 1 & 264 & -263 \\
 90 & 2^1 3^2 5^1 & \text{--} & \text{N} & \text{N} & \text{Y} & \text{--} & 30 & 1 & 14 & 1.1666667 & \text{--} & 31 & 294 & -263 \\
 91 & 7^1 13^1 & \text{--} & \text{Y} & \text{N} & \text{N} & \text{--} & 5 & 1 & 0 & 1.0000000 & \text{--} & 36 & 299 & -263 \\
 92 & 2^2 23^1 & \text{--} & \text{N} & \text{N} & \text{Y} & \text{--} & -7 & 1 & 2 & 1.2857143 & \text{--} & 29 & 299 & -270 \\
 93 & 3^1 31^1 & \text{--} & \text{Y} & \text{N} & \text{N} & \text{--} & 5 & 1 & 0 & 1.0000000 & \text{--} & 34 & 304 & -270 \\
 94 & 2^1 47^1 & \text{--} & \text{Y} & \text{N} & \text{N} & \text{--} & 5 & 1 & 0 & 1.0000000 & \text{--} & 39 & 309 & -270 \\
 95 & 5^1 19^1 & \text{--} & \text{Y} & \text{N} & \text{N} & \text{--} & 5 & 1 & 0 & 1.0000000 & \text{--} & 44 & 314 & -270 \\
 96 & 2^5 3^1 & \text{--} & \text{N} & \text{N} & \text{Y} & \text{--} & 13 & 1 & 8 & 2.0769231 & \text{--} & 57 & 327 & -270 \\
 97 & 97^1 & \text{--} & \text{Y} & \text{Y} & \text{N} & \text{--} & -2 & 1 & 0 & 1.0000000 & \text{--} & 55 & 327 & -272 \\
 98 & 2^1 7^2 & \text{--} & \text{N} & \text{N} & \text{Y} & \text{--} & -7 & 1 & 2 & 1.2857143 & \text{--} & 48 & 327 & -279 \\
 99 & 3^2 11^1 & \text{--} & \text{N} & \text{N} & \text{Y} & \text{--} & -7 & 1 & 2 & 1.2857143 & \text{--} & 41 & 327 & -286 \\
 100 & 2^2 5^2 & \text{--} & \text{N} & \text{N} & \text{Y} & \text{--} & 14 & 1 & 9 & 1.3571429 & \text{--} & 55 & 341 & -286 \\
 101 & 101^1 & \text{--} & \text{Y} & \text{Y} & \text{N} & \text{--} & -2 & 1 & 0 & 1.0000000 & \text{--} & 53 & 341 & -288 \\
 102 & 2^1 3^1 17^1 & \text{--} & \text{Y} & \text{N} & \text{N} & \text{--} & -16 & 1 & 0 & 1.0000000 & \text{--} & 37 & 341 & -304 \\
 103 & 103^1 & \text{--} & \text{Y} & \text{Y} & \text{N} & \text{--} & -2 & 1 & 0 & 1.0000000 & \text{--} & 35 & 341 & -306 \\
 104 & 2^3 13^1 & \text{--} & \text{N} & \text{N} & \text{Y} & \text{--} & 9 & 1 & 4 & 1.5555556 & \text{--} & 44 & 350 & -306 \\
 105 & 3^1 5^1 7^1 & \text{--} & \text{Y} & \text{N} & \text{N} & \text{--} & -16 & 1 & 0 & 1.0000000 & \text{--} & 28 & 350 & -322 \\
 106 & 2^1 53^1 & \text{--} & \text{Y} & \text{N} & \text{N} & \text{--} & 5 & 1 & 0 & 1.0000000 & \text{--} & 33 & 355 & -322 \\
 107 & 107^1 & \text{--} & \text{Y} & \text{Y} & \text{N} & \text{--} & -2 & 1 & 0 & 1.0000000 & \text{--} & 31 & 355 & -324 \\
 108 & 2^2 3^3 & \text{--} & \text{N} & \text{N} & \text{Y} & \text{--} & -23 & 1 & 18 & 1.4782609 & \text{--} & 8 & 355 & -347 \\
 109 & 109^1 & \text{--} & \text{Y} & \text{Y} & \text{N} & \text{--} & -2 & 1 & 0 & 1.0000000 & \text{--} & 6 & 355 & -349 \\
 110 & 2^1 5^1 11^1 & \text{--} & \text{Y} & \text{N} & \text{N} & \text{--} & -16 & 1 & 0 & 1.0000000 & \text{--} & -10 & 355 & -365 \\
 111 & 3^1 37^1 & \text{--} & \text{Y} & \text{N} & \text{N} & \text{--} & 5 & 1 & 0 & 1.0000000 & \text{--} & -5 & 360 & -365 \\
 112 & 2^4 7^1 & \text{--} & \text{N} & \text{N} & \text{Y} & \text{--} & -11 & 1 & 6 & 1.8181818 & \text{--} & -16 & 360 & -376 \\
 113 & 113^1 & \text{--} & \text{Y} & \text{Y} & \text{N} & \text{--} & -2 & 1 & 0 & 1.0000000 & \text{--} & -18 & 360 & -378 \\
 114 & 2^1 3^1 19^1 & \text{--} & \text{Y} & \text{N} & \text{N} & \text{--} & -16 & 1 & 0 & 1.0000000 & \text{--} & -34 & 360 & -394 \\
 115 & 5^1 23^1 & \text{--} & \text{Y} & \text{N} & \text{N} & \text{--} & 5 & 1 & 0 & 1.0000000 & \text{--} & -29 & 365 & -394 \\
 116 & 2^2 29^1 & \text{--} & \text{N} & \text{N} & \text{Y} & \text{--} & -7 & 1 & 2 & 1.2857143 & \text{--} & -36 & 365 & -401 \\
 117 & 3^2 13^1 & \text{--} & \text{N} & \text{N} & \text{Y} & \text{--} & -7 & 1 & 2 & 1.2857143 & \text{--} & -43 & 365 & -408 \\
 118 & 2^1 59^1 & \text{--} & \text{Y} & \text{N} & \text{N} & \text{--} & 5 & 1 & 0 & 1.0000000 & \text{--} & -38 & 370 & -408 \\
 119 & 7^1 17^1 & \text{--} & \text{Y} & \text{N} & \text{N} & \text{--} & 5 & 1 & 0 & 1.0000000 & \text{--} & -33 & 375 & -408 \\
 120 & 2^3 3^1 5^1 & \text{--} & \text{N} & \text{N} & \text{Y} & \text{--} & -48 & 1 & 32 & 1.3333333 & \text{--} & -81 & 375 & -456 \\
 121 & 11^2 & \text{--} & \text{N} & \text{Y} & \text{N} & \text{--} & 2 & 1 & 0 & 1.5000000 & \text{--} & -79 & 377 & -456 \\
 122 & 2^1 61^1 & \text{--} & \text{Y} & \text{N} & \text{N} & \text{--} & 5 & 1 & 0 & 1.0000000 & \text{--} & -74 & 382 & -456 \\
 123 & 3^1 41^1 & \text{--} & \text{Y} & \text{N} & \text{N} & \text{--} & 5 & 1 & 0 & 1.0000000 & \text{--} & -69 & 387 & -456 \\
 124 & 2^2 31^1 & \text{--} & \text{N} & \text{N} & \text{Y} & \text{--} & -7 & 1 & 2 & 1.2857143 & \text{--} & -76 & 387 & -463 \\
\end{array}
}
\end{equation*}

\end{table} 


\newpage
\begin{table}[h!]

\centering

\tiny
\begin{equation*}
\boxed{
\begin{array}{|cc|c|ccc|c|c|ccc|c|ccc}
 n & \mathbf{Primes} & & \mathbf{Sqfree} & \mathbf{PPower} & \bar{\mathbb{S}} & & g^{-1}(n) & 
 \lambda(n) \operatorname{sgn}(g^{-1}(n)) & \lambda(n) g^{-1}(n) - \widehat{f}_1(n) & 
 \frac{\sum\limits_{d|n} C_{\Omega(d)}(d)}{|g^{-1}(n)|} & & G^{-1}(n) & G^{-1}_{+}(n) & G^{-1}_{-}(n) \\ \hline 
 125 & 5^3 & \text{--} & \text{N} & \text{Y} & \text{N} & \text{--} & -2 & 1 & 0 & 2.0000000 & \text{--} & -78 & 387 & -465 \\
 126 & 2^1 3^2 7^1 & \text{--} & \text{N} & \text{N} & \text{Y} & \text{--} & 30 & 1 & 14 & 1.1666667 & \text{--} & -48 & 417 & -465 \\
 127 & 127^1 & \text{--} & \text{Y} & \text{Y} & \text{N} & \text{--} & -2 & 1 & 0 & 1.0000000 & \text{--} & -50 & 417 & -467 \\
 128 & 2^7 & \text{--} & \text{N} & \text{Y} & \text{N} & \text{--} & -2 & 1 & 0 & 4.0000000 & \text{--} & -52 & 417 & -469 \\
 129 & 3^1 43^1 & \text{--} & \text{Y} & \text{N} & \text{N} & \text{--} & 5 & 1 & 0 & 1.0000000 & \text{--} & -47 & 422 & -469 \\
 130 & 2^1 5^1 13^1 & \text{--} & \text{Y} & \text{N} & \text{N} & \text{--} & -16 & 1 & 0 & 1.0000000 & \text{--} & -63 & 422 & -485 \\
 131 & 131^1 & \text{--} & \text{Y} & \text{Y} & \text{N} & \text{--} & -2 & 1 & 0 & 1.0000000 & \text{--} & -65 & 422 & -487 \\
 132 & 2^2 3^1 11^1 & \text{--} & \text{N} & \text{N} & \text{Y} & \text{--} & 30 & 1 & 14 & 1.1666667 & \text{--} & -35 & 452 & -487 \\
 133 & 7^1 19^1 & \text{--} & \text{Y} & \text{N} & \text{N} & \text{--} & 5 & 1 & 0 & 1.0000000 & \text{--} & -30 & 457 & -487 \\
 134 & 2^1 67^1 & \text{--} & \text{Y} & \text{N} & \text{N} & \text{--} & 5 & 1 & 0 & 1.0000000 & \text{--} & -25 & 462 & -487 \\
 135 & 3^3 5^1 & \text{--} & \text{N} & \text{N} & \text{Y} & \text{--} & 9 & 1 & 4 & 1.5555556 & \text{--} & -16 & 471 & -487 \\
 136 & 2^3 17^1 & \text{--} & \text{N} & \text{N} & \text{Y} & \text{--} & 9 & 1 & 4 & 1.5555556 & \text{--} & -7 & 480 & -487 \\
 137 & 137^1 & \text{--} & \text{Y} & \text{Y} & \text{N} & \text{--} & -2 & 1 & 0 & 1.0000000 & \text{--} & -9 & 480 & -489 \\
 138 & 2^1 3^1 23^1 & \text{--} & \text{Y} & \text{N} & \text{N} & \text{--} & -16 & 1 & 0 & 1.0000000 & \text{--} & -25 & 480 & -505 \\
 139 & 139^1 & \text{--} & \text{Y} & \text{Y} & \text{N} & \text{--} & -2 & 1 & 0 & 1.0000000 & \text{--} & -27 & 480 & -507 \\
 140 & 2^2 5^1 7^1 & \text{--} & \text{N} & \text{N} & \text{Y} & \text{--} & 30 & 1 & 14 & 1.1666667 & \text{--} & 3 & 510 & -507 \\
 141 & 3^1 47^1 & \text{--} & \text{Y} & \text{N} & \text{N} & \text{--} & 5 & 1 & 0 & 1.0000000 & \text{--} & 8 & 515 & -507 \\
 142 & 2^1 71^1 & \text{--} & \text{Y} & \text{N} & \text{N} & \text{--} & 5 & 1 & 0 & 1.0000000 & \text{--} & 13 & 520 & -507 \\
 143 & 11^1 13^1 & \text{--} & \text{Y} & \text{N} & \text{N} & \text{--} & 5 & 1 & 0 & 1.0000000 & \text{--} & 18 & 525 & -507 \\
 144 & 2^4 3^2 & \text{--} & \text{N} & \text{N} & \text{Y} & \text{--} & 34 & 1 & 29 & 1.6176471 & \text{--} & 52 & 559 & -507 \\
 145 & 5^1 29^1 & \text{--} & \text{Y} & \text{N} & \text{N} & \text{--} & 5 & 1 & 0 & 1.0000000 & \text{--} & 57 & 564 & -507 \\
 146 & 2^1 73^1 & \text{--} & \text{Y} & \text{N} & \text{N} & \text{--} & 5 & 1 & 0 & 1.0000000 & \text{--} & 62 & 569 & -507 \\
 147 & 3^1 7^2 & \text{--} & \text{N} & \text{N} & \text{Y} & \text{--} & -7 & 1 & 2 & 1.2857143 & \text{--} & 55 & 569 & -514 \\
 148 & 2^2 37^1 & \text{--} & \text{N} & \text{N} & \text{Y} & \text{--} & -7 & 1 & 2 & 1.2857143 & \text{--} & 48 & 569 & -521 \\
 149 & 149^1 & \text{--} & \text{Y} & \text{Y} & \text{N} & \text{--} & -2 & 1 & 0 & 1.0000000 & \text{--} & 46 & 569 & -523 \\
 150 & 2^1 3^1 5^2 & \text{--} & \text{N} & \text{N} & \text{Y} & \text{--} & 30 & 1 & 14 & 1.1666667 & \text{--} & 76 & 599 & -523 \\
 151 & 151^1 & \text{--} & \text{Y} & \text{Y} & \text{N} & \text{--} & -2 & 1 & 0 & 1.0000000 & \text{--} & 74 & 599 & -525 \\
 152 & 2^3 19^1 & \text{--} & \text{N} & \text{N} & \text{Y} & \text{--} & 9 & 1 & 4 & 1.5555556 & \text{--} & 83 & 608 & -525 \\
 153 & 3^2 17^1 & \text{--} & \text{N} & \text{N} & \text{Y} & \text{--} & -7 & 1 & 2 & 1.2857143 & \text{--} & 76 & 608 & -532 \\
 154 & 2^1 7^1 11^1 & \text{--} & \text{Y} & \text{N} & \text{N} & \text{--} & -16 & 1 & 0 & 1.0000000 & \text{--} & 60 & 608 & -548 \\
 155 & 5^1 31^1 & \text{--} & \text{Y} & \text{N} & \text{N} & \text{--} & 5 & 1 & 0 & 1.0000000 & \text{--} & 65 & 613 & -548 \\
 156 & 2^2 3^1 13^1 & \text{--} & \text{N} & \text{N} & \text{Y} & \text{--} & 30 & 1 & 14 & 1.1666667 & \text{--} & 95 & 643 & -548 \\
 157 & 157^1 & \text{--} & \text{Y} & \text{Y} & \text{N} & \text{--} & -2 & 1 & 0 & 1.0000000 & \text{--} & 93 & 643 & -550 \\
 158 & 2^1 79^1 & \text{--} & \text{Y} & \text{N} & \text{N} & \text{--} & 5 & 1 & 0 & 1.0000000 & \text{--} & 98 & 648 & -550 \\
 159 & 3^1 53^1 & \text{--} & \text{Y} & \text{N} & \text{N} & \text{--} & 5 & 1 & 0 & 1.0000000 & \text{--} & 103 & 653 & -550 \\
 160 & 2^5 5^1 & \text{--} & \text{N} & \text{N} & \text{Y} & \text{--} & 13 & 1 & 8 & 2.0769231 & \text{--} & 116 & 666 & -550 \\
 161 & 7^1 23^1 & \text{--} & \text{Y} & \text{N} & \text{N} & \text{--} & 5 & 1 & 0 & 1.0000000 & \text{--} & 121 & 671 & -550 \\
 162 & 2^1 3^4 & \text{--} & \text{N} & \text{N} & \text{Y} & \text{--} & -11 & 1 & 6 & 1.8181818 & \text{--} & 110 & 671 & -561 \\
 163 & 163^1 & \text{--} & \text{Y} & \text{Y} & \text{N} & \text{--} & -2 & 1 & 0 & 1.0000000 & \text{--} & 108 & 671 & -563 \\
 164 & 2^2 41^1 & \text{--} & \text{N} & \text{N} & \text{Y} & \text{--} & -7 & 1 & 2 & 1.2857143 & \text{--} & 101 & 671 & -570 \\
 165 & 3^1 5^1 11^1 & \text{--} & \text{Y} & \text{N} & \text{N} & \text{--} & -16 & 1 & 0 & 1.0000000 & \text{--} & 85 & 671 & -586 \\
 166 & 2^1 83^1 & \text{--} & \text{Y} & \text{N} & \text{N} & \text{--} & 5 & 1 & 0 & 1.0000000 & \text{--} & 90 & 676 & -586 \\
 167 & 167^1 & \text{--} & \text{Y} & \text{Y} & \text{N} & \text{--} & -2 & 1 & 0 & 1.0000000 & \text{--} & 88 & 676 & -588 \\
 168 & 2^3 3^1 7^1 & \text{--} & \text{N} & \text{N} & \text{Y} & \text{--} & -48 & 1 & 32 & 1.3333333 & \text{--} & 40 & 676 & -636 \\
 169 & 13^2 & \text{--} & \text{N} & \text{Y} & \text{N} & \text{--} & 2 & 1 & 0 & 1.5000000 & \text{--} & 42 & 678 & -636 \\
 170 & 2^1 5^1 17^1 & \text{--} & \text{Y} & \text{N} & \text{N} & \text{--} & -16 & 1 & 0 & 1.0000000 & \text{--} & 26 & 678 & -652 \\
 171 & 3^2 19^1 & \text{--} & \text{N} & \text{N} & \text{Y} & \text{--} & -7 & 1 & 2 & 1.2857143 & \text{--} & 19 & 678 & -659 \\
 172 & 2^2 43^1 & \text{--} & \text{N} & \text{N} & \text{Y} & \text{--} & -7 & 1 & 2 & 1.2857143 & \text{--} & 12 & 678 & -666 \\
 173 & 173^1 & \text{--} & \text{Y} & \text{Y} & \text{N} & \text{--} & -2 & 1 & 0 & 1.0000000 & \text{--} & 10 & 678 & -668 \\
 174 & 2^1 3^1 29^1 & \text{--} & \text{Y} & \text{N} & \text{N} & \text{--} & -16 & 1 & 0 & 1.0000000 & \text{--} & -6 & 678 & -684 \\
 175 & 5^2 7^1 & \text{--} & \text{N} & \text{N} & \text{Y} & \text{--} & -7 & 1 & 2 & 1.2857143 & \text{--} & -13 & 678 & -691 \\
 176 & 2^4 11^1 & \text{--} & \text{N} & \text{N} & \text{Y} & \text{--} & -11 & 1 & 6 & 1.8181818 & \text{--} & -24 & 678 & -702 \\
 177 & 3^1 59^1 & \text{--} & \text{Y} & \text{N} & \text{N} & \text{--} & 5 & 1 & 0 & 1.0000000 & \text{--} & -19 & 683 & -702 \\
 178 & 2^1 89^1 & \text{--} & \text{Y} & \text{N} & \text{N} & \text{--} & 5 & 1 & 0 & 1.0000000 & \text{--} & -14 & 688 & -702 \\
 179 & 179^1 & \text{--} & \text{Y} & \text{Y} & \text{N} & \text{--} & -2 & 1 & 0 & 1.0000000 & \text{--} & -16 & 688 & -704 \\
 180 & 2^2 3^2 5^1 & \text{--} & \text{N} & \text{N} & \text{Y} & \text{--} & -74 & 1 & 58 & 1.2162162 & \text{--} & -90 & 688 & -778 \\
 181 & 181^1 & \text{--} & \text{Y} & \text{Y} & \text{N} & \text{--} & -2 & 1 & 0 & 1.0000000 & \text{--} & -92 & 688 & -780 \\
 182 & 2^1 7^1 13^1 & \text{--} & \text{Y} & \text{N} & \text{N} & \text{--} & -16 & 1 & 0 & 1.0000000 & \text{--} & -108 & 688 & -796 \\
 183 & 3^1 61^1 & \text{--} & \text{Y} & \text{N} & \text{N} & \text{--} & 5 & 1 & 0 & 1.0000000 & \text{--} & -103 & 693 & -796 \\
 184 & 2^3 23^1 & \text{--} & \text{N} & \text{N} & \text{Y} & \text{--} & 9 & 1 & 4 & 1.5555556 & \text{--} & -94 & 702 & -796 \\
 185 & 5^1 37^1 & \text{--} & \text{Y} & \text{N} & \text{N} & \text{--} & 5 & 1 & 0 & 1.0000000 & \text{--} & -89 & 707 & -796 \\
 186 & 2^1 3^1 31^1 & \text{--} & \text{Y} & \text{N} & \text{N} & \text{--} & -16 & 1 & 0 & 1.0000000 & \text{--} & -105 & 707 & -812 \\
 187 & 11^1 17^1 & \text{--} & \text{Y} & \text{N} & \text{N} & \text{--} & 5 & 1 & 0 & 1.0000000 & \text{--} & -100 & 712 & -812 \\
 188 & 2^2 47^1 & \text{--} & \text{N} & \text{N} & \text{Y} & \text{--} & -7 & 1 & 2 & 1.2857143 & \text{--} & -107 & 712 & -819 \\
 189 & 3^3 7^1 & \text{--} & \text{N} & \text{N} & \text{Y} & \text{--} & 9 & 1 & 4 & 1.5555556 & \text{--} & -98 & 721 & -819 \\
 190 & 2^1 5^1 19^1 & \text{--} & \text{Y} & \text{N} & \text{N} & \text{--} & -16 & 1 & 0 & 1.0000000 & \text{--} & -114 & 721 & -835 \\
 191 & 191^1 & \text{--} & \text{Y} & \text{Y} & \text{N} & \text{--} & -2 & 1 & 0 & 1.0000000 & \text{--} & -116 & 721 & -837 \\
 192 & 2^6 3^1 & \text{--} & \text{N} & \text{N} & \text{Y} & \text{--} & -15 & 1 & 10 & 2.3333333 & \text{--} & -131 & 721 & -852 \\
 193 & 193^1 & \text{--} & \text{Y} & \text{Y} & \text{N} & \text{--} & -2 & 1 & 0 & 1.0000000 & \text{--} & -133 & 721 & -854 \\
 194 & 2^1 97^1 & \text{--} & \text{Y} & \text{N} & \text{N} & \text{--} & 5 & 1 & 0 & 1.0000000 & \text{--} & -128 & 726 & -854 \\
 195 & 3^1 5^1 13^1 & \text{--} & \text{Y} & \text{N} & \text{N} & \text{--} & -16 & 1 & 0 & 1.0000000 & \text{--} & -144 & 726 & -870 \\
 196 & 2^2 7^2 & \text{--} & \text{N} & \text{N} & \text{Y} & \text{--} & 14 & 1 & 9 & 1.3571429 & \text{--} & -130 & 740 & -870 \\
 197 & 197^1 & \text{--} & \text{Y} & \text{Y} & \text{N} & \text{--} & -2 & 1 & 0 & 1.0000000 & \text{--} & -132 & 740 & -872 \\
 198 & 2^1 3^2 11^1 & \text{--} & \text{N} & \text{N} & \text{Y} & \text{--} & 30 & 1 & 14 & 1.1666667 & \text{--} & -102 & 770 & -872 \\
 199 & 199^1 & \text{--} & \text{Y} & \text{Y} & \text{N} & \text{--} & -2 & 1 & 0 & 1.0000000 & \text{--} & -104 & 770 & -874 \\
 200 & 2^3 5^2 & \text{--} & \text{N} & \text{N} & \text{Y} & \text{--} & -23 & 1 & 18 & 1.4782609 & \text{--} & -127 & 770 & -897 \\
\end{array}
}
\end{equation*}

\end{table} 


\newpage
\begin{table}[h!]

\centering

\tiny
\begin{equation*}
\boxed{
\begin{array}{|cc|c|ccc|c|c|ccc|c|ccc}
 n & \mathbf{Primes} & & \mathbf{Sqfree} & \mathbf{PPower} & \bar{\mathbb{S}} & & g^{-1}(n) & 
 \lambda(n) \operatorname{sgn}(g^{-1}(n)) & \lambda(n) g^{-1}(n) - \widehat{f}_1(n) & 
 \frac{\sum\limits_{d|n} C_{\Omega(d)}(d)}{|g^{-1}(n)|} & & G^{-1}(n) & G^{-1}_{+}(n) & G^{-1}_{-}(n) \\ \hline 
 201 & 3^1 67^1 & \text{--} & \text{Y} & \text{N} & \text{N} & \text{--} & 5 & 1 & 0 & 1.0000000 & \text{--} & -122 & 775 & -897 \\
 202 & 2^1 101^1 & \text{--} & \text{Y} & \text{N} & \text{N} & \text{--} & 5 & 1 & 0 & 1.0000000 & \text{--} & -117 & 780 & -897 \\
 203 & 7^1 29^1 & \text{--} & \text{Y} & \text{N} & \text{N} & \text{--} & 5 & 1 & 0 & 1.0000000 & \text{--} & -112 & 785 & -897 \\
 204 & 2^2 3^1 17^1 & \text{--} & \text{N} & \text{N} & \text{Y} & \text{--} & 30 & 1 & 14 & 1.1666667 & \text{--} & -82 & 815 & -897 \\
 205 & 5^1 41^1 & \text{--} & \text{Y} & \text{N} & \text{N} & \text{--} & 5 & 1 & 0 & 1.0000000 & \text{--} & -77 & 820 & -897 \\
 206 & 2^1 103^1 & \text{--} & \text{Y} & \text{N} & \text{N} & \text{--} & 5 & 1 & 0 & 1.0000000 & \text{--} & -72 & 825 & -897 \\
 207 & 3^2 23^1 & \text{--} & \text{N} & \text{N} & \text{Y} & \text{--} & -7 & 1 & 2 & 1.2857143 & \text{--} & -79 & 825 & -904 \\
 208 & 2^4 13^1 & \text{--} & \text{N} & \text{N} & \text{Y} & \text{--} & -11 & 1 & 6 & 1.8181818 & \text{--} & -90 & 825 & -915 \\
 209 & 11^1 19^1 & \text{--} & \text{Y} & \text{N} & \text{N} & \text{--} & 5 & 1 & 0 & 1.0000000 & \text{--} & -85 & 830 & -915 \\
 210 & 2^1 3^1 5^1 7^1 & \text{--} & \text{Y} & \text{N} & \text{N} & \text{--} & 65 & 1 & 0 & 1.0000000 & \text{--} & -20 & 895 & -915 \\
 211 & 211^1 & \text{--} & \text{Y} & \text{Y} & \text{N} & \text{--} & -2 & 1 & 0 & 1.0000000 & \text{--} & -22 & 895 & -917 \\
 212 & 2^2 53^1 & \text{--} & \text{N} & \text{N} & \text{Y} & \text{--} & -7 & 1 & 2 & 1.2857143 & \text{--} & -29 & 895 & -924 \\
 213 & 3^1 71^1 & \text{--} & \text{Y} & \text{N} & \text{N} & \text{--} & 5 & 1 & 0 & 1.0000000 & \text{--} & -24 & 900 & -924 \\
 214 & 2^1 107^1 & \text{--} & \text{Y} & \text{N} & \text{N} & \text{--} & 5 & 1 & 0 & 1.0000000 & \text{--} & -19 & 905 & -924 \\
 215 & 5^1 43^1 & \text{--} & \text{Y} & \text{N} & \text{N} & \text{--} & 5 & 1 & 0 & 1.0000000 & \text{--} & -14 & 910 & -924 \\
 216 & 2^3 3^3 & \text{--} & \text{N} & \text{N} & \text{Y} & \text{--} & 46 & 1 & 41 & 1.5000000 & \text{--} & 32 & 956 & -924 \\
 217 & 7^1 31^1 & \text{--} & \text{Y} & \text{N} & \text{N} & \text{--} & 5 & 1 & 0 & 1.0000000 & \text{--} & 37 & 961 & -924 \\
 218 & 2^1 109^1 & \text{--} & \text{Y} & \text{N} & \text{N} & \text{--} & 5 & 1 & 0 & 1.0000000 & \text{--} & 42 & 966 & -924 \\
 219 & 3^1 73^1 & \text{--} & \text{Y} & \text{N} & \text{N} & \text{--} & 5 & 1 & 0 & 1.0000000 & \text{--} & 47 & 971 & -924 \\
 220 & 2^2 5^1 11^1 & \text{--} & \text{N} & \text{N} & \text{Y} & \text{--} & 30 & 1 & 14 & 1.1666667 & \text{--} & 77 & 1001 & -924 \\
 221 & 13^1 17^1 & \text{--} & \text{Y} & \text{N} & \text{N} & \text{--} & 5 & 1 & 0 & 1.0000000 & \text{--} & 82 & 1006 & -924 \\
 222 & 2^1 3^1 37^1 & \text{--} & \text{Y} & \text{N} & \text{N} & \text{--} & -16 & 1 & 0 & 1.0000000 & \text{--} & 66 & 1006 & -940 \\
 223 & 223^1 & \text{--} & \text{Y} & \text{Y} & \text{N} & \text{--} & -2 & 1 & 0 & 1.0000000 & \text{--} & 64 & 1006 & -942 \\
 224 & 2^5 7^1 & \text{--} & \text{N} & \text{N} & \text{Y} & \text{--} & 13 & 1 & 8 & 2.0769231 & \text{--} & 77 & 1019 & -942 \\
 225 & 3^2 5^2 & \text{--} & \text{N} & \text{N} & \text{Y} & \text{--} & 14 & 1 & 9 & 1.3571429 & \text{--} & 91 & 1033 & -942 \\
 226 & 2^1 113^1 & \text{--} & \text{Y} & \text{N} & \text{N} & \text{--} & 5 & 1 & 0 & 1.0000000 & \text{--} & 96 & 1038 & -942 \\
 227 & 227^1 & \text{--} & \text{Y} & \text{Y} & \text{N} & \text{--} & -2 & 1 & 0 & 1.0000000 & \text{--} & 94 & 1038 & -944 \\
 228 & 2^2 3^1 19^1 & \text{--} & \text{N} & \text{N} & \text{Y} & \text{--} & 30 & 1 & 14 & 1.1666667 & \text{--} & 124 & 1068 & -944 \\
 229 & 229^1 & \text{--} & \text{Y} & \text{Y} & \text{N} & \text{--} & -2 & 1 & 0 & 1.0000000 & \text{--} & 122 & 1068 & -946 \\
 230 & 2^1 5^1 23^1 & \text{--} & \text{Y} & \text{N} & \text{N} & \text{--} & -16 & 1 & 0 & 1.0000000 & \text{--} & 106 & 1068 & -962 \\
 231 & 3^1 7^1 11^1 & \text{--} & \text{Y} & \text{N} & \text{N} & \text{--} & -16 & 1 & 0 & 1.0000000 & \text{--} & 90 & 1068 & -978 \\
 232 & 2^3 29^1 & \text{--} & \text{N} & \text{N} & \text{Y} & \text{--} & 9 & 1 & 4 & 1.5555556 & \text{--} & 99 & 1077 & -978 \\
 233 & 233^1 & \text{--} & \text{Y} & \text{Y} & \text{N} & \text{--} & -2 & 1 & 0 & 1.0000000 & \text{--} & 97 & 1077 & -980 \\
 234 & 2^1 3^2 13^1 & \text{--} & \text{N} & \text{N} & \text{Y} & \text{--} & 30 & 1 & 14 & 1.1666667 & \text{--} & 127 & 1107 & -980 \\
 235 & 5^1 47^1 & \text{--} & \text{Y} & \text{N} & \text{N} & \text{--} & 5 & 1 & 0 & 1.0000000 & \text{--} & 132 & 1112 & -980 \\
 236 & 2^2 59^1 & \text{--} & \text{N} & \text{N} & \text{Y} & \text{--} & -7 & 1 & 2 & 1.2857143 & \text{--} & 125 & 1112 & -987 \\
 237 & 3^1 79^1 & \text{--} & \text{Y} & \text{N} & \text{N} & \text{--} & 5 & 1 & 0 & 1.0000000 & \text{--} & 130 & 1117 & -987 \\
 238 & 2^1 7^1 17^1 & \text{--} & \text{Y} & \text{N} & \text{N} & \text{--} & -16 & 1 & 0 & 1.0000000 & \text{--} & 114 & 1117 & -1003 \\
 239 & 239^1 & \text{--} & \text{Y} & \text{Y} & \text{N} & \text{--} & -2 & 1 & 0 & 1.0000000 & \text{--} & 112 & 1117 & -1005 \\
 240 & 2^4 3^1 5^1 & \text{--} & \text{N} & \text{N} & \text{Y} & \text{--} & 70 & 1 & 54 & 1.5000000 & \text{--} & 182 & 1187 & -1005 \\
 241 & 241^1 & \text{--} & \text{Y} & \text{Y} & \text{N} & \text{--} & -2 & 1 & 0 & 1.0000000 & \text{--} & 180 & 1187 & -1007 \\
 242 & 2^1 11^2 & \text{--} & \text{N} & \text{N} & \text{Y} & \text{--} & -7 & 1 & 2 & 1.2857143 & \text{--} & 173 & 1187 & -1014 \\
 243 & 3^5 & \text{--} & \text{N} & \text{Y} & \text{N} & \text{--} & -2 & 1 & 0 & 3.0000000 & \text{--} & 171 & 1187 & -1016 \\
 244 & 2^2 61^1 & \text{--} & \text{N} & \text{N} & \text{Y} & \text{--} & -7 & 1 & 2 & 1.2857143 & \text{--} & 164 & 1187 & -1023 \\
 245 & 5^1 7^2 & \text{--} & \text{N} & \text{N} & \text{Y} & \text{--} & -7 & 1 & 2 & 1.2857143 & \text{--} & 157 & 1187 & -1030 \\
 246 & 2^1 3^1 41^1 & \text{--} & \text{Y} & \text{N} & \text{N} & \text{--} & -16 & 1 & 0 & 1.0000000 & \text{--} & 141 & 1187 & -1046 \\
 247 & 13^1 19^1 & \text{--} & \text{Y} & \text{N} & \text{N} & \text{--} & 5 & 1 & 0 & 1.0000000 & \text{--} & 146 & 1192 & -1046 \\
 248 & 2^3 31^1 & \text{--} & \text{N} & \text{N} & \text{Y} & \text{--} & 9 & 1 & 4 & 1.5555556 & \text{--} & 155 & 1201 & -1046 \\
 249 & 3^1 83^1 & \text{--} & \text{Y} & \text{N} & \text{N} & \text{--} & 5 & 1 & 0 & 1.0000000 & \text{--} & 160 & 1206 & -1046 \\
 250 & 2^1 5^3 & \text{--} & \text{N} & \text{N} & \text{Y} & \text{--} & 9 & 1 & 4 & 1.5555556 & \text{--} & 169 & 1215 & -1046 \\
 251 & 251^1 & \text{--} & \text{Y} & \text{Y} & \text{N} & \text{--} & -2 & 1 & 0 & 1.0000000 & \text{--} & 167 & 1215 & -1048 \\
 252 & 2^2 3^2 7^1 & \text{--} & \text{N} & \text{N} & \text{Y} & \text{--} & -74 & 1 & 58 & 1.2162162 & \text{--} & 93 & 1215 & -1122 \\
 253 & 11^1 23^1 & \text{--} & \text{Y} & \text{N} & \text{N} & \text{--} & 5 & 1 & 0 & 1.0000000 & \text{--} & 98 & 1220 & -1122 \\
 254 & 2^1 127^1 & \text{--} & \text{Y} & \text{N} & \text{N} & \text{--} & 5 & 1 & 0 & 1.0000000 & \text{--} & 103 & 1225 & -1122 \\
 255 & 3^1 5^1 17^1 & \text{--} & \text{Y} & \text{N} & \text{N} & \text{--} & -16 & 1 & 0 & 1.0000000 & \text{--} & 87 & 1225 & -1138 \\
 256 & 2^8 & \text{--} & \text{N} & \text{Y} & \text{N} & \text{--} & 2 & 1 & 0 & 4.5000000 & \text{--} & 89 & 1227 & -1138 \\
 257 & 257^1 & \text{--} & \text{Y} & \text{Y} & \text{N} & \text{--} & -2 & 1 & 0 & 1.0000000 & \text{--} & 87 & 1227 & -1140 \\
 258 & 2^1 3^1 43^1 & \text{--} & \text{Y} & \text{N} & \text{N} & \text{--} & -16 & 1 & 0 & 1.0000000 & \text{--} & 71 & 1227 & -1156 \\
 259 & 7^1 37^1 & \text{--} & \text{Y} & \text{N} & \text{N} & \text{--} & 5 & 1 & 0 & 1.0000000 & \text{--} & 76 & 1232 & -1156 \\
 260 & 2^2 5^1 13^1 & \text{--} & \text{N} & \text{N} & \text{Y} & \text{--} & 30 & 1 & 14 & 1.1666667 & \text{--} & 106 & 1262 & -1156 \\
 261 & 3^2 29^1 & \text{--} & \text{N} & \text{N} & \text{Y} & \text{--} & -7 & 1 & 2 & 1.2857143 & \text{--} & 99 & 1262 & -1163 \\
 262 & 2^1 131^1 & \text{--} & \text{Y} & \text{N} & \text{N} & \text{--} & 5 & 1 & 0 & 1.0000000 & \text{--} & 104 & 1267 & -1163 \\
 263 & 263^1 & \text{--} & \text{Y} & \text{Y} & \text{N} & \text{--} & -2 & 1 & 0 & 1.0000000 & \text{--} & 102 & 1267 & -1165 \\
 264 & 2^3 3^1 11^1 & \text{--} & \text{N} & \text{N} & \text{Y} & \text{--} & -48 & 1 & 32 & 1.3333333 & \text{--} & 54 & 1267 & -1213 \\
 265 & 5^1 53^1 & \text{--} & \text{Y} & \text{N} & \text{N} & \text{--} & 5 & 1 & 0 & 1.0000000 & \text{--} & 59 & 1272 & -1213 \\
 266 & 2^1 7^1 19^1 & \text{--} & \text{Y} & \text{N} & \text{N} & \text{--} & -16 & 1 & 0 & 1.0000000 & \text{--} & 43 & 1272 & -1229 \\
 267 & 3^1 89^1 & \text{--} & \text{Y} & \text{N} & \text{N} & \text{--} & 5 & 1 & 0 & 1.0000000 & \text{--} & 48 & 1277 & -1229 \\
 268 & 2^2 67^1 & \text{--} & \text{N} & \text{N} & \text{Y} & \text{--} & -7 & 1 & 2 & 1.2857143 & \text{--} & 41 & 1277 & -1236 \\
 269 & 269^1 & \text{--} & \text{Y} & \text{Y} & \text{N} & \text{--} & -2 & 1 & 0 & 1.0000000 & \text{--} & 39 & 1277 & -1238 \\
 270 & 2^1 3^3 5^1 & \text{--} & \text{N} & \text{N} & \text{Y} & \text{--} & -48 & 1 & 32 & 1.3333333 & \text{--} & -9 & 1277 & -1286 \\
 271 & 271^1 & \text{--} & \text{Y} & \text{Y} & \text{N} & \text{--} & -2 & 1 & 0 & 1.0000000 & \text{--} & -11 & 1277 & -1288 \\
 272 & 2^4 17^1 & \text{--} & \text{N} & \text{N} & \text{Y} & \text{--} & -11 & 1 & 6 & 1.8181818 & \text{--} & -22 & 1277 & -1299 \\
 273 & 3^1 7^1 13^1 & \text{--} & \text{Y} & \text{N} & \text{N} & \text{--} & -16 & 1 & 0 & 1.0000000 & \text{--} & -38 & 1277 & -1315 \\
 274 & 2^1 137^1 & \text{--} & \text{Y} & \text{N} & \text{N} & \text{--} & 5 & 1 & 0 & 1.0000000 & \text{--} & -33 & 1282 & -1315 \\
 275 & 5^2 11^1 & \text{--} & \text{N} & \text{N} & \text{Y} & \text{--} & -7 & 1 & 2 & 1.2857143 & \text{--} & -40 & 1282 & -1322 \\
 276 & 2^2 3^1 23^1 & \text{--} & \text{N} & \text{N} & \text{Y} & \text{--} & 30 & 1 & 14 & 1.1666667 & \text{--} & -10 & 1312 & -1322 \\
 277 & 277^1 & \text{--} & \text{Y} & \text{Y} & \text{N} & \text{--} & -2 & 1 & 0 & 1.0000000 & \text{--} & -12 & 1312 & -1324 \\
\end{array}
}
\end{equation*}

\end{table} 

\newpage
\begin{table}[h!]

\centering

\tiny
\begin{equation*}
\boxed{
\begin{array}{|cc|c|ccc|c|c|ccc|c|ccc}
 n & \mathbf{Primes} & & \mathbf{Sqfree} & \mathbf{PPower} & \bar{\mathbb{S}} & & g^{-1}(n) & 
 \lambda(n) \operatorname{sgn}(g^{-1}(n)) & \lambda(n) g^{-1}(n) - \widehat{f}_1(n) & 
 \frac{\sum\limits_{d|n} C_{\Omega(d)}(d)}{|g^{-1}(n)|} & & G^{-1}(n) & G^{-1}_{+}(n) & G^{-1}_{-}(n) \\ \hline 

 278 & 2^1 139^1 & \text{--} & \text{Y} & \text{N} & \text{N} & \text{--} & 5 & 1 & 0 & 1.0000000 & \text{--} & -7 & 1317 & -1324 \\
 279 & 3^2 31^1 & \text{--} & \text{N} & \text{N} & \text{Y} & \text{--} & -7 & 1 & 2 & 1.2857143 & \text{--} & -14 & 1317 & -1331 \\
 280 & 2^3 5^1 7^1 & \text{--} & \text{N} & \text{N} & \text{Y} & \text{--} & -48 & 1 & 32 & 1.3333333 & \text{--} & -62 & 1317 & -1379 \\
 281 & 281^1 & \text{--} & \text{Y} & \text{Y} & \text{N} & \text{--} & -2 & 1 & 0 & 1.0000000 & \text{--} & -64 & 1317 & -1381 \\
 282 & 2^1 3^1 47^1 & \text{--} & \text{Y} & \text{N} & \text{N} & \text{--} & -16 & 1 & 0 & 1.0000000 & \text{--} & -80 & 1317 & -1397 \\
 283 & 283^1 & \text{--} & \text{Y} & \text{Y} & \text{N} & \text{--} & -2 & 1 & 0 & 1.0000000 & \text{--} & -82 & 1317 & -1399 \\
 284 & 2^2 71^1 & \text{--} & \text{N} & \text{N} & \text{Y} & \text{--} & -7 & 1 & 2 & 1.2857143 & \text{--} & -89 & 1317 & -1406 \\
 285 & 3^1 5^1 19^1 & \text{--} & \text{Y} & \text{N} & \text{N} & \text{--} & -16 & 1 & 0 & 1.0000000 & \text{--} & -105 & 1317 & -1422 \\
 286 & 2^1 11^1 13^1 & \text{--} & \text{Y} & \text{N} & \text{N} & \text{--} & -16 & 1 & 0 & 1.0000000 & \text{--} & -121 & 1317 & -1438 \\
 287 & 7^1 41^1 & \text{--} & \text{Y} & \text{N} & \text{N} & \text{--} & 5 & 1 & 0 & 1.0000000 & \text{--} & -116 & 1322 & -1438 \\
 288 & 2^5 3^2 & \text{--} & \text{N} & \text{N} & \text{Y} & \text{--} & -47 & 1 & 42 & 1.7659574 & \text{--} & -163 & 1322 & -1485 \\
 289 & 17^2 & \text{--} & \text{N} & \text{Y} & \text{N} & \text{--} & 2 & 1 & 0 & 1.5000000 & \text{--} & -161 & 1324 & -1485 \\
 290 & 2^1 5^1 29^1 & \text{--} & \text{Y} & \text{N} & \text{N} & \text{--} & -16 & 1 & 0 & 1.0000000 & \text{--} & -177 & 1324 & -1501 \\
 291 & 3^1 97^1 & \text{--} & \text{Y} & \text{N} & \text{N} & \text{--} & 5 & 1 & 0 & 1.0000000 & \text{--} & -172 & 1329 & -1501 \\
 292 & 2^2 73^1 & \text{--} & \text{N} & \text{N} & \text{Y} & \text{--} & -7 & 1 & 2 & 1.2857143 & \text{--} & -179 & 1329 & -1508 \\
 293 & 293^1 & \text{--} & \text{Y} & \text{Y} & \text{N} & \text{--} & -2 & 1 & 0 & 1.0000000 & \text{--} & -181 & 1329 & -1510 \\
 294 & 2^1 3^1 7^2 & \text{--} & \text{N} & \text{N} & \text{Y} & \text{--} & 30 & 1 & 14 & 1.1666667 & \text{--} & -151 & 1359 & -1510 \\
 295 & 5^1 59^1 & \text{--} & \text{Y} & \text{N} & \text{N} & \text{--} & 5 & 1 & 0 & 1.0000000 & \text{--} & -146 & 1364 & -1510 \\
 296 & 2^3 37^1 & \text{--} & \text{N} & \text{N} & \text{Y} & \text{--} & 9 & 1 & 4 & 1.5555556 & \text{--} & -137 & 1373 & -1510 \\
 297 & 3^3 11^1 & \text{--} & \text{N} & \text{N} & \text{Y} & \text{--} & 9 & 1 & 4 & 1.5555556 & \text{--} & -128 & 1382 & -1510 \\
 298 & 2^1 149^1 & \text{--} & \text{Y} & \text{N} & \text{N} & \text{--} & 5 & 1 & 0 & 1.0000000 & \text{--} & -123 & 1387 & -1510 \\
 299 & 13^1 23^1 & \text{--} & \text{Y} & \text{N} & \text{N} & \text{--} & 5 & 1 & 0 & 1.0000000 & \text{--} & -118 & 1392 & -1510 \\
 300 & 2^2 3^1 5^2 & \text{--} & \text{N} & \text{N} & \text{Y} & \text{--} & -74 & 1 & 58 & 1.2162162 & \text{--} & -192 & 1392 & -1584 \\
 301 & 7^1 43^1 & \text{--} & \text{Y} & \text{N} & \text{N} & \text{--} & 5 & 1 & 0 & 1.0000000 & \text{--} & -187 & 1397 & -1584 \\
 302 & 2^1 151^1 & \text{--} & \text{Y} & \text{N} & \text{N} & \text{--} & 5 & 1 & 0 & 1.0000000 & \text{--} & -182 & 1402 & -1584 \\
 303 & 3^1 101^1 & \text{--} & \text{Y} & \text{N} & \text{N} & \text{--} & 5 & 1 & 0 & 1.0000000 & \text{--} & -177 & 1407 & -1584 \\
 304 & 2^4 19^1 & \text{--} & \text{N} & \text{N} & \text{Y} & \text{--} & -11 & 1 & 6 & 1.8181818 & \text{--} & -188 & 1407 & -1595 \\
 305 & 5^1 61^1 & \text{--} & \text{Y} & \text{N} & \text{N} & \text{--} & 5 & 1 & 0 & 1.0000000 & \text{--} & -183 & 1412 & -1595 \\
 306 & 2^1 3^2 17^1 & \text{--} & \text{N} & \text{N} & \text{Y} & \text{--} & 30 & 1 & 14 & 1.1666667 & \text{--} & -153 & 1442 & -1595 \\
 307 & 307^1 & \text{--} & \text{Y} & \text{Y} & \text{N} & \text{--} & -2 & 1 & 0 & 1.0000000 & \text{--} & -155 & 1442 & -1597 \\
 308 & 2^2 7^1 11^1 & \text{--} & \text{N} & \text{N} & \text{Y} & \text{--} & 30 & 1 & 14 & 1.1666667 & \text{--} & -125 & 1472 & -1597 \\
 309 & 3^1 103^1 & \text{--} & \text{Y} & \text{N} & \text{N} & \text{--} & 5 & 1 & 0 & 1.0000000 & \text{--} & -120 & 1477 & -1597 \\
 310 & 2^1 5^1 31^1 & \text{--} & \text{Y} & \text{N} & \text{N} & \text{--} & -16 & 1 & 0 & 1.0000000 & \text{--} & -136 & 1477 & -1613 \\
 311 & 311^1 & \text{--} & \text{Y} & \text{Y} & \text{N} & \text{--} & -2 & 1 & 0 & 1.0000000 & \text{--} & -138 & 1477 & -1615 \\
 312 & 2^3 3^1 13^1 & \text{--} & \text{N} & \text{N} & \text{Y} & \text{--} & -48 & 1 & 32 & 1.3333333 & \text{--} & -186 & 1477 & -1663 \\
 313 & 313^1 & \text{--} & \text{Y} & \text{Y} & \text{N} & \text{--} & -2 & 1 & 0 & 1.0000000 & \text{--} & -188 & 1477 & -1665 \\
 314 & 2^1 157^1 & \text{--} & \text{Y} & \text{N} & \text{N} & \text{--} & 5 & 1 & 0 & 1.0000000 & \text{--} & -183 & 1482 & -1665 \\
 315 & 3^2 5^1 7^1 & \text{--} & \text{N} & \text{N} & \text{Y} & \text{--} & 30 & 1 & 14 & 1.1666667 & \text{--} & -153 & 1512 & -1665 \\
 316 & 2^2 79^1 & \text{--} & \text{N} & \text{N} & \text{Y} & \text{--} & -7 & 1 & 2 & 1.2857143 & \text{--} & -160 & 1512 & -1672 \\
 317 & 317^1 & \text{--} & \text{Y} & \text{Y} & \text{N} & \text{--} & -2 & 1 & 0 & 1.0000000 & \text{--} & -162 & 1512 & -1674 \\
 318 & 2^1 3^1 53^1 & \text{--} & \text{Y} & \text{N} & \text{N} & \text{--} & -16 & 1 & 0 & 1.0000000 & \text{--} & -178 & 1512 & -1690 \\
 319 & 11^1 29^1 & \text{--} & \text{Y} & \text{N} & \text{N} & \text{--} & 5 & 1 & 0 & 1.0000000 & \text{--} & -173 & 1517 & -1690 \\
 320 & 2^6 5^1 & \text{--} & \text{N} & \text{N} & \text{Y} & \text{--} & -15 & 1 & 10 & 2.3333333 & \text{--} & -188 & 1517 & -1705 \\
 321 & 3^1 107^1 & \text{--} & \text{Y} & \text{N} & \text{N} & \text{--} & 5 & 1 & 0 & 1.0000000 & \text{--} & -183 & 1522 & -1705 \\
 322 & 2^1 7^1 23^1 & \text{--} & \text{Y} & \text{N} & \text{N} & \text{--} & -16 & 1 & 0 & 1.0000000 & \text{--} & -199 & 1522 & -1721 \\
 323 & 17^1 19^1 & \text{--} & \text{Y} & \text{N} & \text{N} & \text{--} & 5 & 1 & 0 & 1.0000000 & \text{--} & -194 & 1527 & -1721 \\
 324 & 2^2 3^4 & \text{--} & \text{N} & \text{N} & \text{Y} & \text{--} & 34 & 1 & 29 & 1.6176471 & \text{--} & -160 & 1561 & -1721 \\
 325 & 5^2 13^1 & \text{--} & \text{N} & \text{N} & \text{Y} & \text{--} & -7 & 1 & 2 & 1.2857143 & \text{--} & -167 & 1561 & -1728 \\
 326 & 2^1 163^1 & \text{--} & \text{Y} & \text{N} & \text{N} & \text{--} & 5 & 1 & 0 & 1.0000000 & \text{--} & -162 & 1566 & -1728 \\
 327 & 3^1 109^1 & \text{--} & \text{Y} & \text{N} & \text{N} & \text{--} & 5 & 1 & 0 & 1.0000000 & \text{--} & -157 & 1571 & -1728 \\
 328 & 2^3 41^1 & \text{--} & \text{N} & \text{N} & \text{Y} & \text{--} & 9 & 1 & 4 & 1.5555556 & \text{--} & -148 & 1580 & -1728 \\
 329 & 7^1 47^1 & \text{--} & \text{Y} & \text{N} & \text{N} & \text{--} & 5 & 1 & 0 & 1.0000000 & \text{--} & -143 & 1585 & -1728 \\
 330 & 2^1 3^1 5^1 11^1 & \text{--} & \text{Y} & \text{N} & \text{N} & \text{--} & 65 & 1 & 0 & 1.0000000 & \text{--} & -78 & 1650 & -1728 \\
 331 & 331^1 & \text{--} & \text{Y} & \text{Y} & \text{N} & \text{--} & -2 & 1 & 0 & 1.0000000 & \text{--} & -80 & 1650 & -1730 \\
 332 & 2^2 83^1 & \text{--} & \text{N} & \text{N} & \text{Y} & \text{--} & -7 & 1 & 2 & 1.2857143 & \text{--} & -87 & 1650 & -1737 \\
 333 & 3^2 37^1 & \text{--} & \text{N} & \text{N} & \text{Y} & \text{--} & -7 & 1 & 2 & 1.2857143 & \text{--} & -94 & 1650 & -1744 \\
 334 & 2^1 167^1 & \text{--} & \text{Y} & \text{N} & \text{N} & \text{--} & 5 & 1 & 0 & 1.0000000 & \text{--} & -89 & 1655 & -1744 \\
 335 & 5^1 67^1 & \text{--} & \text{Y} & \text{N} & \text{N} & \text{--} & 5 & 1 & 0 & 1.0000000 & \text{--} & -84 & 1660 & -1744 \\
 336 & 2^4 3^1 7^1 & \text{--} & \text{N} & \text{N} & \text{Y} & \text{--} & 70 & 1 & 54 & 1.5000000 & \text{--} & -14 & 1730 & -1744 \\
 337 & 337^1 & \text{--} & \text{Y} & \text{Y} & \text{N} & \text{--} & -2 & 1 & 0 & 1.0000000 & \text{--} & -16 & 1730 & -1746 \\
 338 & 2^1 13^2 & \text{--} & \text{N} & \text{N} & \text{Y} & \text{--} & -7 & 1 & 2 & 1.2857143 & \text{--} & -23 & 1730 & -1753 \\
 339 & 3^1 113^1 & \text{--} & \text{Y} & \text{N} & \text{N} & \text{--} & 5 & 1 & 0 & 1.0000000 & \text{--} & -18 & 1735 & -1753 \\
 340 & 2^2 5^1 17^1 & \text{--} & \text{N} & \text{N} & \text{Y} & \text{--} & 30 & 1 & 14 & 1.1666667 & \text{--} & 12 & 1765 & -1753 \\
 341 & 11^1 31^1 & \text{--} & \text{Y} & \text{N} & \text{N} & \text{--} & 5 & 1 & 0 & 1.0000000 & \text{--} & 17 & 1770 & -1753 \\
 342 & 2^1 3^2 19^1 & \text{--} & \text{N} & \text{N} & \text{Y} & \text{--} & 30 & 1 & 14 & 1.1666667 & \text{--} & 47 & 1800 & -1753 \\
 343 & 7^3 & \text{--} & \text{N} & \text{Y} & \text{N} & \text{--} & -2 & 1 & 0 & 2.0000000 & \text{--} & 45 & 1800 & -1755 \\
 344 & 2^3 43^1 & \text{--} & \text{N} & \text{N} & \text{Y} & \text{--} & 9 & 1 & 4 & 1.5555556 & \text{--} & 54 & 1809 & -1755 \\
 345 & 3^1 5^1 23^1 & \text{--} & \text{Y} & \text{N} & \text{N} & \text{--} & -16 & 1 & 0 & 1.0000000 & \text{--} & 38 & 1809 & -1771 \\
 346 & 2^1 173^1 & \text{--} & \text{Y} & \text{N} & \text{N} & \text{--} & 5 & 1 & 0 & 1.0000000 & \text{--} & 43 & 1814 & -1771 \\
 347 & 347^1 & \text{--} & \text{Y} & \text{Y} & \text{N} & \text{--} & -2 & 1 & 0 & 1.0000000 & \text{--} & 41 & 1814 & -1773 \\
 348 & 2^2 3^1 29^1 & \text{--} & \text{N} & \text{N} & \text{Y} & \text{--} & 30 & 1 & 14 & 1.1666667 & \text{--} & 71 & 1844 & -1773 \\
 349 & 349^1 & \text{--} & \text{Y} & \text{Y} & \text{N} & \text{--} & -2 & 1 & 0 & 1.0000000 & \text{--} & 69 & 1844 & -1775 \\
 350 & 2^1 5^2 7^1 & \text{--} & \text{N} & \text{N} & \text{Y} & \text{--} & 30 & 1 & 14 & 1.1666667 & \text{--} & 99 & 1874 & -1775 \\
\end{array}
}
\end{equation*}

\end{table} 

%\NBRef{A03-2020-04026}
%\NBRef{A04-2020-04026}

\newpage
\setcounter{section}{0}
\renewcommand{\thesection}{Appendix \Alph{section}}
\renewcommand{\thesubsection}{\Alph{section}.\arabic{subsection}}

\end{document}
