\documentclass[11pt,reqno,a4letter]{article} 

\usepackage{amsthm,amsfonts,amscd,amsmath}
\usepackage[hidelinks]{hyperref} 
\usepackage{url}
\usepackage[usenames,dvipsnames]{xcolor}
\hypersetup{
    colorlinks,
    linkcolor={green!63!darkgray},
    citecolor={blue!70!white},
    urlcolor={blue!80!white}
}

\usepackage[normalem]{ulem}
\usepackage{graphicx} 
\usepackage{datetime} 
\usepackage{cancel}
\usepackage{subcaption}
\captionsetup{format=hang,labelfont={bf},textfont={small,it}} 
\numberwithin{figure}{section}
\numberwithin{table}{section}

\usepackage{stmaryrd,tikzsymbols,mathabx} 
\usepackage{framed} 
\usepackage{ulem}
\usepackage[T1]{fontenc}
\usepackage{pbsi}


\usepackage{enumitem}
\setlist[itemize]{leftmargin=0.65in}

\usepackage{rotating,adjustbox}

\usepackage{diagbox}
\newcommand{\trianglenk}[2]{$\diagbox{#1}{#2}$}
\newcommand{\trianglenkII}[2]{\diagbox{#1}{#2}}

\let\citep\cite

\newcommand{\undersetbrace}[2]{\underset{\displaystyle{#1}}{\underbrace{#2}}}

\newcommand{\gkpSI}[2]{\ensuremath{\genfrac{\lbrack}{\rbrack}{0pt}{}{#1}{#2}}} 
\newcommand{\gkpSII}[2]{\ensuremath{\genfrac{\lbrace}{\rbrace}{0pt}{}{#1}{#2}}}
\newcommand{\cf}{\textit{cf.\ }} 
\newcommand{\Iverson}[1]{\ensuremath{\left[#1\right]_{\delta}}} 
\newcommand{\floor}[1]{\left\lfloor #1 \right\rfloor} 
\newcommand{\ceiling}[1]{\left\lceil #1 \right\rceil} 
\newcommand{\e}[1]{e\left(#1\right)} 
\newcommand{\seqnum}[1]{\href{http://oeis.org/#1}{\color{ProcessBlue}{\underline{#1}}}}

\usepackage{upgreek,dsfont,amssymb}
\renewcommand{\chi}{\upchi}
\newcommand{\ChiFunc}[1]{\ensuremath{\chi_{\{#1\}}}}
\newcommand{\OneFunc}[1]{\ensuremath{\mathds{1}_{#1}}}

\usepackage{ifthen}
\newcommand{\Hn}[2]{
     \ifthenelse{\equal{#2}{1}}{H_{#1}}{H_{#1}^{\left(#2\right)}}
}

\newcommand{\Floor}[2]{\ensuremath{\left\lfloor \frac{#1}{#2} \right\rfloor}}
\newcommand{\Ceiling}[2]{\ensuremath{\left\lceil \frac{#1}{#2} \right\rceil}}

\DeclareMathOperator{\DGF}{DGF} 
\DeclareMathOperator{\ds}{ds} 
\DeclareMathOperator{\Id}{Id}
\DeclareMathOperator{\fg}{fg}
\DeclareMathOperator{\Div}{div}
\DeclareMathOperator{\rpp}{rpp}
\DeclareMathOperator{\logll}{\ell\ell}

\title{
       \LARGE{
       Lower bounds on the Mertens function $M(x)$ along infinite subsequences for large 
       $x \gg \exp\left(1.1662794 \times 10^{703}\right)$ 
       } 
}
\author{{\Large Maxie Dion Schmidt} \\ 
        %{\normalsize \href{mailto:maxieds@gmail.com}{maxieds@gmail.com}} \\[0.1cm] 
        {\normalsize Georgia Institute of Technology} \\[0.025cm] 
        {\normalsize School of Mathematics} 
} 

\date{\small\underline{Last Revised:} \today \ @\ \hhmmsstime{} \ -- \ Compiled with \LaTeX2e} 

\theoremstyle{plain} 
\newtheorem{theorem}{Theorem}
\newtheorem{conjecture}[theorem]{Conjecture}
\newtheorem{claim}[theorem]{Claim}
\newtheorem{prop}[theorem]{Proposition}
\newtheorem{lemma}[theorem]{Lemma}
\newtheorem{cor}[theorem]{Corollary}
\numberwithin{theorem}{section}

\theoremstyle{definition} 
\newtheorem{example}[theorem]{Example}
\newtheorem{remark}[theorem]{Remark}
\newtheorem{definition}[theorem]{Definition}
\newtheorem{notation}[theorem]{Notation}
\newtheorem{question}[theorem]{Question}
\newtheorem{discussion}[theorem]{Discussion}
\newtheorem{facts}[theorem]{Facts}
\newtheorem{summary}[theorem]{Summary}
\newtheorem{heuristic}[theorem]{Heuristic}

\renewcommand{\arraystretch}{1.25} 

\setlength{\textheight}{9in}
\setlength{\topmargin}{-.18in}
\setlength{\textwidth}{7.65in} 
\setlength{\evensidemargin}{-0.25in} 
\setlength{\oddsidemargin}{-0.25in} 
\setlength{\headsep}{8pt} 
%\setlength{\footskip}{10pt} 

\usepackage{geometry}
%\newgeometry{top=0.65in, bottom=18mm, left=15mm, right=15mm, outer=2in, heightrounded, marginparwidth=1.5in, marginparsep=0.15in}
\newgeometry{top=0.65in, bottom=16mm, left=15mm, right=15mm, heightrounded, marginparwidth=0in, marginparsep=0.15in}

\usepackage{fancyhdr}
\pagestyle{empty}
\pagestyle{fancy}
\fancyhead[RO,RE]{M. D. Schmidt -- Prepared for Myself to Review -- on \today} 
\fancyhead[LO,LE]{}
\fancyheadoffset{0.005\textwidth} 

\setlength{\parindent}{0in}
\setlength{\parskip}{2cm} 

\renewcommand{\thefootnote}{\Alph{footnote}}
\makeatletter
\@addtoreset{footnote}{section}
\makeatother

%\usepackage{marginnote,todonotes}
%\colorlet{NBRefColor}{RoyalBlue!73} 
%\newcommand{\NBRef}[1]{
%     \todo[linecolor=green!85!white,backgroundcolor=orange!50!white,bordercolor=blue!30!black,textcolor=cyan!15!black,shadow,size=\small,fancyline]{
%     \color{NBRefColor}{\textbf{#1}
%     }
%     }
%}
\newcommand{\NBRef}[1]{}  

\newcommand{\SuccSim}[0]{\overset{_{\scriptsize{\blacktriangle}}}{\succsim}} 
\newcommand{\PrecSim}[0]{\overset{_{\scriptsize{\blacktriangle}}}{\precsim}} 
\renewcommand{\SuccSim}[0]{\ensuremath{\gg}} 
\renewcommand{\PrecSim}[0]{\ensuremath{\ll}} 

\renewcommand{\Re}{\operatorname{Re}}
\renewcommand{\Im}{\operatorname{Im}}

\input{glossaries-bibtex/PreambleGlossaries-Mertens}

\usepackage{tikz}
\usetikzlibrary{shapes,arrows}

\usepackage{enumitem} 

\allowdisplaybreaks 

\begin{document} 

\maketitle

\begin{abstract} 
The Mertens function, $M(x) = \sum_{n \leq x} \mu(n)$, is classically 
defined to be the summatory function of the M\"obius function 
$\mu(n)$. The 
Mertens conjecture stating that $|M(x)| < C \cdot \sqrt{x}$ with $C > 0$ for all 
$x \geq 1$ has a well-known disproof due to Odlyzko and t\'{e} Riele given in the early 1980's by computation of 
non-trivial zeta function zeros in conjunction with integral formulas expressing $M(x)$. 
It is conjectured and widely believed that $M(x) / \sqrt{x}$ changes sign infinitely often and grows 
unbounded in the direction of both $\pm \infty$ along subsequences of integers $x \geq 1$. 
Our proof this property of $|M(x)|/\sqrt{x}$, e.g., showing that 
$$\limsup_{x \rightarrow \infty} \frac{|M(x)|}{\sqrt{x}} = +\infty,$$ is not based on 
standard estimates of $M(x)$ we find from Mellin inversion, which are intimately tied to the 
intricate distribution of the non-trivial zeros of the Riemann zeta function. 
There is a distinct stylistic 
flavor and new element of combinatorial analysis 
peppered in with the standard methods from analytic, additive and elementary number theory. 
This stylistic tendency distinguishes 
our methods from other proofs of established upper, rather than lower, bounds on $M(x)$. 

\bigskip 
\noindent
\textbf{Keywords and Phrases:} {\it M\"obius function; Mertens function; summatory function; 
                                    Dirichlet inverse; Liouville lambda function; prime omega function; 
                                    prime counting functions; Dirichlet generating function; 
                                    asymptotic lower bounds; Mertens conjecture. } \\ 
% 11-XX			Number theory
%    11A25  	Arithmetic functions; related numbers; inversion formulas
%    11Y70  	Values of arithmetic functions; tables
%    11-04  	Software, source code, etc. for problems pertaining to number theory
% 11Nxx		Multiplicative number theory
%    11N05  	Distribution of primes
%    11N37  	Asymptotic results on arithmetic functions
%    11N56  	Rate of growth of arithmetic functions
%    11N60  	Distribution functions associated with additive and positive multiplicative functions
%    11N64  	Other results on the distribution of values or the characterization of arithmetic functions
\textbf{Primary Math Subject Classifications (2010):} {\it 11N37; 11A25; 11N60; and 11N64. } 
\end{abstract} 

\bigskip\hrule\bigskip

\newpage
%\section{Reference on abbreviations, special notation and other conventions} 
\label{Appendix_Glossary_NotationConvs}
     \vskip 0in
     \printglossary[type={symbols},
                    title={Reference on special notation and conventions},
                    style={glossstyleSymbol},
                    nogroupskip=true]


%\newpage
%\setcounter{tocdepth}{2}
%\renewcommand{\contentsname}{Listing of major sections and topics} 
%\tableofcontents 

\newpage
\section{Preface: Notation to express asymptotic relations} 

We exphasize that the next careful explanation of the subtle distinctions to our usage of 
what we consider to be traditional notation for asymptotic relations are key to 
understanding our choices of bounding expressions given throughout the article. 
Thus, to avoid any confusion that may linger as we begin to state our new results and bounds on the 
functions we work with in this article, we preface the article starting with this section detailing 
our precise definitions, meanings and assumptions on the uses of certain symbols, operators, and 
relations. The interpretation of this notation forms the core of how we choose 
to convey the growth rates of arithmetic functions on their domain of $x$ within this article 
when $x$ is taken to be very large as $x \rightarrow \infty$ 
\cite[\cf \S 2]{NISTHB} \cite{ACOMB-BOOK}. 

\subsection{Average order, similarity and approximation of asymptotic growth rates} 

\subsubsection{Similarity and average order (expectation)} 

We say that two arithmetic functions $A(x), B(x)$ (with $B > 0$) satisfy the relation $A \sim B$ if 
\[
\lim_{x \rightarrow \infty} \frac{A(x)}{B(x)} = 1. 
\] 
It is sometimes standard to express the \emph{average order} of an arithmetic function f as 
$f \sim h$, even when the values of $f(n)$ may actually non-monotonically 
oscillate in magnitude infinitely often. What the notation $f \sim h$ means when using this 
notation to express the average order of $f$ is that 
$$\frac{1}{x} \cdot \sum_{n \leq x} f(n) \sim h(x).$$ 
For example, in the acceptably classic language of \cite{HARDYWRIGHT} we would normally write that 
$\Omega(n) \sim \log\log n$, even though technically, 
$1 \leq \Omega(n) \leq \frac{\log n}{\log 2}$. 
To be absolutely clear about notation, we intentionally do not re-use the $\sim$ relation by 
instead writing $\mathbb{E}[f(x)] = h(x)$ (as in expectation of $f$), or sometimes 
$\overset{\mathbb{E}}{\sim}$ for convenience,  
to denote that $f$ has a limiting average order growing at the 
rate of $h$. 

%A related conception of $f$ having a so-called \emph{normal order} of $g$ holds whenever 
%$$f(n) = (1+o(1)) g(n), \mathrm{a.e.}, \mathrm{\ as\ } n \rightarrow \infty.$$

\subsubsection{Abel summation} 

The formula we prefer for the Abel summation variant of summation by parts 
to express finite sums of a product of two functions is stated as follows 
\cite[\cf \S 4.3]{APOSTOLANUMT} \footnote{
     Compare to the exact formula for \emph{summation by parts} of any arithmetic functions, $u_n,v_n$, 
     stated as in \cite[\S 2.10(ii)]{NISTHB} for $U_j := u_1+u_2+\cdots+u_j$ when $j \geq 1$: 
     \[
     \sum_{j=1}^{n-1} u_j \cdot v_j = U_{n-1} v_n + \sum_{j=1}^{n-1} U_j \left(v_j - v_{j+1}\right), n \geq 2. 
     \]
}: 
 
\begin{prop}[Abel Summation Integral Formula] 
\label{prop_AbelSummationFormula} 
Suppose that $t > 0$ is real-valued, and that $A(t) \sim \sum_{n \leq t} a(n)$ for some weighting 
arithmetic function $a(n)$ with $A(t)$ continuously differentiable on $(0, \infty)$. Furthermore, suppose that 
$b(n) \sim f(n)$ with $f$ a differentiable function of $n \geq 0$ -- that is, $f^{\prime}(t)$ exists and is smooth for all 
$t \in (0, \infty)$. 
Then for $0 \leq y < x$, where we typcially assume that the bounds of summation satisfy 
$x, y \in \mathbb{Z}^{+}$, we have that 
\[
\sum_{y < n \leq x} a(n) b(n) \sim A(x)b(x) - A(y)b(y) - \int_y^{x} A(t) f^{\prime}(t) dt. 
\] 
\end{prop}
          
\subsubsection{Approximation} 
     
We choose to adopt the convention to write that $f(x) \approx g(x)$ if $|f(x) - g(x)| = O(1)$ as 
$x \rightarrow \infty$. 
That is, we write $f(x) \approx g(x)$ to denote that $f$ is approximately equal to $g$ at $x$ modulo at most a
small constant difference between the functions when $x$ is large. 

\subsubsection{Vinogradov's notation for asymptotics} 

We use the conventional relations $f(x) \gg g(x)$ and $h(x) \ll r(x)$ to symbolically express that we should expect 
$f$ to be ``substantially`` larger than $g$, and respectively $h$ 
to be ``significantly'' smaller than $r$, in asymptotic order 
(e.g., rate of growth when $x$ is large). In practice, we adopt a somewhat looser definition of these symbols which 
allows $f \gg g$ and $h \ll r$ provided that there are constants $C, D \geq 1$ such that whenever $x$ is sufficiently 
large we have that $C \cdot f(x) \geq g(x)$ and $h(x) \leq D \cdot r(x)$. This notation is sometimes called 
\emph{Vinogradov's asymptotic notation}. 

\subsection{Asymptotic expansions and uniformity} 

Because a subset of the results we cite that are proved in the references 
provide statements of 
asymptotic bounds that hold \emph{uniformly} for $x$ large depending on parameters, 
we need to briefly make precise what our preconceptions are of this terminology. 
We introduce the notation for asymptotic expansions of a function $f: \mathbb{R} \rightarrow \mathbb{R}$ from 
\cite[\S 2.1(iii)]{NISTHB}. 

\subsubsection{Ordinary asymptotic expansions of a function} 

Let $\sum_{n} a_n x^{-n}$ denote a formal power series expansion in $x$ where we 
ignore any necessary conditions to guarantee convergence of the series. For each integer $n \geq 1$, suppose that 
\[
f(x) = \sum_{s=0}^{n-1} a_s x^{-s} + O(x^{-n}), 
\]
as $|x| \rightarrow \infty$ where this limiting bound holds for $x \in \mathbb{X}$ in some unbounded set 
$\mathbb{X} \subseteq \mathbb{R}, \mathbb{C}$. 
When such a bound holds, we say that $\sum_s a_s x^{-s}$ is a \emph{Poincar\'{e} asymptotic expansion}, 
or just \emph{asymptotic series expansion}, of $f(x)$ as $x \rightarrow \infty$ along the fixed set $\mathbb{X}$. 
The condition in the previous equation is equivalent to writing 
\[
f(x) \sim a_0 + a_1 x^{-1} + a_2 x^{-2} + \cdots; x \in \mathbb{X}, \mathrm{\ as \ } |x| \rightarrow \infty. 
\]
The prior two characterizations of an asymptotic expansion for $f$ are also equivalent to the 
statement that 
\[
x^n \left(f(x) - \sum_{s=0}^{n-1} a_s x^{-s}\right) \xrightarrow{x \rightarrow \infty} a_n. 
\] 

\subsubsection{Uniform asymptotic expansions of a function} 

Let the set $\mathbb{X}$ from the definition in the last subsection correspond to a 
closed sector of the form 
$$\mathbb{X} := \{x \in \mathbb{C}: \alpha \leq \operatorname{arg}(x) \leq \beta\}.$$ 
Then we say that the asymptotic property 
\[
f(x) = \sum_{s=0}^{n-1} a_s x^{-s} + O(x^{-n}), 
\]
from before holds \emph{uniformly} with respect to $\operatorname{arg}(x) \in [\alpha, \beta]$ as 
$|x| \rightarrow \infty$. 

Another useful, important notion of uniform asymptotic bounds is taken with respect to some parameter $u$ 
(or set of parameters, respectively) that ranges over the point set (point sets, respectively) 
$u \in \mathbb{U}$. In this case, if we have that the $u$-parameterized expressions 
\[
\left\lvert x^n\left(f(u, x) - \sum_{s=0}^{n-1} a_s(u) x^{-s}\right) \right\rvert, 
\]
are bounded for all integers $n \geq 1$ for $x \in \mathbb{X}$ as $|x| \rightarrow \infty$, then we say that 
the asymptotic expansion of $f$ holds \emph{uniformly} for $u \in \mathbb{U}$. 
Now the function $f \equiv f(u, x)$ and the 
asymptotic series coefficients $a_s(u)$ may have an implicit dependence on the parameter $u$. 
If the previous boundedness condition holds for all positive integers $n$, we write that 
\[
f(u, x) \sim \sum_{s=0}^{\infty} a_s(u) x^{-s}; x \in \mathbb{X}, \mathrm{\ as \ } |x| \rightarrow \infty, 
\]
and say that this asymptotic expansion, or bound, holds \emph{uniformly with respect to $u \in \mathbb{U}$}. 
For $u$ taken outside of $\mathbb{U}$, the stated bound may fail to be valid even for $x \in \mathbb{X}$ as 
$|x| \rightarrow \infty$. 

\subsection{Asymptotic densities of subsets of the integers} 

In the proofs given in Section \ref{Section_ProofOfValidityOfAverageOrderLowerBounds} 
of the article, we will require a precise notion of the 
\emph{asymptotic density} of an infinite set $\mathcal{S} \subseteq \mathbb{Z}^{+}$. 
When this limit exists, we denote the asymptotic density of 
$\mathcal{S}$ by $\alpha_{\mathcal{S}} \in [0, 1]$, defined as follows: 
\[
\alpha_{\mathcal{S}} := \lim_{x \rightarrow \infty} \frac{1}{x} \cdot \#\left\{ 
     n \leq x: n \in \mathcal{S} 
     \right\}. 
\]
In other words, if the set $\mathcal{S}$ has asymptotic density $\alpha_{\mathcal{S}}$, then 
for all sufficiently large $x$ 
\[
\alpha_{\mathcal{S}} + o(1) \leq \frac{1}{x} \cdot \#\left\{ 
     n \leq x: n \in \mathcal{S} \right\} \leq 
     \alpha_{\mathcal{S}} + o(1). 
\]
When the limit definition of $\alpha_{\mathcal{S}}$ does not exist, or if by some 
pathology of the way $\mathcal{S}$ is defined, we cannot precisely pin down the limit, we are often 
interested in sets of bounded asymptotic density. We say that $\mathcal{S}$ has 
\emph{bounded asymptotic density} if for all large $x$ there exist constants $0 \leq B \leq C \leq 1$ 
such that 
\[
B + o(1) \leq \frac{1}{x} \cdot \#\left\{ 
     n \leq x: n \in \mathcal{S} \right\} \leq C + o(1), \mathrm{\ as\ } x \rightarrow \infty. 
\]
Clearly, finite and bounded subsets of the positive integers have limiting asymptotic density of zero. 
If the asymptotic density of $\mathcal{S}$ is one, and some property $\mathcal{P}(n)$ holds 
for all $n \in \mathcal{S}$, then we say that $\mathcal{P}(n)$ is true 
\emph{almost everywhere} (on the integers), also abbreviated as holding ``a.e.`` on the positive integers 
as $n \rightarrow \infty$. 

\newpage
\section{An introduction to the Mertens function} 
\label{subSection_MertensMxClassical_Intro} 

\subsection{Definitions} 

Suppose that $n \geq 2$ is a natural number with factorization into 
distinct primes given by 
$n = p_1^{\alpha_1} p_2^{\alpha_2} \cdots p_k^{\alpha_k}$. 
We define the \emph{M\"oebius function} to be the signed indicator function 
of the squarefree integers as follows: 
\[
\mu(n) = \begin{cases} 
     1, & \text{if $n = 1$; } \\ 
     (-1)^k, & \text{if $\alpha_i = 1$, $\forall 1 \leq i \leq k$; } \\ 
     0, & \text{otherwise.} 
     \end{cases} 
\]
There are many other known variants and special properties of the M\"oebius function 
and its generalizations \cite[\cf \S 2]{HANDBOOKNT-2004}. 
A crucial role of the classical $\mu(n)$ forming an inversion relation 
for arithmetic functions convolved with one (e.g., Dirichlet convolutions of the form $g = f \ast 1$) is 
known as \emph{M\"obius inversion}: 
\[
g(n) = (f \ast 1)(n) \iff f(n) = (g \ast \mu)(n), \forall n \geq 1. 
\]
The \emph{Mertens function}, or summatory function of $\mu(n)$, is defined as 
\begin{align*} 
M(x) & = \sum_{n \leq x} \mu(n), x \geq 1. 
\end{align*} 
The sequence of the oscillatory values of this summatory function begins as 
\cite[\seqnum{A002321}]{OEIS} 
\[
\{M(x)\}_{x \geq 1} = \{1, 0, -1, -1, -2, -1, -2, -2, -2, -1, -2, -2, -3, -2, 
     -1, -1, -2, -2, -3, -3, -2, -1, -2, -2, \ldots\}
\]
Clearly, a positive integer $n \geq 1$ is \emph{squarefree}, or contains no (prime power) divisors which are 
squares, if and only if $\mu^2(n) = 0$. 
A related function which counts the 
number of \emph{squarefree} integers $n \leq x$ then satisfies 
\cite[\S 18.6]{HARDYWRIGHT} \cite[\seqnum{A013928}]{OEIS} 
\[ 
Q(x) = \sum_{n \leq x} \mu^2(n) \sim \frac{6x}{\pi^2} + O\left(\sqrt{x}\right). 
\] 
It is known that the asymptotic density of the positively versus negatively 
weighted sets of squarefree numbers are in fact equal as $x \rightarrow \infty$: 
\[
\mu_{+}(x) = \frac{\#\{1 \leq n \leq x: \mu(n) = +1\}}{Q(x)} \overset{\mathbb{E}}{\sim} 
     \mu_{-}(x) = \frac{\#\{1 \leq n \leq x: \mu(n) = -1\}}{Q(x)} 
     \xrightarrow{x \rightarrow \infty} \frac{3}{\pi^2}. 
\]
The actual local oscillations between the approximate densities of the sets 
$\mu_{\pm}(x)$ lend an unpredictable nature to the function and characterize the 
oscillatory sawtooth shaped plot of $M(x)$ over the positive integers. 

\subsection{Properties} 

\subsubsection{Exact formulae} 

The conventional approach to evaluating the behavior of $M(x)$ for large 
$x \rightarrow \infty$ results from a formulation of this summatory 
function as a predictable exact sum involving $x$ and the non-trivial 
zeros of the Riemann zeta function for all real $x > 0$. 
This formula is expressed given the inverse Mellin transformation 
over the reciprocal zeta function. In particular, 
we notice that since 
\[
\frac{1}{\zeta(s)} = \int_1^{\infty} \frac{s \cdot M(x)}{x^{s+1}} dx, 
\]
we obtain that 
\[
M(x) = \lim_{T \rightarrow \infty}\ \frac{1}{2\pi\imath} \int_{T-\imath\infty}^{T+\imath\infty} 
     \frac{x^s}{s \cdot \zeta(s)} ds. 
\] 
This representation, along with the standard Euler product 
representation for the reciprocal zeta function, leads us to the 
exact expression for $M(x)$ for any real $x > 0$ 
given by the next theorem due to Titchmarsh. 
\nocite{TITCHMARSH} 

\begin{theorem}[Analytic Formula for $M(x)$] 
\label{theorem_MxMellinTransformInvFormula} 
Assuming the Riemann Hypothesis (RH), there exists an infinite sequence 
$\{T_k\}_{k \geq 1}$ satisfying $k \leq T_k \leq k+1$ for each $k$ 
such that for any real $x > 0$ 
\[
M(x) = \lim_{k \rightarrow \infty} 
     \sum_{\substack{\rho: \zeta(\rho) = 0 \\ |\Im(\rho)| < T_k}} 
     \frac{x^{\rho}}{\rho \cdot \zeta^{\prime}(\rho)} - 2 + 
     \sum_{n \geq 1} \frac{(-1)^{n-1}}{n \cdot (2n)! \zeta(2n+1)} 
     \left(\frac{2\pi}{x}\right)^{2n} + 
     \frac{\mu(x)}{2} \Iverson{x \in \mathbb{Z}^{+}}. 
\] 
\end{theorem} 

\subsubsection{Upper bounds} 

A historical unconditional bound on the Mertens function due to Walfisz (1963) 
states that there is an absolute constant $C > 0$ such that 
$$M(x) \ll x \cdot \exp\left(-C \cdot \log^{3/5}(x) 
  (\log\log x)^{-3/5}\right).$$ 
Under the assumption of the RH, Soundararajan proved new updated estimates in 2009 
bounding $M(x)$ for large $x$ in the following forms \cite{SOUND-MERTENS-ANNALS}: 
\begin{align*} 
M(x) & \ll \sqrt{x} \cdot \exp\left(\log^{1/2}(x) (\log\log x)^{14}\right), \\ 
M(x) & = O\left(\sqrt{x} \cdot \exp\left( 
     \log^{1/2}(x) (\log\log x)^{5/2+\epsilon}\right)\right),\ 
     \forall \epsilon > 0. 
\end{align*} 
To date, 
considerably less has been conjectured about explicit lower bounds on $|M(x)|$ along 
subsequences. 

\subsection{Conjectures on boundedness and limiting behavior} 

The RH is equivalent to showing that 
$M(x) = O\left(x^{\frac{1}{2}+\varepsilon}\right)$ for any 
$0 < \varepsilon < \frac{1}{2}$. 
There is a rich history to the original statement of the \emph{Mertens conjecture} which 
states that 
\[ 
|M(x)| < C \cdot \sqrt{x},\ \text{ some absolute constant $C > 0$. }
\] 
Mertens conjecture was first verified by Mertens for $C = 1$ and all $x < 10000$. 
Since its beginnings in 1897, the conjecture has been disproven by computation 
of low-lying zeta function zeros in a famous paper by 
Odlyzko and t\'{e} Riele from the early 1980's. 
Since the truth of the Mertens conjecture would have implied the RH, more recent attempts 
at bounding $M(x)$ favor determining the rate at which the function 
$M(x) / \sqrt{x}$ grows without bound towards both $\pm \infty$ along infinite 
subsequences. 

One of the most famous still unanswered questions about the Mertens 
function concerns whether $|M(x)| / \sqrt{x}$ is in actuality unbounded on the 
natural numbers. A precise statement of this 
problem is to produce an affirmative answer whether 
$\limsup_{x \rightarrow \infty} M(x) / \sqrt{x} = +\infty$ and 
$\liminf_{x \rightarrow \infty} M(x) / \sqrt{x} = -\infty$, or 
equivalently whether there is an infinite subsequence of natural numbers 
$\{x_1, x_2, x_3, \ldots\}$ such that the magnitude of 
$M(x_i) x_i^{-1/2}$ grows without bound towards either $\pm \infty$ 
along the subsequence. 
We cite that prior to this point it is only known by computation 
that \cite[\cf \S 4.1]{PRIMEREC} 
\[
\limsup_{x\rightarrow\infty} \frac{M(x)}{\sqrt{x}} > 1.060\ \qquad (\text{now } \geq 1.826054), 
\] 
and 
\[ 
\liminf_{x\rightarrow\infty} \frac{M(x)}{\sqrt{x}} < -1.009\ \qquad (\text{now } \leq -1.837625). 
\] 
Based on work by Odlyzyko and t\'{e} Riele, it seems probable that 
each of these limits should evaluate to $\pm \infty$, respectively 
\cite{ODLYZ-TRIELE,MREVISITED,ORDER-MERTENSFN,HURST-2017}. 

Extensive computational evidence has produced 
a conjecture due to Gonek (among attempts on limiting bounds by others) that in fact the limiting behavior of 
$M(x)$ satisfies that \cite{NG-MERTENS}
$$\limsup_{x \rightarrow \infty} \frac{|M(x)|}{\sqrt{x} \cdot (\log\log x)^{5/4}} = O(1).$$ 
While it seems to be widely believed that $|M(x)| / \sqrt{x}$ tends to $+\infty$ at a logarithmic rate 
along subsequences, infinitely tending factors such as the $(\log\log x)^{\frac{5}{4}}$ in Gonek's conjecture 
do not appear to readily fall out of work on unconditional bounds for $M(x)$ by existing methods. 

\newpage
\section{A summary outline: The core logical steps and components to the proof} 

We offer an initial brief step-by-step summary overview of the critical components 
to our proof outlined in the next section of the introduction below 
that are proved piece-by-piece in the following sections of the article. 
As the proof methodology is new and relies on non-standard elements compared to more 
traditional methods of bounding $M(x)$, we hope that this sketch of the logical components 
to our new argument makes the article easier to parse. 

\subsection{Step-by-step overview} 

The following outline is intended to help 
the reader see our logic and proof methodology as easily and quickly as possible: 
\begin{itemize} 

\item[\textbf{(1)}] We prove a matrix inversion formula relating the summatory 
           functions of an arithmetic function $f$ and its Dirichlet inverse $f^{-1}$ (for $f(1) \neq 0$). 
           See 
           Theorem \ref{theorem_SummatoryFuncsOfDirCvls} in 
           Section \ref{Section_PrelimProofs_Config}.  
\item[\textbf{(2)}] This crucial step provides us with an exact formula for $M(x)$ in terms of $\pi(x)$, the seemingly 
           unconnected prime counting function, and the 
           Dirichlet inverse of the shifted additive function $g(n) := \omega(n)+1$. This 
           formula is stated in \eqref{eqn_Mx_gInvnPixk_formula}.  
\item[\textbf{(3)}] We tighten an updated result from \cite[\S 7]{MV} providing uniform asymptotic formulas for the  
           summatory functions, $\widehat{\pi}_k(x)$, that indicate the parity of 
           $\Omega(n)$ (sign of $\lambda(n)$) 
           for $n \leq x$ using expansions of more combinatorially motivated Dirichlet series 
           (see Theorem \ref{theorem_GFs_SymmFuncs_SumsOfRecipOfPowsOfPrimes}). 
           We use this result to sum $\sum_{n \leq x} \lambda(n) f(n)$ for particular non-negative arithmetic 
           functions $f$ when $x$ is large. 
\item[\textbf{(4)}] We then turn to the average order 
           asymptotics of the quasi-periodic functions, $g^{-1}(n)$, by estimating this inverse function's 
           limiting asymptotics for large $n \leq x$ as $x \rightarrow \infty$ in 
           Section \ref{Section_InvFunc_PreciseExpsAndAsymptotics}. 
           We eventually use these estimates to prove a substantially unique new lower bound formula 
           for the summatory function $G^{-1}(x) := \sum_{n \leq x} g^{-1}(n)$ along prescribed asymptotically large 
           infinite subsequences (see Theorem \ref{theorem_gInv_GeneralAsymptoticsForms}). 
\item[\textbf{(5)}] We spend some interim time in Section \ref{Section_ProofOfValidityOfAverageOrderLowerBounds} 
           carefully working out a rigorous justification for why the limiting lower bounds we obtain from average 
           order case analysis of certain arithmetic function approximations 
           we define are sufficient to prove the limit supremum corollary below 
           (our primary new significant result established the article). 
\item[\textbf{(6)}] When we return to step \textbf{(2)} 
           with our new lower bounds at hand, and bootstrap, we find ``magic'' in the form of 
           showing the unboundedness of $\frac{|M(x)|}{\sqrt{x}}$ 
           along a very large increasing infinite subsequence 
           of positive natural numbers. What we recover is a quick, and rigorous, proof of 
           Corollary \ref{cor_ThePipeDreamResult_v1} given in 
           Section \ref{subSection_TheCoreResultProof}. 
           
\end{itemize} 

\subsection{Diagramatic flowchart of the proof logic with references to results} 

\subsubsection*{Flowchart schematic diagram of proof components: } 

The next flowchart diagramed below shows how the seemingly disparate components of the proof are organized. 
It also indicates how the separate ``lands'' of material and corresponding sets of requisite results 
forming the connected components to steps $\mathcal{A}$, $\mathcal{B}$ and $\mathcal{C}$ (as viewed below) 
combine to form the next core stages of the proof. 

\tikzstyle{CoreComponent} = [diamond, draw, fill=blue!35, text width=4.5em, text badly centered, 
                             node distance=3cm, inner sep=0.1cm]
\tikzstyle{SubComponent} = [rectangle, draw, fill=blue!19, text width=4.5em, text centered, 
                            rounded corners, minimum height=4em, node distance=3cm]
\tikzstyle{MainResultComponent} = [ellipse, draw, fill=purple!45!pink, text width=4.5em, text centered, 
                            rounded corners, minimum height=4em, node distance=3cm]
\tikzstyle{MainResult} = [cloud, draw, fill=green, text width=4.5em, text centered, 
                            rounded corners, minimum height=4em, node distance=3cm]
\tikzstyle{ComponentConnectionLine} = [draw, -latex]

\begin{center}
\resizebox{0.85\textwidth}{!}{ 
\fbox{
\begin{tikzpicture}[node distance = 2cm, auto, font=\Large\sffamily]
%% : == Nodes: 
\node[CoreComponent] (A)  {Step $\mathcal{A}$}; 
\node[SubComponent, left of=A]  (A2) {A.2}; 
\node[right of=A] (CenterDiagram) {            };
\node[CoreComponent, right of=CenterDiagram] (B)  {Step $\mathcal{B}$}; 
\node[SubComponent, right of=B]  (B2) {B.2}; 
%\node[SubComponent, below of=B2]  (B3) {B.3}; 
\node[MainResultComponent, below of=CenterDiagram] (C)  {Step $\mathcal{D}$}; 
\node[MainResult, below of=C] (D)  {\underline{Step $\mathcal{E}$!}}; 
\node[CoreComponent, left of=C, left of=D] (AvgOrderProofs) {Step $\mathcal{C}$};
%% : == Arrows:
\path[ComponentConnectionLine, dashed, style={<->}] (A) -- (A2);
%\path[ComponentConnectionLine, dashed, style={<->}] (A2) -- (C);
\path[ComponentConnectionLine] (AvgOrderProofs) -- (D);
\path[ComponentConnectionLine, dashed, style={<->}] (B) -- (B2);
%\path[ComponentConnectionLine, dashed, style={<->}] (B2) -- (C);
\path[ComponentConnectionLine] (A) -- (C);
\path[ComponentConnectionLine] (B) -- (C);
\path[ComponentConnectionLine] (C) -- (D);
\end{tikzpicture} 
}
}
\end{center}

\subsubsection*{Key to the diagram stages: } 
\begin{itemize}[noitemsep,topsep=0pt,leftmargin=0.95in]

\item[$\blacktriangleright$ \textbf{Step A:}] \textit{Citations and re-statements of existing theorems proved elsewhere. }
     %E.g., statements of non-trivial theorems and key results we need that are proved in the references. 
     \begin{itemize}[noitemsep,topsep=0pt,leftmargin=0.35in] 
     \item[\textbf{A.A:}] Key results and constructions: 
          \begin{itemize}[noitemsep,topsep=0pt,leftmargin=0.25in]
          \item[--] \small{Theorem \ref{theorem_HatPi_ExtInTermsOfGz}} 
          %\item[--] \small{Theorem \ref{theorem_MV_Thm7.20-init_stmt}} 
          \item[--] \small{Corollary \ref{theorem_MV_Thm7.20}} 
          \item[--] \small{The results, lemmas, and facts cited in Section \ref{subSection_OtherFactsAndResults}}
          \end{itemize} 
     \item[\textbf{A.2:}] Lower bounds on the Abel summation based formula for $G^{-1}(x)$: 
          \begin{itemize}[noitemsep,topsep=0pt,leftmargin=0.25in]
          \item[--] \small{Theorem \ref{theorem_GFs_SymmFuncs_SumsOfRecipOfPowsOfPrimes} 
                    (on page \pageref{proofOf_theorem_GFs_SymmFuncs_SumsOfRecipOfPowsOfPrimes})} 
          \item[--] \small{Proposition \ref{cor_PartialSumsOfReciprocalsOfPrimePowers}} 
          \item[--] \small{Theorem \ref{theorem_gInv_GeneralAsymptoticsForms}} 
          %\item[--] \small{Lemma \ref{lemma_CLT_and_AbelSummation}} 
          %\item[--] \small{Lemma \ref{lemma_lowerBoundsOnLambdaFuncParitySummFuncs}} 
          \end{itemize} 
     \end{itemize} 
\item[$\blacktriangleright$ \textbf{Step B:}] \textit{Constructions of an exact formula for $M(x)$. } 
     %The exact formula we prove 
     %uses special arithmetic functions with particularly ``nice'' properties and bounds. 
     \begin{itemize}[noitemsep,topsep=0pt,leftmargin=0.35in] 
     \item[\textbf{B.B:}] Key results and constructions: 
          \begin{itemize}[noitemsep,topsep=0pt,leftmargin=0.25in]
          \item[--] \small{Corollary \ref{cor_Mx_gInvnPixk_formula}} (follows from 
                    Theorem \ref{theorem_SummatoryFuncsOfDirCvls} 
                    proved on page \pageref{proofOf_theorem_SummatoryFuncsOfDirCvls}) 
          %\item[--] \small{Conjecture \ref{lemma_gInv_MxExample} (to a lesser expository only extent)} 
          \item[--] \small{Proposition \ref{prop_SignageDirInvsOfPosBddArithmeticFuncs_v1}} 
          \end{itemize} 
     \item[\textbf{B.2:}] Asymptotics for the component functions $g^{-1}(n)$ and $G^{-1}(x)$: 
          \begin{itemize}[noitemsep,topsep=0pt,leftmargin=0.25in]
          \item[--] \small{Theorem \ref{theorem_Ckn_GeneralAsymptoticsForms} 
                    (on page \pageref{proofOf_theorem_Ckn_GeneralAsymptoticsForms})} 
          \item[--] \small{Lemma \ref{lemma_AnExactFormulaFor_gInvByMobiusInv_v1}} 
          \end{itemize} 
     %\item[\textbf{B.3}] Simplifying formulas for $g^{-1}(n)$ and $G^{-1}(x)$: 
     %     \begin{itemize}[noitemsep,topsep=0pt]
     %     \item[--] \small{Corollary \ref{cor_ASemiForm_ForGInvx_v1}} 
     %     \end{itemize} 
     \end{itemize} 
\item[$\blacktriangleright$ \textbf{Step C:}] \textit{A justification for why lower bounds obtained on average suffice. }
     \begin{itemize}[noitemsep,topsep=0pt,leftmargin=0.35in]
     \item[--] \small{Theorem \ref{theorem_CondAvgOrderGInvxSummatoryFunc_v1} 
               (proved on page \pageref{proofOf_theorem_CondAvgOrderGInvxSummatoryFunc_v1})} 
     \item[--] \small{The lemmas and necessary results we use to build up to a proof that the 
                      hypotheses needed to apply the conclusion of 
                      Theorem \ref{theorem_CondAvgOrderGInvxSummatoryFunc_v1} are 
                      regularly attained for all large $x$ given in 
                      Section \ref{subSection_ProvingTheNecessaryHyps_ThmCondAvgOrderGInvxSummatoryFunc_v1}. } 
     \end{itemize} 
\item[$\blacktriangleright$ \textbf{Step D:}] \textit{Interpreting the exact formula for $M(x)$. } 
     %Key interpretations used in 
     %formulating the lower bounds based on the re-phrased integral formula. 
     \begin{itemize}[noitemsep,topsep=0pt,leftmargin=0.35in]
     \item[--] \small{Proposition \ref{prop_Mx_SBP_IntegralFormula}} 
     \item[--] \small{Theorem \ref{theorem_gInv_GeneralAsymptoticsForms}} 
     \end{itemize} 
\item[$\blacktriangleright$ \textbf{Step E:}] \textit{The Holy Grail. } 
     Proving that 
     $\frac{|M(x)|}{\sqrt{x}}$ grows without bound in the limit supremum sense. 
     \begin{itemize}[noitemsep,topsep=0pt,leftmargin=0.35in]
     \item[--] \small{Corollary \ref{cor_ThePipeDreamResult_v1} (on page \pageref{proofOf_cor_ThePipeDreamResult_v1})} 
     \end{itemize} 

\end{itemize} 

\newpage 
\section{An introduction to our new methodology: A concrete approach to bounding $M(x)$ from below} 

\subsection{Summatory functions over Dirichlet convolutions of arithmetic functions} 

\begin{theorem}[Summatory functions of Dirichlet convolutions] 
\label{theorem_SummatoryFuncsOfDirCvls} 
Let $f,g: \mathbb{Z}^{+} \rightarrow \mathbb{C}$ be any arithmetic functions such that $f(1) \neq 0$. 
Suppose that $F(x) := \sum_{n \leq x} f(n)$ and $H(x) := \sum_{n \leq x} h(n)$ denote the summatory 
functions of $f,g$, respectively, and that $F^{-1}(x)$ denotes the summatory function of the 
Dirichlet inverse $f^{-1}(n)$ of $f$. Then, letting the counting function $\pi_{f \ast h}(x)$ be defined 
as in the first equation below, we have the following equivalent expressions for the 
summatory function of $f \ast h$ for integers $x \geq 1$: 
\begin{align*} 
\pi_{f \ast h}(x) & = \sum_{n \leq x} \sum_{d|n} f(d) h(n/d) \\ 
     & = \sum_{d \leq x} f(d) H\left(\Floor{x}{d}\right) \\ 
     & = \sum_{k=1}^{x} H(k) \left[F\left(\Floor{x}{k}\right) - 
     F\left(\Floor{x}{k+1}\right)\right]. 
\end{align*} 
Moreover, we can invert the linear system determining the coefficients of $H(k)$ for $1 \leq k \leq x$ 
naturally to express $H(x)$ as a linear combination of the original left-hand-side 
summatory function as follows:
\begin{align*} 
H(x) & = \sum_{j=1}^{x} \pi_{f \ast h}(j) \left[F^{-1}\left(\Floor{x}{j}\right) - 
     F^{-1}\left(\Floor{x}{j+1}\right)\right] \\ 
     & = \sum_{n=1}^{x} f^{-1}(n) \pi_{f \ast h}\left(\Floor{x}{n}\right). 
\end{align*} 
\end{theorem} 

\begin{cor}[Convolutions Arising From M\"obius Inversion] 
\label{cor_CvlGAstMu} 
Suppose that $g$ is an arithmetic function with $g(1) \neq 0$. Define the summatory function of 
the convolution of $g$ with $\mu$ by $\widetilde{G}(x) := \sum_{n \leq x} (g \ast \mu)(n)$. 
Then the Mertens function equals 
\[
M(x) = \sum_{k=1}^{x} \left(\sum_{j=\floor{\frac{x}{k+1}}+1}^{\floor{\frac{x}{k}}} g^{-1}(j)\right) 
     \widetilde{G}(k), \forall x \geq 1. 
\]
\end{cor} 

\begin{cor}[A motivating special case] 
\label{cor_Mx_gInvnPixk_formula} 
We have exactly that for all $x \geq 1$ 
\begin{equation} 
\label{eqn_Mx_gInvnPixk_formula} 
M(x) = \sum_{k=1}^{x} (\omega+1)^{-1}(k) \left[\pi\left(\Floor{x}{k}\right) + 1\right]. 
\end{equation} 
\end{cor} 

\subsection{Fixing an exact expression for $M(x)$ in terms of strongly additive functions} 
\label{example_InvertingARecRelForMx_Intro}

From this point on, we fix the notation for the Dirichlet invertible function $g(n) := \omega(n) + 1$ and denote its 
inverse with respect to Dirichlet convolution by $g^{-1}(n) = (\omega+1)^{-1}(n)$. 
We can compute the first few terms for the
Dirichlet inverse sequence of the arithmetic function 
$g(n) := \omega(n) + 1$ from 
Corollary \ref{cor_Mx_gInvnPixk_formula} 
exactly for the first few sequence values as 
(see Table \ref{table_conjecture_Mertens_ginvSeq_approx_values} of the appendix section) 
\[
\{g^{-1}(n)\}_{n \geq 1} = \{1, -2, -2, 2, -2, 5, -2, -2, 2, 5, -2, -7, -2, 5, 5, 2, -2, -7, -2, 
     -7, 5, 5, -2, 9, \ldots \}. 
\] 
The sign of these terms is given by $\operatorname{sgn}(g^{-1}(n)) = \frac{g^{-1}(n)}{|g^{-1}(n)|} = \lambda(n)$ 
(see Proposition \ref{prop_SignageDirInvsOfPosBddArithmeticFuncs_v1}). 
This useful property is inherited from the distinctly 
additive nature of the component function $\omega(n)$. 

Consider the following motivating conjecture: 
\NBRef{A01-2020-04-26}

\begin{conjecture}
\label{lemma_gInv_MxExample} 
We have the following properties characterizing the 
Dirichlet inverse function $g^{-1}(n)$: 
\begin{itemize} 

\item[(A)] $g^{-1}(1) = 1$; 
\item[(B)] For all $n \geq 1$, $\operatorname{sgn}(g^{-1}(n)) = \lambda(n)$; 
\item[(C)] For all squarefree integers $n \geq 1$, we have that 
     \[
     |g^{-1}(n)| = \sum_{m=0}^{\omega(n)} \binom{\omega(n)}{m} \cdot m!. 
     \]
\end{itemize} 
\end{conjecture} 

We illustrate parts (B)--(C) of the conjecture more clearly using 
Table \ref{table_conjecture_Mertens_ginvSeq_approx_values} given starting on 
page \pageref{table_conjecture_Mertens_ginvSeq_approx_values}. 
The realization that the beautiful and remarkably simple combinatorial form of property (C) 
in Conjecture \ref{lemma_gInv_MxExample} holds for all squarefree $n \geq 1$ 
motivates our pursuit of formulas for the inverse functions $g^{-1}(n)$ \footnote{ 
     A proof of this property is not difficult to give using 
     Lemma \ref{lemma_AnExactFormulaFor_gInvByMobiusInv_v1} 
     stated on page \pageref{lemma_AnExactFormulaFor_gInvByMobiusInv_v1}. 
}. 

%\begin{remark}[Comparison to formative methods for bounding $M(x)$]
%Note that since the DGF of $\omega(n)$ is given by 
%$D_{\omega}(s) = P(s) \zeta(s)$ where $P(s)$ is the \emph{prime zeta function}, we do have a 
%Dirichlet series for the inverse functions to invert coefficient-wise using more classical 
%contour integral methods\footnote{
%E.g., using contour integration or the following integral formula for Dirichlet series 
%inversion \cite[\S 11]{APOSTOLANUMT}: 
%\[
%f(n) = \lim_{T \rightarrow \infty} \frac{1}{2T} \int_{-T}^{T} 
%     \frac{n^{\sigma+\imath t}}{\zeta(\sigma+\imath t)(P(\sigma+\imath t) + 1)}, \sigma > 1. 
%\]
%Fr\"oberg has also previously done some preliminary investigation as to the properties of the 
%inversion to find the coefficients of $(1+P(s))^{-1}$ in \cite{FROBERG-1968}. 
%}. 
%However, the uniqueness to our new methods is that our new approach does not rely on typical constructions for 
%bounding $M(x)$ based on estimates of the non-trivial zeros of the Riemann zeta function that have so far 
%been employed to bound the Mertens function from above. 
%That is, we will instead take a more combinatorial tack to investigating bounds on this inverse function 
%sequence in the coming sections. By Corollary \ref{cor_Mx_gInvnPixk_formula}, 
%once we have established bounds on this $g^{-1}(n)$ and its summatory function, we will be able to 
%formulate new lower bounds (in the limit supremum sense) on $M(x)$. 
%\end{remark} 

For natural numbers $n \geq 1, k \geq 0$, let 
\begin{align*} 
C_k(n) := \begin{cases} 
     \varepsilon(n) = \delta_{n,1}, & \text{ if $k = 0$; } \\ 
     \sum\limits_{d|n} \omega(d) C_{k-1}(n/d), & \text{ if $k \geq 1$. } 
     \end{cases} 
\end{align*} 
We have limiting asymptotics on these functions in terms of $n$ and $k$ within a fixed range 
depending on $n$ given by the following theorem: 

\begin{theorem}[Asymptotics for the functions $C_k(n)$] 
\label{theorem_Ckn_GeneralAsymptoticsForms} 
For $k := 0$, we have by definition that $C_0(n) = \delta_{n,1}$. 
For all sufficiently large $n > 1$ and any fixed $1 \leq k \leq \Omega(n)$ 
taken independently of $n$, 
we obtain that the dominant asymptotic term for $C_k(n)$ is given uniformly by 
\[
\mathbb{E}[C_k(n)] \gg (\log\log n)^{2k-1}, \mathrm{\ as\ }n \rightarrow \infty. 
\]
\end{theorem} 

Since we have that 
\begin{equation} 
\label{eqn_AnExactFormulaFor_gInvByMobiusInv_v1} 
(g^{-1} \ast 1)(n) = \lambda(n) \cdot C_{\Omega(n)}(n), \forall n \geq 1, 
\end{equation} 
M\"{o}bius inversion provides us with an exact divisor sum based expression for $g^{-1}(n)$ 
(see Lemma \ref{lemma_AnExactFormulaFor_gInvByMobiusInv_v1}). 
In light of the fact that (see Proposition \ref{prop_Mx_SBP_IntegralFormula}) 
\[
M(x) \sim G^{-1}(x) - \sum_{k=1}^{x/2} G^{-1}(k) \cdot \frac{x}{k^2 \log(x/k)}, 
\]
the formula in \eqref{eqn_AnExactFormulaFor_gInvByMobiusInv_v1} implies that we can establish 
new finite \emph{lower bounds} on $M(x)$ along large infinite subsequences 
by appropriate estimates of the summatory function $G^{-1}(x)$ 
(see Section \ref{Section_InvFunc_PreciseExpsAndAsymptotics}). 

\subsection{Uniform asymptotics from enumerative counting DGFs in Mongomery and Vaughan} 

Our inspiration for the new bounds found in the last sections of this article allows us to sum 
non-negative arithmetic functions weighted by the Liouville lambda function, 
$\lambda(n) = (-1)^{\Omega(n)}$. 
We utilize a somewhat more general 
hybrid generating function and enumerative DGF method 
under which we are able to recover ``good enough'' asymptotics about the summatory functions that 
encapsulate the parity of $\lambda(n)$ through the summatory tally functions $\widehat{\pi}_k(x)$. 
The precise statement of the theorem that we transform to state these new bounds is provided next in 
Theorem \ref{theorem_HatPi_ExtInTermsOfGz}. 

\begin{theorem}[Montgomery and Vaughan]
\label{theorem_HatPi_ExtInTermsOfGz} 
Recall that we have defined 
$$\widehat{\pi}_k(x) := \#\{n \leq x: \Omega(n)=k\}.$$ 
For $R < 2$ we have that 
\[
\widehat{\pi}_k(x) = \mathcal{G}\left(\frac{k-1}{\log\log x}\right) \frac{x}{\log x} 
     \frac{(\log\log x)^{k-1}}{(k-1)!} \left(1 + O_R\left(\frac{k}{(\log\log x)^2}\right)\right),  
\]
uniformly for $1 \leq k \leq R \log\log x$ where 
\[
\mathcal{G}(z) := \frac{1}{\Gamma(z+1)} \times 
     \prod_p \left(1-\frac{z}{p}\right)^{-1} \left(1-\frac{1}{p}\right)^z, z \geq 0. 
\]
\end{theorem} 

The next theorem, proved carefully in Section \ref{Section_MVCh7_GzBounds}, 
is the primary starting point for our new asymptotic lower bounds. 
The proof of this result is combinatorially motivated in so much as it interprets a key 
infinite product factor of $\mathcal{G}(z)$ defined in 
Theorem \ref{theorem_HatPi_ExtInTermsOfGz} 
as corresponding to an ordinary generating function of certain symmetric polynomials 
involving reciprocals of the primes. 

\begin{theorem} 
\label{theorem_GFs_SymmFuncs_SumsOfRecipOfPowsOfPrimes} 
\label{cor_BoundsOnGz_FromMVBook_initial_stmt_v1} 
Suppose that $0 \leq z < 1$ is real-valued. 
We obtain lower bounds of the following form on the function 
$\mathcal{G}(z)$ from Theorem \ref{theorem_HatPi_ExtInTermsOfGz} 
for $A_0 > 0$ an absolute constant and for 
$C_0(z)$ a strictly linear function only in $z$: 
\begin{align*} 
\mathcal{G}(z) \geq A_0 \cdot (1-z)^{3} \cdot C_0(z)^{z}. 
\end{align*} 
It suffices to take the components of the bound given in the previous equation as 
\begin{align*}
A_0 & = \frac{2^{9/16} \exp\left(-\frac{55}{4} \log^2(2)\right)}{ 
     (3e\log 2)^3 \cdot \Gamma\left(\frac{5}{2}\right)} \approx 3.81296 \times 10^{-6} \\ 
C_0(z) & = \frac{4(1-z)}{3e \log 2}. 
\end{align*} 
In particular, with $0 \leq z < 1$ and 
$z \equiv z(k, x) = \frac{k-1}{\log\log x}$, 
or equivalently with $1 \leq k \leq \log\log x$ in Theorem \ref{theorem_HatPi_ExtInTermsOfGz}, 
we have that 
\[
\widehat{\pi}_k(x) \SuccSim \frac{A_0 \cdot x}{\log x \cdot (\log\log x)^4 \cdot (k-1)!} \cdot 
     \left(\frac{4}{3e\log 2}\right)^{\frac{k}{\log\log x}}.
\]
\end{theorem} 

\subsection{Rigorous proofs justifying that so-called average case lower bounds still recover 
            meaningful asymptotics when viewed as bounds that hold more globally} 
\label{subSection_Intro_RigorToTheAverageCaseEstimates} 

\subsubsection{An average-to-global phenomenon for the average case analysis of our new lower bounds} 

\begin{theorem} 
\label{theorem_CondAvgOrderGInvxSummatoryFunc_v1} 
\begin{subequations} 
Let the summatory function $G_E^{-1}(x)$ be defined for $x \geq 1$ by \footnote{ 
     The subscript of $E$ on the function $G_E^{-1}(x)$ is purely for notation and does not correspond to 
     a formal parameter or any implicit dependence on $E$ in the function formula. 
     %In fact, since we are trying to eventually bound $G^{-1}(x)$ from below by this function using the 
     %expectation formulas in 
     %Section \ref{subSection_ProvingTheNecessaryHyps_ThmCondAvgOrderGInvxSummatoryFunc_v1}, 
     %the notation $E$ subscripted on this function 
     %can be viewed in some ways as denoting our 
     %expected lower bounding function -- even 
     %though we have to go to significant lengths to show this property 
     %of expectation holds later in the article. 
}
\begin{equation} 
\label{eqn_GEInvxSummatoryFuncDef_v1} 
G_E^{-1}(x) := \sum_{n \leq \log x} \lambda(n) \times \sum_{\substack{d|n \\ d > e}} 
     (\log d) (\log\log d). 
\end{equation} 
Suppose that $B, C \in (0, 1)$ denote some respectively minimally and maximally defined absolute constants 
such that for a bounded constant $Y \geq 0$, we have that as $x \rightarrow \infty$
\begin{equation} 
\label{eqn_theorem_CondAvgOrderGInvxSummatoryFunc_v1_stmt_tag_v2} 
B + o(1) \leq \frac{1}{x} \cdot \#\left\{n \leq x: |G^{-1}(n)| - |G_E^{-1}(n)| \leq Y\right\} \leq 
     C + o(1). 
\end{equation} 
That is, if for a bounded constant $Y \geq 0$ we have that the set 
\[
\left\{n \leq x: |G^{-1}(n)| - |G_E^{-1}(n)| \leq Y\right\}, 
\]
has bounded asymptotic density in $(0, 1)$ such that the condition in 
\eqref{eqn_theorem_CondAvgOrderGInvxSummatoryFunc_v1_stmt_tag_v2} 
holds for all large $x$, then we take 
\begin{align*} 
B & := \liminf_{x \rightarrow \infty} \frac{1}{x} \cdot \#\left\{n \leq x: |G^{-1}(n)| - |G_E^{-1}(n)| \leq Y\right\} \in (0, 1) \\ 
C & := \limsup_{x \rightarrow \infty} \frac{1}{x} \cdot \#\left\{n \leq x: |G^{-1}(n)| - |G_E^{-1}(n)| \leq Y\right\} \in (0, 1). 
\end{align*} 
If such constants $B, C \in (0, 1)$ exist, then there is some $\varepsilon \in (0, 1)$ (depending on $B,C$) with 
$0 < B - \varepsilon, C+\varepsilon < 1$ such that 
for all sufficiently large $x$ we have at least one point 
$x_0 \in [(B - \varepsilon) x, (C + \varepsilon) x]$ such that 
\[
|G^{-1}(x_0)| \geq \left\lvert G_E^{-1}(x_0) \right\rvert + Y. 
\]
\end{subequations} 
\end{theorem} 
We prove Theorem \ref{theorem_CondAvgOrderGInvxSummatoryFunc_v1}, and 
rigorously justify that its hypotheses are in fact regularly attainable for all large $x$, in 
Section \ref{Section_ProofOfValidityOfAverageOrderLowerBounds}.  
This result combines to allow us to take lower bounds based on average case estimates of 
certain arithmetic functions we have defined to approximate $g^{-1}(n)$ and still recover 
an infinite subsequence along which we can witness the key unboundedness property 
of $|M(x)| / \sqrt{x}$ stated in 
Corollary \ref{cor_ThePipeDreamResult_v1} below. 

\subsubsection{Intuition for average case asymptotics leading to estimates near any large $x$} 

There does not appear to be an easy, nor subtle 
direct recursion between the distinct $g^{-1}$ values, except through auxiliary function sequences. 
However, the distribution of distinct sets of prime exponents is fairly regular with 
$\omega(n)$ and $\Omega(n)$ playing a crucial role in the repitition of common values of 
$g^{-1}(n)$. 
The following observation is suggestive of the quasi-periodicity at play 
with the distinct values of $g^{-1}(n)$ distributed over $n \geq 2$: 

\begin{heuristic}[Symmetry in $g^{-1}(n)$ in the exponents in the prime factorization of $n$] 
Suppose that $n_1, n_2 \geq 2$ are such that their factorizations into distinct primes are 
given by $n_1 = p_1^{\alpha_1} \cdots p_r^{\alpha_r}$ and $n_2 = q_1^{\beta_1} \cdots q_r^{\beta_r}$. 
If $\{\alpha_1, \ldots, \alpha_r\} \equiv \{\beta_1, \ldots, \beta_r\}$ as multisets of prime exponents, 
then $g^{-1}(n_1) = g^{-1}(n_2)$. For example, $g^{-1}$ has the same values on the squarefree integers 
with exactly two, three, and so on prime factors 
(see Table \ref{table_conjecture_Mertens_ginvSeq_approx_values} starting on page 
\pageref{table_conjecture_Mertens_ginvSeq_approx_values}). 
\end{heuristic} 

The next remark then makes clear what our intuiton ought suggest about the relation of 
the actual function values to the average case expectation of $g^{-1}(n)$ for $n \leq x$ when 
$x$ is large. 

\begin{remark}[Essential components of the proof]
Given that we have chosen to work with a representation for $M(x)$ that depends critically on 
the distribution of the values of the additive functions, $\omega(n)$ and $\Omega(n)$, there is 
substantial intuition involved \'{a} priori that suggests our sums over these functions ought 
behave regularly on average. 
Thus when it comes to recovering globally regular behavior from an 
average case analysis of bounds of our new arithmetic functions from below, 
the choice in stating \eqref{eqn_Mx_gInvnPixk_formula} as it depends on the 
canonical additive function examples we have cited is 
\emph{absolutely essential} to the success of our proof.  

Stated precisely, when we define the function 
$\Phi(z) := \frac{1}{\sqrt{2\pi}} \int_{-\infty}^{z} e^{-t^2/2} dt$,  
for any real bounding parameter $z \in (-\infty, +\infty)$, we have that 
\cite[\S 1.7]{IWANIEC-KOWALSKI} 
\[
\#\left\{n \leq x: \frac{\omega(n) - \log\log x}{\sqrt{\log\log x}} \leq z\right\} = 
     \Phi(z) \cdot x + o(1), 
\]
and that uniformly for $-Z \leq z \leq Z$ with respect to any $Z > 0$ \cite[\S 7.4]{MV} 
(\cf Theorem \ref{theorem_MV_Thm7.20-init_stmt})
\begin{align*} 
\#\left\{3 \leq n \leq x: \frac{\Omega(n) - \log\log n}{\sqrt{\log\log n}} \leq z\right\} & = 
     \Phi(z) \cdot x + O_Z\left(\frac{x}{\sqrt{\log\log x}}\right). 
\end{align*} 
That is, notably, since we have an Erd\"os-Kac like theorem for each of 
$\omega(n)$ and $\Omega(n)$ stated as in the above two equations, 
when the bounding parameter is set to $z := 0$, we provably 
should expect these sums involving these canonical additive functions and their 
``nice'' properties to tend towards the 
asymptotic average case behavior infinitely often, and predictably near any large $x$ 
as in Theorem \ref{theorem_CondAvgOrderGInvxSummatoryFunc_v1}. 
\end{remark} 

\subsection{Cracking the classical unboundedness barrier} 

In Section \ref{Section_KeyApplications}, 
we are able to state what forms the culmination of the results 
we carefully build up to in the proofs established in prior sections of the article. 
What we eventually obtain at the conclusion of the section 
is the following important summary corollary that resolves the classical question of the 
unboundedness of the scaled function Mertens function 
$|M(x)| / \sqrt{x}$ in the limit supremum sense: 

\begin{cor}[Unboundedness of the the Mertens function scaled by $\sqrt{x}$] 
\label{cor_ThePipeDreamResult_v1} 
Define the infinite increasing subsequence, 
$\{x_{0,n}\}_{n \geq 1}$, by $x_{0,n} := e^{2e^{e^{e^{2n}}}}$. 
We have that for all sufficiently large 
$y \gg \ceiling{x_{0,1}}+1$ the following bound holds: 
\begin{align*} 
\frac{|M(x_{0,y})|}{\sqrt{x_{0,y}}} & \SuccSim 
     \frac{2 \cdot C_{\ell, 1} \cdot (\log\log \sqrt{x_{0,y}})^{4} \sqrt{\log\log\log \sqrt{x_{0,y}}}}{ 
     (\log\log\log\log \sqrt{x_{0,y}})^{\frac{5}{2}}}, \mathrm{\ as\ } y \rightarrow \infty. 
\end{align*} 
The constant $C_{\ell,1} > 0$ in the previous equation can be taken to be 
\[
C_{\ell,1} := 
     \frac{256 \cdot 2^{1/8}}{59049 \cdot \pi^2 e^8 \log^8(2)} 
     \exp\left(-\frac{55}{2} \log^2(2)\right) 
     \approx 5.51187 \times 10^{-12}. 
\]
\end{cor} 

This is all to say that in establishing the rigorous proof of 
Corollary \ref{cor_ThePipeDreamResult_v1} 
based on our new methods, we not only show that 
\[
\limsup_{x \rightarrow \infty} \frac{|M(x)|}{\sqrt{x}} = +\infty, 
\]
but also set a minimal rate (along a large infinite subsequence) at which this form of the 
scaled Mertens function grows without bound. 

%For technical reasons found in the proof of 
%Theorem \ref{theorem_gInv_GeneralAsymptoticsForms}, 
%the new primary result used to show 
%Corollary \ref{cor_ThePipeDreamResult_v1}, 
%it is more difficult to exactly define a secondary subsequence, $\{\hat{x}_{0,n}\}_{n \geq 1}$, 
%around which we can witness the unboundedness of $M(x) / \sqrt{x}$ towards $-\infty$. 
%In this instance, we would require the less exponentially well defined 
%condition on the sequence that 
%$$\floor{\log\log\log n} \equiv \floor{\log\log\log\log n} \pmod{2}.$$ 

\newpage 
\section{Preliminary proofs of lemmas and new results} 
\label{Section_PrelimProofs_Config} 

The purpose of this section is to provide proofs and statements 
of elementary and otherwise well established facts and results. In particular, the proof of 
Theorem \ref{theorem_SummatoryFuncsOfDirCvls} allows us to easily justify the formula in 
\eqref{eqn_Mx_gInvnPixk_formula}. 
This formula is the crucial formulation that constiutes an exact expression for $M(x)$. 
The indispensible property inherent to the arithmetic functions, $\omega(n)$ and $g^{-1}(n)$, 
that are used to state the formula are strong additivity, which leads to the sign of the inverse function 
$g^{-1}(n)$ being given by $\lambda(n)$. Hence the summatory function of $g^{-1}(n)$ is 
intimately tied to the exact limiting 
distribution of the values of $\Omega(n)$. 

\subsection{Establishing the summatory function properties and inversion identities} 

We will prove Theorem \ref{theorem_SummatoryFuncsOfDirCvls}, a crucial component to our new more combinatorial 
formulations used to bound $M(x)$ in later sections, using matrix methods before moving on. 
Related results on summations of Dirichlet convolutions appear in 
\cite[\S 2.14; \S 3.10; \S 3.12; \cf \S 4.9, p.\ 95]{APOSTOLANUMT}. 

\begin{proof}[Proof of Theorem \ref{theorem_SummatoryFuncsOfDirCvls}] 
\label{proofOf_theorem_SummatoryFuncsOfDirCvls} 
Let $h,g$ be arithmetic functions where $g(1) \neq 0$ 
necessarily has a Dirichlet inverse. Denote the summatory functions of $h$ and $g$, 
respectively, by $H(x) = \sum_{n \leq x} h(n)$ and $G(x) = \sum_{n \leq x} g(n)$. 
We define $\pi_{g \ast h}(x)$ to be the summatory function of the 
Dirichlet convolution of $g$ with $h$: $g \ast h$. 
Then we can easily see that the following expansions hold: 
\begin{align*} 
\pi_{g \ast h}(x) & := \sum_{n=1}^{x} \sum_{d|n} g(n) h(n/d) = \sum_{d=1}^x g(d) H\left(\floor{\frac{x}{d}}\right) \\ 
     & = \sum_{i=1}^x \left[G\left(\floor{\frac{x}{i}}\right) - G\left(\floor{\frac{x}{i+1}}\right)\right] H(i). 
\end{align*} 
We form the matrix of coefficients associated with this system for $H(x)$, and proceed to invert it to express an 
exact solution for this function over all $x \geq 1$. Let the ordinary (initial, non-inverse) matrix entries be denoted by 
\[
g_{x,j} := G\left(\floor{\frac{x}{j}}\right) - G\left(\floor{\frac{x}{j+1}}\right) \equiv G_{x,j} - G_{x,j+1}. 
\]
The matrix we must invert in this problem is lower triangular, with ones on its diagonals -- and hence is invertible. 
Moreover, if we let $\hat{G} := (G_{x,j})$, then this matrix is 
expressable by an invertible shift operation as 
\[
(g_{x,j}) = \hat{G} (I - U^{T}); \qquad U = (\delta_{i,j+1}), (I - U^T)^{-1} = (\Iverson{j \leq i}). 
\]
Here, $U$ is the $N \times N$ matrix whose $(i,j)^{th}$ entries are defined by 
$(U)_{i,j} = \delta_{i+1,j}$. 

It is a useful fact that if we take successive differences of floor functions, we get non-zero behavior at divisors: 
\[
G\left(\floor{\frac{x}{j}}\right) - G\left(\floor{\frac{x-1}{j}}\right) = 
     \begin{cases} 
     g\left(\frac{x}{j}\right), & \text{ if $j | x$; } \\ 
     0, & \text{ otherwise. } 
     \end{cases}
\]
We use this property to shift the matrix $\hat{G}$, and then invert the result, to obtain a matrix involving the 
Dirichlet inverse of $g$: 
\begin{align*} 
\left[(I-U^{T}) \hat{G}\right]^{-1} & = \left(g\left(\frac{x}{j}\right) \Iverson{j|x}\right)^{-1} = 
     \left(g^{-1}\left(\frac{x}{j}\right) \Iverson{j|x}\right). 
\end{align*} 
Now we can express the inverse of the target matrix $(g_{x,j})$ in terms of these Dirichlet inverse functions 
as follows: 
\begin{align*} 
(g_{x,j}) & = (I-U^{T})^{-1} \left(g\left(\frac{x}{j}\right) \Iverson{j|x}\right) (I-U^{T}) \\ 
(g_{x,j})^{-1} & = (I-U^{T})^{-1} \left(g^{-1}\left(\frac{x}{j}\right) \Iverson{j|x}\right) (I-U^{T}) \\ 
     & = \left(\sum_{k=1}^{\floor{\frac{x}{j}}} g^{-1}(k)\right) (I-U^{T}) \\ 
     & = \left(\sum_{k=1}^{\floor{\frac{x}{j}}} g^{-1}(k) - \sum_{k=1}^{\floor{\frac{x}{j+1}}} g^{-1}(k)\right). 
\end{align*} 
Thus the summatory function $H$ is exactly expressed by the inverse vector product of the form 
\begin{align*} 
H(x) & = \sum_{k=1}^x g_{x,k}^{-1} \cdot \pi_{g \ast h}(k) \\ 
     & = \sum_{k=1}^x \left(\sum_{j=\floor{\frac{x}{k+1}}+1}^{\floor{\frac{x}{k}}} g^{-1}(j)\right) \cdot \pi_{g \ast h}(k). 
     \qedhere
\end{align*} 
\end{proof} 

\subsection{Proving the crucial signedness property of $g^{-1}(n)$} 

Let $\chi_{\mathbb{P}}$ denote the characteristic function of the primes, 
$\varepsilon(n) = \delta_{n,1}$ be the multiplicative identity with respect to Dirichlet convolution, 
and denote by $\omega(n)$ the strongly additive function that counts the number of 
distinct prime factors of $n$. Then we can easily prove that 
\begin{equation}
\label{eqn_AntiqueDivisorSumIdent} 
\chi_{\mathbb{P}} + \varepsilon = (\omega + 1) \ast \mu. 
\end{equation} 
When combined with Corollary \ref{cor_CvlGAstMu}, an immediate consequence of 
Theorem \ref{theorem_SummatoryFuncsOfDirCvls}, 
this convolution identity yields the necessary convolution identity that yields the exact 
formula for $M(x)$ stated in \eqref{eqn_Mx_gInvnPixk_formula} of 
Corollary \ref{cor_Mx_gInvnPixk_formula}. 

The proof of the next proposition is essential to our argument given in later sections. 
We try to keep the argument brief while sketching all relevant details to rigorously justifying the key parts 
to the proof of our claim. 

\begin{prop}[The key signedness property of $g^{-1}(n)$]
\label{prop_SignageDirInvsOfPosBddArithmeticFuncs_v1} 
For the Dirichlet invertible function, $g(n) := \omega(n) + 1$ defined such that $g(1) = 1$, at any 
$n \geq 1$, we have that $\operatorname{sgn}(g^{-1}(n)) = \lambda(n)$. 
The notation for the operation given by 
$\operatorname{sgn}(h(n)) = \frac{h(n)}{|h(n)| + \Iverson{h(n) = 0}} \in \{0, \pm 1\}$ denotes the sign 
of the arithmetic function $h$ at $n$. 
\NBRef{A02-2020-04-26}
\end{prop} 
\begin{proof} 
Recall that $D_f(s) := \sum_{n \geq 1} f(n) n^{-s}$ denotes the Dirichlet generating function (DGF) of any 
arithmetic function $f(n)$ which is convergent for all $s \in \mathbb{C}$ satisfying $\Re(s) > \sigma_f$. 
In particular, recall that $D_1(s) = \zeta(s)$, $D_{\mu}(s) = 1 / \zeta(s)$ and $D_{\omega}(s) = P(s) \zeta(s)$. 
Then by \eqref{eqn_AntiqueDivisorSumIdent} and the known property that the DGF of $f^{-1}(n)$ is 
the reciprocal of the DGF of the original arithmetic function $f$, for all $\Re(s) > 1$ we have 
\begin{align} 
\label{eqn_DGF_of_gInvn} 
D_{(\omega+1)^{-1}}(s) = \frac{1}{(P(s)+1) \zeta(s)}. 
\end{align} 
It follows that $(\omega + 1)^{-1}(n) = (h^{-1} \ast \mu)(n)$ when we take 
$h := \chi_{\mathbb{P}} + 1$. 
We show that $\operatorname{sgn}(h^{-1}) = \lambda$. From this fact, it follows by inspection 
that $\operatorname{sgn}(h^{-1} \ast \mu) = \lambda$. The remainder of the proof fills in the 
precise details needed to make this intuition precise. 

By the standard recurrence relation we used to define the Dirichlet inverse function of any 
arithmetic function $h$ such that $h(1) = 1 \neq 0$, 
we have that 
\begin{equation} 
\label{eqn_proof_tag_hInvn_ExactRecFormula_v1}
h^{-1}(n) = \begin{cases} 
            1, & n = 1; \\ 
            -\sum\limits_{\substack{d|n \\ d>1}} h(d) h^{-1}(n/d), & n \geq 2. 
            \end{cases} 
\end{equation} 
For $n \geq 2$, the summands in \eqref{eqn_proof_tag_hInvn_ExactRecFormula_v1} 
can be simply indexed over the primes $p|n$. This observation yields that we can inductively 
expand these sums into nested divisor sums provided the depth of the sums does not exceed the 
capacity to index summations over the primes dividing $n$. Namely, notice that for $n \geq 2$ 
\begin{align*} 
h^{-1}(n) & = -\sum_{p|n} h^{-1}(n/p), && \text{\ if\ } \Omega(n) \geq 1 \\ 
     & = \sum_{p_1|n} \sum_{p_2|\frac{n}{p_1}} h^{-1}\left(\frac{n}{p_1p_2}\right), && \text{\ if\ } \Omega(n) \geq 2 \\ 
     & = -\sum_{p_1|n} \sum_{p_2|\frac{n}{p_1}} \sum_{p_3|\frac{n}{p_1p_2}} h^{-1}\left(\frac{n}{p_1p_2p_3}\right), 
     && \text{\ if\ } \Omega(n) \geq 3. 
\end{align*} 
Then by induction, again with $h^{-1}(1) = 1$, we obtain by expanding the 
nested divisor sums as above to their maximal depth as 
\[
h^{-1}(n) = \lambda(n) \times \sum_{p_1|n} \sum_{p_2|\frac{n}{p_1}} \times \cdots \times 
     \sum_{p_{\Omega(n)}|\frac{n}{p_1p_2 \cdots p_{\Omega(n)-1}}} 1, n \geq 2. 
\]
If for $n \geq 2$ we write the prime factorization of $n$ as 
$n = p_1^{\alpha_1} p_2^{\alpha_2} \cdots p_{\omega(n)}^{\alpha_{\omega(n)}}$ where the exponents $\alpha_i \geq 1$ are all 
non-zero for $1 \leq i \leq \omega(n)$, we can see that 
\begin{align*} 
h^{-1}(n) & \geq \lambda(n) \times 1 \cdot 2 \cdot 3 \cdots \omega(n) = \lambda(n) \times (\omega(n))!, && n \geq 2 \\ 
h^{-1}(n) & \leq \lambda(n) \times (\omega(n))!^{\max(\alpha_1, \alpha_2, \ldots, \alpha_{\omega(n)})}, && n \geq 2. 
\end{align*} 
In other words, what these bounds show is that for all $n \geq 1$ (with $\lambda(1) = 1$) the following property holds: 
\begin{equation} 
\label{eqn_proof_tag_SignedTimesPosConstantFormOf_hInvn_v2}
\operatorname{sgn}(h^{-1}(n)) = \lambda(n). 
\end{equation}
By \eqref{eqn_proof_tag_SignedTimesPosConstantFormOf_hInvn_v2}, we immediately have bounding constants 
$1 \leq C_{1,n}, C_{2,n} < +\infty$ that exist for each $n \geq 1$ so that 
\begin{equation} 
\label{eqn_proof_tag_hInvMunCvl_UpperLowerBounds_v3} 
C_{1,n} \cdot (\lambda \ast \mu)(n) \leq (h^{-1} \ast \mu)(n) \leq C_{2,n} \cdot (\lambda \ast \mu)(n). 
\end{equation} 
Since both $\lambda,\mu$ are multiplicative, the convolution $\lambda \ast \mu$ is multiplicative. 
We know that the values of 
any multiplicative function are uniquely determined by its action at prime powers. 
So we can compute that for any prime $p$ and non-negative integer exponents $\alpha \geq 1$ that 
\begin{align*} 
(\lambda \ast \mu)(p^{\alpha}) & = \sum_{i=0}^{\alpha} \lambda(p^{\alpha-i}) \mu(p^{i}) \\ 
     & = \lambda(p^{\alpha}) - \lambda(p^{\alpha-1}) \\ 
     & = 
     (-1)^{\Omega(p^{\alpha})} - (-1)^{\Omega(p^{\alpha-1})} = 
     (-1)^{\alpha} - (-1)^{\alpha-1} = 
     2 \lambda(p^{\alpha}). 
\end{align*} 
Then by the multiplicativity of $\lambda(n)$, the previous inequalities derived in 
\eqref{eqn_proof_tag_hInvMunCvl_UpperLowerBounds_v3} are re-stated in the form of 
\[
2 C_{1,n} \cdot \lambda(n) \leq h^{-1}(n) \leq 2 C_{2,n} \cdot \lambda(n). 
\] 
Since the absolute constants $C_{1,n}, C_{2,n}$ are always positive for all $n \geq 1$, 
we clearly recover the signedness of $g^{-1}(n)$ as $\lambda(n)$. 
\end{proof} 

\subsection{Statements of other facts and known limiting asymptotic results} 
\label{subSection_OtherFactsAndResults} 

\begin{theorem}[Mertens theorem]
\label{theorem_Mertens_theorem} 
For all $x \geq 2$ we have that 
\[
P_1(x) := \sum_{p \leq x} \frac{1}{p} = \log\log x + B + o(1), 
\]
where 
$B \approx 0.2614972128476427837554$ 
is an explicitly defined absolute constant \footnote{ 
     Exactly, we have that the \emph{Mertens constant} is defined by 
     \[
     B = \gamma + \sum_{m \geq 2} \frac{\mu(m)}{m} \log\left[\zeta(m)\right], 
     \]
     where $\gamma \approx 0.577215664902$ is Euler's gamma constant. 
}.
\end{theorem} 

\begin{cor}[Mertens theorem product form] 
\label{lemma_Gz_productTermV2} 
We have that for all sufficiently large $x \gg 1$ 
\[
\prod_{p \leq x} \left(1 - \frac{1}{p}\right) = \frac{e^{-B}}{\log x}\left( 
     1 + o(1)\right), 
\]
where the notation for the absolute constant $0 < B < 1$ coincides with the definition of 
Mertens constant from Theorem \ref{theorem_Mertens_theorem}. 
Hence, for any real $0 \leq z < 2$ we obtain that 
\[
\prod_{p \leq x} \left(1 - \frac{1}{p}\right)^{z} = 
     \frac{e^{-Bz}}{(\log x)^{z}} \left(1+o(1)\right)^{z} \sim 
     \frac{e^{-Bz}}{(\log x)^{z}}, \mathrm{\ as\ } x \rightarrow \infty. 
\]
\end{cor} 

Proofs of Theorem \ref{theorem_Mertens_theorem} and 
Corollary \ref{lemma_Gz_productTermV2} are found in 
\cite[\S 22.7; \S 22.8]{HARDYWRIGHT}. 

\begin{facts}[Asymptotics for exponential integrals and incomplete gamma functions] 
\label{facts_ExpIntIncGammaFuncs} 
\begin{subequations}
The following two variants of the \emph{exponential integral function} are defined by 
\cite[\S 8.19]{NISTHB} 
\begin{align*} 
\operatorname{Ei}(x) & := \int_{-x}^{\infty} \frac{e^{-t}}{t} dt, \\ 
E_1(z) & := \int_1^{\infty} \frac{e^{-tz}}{t} dt, \Re(z) \geq 0, 
\end{align*} 
where $\operatorname{Ei}(-kz) = -E_1(kz)$ for real $k, z > 0$. 
We have the following inequalities providing 
quasi-polynomial upper and lower bounds on $E_1(z)$ \footnote{ 
     Indeed, these inequalities are usually stated to provide bounds on 
     $\operatorname{Ei}(z)$ when $z > 0$ is real-valued. 
     That the bounds are also valid for $E_1(z)$ at all sufficiently large 
     $z \gg e$ is a matter of convenience we state for reference in this way here 
     to arrive at the nice formulas we see in 
     Proposition \ref{cor_PartialSumsOfReciprocalsOfPrimePowers} 
     and in the proof of 
     Theorem \ref{theorem_GFs_SymmFuncs_SumsOfRecipOfPowsOfPrimes}. 
}: 
\begin{equation}
1-\frac{3}{4} z \leq E_1(z) - \gamma - \log z \leq 1-\frac{3}{4} z + \frac{11}{36} z^2. 
\end{equation}
A related function is the (upper) \emph{incomplete gamma function} defined by \cite[\S 8.4]{NISTHB} 
\[
\Gamma(s, x) = \int_{x}^{\infty} t^{s-1} e^{-t} dt, \Re(s) > 0. 
\]
We have the following properties of $\Gamma(s, x)$: 
\begin{align} 
\label{eqn_IncompleteGamma_PropA} 
\Gamma(s, x) & = (s-1)! \cdot e^{-x} \times \sum_{k=0}^{s-1} \frac{x^k}{k!}, s \in \mathbb{Z}^{+}, \\ 
\label{eqn_IncompleteGamma_PropB} 
\Gamma(s, x) & \sim x^{s-1} \cdot e^{-x}, |x| \rightarrow +\infty. 
\end{align}
\end{subequations}
\end{facts} 

\newpage 
\section{Components to the asymptotic analysis of lower bounds for 
         sums of arithmetic functions weighted by $\lambda(n)$} 
\label{Section_MVCh7_GzBounds} 

In this section, we re-state a couple of key results proved in \cite[\S 7.4]{MV} that we rely on 
to state and prove Corollary \ref{theorem_MV_Thm7.20} stated below. This corollary is important as it shows 
that (signed) summatory functions over $\widehat{\pi}(x)$ 
capture the dominant asymptotics of the full summatory function formed by taking $1 \leq k \leq \log_2(x)$ when 
we truncate and instead sum only up to the uniform bound of $1 \leq k \leq \log\log x$ guaranteed by applying 
Theorem \ref{theorem_HatPi_ExtInTermsOfGz}. 

We also prove 
Theorem \ref{theorem_GFs_SymmFuncs_SumsOfRecipOfPowsOfPrimes} in this section. 
This key theorem allows us to establish a global minimum we can attain on the function $\mathcal{G}(z)$ from 
Theorem \ref{theorem_HatPi_ExtInTermsOfGz} by truncating the formerly stated infinite 
range of the primes $p$ over which we take a component product in the definition of this function. 
This in turn implies the uniform lower bounds on $\widehat{\pi}_k(x)$ guaranteed by that theorem by 
a straightforward manipulation of inequalities. 

\subsection{Key results proved in the reference by Montgomery and Vaughan} 
\label{subSection_MVPrereqResultStmts} 

What the enumeratively-flavored result of Montgomery and Vaughan 
in Theorem \ref{theorem_HatPi_ExtInTermsOfGz} allows us to do is get a 
``good enough'' lower bound on sums of positive and asymptotically bounded arithmetic functions 
weighted by the Liouville lambda function, $\lambda(n) = (-1)^{\Omega(n)}$. 
For comparison, we already have known, more classical bounds due to Erd\"os (and earlier) that 
we can tightly bound \cite{ERDOS-PRIMEK-FUNC,MV} 
\[
\pi_k(x) = (1 + o(1)) \cdot \frac{x}{\log x} \frac{(\log\log x)^{k-1}}{(k-1)!}. 
\] 
We seek to approximate the right-hand-side of $\mathcal{G}(z)$ by only taking the products of the primes 
$p \leq u$, e.g., indexing the component products only over those primes 
$p \in \left\{2,3,5,\ldots,u\right\}$ for some minimal upper bound $u$ (with respect to $x$) 
as $x \rightarrow \infty$. 

We also state the following theorems reproduced from \cite[\S 7.4]{MV} that handle the relative 
scarcity of the distribution of the $\Omega(n)$ for $n \leq x$ such that 
$\Omega(n) > \log\log x$. 

\begin{theorem}[Bounds on exceptional values of $\Omega(n)$ for large $n$] 
\label{theorem_MV_Thm7.20-init_stmt} 
Let 
\begin{align*} 
A(x, r) & := \#\left\{n \leq x: \Omega(n) \leq r \cdot \log\log x\right\}, \\ 
B(x, r) & := \#\left\{n \leq x: \Omega(n) \geq r \cdot \log\log x\right\}. 
\end{align*} 
If $0 < r \leq 1$ and $x \geq 2$, then 
\[
A(x, r) \ll x (\log x)^{r-1 - r\log r}, \text{ \ as\ } x \rightarrow \infty. 
\]
If $1 \leq r \leq R < 2$ and $x \geq 2$, then 
\[
B(x, r) \ll_R x \cdot (\log x)^{r-1-r \log r}, \text{ \ as\ } x \rightarrow \infty. 
\]
\end{theorem} 

\begin{theorem}[Bounds on exceptional values of $\Omega(n)$ for large $n$, MV 7.21] 
\label{theorem_MV_Thm7.21-init_stmt} 
We have that uniformly 
\[
\#\left\{3 \leq n \leq x: \Omega(n) - \log\log n \leq 0\right\} = 
     \frac{x}{2} + O\left(\frac{x}{\sqrt{\log\log x}}\right). 
\]
\end{theorem} 

\begin{remark} 
The proofs of Theorem \ref{theorem_MV_Thm7.20-init_stmt} and 
Theorem \ref{theorem_MV_Thm7.21-init_stmt} 
are found in Chapter 7 of Montgomery and Vaughan. 
The key interpretation we need is the result stated in the next corollary. 
In the previous theorem, the dependence on $R$, and the necessity of using the 
conditional relation $\ll_R$, serves to denote this $R$ as a 
bounding (maximally limiting) parameter on the 
input $r \in (1, R)$ to the functions $B(x, r)$. 
The precise way in which the bound 
stated in this cited theorem depends on this bounded, 
indeterminate paramater $R$ can be reviewed for reference in the proof 
algebra and relations cited in the reference \cite[\S 7]{MV}. 
The role of the parameter $R$ involved in stating the previous theorem 
is notably important as a scalar factor the upper bound on $k \leq R\log\log x$ in 
Theorem \ref{theorem_HatPi_ExtInTermsOfGz} up to which 
we obtain the valid uniform bounds in $x$ on the asymptotics for 
$\widehat{\pi}_k(x)$. 

We have a discrepancy to work out in so much as we 
can only form summatory functions over the $\widehat{\pi}_k(x)$ for 
$1 \leq k \leq R\log\log x$ using the desirable, or ``nice'', asymptotic formulas
guaranteed by Theorem \ref{theorem_HatPi_ExtInTermsOfGz}, even though we can actually 
have contributions from values distributed throughout the range $1 \leq \Omega(n) \leq \log_2(n)$. 
It is then crucial that we can show that the dominant growth of the asymptotic formulas we obtain 
for these summatory functions is captured by summing only over $k$ in the truncated range 
where the uniform formulas hold. In particular, we will require a proof 
that we can discard the terms in the full summatory function 
asymptotic formulas as negligible (up to at most a constant) 
for large $x$ when they happen to fall in the 
limiting exceptional range of $\Omega(n) > R\log\log x$ for $n \leq x$. 
\end{remark} 

\begin{cor} 
\label{theorem_MV_Thm7.20} 
Using the notation for $A(x, r)$ and $B(x, r)$ from 
Theorem \ref{theorem_MV_Thm7.20-init_stmt}, 
we have that for $\delta > 0$, 
\[
o(1) \leq \left\lvert \frac{B(x, 1+\delta)}{A(x, 1)} \right\rvert \ll 2, 
     \mathrm{\ as\ } \delta \rightarrow 0^{+}, x \rightarrow \infty. 
\]
\end{cor} 
\begin{proof} 
The lower bound stated above should be clear. To show that the asymptotic 
upper bound is correct, we compute using Theorem \ref{theorem_MV_Thm7.20-init_stmt} and 
Theorem \ref{theorem_MV_Thm7.21-init_stmt} that 
\begin{align*} 
\left\lvert \frac{B(x, 1+\delta)}{A(x, 1)} \right\rvert & \ll 
     \left\lvert \frac{x \cdot (\log x)^{\delta - \delta\log(1+\delta)}}{ 
     \widehat{\pi}_1(x) + \widehat{\pi}_2(x) + \frac{x}{2} + 
     O\left(\frac{x}{\sqrt{\log\log x}}\right)} \right\rvert \\ 
     & \sim 
     \left\lvert \frac{x \cdot (\log x)^{\delta - \delta\log(1+\delta)}}{ 
     \frac{x}{\log x} + \frac{x \cdot (\log\log x)}{\log x} + \frac{x}{2} + 
     O\left(\frac{x}{\sqrt{\log\log x}}\right)} \right\rvert \\ 
     & = 
     \left\lvert \frac{(\log x)^{1 + \delta - \delta\log(1+\delta)}}{ 
     1 + \log\log x + \frac{\log x}{2} + o(1)}\right\rvert \\ 
     & \xrightarrow{\delta \rightarrow 0^{+}} 
     \left\lvert \frac{(\log x)}{ 
     1 + \log\log x + \frac{\log x}{2} + o(1)} \right\rvert \\ 
     & \sim 2, 
\end{align*} 
as $x \rightarrow \infty$. Notice that since $\mathbb{E}[\Omega(n)] = \log\log n + B$ for $0 < B < 1$, the 
absolute constant from Mertens theorem, 
when we apply this result, when we denote the range of $k > \log\log x$ as holding in the form of 
$k > (1 + \delta) \log\log x$, we can assume that $\delta \rightarrow 0^{+}$ as 
$x \rightarrow \infty$. 
\end{proof} 

We again emphasize that 
Corollary \ref{theorem_MV_Thm7.20} implies that for sums involving $\widehat{\pi}_k(x)$ indexed by $k$, 
we can capture the dominant asymptotic behavior of these sums by taking $k$ in the truncated range 
$1 \leq k \leq \log\log x$, e.g., with $0 \leq z < 1$ in Theorem \ref{theorem_HatPi_ExtInTermsOfGz}. 
This fact will be important when we prove 
Theorem \ref{theorem_gInv_GeneralAsymptoticsForms} in 
Section \ref{Section_KeyApplications} using a sign-weighted 
summatory function in Abel summation that depends on these functions 
(see Lemma \ref{lemma_CLT_and_AbelSummation}). 


\subsection{New results utilizing Theorem \ref{theorem_HatPi_ExtInTermsOfGz}} 
\label{subSection_PartialPrimeProducts_Proofs} 

We will require a handle on partial sums of integer powers of the reciprocal primes as 
functions of the integral exponent and the upper summation index $x$. 
The next corollary is not a triviality as it comes in handy when we take to the next task of 
proving the bound in Theorem \ref{theorem_GFs_SymmFuncs_SumsOfRecipOfPowsOfPrimes}. 
The statement of Proposition \ref{cor_PartialSumsOfReciprocalsOfPrimePowers} 
effectively provides a coarse rate in $x$ below which the reciprocal prime sums tend to 
absolute constants given by the prime zeta function, $P(s)$. We also require the finite-degree 
polynomial dependence of these bounds on $s$ to simplify the computations in the theorem below. 

\begin{prop} 
\label{cor_PartialSumsOfReciprocalsOfPrimePowers} 
For real $s \geq 1$, let 
\[
P_s(x) := \sum_{p \leq x} p^{-s}, x \geq 2. 
\]
When $s := 1$, we have the known bound in Mertens theorem 
(see Theorem \ref{theorem_Mertens_theorem}). 
For all integers $s \geq 2$ 
there is an absolutely defined bounding function $\gamma_1(s, x)$ such that 
\[
P_s(x) \leq \gamma_1(s, x) + o(1), \mathrm{\ as\ } x \rightarrow \infty. 
\] 
It suffices to take the bounding function in the previous equation as 
\begin{align*}
%\gamma_0(z, x) & = -s\log\left(\frac{\log x}{\log 2}\right) + \frac{3}{4}s(s-1) \log(x/2) - 
%     \frac{11}{36} s(s-1)^2 \log^2(x) \\ 
\gamma_1(s, x) & := -s\log\left(\frac{\log x}{\log 2}\right) + \frac{3}{4}s(s-1) \log(x/2) + 
     \frac{11}{36} s(s-1)^2 \log^2(2). 
\end{align*}
\end{prop} 
\NBRef{A05-2020-04-26} 
\begin{proof} 
Let $s > 1$ be real-valued. 
By Abel summation with the summatory function $A(x) = \pi(x) \sim \frac{x}{\log x}$ and where 
our target function smooth function $f(t) = t^{-s}$ with 
$f^{\prime}(t) = -s \cdot t^{-(s+1)}$ at each fixed $s > 1$, we obtain that 
\begin{align*} 
P_s(x) & = \frac{1}{x^s \cdot \log x} + s \cdot \int_2^{x} \frac{dt}{t^s \log t} \\ 
     & = E_1((s-1) \log 2) - E_1((s-1) \log x) + o(1), |x| \rightarrow \infty. 
\end{align*} 
Now using the inequalities in Facts \ref{facts_ExpIntIncGammaFuncs}, we obtain that the 
difference of the exponential integral functions is bounded above and below by 
\begin{align*} 
\frac{P_s(x)}{s} & \geq -\log\left(\frac{\log x}{\log 2}\right) + \frac{3}{4}(s-1) \log(x/2) - 
     \frac{11}{36} (s-1)^2 \log^2(x) \\ 
\frac{P_s(x)}{s} & \leq -\log\left(\frac{\log x}{\log 2}\right) + \frac{3}{4}(s-1) \log(x/2) + 
     \frac{11}{36} (s-1)^2 \log^2(2). 
\end{align*} 
This completes the proof of the bounds cited above in the statement of this lemma. 
\end{proof} 

\NBRef{A06-2020-04-26} 
\begin{proof}[Proof of Theorem \ref{theorem_GFs_SymmFuncs_SumsOfRecipOfPowsOfPrimes}] 
\label{proofOf_theorem_GFs_SymmFuncs_SumsOfRecipOfPowsOfPrimes} 
We have for all integers $0 \leq k \leq m$, and any sequence 
$\{f(n)\}_{n \geq 1}$ with bounded partial sums, that 
\begin{equation} 
\label{eqn_pf_tag_hSymmPolysGF} 
[z^k] \prod_{1 \leq i \leq m} (1-f(i) z)^{-1} = [z^k] \exp\left(\sum_{j \geq 1} 
     \left(\sum_{i=1}^m f(i)^j\right) \frac{z^j}{j}\right), |z| < 1. 
\end{equation} 
In our case we have that $f(i)$ denotes the $i^{th}$ prime in the 
generating function expansion of \eqref{eqn_pf_tag_hSymmPolysGF}. 
Hence, summing over all $p \leq ux$ 
in place of $0 \leq k \leq m$ in the previous formula in tandem with 
Proposition \ref{cor_PartialSumsOfReciprocalsOfPrimePowers}, we obtain that the logarithm of the 
generating function in $z$ obtained when we sum over all $p \leq ux$ for some minimal parameter 
$u$ is given by 
\begin{align*} 
\log\left[\prod_{p \leq ux} \left(1-\frac{z}{p}\right)^{-1}\right] & \geq (B + \log\log (ux)) z + 
     \sum_{j \geq 2} \left[a(ux) + b(ux)(j-1) + c(ux) (j-1)^2\right] z^j \\ 
     & = (B + \log\log (ux)) z - a(ux) \left(1 + \frac{1}{z-1} + z\right) \\ 
     & \phantom{= (B + \log\log (ux)) z\ } + 
     b(ux) \left( 
     1 + \frac{2}{z-1} + \frac{1}{(z-1)^2}\right) \\ 
     & \phantom{= (B + \log\log (ux)) z\ } - 
     c(ux) \left( 
     1 + \frac{4}{z-1} + \frac{5}{(z-1)^2} + \frac{2}{(z-1)^3}\right) \\ 
     & =: \widehat{\mathcal{B}}(u, x; z). 
\end{align*} 
In the previous equations, the lower bounds formed by the functions $(a,b,c)$ 
evaluated at $ux$ are 
given by the corresponding upper bounds from 
Proposition \ref{cor_PartialSumsOfReciprocalsOfPrimePowers} 
due to the leading sign on the previous expansions as 
\begin{align*} 
(a_{\ell}, b_{\ell}, c_{\ell}) & := \left(-\log\left(\frac{\log (ux)}{\log 2}\right), 
     \frac{3}{4} \log\left(\frac{ux}{2}\right), \frac{11}{36} \log^2 2\right). 
\end{align*} 
Now we make a decision to set the uniform bound parameter to a middle ground value of 
$1 < R < 2$ at $R := \frac{3}{2}$ 
(practically, to be truncated and taken as though 
$R \equiv 1$ in sums by the restriction that $0 \leq z < 1$) so that 
$$z \equiv z(k, x) = \frac{k-1}{\log\log x} \in [0, R),$$ for $x \gg 1$ very large. 
Thus $(z-1)^{-m} \in [(-1)^m, 2^m]$ for integers $m \geq 1$, and so we can obtain the 
lower bound stated below. Namely, these bounds on the signed reciprocals of $z-1$ 
lead to an effective bound of the following form: 
\begin{align*} 
\widehat{\mathcal{B}}(u, x; z) & \geq (B + \log\log (ux)) z - a(ux) \left(1 + \frac{1}{z-1} + z\right) \\ 
     & \phantom{= (B + \log\log (ux)) z\ } + 
     b(ux) \left( 
     1 + \frac{2}{z-1} + \frac{1}{(z-1)^2}\right) - 
     45 \cdot c(ux). 
\end{align*} 
Since the function $c(ux)$ is constant, we then also obtain the next bounds. 
\begin{align} 
\notag 
\frac{e^{-Bz}}{(\log (ux))^{z}} \times \exp\left(\widehat{\mathcal{B}}(u, x; z)\right) & \geq 
    \exp\left(-\frac{55}{4} \log^2(2)\right) \times \left(\frac{\log(ux)}{\log 2}\right)^{1 + \frac{1}{z-1} + z} \\ 
\notag 
    & \phantom{\geq \times} \times \left(\frac{ux}{2}\right)^{\frac{3}{4}\left(1 + \frac{2}{z-1} + \frac{1}{(z-1)^2}\right)} \\ 
\label{eqn_proof_simpl_v1} 
     & =: \widehat{\mathcal{C}}(u, x; z) 
\end{align} 
Now we need to determine which values of $u$ minimize the expression for the function defined 
in \eqref{eqn_proof_simpl_v1}. 
For this we will use a somewhat limited elementary method from 
introdutory calculus to determine a global minimum for the products. 

We can symbolically use \emph{Mathematica} to see that 
\[
\frac{\partial}{\partial u}\left[\widehat{\mathcal{C}}(u, x; z)\right] \Biggr\rvert_{u \mapsto u_0} = 0 \implies 
     u_0 \in \left\{\frac{1}{x}, \frac{1}{x} e^{-\frac{4}{3}(z-1)}\right\}. 
\]
When we substitute this outstanding parameter value of $u_0 =: \hat{u}_0 \mapsto \frac{1}{x} e^{-\frac{4}{3}(z-1)}$ 
into the next expression for the second derivative of the same function 
$\widehat{\mathcal{C}}(u, x; z)$ we obtain 
\begin{align*} 
\frac{\partial^2}{{\partial u}^2}\left[\widehat{\mathcal{C}}(u, x; z)\right] \Biggr\rvert_{u = \hat{u}_0} & = 
     \exp\left(-\frac{55}{4} \log^2(2)\right) x^2 2^{\frac{8 z^3-27 z^2+32 z-16}{4 (z-1)^2}} 
     3^{-z+\frac{1}{1-z}+1} e^{\frac{5 z^2-16 z+8}{3 (z-1)}} \times \\ 
     & \phantom{=\times} \times (1-z)^{z+\frac{1}{z-1}-2} z^2
     \log(2)^{\frac{z^2}{1-z}} > 0, 
\end{align*} 
provided that $z < 1$. 
The restriction to $0 \leq z < 1$ is equivalent to requiring that 
$1 \leq k \leq \log\log x$ in Theorem \ref{theorem_HatPi_ExtInTermsOfGz}. 
This restriction on $k$ to note 
leads to a minimum value on the partial product, or lower bound, at this $u = \hat{u}_0$ 
since the second derivative is positive at this critical value for $z$ within this range. 

After substitution of $u = \frac{1}{x} e^{-\frac{4}{3}(z-1)}$ into the expression for 
$\widehat{\mathcal{C}}(u, x; z)$ defined above, we have that 
\[
\widehat{\mathcal{C}}(u, x; z) \geq \exp\left(-\frac{55}{4} \log^2(2)\right) \cdot 2^{\frac{9}{16}} 
     \left(\frac{1-z}{3e\log 2}\right)^3 \times \left(\frac{4(1-z)}{3e\log 2}\right)^z. 
\]
Finally, since $z \equiv z(k, x) = \frac{k-1}{\log\log x}$ and $k \in [0, R\log\log x)$, we obtain 
for minimal $k$ and large $x \gg 1$ that $\Gamma(z+1) \approx 1$, and for $k$ towards the upper range of 
its interval that $\Gamma(z+1) \approx \Gamma(5/2) = \frac{3}{4} \sqrt{\pi}$. 
In total, what we get out of these formulas is stated up to accurate 
constant factor as in the theorem bounds. 
\end{proof} 

\newpage
\section{Average case analysis of bounds on the Dirichlet inverse functions, $g^{-1}(n)$} 
\label{Section_InvFunc_PreciseExpsAndAsymptotics} 

It happens that the property in (C) suggested by 
Conjecture \ref{lemma_gInv_MxExample} along cases of squarefree $n \geq 1$ is not the most 
simple, nor only accurate way to express the limiting behavior of the 
Dirichlet inverse functions $g^{-1}(n)$ we can formulate, 
though it does capture an important characteristic that is true more globally than just at the 
squarefree integers $n \geq 1$. Namely, that these 
functions can be expressed via more simple formulas than inspection of the initial 
repetitive, quasi-periodic sequence properties in the table might otherwise suggest. 
The pages of tabular data given as Table \ref{table_conjecture_Mertens_ginvSeq_approx_values} 
given in the appendix section starting on 
page \pageref{table_conjecture_Mertens_ginvSeq_approx_values} are intended to 
provide clear insight into why we arrived at the convenient approximations to 
$g^{-1}(n)$ proved in this section. The table offers illustrative 
numerical data formed by examining the approximate behavior 
at work here for the asymptotically 
small order cases of $1 \leq n \leq 500$ with \emph{Mathematica}. 

\subsection{Definitions and basic properties of key component function sequences} 

We define the following sequence for integers $n \geq 1, k \geq 0$: 
\begin{align} 
\label{eqn_CknFuncDef_v2} 
C_k(n) := \begin{cases} 
     \varepsilon(n), & \text{ if $k = 0$; } \\ 
     \sum\limits_{d|n} \omega(d) C_{k-1}(n/d), & \text{ if $k \geq 1$. } 
     \end{cases} 
\end{align} 
The sequence of important semi-diagonals of these functions begins as 
\cite[\seqnum{A008480}]{OEIS} 
\[
\{\lambda(n) \cdot C_{\Omega(n)}(n) \}_{n \geq 1} \mapsto \{
     1, -1, -1, 1, -1, 2, -1, -1, 1, 2, -1, -3, -1, 2, 2, 1, -1, -3, -1, \
     -3, 2, 2, -1, 4, 1, 2, \ldots \}. 
\]

\begin{remark}[An effective range of $k$ depending on a fixed large $n$]
Notice that by expanding the recursively-based definition in \eqref{eqn_CknFuncDef_v2} 
out to its maximal depth by nested divisor sums, for fixed $n$, $C_k(n)$ is seen to 
only ever possibly be non-zero for $k \leq \Omega(n)$. 
This observation follows from the fact that 
a minimal condition on the forms of 
divisors $d > 1$ of $n$ requires that $d$ have at least a single prime factor. 
Thus, the effective range of $k$ for fixed $n$ is restricted by the 
conditions of $C_0(n) = \delta_{n,1}$ and that $C_k(n) = 0$, $\forall k > \Omega(n)$. 
That is, for all sufficiently large $n \geq 2$, the contributions from 
summations over $C_k(n)$ are only significant whenever $1 \leq k \leq \Omega(n)$. 
\end{remark} 

\begin{example}[Special cases of the functions $C_k(n)$ for small $k$] 
\label{example_SpCase_Ckn} 
We cite the following special cases which should be easy enough to see on paper by 
explicit computation using \eqref{eqn_CknFuncDef_v2}: 
\NBRef{A07-2020-04-26} 
\begin{align*} 
C_0(n) & = \delta_{n,1} \\ 
C_1(n) & = \omega(n) \\ 
C_2(n) & = d(n) \times \sum_{p|n} \frac{\nu_p(n)}{\nu_p(n)+1} - \gcd\left(\Omega(n), \omega(n)\right). 
\end{align*} 
\end{example} 

\subsection{Uniform asymptotics of $C_k(n)$ for large all $n$ and fixed, bounded $k$} 

Theorem \ref{theorem_Ckn_GeneralAsymptoticsForms} from the introduction is proved next. 
The theorem makes precise what these formulas already 
suggest about the main terms of the growth rates of 
$C_k(n)$ as functions of $k,n$ for limiting cases of $n$ large for fixed $k$ which is bounded in $n$, 
but taken as an independent parameter. 

\NBRef{A08-2020-04-26} 
\begin{proof}[Proof of Theorem \ref{theorem_Ckn_GeneralAsymptoticsForms}] 
\label{proofOf_theorem_Ckn_GeneralAsymptoticsForms} 
We can see by Example \ref{example_SpCase_Ckn} that $C_1(n)$ 
satsfies the formula we must establish when $k := 1$ since $\mathbb{E}[\omega(n)] = \log\log n$. 
We prove our bounds by induction on $k$. 
In particular, suppose that $k \geq 2$ and let the inductive assumption for all $1 \leq m < k$ 
be that 
\[
\mathbb{E}[C_m(n)] \gg (\log\log n)^{2m-1}. 
\] 
Now using the recursive formula we used to define the sequences of $C_k(n)$ in 
\eqref{eqn_CknFuncDef_v2}, we have that as $n \rightarrow \infty$ 
\begin{align*} 
\mathbb{E}[C_k(n)] & = \mathbb{E}\left[\sum_{d|n} \omega(n/d) C_{k-1}(d)\right] \\ 
     & = \frac{1}{n} \times \sum_{d \leq n} C_{k-1}(d) \times \sum_{r=1}^{\Floor{n}{d}} \omega(r) \\ 
     & \sim \sum_{d \leq n} C_{k-1}(d) \left[ 
     \frac{\log\log(n/d) \Iverson{d \leq \frac{n}{e}}}{d} + \frac{B}{d}\right] \\ 
     & \sim \sum_{d \leq \frac{n}{e}} \left[ 
     \sum_{m < d} \frac{\mathbb{E}[C_{k-1}(m)]}{m} \log\log\left(\frac{n}{m}\right) + 
     B \cdot \mathbb{E}[C_{k-1}(d)] + B \cdot \sum_{m < d} \frac{\mathbb{E}[C_{k-1}(m)]}{m} 
     \right] \\ 
     & \gg B \left[n \log n \cdot (\log\log n)^{2k-3} - \log n \cdot (\log\log n)^{2k-3}\right] \times 
     \left(1 + \frac{\log n}{2}\right) \\ 
     & \gg (\log\log n)^{2k-1}. 
\end{align*} 
In transitioning to the last equation from the previous step, we have used that 
$\frac{Bn}{2} \cdot (\log n)^2 \gg (\log\log n)^2$ as $n \rightarrow \infty$. We have also used that for large 
$n \rightarrow \infty$ and fixed $m$, we have by an approximation to the incomplete gamma function that 
\[
\int_{e}^{n} \frac{(\log\log t)^m}{t} \sim (\log n) (\log\log n)^{m}, 
     \mathrm{\ as\ } n \rightarrow \infty. 
\]
Thus the claim holds by mathematical induction. 
\end{proof} 

\begin{remark} 
In Section \ref{Section_ProofOfValidityOfAverageOrderLowerBounds} 
we show that when $k := \Omega(n)$ depends on $n$, then 
\[
\mathbb{E}[C_{\Omega(n)}(n)] \gg (\log n) (\log\log n)^{2\log\log n - 1} \gg \log n \cdot \log\log n. 
\] 
Indeed, for any fixed integral powers $m \geq 1$, whenever $n \rightarrow \infty$ is taken large enough 
we have that 
\[
\mathbb{E}[C_{\Omega(n)}(n)] \gg (\log n)^{m} \cdot \log\log n.  
\]
The estimates above, especially the rightmost lower bound in the previous equation for $m := 1$, 
are much weaker than the sharpest possible estimate we could have obtained working through 
the arithmetic given in the proof of 
Theorem \ref{theorem_Ckn_GeneralAsymptoticsForms}. 
However, it turns out that this statement is sufficient to prove the necessary hypotheses of 
Theorem \ref{theorem_CondAvgOrderGInvxSummatoryFunc_v1} 
are attainable for all large $x$ in 
Section \ref{subSection_ProvingTheNecessaryHyps_ThmCondAvgOrderGInvxSummatoryFunc_v1}. 
\end{remark} 

\subsection{Relating the auxiliary functions $C_k(n)$ to formulas approximating $g^{-1}(n)$} 

\begin{lemma}[An exact formula for $g^{-1}(n)$] 
\label{lemma_AnExactFormulaFor_gInvByMobiusInv_v1} 
For all $n \geq 1$, we have that 
\[
g^{-1}(n) = \sum_{d|n} \mu\left(\frac{n}{d}\right) \lambda(d) C_{\Omega(d)}(d). 
\]
\end{lemma}
\begin{proof} 
We first write out the standard recurrence relation for the Dirichlet inverse of 
$\omega+1$ as 
\begin{align*} 
g^{-1}(n) & = - \sum_{\substack{d|n \\ d>1}} (\omega(d) + 1) g^{-1}(n/d) 
     \quad\implies\quad 
     (g^{-1} \ast 1)(n) = -(\omega \ast g^{-1})(n). 
\end{align*} 
Now by repeatedly expanding the right-hand-side, and removing corner cases in the nested sums with 
$\omega(1) = 0$ by convention, we find inductively that 
\[
(g^{-1} \ast 1)(n) = (-1)^{\Omega(n)} C_{\Omega(n)}(n) = \lambda(n) C_{\Omega(n)}(n). 
\]
The statement follows by M\"obius inversion applied to each side of the last equation. 
\end{proof} 

\begin{cor} 
\label{cor_AnExactFormulaFor_gInvByMobiusInv_nSqFree_v2} 
For all squarefree integers $n \geq 1$, we have that 
\begin{equation} 
\label{eqn_gInvnSqFreeN_exactDivSum_Formula} 
g^{-1}(n) = \lambda(n) \times \sum_{d|n} C_{\Omega(d)}(d). 
\end{equation} 
\end{cor} 
\begin{proof} 
Since $g^{-1}(1) = 1$, clearly the claim is true for $n = 1$. Suppose that $n \geq 2$ and that 
$n$ is squarefree. Then $n = p_1p_2 \cdots p_{\omega(n)}$ where $p_i$ is prime for all 
$1 \leq i \leq \omega(n)$. So we can transform the exact divisor sum guaranteed for all $n$ in 
Lemma \ref{lemma_AnExactFormulaFor_gInvByMobiusInv_v1} into the following: 
\begin{align*} 
g^{-1}(n) & = \sum_{i=1}^{\omega(n)} \sum_{\substack{d|n \\ \omega(d)=i}} (-1)^{\omega(n) - i} (-1)^{i} \cdot 
     C_{\Omega(d)}(d) + \mu(1) \lambda(n) C_1(1) \\ 
     & = \lambda(n) \left[\sum_{i=1}^{\omega(n)} \sum_{\substack{d|n \\ \omega(d)=i}} C_{\Omega(d)}(d) + 1\right] \\ 
     & = \lambda(n) \times \sum_{d|n} C_{\Omega(d)}(d). 
\end{align*} 
The signed computations in the first of the previous equations is justified by noting that $\lambda(n) = (-1)^{\omega(n)}$ 
whenever $n$ is squarefree, and that for $d$ squarefree with $\omega(d) = i$, $\Omega(d) = i$. 
\end{proof} 

%\begin{proof}[Proof of property (C) of Conjecture \ref{lemma_gInv_MxExample}] 
%\label{proofOf_propCFromConj_lemma_gInv_MxExample} 
%We can prove by induction on $\omega(n)$, the number of distinct prime factors of $n \geq 2$, that 
%for all squarefree integers $n \geq 1$, $C_{\Omega(n)}(n) = (\omega(n))!$. Since $g^{-1}(1) = 1$, 
%clearly the conjecture is true for $n = 1$. For squarefree $n \geq 2$, we can prove property (C) 
%directly by applying Lemma \ref{lemma_AnExactFormulaFor_gInvByMobiusInv_v1}. That is, since 
%all divisors of $n$ squarefree are also squarefree, with the number of $d|n$ with exactly $k$ prime factors 
%given by $\binom{\omega(n)}{k}$ for $0 \leq k \leq \omega(n)$, we have that 
%\begin{align*} 
%g^{-1}(n) & = \sum_{k=0}^{\omega(n)} \sum_{\substack{d|n \\ \omega(d) = k}} 
%     \mu(n/d) \lambda(d) C_{\Omega(d)}(d) \\ 
%     & = \sum_{k=0}^{\omega(n)} \binom{\omega(n)}{k} \#\{1 \leq d \leq n: d|n, \omega(d) = k\} \times 
%     (-1)^{\omega(n) - k} \cdot (-1)^{k} \cdot C_{k}\left(p_1 \cdots p_k \Iverson{k > 0} + \Iverson{k = 0}\right) \\ 
%     & = (-1)^{\omega(n)} \times \sum_{k=0}^{\omega(n)} \binom{\omega(n)}{k} \cdot k!. 
%\end{align*} 
%Finally, since $\Omega(n) = \omega(n)$ whenever $n$ is squarefree, we obtain that the leading sign term on 
%the sum in the previous equation is indeed $\lambda(n)$, as expected. 
%\end{proof} 

\begin{cor} 
\label{lemma_BddExpectationOfgInvn} 
We have that 
\[
\frac{6}{\pi^2} \log x \ll \mathbb{E}|g^{-1}(n)| \leq \mathbb{E}\left[\sum_{d|n} C_{\Omega(d)}(d)\right]. 
\]
\end{cor} 
\begin{proof} 
To prove the lower bound, first notice that by 
Lemma \ref{lemma_AnExactFormulaFor_gInvByMobiusInv_v1}, 
Proposition \ref{prop_SignageDirInvsOfPosBddArithmeticFuncs_v1} and the 
complete multiplicativity of $\lambda(n)$, 
we easily obtain that 
\begin{equation} 
\label{eqn_AbsValueOf_gInvn_FornSquareFree_v1} 
|g^{-1}(n)| = \sum_{d|n} \mu^2\left(\frac{n}{d}\right) C_{\Omega(d)}(d). 
\end{equation} 
In particular, since $\mu(n)$ is non-zero only at squarefree integers and 
at any squarefree $n \geq 1$ we have $\mu(n) = (-1)^{\omega(n)} = \lambda(n)$, 
Lemma \ref{lemma_AnExactFormulaFor_gInvByMobiusInv_v1} implies 
\begin{align*} 
|g^{-1}(n)| & = \lambda(n) \times \sum_{d|n} \mu\left(\frac{n}{d}\right) \lambda(d) C_{\Omega(d)}(d) \\ 
     & = \sum_{d|n} \mu^2\left(\frac{n}{d}\right) \lambda\left(\frac{n}{d}\right) 
     \lambda(nd) C_{\Omega(d)}(d) \\ 
     & = \lambda(n^2) \times \sum_{d|n} \mu^2\left(\frac{n}{d}\right) C_{\Omega(d)}(d). 
\end{align*} 
Notice in the above equation 
that $\lambda(n^2) = +1$ for all $n \geq 1$ since the number of distinct 
prime factors (counting multiplicity) of a square integer is necessarily even. 
     
Recall from the introduction that the summatory function of the 
squarefree integers is given by 
\[
Q(x) := \sum_{n \leq x} \mu^2(n) = \frac{6}{\pi^2} x + O(\sqrt{x}). 
\]
Then since $C_{\Omega(d)}(d) \geq 1$ for all $d \geq 1$, we obtain by summing over 
\eqref{eqn_AbsValueOf_gInvn_FornSquareFree_v1} that as $x \rightarrow \infty$ 
\begin{align*} 
\frac{1}{x} \times \sum_{n \leq x} |g^{-1}(n)| & = \frac{1}{x} \times \sum_{d \leq x} 
     C_{\Omega(d)}(d) Q\left(\Floor{x}{d}\right) \\ 
     & \sim \sum_{d \leq x} C_{\Omega(d)}(d) \left[\frac{6}{d \cdot \pi^2} + O\left(\frac{1}{\sqrt{dx}}\right) 
     \right] \\ 
     & \geq \sum_{d \leq x} \left[\frac{6}{d \cdot \pi^2} + O\left(\frac{1}{\sqrt{dx}}\right)\right] \\ 
     & \sim \frac{6}{\pi^2}\left(\log x + \gamma + O\left(\frac{1}{x}\right)\right) + 
     O\left(\frac{1}{\sqrt{x}} \times \int_0^{x} t^{-1/2} dt\right) \\ 
     & = \frac{6}{\pi^2} \log x + O(1). 
\end{align*} 
To prove the upper bound, notice that by 
Lemma \ref{lemma_AnExactFormulaFor_gInvByMobiusInv_v1} and 
Corollary \ref{cor_AnExactFormulaFor_gInvByMobiusInv_nSqFree_v2}, 
\[
|g^{-1}(n)| \leq \sum_{d|n} C_{\Omega(d)}(d). 
\]
Now since both of the above quantities are positive for all $n \geq 1$, 
we must obtain the upper bound on the average order of $|g^{-1}(n)|$
stated above. 
\end{proof} 

\newpage 
\section{A rigorous justification for applying Theorem \ref{theorem_CondAvgOrderGInvxSummatoryFunc_v1}} 
\label{Section_ProofOfValidityOfAverageOrderLowerBounds} 

The point of proving the results in this section before moving onto the core results needed in 
the next section is to provide a rigorous justification for the intuition we sketched in 
Section \ref{subSection_Intro_RigorToTheAverageCaseEstimates} of the introduction. 
That is, we expect our arithmetic functions that are closely 
tied to the additive functions, $\omega(n)$ and $\Omega(n)$, to similarly behave regularly (and 
infinitely often) in accordance with their values being close to the average case for large $x$. 
What we have established so far, and will establish for $G^{-1}(x)$ in 
Section \ref{Section_KeyApplications}, are lower bound estimates that hold essentially only 
\emph{on average}. 
This means that for all sufficiently large $x \rightarrow \infty$, 
we need to show that the expected value lower bounds 
are achieved in asymptotic order more globally 
within predictably some small window (interval) around $x$. 

\subsection{The proof of our central theorem} 

\begin{proof}[Proof of Theorem \ref{theorem_CondAvgOrderGInvxSummatoryFunc_v1}] 
\label{proofOf_theorem_CondAvgOrderGInvxSummatoryFunc_v1} 
The result is obtained simply by contradiction. Suppose that $x$ is so large that the inequalities in the 
hypothesis hold given the fixed bounded $0 \leq Y < +\infty$. 
We have assumed that the constants $B,C \in (0, 1)$ are the tightest possible bounds on the next set as 
$x \rightarrow \infty$ according to their precise definitions given in the theorem statement. 
We need to show that such a 
concrete fixed $\varepsilon \in (0, 1)$ 
satisfying the conditions in the theorem exists (depending only on $B,C$). 

Let $x \geq 1$ be fixed and sufficiently large. 
Suppose that for all $\varepsilon \in (0, 1)$ satisfying $0 < B - \varepsilon, C+\varepsilon < 1$, we have that 
\begin{equation} 
\label{eqn_proof_tag_IneqSetG0x0_DNHold_v1} 
|G^{-1}(x_0)| < |G_E^{-1}(x_0)| + Y, \forall x_0 \in [(B-\varepsilon) x, (C+\varepsilon) x]. 
\end{equation} 
For $n \geq 1$, we have the disjoint set decomposition of the positive integers $n \leq x$ given by  
\[
\{1 \leq n \leq x\} = \{1 \leq n < (B-\varepsilon) x\} \oplus 
     \{(B-\varepsilon) x \leq n \leq (C + \varepsilon) x\} \oplus 
     \{(C+\varepsilon) x < n \leq x\}, 
\]
where the three disjoint sets above are respectively denoted in increasing 
left-to-right order by $\mathcal{D}_i(x)$ for $i = 1,2,3$. 
The set decomposition in the previous equation yields that as 
$x \rightarrow \infty$, if \eqref{eqn_proof_tag_IneqSetG0x0_DNHold_v1} is true, then 
\begin{align} 
\label{eqn_proof_tag_G123xDensityBounds_v1} 
\mathcal{G}_1(x) & := \frac{1}{x} \cdot \#\left\{n \in \mathcal{D}_1(x): |G^{-1}(x_0)| - |G_E^{-1}(x_0)| \leq Y\right\} 
     \in [(B-\varepsilon)^2 + o(1), (B-\varepsilon) (C+\varepsilon) + o(1)] \\ 
\notag 
\mathcal{G}_2(x) & := \frac{1}{x} \cdot \#\left\{n \in \mathcal{D}_2(x): |G^{-1}(x_0)| - |G_E^{-1}(x_0)| \leq Y\right\} 
     \in [B-\varepsilon, C+\varepsilon] \\ 
\notag 
\mathcal{G}_3(x) & := \frac{1}{x} \cdot \#\left\{n \in \mathcal{D}_3(x): |G^{-1}(x_0)| - |G_E^{-1}(x_0)| \leq Y\right\} \\ 
\notag 
     & \phantom{:= \frac{1}{x} \cdot \ } 
     \in [(B-\varepsilon)-(B-\varepsilon) (C+\varepsilon) + o(1), (C+\varepsilon)-(C+\varepsilon)^2 + o(1)]. 
\end{align} 
For $x \geq 1$, let the density of our target set at $x$ be denoted by 
$$\mathcal{G}_0(x) := \frac{1}{x} \cdot \#\left\{n \leq x: |G^{-1}(x_0)| - |G_E^{-1}(x_0)| \leq Y\right\}.$$ 
Then we obtain by summing the respective upper and lower bounds on the densities for the 
disjoint sets given in \eqref{eqn_proof_tag_G123xDensityBounds_v1} above that 
\[
(B-\varepsilon)^2 + B - \varepsilon + (B-\varepsilon) (1 - C - \varepsilon) + o(1) \leq \mathcal{G}_0(x) \leq 
     (B-\varepsilon) (C+\varepsilon) + C + \varepsilon + (C + \varepsilon) (1 - C - \varepsilon) + o(1). 
\]
We show that contrary to our assumption, we can in fact pick any $\varepsilon > 0$ that satisfies 
$B - 2\varepsilon < C, 0 < B - \varepsilon < 1, 0 < C + \varepsilon < 1$, e.g., choosing 
$\varepsilon := \frac{1}{2} \min(B, 1-C)$ will satisfy our requirements. 
Indeed, given such a choice of this parameter, we have that 
\[
C + \varepsilon - \left[(B-\varepsilon) (C+\varepsilon) + C + \varepsilon + (C + \varepsilon) (1 - C - \varepsilon)\right] = 
     -(C + \varepsilon)(1 + B - C - 2\varepsilon) < 0. 
\]
This implies a contradiction to the maximality in the limit supremum sense of our tight bound $C \in (0, 1)$. 
Then we must have that our assumption on $x_0$ is invalid as $x \rightarrow \infty$. 
More to the point, must be such a fixed $\varepsilon > 0$ and such a $x_0 \in [(B-\varepsilon) x, (C+\varepsilon) x]$ 
so that $|G^{-1}(x_0)| \geq |G_E^{-1}(x_0)| + Y$ whenever $x$ is sufficiently large.  
\end{proof} 

\subsection{Verifying the hypotheses in Theorem \ref{theorem_CondAvgOrderGInvxSummatoryFunc_v1} 
            are achieved for all large $x$} 
\label{subSection_ProvingTheNecessaryHyps_ThmCondAvgOrderGInvxSummatoryFunc_v1} 

\subsubsection{Building up to a proof of the necessary hypotheses: Preliminary facts and results} 
\label{subsubSection_ProvingTheNecessaryHyps_PrelimFactPfs} 

\begin{lemma}[Asymptotic densities of exceptional values of positive arithmetic functions] 
\label{lemma_AsymptoticDensitiesOfExceptionalSets_v1}
\label{remark_AsymptoticDensitiesOfExceptionalSets_v1}
Let $F \in C^{1}(0, \infty)$ be a monotone increasing function such that 
$F(x) \not{\rightarrow} 0$ as $x \rightarrow \infty$. 
Suppose that $f$ is an arithmetic function such that $f(n) > 0$ for all $n \geq 1$ that satisfies 
\[
\sum_{n \leq x} f(n) \geq x \cdot F(x) + o(xF(x)), \mathrm{\ as\ } x \rightarrow \infty. 
\]
Let the set defined by 
\begin{align*} 
\mathcal{F}_{-} & := \left\{n \geq 1: f(n) < F(n)\right\}, 
\end{align*} 
have corresponding limiting asymptotic density
\begin{align*} 
\gamma_{-} & := \lim_{x \rightarrow \infty} \frac{1}{x} \cdot \#\{n \leq x: n \in \mathcal{F}_{-}\}. 
\end{align*} 
Then the limit $\gamma_{-}$ exists and $\gamma_{-} = 0$. 
In other words, for almost every sufficiently large $n \rightarrow \infty$, 
$f(n) \geq F(n)$ and $-f(n) \leq -F(n)$. 
\end{lemma} 
\begin{proof} 
First, suppose that the limit we used to define $\gamma_{-}$ exists with $\gamma_{-} \in [0, 1)$. 
By the positivity of $f(n)$, we know that $F(x)$ is positive for all sufficiently large $x$. 
Thus we have that as $x \rightarrow \infty$ 
\begin{align} 
\notag 
\sum_{n \leq x} f(n) & \leq \sum_{\substack{n \leq x \\ n \notin \mathcal{F}_{-}}} f(n) + 
     \sum_{\substack{n \leq x \\ n \in \mathcal{F}_{-}}} F(n) \\ 
\notag 
     & < \sum_{n \leq (1-\gamma_{-}) x} f(n) + 
     \sum_{(1-\gamma_{-}) x \leq n \leq x} F(n) \\ 
\notag 
     & \sim 
     x F(x) - \int_{(1-\gamma_{-}) x}^{x} t F^{\prime}(t) dt \\ 
\label{eqn_proof_tag_GammaMinusDensityIsZero_v1} 
     & = (1-\gamma_{-}) x \cdot F((1-\gamma_{-}) x) + \gamma_{-} x \cdot F(c), 
\end{align} 
integrating by parts and for some $c \in [(1-\gamma_{-}) x, x]$ by the mean value theorem. 
So by \eqref{eqn_proof_tag_GammaMinusDensityIsZero_v1}, we have that 
\begin{align} 
\notag 
\sum_{n \leq x} f(n) & \ll x \cdot F((1-\gamma_{-}) x), 
\end{align} 
which unless $\gamma_{-} = 0$ contradicts our hypothesis on the limiting behavior 
of the summatory function of $f(n)$ by the monotonicity of $F$. 
Notice also that the limiting density cannot be one since if $\gamma_{-} = 1$, then 
\begin{align*} 
\frac{1}{x} \times \sum_{n \leq x} f(n) & < \frac{1}{x} \times \sum_{n \leq x} F(n) \\ 
     & \leq \max_{1 \leq j \leq x} F(j) + o(F(x)) = F(x) + o(F(x)), 
\end{align*} 
since $F$ is monotone increasing on $(0, \infty)$ by assumption. 
Hence, we conclude that $\gamma_{-} = 0$ provided that the limit exists. 

If a limiting value for $\gamma_{-}$ does not exist, then for infinitely many large $x \geq 1$, 
we have that 
\[
M_x := \frac{1}{x} \cdot \#\{n \leq x: n \in \mathcal{F}_{-}\}, 
\]
satisfies $M_x \in (0, 1)$ (where $M_x$ non-monotonically oscillates in value along a subsequence). 
Using a similar method to what we argued above, in this case for infinitely many $x$ 
our assumption on the asymptotic lower bound on 
the summatory function of $f(n)$ does not hold. This contradiction shows that the limit $\gamma_{-}$ 
must in fact exist, and as we have shown above, is then necessarily zero. 
\end{proof} 

We also require the following fact of our notation for average order: 

\begin{prop} 
\label{prop_factA_stmt_v1} 
For sufficiently large $n \rightarrow \infty$, we have that 
$$\mathbb{E}[C_{\Omega(n)}(n)] \gg (\log n) \cdot (\log\log n)^{2\log\log n - 1} \gg 
  \log n \cdot \log\log n, \mathrm{\ as\ } n \rightarrow \infty.$$
\end{prop} 
\begin{proof} 
We must first argue that the set of $n > e$ on which $\Omega(n)$ differs substantially 
from its average order of $\mathbb{E}[\Omega(n)] = \log\log n$ has asymptotic density zero. 
For $\delta, \rho > 0$, let 
\begin{align*} 
\Omega_{+}(\delta, x) & := \frac{1}{x} \cdot \#\{n \leq x: \Omega(n) \geq (1+\delta) \log\log x\} \\ 
\Omega_{-}(\rho, x) & := \frac{1}{x} \cdot \#\{n \leq x: \Omega(n) \leq (1+\rho) \log\log x\}. 
\end{align*} 
We utilize Theorem \ref{theorem_MV_Thm7.20-init_stmt} 
to show each of the following as $x \rightarrow \infty$: 
\begin{align*} 
\Omega_{+}(\delta, x) & \ll (\log x)^{\delta - (1+\delta)\log(1+\delta)} \\ 
\Omega_{-}(\rho, x) & \ll (\log x)^{\rho - (1+\rho)\log(1+\rho)}. 
\end{align*} 
Thus for all $\delta, \rho > 0$ where we take very small 
$\delta, \rho \approx 0^{+}$, we have that 
\begin{equation} 
\label{eqn_proof_tag_OmeganGeqLeqAsymptoticDensityCalcs_v1} 
\Omega_{+}(\delta, x) = o(1), \Omega_{-}(\rho, x) = o(1), \mathrm{\ as\ } x \rightarrow \infty. 
\end{equation} 
The results expanded in 
\eqref{eqn_proof_tag_OmeganGeqLeqAsymptoticDensityCalcs_v1} show that we can expect the 
asymptotic density of the $n \leq x$ where $\Omega(n) \not{\approx} \mathbb{E}[\Omega(n)]$ to be small, 
and tending to zero as $n \rightarrow \infty$. 

Thus with our result for fixed $1 \leq k \leq \Omega(n)$ from 
Theorem \ref{theorem_Ckn_GeneralAsymptoticsForms}, 
we can conclude that 
\begin{align} 
\notag 
\mathbb{E}[C_{\Omega(n)}(n)] & \gg \frac{1}{n} \sum_{d \leq n} (\log\log d)^{2\Omega(d)-1} \\ 
\label{eqn_proof_tag_ECknkEQOmegan_v2} 
     & \sim (\log n) \cdot (\log\log n)^{2\log\log n - 1}, \mathrm{\ as\ } n \rightarrow \infty. 
\end{align} 
Hence, we also have that 
\begin{align*} 
\mathbb{E}[C_{\Omega(n)}(n)] & \gg \log n \cdot \log\log n, \mathrm{\ as\ } n \rightarrow \infty. 
\end{align*} 
To prove that \eqref{eqn_proof_tag_ECknkEQOmegan_v2} is correct, notice that 
for any fixed $m$ we have integrating by parts and applying \eqref{eqn_IncompleteGamma_PropB} at 
large $n \rightarrow \infty$ that \footnote{ 
     In particular, we obtain the following definite integral formula exactly 
     for fixed $m$: 
     \[
     \int_{e}^{n} \frac{(\log\log t)^{m}}{t} dt = (-1)^m \cdot \Gamma(m+1, -\log\log n). 
     \]
}
\begin{align} 
\label{eqn_proof_tag_LargeNIndefIntegralForIncompleteGammaFunc_v1} 
\frac{1}{n} \times \int_{e}^{n} (\log\log t)^m dt & = \frac{1}{n}\left[ 
     n \cdot (\log n) (\log\log n)^m - (\log n) (\log\log n)^m\right] \\ 
\notag 
     & \sim (\log n) (\log\log n)^m. 
\end{align} 
So the claimed two implications follow, one after the other, by a 
perturbed expansion of the binomial series where \cite[\cf \S 6]{GKP} 
\begin{align*} 
\frac{1}{n} \times \int_{e}^{n} (\log\log t)^{2\log\log t - 1} dt & \approx 
     \frac{1}{n} \times \int_{e}^{n} \frac{(1 + \log\log t)^{2\log\log t}}{\log\log t} dt \\ 
     & = \frac{1}{n} \times \int_{e}^{n} 
     \sum_{s \geq 0} \sum_{k=0}^{s} \gkpSI{s}{k} (2\log\log t)^{k} (-1)^{s-k} \times 
     \frac{(\log\log t)^{s-1}}{s!} dt. 
\end{align*} 
Namely, we can integrate the last equation termwise using the 
integral formula in \eqref{eqn_proof_tag_LargeNIndefIntegralForIncompleteGammaFunc_v1} 
from above. 
\end{proof} 

\begin{prop} 
\label{prop_GInvGeqBehavior_v1} 
For all sufficiently large $n$ on a set of asymptotic density one, we have that 
\[
|g^{-1}(n)| \gg \frac{2}{\pi^2} (\log n)^3 (\log\log n) + O\left((\log n)^2 (\log\log n)\right). 
\]
\end{prop} 
\begin{proof} 
An immediate consequence of 
Proposition \ref{prop_factA_stmt_v1} 
is that for all sufficiently large $n$ we have that 
\[
\mathbb{E}\left[C_{\Omega(n)}(n)\right] \gg (\log n)^2 (\log\log n). 
\]
Recall once again that the summatory function of the squarefree integers is denoted by 
\[
Q(x) := \sum_{n \leq x} \mu^2(n) = \frac{6}{\pi^2} x + O(\sqrt{x}). 
\]
Then by Corollary \ref{cor_AnExactFormulaFor_gInvByMobiusInv_nSqFree_v2} 
and since 
\[
|g^{-1}(n)| \leq \sum_{d|n} C_{\Omega(d)}(d), \forall n \geq 1, 
\]
we have that as $n \rightarrow \infty$ 
\begin{align*} 
\mathbb{E}|g^{-1}(n)| & \geq \frac{1}{n} \times \sum_{\substack{m \leq n \\ \mu^2(m) = 1}} \sum_{d|m} C_{\Omega(d)}(d) \\ 
     & \sim \frac{1}{n} \times \sum_{d \leq n} C_{\Omega(d)}(d) Q\left(\frac{n}{d}\right) \\ 
     & = \frac{1}{n} \times \sum_{d \leq n} \mathbb{E}[C_{\Omega(d)}(d)] \cdot d \left( 
     \frac{6}{\pi^2} \frac{n}{d+1} - \frac{6}{\pi^2} \frac{n}{d} + O(1)\right) \\ 
     & \sim \sum_{d \leq n} \mathbb{E}[C_{\Omega(d)}(d)] \left[\frac{6}{\pi^2 \cdot d} + O\left(\frac{1}{n}\right)\right] \\ 
     & \gg \frac{6}{\pi^2} \int_{e}^n \frac{(\log t)^2 (\log\log t)}{t} dt + 
     O\left(\frac{1}{n} \times \int_{e}^n (\log t)^2 (\log\log t) dt\right) \\ 
     & = \frac{2}{\pi^2} \left((\log n)^3 \log\log n - \frac{(\log n)^3}{3}\right) + 
     O\left((\log n)^2 \log\log n\right) \\ 
     & \gg \frac{2}{\pi^2} (\log n)^3 \log\log n + O\left((\log n)^2 \log\log n\right). 
\end{align*} 
So using our observation in Lemma \ref{lemma_AsymptoticDensitiesOfExceptionalSets_v1} 
with $\mathbb{E}|g^{-1}(n)| \gg \frac{6}{\pi^2} \cdot \log n \not{\rightarrow} 0$ by 
Corollary \ref{lemma_BddExpectationOfgInvn}, we have that our statement holds. 
\end{proof} 

\begin{cor} 
\label{cor_GInvGeqBehavior_v2} 
For all sufficiently large $n$ on a set of asymptotic density one, we have that 
\[
\sum_{\substack{d|n \\ d > e}} (\log d) (\log\log d) - |g^{-1}(n)| \leq 0. 
\]
\end{cor} 
\begin{proof} 
First, we see that since $\mathbb{E}[d(n)] = \log n$ \cite[\S 27.11]{NISTHB}, 
for all large enough $n$ on a set of asymptotic density one we have 
\begin{align*} 
\mathbb{E}\left[\sum_{\substack{d|n \\ d > e}} (\log d) (\log\log d)\right] & = 
     \frac{1}{n} \times \sum_{e < d \leq n} (\log d) (\log\log d) \Floor{x}{d} \\ 
     & \sim \int_{e}^{n} \frac{(\log t) (\log\log t)}{t} dt \\ 
     & = \frac{(\log n)^2}{2} (\log\log n) - \frac{(\log n)^2}{4} \\ 
     & \sim \frac{(\log n)^2}{2} (\log\log n). 
\end{align*} 
Now on another set of asymptotic density one, we have that since 
$\mathbb{E}|g^{-1}(n)| \gg \frac{6}{\pi^2} \cdot \log n \not{\rightarrow} 0$ by 
Corollary \ref{lemma_BddExpectationOfgInvn}, 
Lemma \ref{lemma_AsymptoticDensitiesOfExceptionalSets_v1} implies that 
\begin{align*} 
\mathbb{E}\left[\sum_{\substack{d|n \\ d > e}} (\log d) (\log\log d) - |g^{-1}(n)|\right] & = 
     \mathbb{E}\left[\sum_{\substack{d|n \\ d > e}} (\log d) (\log\log d)\right] - 
     \mathbb{E}|g^{-1}(n)| \\ 
     & \ll 
     \frac{(\log n)^2}{2} (\log\log n) - \mathbb{E}|g^{-1}(n)|. 
\end{align*} 
So applying Lemma \ref{lemma_AsymptoticDensitiesOfExceptionalSets_v1} and by 
Proposition \ref{prop_GInvGeqBehavior_v1}, for all large enough $n$ 
within a set of asymptotic density one on the integers, we have that 
\[
\sum_{\substack{d|n \\ d > e}} (\log d) (\log\log d) - |g^{-1}(n)| \ll 0, 
     \mathrm{\ as\ } n \rightarrow \infty. 
\] 
We have implicitly used that the intersection of a finite number of subsets of 
the positive integers with asymptotic density one also has limiting density of one. 
\end{proof} 

\begin{prop} 
\label{prop_DensityOfGInvxPosAndBdd} 
Let the set where $G^{-1}(x)$ is non-positive be defined as 
\[
\mathcal{G}_{-} := \left\{n \leq x: G^{-1}(x) \leq 0\right\}. 
\]
We claim that for all large $x \rightarrow \infty$, the density of this set is 
positive and less than one: 
\[
0 + o(1) < \frac{1}{x} \cdot \#\{n \leq x: n \in \mathcal{G}_{-}\} < 1 + o(1). 
\]
Moreover, if a limiting asymptotic density for $\mathcal{G}_{-}$ exists, it does not 
tend to zero as $x \rightarrow \infty$: 
\[
\lim_{x \rightarrow \infty} \frac{1}{x} \cdot \#\{n \leq x: n \in \mathcal{G}_{-}\} \neq 0. 
\]
\end{prop} 

Note that the proposition above also implies that the corresponding set $\mathcal{G}_{+}$ over which 
$G^{-1}(x) > 0$ has positive density for all $x$ sufficiently large, and that this density does not 
tend to zero as $x \rightarrow \infty$. 
We will prove Proposition \ref{prop_DensityOfGInvxPosAndBdd} after we prove 
Proposition \ref{prop_Mx_SBP_IntegralFormula} in the next section. 

\subsubsection{The proof that the necessary hypotheses in Theorem \ref{theorem_CondAvgOrderGInvxSummatoryFunc_v1} are 
               attained for all large $x$} 
\label{subsubSection_PfOfNecessaryThmHyps} 

\begin{proof}[Proof of the hypotheses of Theorem \ref{theorem_CondAvgOrderGInvxSummatoryFunc_v1}]
Let $G_E^{-1}(x)$ be defined as in \eqref{eqn_GEInvxSummatoryFuncDef_v1} of the theorem. 
We need to find some absolute tight limiting constants 
$B, C \in (0, 1)$ such that as $x \rightarrow \infty$ 
\begin{equation} 
\label{eqn_proof_tag_ThmConstsBCHyp_defs_v1} 
B + o(1) \leq \frac{1}{x} \cdot \#\left\{n \leq x: |G^{-1}(n)| - |G_E^{-1}(n)| \leq Y\right\} \leq 
     C + o(1), 
\end{equation} 
for some bounded constant $0 \leq Y < +\infty$, 
By Corollary \ref{cor_GInvGeqBehavior_v2}, 
for all $n$ sufficiently large within a set $\mathcal{S}_E$ of asymptotic density also one, 
\begin{align} 
\label{eqn_proof_tag_IneqsHold_vb} 
\sum_{\substack{d|n \\ d > e}} (\log d) (\log\log d) - |g^{-1}(n)| & \leq 0, \forall n \in \mathcal{S}_E, 
     \mathrm{\ as\ } n \rightarrow \infty. 
\end{align} 
Now we aim to sum the functions $G^{-1}(x)$ and $G_E^{-1}(x)$ weighted by the same signs on the 
terms at each large enough $n$ that satisfy the condition in \eqref{eqn_proof_tag_IneqsHold_vb}. 

Since the sign of $g^{-1}(n)$ is $\lambda(n)$ as given by 
Proposition \ref{prop_SignageDirInvsOfPosBddArithmeticFuncs_v1}, for all large enough $n \rightarrow \infty$ 
on the set $\mathcal{S}_E$ defined as in 
\eqref{eqn_proof_tag_IneqsHold_vb}, we have that both 
\begin{align*} 
\sum_{\substack{e \leq n \leq x \\ \lambda(n) = +1}} g^{-1}(n) & \geq 
     \sum_{\substack{e \leq n \leq x \\ \lambda(n) = +1}} 
     \sum_{\substack{d|n \\ d > e}} (\log d) (\log\log d) \\ 
\sum_{\substack{e \leq n \leq x \\ \lambda(n) = -1}} g^{-1}(n) & \leq 
     -\sum_{\substack{e \leq n \leq x \\ \lambda(n) = -1}}
     \sum_{\substack{d|n \\ d > e}} (\log d) (\log\log d). 
\end{align*} 
Hence, we have that almost everywhere on $\mathbb{Z}^{+}$ as $x \rightarrow \infty$ the 
following equation is true for some constant offset $0 \leq Y < +\infty$: 
\begin{equation} 
\label{eqn_proof_tag_IneqsHold_vc} 
G^{-1}(x) \geq \sum_{n \leq x} \lambda(n) \sum_{\substack{d|n \\ d > e}} (\log d) (\log\log d) + Y. 
\end{equation} 
Now we notice that the right-hand-side of \eqref{eqn_proof_tag_IneqsHold_vc} corresponds to the definition of 
the function $G_E^{-1}(x)$. Hence, we see that if $G^{-1}(x) \leq 0$ where 
\eqref{eqn_proof_tag_IneqsHold_vc} holds, then also 
$G_E^{-1}(x) \leq 0$, and so letting 
\[
\mathcal{A}_E(Y) := \left\{x \geq 1: 
     G^{-1}(x) \geq G_E^{-1}(x) + Y \wedge G^{-1}(x) \leq 0 
     \right\}, 
\] 
we have that $|G^{-1}(x)| - |G_E^{-1}(x)| \leq Y$, $\forall x \in \mathcal{A}_E(Y)$. 
We still need to show that the density of $\mathcal{A}_E(Y)$ in $\{n \leq x\}$ can be bounded closely 
below and above by some respective constants $B, C \in (0, 1)$ 
for all large enough $x \rightarrow \infty$. 

Using Proposition \ref{prop_DensityOfGInvxPosAndBdd} and that 
\eqref{eqn_proof_tag_IneqsHold_vc} holds almost everywhere on the sufficiently large 
positive integers, 
we can see that there must be some limitingly tight constants 
$B, C \in (0, 1)$ bounding the densities of the infinite set, $\mathcal{A}_E(Y)$, such that the condition 
$|G^{-1}(x)| - |G_E^{-1}(x)| \leq Y$ holds for all large $x$ within this set in the following form: 
\[
B + o(1) \leq \frac{1}{x} \cdot \#\left\{n \leq x: n \in \mathcal{A}_E(Y)\right\} \leq C + o(1), 
     \mathrm{\ as\ } x \rightarrow \infty. 
\] 
That is, for the constant $Y$ taken as in \eqref{eqn_proof_tag_IneqsHold_vc}, 
we have seen that we can select 
\begin{align*} 
B & := \liminf_{x \rightarrow \infty} \frac{1}{x} \cdot \#\left\{n \leq x: n \in \mathcal{A}_E(Y)\right\} \in (0, 1) \\ 
C & := \limsup_{x \rightarrow \infty} \frac{1}{x} \cdot \#\left\{n \leq x: n \in \mathcal{A}_E(Y)\right\} \in (0, 1). 
\end{align*} 
Hence, we have shown that the necessary conditions in hypotheses of 
Theorem \ref{theorem_CondAvgOrderGInvxSummatoryFunc_v1} can in fact be achieved for all 
sufficiently large $x \rightarrow \infty$. 
\end{proof} 

\newpage
\section{Establishing lower bounds for $M(x)$ along infinite subsequences} 
\label{Section_KeyApplications} 

\subsection{The culmination of what we have done so far} 

The summation methods to weight sums of our arithmetic functions according to the sign of 
$\lambda(n)$ (or parity of $\Omega(n)$) we prove within this section are 
reminiscent of the combinatorially motivated sieve methods in 
\cite[\S 17]{OPERADECRIBERO}. 

\begin{prop}
\label{prop_Mx_SBP_IntegralFormula} 
For all sufficiently large $x$, we have that 
\begin{align} 
\label{eqn_pf_tag_v2-restated_v2} 
M(x) & \approx G^{-1}(x) - x \cdot \int_1^{x/2} \frac{G^{-1}(t)}{t^2 \cdot \log(x/t)} dt, 
\end{align} 
where $G^{-1}(x) := \sum_{n \leq x} g^{-1}(n)$ is the summatory function of $g^{-1}(n)$. 
\end{prop} 
\begin{proof} 
We know by applying Corollary \ref{cor_Mx_gInvnPixk_formula} that 
\begin{align} 
\notag
M(x) & = \sum_{k=1}^{x} g^{-1}(k) (\pi(x/k)+1) \\ 
\label{eqn_proof_tag_MxFormulaInitSepTerms_v1} 
     & = G^{-1}(x) + \sum_{k=1}^{x} g^{-1}(k) \pi(x/k), 
\end{align} 
where we can drop the asymptotically unnecessary floored integer-valued arguments to $\pi(x)$ in place of 
its approximation by the monotone non-decreasing $\pi(x) \sim \frac{x}{\log x}$. 
Moreover, we can always 
bound $$\frac{Ax}{\log x} \leq \pi(x) \leq \frac{Bx}{\log x},$$ for suitably defined 
absolute constants, $A,B > 0$. 
Therefore the approximation obtained is valid for all $x > 1$ up to a small constant difference. 

What we now require to sum and simplify the right-hand-side summation from 
\eqref{eqn_proof_tag_MxFormulaInitSepTerms_v1} is an ordinary summation by parts argument. 
Namely, we obtain that for sufficiently large 
$x \geq 2$ \footnote{
     Since $\pi(1) = 0$, the actual range of summation corresponds to 
     $k \in \left[1, \frac{x}{2}\right]$. 
}
\begin{align*} 
\sum_{k=1}^{x} g^{-1}(k) \pi(x/k) & = G^{-1}(x) \pi(1) - \sum_{k=1}^{x-1} G^{-1}(k) \left[ 
     \pi\left(\frac{x}{k}\right) - \pi\left(\frac{x}{k+1}\right)\right] \\ 
     & = -\sum_{k=1}^{x/2} G^{-1}(k) \left[ 
     \pi\left(\frac{x}{k}\right) - \pi\left(\frac{x}{k+1}\right)\right] \\ 
     & \approx -\sum_{k=1}^{x/2} G^{-1}(k) \left[ 
     \frac{x}{k \cdot \log(x/k)} - \frac{x}{(k+1) \cdot \log(x/k)}\right] \\ 
     & \approx -\sum_{k=1}^{x/2} G^{-1}(k) \frac{x}{k^2 \cdot \log(x/k)}. 
\end{align*} 
Since for $x$ large enough the summand is monotonic as $k$ ranges in order over $k \in [1, x/2]$, and 
since the summands in the last equation are smooth functions of $k$ (and $x$), and also since $G^{-1}(x)$ is 
a summatory function with jumps at the positive integers, we can approximate 
$M(x)$ for any finite $x \geq 2$ by 
\[
M(x) \approx G^{-1}(x) - x \cdot \int_1^{x/2} \frac{G^{-1}(t)}{t^2 \cdot \log(x/t)} dt. 
\]
We will later only use unsigned lower bound approximations to this function in the next theorems so that 
the signedness of the summatory function term in the integral formula above 
as $x \rightarrow \infty$ is a moot point entirely. 
\end{proof} 

\begin{proof}[Proof of Proposition \ref{prop_DensityOfGInvxPosAndBdd}] 
Suppose to the contrary that 
\[
\lim_{x \rightarrow \infty} \frac{1}{x} \cdot \#\{n \leq x: n \in \mathcal{G}_{-}\} = 0, 
\]
i.e., that $G^{-1}(x) > 0$ almost everywhere for all integers $x$ sufficiently large. 
We will utilize \eqref{eqn_proof_tag_MxFormulaInitSepTerms_v1} 
from Proposition \ref{prop_Mx_SBP_IntegralFormula} to 
derive a contradiction under this assumption. 
In particular, assuming the above limiting density is zero, we have that 
\begin{equation} 
\label{eqn_proof_tag_MxAbsValueConsequence_v1} 
\frac{|M(x)|}{x} \sim \left\lvert \int_1^{x/2} \frac{|G^{-1}(t)|}{t^2 \cdot \log(x/t)} dt - 
     \frac{|G^{-1}(x)|}{x} \right\rvert, \mathrm{\ a.e. }, \mathrm{\ as\ } x \rightarrow \infty. 
\end{equation} 
So for sufficiently large $x \rightarrow \infty$, for almost every $x$ we have that 
\begin{equation} 
\label{eqn_proof_tag_MxAbsValueConsequence_v2} 
\frac{|M(x)|}{x} \gg \left\lvert \int_1^{x/2} \frac{|\mathbb{E}[g^{-1}(t)]|}{t \cdot \log(x/t)} dt - 
     \mathbb{E}|g^{-1}(x)| \right\rvert. 
\end{equation} 
For any constant $u_0$, 
$\int_{1}^{u_0} \frac{dt}{t^2 \cdot \log(x/t)} = o(1)$ is of lower order 
growth than the primary integral contribution in 
\eqref{eqn_proof_tag_MxAbsValueConsequence_v2} as $x \rightarrow \infty$. 

We also have that  
\[
\int \frac{dt}{t \cdot \log(x/t)} = -\log\log(x/t) + C, 
\]
So since the sequence of $g^{-1}(n)$ is signed, we can bound 
$|\mathbb{E}[g^{-1}(n)]| \geq o(1) \rightarrow 0$ as $n \rightarrow \infty$. 
Combined, it follows that we can bound the right-hand-side of 
\eqref{eqn_proof_tag_MxAbsValueConsequence_v2} from below by 
\begin{equation} 
\label{eqn_proof_tag_MxAbsValueConsequence_v3} 
\frac{|M(x)|}{x} \gg \left\lvert \mathbb{E}[g^{-1}(x)] \right\rvert. 
\end{equation} 
Now since we have assumed that almost everywhere $G^{-1}(x) > 0$ when $x$ is large, 
for infinitely many sufficiently large $x$, we have that 
\begin{align} 
\notag 
\left\lvert \mathbb{E}[g^{-1}(x)] \right\rvert & = \frac{1}{x} \times \left[ 
     \sum_{\substack{n \leq x \\ \lambda(n) = +1}} |g^{-1}(n)| - 
     \sum_{\substack{n \leq x \\ \lambda(n) = -1}} |g^{-1}(n)| \right] \\ 
\notag 
     & \geq \frac{1}{x} \times \left[\sum_{n \leq \left(\frac{1}{2} + \delta_x\right) x} |g^{-1}(n)|\right] (1 + o(1)) \\ 
\label{eqn_proof_tag_MxAbsValueConsequence_v4} 
     & \geq \left(\frac{1}{2} + \delta_x\right)^{-1} \cdot 
     \mathbb{E}\left\lvert g^{-1}\left(\left(\frac{1}{2} + \delta_x\right) x\right) \right\rvert (1 + o(1)), 
\end{align} 
with $\delta_x \in \left(-\frac{1}{2}, \frac{1}{2}\right]$ for all $x$. 
The base factor term of $\frac{1}{2}$ in the upper limit of summation from the previous equation 
above corresponds to the known fact that 
\[
\lim_{x \rightarrow \infty} \frac{1}{x} \cdot \#\{n \leq x: \lambda(n) = +1\} = \frac{1}{2}. 
\]
When we apply Corollary \ref{lemma_BddExpectationOfgInvn} to 
\eqref{eqn_proof_tag_MxAbsValueConsequence_v3} and 
\eqref{eqn_proof_tag_MxAbsValueConsequence_v4}, we obtain that for infinitely many 
sufficiently large $x$ 
\begin{align*} 
\frac{|M(x)|}{x} \gg \frac{6}{\pi^2} \log\left(x\right) (1 + o(1)) 
     \xrightarrow{x \rightarrow \infty} + \infty. 
\end{align*} 
Then we recover a contradiction to the known property that $|M(x)| \leq x$ 
for all $x \geq 1$. 
as $x \rightarrow \infty$ for 
infintely many $x$. 

A similarly phrased argument shows the corresponding result is true for the set $\mathcal{G}_{+}$ on which 
$G^{-1}(x) > 0$. 
Thus, combined, these two consequences show that the limiting density of $\mathcal{G}_{-}$ is positive, 
and in particular, that it cannot tend to zero along infinitely many limiting cases 
as $x \rightarrow \infty$. 
\end{proof} 

\subsubsection{From the routine: Proofs of a few cut-and-dry results} 
\label{subsubSection_RoutineProofsNeededForMainBoundOnGInvxFunc} 

\begin{cor} 
\label{cor_ASemiForm_ForGInvx_v1} 
We have that for sufficiently large $x$, as $x \rightarrow \infty$ that 
``on average'' \footnote{ 
     %E.g., within a predictably bounded interval around each $x$ sufficiently large. 
     This distinction in the statement is necessary since our limiting lower bounds have 
     so far depended on average order estimates of certain sums and arithmetic functions 
     as $n \rightarrow \infty$. 
     %We will rely on the results proved in 
     %Section \ref{Section_ProofOfValidityOfAverageOrderLowerBounds} to justify that these 
     %lower bounds that hold on average can still be reconciled to prove 
     %the key corollary in the next subsection using an infinitely tending subsequence 
     %defined pointwise within intervals. 
} 
\begin{align*} 
\left\lvert G_E^{-1}(x) \right\rvert & \SuccSim 
     \left\lvert 
     \widehat{L}_0\left(\log\log x\right) \times \sum_{e \leq n \leq \log x} 
     \lambda(n) \cdot \log n \cdot \log\log n \right\rvert, 
\end{align*} 
where the function 
\[
\left\lvert \widehat{L}_0(x) \right\rvert \SuccSim 
     \sqrt{\frac{2}{\pi}} \cdot \frac{A_0}{3e \log 2} \cdot 
     \frac{x}{(\log\log x)^{\frac{5}{2} + \log\log x}}, 
\]
and such that $\operatorname{sgn}(\widehat{L}_0(x)) = (-1)^{\floor{\log\log x}}$ 
(as the function is defined inline below). 
\end{cor} 
\NBRef{A10-2020.04-26} 
\begin{proof} 
Using the definition in \eqref{eqn_GEInvxSummatoryFuncDef_v1}, we obtain on average that \footnote{ 
     For any arithmetic functions $f,h$, we have that \cite[\cf \S 3.10; \S 3.12]{APOSTOLANUMT} 
     \[
     \sum_{n \leq x} h(n) \times \sum_{d|n} f(d) = \sum_{d \leq x} f(d) \times \sum_{n=1}^{\Floor{x}{d}} h(dn). 
     \] 
}
\begin{align*} 
\left\lvert G_E^{-1}(x) \right\rvert & = 
     \left\lvert \sum_{n \leq \log x} \lambda(n) 
     \sum_{\substack{d|n \\ d > e}} (\log d) (\log\log d) \right\rvert \\ 
     & = \left\lvert \sum_{e < d \leq \log x} \log d \cdot \log\log d \times 
     \sum_{n=1}^{\Floor{\log x}{d}} \lambda(dn) \right\rvert. 
\end{align*} 
Now we see that by complete additivity of $\Omega(n)$ 
(multiplicativity of $\lambda(n)$) that 
\begin{align*} 
\sum_{n=1}^{\Floor{x}{d}} \lambda(dn) & = \sum_{n=1}^{\Floor{x}{d}} \lambda(d) \lambda(n) 
     = \lambda(d) \sum_{n \leq \Floor{x}{d}} \lambda(n). 
\end{align*} 
Using the result proved in Section \ref{Section_MVCh7_GzBounds} as 
(see Theorem \ref{theorem_GFs_SymmFuncs_SumsOfRecipOfPowsOfPrimes} and 
Corollary \ref{theorem_MV_Thm7.20})
we can establish that \footnote{ 
     See the proof of Lemma \ref{lemma_CLT_and_AbelSummation} below for 
     a justification of the $\gg$ bound. 
}
\begin{align*} 
\left\lvert \sum_{n \leq x} \lambda(n) \right\rvert & \gg 
     \left\lvert \sum_{k \leq \log\log x} (-1)^k \cdot \widehat{\pi}_k(x) \right\rvert 
     =: \left\lvert \widehat{L}_0(x) \right\rvert. 
\end{align*} 
For large enough $x \rightarrow \infty$ and $e \leq d \leq \log x$, 
we can easily prove by bounding each function from above and below that 
\[
\log(x/d) \sim \log x, \log\log(x/d) \sim \log\log x.  
\] 
Then we have that 
$$\left\lvert \widehat{L}_0(\log x) \right\rvert \sim 
 \left\lvert \widehat{L}_0(\log\log (x/d)) \right\rvert,$$ 
for all large $x \rightarrow \infty$ whenever $e \leq d \leq \log x$. 

We note that the precise formula for the 
limiting lower bound stated above for $\widehat{L}_0(x)$ is computed by symbolic summation 
in \emph{Mathematica} using the new bounds on $\widehat{\pi}_k(x)$ guaranteed by 
Theorem \ref{theorem_GFs_SymmFuncs_SumsOfRecipOfPowsOfPrimes} 
(and by applying subsequent standard asymptotic estimates to the resulting formulas, e.g., 
in the form of \eqref{eqn_IncompleteGamma_PropB} and Stirling's formula). 
We also have from this formula that 
$\left\lvert \widehat{L}_0(\log x) \right\rvert \gg \left\lvert \widehat{L}_0(\log\log x) \right\rvert$. 
The inner summation in the lower bound stated for $|G_E^{-1}(x)|$ is correctly indexed only for 
$n \leq \log x$ as the definition of this summatory function depends on bounds on  
$\mathbb{E}[C_{\Omega(n)}(n)]$ from below for $n \leq x$ 
where the functions $C_k(n)$ are only non-zero for large 
$n \geq 1$ when $k \leq \Omega(n) \ll \log x$ (e.g., the upper bound on $\Omega(n)$ 
is valid up to a constant factor). 
\end{proof} 

\begin{lemma} 
\label{lemma_CLT_and_AbelSummation} 
Suppose that $f_k(n)$ is a sequence of arithmetic functions 
such that $f_k(n) > 0$ for all $n > u_0$ and $1 \leq k \leq \Omega(n)$ where 
$f_{\Omega(n)}(n) \SuccSim \widehat{\tau}_{\ell}(n)$ as $n \rightarrow \infty$. We suppose that 
the bounding function $\widehat{\tau}_{\ell}(t)$ is a continuously differentiable function of $t$ for all 
large enough $t \gg u_0$ \footnote{ 
     We will require that $\widehat{\tau}_{\ell}(t) \in C^{1}(\mathbb{R})$ when we apply the 
     Abel summation formula in the proof of Theorem \ref{theorem_gInv_GeneralAsymptoticsForms}. 
     At this point, it is technically an unnecessary condition that is 
     vacously satisfied by assumption (by requirement) 
     and will importantly need to hold only when we specialize to the 
     actual functions employed to form our new bounds in the theorem proof below. 
}.  
We define the $\lambda$-sign-scaled summatory function of $f$ as follows: 
\[
F_{\lambda}(x) := \sum_{\substack{u_0 < n \leq \log x}} 
     \lambda(n) \cdot f_{\Omega(n)}(n). 
\]
Let 
\[
A_{\Omega}^{(\ell)}(t) := \sum_{k=1}^{\floor{\log\log t}} (-1)^k \widehat{\pi}_k^{(\ell)}(t),  
\]
where $\widehat{\pi}_k(x) \geq \widehat{\pi}_k^{(\ell)}(x) \geq 0$ for 
$\widehat{\pi}_k^{(\ell)}(t)$ some smooth monotone non-decreasing 
function of $t$ whenever $t$ sufficiently large. 
Then we have that on average 
\[
|F_{\lambda}(x)| \SuccSim \left\lvert 
     A_{\Omega}^{(\ell)}(\log x) \widehat{\tau}_{\ell}(\log x) - 
     \int_{u_0}^{\log x} 
     A_{\Omega}^{(\ell)}(t) \widehat{\tau}_{\ell}^{\prime}(t) dt 
     \right\rvert.  
\]
\end{lemma}
\begin{proof} 
We can form an accurate $C^{1}(\mathbb{R})$ approximation by the smoothness of 
$\widehat{\pi}_k^{(\ell)}(x)$ that allows us to apply the Abel summation formula using the summatory 
function $A_{\Omega}^{(\ell)}(t)$ for $t$ on any bounded connected subinterval of $[1, \infty)$. 
The second stated formula for $F_{\lambda}(x)$ above is valid by Abel summation whenever 
\[
0 \leq \left\lvert \frac{\displaystyle\sum\limits_{\log\log t < k \leq \frac{\log t}{\log 2}} 
     (-1)^k \widehat{\pi}_k(t)}{A_{\Omega}^{(\ell)}(t)}\right\rvert \ll 2, 
     \mathrm{\ as\ } t \rightarrow \infty. 
\]
What the last equation implies is that the asymptotically dominant terms indicating the parity of 
$\lambda(n)$ are captured up to a constant factor 
by the terms in the range over $k$ summed by 
$A_{\Omega}^{(\ell)}(t)$ for 
sufficiently large $t \rightarrow \infty$. 
Using the arguments in Montgomery and Vaughan \cite[\S 7; Thm.\ 7.20]{MV} used to prove 
Corollary \ref{theorem_MV_Thm7.20}, we can see that 
the assertion above holds in the limit as $t \rightarrow \infty$. 
\end{proof} 

In other words, taking the sum over the summands that defines $A_{\Omega}(x)$ only over the truncated range of 
$k \in [1, \log\log x]$ does not non-trivially change the limiting asymptotically 
dominant terms in the lower bound obtained from using this form of the summatory function in 
conjunction with the Abel summation formula. This property holds even when we should technically 
index over all $k \in [1, \log_2(x)]$ to obtain an exact formula for the summatory weight function. 
The results in Corollary \ref{cor_ASemiForm_ForGInvx_v1} and in 
Lemma \ref{lemma_CLT_and_AbelSummation} combine to provide a key formula used in the 
proof of Theorem \ref{theorem_gInv_GeneralAsymptoticsForms} to bound $G^{-1}(x)$ from 
below in the average case sense. 

We require one more sanity check to our approximations 
used in that proof explored in the next subsection stated in the form of the next lemma. 
Observe that we now use the superscript and subscript notation of 
$(\ell)$ not to denote a formal parameter to 
the functions we define below, but instead to denote that these functions form 
\emph{lower bound} (rather than exact) 
approximations to other forms of the functions without the scripted $(\ell)$. 

\begin{lemma} 
\label{lemma_lowerBoundsOnLambdaFuncParitySummFuncs} 
Suppose that $\widehat{\pi}_k(x) \geq \widehat{\pi}_k^{(\ell)}(x) \geq 0$ 
with $\widehat{\pi}_k^{(\ell)}(x)$ a monotone non-decreasing real-valued function 
for all sufficiently large $x$. 
Let 
\begin{align*} 
A_{\Omega}^{(\ell)}(x) & := \sum_{k \leq \log\log x} (-1)^k \widehat{\pi}_k^{(\ell)}(x) \\ 
A_{\Omega}(x) & := \sum_{k \leq \log\log x} (-1)^k \widehat{\pi}_k(x). 
\end{align*} 
Then for all sufficiently large $x$, we have that 
$$|A_{\Omega}(x)| \gg |A_{\Omega}^{(\ell)}(x)|.$$ 
\end{lemma} 
\begin{proof} 
Given an explicit smooth lower bounding function, $\widehat{\pi}_k^{(\ell)}(x)$, we define the 
similarly smooth and monotone residual terms in approximating $\widehat{\pi}_k(x)$ 
using the following notation: 
\[
\widehat{\pi}_k(x) = \widehat{\pi}_k^{(\ell)}(x) + \widehat{E}_k(x). 
\]
Then we can form the ordinary exact form of the summatory function $A_{\Omega}$ as 
\begin{align*} 
|A_{\Omega}(x)| & = \left\lvert \sum_{k \leq \frac{\log\log x}{2}} 
     \left[\widehat{\pi}_{2k}(x) - \widehat{\pi}_{2k-1}(x)\right] \right\rvert \\ 
     & \geq \left\lvert A_{\Omega}^{(\ell)}(x) - \sum_{k \leq \frac{\log\log x}{2}} \widehat{E}_{2k-1}(x) 
     \right\rvert \\ 
     & \geq 
     \left\lvert A_{\Omega}^{(\ell)}(x) \right\rvert - 
     \left\lvert \sum_{k \leq \frac{\log\log x}{2}} \widehat{E}_{2k-1}(x) 
     \right\rvert. 
\end{align*} 
If the latter sum, denoted 
$$\operatorname{ES}(x) := \sum_{k \leq \frac{\log\log x}{2}} \widehat{E}_{2k-1}(x) \rightarrow \infty,$$ as 
$x \rightarrow \infty$, then we can always find some absolute $C_0 > 0$ (by monotonicity) such that 
$\operatorname{ES}(x) \leq C_0 \cdot A_{\Omega}(x)$. If on the other hand this sum becomes constant as 
$x \rightarrow +\infty$, then we also clearly have another absolute $C_1 > 0$ such that 
$|A_{\Omega}(x)| \geq C_1 \cdot |A_{\Omega}^{(\ell)}(x)|$. 
In either case, the claimed result holds for all large enough $x$. 
\end{proof} 

\subsubsection{A proof of the key bound from below on $G^{-1}(x)$} 

The next central theorem is the last key barrier required to prove 
Corollary \ref{cor_ThePipeDreamResult_v1} 
in the next subsection. For the time being, we will keep track of extraneous positive 
constants that will be dropped when we prove the corollary. 

\begin{theorem}[Asymptotics and bounds for the summatory functions $G^{-1}(x)$] 
\label{theorem_gInv_GeneralAsymptoticsForms}
We define a lower summatory function, $G_{\ell}^{-1}(x)$, 
to provide bounds on the magnitude of $G_E^{-1}(x)$ such that 
$$|G_{\ell}^{-1}(x)| \leq |G_E^{-1}(x)|,$$ 
for all sufficiently large $x \geq e$. 
Let $C_{\ell,1} > 0$ be the absolute constant defined by 
\[
C_{\ell,1} = \frac{8 A_0^2}{9 \pi e^2 \log^2(2)}  = 
     \frac{256 \cdot 2^{1/8}}{59049 \cdot \pi^2 e^8 \log^8(2)} 
     \exp\left(-\frac{55}{2} \log^2(2)\right) 
     \approx 5.51187 \times 10^{-12}.  
\]
We obtain the following limiting estimate for the bounding function 
$G_{\ell}^{-1}(x)$ ``on average'' as $x \rightarrow \infty$:   
\begin{align*} 
 & \left\lvert G_{\ell}^{-1}\left(x\right) \right\rvert
     \SuccSim 
     \frac{C_{\ell,1} \cdot (\log x) (\log\log x)^{4} \sqrt{\log\log\log x}}{ 
     (\log\log\log\log x)^{\frac{5}{2}}}. 
\end{align*} 
\end{theorem} 
\NBRef{A10-2020.04-26} 
\begin{proof} 
Recall from our proof of Corollary \ref{cor_BoundsOnGz_FromMVBook_initial_stmt_v1} that 
a lower bound on the variant prime form counting function, $\widehat{\pi}_k(x)$, is given by 
\[
\widehat{\pi}_k(x) \SuccSim \frac{A_0 \cdot x}{\log x \cdot (\log\log x)^4 \cdot (k-1)!} \cdot 
     \left(\frac{4}{3e\log 2}\right)^{\frac{k}{\log\log x}}, \mathrm{\ as\ } x \rightarrow \infty. 
\]
So we can then form a lower summatory function indicating the signed contributions over the distinct 
parity of $\Omega(n)$ for all $n \leq x$ as follows by applying 
\eqref{eqn_IncompleteGamma_PropA} and Stirling's approximation: 
\begin{align} 
\label{proof_thm_GInvFunc_v0} 
\left\lvert A_{\Omega}^{(\ell)}(t) \right\rvert & = 
     \left\lvert \sum_{k \leq \log\log t} (-1)^k \widehat{\pi}_k(t) \right\rvert \\ 
\notag
     & \SuccSim  
     \sqrt{\frac{2}{\pi}} \cdot \frac{A_0}{3e \log 2} \cdot 
     \frac{t}{(\log\log t)^{\frac{5}{2} + \log\log t}}. 
\end{align} 
The actual sign on this function is given by 
$\operatorname{sgn}(A_{\Omega}^{(\ell)}(t)) = (-1)^{\floor{\log\log t}}$ 
(see Lemma \ref{lemma_lowerBoundsOnLambdaFuncParitySummFuncs}). 

By Corollary \ref{theorem_Ckn_GeneralAsymptoticsForms} 
we recover from the bounded main term approximation to $\mathbb{E}[C_{\Omega(n)}(n)]$ proved in 
Section \ref{Section_ProofOfValidityOfAverageOrderLowerBounds}, denoted here by the smooth function 
$\widehat{\tau}_0(t) = \log t \cdot \log\log t$, that 
\begin{align*} 
\widehat{\tau}_0^{\prime}(t) & = \frac{d}{dt}\left[ 
     \log t \cdot \log\log t 
     \right] \SuccSim \frac{\log\log t}{t}. 
\end{align*} 
As prescribed by Lemma \ref{lemma_CLT_and_AbelSummation} and Corollary \ref{cor_ASemiForm_ForGInvx_v1}, 
we apply Abel summation to imply that we have 
\begin{equation} 
\label{proof_thm_GInvFunc_v1} 
G_{\ell}^{-1}(x) = \widehat{L}_0(\log\log x) \left[
     \widehat{\tau}_0(\log x) A_{\Omega}^{(\ell)}(\log x) - 
     \int_{u_0}^{\log x} 
     \widehat{\tau}_0^{\prime}(t) A_{\Omega}^{(\ell)}(t) dt\right]. 
\end{equation} 
The inner integral term on the rightmost side of \eqref{proof_thm_GInvFunc_v1} 
is summed approximately in the form of 
\begin{align} 
\label{eqn_proof_thm_GInvFunc_v3_approx} 
-\int_{u_0}^{\log x} \widehat{\tau}_0^{\prime}(t) A_{\Omega}^{(\ell)}(t) dt & \approx 
     \sum_{k=u_0+1}^{\frac{1}{2}\log\log\log x} \left( 
     I_{\ell}\left(e^{e^{2k+1}}\right) - 
     I_{\ell}\left(e^{e^{2k}}\right) 
     \right) e^{e^{2k}} \\ 
\notag 
     & \approx 
     C_0(u_0) + 
     (-1)^{\Floor{\log\log\log x}{2}} \times 
     \int_{\frac{\log\log\log x}{2}-\frac{1}{2}}^{\frac{\log\log\log x}{2}} 
     I_{\ell}\left(e^{e^{2k}}\right) 
     e^{e^{2k}} dk. 
\end{align} 
We define the integrand function, 
$I_{\ell}(t) := \widehat{\tau}_0^{\prime}(t) A_{\Omega}^{(\ell)}(t)$, 
as in the previous equations with some limiting simplifications for the 
$k \in \left[\frac{\log\log\log x}{2}-1, \frac{\log\log\log x}{2}\right]$ as 
\begin{align} 
\label{eqn_proof_thm_GInvFunc_v3_approx} 
I_{\ell}\left(e^{e^{2k}}\right) e^{e^{2k}}& \SuccSim 
     \frac{A_0}{3e \sqrt{\pi} \log 2} \cdot \frac{\exp\left(e^{2k}\right)}{2^{2k} \cdot k^{2k+3/2}} 
     =: \widehat{I}_{\ell}(k). 
\end{align} 
So using the lower bound on the increasing integrand in 
\eqref{eqn_proof_thm_GInvFunc_v3_approx}, we find from the mean value theorem 
that \footnote{ 
     We have invoked the simplifications that for sufficiently large $x$, 
     \[
     \exp\left(-\log\log\log x \cdot \log\log\log\log x\right) \SuccSim 
          \exp\left(-(\log\log\log x)^2\right) \SuccSim 
          (\log\log x)^2,  
     \]
     and 
     \[
     \exp\left(-\log\log\log\log x \cdot \log\log\log\log\log x\right) \SuccSim 
          \exp\left(-(\log\log\log\log x)^2\right) \SuccSim 
          (\log\log\log x)^2.  
     \]
} 
\begin{align} 
\notag 
\Biggl\lvert \widehat{L}_0(\log\log x) & 
     \times \int_{\frac{\log\log\log x}{2}-\frac{1}{2}}^{\frac{\log\log\log x}{2}} 
     I_{\ell}\left(e^{e^{2k}}\right) 
     e^{e^{2k}} dk \Biggr\rvert \\ 
\notag 
     & \gg \Biggl\lvert \frac{\widehat{L}_0(\log\log x)}{2} \times  
     \widehat{I}_{\ell}\left(\frac{\log\log\log x}{2}-\frac{1}{2}\right)
     \Biggr\rvert \\ 
\notag 
     & \gg \frac{C_{\ell,1} \cdot (\log x)^{\frac{1}{e}} (\log\log x)}{2 \cdot 
     (\log\log\log x)^{\frac{1}{2} + \log\log\log x} 
     (\log\log\log\log x)^{\frac{5}{2} + \log\log\log\log x}} \\ 
\label{eqn_proof_thm_GInvFunc_v4_approx} 
     & \SuccSim  
     \frac{C_{\ell,1} \cdot (\log x)^{\frac{1}{e}} (\log\log x)^{3} (\log\log\log x)^{\frac{3}{2}}}{ 
     2 \cdot (\log\log\log\log x)^{\frac{5}{2}}}. 
\end{align} 
Moreover, by evaluating $\widehat{I}_{\ell}(t)$ at the upper bound on the integral above, we can 
similarly conclude that 
\[
\Biggl\lvert \widehat{L}_0(\log\log x) 
     \times \int_{\frac{\log\log\log x}{2}-\frac{1}{2}}^{\frac{\log\log\log x}{2}} 
     I_{\ell}\left(e^{e^{2k}}\right) 
     e^{e^{2k}} dk \Biggr\rvert \ll 
     \frac{C_{\ell,1} \cdot (\log x)^{\frac{1}{e}} (\log\log x)}{2 \cdot 
     (\log\log\log x)^{\frac{1}{2} + \log\log\log x} 
     (\log\log\log\log x)^{\frac{5}{2} + \log\log\log\log x}}. 
\]
It is then clear from our prior computations of the growth of 
$A_{\Omega}^{(\ell)}(x)$ and $\widehat{\tau}_0(x)$ 
that the asymptotically dominant behavior of the bound for 
$|G_{\ell}^{-1}(x)|$ cannot correspond to the integral term calculated in 
\eqref{eqn_proof_thm_GInvFunc_v4_approx}. 

To make this observation precise, consider the following expansion for the leading term in 
the Abel summation formula from \eqref{proof_thm_GInvFunc_v1} for comparison with 
\eqref{eqn_proof_thm_GInvFunc_v4_approx}: 
\begin{align} 
\notag 
\left\lvert \widehat{L}_0(\log\log x) \widehat{\tau}_0(\log x) A_{\Omega}^{(\ell)}(\log x) \right\rvert 
\notag 
     & \gg \frac{C_{\ell,1} \cdot (\log x) (\log\log x)^2}{(\log\log\log x)^{\frac{3}{2} + \log\log\log x} 
     (\log\log\log\log x)^{\frac{5}{2} + \log\log\log\log x}} \\ 
\label{eqn_proof_thm_GInvFunc_v5_approx} 
     & \SuccSim 
     \frac{C_{\ell,1} \cdot (\log x) (\log\log x)^{4} \sqrt{\log\log\log x}}{ 
     (\log\log\log\log x)^{\frac{5}{2}}}. 
\end{align} 
We have used the same simplifications noted in the footnote annotated as above in arriving at the 
limiting lower bound given in the previous equation. 
\end{proof} 

%\begin{remark} 
%A good sign of the correctness of our proof given here is that up to a small rational non-zero 
%constant factor distinction, the two terms in 
%\eqref{eqn_proof_thm_GInvFunc_v4_approx} and \eqref{eqn_proof_thm_GInvFunc_v5_approx} 
%summed to contribute asymptotic weight to 
%\eqref{proof_thm_GInvFunc_v1} nearly coincide. 
%Such a thin near coincidence of these terms, one of which forms a product and the other a 
%scaled integral, over the same function triplet should be viewed as a rarity of a  
%phenomenon we just happen to capture exactly by our computations in the proof above. 
%Note that the respective leading 
%sign on the two terms contributing weight to the asymptotics of 
%\eqref{proof_thm_GInvFunc_v1} is given by 
%$\pm (-1)^{\floor{\log\log\log x} + \floor{\log\log\log\log x}}$. 
%\end{remark} 

\subsection{Proof of the unboundedness of the scaled Mertens function along infinite subsequences}
\label{subSection_TheCoreResultProof} 

What we will have shown in total concluding the proof of 
Corollary \ref{cor_ThePipeDreamResult_v1} below is the classically conjectured 
unboundedness property of $M(x)$ in the form of 
\[
\limsup_{x \rightarrow \infty} \frac{|M(x)|}{\sqrt{x}} = +\infty. 
\]
This statement comprises a better than previously known rate of the minimal asymptotic tendencies of 
$|M(x)| / \sqrt{x}$ towards unboundedness along an infinite subsequence. 
Note that this result is still a much weaker condition than the RH as stated. Moreover, 
we must again take time to emphasize that its construction is entirely much differently 
motivated through the encouraging combinatorial structures and additive functions 
we have observed in our new formulas. 

Now we finally address the main conclusion of our arguments given so far: 

\begin{proof}[Proof of Corollary \ref{cor_ThePipeDreamResult_v1}] 
\label{proofOf_cor_ThePipeDreamResult_v1} 
We break up the integral term in 
Proposition \ref{prop_Mx_SBP_IntegralFormula} 
over $t \in [u_0, x/2]$ into two pieces: one that is easily bounded 
from $u_0 \leq t \leq \sqrt{x}$, 
and then another that will conveniently give us our slow-growing tendency towards 
infinity along the subsequence when evaluated using 
Theorem \ref{theorem_gInv_GeneralAsymptoticsForms}. 

We can apply Proposition \ref{prop_Mx_SBP_IntegralFormula} to see that 
for some $x_0 \in \left[\sqrt{x}, \frac{x}{2}\right]$ such that 
\[
\left\lvert G^{-1}(x_0) \right\rvert := 
     \min_{\sqrt{x} \leq t \leq \frac{x}{2}} |G^{-1}(t)|, 
\]
we can bound 
\begin{align} 
\notag 
\frac{|M(x)|}{\sqrt{x}} & = 
     \frac{1}{\sqrt{x}} \left\lvert G^{-1}(x) - x \cdot \int_1^{x/2} \frac{G^{-1}(t)}{ 
     t^2 \cdot \log(x/t)} dt \right\rvert \\ 
\notag 
     & \gg 
     \left\lvert \sqrt{x} \times \int_{\sqrt{x}}^{x/2} \frac{G^{-1}(t)}{ 
     t^2 \cdot \log(x/t)} dt \right\rvert \\ 
\notag 
     & \gg \left\lvert \int_{\sqrt{x_0}}^{\frac{x}{2}} \frac{2 \sqrt{x_0}}{ 
     t^2 \cdot \log\left(x_0\right)} dt \right\rvert 
     \times \left( 
     \min_{\sqrt{x} \leq t \leq \frac{x}{2}} |G^{-1}(t)| 
     \right) \\ 
\label{eqn_MxGInvxLowerBound_stmt_v1} 
     & \gg  
     \frac{2 \left\lvert G^{-1}(x_0) \right\rvert}{\log\left(x_0\right)}. 
\end{align} 
When we assume that $x \mapsto x_y$ is taken along the 
subsequence defined within the intervals defined above, we can transform the bound in the last 
equation into a statement about a lower bound for $|M(x)| / \sqrt{x}$ along an infinitely tending 
subsequence. 
For sufficiently large $y$, this subsequence is guaranteed to exist by our proof of 
Theorem \ref{theorem_CondAvgOrderGInvxSummatoryFunc_v1} 
using the methods we have developed to establish it and the necessary hypotheses in 
Section \ref{Section_ProofOfValidityOfAverageOrderLowerBounds}. 

In particular, the existence of this infinite subsequence 
shows that there is some $x_y$ for each large enough $y \rightarrow \infty$ 
such that $|G^{-1}(x_y)| \gg |G_E^{-1}(x_y)| \gg |G_{\ell}^{-1}(x_y)|$ where 
$x_y \rightarrow \infty$ as $y \rightarrow \infty$. So 
\begin{align} 
\notag 
\frac{|M(x)|}{\sqrt{x}} & \gg \frac{2 \left\lvert G^{-1}(x_0) \right\rvert}{\log\left(x_0\right)} \\ 
\label{eqn_MxGInvxLowerBound_stmt_v2} 
     & \gg \frac{2 C_{\ell,1} \cdot (\log\log x_y)^{4} \sqrt{\log\log\log x_y}}{ 
     (\log\log\log\log x_y)^{\frac{5}{2}}}. 
\end{align} 
We want this sequence $\{x_y\}_{y \geq Y_0}$ for $Y_0$ sufficiently large to correspond to 
$x_y \equiv x_0$ for $x_0$ as in \eqref{eqn_MxGInvxLowerBound_stmt_v1} above. 
Let $\varepsilon_0 := \frac{1}{2} \min(B, 1-C)$ for $B, C \in (0, 1)$ as in the hypotheses of 
Theorem \ref{theorem_CondAvgOrderGInvxSummatoryFunc_v1}. 
Then we have 
if we take $x \mapsto x_{0,y}$ with $x_{0,y} := \exp\left(2e^{e^{e^{2y}}}\right)$, 
we recover from \eqref{eqn_MxGInvxLowerBound_stmt_v2} that 
\begin{align} 
\label{eqn_MxGInvxLowerBound_stmt_v3} 
\frac{|M(x_{0,y})|}{\sqrt{x_{0,y}}} & \gg 
     \frac{2 C_{\ell,1} \cdot (\log\log \left[(B-\varepsilon_0) \sqrt{x_{0,y}}\right])^{4} 
     \sqrt{ \log\log\log \left[(B-\varepsilon_0) \sqrt{x_{0,y}}\right] } }{ 
     (\log\log\log\log \left[(B-\varepsilon_0) \sqrt{x_{0,y}}\right])^{\frac{5}{2}}} \\ 
\notag 
     & \gg 
      \frac{2 C_{\ell,1} \cdot (\log\log \sqrt{x_{0,y}})^{4} 
     \sqrt{ \log\log\log \sqrt{x_{0,y}} } }{ 
     (\log\log\log\log \sqrt{x_{0,y}})^{\frac{5}{2}}}
     \xrightarrow{y \rightarrow \infty} +\infty, 
\end{align} 
along this infinitely tending subsequence. Thus the scaled Mertens function is 
unbounded in the limit supremum sense, as we have claimed. 
\end{proof} 

\newpage 
\renewcommand{\refname}{References} 
\bibliography{glossaries-bibtex/thesis-references}{}
\bibliographystyle{plain}

\newpage
\setcounter{section}{0} 
\renewcommand{\thesection}{T.\arabic{section}} 

\newpage
\section{Table: The Dirichlet inverse function $g^{-1}(n)$ and its summatory function} 
\label{table_conjecture_Mertens_ginvSeq_approx_values}

\begin{table}[h!]

\centering

\tiny
\begin{equation*}
\boxed{
\begin{array}{|cc|c|ccc|c|c|ccc|c|ccc}
 n & \mathbf{Primes} & & \mathbf{Sqfree} & \mathbf{PPower} & \bar{\mathbb{S}} & & g^{-1}(n) & 
 \lambda(n) \operatorname{sgn}(g^{-1}(n)) & \lambda(n) g^{-1}(n) - \widehat{f}_1(n) & 
 \frac{\sum\limits_{d|n} C_{\Omega(d)}(d)}{|g^{-1}(n)|} & & G^{-1}(n) & G^{-1}_{+}(n) & G^{-1}_{-}(n) \\ \hline 
 1 & 1^1 & \text{--} & \text{Y} & \text{N} & \text{N} & \text{--} & 1 & 1 & 0 & 1.0000000 & \text{--} & 1 & 1 & 0 \\
 2 & 2^1 & \text{--} & \text{Y} & \text{Y} & \text{N} & \text{--} & -2 & 1 & 0 & 1.0000000 & \text{--} & -1 & 1 & -2 \\
 3 & 3^1 & \text{--} & \text{Y} & \text{Y} & \text{N} & \text{--} & -2 & 1 & 0 & 1.0000000 & \text{--} & -3 & 1 & -4 \\
 4 & 2^2 & \text{--} & \text{N} & \text{Y} & \text{N} & \text{--} & 2 & 1 & 0 & 1.5000000 & \text{--} & -1 & 3 & -4 \\
 5 & 5^1 & \text{--} & \text{Y} & \text{Y} & \text{N} & \text{--} & -2 & 1 & 0 & 1.0000000 & \text{--} & -3 & 3 & -6 \\
 6 & 2^1 3^1 & \text{--} & \text{Y} & \text{N} & \text{N} & \text{--} & 5 & 1 & 0 & 1.0000000 & \text{--} & 2 & 8 & -6 \\
 7 & 7^1 & \text{--} & \text{Y} & \text{Y} & \text{N} & \text{--} & -2 & 1 & 0 & 1.0000000 & \text{--} & 0 & 8 & -8 \\
 8 & 2^3 & \text{--} & \text{N} & \text{Y} & \text{N} & \text{--} & -2 & 1 & 0 & 2.0000000 & \text{--} & -2 & 8 & -10 \\
 9 & 3^2 & \text{--} & \text{N} & \text{Y} & \text{N} & \text{--} & 2 & 1 & 0 & 1.5000000 & \text{--} & 0 & 10 & -10 \\
 10 & 2^1 5^1 & \text{--} & \text{Y} & \text{N} & \text{N} & \text{--} & 5 & 1 & 0 & 1.0000000 & \text{--} & 5 & 15 & -10 \\
 11 & 11^1 & \text{--} & \text{Y} & \text{Y} & \text{N} & \text{--} & -2 & 1 & 0 & 1.0000000 & \text{--} & 3 & 15 & -12 \\
 12 & 2^2 3^1 & \text{--} & \text{N} & \text{N} & \text{Y} & \text{--} & -7 & 1 & 2 & 1.2857143 & \text{--} & -4 & 15 & -19 \\
 13 & 13^1 & \text{--} & \text{Y} & \text{Y} & \text{N} & \text{--} & -2 & 1 & 0 & 1.0000000 & \text{--} & -6 & 15 & -21 \\
 14 & 2^1 7^1 & \text{--} & \text{Y} & \text{N} & \text{N} & \text{--} & 5 & 1 & 0 & 1.0000000 & \text{--} & -1 & 20 & -21 \\
 15 & 3^1 5^1 & \text{--} & \text{Y} & \text{N} & \text{N} & \text{--} & 5 & 1 & 0 & 1.0000000 & \text{--} & 4 & 25 & -21 \\
 16 & 2^4 & \text{--} & \text{N} & \text{Y} & \text{N} & \text{--} & 2 & 1 & 0 & 2.5000000 & \text{--} & 6 & 27 & -21 \\
 17 & 17^1 & \text{--} & \text{Y} & \text{Y} & \text{N} & \text{--} & -2 & 1 & 0 & 1.0000000 & \text{--} & 4 & 27 & -23 \\
 18 & 2^1 3^2 & \text{--} & \text{N} & \text{N} & \text{Y} & \text{--} & -7 & 1 & 2 & 1.2857143 & \text{--} & -3 & 27 & -30 \\
 19 & 19^1 & \text{--} & \text{Y} & \text{Y} & \text{N} & \text{--} & -2 & 1 & 0 & 1.0000000 & \text{--} & -5 & 27 & -32 \\
 20 & 2^2 5^1 & \text{--} & \text{N} & \text{N} & \text{Y} & \text{--} & -7 & 1 & 2 & 1.2857143 & \text{--} & -12 & 27 & -39 \\
 21 & 3^1 7^1 & \text{--} & \text{Y} & \text{N} & \text{N} & \text{--} & 5 & 1 & 0 & 1.0000000 & \text{--} & -7 & 32 & -39 \\
 22 & 2^1 11^1 & \text{--} & \text{Y} & \text{N} & \text{N} & \text{--} & 5 & 1 & 0 & 1.0000000 & \text{--} & -2 & 37 & -39 \\
 23 & 23^1 & \text{--} & \text{Y} & \text{Y} & \text{N} & \text{--} & -2 & 1 & 0 & 1.0000000 & \text{--} & -4 & 37 & -41 \\
 24 & 2^3 3^1 & \text{--} & \text{N} & \text{N} & \text{Y} & \text{--} & 9 & 1 & 4 & 1.5555556 & \text{--} & 5 & 46 & -41 \\
 25 & 5^2 & \text{--} & \text{N} & \text{Y} & \text{N} & \text{--} & 2 & 1 & 0 & 1.5000000 & \text{--} & 7 & 48 & -41 \\
 26 & 2^1 13^1 & \text{--} & \text{Y} & \text{N} & \text{N} & \text{--} & 5 & 1 & 0 & 1.0000000 & \text{--} & 12 & 53 & -41 \\
 27 & 3^3 & \text{--} & \text{N} & \text{Y} & \text{N} & \text{--} & -2 & 1 & 0 & 2.0000000 & \text{--} & 10 & 53 & -43 \\
 28 & 2^2 7^1 & \text{--} & \text{N} & \text{N} & \text{Y} & \text{--} & -7 & 1 & 2 & 1.2857143 & \text{--} & 3 & 53 & -50 \\
 29 & 29^1 & \text{--} & \text{Y} & \text{Y} & \text{N} & \text{--} & -2 & 1 & 0 & 1.0000000 & \text{--} & 1 & 53 & -52 \\
 30 & 2^1 3^1 5^1 & \text{--} & \text{Y} & \text{N} & \text{N} & \text{--} & -16 & 1 & 0 & 1.0000000 & \text{--} & -15 & 53 & -68 \\
 31 & 31^1 & \text{--} & \text{Y} & \text{Y} & \text{N} & \text{--} & -2 & 1 & 0 & 1.0000000 & \text{--} & -17 & 53 & -70 \\
 32 & 2^5 & \text{--} & \text{N} & \text{Y} & \text{N} & \text{--} & -2 & 1 & 0 & 3.0000000 & \text{--} & -19 & 53 & -72 \\
 33 & 3^1 11^1 & \text{--} & \text{Y} & \text{N} & \text{N} & \text{--} & 5 & 1 & 0 & 1.0000000 & \text{--} & -14 & 58 & -72 \\
 34 & 2^1 17^1 & \text{--} & \text{Y} & \text{N} & \text{N} & \text{--} & 5 & 1 & 0 & 1.0000000 & \text{--} & -9 & 63 & -72 \\
 35 & 5^1 7^1 & \text{--} & \text{Y} & \text{N} & \text{N} & \text{--} & 5 & 1 & 0 & 1.0000000 & \text{--} & -4 & 68 & -72 \\
 36 & 2^2 3^2 & \text{--} & \text{N} & \text{N} & \text{Y} & \text{--} & 14 & 1 & 9 & 1.3571429 & \text{--} & 10 & 82 & -72 \\
 37 & 37^1 & \text{--} & \text{Y} & \text{Y} & \text{N} & \text{--} & -2 & 1 & 0 & 1.0000000 & \text{--} & 8 & 82 & -74 \\
 38 & 2^1 19^1 & \text{--} & \text{Y} & \text{N} & \text{N} & \text{--} & 5 & 1 & 0 & 1.0000000 & \text{--} & 13 & 87 & -74 \\
 39 & 3^1 13^1 & \text{--} & \text{Y} & \text{N} & \text{N} & \text{--} & 5 & 1 & 0 & 1.0000000 & \text{--} & 18 & 92 & -74 \\
 40 & 2^3 5^1 & \text{--} & \text{N} & \text{N} & \text{Y} & \text{--} & 9 & 1 & 4 & 1.5555556 & \text{--} & 27 & 101 & -74 \\
 41 & 41^1 & \text{--} & \text{Y} & \text{Y} & \text{N} & \text{--} & -2 & 1 & 0 & 1.0000000 & \text{--} & 25 & 101 & -76 \\
 42 & 2^1 3^1 7^1 & \text{--} & \text{Y} & \text{N} & \text{N} & \text{--} & -16 & 1 & 0 & 1.0000000 & \text{--} & 9 & 101 & -92 \\
 43 & 43^1 & \text{--} & \text{Y} & \text{Y} & \text{N} & \text{--} & -2 & 1 & 0 & 1.0000000 & \text{--} & 7 & 101 & -94 \\
 44 & 2^2 11^1 & \text{--} & \text{N} & \text{N} & \text{Y} & \text{--} & -7 & 1 & 2 & 1.2857143 & \text{--} & 0 & 101 & -101 \\
 45 & 3^2 5^1 & \text{--} & \text{N} & \text{N} & \text{Y} & \text{--} & -7 & 1 & 2 & 1.2857143 & \text{--} & -7 & 101 & -108 \\
 46 & 2^1 23^1 & \text{--} & \text{Y} & \text{N} & \text{N} & \text{--} & 5 & 1 & 0 & 1.0000000 & \text{--} & -2 & 106 & -108 \\
 47 & 47^1 & \text{--} & \text{Y} & \text{Y} & \text{N} & \text{--} & -2 & 1 & 0 & 1.0000000 & \text{--} & -4 & 106 & -110 \\
 48 & 2^4 3^1 & \text{--} & \text{N} & \text{N} & \text{Y} & \text{--} & -11 & 1 & 6 & 1.8181818 & \text{--} & -15 & 106 & -121 \\
\end{array}
}
\end{equation*}

\bigskip\hrule\smallskip 

\caption*{\textbf{\rm \bf Table \thesection:} 
          \textbf{Computations of $\mathbf{g^{-1}(n) \equiv (\omega+1)^{-1}(n)}$ and related functions 
          for $\mathbf{1 \leq n \leq 500}$.} \\ 
          The column labeled \texttt{Primes} provides the prime factorization of each $n$ so that the values of 
          $\omega(n)$ and $\Omega(n)$ are easily extracted. The columns labeled, respectively, \texttt{Sqfree}, \texttt{PPower} and 
          $\bar{\mathbb{S}}$ list inclusion of $n$ in the sets of squarefree integers, prime powers, and the set $\bar{\mathbb{S}}$ 
          that denotes the positive integers $n$ which are neither squarefree nor prime powers. \\[0.05cm]
          The next two columns provide the 
          explicit values of the inverse function $g^{-1}(n)$ and indicate that the sign of this function at $n$ is given by 
          $\lambda(n)$. \\[0.05cm] 
          The next column shows the small order magnitude differences between the unsigned 
          magnitude of $g^{-1}(n)$ and the summations $\widehat{f}_1(n) := \sum_{k \geq 0} \binom{\omega(n)}{k} \cdot k!$. 
          The following column in order shows the ratio of $\sum_{d|n} C_{\Omega(d)}(d) / |g^{-1}(n)|$ 
          approximated numerically. \\[0.05cm] 
          The last three 
          columns show the summatory function of $g^{-1}(n)$, $G^{-1}(x) := \sum_{n \leq x} g^{-1}(n)$, decomposed into its 
          respective positive and negative summatory function components: $G^{-1}(x) = G^{-1}_{+}(x) + G^{-1}_{-}(x)$ where 
          $G^{-1}_{+}(x) > 0$ and $G^{-1}_{-}(x) < 0$. 
          } 

\end{table}

\newpage
\begin{table}[h!]

\centering

\tiny
\begin{equation*}
\boxed{
\begin{array}{|cc|c|ccc|c|c|ccc|c|ccc}
 n & \mathbf{Primes} & & \mathbf{Sqfree} & \mathbf{PPower} & \bar{\mathbb{S}} & & g^{-1}(n) & 
 \lambda(n) \operatorname{sgn}(g^{-1}(n)) & \lambda(n) g^{-1}(n) - \widehat{f}_1(n) & 
 \frac{\sum\limits_{d|n} C_{\Omega(d)}(d)}{|g^{-1}(n)|} & & G^{-1}(n) & G^{-1}_{+}(n) & G^{-1}_{-}(n) \\ \hline 
 49 & 7^2 & \text{--} & \text{N} & \text{Y} & \text{N} & \text{--} & 2 & 1 & 0 & 1.5000000 & \text{--} & -13 & 108 & -121 \\
 50 & 2^1 5^2 & \text{--} & \text{N} & \text{N} & \text{Y} & \text{--} & -7 & 1 & 2 & 1.2857143 & \text{--} & -20 & 108 & -128 \\
 51 & 3^1 17^1 & \text{--} & \text{Y} & \text{N} & \text{N} & \text{--} & 5 & 1 & 0 & 1.0000000 & \text{--} & -15 & 113 & -128 \\
 52 & 2^2 13^1 & \text{--} & \text{N} & \text{N} & \text{Y} & \text{--} & -7 & 1 & 2 & 1.2857143 & \text{--} & -22 & 113 & -135 \\
 53 & 53^1 & \text{--} & \text{Y} & \text{Y} & \text{N} & \text{--} & -2 & 1 & 0 & 1.0000000 & \text{--} & -24 & 113 & -137 \\
 54 & 2^1 3^3 & \text{--} & \text{N} & \text{N} & \text{Y} & \text{--} & 9 & 1 & 4 & 1.5555556 & \text{--} & -15 & 122 & -137 \\
 55 & 5^1 11^1 & \text{--} & \text{Y} & \text{N} & \text{N} & \text{--} & 5 & 1 & 0 & 1.0000000 & \text{--} & -10 & 127 & -137 \\
 56 & 2^3 7^1 & \text{--} & \text{N} & \text{N} & \text{Y} & \text{--} & 9 & 1 & 4 & 1.5555556 & \text{--} & -1 & 136 & -137 \\
 57 & 3^1 19^1 & \text{--} & \text{Y} & \text{N} & \text{N} & \text{--} & 5 & 1 & 0 & 1.0000000 & \text{--} & 4 & 141 & -137 \\
 58 & 2^1 29^1 & \text{--} & \text{Y} & \text{N} & \text{N} & \text{--} & 5 & 1 & 0 & 1.0000000 & \text{--} & 9 & 146 & -137 \\
 59 & 59^1 & \text{--} & \text{Y} & \text{Y} & \text{N} & \text{--} & -2 & 1 & 0 & 1.0000000 & \text{--} & 7 & 146 & -139 \\
 60 & 2^2 3^1 5^1 & \text{--} & \text{N} & \text{N} & \text{Y} & \text{--} & 30 & 1 & 14 & 1.1666667 & \text{--} & 37 & 176 & -139 \\
 61 & 61^1 & \text{--} & \text{Y} & \text{Y} & \text{N} & \text{--} & -2 & 1 & 0 & 1.0000000 & \text{--} & 35 & 176 & -141 \\
 62 & 2^1 31^1 & \text{--} & \text{Y} & \text{N} & \text{N} & \text{--} & 5 & 1 & 0 & 1.0000000 & \text{--} & 40 & 181 & -141 \\
 63 & 3^2 7^1 & \text{--} & \text{N} & \text{N} & \text{Y} & \text{--} & -7 & 1 & 2 & 1.2857143 & \text{--} & 33 & 181 & -148 \\
 64 & 2^6 & \text{--} & \text{N} & \text{Y} & \text{N} & \text{--} & 2 & 1 & 0 & 3.5000000 & \text{--} & 35 & 183 & -148 \\
 65 & 5^1 13^1 & \text{--} & \text{Y} & \text{N} & \text{N} & \text{--} & 5 & 1 & 0 & 1.0000000 & \text{--} & 40 & 188 & -148 \\
 66 & 2^1 3^1 11^1 & \text{--} & \text{Y} & \text{N} & \text{N} & \text{--} & -16 & 1 & 0 & 1.0000000 & \text{--} & 24 & 188 & -164 \\
 67 & 67^1 & \text{--} & \text{Y} & \text{Y} & \text{N} & \text{--} & -2 & 1 & 0 & 1.0000000 & \text{--} & 22 & 188 & -166 \\
 68 & 2^2 17^1 & \text{--} & \text{N} & \text{N} & \text{Y} & \text{--} & -7 & 1 & 2 & 1.2857143 & \text{--} & 15 & 188 & -173 \\
 69 & 3^1 23^1 & \text{--} & \text{Y} & \text{N} & \text{N} & \text{--} & 5 & 1 & 0 & 1.0000000 & \text{--} & 20 & 193 & -173 \\
 70 & 2^1 5^1 7^1 & \text{--} & \text{Y} & \text{N} & \text{N} & \text{--} & -16 & 1 & 0 & 1.0000000 & \text{--} & 4 & 193 & -189 \\
 71 & 71^1 & \text{--} & \text{Y} & \text{Y} & \text{N} & \text{--} & -2 & 1 & 0 & 1.0000000 & \text{--} & 2 & 193 & -191 \\
 72 & 2^3 3^2 & \text{--} & \text{N} & \text{N} & \text{Y} & \text{--} & -23 & 1 & 18 & 1.4782609 & \text{--} & -21 & 193 & -214 \\
 73 & 73^1 & \text{--} & \text{Y} & \text{Y} & \text{N} & \text{--} & -2 & 1 & 0 & 1.0000000 & \text{--} & -23 & 193 & -216 \\
 74 & 2^1 37^1 & \text{--} & \text{Y} & \text{N} & \text{N} & \text{--} & 5 & 1 & 0 & 1.0000000 & \text{--} & -18 & 198 & -216 \\
 75 & 3^1 5^2 & \text{--} & \text{N} & \text{N} & \text{Y} & \text{--} & -7 & 1 & 2 & 1.2857143 & \text{--} & -25 & 198 & -223 \\
 76 & 2^2 19^1 & \text{--} & \text{N} & \text{N} & \text{Y} & \text{--} & -7 & 1 & 2 & 1.2857143 & \text{--} & -32 & 198 & -230 \\
 77 & 7^1 11^1 & \text{--} & \text{Y} & \text{N} & \text{N} & \text{--} & 5 & 1 & 0 & 1.0000000 & \text{--} & -27 & 203 & -230 \\
 78 & 2^1 3^1 13^1 & \text{--} & \text{Y} & \text{N} & \text{N} & \text{--} & -16 & 1 & 0 & 1.0000000 & \text{--} & -43 & 203 & -246 \\
 79 & 79^1 & \text{--} & \text{Y} & \text{Y} & \text{N} & \text{--} & -2 & 1 & 0 & 1.0000000 & \text{--} & -45 & 203 & -248 \\
 80 & 2^4 5^1 & \text{--} & \text{N} & \text{N} & \text{Y} & \text{--} & -11 & 1 & 6 & 1.8181818 & \text{--} & -56 & 203 & -259 \\
 81 & 3^4 & \text{--} & \text{N} & \text{Y} & \text{N} & \text{--} & 2 & 1 & 0 & 2.5000000 & \text{--} & -54 & 205 & -259 \\
 82 & 2^1 41^1 & \text{--} & \text{Y} & \text{N} & \text{N} & \text{--} & 5 & 1 & 0 & 1.0000000 & \text{--} & -49 & 210 & -259 \\
 83 & 83^1 & \text{--} & \text{Y} & \text{Y} & \text{N} & \text{--} & -2 & 1 & 0 & 1.0000000 & \text{--} & -51 & 210 & -261 \\
 84 & 2^2 3^1 7^1 & \text{--} & \text{N} & \text{N} & \text{Y} & \text{--} & 30 & 1 & 14 & 1.1666667 & \text{--} & -21 & 240 & -261 \\
 85 & 5^1 17^1 & \text{--} & \text{Y} & \text{N} & \text{N} & \text{--} & 5 & 1 & 0 & 1.0000000 & \text{--} & -16 & 245 & -261 \\
 86 & 2^1 43^1 & \text{--} & \text{Y} & \text{N} & \text{N} & \text{--} & 5 & 1 & 0 & 1.0000000 & \text{--} & -11 & 250 & -261 \\
 87 & 3^1 29^1 & \text{--} & \text{Y} & \text{N} & \text{N} & \text{--} & 5 & 1 & 0 & 1.0000000 & \text{--} & -6 & 255 & -261 \\
 88 & 2^3 11^1 & \text{--} & \text{N} & \text{N} & \text{Y} & \text{--} & 9 & 1 & 4 & 1.5555556 & \text{--} & 3 & 264 & -261 \\
 89 & 89^1 & \text{--} & \text{Y} & \text{Y} & \text{N} & \text{--} & -2 & 1 & 0 & 1.0000000 & \text{--} & 1 & 264 & -263 \\
 90 & 2^1 3^2 5^1 & \text{--} & \text{N} & \text{N} & \text{Y} & \text{--} & 30 & 1 & 14 & 1.1666667 & \text{--} & 31 & 294 & -263 \\
 91 & 7^1 13^1 & \text{--} & \text{Y} & \text{N} & \text{N} & \text{--} & 5 & 1 & 0 & 1.0000000 & \text{--} & 36 & 299 & -263 \\
 92 & 2^2 23^1 & \text{--} & \text{N} & \text{N} & \text{Y} & \text{--} & -7 & 1 & 2 & 1.2857143 & \text{--} & 29 & 299 & -270 \\
 93 & 3^1 31^1 & \text{--} & \text{Y} & \text{N} & \text{N} & \text{--} & 5 & 1 & 0 & 1.0000000 & \text{--} & 34 & 304 & -270 \\
 94 & 2^1 47^1 & \text{--} & \text{Y} & \text{N} & \text{N} & \text{--} & 5 & 1 & 0 & 1.0000000 & \text{--} & 39 & 309 & -270 \\
 95 & 5^1 19^1 & \text{--} & \text{Y} & \text{N} & \text{N} & \text{--} & 5 & 1 & 0 & 1.0000000 & \text{--} & 44 & 314 & -270 \\
 96 & 2^5 3^1 & \text{--} & \text{N} & \text{N} & \text{Y} & \text{--} & 13 & 1 & 8 & 2.0769231 & \text{--} & 57 & 327 & -270 \\
 97 & 97^1 & \text{--} & \text{Y} & \text{Y} & \text{N} & \text{--} & -2 & 1 & 0 & 1.0000000 & \text{--} & 55 & 327 & -272 \\
 98 & 2^1 7^2 & \text{--} & \text{N} & \text{N} & \text{Y} & \text{--} & -7 & 1 & 2 & 1.2857143 & \text{--} & 48 & 327 & -279 \\
 99 & 3^2 11^1 & \text{--} & \text{N} & \text{N} & \text{Y} & \text{--} & -7 & 1 & 2 & 1.2857143 & \text{--} & 41 & 327 & -286 \\
 100 & 2^2 5^2 & \text{--} & \text{N} & \text{N} & \text{Y} & \text{--} & 14 & 1 & 9 & 1.3571429 & \text{--} & 55 & 341 & -286 \\
 101 & 101^1 & \text{--} & \text{Y} & \text{Y} & \text{N} & \text{--} & -2 & 1 & 0 & 1.0000000 & \text{--} & 53 & 341 & -288 \\
 102 & 2^1 3^1 17^1 & \text{--} & \text{Y} & \text{N} & \text{N} & \text{--} & -16 & 1 & 0 & 1.0000000 & \text{--} & 37 & 341 & -304 \\
 103 & 103^1 & \text{--} & \text{Y} & \text{Y} & \text{N} & \text{--} & -2 & 1 & 0 & 1.0000000 & \text{--} & 35 & 341 & -306 \\
 104 & 2^3 13^1 & \text{--} & \text{N} & \text{N} & \text{Y} & \text{--} & 9 & 1 & 4 & 1.5555556 & \text{--} & 44 & 350 & -306 \\
 105 & 3^1 5^1 7^1 & \text{--} & \text{Y} & \text{N} & \text{N} & \text{--} & -16 & 1 & 0 & 1.0000000 & \text{--} & 28 & 350 & -322 \\
 106 & 2^1 53^1 & \text{--} & \text{Y} & \text{N} & \text{N} & \text{--} & 5 & 1 & 0 & 1.0000000 & \text{--} & 33 & 355 & -322 \\
 107 & 107^1 & \text{--} & \text{Y} & \text{Y} & \text{N} & \text{--} & -2 & 1 & 0 & 1.0000000 & \text{--} & 31 & 355 & -324 \\
 108 & 2^2 3^3 & \text{--} & \text{N} & \text{N} & \text{Y} & \text{--} & -23 & 1 & 18 & 1.4782609 & \text{--} & 8 & 355 & -347 \\
 109 & 109^1 & \text{--} & \text{Y} & \text{Y} & \text{N} & \text{--} & -2 & 1 & 0 & 1.0000000 & \text{--} & 6 & 355 & -349 \\
 110 & 2^1 5^1 11^1 & \text{--} & \text{Y} & \text{N} & \text{N} & \text{--} & -16 & 1 & 0 & 1.0000000 & \text{--} & -10 & 355 & -365 \\
 111 & 3^1 37^1 & \text{--} & \text{Y} & \text{N} & \text{N} & \text{--} & 5 & 1 & 0 & 1.0000000 & \text{--} & -5 & 360 & -365 \\
 112 & 2^4 7^1 & \text{--} & \text{N} & \text{N} & \text{Y} & \text{--} & -11 & 1 & 6 & 1.8181818 & \text{--} & -16 & 360 & -376 \\
 113 & 113^1 & \text{--} & \text{Y} & \text{Y} & \text{N} & \text{--} & -2 & 1 & 0 & 1.0000000 & \text{--} & -18 & 360 & -378 \\
 114 & 2^1 3^1 19^1 & \text{--} & \text{Y} & \text{N} & \text{N} & \text{--} & -16 & 1 & 0 & 1.0000000 & \text{--} & -34 & 360 & -394 \\
 115 & 5^1 23^1 & \text{--} & \text{Y} & \text{N} & \text{N} & \text{--} & 5 & 1 & 0 & 1.0000000 & \text{--} & -29 & 365 & -394 \\
 116 & 2^2 29^1 & \text{--} & \text{N} & \text{N} & \text{Y} & \text{--} & -7 & 1 & 2 & 1.2857143 & \text{--} & -36 & 365 & -401 \\
 117 & 3^2 13^1 & \text{--} & \text{N} & \text{N} & \text{Y} & \text{--} & -7 & 1 & 2 & 1.2857143 & \text{--} & -43 & 365 & -408 \\
 118 & 2^1 59^1 & \text{--} & \text{Y} & \text{N} & \text{N} & \text{--} & 5 & 1 & 0 & 1.0000000 & \text{--} & -38 & 370 & -408 \\
 119 & 7^1 17^1 & \text{--} & \text{Y} & \text{N} & \text{N} & \text{--} & 5 & 1 & 0 & 1.0000000 & \text{--} & -33 & 375 & -408 \\
 120 & 2^3 3^1 5^1 & \text{--} & \text{N} & \text{N} & \text{Y} & \text{--} & -48 & 1 & 32 & 1.3333333 & \text{--} & -81 & 375 & -456 \\
 121 & 11^2 & \text{--} & \text{N} & \text{Y} & \text{N} & \text{--} & 2 & 1 & 0 & 1.5000000 & \text{--} & -79 & 377 & -456 \\
 122 & 2^1 61^1 & \text{--} & \text{Y} & \text{N} & \text{N} & \text{--} & 5 & 1 & 0 & 1.0000000 & \text{--} & -74 & 382 & -456 \\
 123 & 3^1 41^1 & \text{--} & \text{Y} & \text{N} & \text{N} & \text{--} & 5 & 1 & 0 & 1.0000000 & \text{--} & -69 & 387 & -456 \\
 124 & 2^2 31^1 & \text{--} & \text{N} & \text{N} & \text{Y} & \text{--} & -7 & 1 & 2 & 1.2857143 & \text{--} & -76 & 387 & -463 \\
\end{array}
}
\end{equation*}

\end{table} 


\newpage
\begin{table}[h!]

\centering

\tiny
\begin{equation*}
\boxed{
\begin{array}{|cc|c|ccc|c|c|ccc|c|ccc}
 n & \mathbf{Primes} & & \mathbf{Sqfree} & \mathbf{PPower} & \bar{\mathbb{S}} & & g^{-1}(n) & 
 \lambda(n) \operatorname{sgn}(g^{-1}(n)) & \lambda(n) g^{-1}(n) - \widehat{f}_1(n) & 
 \frac{\sum\limits_{d|n} C_{\Omega(d)}(d)}{|g^{-1}(n)|} & & G^{-1}(n) & G^{-1}_{+}(n) & G^{-1}_{-}(n) \\ \hline 
 125 & 5^3 & \text{--} & \text{N} & \text{Y} & \text{N} & \text{--} & -2 & 1 & 0 & 2.0000000 & \text{--} & -78 & 387 & -465 \\
 126 & 2^1 3^2 7^1 & \text{--} & \text{N} & \text{N} & \text{Y} & \text{--} & 30 & 1 & 14 & 1.1666667 & \text{--} & -48 & 417 & -465 \\
 127 & 127^1 & \text{--} & \text{Y} & \text{Y} & \text{N} & \text{--} & -2 & 1 & 0 & 1.0000000 & \text{--} & -50 & 417 & -467 \\
 128 & 2^7 & \text{--} & \text{N} & \text{Y} & \text{N} & \text{--} & -2 & 1 & 0 & 4.0000000 & \text{--} & -52 & 417 & -469 \\
 129 & 3^1 43^1 & \text{--} & \text{Y} & \text{N} & \text{N} & \text{--} & 5 & 1 & 0 & 1.0000000 & \text{--} & -47 & 422 & -469 \\
 130 & 2^1 5^1 13^1 & \text{--} & \text{Y} & \text{N} & \text{N} & \text{--} & -16 & 1 & 0 & 1.0000000 & \text{--} & -63 & 422 & -485 \\
 131 & 131^1 & \text{--} & \text{Y} & \text{Y} & \text{N} & \text{--} & -2 & 1 & 0 & 1.0000000 & \text{--} & -65 & 422 & -487 \\
 132 & 2^2 3^1 11^1 & \text{--} & \text{N} & \text{N} & \text{Y} & \text{--} & 30 & 1 & 14 & 1.1666667 & \text{--} & -35 & 452 & -487 \\
 133 & 7^1 19^1 & \text{--} & \text{Y} & \text{N} & \text{N} & \text{--} & 5 & 1 & 0 & 1.0000000 & \text{--} & -30 & 457 & -487 \\
 134 & 2^1 67^1 & \text{--} & \text{Y} & \text{N} & \text{N} & \text{--} & 5 & 1 & 0 & 1.0000000 & \text{--} & -25 & 462 & -487 \\
 135 & 3^3 5^1 & \text{--} & \text{N} & \text{N} & \text{Y} & \text{--} & 9 & 1 & 4 & 1.5555556 & \text{--} & -16 & 471 & -487 \\
 136 & 2^3 17^1 & \text{--} & \text{N} & \text{N} & \text{Y} & \text{--} & 9 & 1 & 4 & 1.5555556 & \text{--} & -7 & 480 & -487 \\
 137 & 137^1 & \text{--} & \text{Y} & \text{Y} & \text{N} & \text{--} & -2 & 1 & 0 & 1.0000000 & \text{--} & -9 & 480 & -489 \\
 138 & 2^1 3^1 23^1 & \text{--} & \text{Y} & \text{N} & \text{N} & \text{--} & -16 & 1 & 0 & 1.0000000 & \text{--} & -25 & 480 & -505 \\
 139 & 139^1 & \text{--} & \text{Y} & \text{Y} & \text{N} & \text{--} & -2 & 1 & 0 & 1.0000000 & \text{--} & -27 & 480 & -507 \\
 140 & 2^2 5^1 7^1 & \text{--} & \text{N} & \text{N} & \text{Y} & \text{--} & 30 & 1 & 14 & 1.1666667 & \text{--} & 3 & 510 & -507 \\
 141 & 3^1 47^1 & \text{--} & \text{Y} & \text{N} & \text{N} & \text{--} & 5 & 1 & 0 & 1.0000000 & \text{--} & 8 & 515 & -507 \\
 142 & 2^1 71^1 & \text{--} & \text{Y} & \text{N} & \text{N} & \text{--} & 5 & 1 & 0 & 1.0000000 & \text{--} & 13 & 520 & -507 \\
 143 & 11^1 13^1 & \text{--} & \text{Y} & \text{N} & \text{N} & \text{--} & 5 & 1 & 0 & 1.0000000 & \text{--} & 18 & 525 & -507 \\
 144 & 2^4 3^2 & \text{--} & \text{N} & \text{N} & \text{Y} & \text{--} & 34 & 1 & 29 & 1.6176471 & \text{--} & 52 & 559 & -507 \\
 145 & 5^1 29^1 & \text{--} & \text{Y} & \text{N} & \text{N} & \text{--} & 5 & 1 & 0 & 1.0000000 & \text{--} & 57 & 564 & -507 \\
 146 & 2^1 73^1 & \text{--} & \text{Y} & \text{N} & \text{N} & \text{--} & 5 & 1 & 0 & 1.0000000 & \text{--} & 62 & 569 & -507 \\
 147 & 3^1 7^2 & \text{--} & \text{N} & \text{N} & \text{Y} & \text{--} & -7 & 1 & 2 & 1.2857143 & \text{--} & 55 & 569 & -514 \\
 148 & 2^2 37^1 & \text{--} & \text{N} & \text{N} & \text{Y} & \text{--} & -7 & 1 & 2 & 1.2857143 & \text{--} & 48 & 569 & -521 \\
 149 & 149^1 & \text{--} & \text{Y} & \text{Y} & \text{N} & \text{--} & -2 & 1 & 0 & 1.0000000 & \text{--} & 46 & 569 & -523 \\
 150 & 2^1 3^1 5^2 & \text{--} & \text{N} & \text{N} & \text{Y} & \text{--} & 30 & 1 & 14 & 1.1666667 & \text{--} & 76 & 599 & -523 \\
 151 & 151^1 & \text{--} & \text{Y} & \text{Y} & \text{N} & \text{--} & -2 & 1 & 0 & 1.0000000 & \text{--} & 74 & 599 & -525 \\
 152 & 2^3 19^1 & \text{--} & \text{N} & \text{N} & \text{Y} & \text{--} & 9 & 1 & 4 & 1.5555556 & \text{--} & 83 & 608 & -525 \\
 153 & 3^2 17^1 & \text{--} & \text{N} & \text{N} & \text{Y} & \text{--} & -7 & 1 & 2 & 1.2857143 & \text{--} & 76 & 608 & -532 \\
 154 & 2^1 7^1 11^1 & \text{--} & \text{Y} & \text{N} & \text{N} & \text{--} & -16 & 1 & 0 & 1.0000000 & \text{--} & 60 & 608 & -548 \\
 155 & 5^1 31^1 & \text{--} & \text{Y} & \text{N} & \text{N} & \text{--} & 5 & 1 & 0 & 1.0000000 & \text{--} & 65 & 613 & -548 \\
 156 & 2^2 3^1 13^1 & \text{--} & \text{N} & \text{N} & \text{Y} & \text{--} & 30 & 1 & 14 & 1.1666667 & \text{--} & 95 & 643 & -548 \\
 157 & 157^1 & \text{--} & \text{Y} & \text{Y} & \text{N} & \text{--} & -2 & 1 & 0 & 1.0000000 & \text{--} & 93 & 643 & -550 \\
 158 & 2^1 79^1 & \text{--} & \text{Y} & \text{N} & \text{N} & \text{--} & 5 & 1 & 0 & 1.0000000 & \text{--} & 98 & 648 & -550 \\
 159 & 3^1 53^1 & \text{--} & \text{Y} & \text{N} & \text{N} & \text{--} & 5 & 1 & 0 & 1.0000000 & \text{--} & 103 & 653 & -550 \\
 160 & 2^5 5^1 & \text{--} & \text{N} & \text{N} & \text{Y} & \text{--} & 13 & 1 & 8 & 2.0769231 & \text{--} & 116 & 666 & -550 \\
 161 & 7^1 23^1 & \text{--} & \text{Y} & \text{N} & \text{N} & \text{--} & 5 & 1 & 0 & 1.0000000 & \text{--} & 121 & 671 & -550 \\
 162 & 2^1 3^4 & \text{--} & \text{N} & \text{N} & \text{Y} & \text{--} & -11 & 1 & 6 & 1.8181818 & \text{--} & 110 & 671 & -561 \\
 163 & 163^1 & \text{--} & \text{Y} & \text{Y} & \text{N} & \text{--} & -2 & 1 & 0 & 1.0000000 & \text{--} & 108 & 671 & -563 \\
 164 & 2^2 41^1 & \text{--} & \text{N} & \text{N} & \text{Y} & \text{--} & -7 & 1 & 2 & 1.2857143 & \text{--} & 101 & 671 & -570 \\
 165 & 3^1 5^1 11^1 & \text{--} & \text{Y} & \text{N} & \text{N} & \text{--} & -16 & 1 & 0 & 1.0000000 & \text{--} & 85 & 671 & -586 \\
 166 & 2^1 83^1 & \text{--} & \text{Y} & \text{N} & \text{N} & \text{--} & 5 & 1 & 0 & 1.0000000 & \text{--} & 90 & 676 & -586 \\
 167 & 167^1 & \text{--} & \text{Y} & \text{Y} & \text{N} & \text{--} & -2 & 1 & 0 & 1.0000000 & \text{--} & 88 & 676 & -588 \\
 168 & 2^3 3^1 7^1 & \text{--} & \text{N} & \text{N} & \text{Y} & \text{--} & -48 & 1 & 32 & 1.3333333 & \text{--} & 40 & 676 & -636 \\
 169 & 13^2 & \text{--} & \text{N} & \text{Y} & \text{N} & \text{--} & 2 & 1 & 0 & 1.5000000 & \text{--} & 42 & 678 & -636 \\
 170 & 2^1 5^1 17^1 & \text{--} & \text{Y} & \text{N} & \text{N} & \text{--} & -16 & 1 & 0 & 1.0000000 & \text{--} & 26 & 678 & -652 \\
 171 & 3^2 19^1 & \text{--} & \text{N} & \text{N} & \text{Y} & \text{--} & -7 & 1 & 2 & 1.2857143 & \text{--} & 19 & 678 & -659 \\
 172 & 2^2 43^1 & \text{--} & \text{N} & \text{N} & \text{Y} & \text{--} & -7 & 1 & 2 & 1.2857143 & \text{--} & 12 & 678 & -666 \\
 173 & 173^1 & \text{--} & \text{Y} & \text{Y} & \text{N} & \text{--} & -2 & 1 & 0 & 1.0000000 & \text{--} & 10 & 678 & -668 \\
 174 & 2^1 3^1 29^1 & \text{--} & \text{Y} & \text{N} & \text{N} & \text{--} & -16 & 1 & 0 & 1.0000000 & \text{--} & -6 & 678 & -684 \\
 175 & 5^2 7^1 & \text{--} & \text{N} & \text{N} & \text{Y} & \text{--} & -7 & 1 & 2 & 1.2857143 & \text{--} & -13 & 678 & -691 \\
 176 & 2^4 11^1 & \text{--} & \text{N} & \text{N} & \text{Y} & \text{--} & -11 & 1 & 6 & 1.8181818 & \text{--} & -24 & 678 & -702 \\
 177 & 3^1 59^1 & \text{--} & \text{Y} & \text{N} & \text{N} & \text{--} & 5 & 1 & 0 & 1.0000000 & \text{--} & -19 & 683 & -702 \\
 178 & 2^1 89^1 & \text{--} & \text{Y} & \text{N} & \text{N} & \text{--} & 5 & 1 & 0 & 1.0000000 & \text{--} & -14 & 688 & -702 \\
 179 & 179^1 & \text{--} & \text{Y} & \text{Y} & \text{N} & \text{--} & -2 & 1 & 0 & 1.0000000 & \text{--} & -16 & 688 & -704 \\
 180 & 2^2 3^2 5^1 & \text{--} & \text{N} & \text{N} & \text{Y} & \text{--} & -74 & 1 & 58 & 1.2162162 & \text{--} & -90 & 688 & -778 \\
 181 & 181^1 & \text{--} & \text{Y} & \text{Y} & \text{N} & \text{--} & -2 & 1 & 0 & 1.0000000 & \text{--} & -92 & 688 & -780 \\
 182 & 2^1 7^1 13^1 & \text{--} & \text{Y} & \text{N} & \text{N} & \text{--} & -16 & 1 & 0 & 1.0000000 & \text{--} & -108 & 688 & -796 \\
 183 & 3^1 61^1 & \text{--} & \text{Y} & \text{N} & \text{N} & \text{--} & 5 & 1 & 0 & 1.0000000 & \text{--} & -103 & 693 & -796 \\
 184 & 2^3 23^1 & \text{--} & \text{N} & \text{N} & \text{Y} & \text{--} & 9 & 1 & 4 & 1.5555556 & \text{--} & -94 & 702 & -796 \\
 185 & 5^1 37^1 & \text{--} & \text{Y} & \text{N} & \text{N} & \text{--} & 5 & 1 & 0 & 1.0000000 & \text{--} & -89 & 707 & -796 \\
 186 & 2^1 3^1 31^1 & \text{--} & \text{Y} & \text{N} & \text{N} & \text{--} & -16 & 1 & 0 & 1.0000000 & \text{--} & -105 & 707 & -812 \\
 187 & 11^1 17^1 & \text{--} & \text{Y} & \text{N} & \text{N} & \text{--} & 5 & 1 & 0 & 1.0000000 & \text{--} & -100 & 712 & -812 \\
 188 & 2^2 47^1 & \text{--} & \text{N} & \text{N} & \text{Y} & \text{--} & -7 & 1 & 2 & 1.2857143 & \text{--} & -107 & 712 & -819 \\
 189 & 3^3 7^1 & \text{--} & \text{N} & \text{N} & \text{Y} & \text{--} & 9 & 1 & 4 & 1.5555556 & \text{--} & -98 & 721 & -819 \\
 190 & 2^1 5^1 19^1 & \text{--} & \text{Y} & \text{N} & \text{N} & \text{--} & -16 & 1 & 0 & 1.0000000 & \text{--} & -114 & 721 & -835 \\
 191 & 191^1 & \text{--} & \text{Y} & \text{Y} & \text{N} & \text{--} & -2 & 1 & 0 & 1.0000000 & \text{--} & -116 & 721 & -837 \\
 192 & 2^6 3^1 & \text{--} & \text{N} & \text{N} & \text{Y} & \text{--} & -15 & 1 & 10 & 2.3333333 & \text{--} & -131 & 721 & -852 \\
 193 & 193^1 & \text{--} & \text{Y} & \text{Y} & \text{N} & \text{--} & -2 & 1 & 0 & 1.0000000 & \text{--} & -133 & 721 & -854 \\
 194 & 2^1 97^1 & \text{--} & \text{Y} & \text{N} & \text{N} & \text{--} & 5 & 1 & 0 & 1.0000000 & \text{--} & -128 & 726 & -854 \\
 195 & 3^1 5^1 13^1 & \text{--} & \text{Y} & \text{N} & \text{N} & \text{--} & -16 & 1 & 0 & 1.0000000 & \text{--} & -144 & 726 & -870 \\
 196 & 2^2 7^2 & \text{--} & \text{N} & \text{N} & \text{Y} & \text{--} & 14 & 1 & 9 & 1.3571429 & \text{--} & -130 & 740 & -870 \\
 197 & 197^1 & \text{--} & \text{Y} & \text{Y} & \text{N} & \text{--} & -2 & 1 & 0 & 1.0000000 & \text{--} & -132 & 740 & -872 \\
 198 & 2^1 3^2 11^1 & \text{--} & \text{N} & \text{N} & \text{Y} & \text{--} & 30 & 1 & 14 & 1.1666667 & \text{--} & -102 & 770 & -872 \\
 199 & 199^1 & \text{--} & \text{Y} & \text{Y} & \text{N} & \text{--} & -2 & 1 & 0 & 1.0000000 & \text{--} & -104 & 770 & -874 \\
 200 & 2^3 5^2 & \text{--} & \text{N} & \text{N} & \text{Y} & \text{--} & -23 & 1 & 18 & 1.4782609 & \text{--} & -127 & 770 & -897 \\
\end{array}
}
\end{equation*}

\end{table} 


\newpage
\begin{table}[h!]

\centering

\tiny
\begin{equation*}
\boxed{
\begin{array}{|cc|c|ccc|c|c|ccc|c|ccc}
 n & \mathbf{Primes} & & \mathbf{Sqfree} & \mathbf{PPower} & \bar{\mathbb{S}} & & g^{-1}(n) & 
 \lambda(n) \operatorname{sgn}(g^{-1}(n)) & \lambda(n) g^{-1}(n) - \widehat{f}_1(n) & 
 \frac{\sum\limits_{d|n} C_{\Omega(d)}(d)}{|g^{-1}(n)|} & & G^{-1}(n) & G^{-1}_{+}(n) & G^{-1}_{-}(n) \\ \hline 
 201 & 3^1 67^1 & \text{--} & \text{Y} & \text{N} & \text{N} & \text{--} & 5 & 1 & 0 & 1.0000000 & \text{--} & -122 & 775 & -897 \\
 202 & 2^1 101^1 & \text{--} & \text{Y} & \text{N} & \text{N} & \text{--} & 5 & 1 & 0 & 1.0000000 & \text{--} & -117 & 780 & -897 \\
 203 & 7^1 29^1 & \text{--} & \text{Y} & \text{N} & \text{N} & \text{--} & 5 & 1 & 0 & 1.0000000 & \text{--} & -112 & 785 & -897 \\
 204 & 2^2 3^1 17^1 & \text{--} & \text{N} & \text{N} & \text{Y} & \text{--} & 30 & 1 & 14 & 1.1666667 & \text{--} & -82 & 815 & -897 \\
 205 & 5^1 41^1 & \text{--} & \text{Y} & \text{N} & \text{N} & \text{--} & 5 & 1 & 0 & 1.0000000 & \text{--} & -77 & 820 & -897 \\
 206 & 2^1 103^1 & \text{--} & \text{Y} & \text{N} & \text{N} & \text{--} & 5 & 1 & 0 & 1.0000000 & \text{--} & -72 & 825 & -897 \\
 207 & 3^2 23^1 & \text{--} & \text{N} & \text{N} & \text{Y} & \text{--} & -7 & 1 & 2 & 1.2857143 & \text{--} & -79 & 825 & -904 \\
 208 & 2^4 13^1 & \text{--} & \text{N} & \text{N} & \text{Y} & \text{--} & -11 & 1 & 6 & 1.8181818 & \text{--} & -90 & 825 & -915 \\
 209 & 11^1 19^1 & \text{--} & \text{Y} & \text{N} & \text{N} & \text{--} & 5 & 1 & 0 & 1.0000000 & \text{--} & -85 & 830 & -915 \\
 210 & 2^1 3^1 5^1 7^1 & \text{--} & \text{Y} & \text{N} & \text{N} & \text{--} & 65 & 1 & 0 & 1.0000000 & \text{--} & -20 & 895 & -915 \\
 211 & 211^1 & \text{--} & \text{Y} & \text{Y} & \text{N} & \text{--} & -2 & 1 & 0 & 1.0000000 & \text{--} & -22 & 895 & -917 \\
 212 & 2^2 53^1 & \text{--} & \text{N} & \text{N} & \text{Y} & \text{--} & -7 & 1 & 2 & 1.2857143 & \text{--} & -29 & 895 & -924 \\
 213 & 3^1 71^1 & \text{--} & \text{Y} & \text{N} & \text{N} & \text{--} & 5 & 1 & 0 & 1.0000000 & \text{--} & -24 & 900 & -924 \\
 214 & 2^1 107^1 & \text{--} & \text{Y} & \text{N} & \text{N} & \text{--} & 5 & 1 & 0 & 1.0000000 & \text{--} & -19 & 905 & -924 \\
 215 & 5^1 43^1 & \text{--} & \text{Y} & \text{N} & \text{N} & \text{--} & 5 & 1 & 0 & 1.0000000 & \text{--} & -14 & 910 & -924 \\
 216 & 2^3 3^3 & \text{--} & \text{N} & \text{N} & \text{Y} & \text{--} & 46 & 1 & 41 & 1.5000000 & \text{--} & 32 & 956 & -924 \\
 217 & 7^1 31^1 & \text{--} & \text{Y} & \text{N} & \text{N} & \text{--} & 5 & 1 & 0 & 1.0000000 & \text{--} & 37 & 961 & -924 \\
 218 & 2^1 109^1 & \text{--} & \text{Y} & \text{N} & \text{N} & \text{--} & 5 & 1 & 0 & 1.0000000 & \text{--} & 42 & 966 & -924 \\
 219 & 3^1 73^1 & \text{--} & \text{Y} & \text{N} & \text{N} & \text{--} & 5 & 1 & 0 & 1.0000000 & \text{--} & 47 & 971 & -924 \\
 220 & 2^2 5^1 11^1 & \text{--} & \text{N} & \text{N} & \text{Y} & \text{--} & 30 & 1 & 14 & 1.1666667 & \text{--} & 77 & 1001 & -924 \\
 221 & 13^1 17^1 & \text{--} & \text{Y} & \text{N} & \text{N} & \text{--} & 5 & 1 & 0 & 1.0000000 & \text{--} & 82 & 1006 & -924 \\
 222 & 2^1 3^1 37^1 & \text{--} & \text{Y} & \text{N} & \text{N} & \text{--} & -16 & 1 & 0 & 1.0000000 & \text{--} & 66 & 1006 & -940 \\
 223 & 223^1 & \text{--} & \text{Y} & \text{Y} & \text{N} & \text{--} & -2 & 1 & 0 & 1.0000000 & \text{--} & 64 & 1006 & -942 \\
 224 & 2^5 7^1 & \text{--} & \text{N} & \text{N} & \text{Y} & \text{--} & 13 & 1 & 8 & 2.0769231 & \text{--} & 77 & 1019 & -942 \\
 225 & 3^2 5^2 & \text{--} & \text{N} & \text{N} & \text{Y} & \text{--} & 14 & 1 & 9 & 1.3571429 & \text{--} & 91 & 1033 & -942 \\
 226 & 2^1 113^1 & \text{--} & \text{Y} & \text{N} & \text{N} & \text{--} & 5 & 1 & 0 & 1.0000000 & \text{--} & 96 & 1038 & -942 \\
 227 & 227^1 & \text{--} & \text{Y} & \text{Y} & \text{N} & \text{--} & -2 & 1 & 0 & 1.0000000 & \text{--} & 94 & 1038 & -944 \\
 228 & 2^2 3^1 19^1 & \text{--} & \text{N} & \text{N} & \text{Y} & \text{--} & 30 & 1 & 14 & 1.1666667 & \text{--} & 124 & 1068 & -944 \\
 229 & 229^1 & \text{--} & \text{Y} & \text{Y} & \text{N} & \text{--} & -2 & 1 & 0 & 1.0000000 & \text{--} & 122 & 1068 & -946 \\
 230 & 2^1 5^1 23^1 & \text{--} & \text{Y} & \text{N} & \text{N} & \text{--} & -16 & 1 & 0 & 1.0000000 & \text{--} & 106 & 1068 & -962 \\
 231 & 3^1 7^1 11^1 & \text{--} & \text{Y} & \text{N} & \text{N} & \text{--} & -16 & 1 & 0 & 1.0000000 & \text{--} & 90 & 1068 & -978 \\
 232 & 2^3 29^1 & \text{--} & \text{N} & \text{N} & \text{Y} & \text{--} & 9 & 1 & 4 & 1.5555556 & \text{--} & 99 & 1077 & -978 \\
 233 & 233^1 & \text{--} & \text{Y} & \text{Y} & \text{N} & \text{--} & -2 & 1 & 0 & 1.0000000 & \text{--} & 97 & 1077 & -980 \\
 234 & 2^1 3^2 13^1 & \text{--} & \text{N} & \text{N} & \text{Y} & \text{--} & 30 & 1 & 14 & 1.1666667 & \text{--} & 127 & 1107 & -980 \\
 235 & 5^1 47^1 & \text{--} & \text{Y} & \text{N} & \text{N} & \text{--} & 5 & 1 & 0 & 1.0000000 & \text{--} & 132 & 1112 & -980 \\
 236 & 2^2 59^1 & \text{--} & \text{N} & \text{N} & \text{Y} & \text{--} & -7 & 1 & 2 & 1.2857143 & \text{--} & 125 & 1112 & -987 \\
 237 & 3^1 79^1 & \text{--} & \text{Y} & \text{N} & \text{N} & \text{--} & 5 & 1 & 0 & 1.0000000 & \text{--} & 130 & 1117 & -987 \\
 238 & 2^1 7^1 17^1 & \text{--} & \text{Y} & \text{N} & \text{N} & \text{--} & -16 & 1 & 0 & 1.0000000 & \text{--} & 114 & 1117 & -1003 \\
 239 & 239^1 & \text{--} & \text{Y} & \text{Y} & \text{N} & \text{--} & -2 & 1 & 0 & 1.0000000 & \text{--} & 112 & 1117 & -1005 \\
 240 & 2^4 3^1 5^1 & \text{--} & \text{N} & \text{N} & \text{Y} & \text{--} & 70 & 1 & 54 & 1.5000000 & \text{--} & 182 & 1187 & -1005 \\
 241 & 241^1 & \text{--} & \text{Y} & \text{Y} & \text{N} & \text{--} & -2 & 1 & 0 & 1.0000000 & \text{--} & 180 & 1187 & -1007 \\
 242 & 2^1 11^2 & \text{--} & \text{N} & \text{N} & \text{Y} & \text{--} & -7 & 1 & 2 & 1.2857143 & \text{--} & 173 & 1187 & -1014 \\
 243 & 3^5 & \text{--} & \text{N} & \text{Y} & \text{N} & \text{--} & -2 & 1 & 0 & 3.0000000 & \text{--} & 171 & 1187 & -1016 \\
 244 & 2^2 61^1 & \text{--} & \text{N} & \text{N} & \text{Y} & \text{--} & -7 & 1 & 2 & 1.2857143 & \text{--} & 164 & 1187 & -1023 \\
 245 & 5^1 7^2 & \text{--} & \text{N} & \text{N} & \text{Y} & \text{--} & -7 & 1 & 2 & 1.2857143 & \text{--} & 157 & 1187 & -1030 \\
 246 & 2^1 3^1 41^1 & \text{--} & \text{Y} & \text{N} & \text{N} & \text{--} & -16 & 1 & 0 & 1.0000000 & \text{--} & 141 & 1187 & -1046 \\
 247 & 13^1 19^1 & \text{--} & \text{Y} & \text{N} & \text{N} & \text{--} & 5 & 1 & 0 & 1.0000000 & \text{--} & 146 & 1192 & -1046 \\
 248 & 2^3 31^1 & \text{--} & \text{N} & \text{N} & \text{Y} & \text{--} & 9 & 1 & 4 & 1.5555556 & \text{--} & 155 & 1201 & -1046 \\
 249 & 3^1 83^1 & \text{--} & \text{Y} & \text{N} & \text{N} & \text{--} & 5 & 1 & 0 & 1.0000000 & \text{--} & 160 & 1206 & -1046 \\
 250 & 2^1 5^3 & \text{--} & \text{N} & \text{N} & \text{Y} & \text{--} & 9 & 1 & 4 & 1.5555556 & \text{--} & 169 & 1215 & -1046 \\
 251 & 251^1 & \text{--} & \text{Y} & \text{Y} & \text{N} & \text{--} & -2 & 1 & 0 & 1.0000000 & \text{--} & 167 & 1215 & -1048 \\
 252 & 2^2 3^2 7^1 & \text{--} & \text{N} & \text{N} & \text{Y} & \text{--} & -74 & 1 & 58 & 1.2162162 & \text{--} & 93 & 1215 & -1122 \\
 253 & 11^1 23^1 & \text{--} & \text{Y} & \text{N} & \text{N} & \text{--} & 5 & 1 & 0 & 1.0000000 & \text{--} & 98 & 1220 & -1122 \\
 254 & 2^1 127^1 & \text{--} & \text{Y} & \text{N} & \text{N} & \text{--} & 5 & 1 & 0 & 1.0000000 & \text{--} & 103 & 1225 & -1122 \\
 255 & 3^1 5^1 17^1 & \text{--} & \text{Y} & \text{N} & \text{N} & \text{--} & -16 & 1 & 0 & 1.0000000 & \text{--} & 87 & 1225 & -1138 \\
 256 & 2^8 & \text{--} & \text{N} & \text{Y} & \text{N} & \text{--} & 2 & 1 & 0 & 4.5000000 & \text{--} & 89 & 1227 & -1138 \\
 257 & 257^1 & \text{--} & \text{Y} & \text{Y} & \text{N} & \text{--} & -2 & 1 & 0 & 1.0000000 & \text{--} & 87 & 1227 & -1140 \\
 258 & 2^1 3^1 43^1 & \text{--} & \text{Y} & \text{N} & \text{N} & \text{--} & -16 & 1 & 0 & 1.0000000 & \text{--} & 71 & 1227 & -1156 \\
 259 & 7^1 37^1 & \text{--} & \text{Y} & \text{N} & \text{N} & \text{--} & 5 & 1 & 0 & 1.0000000 & \text{--} & 76 & 1232 & -1156 \\
 260 & 2^2 5^1 13^1 & \text{--} & \text{N} & \text{N} & \text{Y} & \text{--} & 30 & 1 & 14 & 1.1666667 & \text{--} & 106 & 1262 & -1156 \\
 261 & 3^2 29^1 & \text{--} & \text{N} & \text{N} & \text{Y} & \text{--} & -7 & 1 & 2 & 1.2857143 & \text{--} & 99 & 1262 & -1163 \\
 262 & 2^1 131^1 & \text{--} & \text{Y} & \text{N} & \text{N} & \text{--} & 5 & 1 & 0 & 1.0000000 & \text{--} & 104 & 1267 & -1163 \\
 263 & 263^1 & \text{--} & \text{Y} & \text{Y} & \text{N} & \text{--} & -2 & 1 & 0 & 1.0000000 & \text{--} & 102 & 1267 & -1165 \\
 264 & 2^3 3^1 11^1 & \text{--} & \text{N} & \text{N} & \text{Y} & \text{--} & -48 & 1 & 32 & 1.3333333 & \text{--} & 54 & 1267 & -1213 \\
 265 & 5^1 53^1 & \text{--} & \text{Y} & \text{N} & \text{N} & \text{--} & 5 & 1 & 0 & 1.0000000 & \text{--} & 59 & 1272 & -1213 \\
 266 & 2^1 7^1 19^1 & \text{--} & \text{Y} & \text{N} & \text{N} & \text{--} & -16 & 1 & 0 & 1.0000000 & \text{--} & 43 & 1272 & -1229 \\
 267 & 3^1 89^1 & \text{--} & \text{Y} & \text{N} & \text{N} & \text{--} & 5 & 1 & 0 & 1.0000000 & \text{--} & 48 & 1277 & -1229 \\
 268 & 2^2 67^1 & \text{--} & \text{N} & \text{N} & \text{Y} & \text{--} & -7 & 1 & 2 & 1.2857143 & \text{--} & 41 & 1277 & -1236 \\
 269 & 269^1 & \text{--} & \text{Y} & \text{Y} & \text{N} & \text{--} & -2 & 1 & 0 & 1.0000000 & \text{--} & 39 & 1277 & -1238 \\
 270 & 2^1 3^3 5^1 & \text{--} & \text{N} & \text{N} & \text{Y} & \text{--} & -48 & 1 & 32 & 1.3333333 & \text{--} & -9 & 1277 & -1286 \\
 271 & 271^1 & \text{--} & \text{Y} & \text{Y} & \text{N} & \text{--} & -2 & 1 & 0 & 1.0000000 & \text{--} & -11 & 1277 & -1288 \\
 272 & 2^4 17^1 & \text{--} & \text{N} & \text{N} & \text{Y} & \text{--} & -11 & 1 & 6 & 1.8181818 & \text{--} & -22 & 1277 & -1299 \\
 273 & 3^1 7^1 13^1 & \text{--} & \text{Y} & \text{N} & \text{N} & \text{--} & -16 & 1 & 0 & 1.0000000 & \text{--} & -38 & 1277 & -1315 \\
 274 & 2^1 137^1 & \text{--} & \text{Y} & \text{N} & \text{N} & \text{--} & 5 & 1 & 0 & 1.0000000 & \text{--} & -33 & 1282 & -1315 \\
 275 & 5^2 11^1 & \text{--} & \text{N} & \text{N} & \text{Y} & \text{--} & -7 & 1 & 2 & 1.2857143 & \text{--} & -40 & 1282 & -1322 \\
 276 & 2^2 3^1 23^1 & \text{--} & \text{N} & \text{N} & \text{Y} & \text{--} & 30 & 1 & 14 & 1.1666667 & \text{--} & -10 & 1312 & -1322 \\
 277 & 277^1 & \text{--} & \text{Y} & \text{Y} & \text{N} & \text{--} & -2 & 1 & 0 & 1.0000000 & \text{--} & -12 & 1312 & -1324 \\
\end{array}
}
\end{equation*}

\end{table} 

\newpage
\begin{table}[h!]

\centering

\tiny
\begin{equation*}
\boxed{
\begin{array}{|cc|c|ccc|c|c|ccc|c|ccc}
 n & \mathbf{Primes} & & \mathbf{Sqfree} & \mathbf{PPower} & \bar{\mathbb{S}} & & g^{-1}(n) & 
 \lambda(n) \operatorname{sgn}(g^{-1}(n)) & \lambda(n) g^{-1}(n) - \widehat{f}_1(n) & 
 \frac{\sum\limits_{d|n} C_{\Omega(d)}(d)}{|g^{-1}(n)|} & & G^{-1}(n) & G^{-1}_{+}(n) & G^{-1}_{-}(n) \\ \hline 

 278 & 2^1 139^1 & \text{--} & \text{Y} & \text{N} & \text{N} & \text{--} & 5 & 1 & 0 & 1.0000000 & \text{--} & -7 & 1317 & -1324 \\
 279 & 3^2 31^1 & \text{--} & \text{N} & \text{N} & \text{Y} & \text{--} & -7 & 1 & 2 & 1.2857143 & \text{--} & -14 & 1317 & -1331 \\
 280 & 2^3 5^1 7^1 & \text{--} & \text{N} & \text{N} & \text{Y} & \text{--} & -48 & 1 & 32 & 1.3333333 & \text{--} & -62 & 1317 & -1379 \\
 281 & 281^1 & \text{--} & \text{Y} & \text{Y} & \text{N} & \text{--} & -2 & 1 & 0 & 1.0000000 & \text{--} & -64 & 1317 & -1381 \\
 282 & 2^1 3^1 47^1 & \text{--} & \text{Y} & \text{N} & \text{N} & \text{--} & -16 & 1 & 0 & 1.0000000 & \text{--} & -80 & 1317 & -1397 \\
 283 & 283^1 & \text{--} & \text{Y} & \text{Y} & \text{N} & \text{--} & -2 & 1 & 0 & 1.0000000 & \text{--} & -82 & 1317 & -1399 \\
 284 & 2^2 71^1 & \text{--} & \text{N} & \text{N} & \text{Y} & \text{--} & -7 & 1 & 2 & 1.2857143 & \text{--} & -89 & 1317 & -1406 \\
 285 & 3^1 5^1 19^1 & \text{--} & \text{Y} & \text{N} & \text{N} & \text{--} & -16 & 1 & 0 & 1.0000000 & \text{--} & -105 & 1317 & -1422 \\
 286 & 2^1 11^1 13^1 & \text{--} & \text{Y} & \text{N} & \text{N} & \text{--} & -16 & 1 & 0 & 1.0000000 & \text{--} & -121 & 1317 & -1438 \\
 287 & 7^1 41^1 & \text{--} & \text{Y} & \text{N} & \text{N} & \text{--} & 5 & 1 & 0 & 1.0000000 & \text{--} & -116 & 1322 & -1438 \\
 288 & 2^5 3^2 & \text{--} & \text{N} & \text{N} & \text{Y} & \text{--} & -47 & 1 & 42 & 1.7659574 & \text{--} & -163 & 1322 & -1485 \\
 289 & 17^2 & \text{--} & \text{N} & \text{Y} & \text{N} & \text{--} & 2 & 1 & 0 & 1.5000000 & \text{--} & -161 & 1324 & -1485 \\
 290 & 2^1 5^1 29^1 & \text{--} & \text{Y} & \text{N} & \text{N} & \text{--} & -16 & 1 & 0 & 1.0000000 & \text{--} & -177 & 1324 & -1501 \\
 291 & 3^1 97^1 & \text{--} & \text{Y} & \text{N} & \text{N} & \text{--} & 5 & 1 & 0 & 1.0000000 & \text{--} & -172 & 1329 & -1501 \\
 292 & 2^2 73^1 & \text{--} & \text{N} & \text{N} & \text{Y} & \text{--} & -7 & 1 & 2 & 1.2857143 & \text{--} & -179 & 1329 & -1508 \\
 293 & 293^1 & \text{--} & \text{Y} & \text{Y} & \text{N} & \text{--} & -2 & 1 & 0 & 1.0000000 & \text{--} & -181 & 1329 & -1510 \\
 294 & 2^1 3^1 7^2 & \text{--} & \text{N} & \text{N} & \text{Y} & \text{--} & 30 & 1 & 14 & 1.1666667 & \text{--} & -151 & 1359 & -1510 \\
 295 & 5^1 59^1 & \text{--} & \text{Y} & \text{N} & \text{N} & \text{--} & 5 & 1 & 0 & 1.0000000 & \text{--} & -146 & 1364 & -1510 \\
 296 & 2^3 37^1 & \text{--} & \text{N} & \text{N} & \text{Y} & \text{--} & 9 & 1 & 4 & 1.5555556 & \text{--} & -137 & 1373 & -1510 \\
 297 & 3^3 11^1 & \text{--} & \text{N} & \text{N} & \text{Y} & \text{--} & 9 & 1 & 4 & 1.5555556 & \text{--} & -128 & 1382 & -1510 \\
 298 & 2^1 149^1 & \text{--} & \text{Y} & \text{N} & \text{N} & \text{--} & 5 & 1 & 0 & 1.0000000 & \text{--} & -123 & 1387 & -1510 \\
 299 & 13^1 23^1 & \text{--} & \text{Y} & \text{N} & \text{N} & \text{--} & 5 & 1 & 0 & 1.0000000 & \text{--} & -118 & 1392 & -1510 \\
 300 & 2^2 3^1 5^2 & \text{--} & \text{N} & \text{N} & \text{Y} & \text{--} & -74 & 1 & 58 & 1.2162162 & \text{--} & -192 & 1392 & -1584 \\
 301 & 7^1 43^1 & \text{--} & \text{Y} & \text{N} & \text{N} & \text{--} & 5 & 1 & 0 & 1.0000000 & \text{--} & -187 & 1397 & -1584 \\
 302 & 2^1 151^1 & \text{--} & \text{Y} & \text{N} & \text{N} & \text{--} & 5 & 1 & 0 & 1.0000000 & \text{--} & -182 & 1402 & -1584 \\
 303 & 3^1 101^1 & \text{--} & \text{Y} & \text{N} & \text{N} & \text{--} & 5 & 1 & 0 & 1.0000000 & \text{--} & -177 & 1407 & -1584 \\
 304 & 2^4 19^1 & \text{--} & \text{N} & \text{N} & \text{Y} & \text{--} & -11 & 1 & 6 & 1.8181818 & \text{--} & -188 & 1407 & -1595 \\
 305 & 5^1 61^1 & \text{--} & \text{Y} & \text{N} & \text{N} & \text{--} & 5 & 1 & 0 & 1.0000000 & \text{--} & -183 & 1412 & -1595 \\
 306 & 2^1 3^2 17^1 & \text{--} & \text{N} & \text{N} & \text{Y} & \text{--} & 30 & 1 & 14 & 1.1666667 & \text{--} & -153 & 1442 & -1595 \\
 307 & 307^1 & \text{--} & \text{Y} & \text{Y} & \text{N} & \text{--} & -2 & 1 & 0 & 1.0000000 & \text{--} & -155 & 1442 & -1597 \\
 308 & 2^2 7^1 11^1 & \text{--} & \text{N} & \text{N} & \text{Y} & \text{--} & 30 & 1 & 14 & 1.1666667 & \text{--} & -125 & 1472 & -1597 \\
 309 & 3^1 103^1 & \text{--} & \text{Y} & \text{N} & \text{N} & \text{--} & 5 & 1 & 0 & 1.0000000 & \text{--} & -120 & 1477 & -1597 \\
 310 & 2^1 5^1 31^1 & \text{--} & \text{Y} & \text{N} & \text{N} & \text{--} & -16 & 1 & 0 & 1.0000000 & \text{--} & -136 & 1477 & -1613 \\
 311 & 311^1 & \text{--} & \text{Y} & \text{Y} & \text{N} & \text{--} & -2 & 1 & 0 & 1.0000000 & \text{--} & -138 & 1477 & -1615 \\
 312 & 2^3 3^1 13^1 & \text{--} & \text{N} & \text{N} & \text{Y} & \text{--} & -48 & 1 & 32 & 1.3333333 & \text{--} & -186 & 1477 & -1663 \\
 313 & 313^1 & \text{--} & \text{Y} & \text{Y} & \text{N} & \text{--} & -2 & 1 & 0 & 1.0000000 & \text{--} & -188 & 1477 & -1665 \\
 314 & 2^1 157^1 & \text{--} & \text{Y} & \text{N} & \text{N} & \text{--} & 5 & 1 & 0 & 1.0000000 & \text{--} & -183 & 1482 & -1665 \\
 315 & 3^2 5^1 7^1 & \text{--} & \text{N} & \text{N} & \text{Y} & \text{--} & 30 & 1 & 14 & 1.1666667 & \text{--} & -153 & 1512 & -1665 \\
 316 & 2^2 79^1 & \text{--} & \text{N} & \text{N} & \text{Y} & \text{--} & -7 & 1 & 2 & 1.2857143 & \text{--} & -160 & 1512 & -1672 \\
 317 & 317^1 & \text{--} & \text{Y} & \text{Y} & \text{N} & \text{--} & -2 & 1 & 0 & 1.0000000 & \text{--} & -162 & 1512 & -1674 \\
 318 & 2^1 3^1 53^1 & \text{--} & \text{Y} & \text{N} & \text{N} & \text{--} & -16 & 1 & 0 & 1.0000000 & \text{--} & -178 & 1512 & -1690 \\
 319 & 11^1 29^1 & \text{--} & \text{Y} & \text{N} & \text{N} & \text{--} & 5 & 1 & 0 & 1.0000000 & \text{--} & -173 & 1517 & -1690 \\
 320 & 2^6 5^1 & \text{--} & \text{N} & \text{N} & \text{Y} & \text{--} & -15 & 1 & 10 & 2.3333333 & \text{--} & -188 & 1517 & -1705 \\
 321 & 3^1 107^1 & \text{--} & \text{Y} & \text{N} & \text{N} & \text{--} & 5 & 1 & 0 & 1.0000000 & \text{--} & -183 & 1522 & -1705 \\
 322 & 2^1 7^1 23^1 & \text{--} & \text{Y} & \text{N} & \text{N} & \text{--} & -16 & 1 & 0 & 1.0000000 & \text{--} & -199 & 1522 & -1721 \\
 323 & 17^1 19^1 & \text{--} & \text{Y} & \text{N} & \text{N} & \text{--} & 5 & 1 & 0 & 1.0000000 & \text{--} & -194 & 1527 & -1721 \\
 324 & 2^2 3^4 & \text{--} & \text{N} & \text{N} & \text{Y} & \text{--} & 34 & 1 & 29 & 1.6176471 & \text{--} & -160 & 1561 & -1721 \\
 325 & 5^2 13^1 & \text{--} & \text{N} & \text{N} & \text{Y} & \text{--} & -7 & 1 & 2 & 1.2857143 & \text{--} & -167 & 1561 & -1728 \\
 326 & 2^1 163^1 & \text{--} & \text{Y} & \text{N} & \text{N} & \text{--} & 5 & 1 & 0 & 1.0000000 & \text{--} & -162 & 1566 & -1728 \\
 327 & 3^1 109^1 & \text{--} & \text{Y} & \text{N} & \text{N} & \text{--} & 5 & 1 & 0 & 1.0000000 & \text{--} & -157 & 1571 & -1728 \\
 328 & 2^3 41^1 & \text{--} & \text{N} & \text{N} & \text{Y} & \text{--} & 9 & 1 & 4 & 1.5555556 & \text{--} & -148 & 1580 & -1728 \\
 329 & 7^1 47^1 & \text{--} & \text{Y} & \text{N} & \text{N} & \text{--} & 5 & 1 & 0 & 1.0000000 & \text{--} & -143 & 1585 & -1728 \\
 330 & 2^1 3^1 5^1 11^1 & \text{--} & \text{Y} & \text{N} & \text{N} & \text{--} & 65 & 1 & 0 & 1.0000000 & \text{--} & -78 & 1650 & -1728 \\
 331 & 331^1 & \text{--} & \text{Y} & \text{Y} & \text{N} & \text{--} & -2 & 1 & 0 & 1.0000000 & \text{--} & -80 & 1650 & -1730 \\
 332 & 2^2 83^1 & \text{--} & \text{N} & \text{N} & \text{Y} & \text{--} & -7 & 1 & 2 & 1.2857143 & \text{--} & -87 & 1650 & -1737 \\
 333 & 3^2 37^1 & \text{--} & \text{N} & \text{N} & \text{Y} & \text{--} & -7 & 1 & 2 & 1.2857143 & \text{--} & -94 & 1650 & -1744 \\
 334 & 2^1 167^1 & \text{--} & \text{Y} & \text{N} & \text{N} & \text{--} & 5 & 1 & 0 & 1.0000000 & \text{--} & -89 & 1655 & -1744 \\
 335 & 5^1 67^1 & \text{--} & \text{Y} & \text{N} & \text{N} & \text{--} & 5 & 1 & 0 & 1.0000000 & \text{--} & -84 & 1660 & -1744 \\
 336 & 2^4 3^1 7^1 & \text{--} & \text{N} & \text{N} & \text{Y} & \text{--} & 70 & 1 & 54 & 1.5000000 & \text{--} & -14 & 1730 & -1744 \\
 337 & 337^1 & \text{--} & \text{Y} & \text{Y} & \text{N} & \text{--} & -2 & 1 & 0 & 1.0000000 & \text{--} & -16 & 1730 & -1746 \\
 338 & 2^1 13^2 & \text{--} & \text{N} & \text{N} & \text{Y} & \text{--} & -7 & 1 & 2 & 1.2857143 & \text{--} & -23 & 1730 & -1753 \\
 339 & 3^1 113^1 & \text{--} & \text{Y} & \text{N} & \text{N} & \text{--} & 5 & 1 & 0 & 1.0000000 & \text{--} & -18 & 1735 & -1753 \\
 340 & 2^2 5^1 17^1 & \text{--} & \text{N} & \text{N} & \text{Y} & \text{--} & 30 & 1 & 14 & 1.1666667 & \text{--} & 12 & 1765 & -1753 \\
 341 & 11^1 31^1 & \text{--} & \text{Y} & \text{N} & \text{N} & \text{--} & 5 & 1 & 0 & 1.0000000 & \text{--} & 17 & 1770 & -1753 \\
 342 & 2^1 3^2 19^1 & \text{--} & \text{N} & \text{N} & \text{Y} & \text{--} & 30 & 1 & 14 & 1.1666667 & \text{--} & 47 & 1800 & -1753 \\
 343 & 7^3 & \text{--} & \text{N} & \text{Y} & \text{N} & \text{--} & -2 & 1 & 0 & 2.0000000 & \text{--} & 45 & 1800 & -1755 \\
 344 & 2^3 43^1 & \text{--} & \text{N} & \text{N} & \text{Y} & \text{--} & 9 & 1 & 4 & 1.5555556 & \text{--} & 54 & 1809 & -1755 \\
 345 & 3^1 5^1 23^1 & \text{--} & \text{Y} & \text{N} & \text{N} & \text{--} & -16 & 1 & 0 & 1.0000000 & \text{--} & 38 & 1809 & -1771 \\
 346 & 2^1 173^1 & \text{--} & \text{Y} & \text{N} & \text{N} & \text{--} & 5 & 1 & 0 & 1.0000000 & \text{--} & 43 & 1814 & -1771 \\
 347 & 347^1 & \text{--} & \text{Y} & \text{Y} & \text{N} & \text{--} & -2 & 1 & 0 & 1.0000000 & \text{--} & 41 & 1814 & -1773 \\
 348 & 2^2 3^1 29^1 & \text{--} & \text{N} & \text{N} & \text{Y} & \text{--} & 30 & 1 & 14 & 1.1666667 & \text{--} & 71 & 1844 & -1773 \\
 349 & 349^1 & \text{--} & \text{Y} & \text{Y} & \text{N} & \text{--} & -2 & 1 & 0 & 1.0000000 & \text{--} & 69 & 1844 & -1775 \\
 350 & 2^1 5^2 7^1 & \text{--} & \text{N} & \text{N} & \text{Y} & \text{--} & 30 & 1 & 14 & 1.1666667 & \text{--} & 99 & 1874 & -1775 \\
\end{array}
}
\end{equation*}

\end{table} 

\newpage
\begin{table}[h!]

\centering
\tiny
\begin{equation*}
\boxed{
\begin{array}{|cc|c|ccc|c|c|ccc|c|ccc}
 n & \mathbf{Primes} & & \mathbf{Sqfree} & \mathbf{PPower} & \bar{\mathbb{S}} & & g^{-1}(n) & 
 \lambda(n) \operatorname{sgn}(g^{-1}(n)) & \lambda(n) g^{-1}(n) - \widehat{f}_1(n) & 
 \frac{\sum\limits_{d|n} C_{\Omega(d)}(d)}{|g^{-1}(n)|} & & G^{-1}(n) & G^{-1}_{+}(n) & G^{-1}_{-}(n) \\ \hline 

 351 & 3^3 13^1 & \text{--} & \text{N} & \text{N} & \text{Y} & \text{--} & 9 & 1 & 4 & 1.5555556 & \text{--} & 108 & 1883 & -1775 \\
 352 & 2^5 11^1 & \text{--} & \text{N} & \text{N} & \text{Y} & \text{--} & 13 & 1 & 8 & 2.0769231 & \text{--} & 121 & 1896 & -1775 \\
 353 & 353^1 & \text{--} & \text{Y} & \text{Y} & \text{N} & \text{--} & -2 & 1 & 0 & 1.0000000 & \text{--} & 119 & 1896 & -1777 \\
 354 & 2^1 3^1 59^1 & \text{--} & \text{Y} & \text{N} & \text{N} & \text{--} & -16 & 1 & 0 & 1.0000000 & \text{--} & 103 & 1896 & -1793 \\
 355 & 5^1 71^1 & \text{--} & \text{Y} & \text{N} & \text{N} & \text{--} & 5 & 1 & 0 & 1.0000000 & \text{--} & 108 & 1901 & -1793 \\
 356 & 2^2 89^1 & \text{--} & \text{N} & \text{N} & \text{Y} & \text{--} & -7 & 1 & 2 & 1.2857143 & \text{--} & 101 & 1901 & -1800 \\
 357 & 3^1 7^1 17^1 & \text{--} & \text{Y} & \text{N} & \text{N} & \text{--} & -16 & 1 & 0 & 1.0000000 & \text{--} & 85 & 1901 & -1816 \\
 358 & 2^1 179^1 & \text{--} & \text{Y} & \text{N} & \text{N} & \text{--} & 5 & 1 & 0 & 1.0000000 & \text{--} & 90 & 1906 & -1816 \\
 359 & 359^1 & \text{--} & \text{Y} & \text{Y} & \text{N} & \text{--} & -2 & 1 & 0 & 1.0000000 & \text{--} & 88 & 1906 & -1818 \\
 360 & 2^3 3^2 5^1 & \text{--} & \text{N} & \text{N} & \text{Y} & \text{--} & 145 & 1 & 129 & 1.3034483 & \text{--} & 233 & 2051 & -1818 \\
 361 & 19^2 & \text{--} & \text{N} & \text{Y} & \text{N} & \text{--} & 2 & 1 & 0 & 1.5000000 & \text{--} & 235 & 2053 & -1818 \\
 362 & 2^1 181^1 & \text{--} & \text{Y} & \text{N} & \text{N} & \text{--} & 5 & 1 & 0 & 1.0000000 & \text{--} & 240 & 2058 & -1818 \\
 363 & 3^1 11^2 & \text{--} & \text{N} & \text{N} & \text{Y} & \text{--} & -7 & 1 & 2 & 1.2857143 & \text{--} & 233 & 2058 & -1825 \\
 364 & 2^2 7^1 13^1 & \text{--} & \text{N} & \text{N} & \text{Y} & \text{--} & 30 & 1 & 14 & 1.1666667 & \text{--} & 263 & 2088 & -1825 \\
 365 & 5^1 73^1 & \text{--} & \text{Y} & \text{N} & \text{N} & \text{--} & 5 & 1 & 0 & 1.0000000 & \text{--} & 268 & 2093 & -1825 \\
 366 & 2^1 3^1 61^1 & \text{--} & \text{Y} & \text{N} & \text{N} & \text{--} & -16 & 1 & 0 & 1.0000000 & \text{--} & 252 & 2093 & -1841 \\
 367 & 367^1 & \text{--} & \text{Y} & \text{Y} & \text{N} & \text{--} & -2 & 1 & 0 & 1.0000000 & \text{--} & 250 & 2093 & -1843 \\
 368 & 2^4 23^1 & \text{--} & \text{N} & \text{N} & \text{Y} & \text{--} & -11 & 1 & 6 & 1.8181818 & \text{--} & 239 & 2093 & -1854 \\
 369 & 3^2 41^1 & \text{--} & \text{N} & \text{N} & \text{Y} & \text{--} & -7 & 1 & 2 & 1.2857143 & \text{--} & 232 & 2093 & -1861 \\
 370 & 2^1 5^1 37^1 & \text{--} & \text{Y} & \text{N} & \text{N} & \text{--} & -16 & 1 & 0 & 1.0000000 & \text{--} & 216 & 2093 & -1877 \\
 371 & 7^1 53^1 & \text{--} & \text{Y} & \text{N} & \text{N} & \text{--} & 5 & 1 & 0 & 1.0000000 & \text{--} & 221 & 2098 & -1877 \\
 372 & 2^2 3^1 31^1 & \text{--} & \text{N} & \text{N} & \text{Y} & \text{--} & 30 & 1 & 14 & 1.1666667 & \text{--} & 251 & 2128 & -1877 \\
 373 & 373^1 & \text{--} & \text{Y} & \text{Y} & \text{N} & \text{--} & -2 & 1 & 0 & 1.0000000 & \text{--} & 249 & 2128 & -1879 \\
 374 & 2^1 11^1 17^1 & \text{--} & \text{Y} & \text{N} & \text{N} & \text{--} & -16 & 1 & 0 & 1.0000000 & \text{--} & 233 & 2128 & -1895 \\
 375 & 3^1 5^3 & \text{--} & \text{N} & \text{N} & \text{Y} & \text{--} & 9 & 1 & 4 & 1.5555556 & \text{--} & 242 & 2137 & -1895 \\
 376 & 2^3 47^1 & \text{--} & \text{N} & \text{N} & \text{Y} & \text{--} & 9 & 1 & 4 & 1.5555556 & \text{--} & 251 & 2146 & -1895 \\
 377 & 13^1 29^1 & \text{--} & \text{Y} & \text{N} & \text{N} & \text{--} & 5 & 1 & 0 & 1.0000000 & \text{--} & 256 & 2151 & -1895 \\
 378 & 2^1 3^3 7^1 & \text{--} & \text{N} & \text{N} & \text{Y} & \text{--} & -48 & 1 & 32 & 1.3333333 & \text{--} & 208 & 2151 & -1943 \\
 379 & 379^1 & \text{--} & \text{Y} & \text{Y} & \text{N} & \text{--} & -2 & 1 & 0 & 1.0000000 & \text{--} & 206 & 2151 & -1945 \\
 380 & 2^2 5^1 19^1 & \text{--} & \text{N} & \text{N} & \text{Y} & \text{--} & 30 & 1 & 14 & 1.1666667 & \text{--} & 236 & 2181 & -1945 \\
 381 & 3^1 127^1 & \text{--} & \text{Y} & \text{N} & \text{N} & \text{--} & 5 & 1 & 0 & 1.0000000 & \text{--} & 241 & 2186 & -1945 \\
 382 & 2^1 191^1 & \text{--} & \text{Y} & \text{N} & \text{N} & \text{--} & 5 & 1 & 0 & 1.0000000 & \text{--} & 246 & 2191 & -1945 \\
 383 & 383^1 & \text{--} & \text{Y} & \text{Y} & \text{N} & \text{--} & -2 & 1 & 0 & 1.0000000 & \text{--} & 244 & 2191 & -1947 \\
 384 & 2^7 3^1 & \text{--} & \text{N} & \text{N} & \text{Y} & \text{--} & 17 & 1 & 12 & 2.5882353 & \text{--} & 261 & 2208 & -1947 \\
 385 & 5^1 7^1 11^1 & \text{--} & \text{Y} & \text{N} & \text{N} & \text{--} & -16 & 1 & 0 & 1.0000000 & \text{--} & 245 & 2208 & -1963 \\
 386 & 2^1 193^1 & \text{--} & \text{Y} & \text{N} & \text{N} & \text{--} & 5 & 1 & 0 & 1.0000000 & \text{--} & 250 & 2213 & -1963 \\
 387 & 3^2 43^1 & \text{--} & \text{N} & \text{N} & \text{Y} & \text{--} & -7 & 1 & 2 & 1.2857143 & \text{--} & 243 & 2213 & -1970 \\
 388 & 2^2 97^1 & \text{--} & \text{N} & \text{N} & \text{Y} & \text{--} & -7 & 1 & 2 & 1.2857143 & \text{--} & 236 & 2213 & -1977 \\
 389 & 389^1 & \text{--} & \text{Y} & \text{Y} & \text{N} & \text{--} & -2 & 1 & 0 & 1.0000000 & \text{--} & 234 & 2213 & -1979 \\
 390 & 2^1 3^1 5^1 13^1 & \text{--} & \text{Y} & \text{N} & \text{N} & \text{--} & 65 & 1 & 0 & 1.0000000 & \text{--} & 299 & 2278 & -1979 \\
 391 & 17^1 23^1 & \text{--} & \text{Y} & \text{N} & \text{N} & \text{--} & 5 & 1 & 0 & 1.0000000 & \text{--} & 304 & 2283 & -1979 \\
 392 & 2^3 7^2 & \text{--} & \text{N} & \text{N} & \text{Y} & \text{--} & -23 & 1 & 18 & 1.4782609 & \text{--} & 281 & 2283 & -2002 \\
 393 & 3^1 131^1 & \text{--} & \text{Y} & \text{N} & \text{N} & \text{--} & 5 & 1 & 0 & 1.0000000 & \text{--} & 286 & 2288 & -2002 \\
 394 & 2^1 197^1 & \text{--} & \text{Y} & \text{N} & \text{N} & \text{--} & 5 & 1 & 0 & 1.0000000 & \text{--} & 291 & 2293 & -2002 \\
 395 & 5^1 79^1 & \text{--} & \text{Y} & \text{N} & \text{N} & \text{--} & 5 & 1 & 0 & 1.0000000 & \text{--} & 296 & 2298 & -2002 \\
 396 & 2^2 3^2 11^1 & \text{--} & \text{N} & \text{N} & \text{Y} & \text{--} & -74 & 1 & 58 & 1.2162162 & \text{--} & 222 & 2298 & -2076 \\
 397 & 397^1 & \text{--} & \text{Y} & \text{Y} & \text{N} & \text{--} & -2 & 1 & 0 & 1.0000000 & \text{--} & 220 & 2298 & -2078 \\
 398 & 2^1 199^1 & \text{--} & \text{Y} & \text{N} & \text{N} & \text{--} & 5 & 1 & 0 & 1.0000000 & \text{--} & 225 & 2303 & -2078 \\
 399 & 3^1 7^1 19^1 & \text{--} & \text{Y} & \text{N} & \text{N} & \text{--} & -16 & 1 & 0 & 1.0000000 & \text{--} & 209 & 2303 & -2094 \\
 400 & 2^4 5^2 & \text{--} & \text{N} & \text{N} & \text{Y} & \text{--} & 34 & 1 & 29 & 1.6176471 & \text{--} & 243 & 2337 & -2094 \\
 401 & 401^1 & \text{--} & \text{Y} & \text{Y} & \text{N} & \text{--} & -2 & 1 & 0 & 1.0000000 & \text{--} & 241 & 2337 & -2096 \\
 402 & 2^1 3^1 67^1 & \text{--} & \text{Y} & \text{N} & \text{N} & \text{--} & -16 & 1 & 0 & 1.0000000 & \text{--} & 225 & 2337 & -2112 \\
 403 & 13^1 31^1 & \text{--} & \text{Y} & \text{N} & \text{N} & \text{--} & 5 & 1 & 0 & 1.0000000 & \text{--} & 230 & 2342 & -2112 \\
 404 & 2^2 101^1 & \text{--} & \text{N} & \text{N} & \text{Y} & \text{--} & -7 & 1 & 2 & 1.2857143 & \text{--} & 223 & 2342 & -2119 \\
 405 & 3^4 5^1 & \text{--} & \text{N} & \text{N} & \text{Y} & \text{--} & -11 & 1 & 6 & 1.8181818 & \text{--} & 212 & 2342 & -2130 \\
 406 & 2^1 7^1 29^1 & \text{--} & \text{Y} & \text{N} & \text{N} & \text{--} & -16 & 1 & 0 & 1.0000000 & \text{--} & 196 & 2342 & -2146 \\
 407 & 11^1 37^1 & \text{--} & \text{Y} & \text{N} & \text{N} & \text{--} & 5 & 1 & 0 & 1.0000000 & \text{--} & 201 & 2347 & -2146 \\
 408 & 2^3 3^1 17^1 & \text{--} & \text{N} & \text{N} & \text{Y} & \text{--} & -48 & 1 & 32 & 1.3333333 & \text{--} & 153 & 2347 & -2194 \\
 409 & 409^1 & \text{--} & \text{Y} & \text{Y} & \text{N} & \text{--} & -2 & 1 & 0 & 1.0000000 & \text{--} & 151 & 2347 & -2196 \\
 410 & 2^1 5^1 41^1 & \text{--} & \text{Y} & \text{N} & \text{N} & \text{--} & -16 & 1 & 0 & 1.0000000 & \text{--} & 135 & 2347 & -2212 \\
 411 & 3^1 137^1 & \text{--} & \text{Y} & \text{N} & \text{N} & \text{--} & 5 & 1 & 0 & 1.0000000 & \text{--} & 140 & 2352 & -2212 \\
 412 & 2^2 103^1 & \text{--} & \text{N} & \text{N} & \text{Y} & \text{--} & -7 & 1 & 2 & 1.2857143 & \text{--} & 133 & 2352 & -2219 \\
 413 & 7^1 59^1 & \text{--} & \text{Y} & \text{N} & \text{N} & \text{--} & 5 & 1 & 0 & 1.0000000 & \text{--} & 138 & 2357 & -2219 \\
 414 & 2^1 3^2 23^1 & \text{--} & \text{N} & \text{N} & \text{Y} & \text{--} & 30 & 1 & 14 & 1.1666667 & \text{--} & 168 & 2387 & -2219 \\
 415 & 5^1 83^1 & \text{--} & \text{Y} & \text{N} & \text{N} & \text{--} & 5 & 1 & 0 & 1.0000000 & \text{--} & 173 & 2392 & -2219 \\
 416 & 2^5 13^1 & \text{--} & \text{N} & \text{N} & \text{Y} & \text{--} & 13 & 1 & 8 & 2.0769231 & \text{--} & 186 & 2405 & -2219 \\
 417 & 3^1 139^1 & \text{--} & \text{Y} & \text{N} & \text{N} & \text{--} & 5 & 1 & 0 & 1.0000000 & \text{--} & 191 & 2410 & -2219 \\
 418 & 2^1 11^1 19^1 & \text{--} & \text{Y} & \text{N} & \text{N} & \text{--} & -16 & 1 & 0 & 1.0000000 & \text{--} & 175 & 2410 & -2235 \\
 419 & 419^1 & \text{--} & \text{Y} & \text{Y} & \text{N} & \text{--} & -2 & 1 & 0 & 1.0000000 & \text{--} & 173 & 2410 & -2237 \\
 420 & 2^2 3^1 5^1 7^1 & \text{--} & \text{N} & \text{N} & \text{Y} & \text{--} & -155 & 1 & 90 & 1.1032258 & \text{--} & 18 & 2410 & -2392 \\
 421 & 421^1 & \text{--} & \text{Y} & \text{Y} & \text{N} & \text{--} & -2 & 1 & 0 & 1.0000000 & \text{--} & 16 & 2410 & -2394 \\
 422 & 2^1 211^1 & \text{--} & \text{Y} & \text{N} & \text{N} & \text{--} & 5 & 1 & 0 & 1.0000000 & \text{--} & 21 & 2415 & -2394 \\
 423 & 3^2 47^1 & \text{--} & \text{N} & \text{N} & \text{Y} & \text{--} & -7 & 1 & 2 & 1.2857143 & \text{--} & 14 & 2415 & -2401 \\
 424 & 2^3 53^1 & \text{--} & \text{N} & \text{N} & \text{Y} & \text{--} & 9 & 1 & 4 & 1.5555556 & \text{--} & 23 & 2424 & -2401 \\
 425 & 5^2 17^1 & \text{--} & \text{N} & \text{N} & \text{Y} & \text{--} & -7 & 1 & 2 & 1.2857143 & \text{--} & 16 & 2424 & -2408 \\
\end{array}
}
\end{equation*}

\end{table} 

\newpage
\begin{table}[h!]

\centering
\tiny
\begin{equation*}
\boxed{
\begin{array}{|cc|c|ccc|c|c|ccc|c|ccc}
 n & \mathbf{Primes} & & \mathbf{Sqfree} & \mathbf{PPower} & \bar{\mathbb{S}} & & g^{-1}(n) & 
 \lambda(n) \operatorname{sgn}(g^{-1}(n)) & \lambda(n) g^{-1}(n) - \widehat{f}_1(n) & 
 \frac{\sum\limits_{d|n} C_{\Omega(d)}(d)}{|g^{-1}(n)|} & & G^{-1}(n) & G^{-1}_{+}(n) & G^{-1}_{-}(n) \\ \hline 

 426 & 2^1 3^1 71^1 & \text{--} & \text{Y} & \text{N} & \text{N} & \text{--} & -16 & 1 & 0 & 1.0000000 & \text{--} & 0 & 2424 & -2424 \\
 427 & 7^1 61^1 & \text{--} & \text{Y} & \text{N} & \text{N} & \text{--} & 5 & 1 & 0 & 1.0000000 & \text{--} & 5 & 2429 & -2424 \\
 428 & 2^2 107^1 & \text{--} & \text{N} & \text{N} & \text{Y} & \text{--} & -7 & 1 & 2 & 1.2857143 & \text{--} & -2 & 2429 & -2431 \\
 429 & 3^1 11^1 13^1 & \text{--} & \text{Y} & \text{N} & \text{N} & \text{--} & -16 & 1 & 0 & 1.0000000 & \text{--} & -18 & 2429 & -2447 \\
 430 & 2^1 5^1 43^1 & \text{--} & \text{Y} & \text{N} & \text{N} & \text{--} & -16 & 1 & 0 & 1.0000000 & \text{--} & -34 & 2429 & -2463 \\
 431 & 431^1 & \text{--} & \text{Y} & \text{Y} & \text{N} & \text{--} & -2 & 1 & 0 & 1.0000000 & \text{--} & -36 & 2429 & -2465 \\
 432 & 2^4 3^3 & \text{--} & \text{N} & \text{N} & \text{Y} & \text{--} & -80 & 1 & 75 & 1.5625000 & \text{--} & -116 & 2429 & -2545 \\
 433 & 433^1 & \text{--} & \text{Y} & \text{Y} & \text{N} & \text{--} & -2 & 1 & 0 & 1.0000000 & \text{--} & -118 & 2429 & -2547 \\
 434 & 2^1 7^1 31^1 & \text{--} & \text{Y} & \text{N} & \text{N} & \text{--} & -16 & 1 & 0 & 1.0000000 & \text{--} & -134 & 2429 & -2563 \\
 435 & 3^1 5^1 29^1 & \text{--} & \text{Y} & \text{N} & \text{N} & \text{--} & -16 & 1 & 0 & 1.0000000 & \text{--} & -150 & 2429 & -2579 \\
 436 & 2^2 109^1 & \text{--} & \text{N} & \text{N} & \text{Y} & \text{--} & -7 & 1 & 2 & 1.2857143 & \text{--} & -157 & 2429 & -2586 \\
 437 & 19^1 23^1 & \text{--} & \text{Y} & \text{N} & \text{N} & \text{--} & 5 & 1 & 0 & 1.0000000 & \text{--} & -152 & 2434 & -2586 \\
 438 & 2^1 3^1 73^1 & \text{--} & \text{Y} & \text{N} & \text{N} & \text{--} & -16 & 1 & 0 & 1.0000000 & \text{--} & -168 & 2434 & -2602 \\
 439 & 439^1 & \text{--} & \text{Y} & \text{Y} & \text{N} & \text{--} & -2 & 1 & 0 & 1.0000000 & \text{--} & -170 & 2434 & -2604 \\
 440 & 2^3 5^1 11^1 & \text{--} & \text{N} & \text{N} & \text{Y} & \text{--} & -48 & 1 & 32 & 1.3333333 & \text{--} & -218 & 2434 & -2652 \\
 441 & 3^2 7^2 & \text{--} & \text{N} & \text{N} & \text{Y} & \text{--} & 14 & 1 & 9 & 1.3571429 & \text{--} & -204 & 2448 & -2652 \\
 442 & 2^1 13^1 17^1 & \text{--} & \text{Y} & \text{N} & \text{N} & \text{--} & -16 & 1 & 0 & 1.0000000 & \text{--} & -220 & 2448 & -2668 \\
 443 & 443^1 & \text{--} & \text{Y} & \text{Y} & \text{N} & \text{--} & -2 & 1 & 0 & 1.0000000 & \text{--} & -222 & 2448 & -2670 \\
 444 & 2^2 3^1 37^1 & \text{--} & \text{N} & \text{N} & \text{Y} & \text{--} & 30 & 1 & 14 & 1.1666667 & \text{--} & -192 & 2478 & -2670 \\
 445 & 5^1 89^1 & \text{--} & \text{Y} & \text{N} & \text{N} & \text{--} & 5 & 1 & 0 & 1.0000000 & \text{--} & -187 & 2483 & -2670 \\
 446 & 2^1 223^1 & \text{--} & \text{Y} & \text{N} & \text{N} & \text{--} & 5 & 1 & 0 & 1.0000000 & \text{--} & -182 & 2488 & -2670 \\
 447 & 3^1 149^1 & \text{--} & \text{Y} & \text{N} & \text{N} & \text{--} & 5 & 1 & 0 & 1.0000000 & \text{--} & -177 & 2493 & -2670 \\
 448 & 2^6 7^1 & \text{--} & \text{N} & \text{N} & \text{Y} & \text{--} & -15 & 1 & 10 & 2.3333333 & \text{--} & -192 & 2493 & -2685 \\
 449 & 449^1 & \text{--} & \text{Y} & \text{Y} & \text{N} & \text{--} & -2 & 1 & 0 & 1.0000000 & \text{--} & -194 & 2493 & -2687 \\
 450 & 2^1 3^2 5^2 & \text{--} & \text{N} & \text{N} & \text{Y} & \text{--} & -74 & 1 & 58 & 1.2162162 & \text{--} & -268 & 2493 & -2761 \\
 451 & 11^1 41^1 & \text{--} & \text{Y} & \text{N} & \text{N} & \text{--} & 5 & 1 & 0 & 1.0000000 & \text{--} & -263 & 2498 & -2761 \\
 452 & 2^2 113^1 & \text{--} & \text{N} & \text{N} & \text{Y} & \text{--} & -7 & 1 & 2 & 1.2857143 & \text{--} & -270 & 2498 & -2768 \\
 453 & 3^1 151^1 & \text{--} & \text{Y} & \text{N} & \text{N} & \text{--} & 5 & 1 & 0 & 1.0000000 & \text{--} & -265 & 2503 & -2768 \\
 454 & 2^1 227^1 & \text{--} & \text{Y} & \text{N} & \text{N} & \text{--} & 5 & 1 & 0 & 1.0000000 & \text{--} & -260 & 2508 & -2768 \\
 455 & 5^1 7^1 13^1 & \text{--} & \text{Y} & \text{N} & \text{N} & \text{--} & -16 & 1 & 0 & 1.0000000 & \text{--} & -276 & 2508 & -2784 \\
 456 & 2^3 3^1 19^1 & \text{--} & \text{N} & \text{N} & \text{Y} & \text{--} & -48 & 1 & 32 & 1.3333333 & \text{--} & -324 & 2508 & -2832 \\
 457 & 457^1 & \text{--} & \text{Y} & \text{Y} & \text{N} & \text{--} & -2 & 1 & 0 & 1.0000000 & \text{--} & -326 & 2508 & -2834 \\
 458 & 2^1 229^1 & \text{--} & \text{Y} & \text{N} & \text{N} & \text{--} & 5 & 1 & 0 & 1.0000000 & \text{--} & -321 & 2513 & -2834 \\
 459 & 3^3 17^1 & \text{--} & \text{N} & \text{N} & \text{Y} & \text{--} & 9 & 1 & 4 & 1.5555556 & \text{--} & -312 & 2522 & -2834 \\
 460 & 2^2 5^1 23^1 & \text{--} & \text{N} & \text{N} & \text{Y} & \text{--} & 30 & 1 & 14 & 1.1666667 & \text{--} & -282 & 2552 & -2834 \\
 461 & 461^1 & \text{--} & \text{Y} & \text{Y} & \text{N} & \text{--} & -2 & 1 & 0 & 1.0000000 & \text{--} & -284 & 2552 & -2836 \\
 462 & 2^1 3^1 7^1 11^1 & \text{--} & \text{Y} & \text{N} & \text{N} & \text{--} & 65 & 1 & 0 & 1.0000000 & \text{--} & -219 & 2617 & -2836 \\
 463 & 463^1 & \text{--} & \text{Y} & \text{Y} & \text{N} & \text{--} & -2 & 1 & 0 & 1.0000000 & \text{--} & -221 & 2617 & -2838 \\
 464 & 2^4 29^1 & \text{--} & \text{N} & \text{N} & \text{Y} & \text{--} & -11 & 1 & 6 & 1.8181818 & \text{--} & -232 & 2617 & -2849 \\
 465 & 3^1 5^1 31^1 & \text{--} & \text{Y} & \text{N} & \text{N} & \text{--} & -16 & 1 & 0 & 1.0000000 & \text{--} & -248 & 2617 & -2865 \\
 466 & 2^1 233^1 & \text{--} & \text{Y} & \text{N} & \text{N} & \text{--} & 5 & 1 & 0 & 1.0000000 & \text{--} & -243 & 2622 & -2865 \\
 467 & 467^1 & \text{--} & \text{Y} & \text{Y} & \text{N} & \text{--} & -2 & 1 & 0 & 1.0000000 & \text{--} & -245 & 2622 & -2867 \\
 468 & 2^2 3^2 13^1 & \text{--} & \text{N} & \text{N} & \text{Y} & \text{--} & -74 & 1 & 58 & 1.2162162 & \text{--} & -319 & 2622 & -2941 \\
 469 & 7^1 67^1 & \text{--} & \text{Y} & \text{N} & \text{N} & \text{--} & 5 & 1 & 0 & 1.0000000 & \text{--} & -314 & 2627 & -2941 \\
 470 & 2^1 5^1 47^1 & \text{--} & \text{Y} & \text{N} & \text{N} & \text{--} & -16 & 1 & 0 & 1.0000000 & \text{--} & -330 & 2627 & -2957 \\
 471 & 3^1 157^1 & \text{--} & \text{Y} & \text{N} & \text{N} & \text{--} & 5 & 1 & 0 & 1.0000000 & \text{--} & -325 & 2632 & -2957 \\
 472 & 2^3 59^1 & \text{--} & \text{N} & \text{N} & \text{Y} & \text{--} & 9 & 1 & 4 & 1.5555556 & \text{--} & -316 & 2641 & -2957 \\
 473 & 11^1 43^1 & \text{--} & \text{Y} & \text{N} & \text{N} & \text{--} & 5 & 1 & 0 & 1.0000000 & \text{--} & -311 & 2646 & -2957 \\
 474 & 2^1 3^1 79^1 & \text{--} & \text{Y} & \text{N} & \text{N} & \text{--} & -16 & 1 & 0 & 1.0000000 & \text{--} & -327 & 2646 & -2973 \\
 475 & 5^2 19^1 & \text{--} & \text{N} & \text{N} & \text{Y} & \text{--} & -7 & 1 & 2 & 1.2857143 & \text{--} & -334 & 2646 & -2980 \\
 476 & 2^2 7^1 17^1 & \text{--} & \text{N} & \text{N} & \text{Y} & \text{--} & 30 & 1 & 14 & 1.1666667 & \text{--} & -304 & 2676 & -2980 \\
 477 & 3^2 53^1 & \text{--} & \text{N} & \text{N} & \text{Y} & \text{--} & -7 & 1 & 2 & 1.2857143 & \text{--} & -311 & 2676 & -2987 \\
 478 & 2^1 239^1 & \text{--} & \text{Y} & \text{N} & \text{N} & \text{--} & 5 & 1 & 0 & 1.0000000 & \text{--} & -306 & 2681 & -2987 \\
 479 & 479^1 & \text{--} & \text{Y} & \text{Y} & \text{N} & \text{--} & -2 & 1 & 0 & 1.0000000 & \text{--} & -308 & 2681 & -2989 \\
 480 & 2^5 3^1 5^1 & \text{--} & \text{N} & \text{N} & \text{Y} & \text{--} & -96 & 1 & 80 & 1.6666667 & \text{--} & -404 & 2681 & -3085 \\
 481 & 13^1 37^1 & \text{--} & \text{Y} & \text{N} & \text{N} & \text{--} & 5 & 1 & 0 & 1.0000000 & \text{--} & -399 & 2686 & -3085 \\
 482 & 2^1 241^1 & \text{--} & \text{Y} & \text{N} & \text{N} & \text{--} & 5 & 1 & 0 & 1.0000000 & \text{--} & -394 & 2691 & -3085 \\
 483 & 3^1 7^1 23^1 & \text{--} & \text{Y} & \text{N} & \text{N} & \text{--} & -16 & 1 & 0 & 1.0000000 & \text{--} & -410 & 2691 & -3101 \\
 484 & 2^2 11^2 & \text{--} & \text{N} & \text{N} & \text{Y} & \text{--} & 14 & 1 & 9 & 1.3571429 & \text{--} & -396 & 2705 & -3101 \\
 485 & 5^1 97^1 & \text{--} & \text{Y} & \text{N} & \text{N} & \text{--} & 5 & 1 & 0 & 1.0000000 & \text{--} & -391 & 2710 & -3101 \\
 486 & 2^1 3^5 & \text{--} & \text{N} & \text{N} & \text{Y} & \text{--} & 13 & 1 & 8 & 2.0769231 & \text{--} & -378 & 2723 & -3101 \\
 487 & 487^1 & \text{--} & \text{Y} & \text{Y} & \text{N} & \text{--} & -2 & 1 & 0 & 1.0000000 & \text{--} & -380 & 2723 & -3103 \\
 488 & 2^3 61^1 & \text{--} & \text{N} & \text{N} & \text{Y} & \text{--} & 9 & 1 & 4 & 1.5555556 & \text{--} & -371 & 2732 & -3103 \\
 489 & 3^1 163^1 & \text{--} & \text{Y} & \text{N} & \text{N} & \text{--} & 5 & 1 & 0 & 1.0000000 & \text{--} & -366 & 2737 & -3103 \\
 490 & 2^1 5^1 7^2 & \text{--} & \text{N} & \text{N} & \text{Y} & \text{--} & 30 & 1 & 14 & 1.1666667 & \text{--} & -336 & 2767 & -3103 \\
 491 & 491^1 & \text{--} & \text{Y} & \text{Y} & \text{N} & \text{--} & -2 & 1 & 0 & 1.0000000 & \text{--} & -338 & 2767 & -3105 \\
 492 & 2^2 3^1 41^1 & \text{--} & \text{N} & \text{N} & \text{Y} & \text{--} & 30 & 1 & 14 & 1.1666667 & \text{--} & -308 & 2797 & -3105 \\
 493 & 17^1 29^1 & \text{--} & \text{Y} & \text{N} & \text{N} & \text{--} & 5 & 1 & 0 & 1.0000000 & \text{--} & -303 & 2802 & -3105 \\
 494 & 2^1 13^1 19^1 & \text{--} & \text{Y} & \text{N} & \text{N} & \text{--} & -16 & 1 & 0 & 1.0000000 & \text{--} & -319 & 2802 & -3121 \\
 495 & 3^2 5^1 11^1 & \text{--} & \text{N} & \text{N} & \text{Y} & \text{--} & 30 & 1 & 14 & 1.1666667 & \text{--} & -289 & 2832 & -3121 \\
 496 & 2^4 31^1 & \text{--} & \text{N} & \text{N} & \text{Y} & \text{--} & -11 & 1 & 6 & 1.8181818 & \text{--} & -300 & 2832 & -3132 \\
 497 & 7^1 71^1 & \text{--} & \text{Y} & \text{N} & \text{N} & \text{--} & 5 & 1 & 0 & 1.0000000 & \text{--} & -295 & 2837 & -3132 \\
 498 & 2^1 3^1 83^1 & \text{--} & \text{Y} & \text{N} & \text{N} & \text{--} & -16 & 1 & 0 & 1.0000000 & \text{--} & -311 & 2837 & -3148 \\
 499 & 499^1 & \text{--} & \text{Y} & \text{Y} & \text{N} & \text{--} & -2 & 1 & 0 & 1.0000000 & \text{--} & -313 & 2837 & -3150 \\
 500 & 2^2 5^3 & \text{--} & \text{N} & \text{N} & \text{Y} & \text{--} & -23 & 1 & 18 & 1.4782609 & \text{--} & -336 & 2837 & -3173 \\  
\end{array}
}
\end{equation*}

\end{table} 

%\NBRef{A03-2020-04026}
%\NBRef{A04-2020-04026}

\newpage
\setcounter{section}{0}
\renewcommand{\thesection}{Appendix \Alph{section}}
\renewcommand{\thesubsection}{\Alph{section}.\arabic{subsection}}

\end{document}
