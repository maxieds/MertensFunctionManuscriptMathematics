\documentclass[11pt,reqno,a4letter]{article} 

\usepackage{amsfonts,amscd,amsmath}
\usepackage[hidelinks]{hyperref} 
\usepackage{url}
\usepackage[usenames,dvipsnames]{xcolor}
\hypersetup{
    colorlinks,
    linkcolor={green!63!darkgray},
    citecolor={blue!70!white},
    urlcolor={blue!80!white}
}

\usepackage[normalem]{ulem}
\usepackage{graphicx} 
\usepackage{datetime} 
\usepackage{cancel}
\usepackage{subcaption}
\captionsetup{format=hang,labelfont={bf},textfont={small,it}} 
\numberwithin{figure}{section}
\numberwithin{table}{section}

\usepackage{stmaryrd,tikzsymbols,mathabx} 
\usepackage{framed} 
\usepackage{ulem}
\usepackage[T1]{fontenc}
\usepackage{pbsi}


\usepackage{enumitem}
\setlist[itemize]{leftmargin=0.65in}

\usepackage{rotating,adjustbox}

\usepackage{diagbox}
\newcommand{\trianglenk}[2]{$\diagbox{#1}{#2}$}
\newcommand{\trianglenkII}[2]{\diagbox{#1}{#2}}

\let\citep\cite

\newcommand{\undersetbrace}[2]{\underset{\displaystyle{#1}}{\underbrace{#2}}}

\newcommand{\gkpSI}[2]{\ensuremath{\genfrac{\lbrack}{\rbrack}{0pt}{}{#1}{#2}}} 
\newcommand{\gkpSII}[2]{\ensuremath{\genfrac{\lbrace}{\rbrace}{0pt}{}{#1}{#2}}}
\newcommand{\cf}{\textit{cf.\ }} 
\newcommand{\Iverson}[1]{\ensuremath{\left[#1\right]_{\delta}}} 
\newcommand{\floor}[1]{\left\lfloor #1 \right\rfloor} 
\newcommand{\ceiling}[1]{\left\lceil #1 \right\rceil} 
\newcommand{\e}[1]{e\left(#1\right)} 
\newcommand{\seqnum}[1]{\href{http://oeis.org/#1}{\color{ProcessBlue}{\underline{#1}}}}

\usepackage{upgreek,dsfont,amssymb}
\renewcommand{\chi}{\upchi}
\newcommand{\ChiFunc}[1]{\ensuremath{\chi_{\{#1\}}}}
\newcommand{\OneFunc}[1]{\ensuremath{\mathds{1}_{#1}}}

\usepackage{ifthen}
\newcommand{\Hn}[2]{
     \ifthenelse{\equal{#2}{1}}{H_{#1}}{H_{#1}^{\left(#2\right)}}
}

\newcommand{\Floor}[2]{\ensuremath{\left\lfloor \frac{#1}{#2} \right\rfloor}}
\newcommand{\Ceiling}[2]{\ensuremath{\left\lceil \frac{#1}{#2} \right\rceil}}

\DeclareMathOperator{\DGF}{DGF} 
\DeclareMathOperator{\ds}{ds} 
\DeclareMathOperator{\Id}{Id}
\DeclareMathOperator{\fg}{fg}
\DeclareMathOperator{\Div}{div}
\DeclareMathOperator{\rpp}{rpp}
\DeclareMathOperator{\logll}{\ell\ell}

\title{
       \LARGE{
       Lower bounds on the summatory function of the M\"obius function along infinite subsequences 
       } 
}
\author{{\Large Maxie Dion Schmidt} \\ 
        %{\normalsize \href{mailto:maxieds@gmail.com}{maxieds@gmail.com}} \\[0.1cm] 
        {\normalsize Georgia Institute of Technology} \\[0.025cm] 
        {\normalsize School of Mathematics} 
} 

\date{\small\underline{Last Revised:} \today \ @\ \hhmmsstime{} \ -- \ Compiled with \LaTeX2e} 

%\usepackage[amsmath,thmmarks,framed]{ntheorem}
%\usepackage{framed} 
%\definecolor{thmborder}{rgb}{0.419601,0.4941,0.55294} 
%\renewcommand*\FrameCommand{{\color{thmborder}\vrule width 5pt \hspace{10pt}}}

\usepackage{amsthm} 

\theoremstyle{plain} 
\newtheorem{theorem}{Theorem}
\newtheorem{conjecture}[theorem]{Conjecture}
\newtheorem{claim}[theorem]{Claim}
\newtheorem{prop}[theorem]{Proposition}
\newtheorem{lemma}[theorem]{Lemma}
\newtheorem{cor}[theorem]{Corollary}
\numberwithin{theorem}{section}

\theoremstyle{definition} 
\newtheorem{example}[theorem]{Example}
\newtheorem{remark}[theorem]{Remark}
\newtheorem{definition}[theorem]{Definition}
\newtheorem{notation}[theorem]{Notation}
\newtheorem{question}[theorem]{Question}
\newtheorem{discussion}[theorem]{Discussion}
\newtheorem{facts}[theorem]{Facts}
\newtheorem{summary}[theorem]{Summary}
\newtheorem{heuristic}[theorem]{Heuristic}

\renewcommand{\arraystretch}{1.25} 

\setlength{\textheight}{9in}
\setlength{\topmargin}{-.18in}
\setlength{\textwidth}{7.65in} 
\setlength{\evensidemargin}{-0.25in} 
\setlength{\oddsidemargin}{-0.25in} 
\setlength{\headsep}{8pt} 
%\setlength{\footskip}{10pt} 

\usepackage{geometry}
%\newgeometry{top=0.65in, bottom=18mm, left=15mm, right=15mm, outer=2in, heightrounded, marginparwidth=1.5in, marginparsep=0.15in}
\newgeometry{top=0.65in, bottom=16mm, left=15mm, right=15mm, heightrounded, marginparwidth=0in, marginparsep=0.15in}

\usepackage{fancyhdr}
\pagestyle{empty}
\pagestyle{fancy}
\fancyhead[RO,RE]{Maxie Dion Schmidt -- \today} 
\fancyhead[LO,LE]{}
\fancyheadoffset{0.005\textwidth} 

\setlength{\parindent}{0in}
\setlength{\parskip}{2cm} 

\renewcommand{\thefootnote}{\textbf{\ \Alph{footnote}}}
\makeatletter
\@addtoreset{footnote}{section}
\makeatother

%\usepackage{marginnote,todonotes}
%\colorlet{NBRefColor}{RoyalBlue!73} 
%\newcommand{\NBRef}[1]{
%     \todo[linecolor=green!85!white,backgroundcolor=orange!50!white,bordercolor=blue!30!black,textcolor=cyan!15!black,shadow,size=\small,fancyline]{
%     \color{NBRefColor}{\textbf{#1}
%     }
%     }
%}
\newcommand{\NBRef}[1]{}  

\newcommand{\SuccSim}[0]{\overset{_{\scriptsize{\blacktriangle}}}{\succsim}} 
\newcommand{\PrecSim}[0]{\overset{_{\scriptsize{\blacktriangle}}}{\precsim}} 
\renewcommand{\SuccSim}[0]{\ensuremath{\gg}} 
\renewcommand{\PrecSim}[0]{\ensuremath{\ll}} 

\renewcommand{\Re}{\operatorname{Re}}
\renewcommand{\Im}{\operatorname{Im}}

\input{glossaries-bibtex/PreambleGlossaries-Mertens}

\usepackage{tikz}
\usetikzlibrary{shapes,arrows}

\usepackage{enumitem} 

\allowdisplaybreaks 

\begin{document} 

\maketitle

\begin{abstract} 
The Mertens function, $M(x) = \sum_{n \leq x} \mu(n)$, is classically 
defined as the summatory function of the M\"obius function $\mu(n)$. 
The Mertens conjecture stating that $|M(x)| < C \cdot \sqrt{x}$ with come absolute $C > 0$ for all 
$x \geq 1$ has a well-known disproof due to Odlyzko and t\'{e} Riele given in the early 1980's by computation of 
non-trivial zeta function zeros in conjunction with integral formulas expressing $M(x)$. 
It is conjectured that $M(x) / \sqrt{x}$ changes sign infinitely often and grows 
unbounded in the direction of both $\pm \infty$ along subsequences of integers $x \geq 1$. 
We prove a weaker property related to the unboundedness of $|M(x)| / \sqrt{x}$ by showing that 
$$\limsup_{x \rightarrow \infty} \frac{|M(x)| (\log\log x)^{\frac{3}{4}} (\log\log\log x)^{2}}{ 
  \sqrt{x} \cdot (\log x)^{\frac{1}{2}}} > 0.$$ 
There is a distinct stylistic 
flavor and new element of combinatorial analysis to our proof 
peppered in with the standard methods from analytic, additive and elementary number theory. 
This stylistic tendency distinguishes 
our methods from other proofs of established upper, rather than lower, bounds on $M(x)$. 

\bigskip 
\noindent
\textbf{Keywords and Phrases:} {\it M\"obius function; Mertens function; summatory function; 
                                    Dirichlet inverse; Liouville lambda function; prime omega function; 
                                    prime counting functions; Dirichlet generating function; 
                                    asymptotic lower bounds; Mertens conjecture. } \\ 
% 11-XX			Number theory
%    11A25  	Arithmetic functions; related numbers; inversion formulas
%    11Y70  	Values of arithmetic functions; tables
%    11-04  	Software, source code, etc. for problems pertaining to number theory
% 11Nxx		Multiplicative number theory
%    11N05  	Distribution of primes
%    11N37  	Asymptotic results on arithmetic functions
%    11N56  	Rate of growth of arithmetic functions
%    11N60  	Distribution functions associated with additive and positive multiplicative functions
%    11N64  	Other results on the distribution of values or the characterization of arithmetic functions
\textbf{Math Subject Classifications (MSC 2010):} {\it 11N37; 11A25; 11N60; and 11N64. } 
\end{abstract} 

%\bigskip\hrule\bigskip

\newpage
%\section{Reference on abbreviations, special notation and other conventions} 
\label{Appendix_Glossary_NotationConvs}
     \vskip 0in
     \printglossary[type={symbols},
                    title={Glossary of special notation and conventions},
                    style={glossstyleSymbol},
                    nogroupskip=true]


%\newpage
%\setcounter{tocdepth}{2}
%\renewcommand{\contentsname}{Listing of major sections and topics} 
%\tableofcontents 

\newpage
\section{Introduction} 
\label{subSection_MertensMxClassical_Intro} 

\subsection{Definitions} 

Suppose that $n \geq 2$ is a natural number with factorization into 
distinct primes given by 
$n = p_1^{\alpha_1} p_2^{\alpha_2} \cdots p_r^{\alpha_r}$ so that $r = \omega(n)$. 
We define the \emph{M\"oebius function} to be the signed indicator function 
of the squarefree integers as follows: 
\[
\mu(n) = \begin{cases} 
     1, & \text{if $n = 1$; } \\ 
     (-1)^k, & \text{if $\alpha_i = 1$, $\forall 1 \leq i \leq k$; } \\ 
     0, & \text{otherwise.} 
     \end{cases} 
\]
There are many other variants and special properties of the M\"oebius function 
and its generalizations \cite[\cf \S 2]{HANDBOOKNT-2004}. 
A crucial role of the classical $\mu(n)$ forms an inversion relation 
for arithmetic functions convolved with one by \emph{M\"obius inversion}: 
\[
g(n) = (f \ast 1)(n) \iff f(n) = (g \ast \mu)(n), \forall n \geq 1. 
\]
The \emph{Mertens function}, or summatory function of $\mu(n)$, is defined as 
\begin{align*} 
M(x) & = \sum_{n \leq x} \mu(n), x \geq 1. 
\end{align*} 
The sequence of the oscillatory values of this summatory function begins as 
\cite[\seqnum{A002321}]{OEIS} 
\[
\{M(x)\}_{x \geq 1} = \{1, 0, -1, -1, -2, -1, -2, -2, -2, -1, -2, -2, -3, -2, 
     -1, -1, -2, -2, -3, -3, -2, -1, -2, -2, \ldots\}
\]
Clearly, a positive integer $n \geq 1$ is \emph{squarefree}, or contains no (prime power) divisors which are 
squares, if and only if $\mu^2(n) = 0$. 
A related summatory function which counts the 
number of \emph{squarefree} integers $n \leq x$ then satisfies 
\cite[\S 18.6]{HARDYWRIGHT} \cite[\seqnum{A013928}]{OEIS} 
\[ 
Q(x) = \sum_{n \leq x} \mu^2(n) \sim \frac{6x}{\pi^2} + O\left(\sqrt{x}\right). 
\] 
It is known that the asymptotic density of the positively versus negatively 
weighted sets of squarefree numbers are in fact equal as $x \rightarrow \infty$: 
\[
\mu_{+}(x) = \frac{\#\{1 \leq n \leq x: \mu(n) = +1\}}{Q(x)} \overset{\mathbb{E}}{\sim} 
     \mu_{-}(x) = \frac{\#\{1 \leq n \leq x: \mu(n) = -1\}}{Q(x)} 
     \xrightarrow{x \rightarrow \infty} \frac{3}{\pi^2}. 
\]
The actual local oscillations between the approximate densities of the sets 
$\mu_{\pm}(x)$ lend an unpredictable nature to the function and characterize the 
oscillatory sawtooth shaped plot of $M(x)$ over the positive integers. 

\subsection{Properties} 

One conventional approach to evaluating the behavior of $M(x)$ for large 
$x \rightarrow \infty$ results from a formulation of this summatory 
function as a predictable exact sum involving $x$ and the non-trivial 
zeros of the Riemann zeta function for all real $x > 0$. 
This formula is expressed given the inverse Mellin transformation 
over the reciprocal zeta function. In particular, 
we notice that since 
\[
\frac{1}{\zeta(s)} = \prod_{p} \left(1 - \frac{1}{p^s}\right) = 
     \int_1^{\infty} \frac{s \cdot M(x)}{x^{s+1}} dx, 
\]
we obtain that 
\[
M(x) = \lim_{T \rightarrow \infty}\ \frac{1}{2\pi\imath} \int_{T-\imath\infty}^{T+\imath\infty} 
     \frac{x^s}{s \cdot \zeta(s)} ds. 
\] 
This representation, along with the standard Euler product 
representation for the reciprocal zeta function cited in the first equation above, leads us to the 
exact expression for $M(x)$ for any real $x > 0$ 
given by the next theorem due to Titchmarsh. 
\nocite{TITCHMARSH} 

\begin{theorem}[Analytic Formula for $M(x)$] 
\label{theorem_MxMellinTransformInvFormula} 
Assuming the Riemann Hypothesis (RH), there exists an infinite sequence 
$\{T_k\}_{k \geq 1}$ satisfying $k \leq T_k \leq k+1$ for each $k$ 
such that for any real $x > 0$ 
\[
M(x) = \lim_{k \rightarrow \infty} 
     \sum_{\substack{\rho: \zeta(\rho) = 0 \\ |\Im(\rho)| < T_k}} 
     \frac{x^{\rho}}{\rho \cdot \zeta^{\prime}(\rho)} - 2 + 
     \sum_{n \geq 1} \frac{(-1)^{n-1}}{n \cdot (2n)! \zeta(2n+1)} 
     \left(\frac{2\pi}{x}\right)^{2n} + 
     \frac{\mu(x)}{2} \Iverson{x \in \mathbb{Z}^{+}}. 
\] 
\end{theorem} 

A historical unconditional bound on the Mertens function due to Walfisz (1963) 
states that there is an absolute constant $C > 0$ such that 
$$M(x) \ll x \cdot \exp\left(-C \cdot \log^{3/5}(x) 
  (\log\log x)^{-3/5}\right).$$ 
Under the assumption of the RH, Soundararajan recently proved new updated estimates 
bounding $M(x)$ for large $x$ in the following forms \cite{SOUND-MERTENS-ANNALS}: 
\begin{align*} 
M(x) & \ll \sqrt{x} \cdot \exp\left(\log^{1/2}(x) (\log\log x)^{14}\right), \\ 
M(x) & = O\left(\sqrt{x} \cdot \exp\left( 
     \log^{1/2}(x) (\log\log x)^{5/2+\epsilon}\right)\right),\ 
     \forall \epsilon > 0. 
\end{align*} 

\subsection{Conjectures on boundedness and limiting behavior} 

The RH is equivalent to showing that 
$M(x) = O\left(x^{\frac{1}{2}+\varepsilon}\right)$ for any 
$0 < \varepsilon < \frac{1}{2}$. 
There is a rich history to the original statement of the \emph{Mertens conjecture} which 
posits that 
\[ 
|M(x)| < C \cdot \sqrt{x},\ \text{ for some absolute constant $C > 0$. }
\] 
The conjecture was first verified by Mertens for $C = 1$ and all $x < 10000$. 
Since its beginnings in 1897, the Mertens conjecture has been disproven by computation 
of non-trivial simple zeta function zeros with comparitively small imaginary parts in a famous paper by 
Odlyzko and t\'{e} Riele from the early 1980's \cite{ODLYZ-TRIELE}. 
Since the truth of the conjecture would have implied the RH, more recent attempts 
at bounding $M(x)$ consider determining the rates at which the function 
$M(x) / \sqrt{x}$ grows with or without bound towards both $\pm \infty$ along infinite 
subsequences. 

One of the most famous still unanswered questions about the Mertens 
function concerns whether $|M(x)| / \sqrt{x}$ is in actuality unbounded on the 
natural numbers. A precise statement of this 
problem is to produce an affirmative answer whether 
$\limsup_{x \rightarrow \infty} M(x) / \sqrt{x} = +\infty$ and 
$\liminf_{x \rightarrow \infty} M(x) / \sqrt{x} = -\infty$, or 
equivalently whether there are an infinite subsequences of natural numbers 
$\{x_1, x_2, x_3, \ldots\}$ such that the magnitude of 
$M(x_i) x_i^{-1/2}$ grows without bound towards either $\pm \infty$ 
along the subsequence. 
We cite that prior to this point it is only known by computation 
that \cite[\cf \S 4.1]{PRIMEREC} 
\cite[\cf \seqnum{A051400}; \seqnum{A051401}]{OEIS} 
\[
\limsup_{x\rightarrow\infty} \frac{M(x)}{\sqrt{x}} > 1.060\ \qquad (\text{now } \geq 1.826054), 
\] 
and 
\[ 
\liminf_{x\rightarrow\infty} \frac{M(x)}{\sqrt{x}} < -1.009\ \qquad (\text{now } \leq -1.837625). 
\] 
Based on work by Odlyzyko and t\'{e} Riele, it seems probable that 
each of these limits should evaluate to $\pm \infty$, respectively 
\cite{ODLYZ-TRIELE,MREVISITED,ORDER-MERTENSFN,HURST-2017}. 
Extensive computational evidence has produced 
a conjecture due to Gonek (among attempts on exact bounds by others) that in fact the limiting behavior of 
$M(x)$ satisfies \cite{NG-MERTENS}
$$\limsup_{x \rightarrow \infty} \frac{|M(x)|}{\sqrt{x} \cdot (\log\log x)^{5/4}} = O(1).$$ 
%While it seems to be widely believed that $|M(x)| / \sqrt{x}$ tends to $+\infty$ at some 
%logarithmically scaled rate along subsequences, the 
%infinitely tending factors such as the $(\log\log x)^{\frac{5}{4}}$ in Gonek's conjecture 
%do not appear to readily fall out of work on unconditional bounds for $M(x)$ by existing methods. 

\newpage
\section{An overview of the core logical steps and components to the proof} 

We offer an initial step-by-step summary overview of the core components 
to our proof outlined in the next. 
As our proof methodology is new and relies on non-standard elements compared to more 
traditional methods of bounding $M(x)$, we hope that this sketch of the logical components 
to this argument makes the article easier to parse. 

\subsection{Step-by-step overview} 

\begin{itemize} 

\item[\textbf{(1)}] We prove a matrix inversion formula relating the summatory 
           functions of an arithmetic function $f$ and its Dirichlet inverse $f^{-1}$ (for $f(1) \neq 0$). 
           See 
           Theorem \ref{theorem_SummatoryFuncsOfDirCvls} in 
           Section \ref{Section_PrelimProofs_Config}.  
\item[\textbf{(2)}] This crucial step provides us with an exact formula for $M(x)$ in terms of $\pi(x)$, the seemingly 
           unconnected prime counting function, and the 
           Dirichlet inverse of the shifted additive function $g(n) := \omega(n) + 1$. This 
           formula is stated in \eqref{eqn_Mx_gInvnPixk_formula}. 
           
           The strong additivity of $\omega(n)$ imparts the characteristic signedness of 
           $\operatorname{sgn}(g^{-1}(n)) = \lambda(n)$ for all $n \geq 1$, which is weighted 
           according to the parity of $\Omega(n)$. 
           The link relating \eqref{eqn_Mx_gInvnPixk_formula} to canonical additive functions and their 
           distributions then lends a recent distinguishing element to the 
           success of the methods in our proof. 
\item[\textbf{(3)}] We tighten an updated result from \cite[\S 7]{MV} providing uniform asymptotic formulas for the  
           summatory functions, $\widehat{\pi}_k(x)$, that indicate the parity of 
           $\Omega(n)$ (sign of $\lambda(n)$) 
           for $n \leq x$ and $1 \leq k \leq \log\log x$. These formulas are proved using expansions of 
           more combinatorially motivated Dirichlet series 
           (see Theorem \ref{theorem_GFs_SymmFuncs_SumsOfRecipOfPowsOfPrimes}). 
           We use this result to sum $\sum_{n \leq x} \lambda(n) f(n)$ for particular non-negative arithmetic 
           functions $f$ when $x$ is large. 
\item[\textbf{(4)}] We then turn to bounding the 
           asymptotics of the quasi-periodic functions, $g^{-1}(n)$, by estimating this inverse function's 
           limiting order for large $n \leq x$ as $x \rightarrow \infty$ in 
           Section \ref{Section_InvFunc_PreciseExpsAndAsymptotics}. 
           We eventually use these estimates to prove a substantially unique new lower bound formula 
           for the summatory function $G^{-1}(x) := \sum_{n \leq x} g^{-1}(n)$ along certain asymptotically large 
           infinite subsequences (see Theorem \ref{theorem_gInv_GeneralAsymptoticsForms}). 
\item[\textbf{(5)}] We spend some interim time in Section \ref{Section_ProofOfValidityOfAverageOrderLowerBounds} 
           carefully working out a rigorous justification for why the limiting lower bounds we obtain from average 
           order case analysis of our arithmetic function approximations 
           to $g^{-1}(n)$ are sufficient to prove the corollary on the unboundedness of $M(x)$ below. 
\item[\textbf{(6)}] When we return to step \textbf{(2)} 
           with our new lower bounds at hand, we have a new unconditional proof of the 
           unboundedness of $\frac{|M(x)| \log x}{\sqrt{x}}$ 
           along a very large increasing infinite subsequence 
           of positive natural numbers. What we recover is a quick, and rigorous, proof of 
           Corollary \ref{cor_ThePipeDreamResult_v1} given in 
           Section \ref{subSection_TheCoreResultProof}. 
           
\end{itemize} 

\subsection{Schematic flowchart of the proof logic} 

The next flowchart diagramed below shows how the seemingly disparate components of the proof are organized. 
%It also indicates how the separate ``lands'' of material and corresponding sets of requisite results 
%forming the connected components to steps $\mathcal{A}$, $\mathcal{B}$ and $\mathcal{C}$ (as viewed below) 
%combine to form the next core stages of the proof. 

\tikzstyle{CoreComponent} = [diamond, draw, fill=blue!35, text width=4.5em, text badly centered, 
                             node distance=3cm, inner sep=0.1cm]
\tikzstyle{SubComponent} = [rectangle, draw, fill=blue!19, text width=4.5em, text centered, 
                            rounded corners, minimum height=4em, node distance=3cm]
\tikzstyle{MainResultComponent} = [ellipse, draw, fill=purple!45!pink, text width=4.5em, text centered, 
                            rounded corners, minimum height=4em, node distance=3cm]
\tikzstyle{MainResult} = [cloud, draw, fill=green, text width=4.5em, text centered, 
                            rounded corners, minimum height=4em, node distance=3cm]
\tikzstyle{ComponentConnectionLine} = [draw, -latex]

\begin{center}
\resizebox{0.85\textwidth}{!}{ 
\fbox{
\begin{tikzpicture}[node distance = 2cm, auto, font=\Large\sffamily]
%% : == Nodes: 
\node[CoreComponent] (A)  {Step $\mathcal{A}$}; 
\node[SubComponent, left of=A]  (A2) {A.2}; 
\node[right of=A] (CenterDiagram) {            };
\node[CoreComponent, right of=CenterDiagram] (B)  {Step $\mathcal{B}$}; 
\node[SubComponent, right of=B]  (B2) {B.2}; 
%\node[SubComponent, below of=B2]  (B3) {B.3}; 
\node[MainResultComponent, below of=CenterDiagram] (C)  {Step $\mathcal{D}$}; 
\node[MainResult, below of=C] (D)  {\underline{Step $\mathcal{E}$!}}; 
\node[CoreComponent, left of=C, left of=D] (AvgOrderProofs) {Step $\mathcal{C}$};
%% : == Arrows:
\path[ComponentConnectionLine, dashed, style={<->}] (A) -- (A2);
%\path[ComponentConnectionLine, dashed, style={<->}] (A2) -- (C);
\path[ComponentConnectionLine] (AvgOrderProofs) -- (D);
\path[ComponentConnectionLine, dashed, style={<->}] (B) -- (B2);
%\path[ComponentConnectionLine, dashed, style={<->}] (B2) -- (C);
\path[ComponentConnectionLine] (A) -- (C);
\path[ComponentConnectionLine] (B) -- (C);
\path[ComponentConnectionLine] (C) -- (D);
\end{tikzpicture} 
}
}
\end{center}

\subsubsection*{\underline{Legend to the diagram stages:}} 
\begin{itemize}[noitemsep,topsep=0pt,leftmargin=0.95in]

\item[$\blacktriangleright$ \textbf{Step A:}] \textit{Citations and re-statements of existing theorems proved elsewhere. }
     %E.g., statements of non-trivial theorems and key results we need that are proved in the references. 
     \begin{itemize}[noitemsep,topsep=0pt,leftmargin=0.35in] 
     \item[\textbf{A.A:}] Key results and constructions: 
          \begin{itemize}[noitemsep,topsep=0pt,leftmargin=0.25in]
          \item[--] \small{Theorem \ref{theorem_HatPi_ExtInTermsOfGz}} 
          %\item[--] \small{Theorem \ref{theorem_MV_Thm7.20-init_stmt}} 
          \item[--] \small{Corollary \ref{theorem_MV_Thm7.20}} 
          \item[--] \small{The results, lemmas, and facts cited in Section \ref{subSection_OtherFactsAndResults}}
          \end{itemize} 
     \item[\textbf{A.2:}] Lower bounds on the Abel summation based formula for $G^{-1}(x)$: 
          \begin{itemize}[noitemsep,topsep=0pt,leftmargin=0.25in]
          \item[--] \small{Theorem \ref{theorem_GFs_SymmFuncs_SumsOfRecipOfPowsOfPrimes} 
                    (on page \pageref{proofOf_theorem_GFs_SymmFuncs_SumsOfRecipOfPowsOfPrimes})} 
          \item[--] \small{Proposition \ref{cor_PartialSumsOfReciprocalsOfPrimePowers}} 
          \item[--] \small{Theorem \ref{theorem_gInv_GeneralAsymptoticsForms}} 
          %\item[--] \small{Lemma \ref{lemma_CLT_and_AbelSummation}} 
          %\item[--] \small{Lemma \ref{lemma_lowerBoundsOnLambdaFuncParitySummFuncs}} 
          \end{itemize} 
     \end{itemize} 
\item[$\blacktriangleright$ \textbf{Step B:}] \textit{Constructions of an exact formula for $M(x)$. } 
     %The exact formula we prove 
     %uses special arithmetic functions with particularly ``nice'' properties and bounds. 
     \begin{itemize}[noitemsep,topsep=0pt,leftmargin=0.35in] 
     \item[\textbf{B.B:}] Key results and constructions: 
          \begin{itemize}[noitemsep,topsep=0pt,leftmargin=0.25in]
          \item[--] \small{Corollary \ref{cor_Mx_gInvnPixk_formula}} (follows from 
                    Theorem \ref{theorem_SummatoryFuncsOfDirCvls} 
                    proved on page \pageref{proofOf_theorem_SummatoryFuncsOfDirCvls}) 
          %\item[--] \small{Conjecture \ref{lemma_gInv_MxExample} (to a lesser expository only extent)} 
          \item[--] \small{Proposition \ref{prop_SignageDirInvsOfPosBddArithmeticFuncs_v1}} 
          \end{itemize} 
     \item[\textbf{B.2:}] Asymptotics for the component functions $g^{-1}(n)$ and $G^{-1}(x)$: 
          \begin{itemize}[noitemsep,topsep=0pt,leftmargin=0.25in]
          \item[--] \small{Theorem \ref{theorem_Ckn_GeneralAsymptoticsForms} 
                    (on page \pageref{proofOf_theorem_Ckn_GeneralAsymptoticsForms})} 
          \item[--] \small{Lemma \ref{lemma_AnExactFormulaFor_gInvByMobiusInv_v1}} 
          \end{itemize} 
     %\item[\textbf{B.3}] Simplifying formulas for $g^{-1}(n)$ and $G^{-1}(x)$: 
     %     \begin{itemize}[noitemsep,topsep=0pt]
     %     \item[--] \small{Corollary \ref{cor_ASemiForm_ForGInvx_v1}} 
     %     \end{itemize} 
     \end{itemize} 
\item[$\blacktriangleright$ \textbf{Step C:}] \textit{A justification for 
     why lower bounds obtained roughly ``on average'' suffice. }
     \begin{itemize}[noitemsep,topsep=0pt,leftmargin=0.35in]
     \item[--] \small{The results proved in Section \ref{Section_ProofOfValidityOfAverageOrderLowerBounds}} 
     \end{itemize} 
\item[$\blacktriangleright$ \textbf{Step D:}] \textit{Interpreting the exact formula for $M(x)$. } 
     %Key interpretations used in 
     %formulating the lower bounds based on the re-phrased integral formula. 
     \begin{itemize}[noitemsep,topsep=0pt,leftmargin=0.35in]
     \item[--] \small{Proposition \ref{prop_Mx_SBP_IntegralFormula}} 
     \item[--] \small{Theorem \ref{theorem_gInv_GeneralAsymptoticsForms}} 
     \end{itemize} 
\item[$\blacktriangleright$ \textbf{Step E:}] \textit{The Holy Grail. } 
     Proving that 
     $\frac{|M(x)|}{\sqrt{x}}$ grows without bound in the limit supremum sense. 
     \begin{itemize}[noitemsep,topsep=0pt,leftmargin=0.35in]
     \item[--] \small{Corollary \ref{cor_ThePipeDreamResult_v1} (on page \pageref{proofOf_cor_ThePipeDreamResult_v1})} 
     \end{itemize} 

\end{itemize} 

\newpage 
\section{A concrete new approach for bounding $M(x)$ from below} 

\subsection{Summatory functions of Dirichlet convolutions of arithmetic functions} 

\begin{theorem}[Summatory functions of Dirichlet convolutions] 
\label{theorem_SummatoryFuncsOfDirCvls} 
Let $f,h: \mathbb{Z}^{+} \rightarrow \mathbb{C}$ be any arithmetic functions such that $f(1) \neq 0$. 
Suppose that $F(x) := \sum_{n \leq x} f(n)$ and $H(x) := \sum_{n \leq x} h(n)$ denote the summatory 
functions of $f,h$, respectively, and that $F^{-1}(x)$ denotes the summatory function of the 
Dirichlet inverse $f^{-1}$ of $f$. Then, letting the counting function $\pi_{f \ast h}(x)$ be defined 
as in the first equation below, we have the following equivalent expressions for the 
summatory function of $f \ast h$ for all integers $x \geq 1$: 
\begin{align*} 
\pi_{f \ast h}(x) & = \sum_{n \leq x} \sum_{d|n} f(d) h(n/d) \\ 
     & = \sum_{d \leq x} f(d) H\left(\Floor{x}{d}\right) \\ 
     & = \sum_{k=1}^{x} H(k) \left[F\left(\Floor{x}{k}\right) - 
     F\left(\Floor{x}{k+1}\right)\right]. 
\end{align*} 
Moreover, we can invert the linear system determining the coefficients of $H(k)$ for $1 \leq k \leq x$ 
naturally to express $H(x)$ as a linear combination of the original left-hand-side 
summatory function as follows:
\begin{align*} 
H(x) & = \sum_{j=1}^{x} \pi_{f \ast h}(j) \left[F^{-1}\left(\Floor{x}{j}\right) - 
     F^{-1}\left(\Floor{x}{j+1}\right)\right] \\ 
     & = \sum_{n=1}^{x} f^{-1}(n) \pi_{f \ast h}\left(\Floor{x}{n}\right). 
\end{align*} 
\end{theorem} 

\begin{cor}[Convolutions Arising From M\"obius Inversion] 
\label{cor_CvlGAstMu} 
Suppose that $g$ is an arithmetic function on the positive integers such that 
$g(1) \neq 0$. Define the summatory function of 
the convolution of $g$ with $\mu$ by $\widetilde{G}(x) := \sum_{n \leq x} (g \ast \mu)(n)$. 
Then the Mertens function equals 
\[
M(x) = \sum_{k=1}^{x} \left(\sum_{j=\floor{\frac{x}{k+1}}+1}^{\floor{\frac{x}{k}}} g^{-1}(j)\right) 
     \widetilde{G}(k), \forall x \geq 1. 
\]
\end{cor} 

\begin{cor}[A motivating special case] 
\label{cor_Mx_gInvnPixk_formula} 
We have exactly that for all $x \geq 1$ 
\begin{equation} 
\label{eqn_Mx_gInvnPixk_formula} 
M(x) = \sum_{k=1}^{x} (\omega+1)^{-1}(k) \left[\pi\left(\Floor{x}{k}\right) + 1\right]. 
\end{equation} 
\end{cor} 

\subsection{An exact expression for $M(x)$ in terms of strongly additive functions} 
\label{example_InvertingARecRelForMx_Intro}

From this point on, we fix the notation for the Dirichlet invertible function $g(n) := \omega(n) + 1$ and denote its 
inverse with respect to Dirichlet convolution by $g^{-1}(n) = (\omega+1)^{-1}(n)$. 
We can compute the first few terms for the
Dirichlet inverse of $g(n)$ exactly for the first few sequence values as 
(see Table \ref{table_conjecture_Mertens_ginvSeq_approx_values} of the appendix section) 
\[
\{g^{-1}(n)\}_{n \geq 1} = \{1, -2, -2, 2, -2, 5, -2, -2, 2, 5, -2, -7, -2, 5, 5, 2, -2, -7, -2, 
     -7, 5, 5, -2, 9, \ldots \}. 
\] 
The sign of these terms is given by $\operatorname{sgn}(g^{-1}(n)) = \frac{g^{-1}(n)}{|g^{-1}(n)|} = \lambda(n)$ 
(see Proposition \ref{prop_SignageDirInvsOfPosBddArithmeticFuncs_v1}). 
This useful property is inherited from the distinctly 
additive nature of the component function $\omega(n)$\footnote{ 
     Indeed, for any non-negative additive arithmetic function $a(n)$, 
     $(a+1)^{-1}(n)$ has leading sign given by $\lambda(n)$ for any $n \geq 1$. 
     For multiplicative $f$, we obtain a related condition that 
     $\operatorname{sgn}(f(n)) = (-1)^{\omega(n)}$ for all $n \geq 1$. 
}. 

There does not appear to be an easy, nor subtle 
direct recursion between the distinct values of $g^{-1}(n)$, except through auxiliary function sequences. 
However, the distribution of distinct sets of prime exponents is fairly regular so that 
$\omega(n)$ and $\Omega(n)$ play a crucial role in the repitition of common values of 
$g^{-1}(n)$. 
The following observation is suggestive of the quasi-periodicity of the distribution of 
distinct values of $g^{-1}(n)$ over $n \geq 2$: 

\begin{heuristic}[Symmetry in $g^{-1}(n)$ in the exponents in the prime factorization of $n$] 
Suppose that $n_1, n_2 \geq 2$ are such that their factorizations into distinct primes are 
given by $n_1 = p_1^{\alpha_1} \cdots p_r^{\alpha_r}$ and $n_2 = q_1^{\beta_1} \cdots q_r^{\beta_r}$ 
for some $r \geq 1$. 
If $\{\alpha_1, \ldots, \alpha_r\} \equiv \{\beta_1, \ldots, \beta_r\}$ as multisets of prime exponents, 
then $g^{-1}(n_1) = g^{-1}(n_2)$. For example, $g^{-1}$ has the same values on the squarefree integers 
with exactly two, three, and so on prime factors 
(compare with Table \ref{table_conjecture_Mertens_ginvSeq_approx_values} starting on page 
\pageref{table_conjecture_Mertens_ginvSeq_approx_values}). 
\end{heuristic} 

%The exact formula proved in \eqref{eqn_proof_tag_hInvn_ExactNestedSumFormula_v2} of the next section 
%provides a more combinatorial framework for enumerating groups of common values of 
%$|g^{-1}(n)|$ for all $n \leq x$ as $x \rightarrow \infty$. This construction is useful provided we have 
%sufficiently accurate asymptotics on the distribution of the distinct exponent patterns appearing the 
%the prime factorizations of the $n \leq x$. It is evident from the formula that for $n \geq 2$ 
%$|g^{-1}(n)|$ is polynomial in the exponents, $\{\alpha_1, \ldots, \alpha_{\omega(n)}\}$, 
%corresponding to the prime factorization of 
%$n := p_1^{\alpha_1} \cdots p_{\omega(n)}^{\alpha_{\omega(n)}}$. 

\NBRef{A01-2020-04-26}
\begin{conjecture}
\label{lemma_gInv_MxExample} 
We have the following properties characterizing the 
Dirichlet inverse function $g^{-1}(n)$: 
\begin{itemize} 

\item[\textbf{(A)}] $g^{-1}(1) = 1$; 
\item[\textbf{(B)}] For all $n \geq 1$, $\operatorname{sgn}(g^{-1}(n)) = \lambda(n)$; 
\item[\textbf{(C)}] For all squarefree integers $n \geq 1$, we have that 
     \[
     |g^{-1}(n)| = \sum_{m=0}^{\omega(n)} \binom{\omega(n)}{m} \cdot m!. 
     \]
\end{itemize} 
\end{conjecture} 

We illustrate parts (B)--(C) of the conjecture more clearly using 
Table \ref{table_conjecture_Mertens_ginvSeq_approx_values} given starting on 
page \pageref{table_conjecture_Mertens_ginvSeq_approx_values}. 
The realization that the beautiful and remarkably simple combinatorial form of property (C) 
in Conjecture \ref{lemma_gInv_MxExample} holds for all squarefree $n \geq 1$ 
motivates our pursuit of formulas for the inverse functions $g^{-1}(n)$ 
expressed by sums of auxiliary sequences of arithmetic functions\footnote{ 
     A proof of this property is not difficult to give using 
     Lemma \ref{lemma_AnExactFormulaFor_gInvByMobiusInv_v1} 
     stated on page \pageref{lemma_AnExactFormulaFor_gInvByMobiusInv_v1}. 
} 
(see Section \ref{Section_InvFunc_PreciseExpsAndAsymptotics}). 

%\begin{remark}[Comparison to formative methods for bounding $M(x)$]
%Note that since the DGF of $\omega(n)$ is given by 
%$D_{\omega}(s) = P(s) \zeta(s)$ where $P(s)$ is the \emph{prime zeta function}, we do have a 
%Dirichlet series for the inverse functions to invert coefficient-wise using more classical 
%contour integral methods\footnote{
%E.g., using contour integration or the following integral formula for Dirichlet series 
%inversion \cite[\S 11]{APOSTOLANUMT}: 
%\[
%f(n) = \lim_{T \rightarrow \infty} \frac{1}{2T} \int_{-T}^{T} 
%     \frac{n^{\sigma+\imath t}}{\zeta(\sigma+\imath t)(P(\sigma+\imath t) + 1)}, \sigma > 1. 
%\]
%Fr\"oberg has also previously done some preliminary investigation as to the properties of the 
%inversion to find the coefficients of $(1+P(s))^{-1}$ in \cite{FROBERG-1968}. 
%}. 
%However, the uniqueness to our new methods is that our new approach does not rely on typical constructions for 
%bounding $M(x)$ based on estimates of the non-trivial zeros of the Riemann zeta function that have so far 
%been employed to bound the Mertens function from above. 
%That is, we will instead take a more combinatorial tack to investigating bounds on this inverse function 
%sequence in the coming sections. By Corollary \ref{cor_Mx_gInvnPixk_formula}, 
%once we have established bounds on this $g^{-1}(n)$ and its summatory function, we will be able to 
%formulate new lower bounds (in the limit supremum sense) on $M(x)$. 
%\end{remark} 

For natural numbers $n \geq 1, k \geq 0$, let 
\begin{align*} 
C_k(n) := \begin{cases} 
     \varepsilon(n) = \delta_{n,1}, & \text{ if $k = 0$; } \\ 
     \sum\limits_{d|n} \omega(d) C_{k-1}(n/d), & \text{ if $k \geq 1$. } 
     \end{cases} 
\end{align*} 
For any $n \geq 1$, we can prove that (see Lemma \ref{lemma_AnExactFormulaFor_gInvByMobiusInv_v1})
\begin{equation} 
\label{eqn_AnExactFormulaFor_gInvByMobiusInv_v2} 
g^{-1}(n) = \lambda(n) \times \sum_{d|n} \mu^2\left(\frac{n}{d}\right) C_{\Omega(d)}(d). 
\end{equation} 
In light of the fact that (see Proposition \ref{prop_Mx_SBP_IntegralFormula}) 
\[
M(x) \approx G^{-1}(x) - \sum_{k=1}^{x/2} G^{-1}(k) \cdot \frac{x}{k^2 \log(x/k)}, 
\]
the formula in \eqref{eqn_AnExactFormulaFor_gInvByMobiusInv_v2} 
implies that we can establish new \emph{lower bounds} on $M(x)$ along large infinite subsequences 
by appropriate estimates of the summatory function $G^{-1}(x)$\footnote{ 
     We can also prove that 
     \[
     M(x) = G^{-1}(x) + \sum_{p \leq x} G^{-1}\left(\Floor{x}{p}\right), 
     \] 
     by inversion since 
     \[
     G^{-1}(x) = \sum_{d \leq x} (g^{-1} \ast 1)(d) M\left(\Floor{x}{d}\right), 
     \]
     with $(g^{-1} \ast 1)^{-1} = g \ast \mu = \chi_{\mathbb{P}} + \varepsilon$ 
     defined such that $\chi_{\mathbb{P}}$ is the characteristic function of the primes. 
}. 

\subsection{Uniform asymptotics from enumerative counting DGFs in Mongomery and Vaughan} 

Our inspiration for the new bounds found in the last sections of this article allows us to sum 
non-negative arithmetic functions weighted by the Liouville lambda function, 
$\lambda(n) = (-1)^{\Omega(n)}$. 
We utilize a somewhat more general 
hybrid generating function and enumerative DGF method 
under which we are able to recover ``good enough'' asymptotics about the summatory functions that 
encapsulate the parity of $\Omega(n)$ (or sign of $\lambda(n)$) 
through the summatory tally functions $\widehat{\pi}_k(x)$ 
(see Section \ref{subSection_MVPrereqResultStmts}). 
%The precise statement of the result that we transform to state these new bounds is provided next in 
%Theorem \ref{theorem_HatPi_ExtInTermsOfGz} 
%(see Section \ref{subSection_MVPrereqResultStmts}). 

\begin{theorem}[Montgomery and Vaughan]
\label{theorem_HatPi_ExtInTermsOfGz} 
Recall that we have defined 
$$\widehat{\pi}_k(x) := \#\{n \leq x: \Omega(n)=k\}.$$ 
For $R < 2$ we have that 
\[
\widehat{\pi}_k(x) = \mathcal{G}\left(\frac{k-1}{\log\log x}\right) \frac{x}{\log x} 
     \frac{(\log\log x)^{k-1}}{(k-1)!} \left(1 + O_R\left(\frac{k}{(\log\log x)^2}\right)\right),  
\]
uniformly for $1 \leq k \leq R \log\log x$ where 
\[
\mathcal{G}(z) := \frac{1}{\Gamma(z+1)} \times 
     \prod_p \left(1-\frac{z}{p}\right)^{-1} \left(1-\frac{1}{p}\right)^z, z \geq 0. 
\]
\end{theorem} 

%The next theorem, proved carefully in Section \ref{Section_MVCh7_GzBounds}, 
%is the primary starting point for our new asymptotic lower bounds. 
The proof of the next result is combinatorially motivated in so much as it interprets 
lower bounds on a key infinite product factor of $\mathcal{G}(z)$ defined in 
Theorem \ref{theorem_HatPi_ExtInTermsOfGz} 
as corresponding to an ordinary generating function of certain 
homogeneous symmetric polynomials involving reciprocals of the primes. 

\begin{theorem} 
\label{theorem_GFs_SymmFuncs_SumsOfRecipOfPowsOfPrimes} 
\label{cor_BoundsOnGz_FromMVBook_initial_stmt_v1} 
For all large $x$ we have uniformly for $1 \leq k \leq \log\log x$ that 
\[
\widehat{\pi}_k(x) \SuccSim \frac{4}{3\sqrt{\pi}} \frac{x}{\log x} \left(\frac{\log 2}{\log x}\right)^{ 
     \frac{k-1}{\log\log x}} \frac{(\log\log x)^{k-1}}{(k-1)!} \left( 
     1 + O\left(\frac{k}{(\log\log x)^2}\right) 
     \right).
\]
\end{theorem} 

\subsection{Cracking the classical unboundedness barrier} 

In Section \ref{Section_KeyApplications}, 
we are able to state what forms a culmination of the results 
we carefully build up to in the proofs established in prior sections of the article. 
What we eventually obtain at the conclusion of the section 
is the next important summary corollary that partially 
resolves the classical question of the 
unboundedness of the scaled function Mertens function 
$q(x) := |M(x)| / \sqrt{x}$ in the limit supremum sense. 

\begin{cor}[Unboundedness of the the Mertens function, $q(x)$] 
\label{cor_ThePipeDreamResult_v1} 
We have that 
\[
\limsup_{x \rightarrow \infty} \frac{|M(x)|}{\sqrt{x}} = +\infty. 
\]
\end{cor} 

In establishing the rigorous proof of 
Corollary \ref{cor_ThePipeDreamResult_v1} 
based on our new methods, we not only show unboundedness of 
$q(x)$, but also set a minimal rate (along a large infinite subsequence) 
at which this form of the 
scaled Mertens function grows without bound. 

\newpage 
\section{Preliminary proofs of new results} 
\label{Section_PrelimProofs_Config} 

\subsection{Establishing the summatory function properties and inversion identities} 

We will first prove Theorem \ref{theorem_SummatoryFuncsOfDirCvls} 
using matrix methods and similarity transforms by shift matrices. 
Related results on summations of Dirichlet convolutions appear in 
\cite[\S 2.14; \S 3.10; \S 3.12; \cf \S 4.9, p.\ 95]{APOSTOLANUMT}. 

\begin{proof}[Proof of Theorem \ref{theorem_SummatoryFuncsOfDirCvls}] 
\label{proofOf_theorem_SummatoryFuncsOfDirCvls} 
Let $h,g$ be arithmetic functions such that $g(1) \neq 0$. 
Denote the summatory functions of $h$ and $g$, 
respectively, by $H(x) = \sum_{n \leq x} h(n)$ and $G(x) = \sum_{n \leq x} g(n)$. 
We define $\pi_{g \ast h}(x)$ to be the summatory function of the 
Dirichlet convolution of $g$ with $h$: $g \ast h$. 
Then we can readily see that the following initial formulas hold for all $x \geq 1$: 
\begin{align*} 
\pi_{g \ast h}(x) & := \sum_{n=1}^{x} \sum_{d|n} g(n) h(n/d) = \sum_{d=1}^x g(d) H\left(\floor{\frac{x}{d}}\right) \\ 
     & = \sum_{i=1}^x \left[G\left(\floor{\frac{x}{i}}\right) - G\left(\floor{\frac{x}{i+1}}\right)\right] H(i). 
\end{align*} 
We form the matrix of coefficients associated with this linear system defining $H(n)$ for all $n \leq x$. 
We then invert the system to express an 
exact solution for $H(x)$ at any $x \geq 1$. 
Let the matrix entries be denoted by 
\[
g_{x,j} := G\left(\floor{\frac{x}{j}}\right) - G\left(\floor{\frac{x}{j+1}}\right) \equiv G_{x,j} - G_{x,j+1}, 
\] 
where 
\[
G_{x,j} := G\left(\Floor{x}{j}\right), \forall 1 \leq j \leq x. 
\]
The matrix we must invert in this problem is lower triangular, with ones on its diagonals, and hence is invertible. 
Moreover, if we let $\hat{G} := (G_{x,j})$, then this matrix is 
expressable by an invertible shift operation as 
\[
(g_{x,j}) = \hat{G} (I - U^{T}). 
\]
Here, $U$ is a square matrix with finite dimensions 
whose $(i,j)^{th}$ entries are defined by $(U)_{i,j} = \delta_{i+1,j}$ such that 
\[
\left[(I - U^T)^{-1}\right]_{i,j} = \Iverson{j \leq i}. 
\]
It is a useful fact that if we take successive differences in $x$ of the 
floor of certain fractions, $\Floor{x}{j}$, we get non-zero behavior at the 
divisors of $x$: 
\[
G\left(\floor{\frac{x}{j}}\right) - G\left(\floor{\frac{x-1}{j}}\right) = 
     \begin{cases} 
     g\left(\frac{x}{j}\right), & \text{ if $j | x$; } \\ 
     0, & \text{ otherwise. } 
     \end{cases}
\]
We use this property to shift the matrix $\hat{G}$, and then invert the result to obtain a matrix involving the 
Dirichlet inverse of $g$ of the following form: 
\begin{align*} 
\left[(I-U^{T}) \hat{G}\right]^{-1} & = \left(g\left(\frac{x}{j}\right) \Iverson{j|x}\right)^{-1} = 
     \left(g^{-1}\left(\frac{x}{j}\right) \Iverson{j|x}\right). 
\end{align*} 
Now we can express the inverse of the target matrix, 
$$(g_{x,j}) = (I-U^{T})^{-1} \left(g\left(\frac{x}{j}\right) \Iverson{j|x}\right) (I-U^{T}),$$
using a similarity transformation conjugated by shift operators as follows: 
\begin{align*} 
(g_{x,j})^{-1} & = (I-U^{T})^{-1} \left(g^{-1}\left(\frac{x}{j}\right) \Iverson{j|x}\right) (I-U^{T}) \\ 
     & = \left(\sum_{k=1}^{\floor{\frac{x}{j}}} g^{-1}(k)\right) (I-U^{T}) \\ 
     & = \left(\sum_{k=1}^{\floor{\frac{x}{j}}} g^{-1}(k) - \sum_{k=1}^{\floor{\frac{x}{j+1}}} g^{-1}(k)\right). 
\end{align*} 
Hence, the summatory function $H(x)$ is exactly expressed for any $x \geq 1$ 
by a vector product with the inverse matrix from the previous equation in the form of 
\begin{align*} 
H(x) & = \sum_{k=1}^x g_{x,k}^{-1} \cdot \pi_{g \ast h}(k) \\ 
     & = \sum_{k=1}^x \left(\sum_{j=\floor{\frac{x}{k+1}}+1}^{\floor{\frac{x}{k}}} g^{-1}(j)\right) \cdot \pi_{g \ast h}(k). 
     \qedhere
\end{align*} 
\end{proof} 

\subsection{Proving the characteristic signedness property of $g^{-1}(n)$} 

Let $\chi_{\mathbb{P}}$ denote the characteristic function of the primes, 
$\varepsilon(n) = \delta_{n,1}$ be the multiplicative identity with respect to Dirichlet convolution, 
and denote by $\omega(n)$ the strongly additive function that counts the number of 
distinct prime factors of $n$. Then we can easily prove that 
\begin{equation}
\label{eqn_AntiqueDivisorSumIdent} 
\chi_{\mathbb{P}} + \varepsilon = (\omega + 1) \ast \mu. 
\end{equation} 
When combined with Corollary \ref{cor_CvlGAstMu} 
this convolution identity yields the exact 
formula for $M(x)$ stated in \eqref{eqn_Mx_gInvnPixk_formula} of 
Corollary \ref{cor_Mx_gInvnPixk_formula}. 

\begin{prop}[The key signedness property of $g^{-1}(n)$]
\label{prop_SignageDirInvsOfPosBddArithmeticFuncs_v1} 
Let the operator 
$\operatorname{sgn}(h(n)) = \frac{h(n)}{|h(n)| + \Iverson{h(n) = 0}} \in \{0, \pm 1\}$ denote the sign 
of the arithmetic function $h$ at integers $n \geq 1$. 
For the Dirichlet invertible function, $g(n) := \omega(n) + 1$, 
we have that $\operatorname{sgn}(g^{-1}(n)) = \lambda(n)$ for all $n \geq 1$. 
\NBRef{A02-2020-04-26}
\end{prop} 
\begin{proof} 
Recall that $D_f(s) := \sum_{n \geq 1} f(n) n^{-s}$ denotes the 
\emph{Dirichlet generating function} (DGF) of any 
arithmetic function $f(n)$ which is convergent for all $s \in \mathbb{C}$ satisfying 
$\Re(s) > \sigma_f$ for $\sigma_f$ the abcissa of convergence of the series. 
Recall that $D_1(s) = \zeta(s)$, $D_{\mu}(s) = 1 / \zeta(s)$ and $D_{\omega}(s) = P(s) \zeta(s)$. 
Then by \eqref{eqn_AntiqueDivisorSumIdent} and the known property that the DGF of $f^{-1}(n)$ is 
the reciprocal of the DGF of any invertible arithmetic function $f$, for all $\Re(s) > 1$ we have 
\begin{align} 
\label{eqn_DGF_of_gInvn} 
D_{(\omega+1)^{-1}}(s) = \frac{1}{(P(s)+1) \zeta(s)}. 
\end{align} 
It follows that $(\omega + 1)^{-1}(n) = (h^{-1} \ast \mu)(n)$ when we take 
$h := \chi_{\mathbb{P}} + 1$. 
We first show that $\operatorname{sgn}(h^{-1}) = \lambda$. From this fact, it follows by inspection 
that $\operatorname{sgn}(h^{-1} \ast \mu) = \lambda$. The remainder of the proof fills in the 
precise details needed to make this intuition rigorous. 

By the standard recurrence relation that defines the Dirichlet inverse function of any 
arithmetic function $h$, we have that \cite[\S 2.7]{APOSTOLANUMT} 
\begin{equation} 
\label{eqn_proof_tag_hInvn_ExactRecFormula_v1}
h^{-1}(n) = \begin{cases} 
            1, & n = 1; \\ 
            -\sum\limits_{\substack{d|n \\ d>1}} h(d) h^{-1}(n/d), & n \geq 2. 
            \end{cases} 
\end{equation} 
For $n \geq 2$, the summands in \eqref{eqn_proof_tag_hInvn_ExactRecFormula_v1} 
can be simply indexed over the primes $p|n$ given our definition of $h$ from above. 
This observation yields that we can inductively 
expand these sums into nested divisor sums provided the depth of the sums does not exceed the 
capacity to index summations over the primes dividing $n$. Namely, notice that for $n \geq 2$ 
\begin{align*} 
h^{-1}(n) & = -\sum_{p|n} h^{-1}(n/p), && \text{\ if\ } \Omega(n) \geq 1 \\ 
     & = \sum_{p_1|n} \sum_{p_2|\frac{n}{p_1}} h^{-1}\left(\frac{n}{p_1p_2}\right), && \text{\ if\ } \Omega(n) \geq 2 \\ 
     & = -\sum_{p_1|n} \sum_{p_2|\frac{n}{p_1}} \sum_{p_3|\frac{n}{p_1p_2}} h^{-1}\left(\frac{n}{p_1p_2p_3}\right), 
     && \text{\ if\ } \Omega(n) \geq 3. 
\end{align*} 
Then by induction, again with $h^{-1}(1) = h(1) = 1$, we expand these 
nested divisor sums as above to the maximal possible depth as 
\begin{equation} 
\label{eqn_proof_tag_hInvn_ExactNestedSumFormula_v2} 
\lambda(n) \cdot h^{-1}(n) = \sum_{p_1|n} \sum_{p_2|\frac{n}{p_1}} \times \cdots \times 
     \sum_{p_{\Omega(n)}|\frac{n}{p_1p_2 \cdots p_{\Omega(n)-1}}} 1, n \geq 2. 
\end{equation} 
If for $n \geq 2$ we write the prime factorization of $n$ as 
$n = p_1^{\alpha_1} p_2^{\alpha_2} \cdots p_{\omega(n)}^{\alpha_{\omega(n)}}$ where the exponents $\alpha_i \geq 1$ 
for all $1 \leq i \leq \omega(n)$, we can see that\footnote{
     In fact, we recover that 
     \[
     \lambda(n) h^{-1}(n) = \frac{(\alpha_1+\cdots+\alpha_{\omega(n)})!}{ 
          \alpha_1! \alpha_2! \cdots \alpha_{\omega(n)}!}, 
     \]
     so that since $h^{-1} = g^{-1} \ast 1$ by the DGF above, when $n \geq 1$ is 
     squarefree, we recover property (C) stated in 
     Conjecture \ref{lemma_gInv_MxExample}. 
} 
\begin{align} 
\label{eqn_proof_tag_hInvn_AbsBounds_v3} 
|h^{-1}(n)| & \geq (\omega(n))! && =: h_{\ell}^{-1}(n), n \geq 2, \\ 
\notag 
|h^{-1}(n)| & \leq (\omega(n))!^{\max(\alpha_1, \alpha_2, \ldots, \alpha_{\omega(n)})} && =: h_u^{-1}(n), n \geq 2, 
\end{align} 
where the bounding functions $h_{\ell}^{-1}(n), h_{u}^{-1}(n) > 0$ are positive for all $n \geq 1$. 
What these bounds show is that for all $n \geq 1$ (with $\lambda(1) = 1$) the following property holds: 
\begin{equation} 
\notag 
%\label{eqn_proof_tag_SignedTimesPosConstantFormOf_hInvn_v2}
\operatorname{sgn}(h^{-1}(n)) = \lambda(n). 
\end{equation} 
Since $\lambda$ is completely multiplicative, and since $\mu(n) = \lambda(n)$ whenever $n$ is squarefree, 
we obtain that 
\[
g^{-1}(n) = (h^{-1} \ast \mu)(n) = \lambda(n) \times \sum_{d|n} \mu^2\left(\frac{n}{d}\right) |h^{-1}(n)|, n \geq 1. 
\]
Finally, since $|h^{-1}(n)| > 0$ for all $n \geq 1$ by the bounds we proved in 
\eqref{eqn_proof_tag_hInvn_AbsBounds_v3}, the previous equation implies our result. 
\end{proof} 

\subsection{Statements of other facts and known limiting asymptotics} 
\label{subSection_OtherFactsAndResults} 

\begin{theorem}[Mertens theorem]
\label{theorem_Mertens_theorem} 
For all $x \geq 2$ we have that 
\[
P_1(x) := \sum_{p \leq x} \frac{1}{p} = \log\log x + B + o(1), 
\]
where 
$B \approx 0.2614972128476427837554$ 
is an absolute constant\footnote{ 
     Exactly, we have that the \emph{Mertens constant} is defined by 
     \[
     B = \gamma + \sum_{m \geq 2} \frac{\mu(m)}{m} \log\left[\zeta(m)\right], 
     \]
     where $\gamma \approx 0.577215664902$ is Euler's gamma constant. 
}.
\end{theorem} 

\begin{cor}[Product form of Mertens theorem] 
\label{lemma_Gz_productTermV2} 
We have that for all sufficiently large $x \gg 2$ 
\[
\prod_{p \leq x} \left(1 - \frac{1}{p}\right) = \frac{e^{-B}}{\log x}\left( 
     1 + o(1)\right), 
\]
where the notation for the absolute constant $0 < B < 1$ coincides with the definition of 
Mertens constant from Theorem \ref{theorem_Mertens_theorem}. 
Hence, for any real $z \geq 0$ we obtain that 
\[
\prod_{p \leq x} \left(1 - \frac{1}{p}\right)^{z} = 
     \frac{e^{-Bz}}{(\log x)^{z}} \left(1+o(1)\right)^{z} \sim 
     \frac{e^{-Bz}}{(\log x)^{z}}, \mathrm{\ as\ } x \rightarrow \infty. 
\]
\end{cor} 

Proofs of Theorem \ref{theorem_Mertens_theorem} and 
Corollary \ref{lemma_Gz_productTermV2} are found in 
\cite[\S 22.7; \S 22.8]{HARDYWRIGHT}. 

\begin{facts}[Exponential integrals and the incomplete gamma function] 
\label{facts_ExpIntIncGammaFuncs} 
\begin{subequations}
The following two variants of the \emph{exponential integral function} are defined by the 
integral representations \cite[\S 8.19]{NISTHB} 
\begin{align*} 
\operatorname{Ei}(x) & := \int_{-x}^{\infty} \frac{e^{-t}}{t} dt, \\ 
E_1(z) & := \int_1^{\infty} \frac{e^{-tz}}{t} dt, \Re(z) \geq 0. 
\end{align*} 
These two functions are related by $\operatorname{Ei}(-kz) = -E_1(kz)$ for real $k, z > 0$. 
We have the following inequalities providing 
quasi-polynomial upper and lower bounds on $\operatorname{Ei}(\pm x)$ for real $x > 0$: 
\begin{align}
\gamma + \log x - x \leq & \operatorname{Ei}(-x) \leq \gamma + \log x - x + \frac{x^2}{4}, \\ 
\notag 
1 + \gamma + \log x -\frac{3}{4} x \leq & \operatorname{Ei}(x) \phantom{-} \leq 
     1 + \gamma + \log x -\frac{3}{4} x + \frac{11}{36} x^2. 
\end{align}
The (upper) \emph{incomplete gamma function} is defined by \cite[\S 8.4]{NISTHB} 
\[
\Gamma(s, x) = \int_{x}^{\infty} t^{s-1} e^{-t} dt, \Re(s) > 0. 
\]
We have the following properties of $\Gamma(s, x)$: 
\begin{align} 
\label{eqn_IncompleteGamma_PropA} 
\Gamma(s, x) & = (s-1)! \cdot e^{-x} \times \sum_{k=0}^{s-1} \frac{x^k}{k!}, s \in \mathbb{Z}^{+}, \\ 
\label{eqn_IncompleteGamma_PropB} 
\Gamma(s, x) & \sim x^{s-1} \cdot e^{-x}, \mathrm{\ as\ } x \rightarrow \infty. 
\end{align}
\end{subequations}
\end{facts} 

\newpage 
\section{Components to the asymptotic analysis of lower bounds for 
         sums of arithmetic functions weighted by $\lambda(n)$} 
\label{Section_MVCh7_GzBounds} 

\subsection{A discussion of the results proved by Montgomery and Vaughan} 
\label{subSection_MVPrereqResultStmts} 

\begin{remark}[Intuition and constructions in Theorem \ref{theorem_HatPi_ExtInTermsOfGz}] 
\label{remark_intuitionConstrIn_theorem_HatPi_ExtInTermsOfGz} 
For $|z| < 2$ and $\Re(s) > 1$, let 
\begin{equation} 
\label{eqn_IntuitionMVThm_FszFuncDef_v1} 
F(s, z) := \prod_{p} \left(1 - \frac{z}{p^s}\right)^{-1} \left(1 - \frac{1}{p^s}\right)^{z}, 
\end{equation} 
and define the DGF coefficients, $a_z(n)$ for $n \geq 1$, by the relation 
\[
\zeta(s)^{z} \cdot F(s, z) := \sum_{n \geq 1} \frac{a_z(n)}{n^s}, \Re(s) > 1. 
\]
Suppose that $A_z(x) := \sum_{n \leq x} a_z(n)$ for $x \geq 1$. Then for the choice of the 
function $F(s, z)$ defined in \eqref{eqn_IntuitionMVThm_FszFuncDef_v1}, we obtain 
\[
A_z(x) = \sum_{n \leq x} z^{\Omega(n)} = \sum_{k \geq 0} \widehat{\pi}_k(x) z^k. 
\]
Thus for $r < 2$, by Cauchy's integral formula we have 
\[
\widehat{\pi}_k(x) = \frac{1}{2\pi\imath} \int_{|z|=r} \frac{A_z(x)}{z^{k+1}} dz. 
\]
Selecting $r := \frac{k-1}{\log\log x}$ leads to the uniform asymptotic formulas for 
$\widehat{\pi}_k(x)$ given in 
Theorem \ref{theorem_HatPi_ExtInTermsOfGz}. 
\end{remark} 

We also require the next theorems reproduced from \cite[\S 7.4]{MV} that handle the relative 
scarcity of the distribution of the $\Omega(n)$ for $n \leq x$ such that 
$\Omega(n) > \log\log x$. 

\begin{theorem}[Upper bounds on exceptional values of $\Omega(n)$ for large $n$] 
\label{theorem_MV_Thm7.20-init_stmt} 
Let 
\begin{align*} 
A(x, r) & := \#\left\{n \leq x: \Omega(n) \leq r \cdot \log\log x\right\}, \\ 
B(x, r) & := \#\left\{n \leq x: \Omega(n) \geq r \cdot \log\log x\right\}. 
\end{align*} 
If $0 < r \leq 1$ and $x \geq 2$, then 
\[
A(x, r) \ll x (\log x)^{r-1 - r\log r}, \text{ \ as\ } x \rightarrow \infty. 
\]
If $1 \leq r \leq R < 2$ and $x \geq 2$, then 
\[
B(x, r) \ll_R x \cdot (\log x)^{r-1-r \log r}, \text{ \ as\ } x \rightarrow \infty. 
\]
\end{theorem} 

\begin{theorem}[Exact bounds on exceptional values of $\Omega(n)$ for large $n$] 
\label{theorem_MV_Thm7.21-init_stmt} 
We have that uniformly 
\[
\#\left\{3 \leq n \leq x: \Omega(n) - \log\log n \leq 0\right\} = 
     \frac{x}{2} + O\left(\frac{x}{\sqrt{\log\log x}}\right). 
\]
\end{theorem} 

\begin{remark} 
The proofs of Theorem \ref{theorem_MV_Thm7.20-init_stmt} and 
Theorem \ref{theorem_MV_Thm7.21-init_stmt} 
are found in \S{7.4} of Montgomery and Vaughan. 
The key interpretation we need is the result stated in the next corollary. 
The precise way in which the bound 
stated in the previous theorem depends on the 
indeterminate paramater $R$ can be reviewed for reference in the proof 
algebra and relations cited in the reference \cite[\S 7]{MV}. 
The role of the parameter $R$ involved in stating the previous theorem 
is more notably critical as the scalar factor the upper bound on $k \leq R\log\log x$ in 
Theorem \ref{theorem_HatPi_ExtInTermsOfGz} up to which 
we obtain the valid uniform bounds in $x$ on the asymptotic formulas for 
$\widehat{\pi}_k(x)$. 

We have a discrepancy to work out in so much as we 
can only form summatory functions over the $\widehat{\pi}_k(x)$ for 
$1 \leq k \leq R\log\log x$ using the asymptotic formulas
guaranteed by Theorem \ref{theorem_HatPi_ExtInTermsOfGz}, even though we can actually 
have contributions from values distributed throughout the range $1 \leq \Omega(n) \leq \log_2(n)$ 
infinitely often. 
It is then crucial that we can show that the dominant growth of the asymptotic formulas we obtain 
for these summatory functions is captured by summing only over $k$ in the truncated range 
where the uniform bounds hold. 
\end{remark} 

\begin{cor} 
\label{theorem_MV_Thm7.20} 
Using the notation for $A(x, r)$ and $B(x, r)$ from 
Theorem \ref{theorem_MV_Thm7.20-init_stmt}, 
we have that for $\delta > 0$, 
\[
o(1) \leq \left\lvert \frac{B(x, 1+\delta)}{A(x, 1)} \right\rvert \ll 2, 
     \mathrm{\ as\ } \delta \rightarrow 0^{+}, x \rightarrow \infty. 
\]
\end{cor} 
\begin{proof} 
The lower bound stated above should be clear. To show that the asymptotic 
upper bound is correct, we compute using Theorem \ref{theorem_MV_Thm7.20-init_stmt} and 
Theorem \ref{theorem_MV_Thm7.21-init_stmt} that 
\begin{align*} 
\left\lvert \frac{B(x, 1+\delta)}{A(x, 1)} \right\rvert & \ll 
     \left\lvert \frac{x \cdot (\log x)^{\delta - \delta\log(1+\delta)}}{ 
     O(1) + \frac{x}{2} + 
     O\left(\frac{x}{\sqrt{\log\log x}}\right)} \right\rvert 
     \sim 
     \left\lvert \frac{(\log x)^{\delta - \delta\log(1+\delta)}}{ 
     \frac{1}{2} + o(1)}\right\rvert 
     \xrightarrow{\delta \rightarrow 0^{+}} 2, 
\end{align*} 
as $x \rightarrow \infty$. Notice that since $\mathbb{E}[\Omega(n)] = \log\log n + B$, with $0 < B < 1$ the 
absolute constant from Mertens theorem, 
when we denote the range of $k > \log\log x$ as holding in the form of 
$k > (1 + \delta) \log\log x$ for $\delta > 0$, we can assume that $\delta \rightarrow 0^{+}$ as 
$x \rightarrow \infty$. 
\end{proof} 

\subsection{New results based on refinements of Theorem \ref{theorem_HatPi_ExtInTermsOfGz}} 
\label{subSection_PartialPrimeProducts_Proofs} 

What the enumeratively flavored result 
in Theorem \ref{theorem_HatPi_ExtInTermsOfGz} allows us to do is get a 
sufficient lower bound on sums of positive and asymptotically bounded arithmetic functions 
weighted by the Liouville lambda function, $\lambda(n) = (-1)^{\Omega(n)}$. 
We seek to approximate $\mathcal{G}(z)$ defined in this theorem 
by only taking finite products of the primes in the factor of 
$\prod_{p} (1 - z/p)^{-1}$ for 
$p \leq ux$, e.g., indexing the component products only over those primes 
$p \in \left\{2,3,5,\ldots,x\right\}$ as $x \rightarrow \infty$. 
We can extend the argument behind the constructions sketched in 
Remark \ref{remark_intuitionConstrIn_theorem_HatPi_ExtInTermsOfGz} to 
justify that it suffices to take only these finite products to obtain a lower bound on 
$\widehat{\pi}_k(x)$. 

\begin{prop} 
\label{cor_PartialSumsOfReciprocalsOfPrimePowers} 
For real $s \geq 1$, let 
\[
P_s(x) := \sum_{p \leq x} p^{-s}, x \geq 2. 
\]
When $s := 1$, we have the asymptotic formula from Mertens theorem 
(see Theorem \ref{theorem_Mertens_theorem}). 
For all integers $s \geq 2$ 
there is an absolutely defined bounding function $\gamma_0(s, x)$ such that 
\[
\gamma_0(s, x) + o(1) \leq P_s(x), \mathrm{\ as\ } x \rightarrow \infty. 
\] 
It suffices to take the bound in the previous equation as 
\begin{align*} 
\gamma_0(z, x) & = s\log\left(\frac{\log x}{\log 2}\right) - 
     s(s-1) \log\left(\frac{x}{2}\right) - 
     \frac{1}{4} s(s-1)^2 \log^2(2). 
\end{align*}
\end{prop} 
\NBRef{A05-2020-04-26} 
\begin{proof} 
Let $s > 1$ be real-valued. 
By Abel summation with the summatory function $A(x) = \pi(x) \sim \frac{x}{\log x}$ and where 
our target function smooth function is $f(t) = t^{-s}$ with 
$f^{\prime}(t) = -s \cdot t^{-(s+1)}$, we obtain that 
\begin{align*} 
P_s(x) & = \frac{1}{x^s \cdot \log x} + s \cdot \int_2^{x} \frac{dt}{t^s \log t} \\ 
     & = \operatorname{Ei}(-(s-1) \log x) - \operatorname{Ei}(-(s-1) \log 2) + o(1), 
     \mathrm{\ as\ } x \rightarrow \infty. 
\end{align*} 
Now using the inequalities in Facts \ref{facts_ExpIntIncGammaFuncs}, we obtain that the 
difference of the exponential integral functions is bounded above and below by 
\begin{align*} 
\frac{P_s(x)}{s} & \geq \log\left(\frac{\log x}{\log 2}\right) - (s-1) \log\left(\frac{x}{2}\right) - 
     \frac{1}{4} (s-1)^2 \log^2(2) \\ 
\frac{P_s(x)}{s} & \leq \log\left(\frac{\log x}{\log 2}\right) - (s-1) \log\left(\frac{x}{2}\right) + 
     \frac{1}{4} (s-1)^2 \log^2(x). 
\end{align*} 
This completes the proof of the bound cited above in this lemma. 
\end{proof} 

\NBRef{A06-2020-04-26} 
\begin{proof}[Proof of Theorem \ref{theorem_GFs_SymmFuncs_SumsOfRecipOfPowsOfPrimes}] 
\label{proofOf_theorem_GFs_SymmFuncs_SumsOfRecipOfPowsOfPrimes} 
Notice that for real $0 \leq z < 2$ and any prime $p \geq 2$, we have that 
$\left(1-z/p\right)^{-1} \geq 1$ with equality if and only if $z := 0$. 
So if for $0 \leq z < 2$ and integers $x \geq 2$, we define the function 
\[
\widehat{P}(z, x) := \prod_{p \leq x} \left(1 - \frac{z}{p}\right)^{-1}, 
\]
then the right-hand-side product is finite as $x \rightarrow \infty$. 
Moreover, for $x \geq 2$ the product function $\widehat{P}(z, x)$ is a non-decreasing 
function of $x$ when $0 \leq z < 2$. Thus, if we form an approximation to the function 
$\mathcal{G}(z)$ from Theorem \ref{theorem_HatPi_ExtInTermsOfGz} by truncating the 
product factor corresponding to the limiting case of 
$\widehat{P}(z, x)$ above by indexing only for primes $0 \leq p \leq x$, then what we 
obtain in the formula for $\widehat{\pi}_k(x)$ 
guaranteed by the theorem is indeed a \emph{lower bound} for the summatory function. 

For fixed, finite $x \geq 2$ let 
\[
\mathbb{P}_x := \left\{n \geq 1: \mathrm{ all\ prime\ factors\ } 
     p|n \mathrm{\ satisfy\ } p \leq x\right\}. 
\]
Then we can see as in the constructions from Montgomery and Vaughan sketeched in 
Remark \ref{remark_intuitionConstrIn_theorem_HatPi_ExtInTermsOfGz} that 
\begin{equation} 
\label{eqn_proof_tag_PHatFiniteTruncProdFactorOfGz_v2} 
\prod_{p \leq x} \left(1 - \frac{z}{p^s}\right)^{-1} = \sum_{n \in \mathbb{P}_x} 
     \frac{z^{\Omega(n)}}{n^s}, x \geq 2. 
\end{equation} 
So by extending the argument we employed in the remark summarizing the proof given in 
\cite[\S 7.4]{MV}, we have that the formulas for 
\[
A_z(x) := \sum_{n \leq x} z^{\Omega(n)} = \sum_{k \geq 0} \widehat{\pi}_k(x) z^k, 
\]
depending on approximations (or inputs) to $\mathcal{G}(z)$ will also be valid by taking 
the finite products in \eqref{eqn_proof_tag_PHatFiniteTruncProdFactorOfGz_v2}. 
This happens since the products of all non-negative integral powers of the 
primes $p \leq x$ generate the integers $\{1 \leq n \leq x\}$ as a subset. 
So the bound on $\widehat{\pi}_k(x)$ that depends on $\mathcal{G}(z)$ formed by such 
an approximation to this product factor yields a new asymptotic formula for these 
summatory functions from below for each sufficiently large finite $x$ as 
$x \rightarrow \infty$. 

We must now find effective bounds on the truncated products in 
\eqref{eqn_proof_tag_PHatFiniteTruncProdFactorOfGz_v2} 
that are both meaningful and still simple enough to use in our new formulas. 
We have for all integers $0 \leq k < +\infty$, and any sequence 
$\{f(n)\}_{n \geq 1}$ with bounded partial sums, that 
\cite[\S 2]{MACDONALD-SYMFUNCS} 
\begin{equation} 
\label{eqn_pf_tag_hSymmPolysGF} 
[z^k] \prod_{i \geq 1} (1-f(i) z)^{-1} = [z^k] \exp\left(\sum_{j \geq 1} 
     \left(\sum_{i=1}^m f(i)^j\right) \frac{z^j}{j}\right), |z| < 1. 
\end{equation} 
In our case we have that $f(i)$ denotes the $i^{th}$ prime in the 
generating function expansion of \eqref{eqn_pf_tag_hSymmPolysGF}. 
It follows from Proposition \ref{cor_PartialSumsOfReciprocalsOfPrimePowers} that 
for $0 \leq z < 1$ we obtain 
\begin{align} 
\notag 
\log\left[\prod_{p \leq x} \left(1-\frac{z}{p}\right)^{-1}\right] & \geq (B + \log\log x) z + 
     \sum_{j \geq 2} \left[a(x) - b(x)(j-1) - c(x) (j-1)^2\right] z^j \\ 
\notag 
     & = (B + \log\log x) z - a(x) \left(1 - \frac{1}{1-z} + z\right) \\ 
\notag 
     & \phantom{= (B + \log\log x) z\ } + 
     b(x) \left( 
     1 - \frac{2}{1-z} + \frac{1}{(1-z)^2}\right) \\ 
\notag 
     & \phantom{= (B + \log\log x) z\ } + 
     c(x) \left( 
     1 - \frac{4}{1-z} + \frac{5}{(1-z)^2} - \frac{2}{(1-z)^3}\right) \\ 
\label{eqn_proof_tag_PHatFiniteTruncProdFactorOfGz_v3} 
     & =: \widehat{\mathcal{B}}(x; z). 
\end{align} 
In the previous equations, the lower bounds formed by the functions 
$(a,b,c) \equiv (a_{\ell}, b_{\ell}, c_{\ell})$ 
evaluated at $x$ are given by the corresponding lower bounds from 
Proposition \ref{cor_PartialSumsOfReciprocalsOfPrimePowers} as 
\begin{align*} 
(a_{\ell}, b_{\ell}, c_{\ell}) & := \left(\log\left(\frac{\log x}{\log 2}\right), 
     \log\left(\frac{x}{2}\right), \frac{1}{4} \log^2 2\right). 
\end{align*} 
We make a decision to set the uniform bound parameter to a middle ground value of 
$1 < R < 2$ at $R := \frac{3}{2}$ 
so that $$z \equiv z(k, x) = \frac{k-1}{\log\log x} \in [0, R),$$ 
in the notation of Theorem \ref{theorem_HatPi_ExtInTermsOfGz} 
for $x \gg 1$ sufficiently large. 
Thus $(z-1)^{-m} \in [(-1)^m, 2^m]$ for integers $m \geq 1$, which implies that 
\[
c_{\ell}(x) \left( 
     1 + \frac{4}{z-1} + \frac{5}{(z-1)^2} + \frac{2}{(z-1)^3}\right) \geq 0. 
\]
Moreover, by taking the first derivative of the function 
\[
f(z) := 1 - \frac{2}{1-z} + \frac{1}{(1-z)^2}, z \in [0, 1), 
\]
we can compute that its extrema occur at $z := 0$, or as $z \rightarrow 1^{-}$. 
In the former case, we obtain that the limiting exponent is zero. 
Clearly, as $z \rightarrow 1$, $f(z)$ becomes unbounded, which would lead to 
a zero tending lower bound that is also inconsistent with the known bounds we 
obtain when $k := 1, 2$ in \cite[\S 7.4,\ p. 228]{MV}. 
We thus let this term become 
zero-valued when we evaluate the function $\widehat{\mathcal{B}}(x; z)$ in practice. 
We also conclude that 
\[
1 + z - \frac{1}{1-z} \geq z, z \in [0, 1). 
\]
Then we obtain the next effective lower bounds on the function from 
\eqref{eqn_proof_tag_PHatFiniteTruncProdFactorOfGz_v3}:  
\begin{align} 
\label{eqn_proof_tag_simpl_v1} 
\frac{e^{-Bz}}{(\log x)^{z}} \times \exp\left(\widehat{\mathcal{B}}(u, x; z)\right) & \gg 
     \left(\frac{\log 2}{\log x}\right)^{z} =: \widehat{\mathcal{C}}(u, x; z) 
\end{align} 
Indeed, setting $z := 0$ on these select terms in 
\eqref{eqn_proof_tag_PHatFiniteTruncProdFactorOfGz_v3} does still in fact lead to lower bounds 
up to a constant by considering the series expansions of the product term we have estimated as a 
series in $z$. 
%In particular, what we need to show to justify this step is that as $x \rightarrow \infty$ 
%     \[
%     \mathcal{G}\left(\frac{k-1}{\log\log x}\right) \gg 
%          \left(\frac{\log 2}{\log x}\right)^{\frac{k-1}{\log\log x}}. 
%     \] 
%     As we approach large limiting cases of $x$, we approximate 
%     (bounding $\Gamma(1+z)$ by constants as below) 
%     \[
%     \mathcal{G}\left(\frac{k-1}{\log\log x}\right) \sim 
%          %\frac{1}{\Gamma\left(\frac{k-1}{\log\log x} + 1\right)} \times 
%          C \times 
%          \left(\sum_{n \geq 1} \frac{1}{n} \cdot \left(\frac{k-1}{\log\log x}\right)^{\Omega(n)}\right) \times 
%          \frac{\exp\left(-\frac{B(k-1)}{\log\log x}\right)}{(\log x)^{\frac{k-1}{\log\log x}}}. 
%     \]
%     Now taking the left-hand-side series terms over only the primes $p$, we can justify that 
%     \begin{align*} 
%     \lim_{x \rightarrow \infty} \left[ \sum_{p \leq x} \left(\frac{k}{\log\log x}\right) \cdot 
%          \frac{1}{p} \times \left(1 + o\left(\frac{k}{\log\log x}\right)\right) 
%          \right] & \sim k 
%          \gg (\log 2)^{\frac{k}{\log\log x}} \gg \log 2. 
%     \end{align*} 
%     %The point of the lower bound stated in this way is less about precision, and much more about a 
%     %smoother way towards summing weights of $\lambda(n)$ according to the parity of $\Omega(n) = k$ 
%     %for use with Abel summation. 

Finally, to finish our proof of the new form of the lower bound on $\mathcal{G}(z)$ in 
Theorem \ref{theorem_HatPi_ExtInTermsOfGz}, 
we need to bound the reciprocal factor of $\Gamma(z+1)$. 
Since $z \equiv z(k, x) = \frac{k-1}{\log\log x}$ and 
$k \in [1, R\log\log x]$, or with $z \in [0, R)$, 
we obtain for minimal $k$ and all large enough $x \gg 1$ that 
$\Gamma(z+1) \approx 1$, and for $k$ towards the upper range of 
its interval that $\Gamma(z+1) \leq \Gamma(5/2) = \frac{3}{4} \sqrt{\pi}$. 
\end{proof} 

\newpage
\section{Average case analysis of bounds on the Dirichlet inverse functions, $g^{-1}(n)$} 
\label{Section_InvFunc_PreciseExpsAndAsymptotics} 

The property in (C) of 
Conjecture \ref{lemma_gInv_MxExample} along squarefree $n \geq 1$ 
captures an important characteristic of $g^{-1}(n)$ that holds more globally 
for all $n \geq 1$. In particular, these 
functions can be expressed via more simple formulas than inspection of the 
first few initial values of the repetitive, quasi-periodic sequence otherwise suggests. 
The pages of tabular data given as Table \ref{table_conjecture_Mertens_ginvSeq_approx_values} 
in the appendix section starting on 
page \pageref{table_conjecture_Mertens_ginvSeq_approx_values} are intended to 
provide clear insight into why we arrived at the convenient approximations to 
$g^{-1}(n)$ proved in this section. The table offers illustrative 
numerical data formed by examining the approximate behavior 
at hand for the first several cases of $1 \leq n \leq 500$ with \emph{Mathematica}. 

\subsection{Definitions and basic properties of component function sequences} 

We define the following sequence for integers $n \geq 1, k \geq 0$: 
\begin{align} 
\label{eqn_CknFuncDef_v2} 
C_k(n) := \begin{cases} 
     \varepsilon(n), & \text{ if $k = 0$; } \\ 
     \sum\limits_{d|n} \omega(d) C_{k-1}(n/d), & \text{ if $k \geq 1$. } 
     \end{cases} 
\end{align} 
The sequence of important semi-diagonals of these functions begins as 
\cite[\seqnum{A008480}]{OEIS} 
\[
\{\lambda(n) \cdot C_{\Omega(n)}(n) \}_{n \geq 1} \mapsto \{
     1, -1, -1, 1, -1, 2, -1, -1, 1, 2, -1, -3, -1, 2, 2, 1, -1, -3, -1, \
     -3, 2, 2, -1, 4, 1, 2, \ldots \}. 
\]

\begin{remark}[An effective range of $k$ depending on a fixed large $n$]
Notice that by expanding the recursively-based definition in \eqref{eqn_CknFuncDef_v2} 
out to its maximal depth by nested divisor sums, for fixed $n$, $C_k(n)$ is seen to 
only ever possibly be non-zero for $k \leq \Omega(n)$. 
Thus, the effective range of $k$ for fixed $n$ is restricted by the 
conditions of $C_0(n) = \delta_{n,1}$ and that $C_k(n) = 0$, $\forall k > \Omega(n)$ 
whenever $n \geq 2$. 
\end{remark} 

\begin{example}[Special cases of the functions $C_k(n)$ for small $k$] 
\label{example_SpCase_Ckn} 
We cite the following special cases which are verified by 
explicit computation using \eqref{eqn_CknFuncDef_v2}\footnote{ 
     For all $n,k \geq 2$, we have the following recurrence 
     relation satisfied by $C_k(n)$ between successive values of $k$: 
     \begin{equation*}
     C_k(n) = \sum_{p|n} \sum_{d\rvert\frac{n}{p^{\nu_p(n)}}} \sum_{i=0}^{\nu_p(n)-1} 
          C_{k-1}\left(dp^i\right), n \geq 1. 
     \end{equation*}
}: 
\NBRef{A07-2020-04-26} 
\begin{align*} 
C_0(n) & = \delta_{n,1} \\ 
C_1(n) & = \omega(n) \\ 
C_2(n) & = d(n) \times \sum_{p|n} \frac{\nu_p(n)}{\nu_p(n)+1} - \gcd\left(\Omega(n), \omega(n)\right). 
\end{align*} 
\end{example} 

\subsection{Uniform asymptotics of $C_k(n)$ for large all $n$ and fixed, bounded $k$} 

The next theorem makes precise what these formulas already 
suggest about the main terms of the growth rates of 
$C_k(n)$ as functions of $k,n$ for limiting cases of $n$ large and fixed $k$ which is 
necessarily bounded in $n$, but still taken as an independent parameter. 

\begin{theorem}[Asymptotics for the functions $C_k(n)$] 
\label{theorem_Ckn_GeneralAsymptoticsForms} 
For $k := 0$, we have by definition that $C_0(n) = \delta_{n,1}$. 
For all sufficiently large $n > 1$ and any fixed $1 \leq k \leq \Omega(n)$ 
taken independently of $n$, 
we obtain that the dominant asymptotic term for $C_k(n)$ is given uniformly by 
\[
\mathbb{E}[C_k(n)] \gg (\log\log n)^{2k-1}, \mathrm{\ as\ }n \rightarrow \infty. 
\]
\end{theorem} 
\NBRef{A08-2020-04-26} 
\begin{proof} 
\label{proofOf_theorem_Ckn_GeneralAsymptoticsForms} 
We prove our bounds by induction on $k$. 
We can see by Example \ref{example_SpCase_Ckn} that $C_1(n)$ 
satsfies the formula we must establish when $k := 1$ since $\mathbb{E}[\omega(n)] = \log\log n$. 
Suppose that $k \geq 2$ and let the inductive assumption state that for all $1 \leq m < k$ 
\[
\mathbb{E}[C_m(n)] \gg (\log\log n)^{2m-1}. 
\] 
Now using the recursive formula we used to define the sequences of $C_k(n)$ in 
\eqref{eqn_CknFuncDef_v2}, we have that as $n \rightarrow \infty$\footnote{ 
     For all large $x \gg 2$ the summatory function of $\omega(n)$ satisfies 
     \cite[\S 22.10]{HARDYWRIGHT} 
     \[
     \sum_{n \leq x} \omega(n) = x \log\log x + Bx + O\left(\frac{x}{\log x}\right). 
     \]
}
\begin{align*} 
\mathbb{E}[C_k(n)] & = \mathbb{E}\left[\sum_{d|n} \omega(n/d) C_{k-1}(d)\right] \\ 
     & = \frac{1}{n} \times \sum_{d \leq n} C_{k-1}(d) \times \sum_{r=1}^{\Floor{n}{d}} \omega(r) \\ 
     & \sim \sum_{d \leq n} C_{k-1}(d) \left[ 
     \frac{\log\log(n/d) \Iverson{d \leq \frac{n}{e}}}{d} + \frac{B}{d}\right] \\ 
     & \sim \sum_{d \leq \frac{n}{e}} \left[ 
     \sum_{m < d} \frac{\mathbb{E}[C_{k-1}(m)]}{m} \log\log\left(\frac{n}{m}\right) + 
     B \cdot \mathbb{E}[C_{k-1}(d)] + B \cdot \sum_{m < d} \frac{\mathbb{E}[C_{k-1}(m)]}{m} 
     \right] \\ 
     & \gg B \left[n \log n \cdot (\log\log n)^{2k-3} - \log n \cdot (\log\log n)^{2k-3}\right] \times 
     \left(1 + \frac{\log n}{2}\right) \\ 
     & \gg (\log\log n)^{2k-1}. 
\end{align*} 
In transitioning to the last equation from the previous step, we have used that 
$\frac{Bn}{2} \cdot (\log n)^2 \gg (\log\log n)^2$ as $n \rightarrow \infty$. We have also used that for large 
$n \rightarrow \infty$ and fixed $m$, we have by an approximation to the incomplete gamma function given by 
\[
\int_{e}^{n} \frac{(\log\log t)^m}{t} \sim (\log n) (\log\log n)^{m}, 
     \mathrm{\ as\ } n \rightarrow \infty. 
\]
Thus the claim holds by mathematical induction whenever $n \rightarrow \infty$ is large and 
$1 \leq k \leq \Omega(n)$. 
\end{proof} 

\subsection{Relating the auxiliary functions $C_k(n)$ to formulas approximating $g^{-1}(n)$} 

\begin{lemma}[An exact formula for $g^{-1}(n)$] 
\label{lemma_AnExactFormulaFor_gInvByMobiusInv_v1} 
For all $n \geq 1$, we have that 
\[
g^{-1}(n) = \sum_{d|n} \mu\left(\frac{n}{d}\right) \lambda(d) C_{\Omega(d)}(d). 
\]
\end{lemma}
\begin{proof} 
We first write out the standard recurrence relation for the Dirichlet inverse of 
$\omega+1$ as 
\begin{align*} 
g^{-1}(n) & = - \sum_{\substack{d|n \\ d>1}} (\omega(d) + 1) g^{-1}(n/d) 
     \quad\implies\quad 
     (g^{-1} \ast 1)(n) = -(\omega \ast g^{-1})(n). 
\end{align*} 
Now by repeatedly expanding the right-hand-side, and removing corner cases in the nested sums with 
$\omega(1) = 0$, we find inductively that 
\[
(g^{-1} \ast 1)(n) = (-1)^{\Omega(n)} C_{\Omega(n)}(n) = \lambda(n) C_{\Omega(n)}(n). 
\]
The statement then follows by M\"obius inversion applied to each side of the last equation. 
\end{proof} 

\begin{cor} 
\label{cor_AnExactFormulaFor_gInvByMobiusInv_nSqFree_v2} 
For all squarefree integers $n \geq 1$, we have that 
\begin{equation} 
\label{eqn_gInvnSqFreeN_exactDivSum_Formula} 
g^{-1}(n) = \lambda(n) \times \sum_{d|n} C_{\Omega(d)}(d). 
\end{equation} 
\end{cor} 
\begin{proof} 
Since $g^{-1}(1) = 1$, clearly the claim is true for $n = 1$. Suppose that $n \geq 2$ and that 
$n$ is squarefree. Then $n = p_1p_2 \cdots p_{\omega(n)}$ where $p_i$ is prime for all 
$1 \leq i \leq \omega(n)$. So we can transform the exact divisor sum guaranteed for all $n$ in 
Lemma \ref{lemma_AnExactFormulaFor_gInvByMobiusInv_v1} as follows: 
\begin{align*} 
g^{-1}(n) & = \sum_{i=0}^{\omega(n)} \sum_{\substack{d|n \\ \omega(d)=i}} (-1)^{\omega(n) - i} (-1)^{i} \cdot 
     C_{\Omega(d)}(d) \\ 
     & = \lambda(n) \times \sum_{i=0}^{\omega(n)} \sum_{\substack{d|n \\ \omega(d)=i}} C_{\Omega(d)}(d) \\ 
     & = \lambda(n) \times \sum_{d|n} C_{\Omega(d)}(d). 
\end{align*} 
The signed contributions in the first of the previous equations is 
justified by noting that $\lambda(n) = (-1)^{\omega(n)}$ 
whenever $n$ is squarefree, and that for $d$ squarefree with 
$\omega(d) = i$, $\Omega(d) = i$. 
\end{proof} 

%\begin{proof}[Proof of property (C) of Conjecture \ref{lemma_gInv_MxExample}] 
%\label{proofOf_propCFromConj_lemma_gInv_MxExample} 
%We can prove by induction on $\omega(n)$, the number of distinct prime factors of $n \geq 2$, that 
%for all squarefree integers $n \geq 1$, $C_{\Omega(n)}(n) = (\omega(n))!$. Since $g^{-1}(1) = 1$, 
%clearly the conjecture is true for $n = 1$. For squarefree $n \geq 2$, we can prove property (C) 
%directly by applying Lemma \ref{lemma_AnExactFormulaFor_gInvByMobiusInv_v1}. That is, since 
%all divisors of $n$ squarefree are also squarefree, with the number of $d|n$ with exactly $k$ prime factors 
%given by $\binom{\omega(n)}{k}$ for $0 \leq k \leq \omega(n)$, we have that 
%\begin{align*} 
%g^{-1}(n) & = \sum_{k=0}^{\omega(n)} \sum_{\substack{d|n \\ \omega(d) = k}} 
%     \mu(n/d) \lambda(d) C_{\Omega(d)}(d) \\ 
%     & = \sum_{k=0}^{\omega(n)} \binom{\omega(n)}{k} \#\{1 \leq d \leq n: d|n, \omega(d) = k\} \times 
%     (-1)^{\omega(n) - k} \cdot (-1)^{k} \cdot C_{k}\left(p_1 \cdots p_k \Iverson{k > 0} + \Iverson{k = 0}\right) \\ 
%     & = (-1)^{\omega(n)} \times \sum_{k=0}^{\omega(n)} \binom{\omega(n)}{k} \cdot k!. 
%\end{align*} 
%Finally, since $\Omega(n) = \omega(n)$ whenever $n$ is squarefree, we obtain that the leading sign term on 
%the sum in the previous equation is indeed $\lambda(n)$, as expected. 
%\end{proof} 

\begin{cor} 
\label{lemma_BddExpectationOfgInvn} 
We have that 
\[
\frac{6}{\pi^2} (\log n) (\log\log n) \ll 
     \mathbb{E}|g^{-1}(n)| \leq 
     \mathbb{E}\left[\sum_{d|n} C_{\Omega(d)}(d)\right]. 
\]
\end{cor} 
\begin{proof} 
To prove the lower bound, first notice that by 
Lemma \ref{lemma_AnExactFormulaFor_gInvByMobiusInv_v1}, 
Proposition \ref{prop_SignageDirInvsOfPosBddArithmeticFuncs_v1} and the 
complete multiplicativity of $\lambda(n)$, 
we easily obtain that 
\begin{equation} 
\label{eqn_AbsValueOf_gInvn_FornSquareFree_v1} 
|g^{-1}(n)| = \sum_{d|n} \mu^2\left(\frac{n}{d}\right) C_{\Omega(d)}(d). 
\end{equation} 
In particular, since $\mu(n)$ is non-zero only at squarefree integers and 
at any squarefree $n \geq 1$ we have $\mu(n) = (-1)^{\omega(n)} = \lambda(n)$, 
Lemma \ref{lemma_AnExactFormulaFor_gInvByMobiusInv_v1} implies 
\begin{align*} 
|g^{-1}(n)| & = \lambda(n) \times \sum_{d|n} \mu\left(\frac{n}{d}\right) \lambda(d) C_{\Omega(d)}(d) \\ 
     & = \sum_{d|n} \mu^2\left(\frac{n}{d}\right) \lambda\left(\frac{n}{d}\right) 
     \lambda(nd) C_{\Omega(d)}(d) \\ 
     & = \lambda(n^2) \times \sum_{d|n} \mu^2\left(\frac{n}{d}\right) C_{\Omega(d)}(d). 
\end{align*} 
Notice in the above equation 
that $\lambda(n^2) = +1$ for all $n \geq 1$ since the number of distinct 
prime factors (counting multiplicity) of any square integer is necessarily even. 
     
Recall from the introduction that the summatory function of the 
squarefree integers is given by 
\[
Q(x) := \sum_{n \leq x} \mu^2(n) = \frac{6}{\pi^2} x + O(\sqrt{x}). 
\]
Then since $C_{\Omega(d)}(d) \geq 1$ for all $d \geq 1$, and since 
$\mathbb{E}[C_k(d)]$ is minimized when $k := 1$, we obtain by summing over 
\eqref{eqn_AbsValueOf_gInvn_FornSquareFree_v1} that as $x \rightarrow \infty$ 
\begin{align*} 
\frac{1}{x} \times \sum_{n \leq x} |g^{-1}(n)| & = \frac{1}{x} \times \sum_{d \leq x} 
     C_{\Omega(d)}(d) Q\left(\Floor{x}{d}\right) \\ 
     & \sim \sum_{d \leq x} C_{\Omega(d)}(d) \left[\frac{6}{d \cdot \pi^2} + O\left(\frac{1}{\sqrt{dx}}\right) 
     \right] \\ 
     & \geq \sum_{d \leq x} \left[\frac{6 \cdot C_{\Omega(d)}(d)}{d \cdot \pi^2} + 
     O\left(\frac{1}{\sqrt{dx}}\right)\right] \\ 
     & = \frac{6}{\pi^2} \left[\mathbb{E}[C_{\Omega(x)}(x)] + \sum_{d<x} 
     \frac{\mathbb{E}[C_{\Omega(d)}(d)]}{d}\right] + 
     O\left(\frac{1}{\sqrt{x}} \times \int_0^{x} t^{-1/2} dt\right) \\ 
     & \gg \frac{6}{\pi^2} \left[\sum_{e \leq d \leq x} 
     \frac{\log\log d}{d}\right] + O(1) \\ 
     & \sim \frac{6}{\pi^2} \times \int_{e}^{x} \frac{\log\log t}{t} dt + O(1) \\ 
     & \gg \frac{6}{\pi^2} (\log x) (\log\log x). 
\end{align*} 
To prove the upper bound, notice that by 
Lemma \ref{lemma_AnExactFormulaFor_gInvByMobiusInv_v1} and 
Corollary \ref{cor_AnExactFormulaFor_gInvByMobiusInv_nSqFree_v2}, 
\[
|g^{-1}(n)| \leq \sum_{d|n} C_{\Omega(d)}(d). 
\]
Now since both of the above quantities are positive for all $n \geq 1$, 
we must obtain the upper bound on the average order of $|g^{-1}(n)|$
stated above. 
\end{proof} 

\newpage
\section{Lower bounds for $M(x)$ along infinite subsequences} 
\label{Section_KeyApplications} 

\subsection{The culmination of what we have done so far} 

%The summation methods to weight sums of our arithmetic functions according to the sign of 
%$\lambda(n)$ (or parity of $\Omega(n)$) we prove within this section are 
%reminiscent of the combinatorially motivated sieve methods in 
%\cite[\S 17]{OPERADECRIBERO}. 
\nocite{OPERADECRIBERO}

\begin{prop}
\label{prop_Mx_SBP_IntegralFormula} 
For all sufficiently large $x$, we have that 
\begin{align} 
\label{eqn_pf_tag_v2-restated_v2} 
M(x) & \approx G^{-1}(x) - x \cdot \int_1^{x/2} \frac{G^{-1}(t)}{t^2 \cdot \log(x/t)} dt. 
\end{align} 
\end{prop} 
\begin{proof} 
We know by applying Corollary \ref{cor_Mx_gInvnPixk_formula} that 
\begin{align} 
\notag
M(x) & = \sum_{k=1}^{x} g^{-1}(k) (\pi(x/k)+1) \\ 
\label{eqn_proof_tag_MxFormulaInitSepTerms_v1} 
     & \approx G^{-1}(x) + \sum_{k=1}^{x} g^{-1}(k) \pi(x/k), 
\end{align} 
We can replace the asymptotically unnecessary floored integer-valued arguments to $\pi(x)$ 
in \eqref{eqn_proof_tag_MxFormulaInitSepTerms_v1} using 
its approximation by the monotone non-decreasing asymptotic order, $\pi(x) \sim \frac{x}{\log x}$. 
Moreover, we can always 
bound $$\frac{Ax}{\log x} \leq \pi(x) \leq \frac{Bx}{\log x},$$ for suitably defined 
absolute constants, $A,B > 0$ whenever $x \geq 2$. 
Therefore the approximation obtained by replacing $\pi(x)$ by the dominant term in its 
limting asymptotic formula is actually valid for all $x > 1$ up to at most 
a small constant difference. 

What we require to sum and simplify the right-hand-side terms from 
\eqref{eqn_proof_tag_MxFormulaInitSepTerms_v1} essentially follows from 
summation by parts\footnote{
     For any arithmetic functions, $u_n,v_n$, 
     with $U_j := u_1+u_2+\cdots+u_j$ for $j \geq 1$, we have that 
     \cite[\S 2.10(ii)]{NISTHB} 
     \[
     \sum_{j=1}^{n-1} u_j \cdot v_j = U_{n-1} v_n + 
          \sum_{j=1}^{n-1} U_j \left(v_j - v_{j+1}\right), n \geq 2. 
     \]
}. 
In particular, we argue that for sufficiently large 
$x \geq 2$ we can approximate\footnote{
     Since $\pi(1) = 0$, the actual range of summation corresponds to 
     $k \in \left[1, \frac{x}{2}\right]$. 
}
\begin{align*} 
\sum_{k=1}^{x} g^{-1}(k) \pi(x/k) & = G^{-1}(x) \pi(1) - \sum_{k=1}^{x-1} G^{-1}(k) \left[ 
     \pi\left(\frac{x}{k}\right) - \pi\left(\frac{x}{k+1}\right)\right] \\ 
     & = -\sum_{k=1}^{x/2} G^{-1}(k) \left[ 
     \pi\left(\frac{x}{k}\right) - \pi\left(\frac{x}{k+1}\right)\right] \\ 
     & \approx -\sum_{k=1}^{x/2} G^{-1}(k) \left[ 
     \frac{x}{k \cdot \log(x/k)} - \frac{x}{(k+1) \cdot \log(x/k)}\right] \\ 
     & \approx -\sum_{k=1}^{x/2} G^{-1}(k) \frac{x}{k^2 \cdot \log(x/k)}. 
\end{align*} 
Since for $x$ large enough the summand factor $\frac{x}{k^2 \cdot \log(x/k)}$ 
is monotonic as $k$ ranges over $k \in [1, x/2]$ in ascending order, since this 
summand factor is a smooth function of $k$ (and $x$), and since $G^{-1}(x)$ is 
a summatory function with jumps only at the positive integers, we can approximate 
$M(x)$ for any finite $x \geq 2$ by 
\[
M(x) \approx G^{-1}(x) - x \cdot \int_1^{x/2} \frac{G^{-1}(t)}{t^2 \cdot \log(x/t)} dt. 
\]
We will later only use unsigned lower bound approximations to this function in the next theorems so that 
the signedness of the summatory function term in the integral formula above 
as $x \rightarrow \infty$ is a moot point entirely. 
\end{proof} 

\subsection{Establishing initial lower bounds on the summatory functions $G^{-1}(x)$} 
\label{Section_ProofOfValidityOfAverageOrderLowerBounds} 

Let the summatory function $G_E^{-1}(x)$ be defined for $x \geq 1$ by\footnote{ 
     The subscript of $E$ (as in to be evaluated in expectation) 
     on the function $G_E^{-1}(x)$ is purely for formality of 
     notation and does not correspond to 
     an actual parameter or any implicit dependence on $E$ in the formula 
     that defines this function. 
}
\begin{equation} 
\label{eqn_GEInvxSummatoryFuncDef_v1} 
G_E^{-1}(x) := \sum_{n \leq (\log x)^{\frac{3}{2}} (\log\log x)} \lambda(n) \times 
     \sum_{\substack{d|n \\ d > e}} \frac{(\log d)^{\frac{1}{4}}}{\log\log d}. 
\end{equation} 

\begin{theorem} 
\label{theorem_GInvxLowerBoundByGEInvx_v1} 
For almost all sufficiently large integers $x \rightarrow \infty$, we have that 
\[
|G^{-1}(x)| \gg |G_E^{-1}(x)|. 
\]
\end{theorem} 
\begin{proof} 
We bound the magnitude of each respective function respectively below and above in the worst 
cases of the cancellation imparted by the signage on the otherwise 
positive terms when weighted by $\lambda(n)$. 
First, consider the following upper bound on $|G_E^{-1}(x)|$: 
\begin{align*} 
|G_E^{-1}(x)| & = \left\lvert \sum_{\substack{e \leq n \leq (\log x)^{\frac{3}{2}} (\log\log x)}} \lambda(n) \times 
     \sum_{\substack{d|n \\ d > e}} \frac{(\log d)^{\frac{1}{4}}}{\log\log d} 
     \right\rvert \\ 
     & \ll \sum_{e < d \leq (\log x)^{\frac{3}{2}} (\log\log x)} \frac{(\log d)^{\frac{1}{4}}}{\log\log d} \cdot 
     \Floor{(\log x)^{\frac{3}{2}} (\log\log x)}{d} \\ 
     & \approx (\log x)^{\frac{3}{2}} (\log\log x) \times 
     \int_{e}^{(\log x)^{\frac{3}{2}} (\log\log x)} \frac{(\log t)^{\frac{1}{4}}}{t \cdot \log\log t} dt \\ 
     & = (\log x)^{\frac{3}{2}} (\log\log x) \times 
     \operatorname{Ei}\left(\frac{5}{4} \log\log\left((\log x)^{\frac{3}{2}} (\log\log x)\right)\right) \\ 
     & \ll \frac{25}{64} \cdot (\log x)^{\frac{3}{2}} (\log\log x) (\log\log\log x)^2. 
\end{align*} 
Next, we bound the summatory function $|G^{-1}(x)|$ from below. 
Notice that in applying the lower bound for $\mathbb{E}|g^{-1}(n)|$ from 
Corollary \ref{lemma_BddExpectationOfgInvn}, we obtain that for large $n \rightarrow \infty$ and any 
fixed positive constant $C > 0$, we obtain that 
\begin{equation} 
\label{eqn_proof_tag_GInvnLowerBound_rule_v0} 
\mathbb{E}|g^{-1}(Cn)| \gg \frac{6}{\pi^2}\left[(\log n)(\log\log n) + \log(C) (\log\log n)\right]. 
\end{equation} 
We define the following densities for large $x \geq 2$: 
\begin{align*} 
\mathcal{L}_{+}(x) := \frac{1}{n} \cdot \#\{n \leq x: \lambda(n) = +1\} \\ 
\mathcal{L}_{-}(x) := \frac{1}{n} \cdot \#\{n \leq x: \lambda(n) = -1\}. 
\end{align*} 
We know that \cite[\cf \S 1]{TAO-VALUEPATTERNS} 
\[
\lim_{x \rightarrow \infty} \mathcal{L}_{+}(x) = \lim_{x \rightarrow \infty} \mathcal{L}_{-}(x) = 
     \frac{1}{2}, 
\]
so that as $x \rightarrow \infty$, the two densities $\mathcal{L}_{+}(x),\mathcal{L}_{-}(x)$ may fluctuate, 
but cannot grow too far apart over extended intervals. 

Then we compute that for almost every sufficiently large $n \rightarrow \infty$: 
\begin{align*} 
\frac{|G^{-1}(x)|}{x} & = \frac{1}{n} \times \left\lvert 
     \sum_{\substack{d \leq x \\ \lambda(d)=+1}} |g^{-1}(d)| - 
     \sum_{\substack{d \leq x \\ \lambda(d)=-1}} |g^{-1}(d)| 
     \right\rvert \\ 
     & \gg \left\lvert 
     \mathbb{E}|g^{-1}\left(\mathcal{L}_{+}(x) x\right)| - 
     \mathbb{E}|g^{-1}\left((1-\mathcal{L}_{+}(x)) x\right)| 
     \right\rvert. 
\end{align*} 
So by applying the lower bound on the average order expectations from 
Corollary \ref{lemma_BddExpectationOfgInvn} along with 
\eqref{eqn_proof_tag_GInvnLowerBound_rule_v0}, we obtain for almost every 
large enough $x$ (i.e., with the exception of $x$ on a set of asymptotic density zero) that 
\begin{align} 
\notag 
|G^{-1}(x)| & \gg \frac{6x}{\pi^2} \biggl\lvert 
     \mathcal{L}_{+}(x) (\log x) (\log\log x) + (1-\mathcal{L}_{+}(x)) (\log x) (\log\log x) \\ 
\notag 
     & \phantom{\gg \frac{6x}{\pi^2} \biggl\lvert\ } + 
     \left(\mathcal{L}_{+}(x) \log\left[\mathcal{L}_{+}(x)\right] + 
     (1-\mathcal{L}_{+}(x)) \log\left[1-\mathcal{L}_{+}(x)\right]\right) (\log\log n) 
     \biggr\rvert \\ 
\label{eqn_proof_tag_GInvnLowerBound_stmt_v1} 
     & = \frac{6x}{\pi^2} \left\lvert (\log x)(\log\log x) + 
     \left(\log\left[1-\mathcal{L}_{+}(x)\right] + \mathcal{L}_{+}(x) \cdot \log\left[ 
     \frac{\mathcal{L}_{+}(x)}{1-\mathcal{L}_{+}(x)}\right]\right) (\log\log x) 
     \right\rvert, 
     \mathrm{\ as\ } x \rightarrow \infty. 
\end{align} 
Note for verification, that the largest $n \leq 500$ such that we have $G^{-1}(n) = 0$ in 
Table \ref{table_conjecture_Mertens_ginvSeq_approx_values} 
(see page \pageref{table_conjecture_Mertens_ginvSeq_approx_values_LastPage}) 
is given by $n := 426$. 
In this case, we have that $\mathcal{L}_{+}(n) \approx 0.485915$, 
$\mathcal{L}_{-}(n) \approx 0.514845$, and that our lower bound on the right-hand-side of 
\eqref{eqn_proof_tag_GInvnLowerBound_stmt_v1} gives approximately that 
$|G^{-1}(n)| \succsim 5.86971$, where it happens that $G^{-1}(427) = 5$. 

Finally, since for all sufficiently large $x \rightarrow \infty$, we have that 
\[
\frac{6x}{\pi^2} (\log x) (\log\log x) \gg 
     \frac{25}{64} \cdot (\log x)^{\frac{3}{2}} (\log\log x) (\log\log\log x)^2, 
\] 
we have that our claimed relation between the two key summatory functions holds. 
\end{proof} 

\subsubsection{A few more necessary results} 
\label{subsubSection_RoutineProofsNeededForMainBoundOnGInvxFunc} 

We now use the superscript and subscript notation of 
$(\ell)$ not to denote a formal parameter to 
the functions we define below, but instead to denote that these functions form 
\emph{lower bound} (rather than exact) 
approximations to other forms of the functions without the scripted $(\ell)$. 

\begin{lemma} 
\label{lemma_lowerBoundsOnLambdaFuncParitySummFuncs} 
Suppose that $\widehat{\pi}_k(x) \geq \widehat{\pi}_k^{(\ell)}(x) \geq 0$ 
for $\widehat{\pi}_k^{(\ell)}(x)$ a monotone real-valued function of $x$ 
for all integers $k \geq 1$ whenever $x \geq 2$ is sufficiently large. 
Let 
\begin{align*} 
A_{\Omega}^{(\ell)}(x) & := \sum_{k \leq \log\log x} (-1)^k \widehat{\pi}_k^{(\ell)}(x) \\ 
A_{\Omega}(x) & := \sum_{k \leq \log\log x} (-1)^k \widehat{\pi}_k(x). 
\end{align*} 
Then for all sufficiently large $x$, we have that 
$$|A_{\Omega}(x)| \gg |A_{\Omega}^{(\ell)}(x)|.$$ 
\end{lemma} 
\begin{proof} 
Given an explicit smooth lower bounding function, $\widehat{\pi}_k^{(\ell)}(x)$, we define the 
similarly smooth and monotone residual terms in approximating $\widehat{\pi}_k(x)$ 
using the following notation: 
\[
\widehat{\pi}_k(x) = \widehat{\pi}_k^{(\ell)}(x) + \widehat{E}_k(x). 
\]
Then we can form the ordinary exact form of the summatory function as 
\begin{align*} 
|A_{\Omega}(x)| & \gg \left\lvert \sum_{k \leq \frac{\log\log x}{2}} 
     \left[\widehat{\pi}_{2k}(x) - \widehat{\pi}_{2k-1}(x)\right] \right\rvert \\ 
     & \geq \left\lvert A_{\Omega}^{(\ell)}(x) - \sum_{k \leq \frac{\log\log x}{2}} 
     \left[\widehat{E}_{2k}(x) - \widehat{E}_{2k-1}(x)\right] 
     \right\rvert \\ 
     & \geq 
     \left\lvert A_{\Omega}^{(\ell)}(x) \right\rvert - 
     \left\lvert \sum_{k \leq \frac{\log\log x}{2}} 
     \left[\widehat{E}_{2k}(x) - \widehat{E}_{2k-1}(x)\right]
     \right\rvert. 
\end{align*} 
If the latter sum, denoted 
$$\operatorname{ES}(x) := \left\lvert \sum_{k \leq \frac{\log\log x}{2}} 
   \left[\widehat{E}_{2k}(x) - \widehat{E}_{2k-1}(x)\right] \right\rvert \rightarrow \infty,$$ as 
$x \rightarrow \infty$, then we can always find some absolute $C_0 > 0$ (by monotonicity) such that 
$\operatorname{ES}(x) \leq C_0 \cdot A_{\Omega}(x)$: 
\begin{align*} 
\operatorname{ES}(x) & = \left\lvert A_{\Omega}(x) - A_{\Omega}^{(\ell)}(x) \right\rvert 
     \leq \left\lvert |A_{\Omega}(x)| + 
     \left\lvert A_{\Omega}^{(\ell)}(x) \right\rvert \right\rvert 
     \ll 2 \left\lvert A_{\Omega}(x) \right\rvert. 
\end{align*} 
If on the other hand this sum becomes constant, or is bounded 
as $x \rightarrow +\infty$, then we also clearly have another absolute $C_1 > 0$ such that 
$|A_{\Omega}(x)| \geq C_1 \cdot |A_{\Omega}^{(\ell)}(x)|$. 
In either case, the claimed result holds for all large enough $x$. 
\end{proof} 

\begin{lemma} 
\label{lemma_CLT_and_AbelSummation} 
Suppose that $f(n)$ is an arithmetic functions 
such that $f(n) > 0$ for all $n > u_0$ where 
$f(n) \SuccSim \widehat{\tau}_{\ell}(n)$ as $n \rightarrow \infty$. Assume that 
the bounding function $\widehat{\tau}_{\ell}(t)$ is a non-negative 
continuously differentiable function of $t$ for all 
large enough $t \gg u_0$.  
We define the $\lambda$-sign-scaled summatory function of $f$ as follows: 
\[
F_{\lambda}(x) := \sum_{\substack{u_0 < n \leq x}} \lambda(n) \cdot f(n). 
\]
Let 
\begin{align*} 
A_{\Omega}^{(\ell)}(t) & := \sum_{k=1}^{\floor{\log\log t}} (-1)^k \widehat{\pi}_k^{(\ell)}(t), \\ 
A_{\Omega}(t) & := \sum_{k=1}^{\floor{\log\log t}} (-1)^k \widehat{\pi}_k(t), 
\end{align*} 
where $\widehat{\pi}_k(x) \geq \widehat{\pi}_k^{(\ell)}(x) \geq 0$ for 
$\widehat{\pi}_k^{(\ell)}(t)$ some smooth monotone 
function of $t$ at all sufficiently large $t \rightarrow \infty$. 
Then we have that 
\begin{equation} 
\label{eqn_Flambdax_RHA_AbelSummationFormula_v1} 
|F_{\lambda}(x)| \SuccSim \left\lvert 
     \left\lvert A_{\Omega}^{(\ell)}(x) \widehat{\tau}_{\ell}(x) \right\rvert - 
     \left\lvert \int_{u_0}^{x} A_{\Omega}^{(\ell)}(t) \widehat{\tau}_{\ell}^{\prime}(t) dt 
     \right\rvert \right\rvert.  
\end{equation} 
\end{lemma}
\begin{proof} 
We can form an accurate $C^{1}(\mathbb{R})$ approximation by the smoothness of 
$\widehat{\pi}_k^{(\ell)}(x)$ that allows us to apply the Abel summation formula using the summatory 
function $A_{\Omega}^{(\ell)}(t)$ for $t$ on any bounded connected subinterval of $[1, \infty)$. 
The stated lower bound formula for $F_{\lambda}(x)$ in 
\eqref{eqn_Flambdax_RHA_AbelSummationFormula_v1} 
above is valid by Abel summation and by 
applying Lemma \ref{lemma_lowerBoundsOnLambdaFuncParitySummFuncs}. 
In particular, whenever 
\[
0 \leq \left\lvert \frac{\displaystyle\sum\limits_{\log\log t < k \leq \frac{\log t}{\log 2}} 
     (-1)^k \widehat{\pi}_k(t)}{A_{\Omega}(t)}\right\rvert \ll 2, 
     \mathrm{\ as\ } t \rightarrow \infty. 
\]
the asymptotically dominant terms indicating the parity of 
$\lambda(n)$ are captured up to a constant factor 
by the terms in the range over $k$ summed by $A_{\Omega}(t)$ for 
sufficiently large $t \rightarrow \infty$. 
In other words, taking the sum over the summands that defines $A_{\Omega}(x)$ only over the truncated range of 
$k \in [1, \log\log x]$ does not non-trivially change the limiting asymptotically 
dominant terms in the lower bound obtained from using this form of the summatory function in 
conjunction with the claimed Abel summation formula. 
This property holds even when we should technically 
index over all $k \in [1, \log_2(x)]$ to obtain an exact formula for the summatory weight function.  
By Corollary \ref{theorem_MV_Thm7.20}, we have that 
the assertion above holds as $t \rightarrow \infty$. 

Secondly, observe that provided sufficiently smoothness (differentiability) of 
close approximations to $A_{\Omega}(t)$ (to $f(t)$) on $(u_0, x)$, we have that 
\begin{align*} 
|F_{\lambda}(x)| & \geq \left\lvert |A_{\Omega}(x) f(x)| - \int_{u_0}^{x} 
     |A_{\Omega}(t) f^{\prime}(t)| dt \right\rvert \\ 
     & \gg \left\lvert \left\lvert A_{\Omega}^{(\ell)} (x) \widehat{\tau}_{\ell}(x) \right\rvert - 
     \int_{u_0}^{x} 
     \left\lvert A_{\Omega}^{(\ell)}(t) \widehat{\tau}_{\ell}^{\prime}(t) 
     \right\rvert dt \right\rvert \\ 
     & \gg \left\lvert \left\lvert A_{\Omega}^{(\ell)} (x) \widehat{\tau}_{\ell}(x) \right\rvert - 
     \left\lvert \int_{u_0}^{x} 
     A_{\Omega}^{(\ell)}(t) \widehat{\tau}_{\ell}^{\prime}(t) dt \right\rvert \right\rvert. 
\end{align*} 
The previous equations follow from the ordinary Abel summation method by 
applying the argument in 
Lemma \ref{lemma_lowerBoundsOnLambdaFuncParitySummFuncs} and using the triangle inequality. 
\end{proof} 

\begin{cor} 
\label{cor_ASemiForm_ForGInvx_v1} 
We have that for almost every sufficiently large $x$, that as $x \rightarrow \infty$ 
\begin{align*} 
\left\lvert G_E^{-1}(x) \right\rvert & \SuccSim 
     \frac{2\sqrt{2} e \log 2}{3\pi} \times 
     \frac{(\log x)^{\frac{3}{2}} (\log\log x)}{(\log\log x) \sqrt{\log\log\log x}} \times 
     \left\lvert \sum_{e < d \leq \log x} 
     \frac{\lambda(d) (\log d)^{\frac{1}{4}}}{d \cdot \log\log d} 
     \right\rvert. 
\end{align*} 
\end{cor} 
\NBRef{A10-2020.04-26} 
\begin{proof} 
Using the definition in \eqref{eqn_GEInvxSummatoryFuncDef_v1}, we obtain on average that\footnote{ 
     For any arithmetic functions $f,h$, we have that \cite[\cf \S 3.10; \S 3.12]{APOSTOLANUMT} 
     \[
     \sum_{n \leq x} h(n) \times \sum_{d|n} f(d) = \sum_{d \leq x} f(d) \times \sum_{n=1}^{\Floor{x}{d}} h(dn). 
     \] 
}
\begin{align*} 
\left\lvert G_E^{-1}(x) \right\rvert & = 
     \left\lvert \sum_{n \leq (\log x)^{\frac{3}{2}} (\log\log x)} \lambda(n) \times 
     \sum_{\substack{d|n \\ d > e}} \frac{\lambda(d) (\log d)^{\frac{1}{4}}}{\log\log d} \right\rvert \\ 
     & = \left\lvert \sum_{e < d \leq (\log x)^{\frac{3}{2}} (\log\log x)} 
     \frac{(\log d)^{\frac{1}{4}}}{\log\log d} \times 
     \sum_{n=1}^{\Floor{\log x}{d}} \lambda(dn) \right\rvert. 
\end{align*} 
We see that by complete additivity of $\Omega(n)$ 
(complete multiplicativity of $\lambda(n)$) that 
\begin{align*} 
\sum_{n=1}^{\Floor{x}{d}} \lambda(dn) & = \sum_{n=1}^{\Floor{x}{d}} \lambda(d) \times \lambda(n) 
     = \lambda(d) \times \sum_{n \leq \Floor{x}{d}} \lambda(n). 
\end{align*} 
Now using Theorem \ref{theorem_GFs_SymmFuncs_SumsOfRecipOfPowsOfPrimes} and 
Lemma \ref{lemma_lowerBoundsOnLambdaFuncParitySummFuncs}, 
we can establish that 
\begin{align} 
\label{eqn_proof_tag_GEInvxLowerBound_v1} 
\left\lvert \sum_{n \leq x} \lambda(n) \right\rvert & \gg 
     \left\lvert \sum_{k \leq \log\log x} (-1)^k \cdot \widehat{\pi}_k(x) \right\rvert \\ 
\notag 
     & \gg \frac{2\sqrt{2} e \log 2}{3\pi} \cdot 
     \frac{x}{(\log x) \sqrt{\log\log x}} 
     =: \widehat{L}_0(x). 
\end{align} 
The sign of the sum obtained by taking the right-hand-side of 
\eqref{eqn_proof_tag_GEInvxLowerBound_v1} without the 
absolute value operation is given by $(-1)^{\floor{\log\log x}}$. 
The precise formula for the 
limiting lower bound stated above for $\widehat{L}_0(x)$ is computed by symbolic summation 
in \emph{Mathematica} using the new bounds on $\widehat{\pi}_k(x)$ guaranteed by 
the theorem, and then by applying subsequent standard asymptotic estimates to the 
resulting formulas for large $x \rightarrow \infty$, e.g., 
in the form of \eqref{eqn_IncompleteGamma_PropB} and Stirling's formula. 
It follows that 
\begin{align} 
\label{eqn_proof_tag_GEInvxLowerBound_v2} 
|G_E^{-1}(x)| & \gg \left\lvert \sum_{e < d \leq (\log x)^{\frac{3}{2}} (\log\log x)} 
     \frac{\lambda(d) (\log d)^{\frac{1}{4}}}{\log\log d} \times 
     (-1)^{\floor{\log\log\left(\frac{(\log x)^{\frac{3}{2}} (\log\log x)}{d}\right)}} \cdot 
     \widehat{L}_0\left(\frac{(\log x)^{\frac{3}{2}} (\log\log x)}{d}\right) \right\rvert. 
\end{align} 
\textbf{Outline for the remainder of the proof.} 
We sketch the following core sections remaining to prove our claimed lower bound on 
$|G_E^{-1}(x)|$: 
\begin{itemize}[itemsep=0pt,topsep=4pt,leftmargin=0.75in] 
\item[\textbf{(A)}] We identify an initial subinterval of our full bounds 
     on the summation defined by 
     \eqref{eqn_GEInvxSummatoryFuncDef_v1}. 
     On this subinterval we prove that we can expect 
     constant sign term contributions resulting from the inputs to the function $\widehat{L}_0$ 
     involving (a priori) both $d,x$ for $x$ large and $d$ on this subinterval. 
     This consideration keeps the sign imparted by $\lambda(d)$ intact 
     in the resulting formula. 
     What we are looking for here is a method to discard the local 
     signedness from the otherwise 
     easily bounded function $\widehat{L}_0$ evaluated as the bivariate 
     function of $d,x$ from the equation above. 
\item[\textbf{(B)}] We then factor out easily bounded terms from the expansion of the 
     monotone $\widehat{L}_0$ on this interval. 
\item[\textbf{(C)}] We define and determine additional characteristic formulas we will 
     refer to in later sections for the resulting lower bounds that are formed by 
     restricting the range of $d$ in 
     \eqref{eqn_proof_tag_GEInvxLowerBound_v2} 
     to just this initial range. 
\item[\textbf{(D)}] Finally, being absolutely rigorous and careful with this approach, 
     we must argue precisely 
     that the oscillatory, signed terms from the upper end of the deleted interval 
     cannot generate trivial bounds by cancellation with the stated lower bounds. 
\end{itemize} 
The arguments used to establish the form of the lower bounds stated in this 
corollary are longer, and are certainly more involved than the 
proofs of our previous results given in this section so far. 
Since further disassembly of these distinct parts to proving the stated bound into 
smaller external lemmas makes 
the overall logic to the article harder to interpret, 
we will continue to label subsections of the remaining proof components 
corresponding to the headings itemized as above to maintain clarity. \\ 
\textbf{Part A.} 
We will simplify \eqref{eqn_proof_tag_GEInvxLowerBound_v2} using an appeal to accessible contiguous 
ranges of consecutive integers over which we obtain 
effectively constant sign contributions from the 
function $\widehat{L}_0((\log x)^{\frac{3}{2}} (\log\log x) / d)$ as a function of both $x,d$. 
An initial contiguous interval is not difficult to extract for large $x$, though for 
general $d \in \left(e, (\log x)^{\frac{3}{2}} (\log\log x)\right)$, the 
sign contributions from this weight function are muddled 
by a dual dependence on the fluctuations of both the fractional part of 
logarithmic functions of $x$ and 
on the precise location of $d$ within the interval. 
The idea is to identify this initial accesible interval case, and then prove that we can 
form a lower bound on $G_E^{-1}(x)$ by truncating and summing only over the $d$ in this range. 

In particular, consider that 
\begin{align*} 
\log\log\left(\frac{(\log x)^{\frac{3}{2}} (\log\log x)}{d}\right) & = 
     \log\log\left((\log x)^{\frac{3}{2}} (\log\log x)\right) \\ 
     & \phantom{=\ } + \log\left(1 - 
     \frac{\log d}{(\log x)^{\frac{3}{2}} (\log\log x) \log\left( 
     (\log x)^{\frac{3}{2}} (\log\log x)\right)}\right), 
     \mathrm{\ as\ } x \rightarrow \infty. 
\end{align*} 
If we take $d \in (e, \log x] =: \mathcal{R}_x$, we have that 
$$\frac{\log d}{(\log x)^{\frac{3}{2}} (\log\log x) \log\left( 
 (\log x)^{\frac{3}{2}} (\log\log x)\right)} = o(1) \rightarrow 0,$$ 
as $x \rightarrow \infty$. 
So it stands to reason that for $d$ taken within $\mathcal{R}_x$, 
we expect that for almost every $x$ there are at most 
a handful of negligible cases of comparitively small order $d \leq d_0(x)$ such that 
\[
\floor{\log\log\left(\frac{(\log x)^{\frac{3}{2}} (\log\log x)}{d}\right)} \sim 
     \floor{\log\log\left((\log x)^{\frac{3}{2}} (\log\log x)\right) + o(1)}, 
\]
changes in parity transitioning from $d_0(x)-1$ to $d_0(x)$. 
An argument making this assertion precise brings leads us to 
two primary cases that rely on the distribution of the fractional parts 
of $\left\{(\log x)^{\frac{3}{2}} (\log\log x)\right\}$ within $[0, 1)$ for 
large integers $x \rightarrow \infty$. 
Consider the following points justifying that we obtain the desired (almost) 
constant sign property for all sufficiently large fixed $x$ and any 
$\log d \in \mathcal{R}_x$: 
\begin{itemize}[itemsep=0pt,topsep=0pt,leftmargin=0.35in] 
\item[\textbf{(1)}] If the fractional part 
     $\left\{\log\log\left((\log x)^{\frac{3}{2}} (\log\log x)\right)\right\} = 0$, then 
     \begin{align*} 
     \floor{\log\log\left(\frac{(\log x)^{\frac{3}{2}} (\log\log x)}{d}\right)} & = 
          \floor{\log\log\left((\log x)^{\frac{3}{2}} (\log\log x)\right)} \\ 
          & \phantom{=\ } + 
          \floor{-\frac{\log d}{(\log x)^{\frac{3}{2}} (\log\log x) \log\left( 
          (\log x)^{\frac{3}{2}} (\log\log x)\right)}}. 
     \end{align*} 
     This implies that provided that 
     \[
     -1 \leq -\frac{\log d}{(\log x)^{\frac{3}{2}} (\log\log x) \log\left( 
          (\log x)^{\frac{3}{2}} (\log\log x)\right)} < 0, 
     \]
     we obtain a constant sign term for 
     $\operatorname{sgn}\left[\widehat{L}_0\left(\frac{(\log x)^{\frac{3}{2}} (\log\log x)}{d}\right)\right]$. 
     Since $d$ is positive and maximized at $\log x$, 
     this condition clearly happens for all sufficiently large $x$. 
\item[\textbf{(2)}] If the fractional part $\{\log\log\log\log x\} \in (0, 1)$, then 
     \begin{align*} 
     & \floor{\log\log\left(\frac{(\log x)^{\frac{3}{2}} (\log\log x)}{d}\right)} = 
          \floor{\log\log\left((\log x)^{\frac{3}{2}} (\log\log x)\right)} \\ 
          & \phantom{\qquad =\ } + 
          \floor{\left\{\log\log\left((\log x)^{\frac{3}{2}} (\log\log x)\right)\right\} - 
          \frac{\log d}{(\log x)^{\frac{3}{2}} (\log\log x) \log\left( 
          (\log x)^{\frac{3}{2}} (\log\log x)\right)}}. 
     \end{align*} 
     Let the shorthands $f_x := \left\{\log\log\left((\log x)^{\frac{3}{2}} (\log\log x)\right)\right\}$ and 
     $\mathcal{B}(x) := (\log x)^{\frac{3}{2}} (\log\log x) \log\left((\log x)^{\frac{3}{2}} (\log\log x)\right)$. 
     We require that 
     \begin{align*} 
     -1 & \leq f_x - \frac{\log d}{\mathcal{B}(x)} < 0 \iff 
          (1 + f_x) \cdot \mathcal{B}(x) \geq \log d > 0, 
     \end{align*} 
     which is similarly clearly attained as $x \rightarrow \infty$. 
\end{itemize} 
In either case, we obtain the constant sign term on the contribution from 
$\widehat{L}_0$ for $d$ on this subinterval, $\mathcal{R}_x$. \\ 
\textbf{Part B.} 
Then provided that the sign term involving both $d$ and $x$ 
from \eqref{eqn_proof_tag_GEInvxLowerBound_v2} does not change for $d$ within our new interval, 
$\mathcal{R}_x$, 
we can factor out the dependence of the sign on the monotonically 
decreasing function 
$\widehat{L}_0\left((\log x)^{\frac{3}{2}} (\log\log x)/d\right)$ 
in the variable $d$ as we sum along the lower interval $\mathcal{R}_x$. 
We can see that this function is decreasing 
for $d \in \mathcal{R}_x$ by computing its partial derivative with respect to $d$ and 
evaluating the asymptotically dominant terms with leading negative sign as 
$x \rightarrow \infty$. 
So we determine that we should select $d := \log x$ in 
\eqref{eqn_proof_tag_GEInvxLowerBound_v2} to 
obtain a global lower bound on $|G_E^{-1}(x)|$ if we truncate the sum 
defined by \eqref{eqn_GEInvxSummatoryFuncDef_v1} to include only the 
indices $d \in \mathcal{R}_x$. \\ 
\textbf{Part C.} 
Let the magnitudes of the oscillatory remainder term sums be 
defined for all sufficiently large $x$ by 
\[
R_E(x) := \left\lvert \sum_{\log x < d < \frac{(\log x)^{\frac{3}{2}} (\log\log x)}{e}} 
     \frac{\lambda(d) (\log d)^{\frac{1}{4}}}{\log\log d} \times 
     (-1)^{\floor{\log\log\left(\frac{(\log x)^{\frac{3}{2}} (\log\log x)}{d}\right)}} \cdot 
     \widehat{L}_0\left(\frac{(\log x)^{\frac{3}{2}} (\log\log x)}{d}\right) \right\rvert. 
\]
Next, let the function $T_E(x)$ correspond to the 
easily factored dependence of the less simply integrable factors 
in $\widehat{L}_0$ when we set $d := \log x$. 
It is defined for all large enough $x$ as 
\begin{equation} 
\label{eqn_proof_tag_TExFuncDefAndBounds_v1} 
T_E(x) := \frac{1}{\log\left[(\log x)^{\frac{3}{2}}\right] 
     \sqrt{\log\log\left[(\log x)^{\frac{3}{2}}\right]}} \gg 
     \frac{1}{(\log\log x) \sqrt{\log\log\log x}}. 
\end{equation} 
Then, as we argued before, we see that as $x \rightarrow \infty$ 
\begin{align} 
\notag 
S_{E,1}(x) & := \left\lvert \sum_{e < d \leq (\log x)^{\frac{3}{2}} (\log\log x)} 
     \frac{\lambda(d) (\log d)^{\frac{1}{4}}}{\log\log d} \times 
     (-1)^{\floor{\log\log\left(\frac{(\log x)^{\frac{3}{2}} (\log\log x)}{d}\right)}} 
     \widehat{L}_0\left(\frac{(\log x)^{\frac{3}{2}} (\log\log x)}{d}\right) 
     \right\rvert \\ 
\label{eqn_proof_tag_SE1xFuncExp_v1} 
     & \gg \frac{2\sqrt{2} e \log 2}{3\pi} \times (\log x)^{\frac{3}{2}} (\log\log x) 
     T_E(x) \times 
     \left\lvert \sum_{e < d \leq \log x} 
     \frac{\lambda(d) (\log d)^{\frac{1}{4}}}{d \cdot \log\log d} 
     \right\rvert \\ 
\notag 
     & \gg \frac{2\sqrt{2} e \log 2}{3\pi} \times (\log x)^{\frac{3}{2}} (\log\log x) 
     T_E(x) \times 
     \left\lvert A_{\Omega}^{(\ell)}(\log x) \widehat{\tau}_0(\log x) - 
     \int_e^{\log x} A_{\Omega}^{(\ell)}(t) \widehat{\tau}_0^{\prime}(t) dt 
     \right\rvert, 
\end{align} 
where we select the functions $\widehat{\tau}_0(t) := \frac{(\log t)^{1/4}}{t \cdot \log\log t}$ and 
$-\widehat{\tau}^{\prime}_0(t) \gg \frac{(\log t)^{1/4}}{t^2 \cdot \log\log t}$ in the notation of 
Lemma \ref{lemma_CLT_and_AbelSummation}. 

What we then obtain from 
\eqref{eqn_proof_tag_GEInvxLowerBound_v2} and \eqref{eqn_proof_tag_SE1xFuncExp_v1} 
is the following lower bound by the triangle inequality 
that holds for all sufficiently large $x$: 
\begin{align} 
\label{eqn_proof_tag_GEInvxLowerBound_v3}
|G_E^{-1}(x)| & \gg 
     \Biggl\lvert S_{E,1}(x) - R_E(x) \Biggr\rvert \gg S_{E,1}(x), \mathrm{\ as\ } 
     x \rightarrow \infty. 
\end{align} 
We have claimed that in fact we can drop the sum terms over upper range of $d$ and still 
obtain the asymptotic lower bound on $|G_E^{-1}(x)|$ as $x \rightarrow \infty$ on the 
right-hand-side of \eqref{eqn_proof_tag_GEInvxLowerBound_v3}. 
To justify this step in the proof, 
we will provide limiting lower bounds on $R_E(x)$ that show that the 
contribution from these terms in absolute value exceeds the magnitude of the 
corresponding sums over $d \in \mathcal{R}_x$ when $x$ is large. \\ 
\textbf{Part D.} 
In Theorem \ref{theorem_gInv_GeneralAsymptoticsForms} stated in the next section below, 
we prove lower bounds on the sums we used to 
define $S_{E,1}(x)$ above of the form 
\[
S_{E,1}(x) \gg \frac{(\log x)^{\frac{3}{2}}}{2 \cdot  
     (\log\log x)^{\frac{3}{4}} (\log\log\log x)^2}, 
\]
where the lower bounds on the right-hand-side of the previous equation are clearly 
$o\left((\log x)^{\frac{3}{2}}\right)$, though still grows 
without bound as $x \rightarrow \infty$. 
In contrast, we can bound from below to show that the contribution from 
$R_E(x)$ is at least on the order of a constant 
times $(\log x)^{\frac{3}{2}}$. To obtain this lower bound, consider 
that since $\frac{(\log d)^{\frac{1}{4}}}{d \cdot \log\log d}$ 
is monotone decreasing for all large enough $d > e$, we obtain the smallest possible magnitude on the sum 
by alternating signs on consecutive terms in the sum. 
We can then bound the sum as $x \rightarrow \infty$ by 
\begin{align*} 
\frac{R_E(x)}{(\log x)^{\frac{3}{2}} (\log\log x)} & \gg 
     \left\lvert o(1) + 
     \sum_{\log x < d < \frac{(\log x)^{\frac{3}{2}} (\log\log x)}{2e}} 
     \frac{\log(2d)^{1/4}}{2d \cdot \log\log(2d)} - \frac{\log(2d+1)^{1/4}}{(2d+1) \log\log(2d+1)} 
     \right\rvert \\ 
     & \sim \left\lvert o(1) + 
     \sum_{\log x < d < \frac{(\log x)^{\frac{3}{2}} (\log\log x)}{2e}} 
     \frac{\log(2d)^{1/4}}{2d \cdot \log\log(2d)} - 
     \frac{1}{(2d+1)} \frac{\left(\log(2d) + \frac{1}{2d}\right)^{1/4}}{\left( 
     \log\log(2d) + \frac{1}{2d \cdot \log(2d)}\right)} 
     \right\rvert \\ 
     & \approx \left\lvert o(1) + 
     \sum_{\log x < d < \frac{(\log x)^{\frac{3}{2}} (\log\log x)}{2e}} 
     \frac{\log(2d)^{1/4}}{\log\log(2d)} \left[
     \frac{1}{2d} - 
     \frac{\left(1 + \frac{1}{2d \cdot \log(2d)}\right)^{1/4}}{ 
     (2d+1) \left(1 + \frac{1}{2d \cdot \log(2d) \log\log(2d)}\right)} 
     \right] \right\rvert. 
\end{align*} 
Then by an appeal to binomial and geometric series expansions, we obtain that the significant 
terms in the above sum are given by\footnote{ 
     In particular, with $|2d \log(2d)|^{-1}, |2d \log(2d) \log\log(2d)|^{-1} < 1$ we can compute that 
     \begin{align*} 
     \left(1 + \frac{1}{2d \cdot \log(2d)}\right)^{\frac{1}{4}} \times 
          \left(1 + \frac{1}{2d \cdot \log(2d) \cdot \log\log(2d)}\right)^{-1} & = 
          1 + \frac{1}{8d \cdot \log(2d)} - \frac{1}{2d \cdot \log(2d) \cdot \log\log(2d)} + 
          O\left(\frac{1}{d^2}\right). 
     \end{align*} 
}
\begin{align*} 
\frac{R_E(x)}{(\log x)^{\frac{3}{2}} (\log\log x)} & \gg 
     %\Biggl\lvert o(1) + 
     %\sum_{\log x < d < \frac{(\log x)^{\frac{3}{2}} (\log\log x)}{2e}} 
     %\frac{\log(2d)^{1/4}}{\log\log(2d)} \Biggl[ 
     %\frac{1}{2d} \\ 
     %& \phantom{\gg\Biggl\lvert o(1) + \sum_{\log x < d < \frac{(\log x)^{\frac{3}{2}} (\log\log x)}{2e}}\ } - 
     %\frac{1}{(2d+1)} \left(1 + \frac{1}{8d \cdot \log(2d)} - 
     %\frac{1}{2d \cdot \log(2d) \log\log(2d)}\right) 
     %\Biggr] \Biggr\rvert \\ 
     %& = 
     \left\lvert o(1) + 
     \sum_{\log x < d < \frac{(\log x)^{\frac{3}{2}} (\log\log x)}{2e}} 
     O\left(\frac{\log(2d)^{1/4}}{2d (2d+1) \cdot \log\log(2d)}\right) \right\rvert = 
     O\left(1\right). 
\end{align*} 
What we obtain from the previous several caclulations 
is that the magnitude of $R_E(x)$ always exceeds that of the lower 
bound we establish in Theorem \ref{theorem_gInv_GeneralAsymptoticsForms} for the 
sums over $d \in \mathcal{R}_x$ as $x \rightarrow \infty$. 
In total, we obtain the lower bounds on $G_E(x)$ that correspond to the 
smaller order terms resulting from the first summation ranges above to be bounded by the 
functions stated in Theorem \ref{theorem_gInv_GeneralAsymptoticsForms} below. 
\end{proof} 

\subsubsection{The proof of a central lower bound on the magnitude of $G_{E}^{-1}(x)$} 

The next central theorem is the last key barrier required to prove 
Corollary \ref{cor_ThePipeDreamResult_v1} 
in the next subsection. 
Combined with Theorem \ref{theorem_GInvxLowerBoundByGEInvx_v1} 
proved in the last section, the new lower bounds we establish below provide us 
with a sufficient mechanism to bound the formula from 
Proposition \ref{prop_Mx_SBP_IntegralFormula}. 
Since these lower bounds tend to $+\infty$ as $x \rightarrow \infty$ (along an 
infinite subsequence of positive integers), this is a sufficient condition to 
guarantee the unboundedness of the scaled Mertens function of the form 
claimed in the corollary. 

\begin{theorem}[Asymptotics and bounds for the summatory function $G^{-1}(x)$] 
\label{theorem_gInv_GeneralAsymptoticsForms}
We define a lower summatory function, $G_{\ell}^{-1}(x)$, 
to provide bounds on the magnitude of $G_E^{-1}(x)$ such that 
$$|G_E^{-1}(x)| \gg |G_{\ell}^{-1}(x)|,$$ 
for all sufficiently large $x > e$. 
Let $C_{\ell,1} > 0$ be the absolute constant defined by 
\[
C_{\ell,1} = \frac{8 e^2 \log^2(2)}{9 \pi^2} \approx 0.319733.  
\]
We obtain the following limiting estimate for the bounding function 
$G_{\ell}^{-1}(x)$ as $x \rightarrow \infty$:   
\begin{align*} 
 & \left\lvert G_{\ell}^{-1}\left(x\right) \right\rvert
     \SuccSim 
     \frac{C_{\ell,1} \cdot (\log x)^{\frac{3}{2}}}{2 \cdot  
     (\log\log x)^{\frac{3}{4}} (\log\log\log x)^2}. 
\end{align*} 
\end{theorem} 
\NBRef{A10-2020.04-26} 
\begin{proof} 
Recall from our proof of Corollary \ref{cor_BoundsOnGz_FromMVBook_initial_stmt_v1} that 
a lower bound on the variant prime form counting function, $\widehat{\pi}_k(x)$, is given by 
\[
\widehat{\pi}_k(x) \SuccSim \frac{4}{3\sqrt{\pi}} \frac{x}{\log x} \left(\frac{\log 2}{\log x}\right)^{ 
     \frac{k-1}{\log\log x}} \frac{(\log\log x)^{k-1}}{(k-1)!} \left( 
     1 + O\left(\frac{k}{(\log\log x)^2}\right) 
     \right), \mathrm{\ as\ } x \rightarrow \infty. 
\]
We can then form a lower summatory function indicating the signed contributions over the distinct 
parity of $\Omega(n)$ for all $n \leq x$ as follows by applying 
\eqref{eqn_IncompleteGamma_PropA} and Stirling's approximation as already noted in the 
proof of Corollary \ref{cor_ASemiForm_ForGInvx_v1} given above: 
\begin{align} 
\notag 
\left\lvert A_{\Omega}^{(\ell)}(t) \right\rvert & = 
     \left\lvert \sum_{k \leq \log\log t} (-1)^k \widehat{\pi}_k(t) \right\rvert \\ 
\notag 
     & \SuccSim  
     \frac{2\sqrt{2} e \log 2}{3\pi} \cdot 
     \frac{t}{(\log t) \sqrt{\log\log t}} \left( 
     1 + O\left(\frac{1}{\log\log t}\right) 
     \right) \\ 
\label{proof_thm_GInvFunc_v0} 
     & \gg \frac{2\sqrt{2} e \log 2}{3\pi} \cdot 
     \frac{t}{(\log t) \sqrt{\log\log t}}, 
     \mathrm{\ as\ } t \rightarrow \infty. 
\end{align} 
The actual sign on this function is given by 
$\operatorname{sgn}(A_{\Omega}^{(\ell)}(t)) = (-1)^{\floor{\log\log t}}$ 
(see Lemma \ref{lemma_lowerBoundsOnLambdaFuncParitySummFuncs}). 
By Lemma \ref{lemma_CLT_and_AbelSummation}
we know that this summatory function forms a lower bound in absolute value for the 
actual weight of the signed terms indicated by $\lambda(n)$. 

As we determined in \eqref{eqn_proof_tag_SE1xFuncExp_v1} from the proof of 
Corollary \ref{cor_ASemiForm_ForGInvx_v1}, we take the function 
$\widehat{\tau}_0(t) = \frac{(\log t)^{1/4}}{t \cdot \log\log t}$ that satisfies 
\begin{align*} 
-\widehat{\tau}_0^{\prime}(t) & = -\frac{d}{dt}\left[ 
     \frac{(\log t)^{\frac{1}{4}}}{t \cdot \log\log t} 
     \right] \SuccSim \frac{(\log t)^{1/4}}{t^2 \cdot \log\log t}. 
\end{align*} 
Moreover, we have using the notation from the proof above that we can select 
the initial form of the lower bound function $G_{\ell}^{-1}(x)$ to be defined as follows: 
\begin{equation} 
\label{proof_thm_GInvFunc_v1} 
G_{\ell}^{-1}(x) := \frac{2\sqrt{2} e \log 2}{3\pi} \cdot 
     (\log x)^{\frac{3}{2}} (\log\log x) \cdot 
     T_E(x) \times 
     \left\lvert A_{\Omega}^{(\ell)}(\log x) \widehat{\tau}_0(\log x) - 
     \int_e^{\log x} A_{\Omega}^{(\ell)}(t) \widehat{\tau}_0^{\prime}(t) dt 
     \right\rvert. 
\end{equation} 
The inner integral term on the rightmost side of \eqref{proof_thm_GInvFunc_v1} 
is summed approximately by splitting the terms weighted by 
$(-1)^{\floor{\log\log t}}$ in the form of\footnote{ 
     That is, we form the disjoint union of the range of integration into 
     subintervals along which the signedness of the integrands are constant 
     according to 
     \[
     \left\{e^e \leq t \leq (\log x)^{\frac{3}{2}} (\log\log x): 
          (-1)^{\floor{\log\log t}} = +1\right\} = 
          \left(\bigcup_{k=1}^{\frac{1}{2} (\log x)^{\frac{3}{2}} (\log\log x)} 
          \left[e^{e^{2k}}, e^{e^{2k+1}}\right)\right) \bigcup 
          \mathcal{S}_{0,+}, 
     \]
     where $|\mathcal{S}_{0,+}| \leq \frac{1}{2}$. We can similarly split the interval 
     of integration corresponding to the negatively biased terms on the 
     unsigned integrand functions for 
     $t \in \left[e^e, (\log x)^{\frac{3}{2}} (\log\log x)\right]$. 
} 
\begin{align} 
\notag 
\frac{2\sqrt{2} e \log 2}{3\pi} \times 
     \left\lvert \int_{e}^{\log x} A_{\Omega}^{(\ell)}(t)\widehat{\tau}_0^{\prime}(t) dt 
     \right\rvert & \gg \frac{2\sqrt{2} e \log 2}{3\pi} \times 
     \left\lvert \sum_{k=e+1}^{\frac{1}{2} \log\log\left[(\log x)^{\frac{3}{2}} (\log\log x)\right]} \left[ 
     I_{\ell}\left(e^{e^{2k+1}}\right) e^{e^{2k+1}} - 
     I_{\ell}\left(e^{e^{2k}}\right) e^{e^{2k}} 
     \right] \right\rvert \\ 
\label{eqn_proof_thm_GInvFunc_v3_approx} 
     & \gg 
     \frac{2\sqrt{2} e \log 2}{3\pi} \times 
     \left\lvert 
     \int_{\frac{1}{2} \log\log\left[(\log x)^{\frac{3}{2}} (\log\log x)\right]-\frac{1}{2}}^{ 
     \frac{1}{2} \log\log\left[(\log x)^{\frac{3}{2}} (\log\log x)\right]} 
     I_{\ell}\left(e^{e^{2k}}\right) 
     e^{e^{2k}} dk \right\rvert. 
\end{align} 
We express the integrand function, 
$$I_{\ell}(t) := \frac{2\sqrt{2} e \log 2}{3\pi} \times 
  \widehat{\tau}_0^{\prime}(t) A_{\Omega}^{(\ell)}(t),$$ 
defined implicitly as in \eqref{eqn_proof_thm_GInvFunc_v3_approx} as the following function of $k$: 
\begin{align} 
\label{eqn_proof_thm_GInvFunc_v3v2_approx} 
I_{\ell}\left(e^{e^{2k}}\right) e^{e^{2k}}& \SuccSim 
     \frac{2 e^2 \log^2(2)}{9 \pi^2} \cdot \frac{e^{-\frac{3k}{2}}}{k^{\frac{3}{2}}} 
     =: \widehat{I}_{\ell}(k). 
\end{align} 
So upon input of the upper bound on the range of integration in 
\eqref{eqn_proof_thm_GInvFunc_v3_approx}, at the point 
$k := \frac{\log\log\left[(\log x)^{\frac{3}{2}} (\log\log x)\right]}{2}$, 
we find from the mean value theorem with the monotone function 
from \eqref{eqn_proof_thm_GInvFunc_v3v2_approx} that 
\begin{align} 
\notag 
\frac{2\sqrt{2} e \log 2}{3\pi} & \times (\log x)^{\frac{3}{2}} (\log\log x) \times T_E(x) \times 
     \left\lvert 
     \int_{\frac{1}{2} \log\log\left[(\log x)^{\frac{3}{2}} (\log\log x)\right] - 
     \frac{1}{2}}^{\frac{1}{2} \log\log\left[(\log x)^{\frac{3}{2}} (\log\log x)\right]} 
     I_{\ell}\left(e^{e^{2k}}\right) 
     e^{e^{2k}} dk \right\rvert \\ 
\notag 
     & \gg \frac{2\sqrt{2} e \log 2}{3\pi} \times (\log x)^{\frac{3}{2}} (\log\log x) \times T_E(x) \times 
     \Biggl\lvert 
     \widehat{I}_{\ell}\left(\frac{1}{2} \log\log\left[(\log x)^{\frac{3}{2}} (\log\log x)\right]\right)
     \Biggr\rvert \\ 
\label{eqn_proof_thm_GInvFunc_v4_approx} 
     & \gg \frac{C_{\ell,1} \cdot (\log x)^{\frac{3}{2}}}{2 \cdot  
     (\log\log x)^{\frac{3}{4}} (\log\log\log x)^2}. 
\end{align} 
Similarly, by evaluating $\widehat{I}_{\ell}(t)$ at the 
lower bound on the integral above with 
$k := \frac{\log\log\left[(\log x)^{\frac{3}{2}} (\log\log x)\right] - 1}{2}$, we can similarly conclude that 
\begin{align} 
\notag 
\frac{2\sqrt{2} e \log 2}{3\pi} & \times (\log x)^{\frac{3}{2}} (\log\log x) \times T_E(x) \times 
     \left\lvert 
     \int_{\frac{1}{2} \log\log\left[(\log x)^{\frac{3}{2}} (\log\log x)\right] - 
     \frac{1}{2}}^{\frac{1}{2} \log\log\left[(\log x)^{\frac{3}{2}} (\log\log x)\right]} 
     I_{\ell}\left(e^{e^{2k}}\right) 
     e^{e^{2k}} dk \right\rvert \\ 
\label{eqn_proof_thm_GInvFunc_v4v2_approx} 
     & \ll 
     \frac{e^{3/4} \cdot C_{\ell,1} \cdot (\log x)^{\frac{3}{2}}}{2 \cdot  
     (\log\log x)^{\frac{3}{4}} (\log\log\log x)^2}. 
\end{align} 
To make it clear which terms in \eqref{proof_thm_GInvFunc_v1} 
the limiting lower bounds correspond to, consider the following expansion for the leading term in 
the Abel summation formula from \eqref{proof_thm_GInvFunc_v1} for comparison with 
\eqref{eqn_proof_thm_GInvFunc_v4_approx}: 
\begin{align} 
\notag 
\frac{2\sqrt{2} e \log 2}{3\pi} & \times (\log x)^{\frac{3}{2}} (\log\log x) \times T_E(x) \times 
     \left\lvert \widehat{\tau}_0(\log x) A_{\Omega}^{(\ell)}(\log x) \right\rvert \\ 
\label{eqn_proof_thm_GInvFunc_v5_approx} 
     & \gg \frac{C_{\ell,1} \cdot (\log x)^{\frac{3}{2}}}{ 
     (\log\log x)^{\frac{3}{4}} (\log\log\log x)^2}. 
\end{align} 
Hence, by 
Lemma \ref{lemma_lowerBoundsOnLambdaFuncParitySummFuncs} and the triangle inequality, 
we conclude that we can take $\left\lvert G_{\ell}^{-1}\left(x\right) \right\rvert$ 
bounded below by the term in 
\eqref{eqn_proof_thm_GInvFunc_v4_approx}. 
\end{proof} 

\begin{remark} 
What is key to observe about the distinct lower bounds obtained in the proof of the 
previous theorem 
is that each of them scaled by $(\log x)^{-1}$ 
monotonically increases without bound as $x \rightarrow \infty$. In particular, the remaining factor 
after rescaling dominates the asymptotics of the reciprocal powers of iterated logarithms. 
It is fortunate, and a sign of our correct calculations, that up to distinct constant factors, the 
asymptotic orders of each of 
\eqref{eqn_proof_thm_GInvFunc_v4_approx}, 
\eqref{eqn_proof_thm_GInvFunc_v4v2_approx} and 
\eqref{eqn_proof_thm_GInvFunc_v5_approx} 
match identically. We expect this correspondence to be somewhat of a rarity that still 
coincides in these cases even though one of these terms 
is formed by a product of component functions, where the other two correspond to 
distinct particular values of another separate product of related functions over 
which we perform a definite integral operation. 
\end{remark} 

\subsection{Proof of the unboundedness of the scaled Mertens function}
\label{subSection_TheCoreResultProof} 

We finally address the main conclusion of our arguments given so far with the 
following proof: 

\begin{proof}[Proof of Corollary \ref{cor_ThePipeDreamResult_v1}] 
\label{proofOf_cor_ThePipeDreamResult_v1} 
We break up the integral term in 
Proposition \ref{prop_Mx_SBP_IntegralFormula} 
over $t \in [u_0, x/2]$ into two pieces: one that is easily bounded 
from $u_0 \leq t \leq \sqrt{x}$, 
and then another that will conveniently give us our slow-growing tendency towards 
infinity along the subsequence when evaluated using 
Theorem \ref{theorem_gInv_GeneralAsymptoticsForms}. 
Given a fixed large infinitely tending $x$, we have some (at least one) point 
$x_0 \in \left[\sqrt{x}, \frac{x}{2}\right]$ defined such that 
$|G^{-1}(t)|$ is minimal and non-vanishing as 
\[
\left\lvert G^{-1}(x_0) \right\rvert := 
     \min_{\substack{\sqrt{x} \leq t \leq \frac{x}{2} \\ G^{-1}(t) \neq 0}} |G^{-1}(t)|. 
\]
We can then apply Proposition \ref{prop_Mx_SBP_IntegralFormula} to bound 
\begin{align} 
\notag 
\frac{|M(x)|}{\sqrt{x}} & = 
     \frac{1}{\sqrt{x}} \left\lvert G^{-1}(x) - x \cdot \int_1^{x/2} \frac{G^{-1}(t)}{ 
     t^2 \cdot \log(x/t)} dt \right\rvert \\ 
\label{eqn_MxGInvxLowerBound_stmt_v0} 
     & \gg 
     \left\lvert \left\lvert \frac{G^{-1}(x)}{\sqrt{x}} \right\rvert - \sqrt{x} \left\lvert 
     \int_1^{x/2} \frac{G^{-1}(t)}{ 
     t^2 \cdot \log(x/t)} dt \right\rvert \right\rvert \\ 
\notag 
     & \gg 
     \left\lvert \sqrt{x} \times \int_{\sqrt{x}}^{x/2} \frac{G^{-1}(t)}{ 
     t^2 \cdot \log(x/t)} dt \right\rvert \\ 
\notag 
     & \gg \left\lvert \int_{\sqrt{x_0}}^{\frac{x}{2}} \frac{2 \sqrt{x_0}}{ 
     t^2 \cdot \log\left(x_0\right)} dt \right\rvert 
     \times \left( 
     \min\limits_{\substack{\sqrt{x} \leq t \leq \frac{x}{2} \\ G^{-1}(t) \neq 0}} |G^{-1}(t)| 
     \right) \\ 
\label{eqn_MxGInvxLowerBound_stmt_v1} 
     & \gg  
     \frac{2 \left\lvert G^{-1}(x_0) \right\rvert}{\log\left(x_0\right)}. 
\end{align} 
To see the complete logic to the bound we arrived at in \eqref{eqn_MxGInvxLowerBound_stmt_v1}, 
observe that the difference of terms we have in 
\eqref{eqn_MxGInvxLowerBound_stmt_v0} is dominated first by 
our result in \eqref{eqn_proof_tag_GInvnLowerBound_stmt_v1} from the proof of 
Theorem \ref{theorem_GInvxLowerBoundByGEInvx_v1} as 
\[
\frac{|G^{-1}(x)|}{\sqrt{x}} \gg \frac{6\sqrt{x}}{\pi^2} (\log x) (\log\log x), 
     \mathrm{\ for\ a.e.\ } x \rightarrow \infty, 
\]
and second, by considering that by the mean value theorem, for 
some $c_0 \in [1, \sqrt{x}]$ and $c_1 \in \left[\sqrt{x}, \frac{x}{2}\right]$ we have 
\begin{align*} 
\sqrt{x} & \left\lvert \int_1^{x/2} \frac{G^{-1}(t)}{t^2 \cdot \log(x/t)} dt \right\rvert \\ 
     & \gg \left\lvert \frac{\sqrt{x} \cdot G^{-1}(c_0)}{c_0} \int_1^{\sqrt{x}} \frac{dt}{t \log(x/t)} + 
     \sqrt{x} \cdot G^{-1}(c_1) \int_{\sqrt{x}}^{x/2} \frac{dt}{t^2 \log(x)} \right\rvert \\ 
     & \gg \left\lvert \left(\min\limits_{\substack{1 \leq c \leq \sqrt{x} \\ G^{-1}(c) \neq 0}} 
     G^{-1}(c)\right) \log\log x + 
     \left(\min\limits_{\substack{\sqrt{x} \leq c \leq \frac{x}{2} \\ G^{-1}(c) \neq 0}} 
     G^{-1}(c)\right) \left(\frac{1}{\log x} + o\left(\frac{1}{\log x}\right)\right) 
     \right\rvert. 
\end{align*} 
Then by Theorem \ref{theorem_GInvxLowerBoundByGEInvx_v1} proved in 
Section \ref{Section_ProofOfValidityOfAverageOrderLowerBounds}, the result in 
\eqref{eqn_MxGInvxLowerBound_stmt_v1} implies that 
\begin{align} 
\label{eqn_MxGInvxLowerBound_stmt_v2} 
\frac{|M(x)|}{\sqrt{x}} & \gg \frac{2 \left\lvert G_{E}^{-1}(x_0) \right\rvert}{\log\left(x_0\right)}. 
\end{align} 
Define the infinite increasing subsequence, 
$\{x_{0,y}\}_{y \geq Y_0}$, by $x_{0,y} := e^{2e^{e^{2y}}}$ for some sufficiently 
large finite integer $Y_0 \gg 1$. 
When we assume that $x \mapsto x_{0,y}$ is taken along this subsequence, 
we can transform the bound in the last 
equation into a statement about a lower bound for $|M(x)| \log x / \sqrt{x}$ 
along an infinitely tending subsequence in the following form by 
applying Theorem \ref{theorem_gInv_GeneralAsymptoticsForms} to 
\eqref{eqn_MxGInvxLowerBound_stmt_v2}: 
\begin{align} 
\label{eqn_MxGInvxLowerBound_stmt_v3} 
\frac{|M(x_{0,y})|}{\sqrt{x_{0,y}}} & \gg 
     \frac{C_{\ell,1} \cdot (\log \sqrt{x_{0,y}})^{\frac{1}{2}}}{ 
     (\log\log \sqrt{x_{0,y}})^{\frac{3}{4}} (\log\log\log \sqrt{x_{0,y}})^2}, 
     \mathrm{\ as\ } y \rightarrow \infty. 
\end{align} 
Notice by a small, but insightful, 
technicality in stating \eqref{eqn_MxGInvxLowerBound_stmt_v3}, 
we are not actually asserting that 
$|M(x)| \log x / \sqrt{x}$ grows unbounded along the precise subsequence of 
$x \mapsto x_{0,y}$. Rather, we are asserting that the unboundedness of this function 
can be witnessed along some subsequence whose points are taken within an interval as 
$\hat{x}_{0,y} \in \left[\sqrt{x_{0,y}}, \frac{x_{0,y}}{2}\right]$. 
We choose to state the lower bound given on the right-hand-side of 
\eqref{eqn_MxGInvxLowerBound_stmt_v3} using the 
monotonicity of the lower bound on $|G_E^{-1}(x)|$ we proved in 
Theorem \ref{theorem_gInv_GeneralAsymptoticsForms} without the need for a conditionally 
defined asymptotic growth rate. 
We also can verify that for sufficiently large $y \rightarrow \infty$, this infinitely 
tending subsequence is well defined as $\hat{x}_{0,y+1} > \hat{x}_{0,y}$ for all 
sufficiently large $y \geq Y_0$. 

Finally, we evaluate the following limit to show unboundedness: 
\[
\lim_{x \rightarrow \infty} \left[\frac{(\log x)^{\frac{1}{2}}}{ 
     (\log\log x)^{\frac{3}{4}} (\log\log\log x)^2}  
     \right] = +\infty. 
\]
The scaled Mertens function is then 
unbounded in the limit supremum sense, as we have claimed, since the right-hand-side of 
\eqref{eqn_MxGInvxLowerBound_stmt_v3} tends to positive infinity as 
$x_{0,y} \rightarrow \infty$, or equivalently as $y \rightarrow \infty$. 
\end{proof} 

\newpage 
\renewcommand{\refname}{References} 
\bibliography{glossaries-bibtex/thesis-references}{}
\bibliographystyle{plain}

\newpage
\setcounter{section}{0} 
\renewcommand{\thesection}{T.\arabic{section}} 

\section{Table: The Dirichlet inverse function $g^{-1}(n)$ and the 
         distribution of its summatory function} 
\label{table_conjecture_Mertens_ginvSeq_approx_values}

\begin{table}[h!]

\centering

\tiny
\begin{equation*}
\boxed{
\begin{array}{cc|cc|ccc|cc|ccc}
 n & \mathbf{Primes} & \mathbf{Sqfree} & \mathbf{PPower} & g^{-1}(n) & 
 \lambda(n) g^{-1}(n) - \widehat{f}_1(n) & 
 \frac{\sum_{d|n} C_{\Omega(d)}(d)}{|g^{-1}(n)|} & 
 \mathcal{L}_{+}(n) & \mathcal{L}_{-}(n) & 
 G^{-1}(n) & G^{-1}_{+}(n) & G^{-1}_{-}(n) \\ \hline 
1 & 1^1 & \text{Y} & \text{N} & 1 & 0 & 1.0000000 & 1.000000 & 0.000000 & 1 & 1 & 0 \\
 2 & 2^1 & \text{Y} & \text{Y} & -2 & 0 & 1.0000000 & 0.500000 & 0.500000 & -1 & 1 & -2 \\
 3 & 3^1 & \text{Y} & \text{Y} & -2 & 0 & 1.0000000 & 0.333333 & 0.666667 & -3 & 1 & -4 \\
 4 & 2^2 & \text{N} & \text{Y} & 2 & 0 & 1.5000000 & 0.500000 & 0.500000 & -1 & 3 & -4 \\
 5 & 5^1 & \text{Y} & \text{Y} & -2 & 0 & 1.0000000 & 0.400000 & 0.600000 & -3 & 3 & -6 \\
 6 & 2^1 3^1 & \text{Y} & \text{N} & 5 & 0 & 1.0000000 & 0.500000 & 0.500000 & 2 & 8 & -6 \\
 7 & 7^1 & \text{Y} & \text{Y} & -2 & 0 & 1.0000000 & 0.428571 & 0.571429 & 0 & 8 & -8 \\
 8 & 2^3 & \text{N} & \text{Y} & -2 & 0 & 2.0000000 & 0.375000 & 0.625000 & -2 & 8 & -10 \\
 9 & 3^2 & \text{N} & \text{Y} & 2 & 0 & 1.5000000 & 0.444444 & 0.555556 & 0 & 10 & -10 \\
 10 & 2^1 5^1 & \text{Y} & \text{N} & 5 & 0 & 1.0000000 & 0.500000 & 0.500000 & 5 & 15 & -10 \\
 11 & 11^1 & \text{Y} & \text{Y} & -2 & 0 & 1.0000000 & 0.454545 & 0.545455 & 3 & 15 & -12 \\
 12 & 2^2 3^1 & \text{N} & \text{N} & -7 & 2 & 1.2857143 & 0.416667 & 0.583333 & -4 & 15 & -19 \\
 13 & 13^1 & \text{Y} & \text{Y} & -2 & 0 & 1.0000000 & 0.384615 & 0.615385 & -6 & 15 & -21 \\
 14 & 2^1 7^1 & \text{Y} & \text{N} & 5 & 0 & 1.0000000 & 0.428571 & 0.571429 & -1 & 20 & -21 \\
 15 & 3^1 5^1 & \text{Y} & \text{N} & 5 & 0 & 1.0000000 & 0.466667 & 0.533333 & 4 & 25 & -21 \\
 16 & 2^4 & \text{N} & \text{Y} & 2 & 0 & 2.5000000 & 0.500000 & 0.500000 & 6 & 27 & -21 \\
 17 & 17^1 & \text{Y} & \text{Y} & -2 & 0 & 1.0000000 & 0.470588 & 0.529412 & 4 & 27 & -23 \\
 18 & 2^1 3^2 & \text{N} & \text{N} & -7 & 2 & 1.2857143 & 0.444444 & 0.555556 & -3 & 27 & -30 \\
 19 & 19^1 & \text{Y} & \text{Y} & -2 & 0 & 1.0000000 & 0.421053 & 0.578947 & -5 & 27 & -32 \\
 20 & 2^2 5^1 & \text{N} & \text{N} & -7 & 2 & 1.2857143 & 0.400000 & 0.600000 & -12 & 27 & -39 \\
 21 & 3^1 7^1 & \text{Y} & \text{N} & 5 & 0 & 1.0000000 & 0.428571 & 0.571429 & -7 & 32 & -39 \\
 22 & 2^1 11^1 & \text{Y} & \text{N} & 5 & 0 & 1.0000000 & 0.454545 & 0.545455 & -2 & 37 & -39 \\
 23 & 23^1 & \text{Y} & \text{Y} & -2 & 0 & 1.0000000 & 0.434783 & 0.565217 & -4 & 37 & -41 \\
 24 & 2^3 3^1 & \text{N} & \text{N} & 9 & 4 & 1.5555556 & 0.458333 & 0.541667 & 5 & 46 & -41 \\
 25 & 5^2 & \text{N} & \text{Y} & 2 & 0 & 1.5000000 & 0.480000 & 0.520000 & 7 & 48 & -41 \\
 26 & 2^1 13^1 & \text{Y} & \text{N} & 5 & 0 & 1.0000000 & 0.500000 & 0.500000 & 12 & 53 & -41 \\
 27 & 3^3 & \text{N} & \text{Y} & -2 & 0 & 2.0000000 & 0.481481 & 0.518519 & 10 & 53 & -43 \\
 28 & 2^2 7^1 & \text{N} & \text{N} & -7 & 2 & 1.2857143 & 0.464286 & 0.535714 & 3 & 53 & -50 \\
 29 & 29^1 & \text{Y} & \text{Y} & -2 & 0 & 1.0000000 & 0.448276 & 0.551724 & 1 & 53 & -52 \\
 30 & 2^1 3^1 5^1 & \text{Y} & \text{N} & -16 & 0 & 1.0000000 & 0.433333 & 0.566667 & -15 & 53 & -68 \\
 31 & 31^1 & \text{Y} & \text{Y} & -2 & 0 & 1.0000000 & 0.419355 & 0.580645 & -17 & 53 & -70 \\
 32 & 2^5 & \text{N} & \text{Y} & -2 & 0 & 3.0000000 & 0.406250 & 0.593750 & -19 & 53 & -72 \\
 33 & 3^1 11^1 & \text{Y} & \text{N} & 5 & 0 & 1.0000000 & 0.424242 & 0.575758 & -14 & 58 & -72 \\
 34 & 2^1 17^1 & \text{Y} & \text{N} & 5 & 0 & 1.0000000 & 0.441176 & 0.558824 & -9 & 63 & -72 \\
 35 & 5^1 7^1 & \text{Y} & \text{N} & 5 & 0 & 1.0000000 & 0.457143 & 0.542857 & -4 & 68 & -72 \\
 36 & 2^2 3^2 & \text{N} & \text{N} & 14 & 9 & 1.3571429 & 0.472222 & 0.527778 & 10 & 82 & -72 \\
 37 & 37^1 & \text{Y} & \text{Y} & -2 & 0 & 1.0000000 & 0.459459 & 0.540541 & 8 & 82 & -74 \\
 38 & 2^1 19^1 & \text{Y} & \text{N} & 5 & 0 & 1.0000000 & 0.473684 & 0.526316 & 13 & 87 & -74 \\
 39 & 3^1 13^1 & \text{Y} & \text{N} & 5 & 0 & 1.0000000 & 0.487179 & 0.512821 & 18 & 92 & -74 \\
 40 & 2^3 5^1 & \text{N} & \text{N} & 9 & 4 & 1.5555556 & 0.500000 & 0.500000 & 27 & 101 & -74 \\
 41 & 41^1 & \text{Y} & \text{Y} & -2 & 0 & 1.0000000 & 0.487805 & 0.512195 & 25 & 101 & -76 \\
 42 & 2^1 3^1 7^1 & \text{Y} & \text{N} & -16 & 0 & 1.0000000 & 0.476190 & 0.523810 & 9 & 101 & -92 \\
 43 & 43^1 & \text{Y} & \text{Y} & -2 & 0 & 1.0000000 & 0.465116 & 0.534884 & 7 & 101 & -94 \\
 44 & 2^2 11^1 & \text{N} & \text{N} & -7 & 2 & 1.2857143 & 0.454545 & 0.545455 & 0 & 101 & -101 \\
 45 & 3^2 5^1 & \text{N} & \text{N} & -7 & 2 & 1.2857143 & 0.444444 & 0.555556 & -7 & 101 & -108 \\
 46 & 2^1 23^1 & \text{Y} & \text{N} & 5 & 0 & 1.0000000 & 0.456522 & 0.543478 & -2 & 106 & -108 \\
 47 & 47^1 & \text{Y} & \text{Y} & -2 & 0 & 1.0000000 & 0.446809 & 0.553191 & -4 & 106 & -110 \\
 48 & 2^4 3^1 & \text{N} & \text{N} & -11 & 6 & 1.8181818 & 0.437500 & 0.562500 & -15 & 106 & -121 \\ 
\end{array}
}
\end{equation*}

\bigskip\hrule\smallskip 

\captionsetup{singlelinecheck=off} 
\caption*{\textbf{\rm \bf Table \thesection:} 
          \textbf{Computations with $\mathbf{g^{-1}(n) \equiv (\omega+1)^{-1}(n)}$ 
          for $\mathbf{1 \leq n \leq 500}$.} 
          \begin{itemize}[noitemsep,topsep=0pt,leftmargin=0.23in] 
          \item[$\blacktriangleright$] 
          The column labeled \texttt{Primes} provides the prime factorization of each $n$ so that the values of 
          $\omega(n)$ and $\Omega(n)$ are easily extracted. 
          The columns labeled \texttt{Sqfree} and \texttt{PPower}, respectively, 
          list inclusion of $n$ in the sets of squarefree integers and the prime powers. 
          \item[$\blacktriangleright$] 
          The next three columns provide the 
          explicit values of the inverse function $g^{-1}(n)$ and compare its explicit value with other estimates. 
          We define the function $\widehat{f}_1(n) := \sum_{k=0}^{\omega(n)} \binom{\omega(n)}{k} \cdot k!$. 
          \item[$\blacktriangleright$] 
          The last several columns indicate properties of the summatory function of $g^{-1}(n)$. 
          The notation for the densities of the sign weight of $g^{-1}(n)$ is defined as 
          $\mathcal{L}_{\pm}(x) := \frac{1}{n} \cdot \#\left\{n \leq x: \lambda(n) = \pm 1\right\}$. 
          The last three 
          columns then show the explicit components to the signed summatory function, 
          $G^{-1}(x) := \sum_{n \leq x} g^{-1}(n)$, decomposed into its 
          respective positive and negative magnitude sum contributions: $G^{-1}(x) = G^{-1}_{+}(x) + G^{-1}_{-}(x)$ where 
          $G^{-1}_{+}(x) > 0$ and $G^{-1}_{-}(x) < 0$ for all $x \geq 1$. 
          \end{itemize} 
          } 

\end{table}

\newpage
\begin{table}[h!]

\centering

\tiny
\begin{equation*}
\boxed{
\begin{array}{cc|cc|ccc|cc|ccc}
 n & \mathbf{Primes} & \mathbf{Sqfree} & \mathbf{PPower} & g^{-1}(n) & 
 \lambda(n) g^{-1}(n) - \widehat{f}_1(n) & 
 \frac{\sum_{d|n} C_{\Omega(d)}(d)}{|g^{-1}(n)|} & 
 \mathcal{L}_{+}(n) & \mathcal{L}_{-}(n) & 
 G^{-1}(n) & G^{-1}_{+}(n) & G^{-1}_{-}(n) \\ \hline 
 49 & 7^2 & \text{N} & \text{Y} & 2 & 0 & 1.5000000 & 0.448980 & 0.551020 & -13 & 108 & -121 \\
 50 & 2^1 5^2 & \text{N} & \text{N} & -7 & 2 & 1.2857143 & 0.440000 & 0.560000 & -20 & 108 & -128 \\
 51 & 3^1 17^1 & \text{Y} & \text{N} & 5 & 0 & 1.0000000 & 0.450980 & 0.549020 & -15 & 113 & -128 \\
 52 & 2^2 13^1 & \text{N} & \text{N} & -7 & 2 & 1.2857143 & 0.442308 & 0.557692 & -22 & 113 & -135 \\
 53 & 53^1 & \text{Y} & \text{Y} & -2 & 0 & 1.0000000 & 0.433962 & 0.566038 & -24 & 113 & -137 \\
 54 & 2^1 3^3 & \text{N} & \text{N} & 9 & 4 & 1.5555556 & 0.444444 & 0.555556 & -15 & 122 & -137 \\
 55 & 5^1 11^1 & \text{Y} & \text{N} & 5 & 0 & 1.0000000 & 0.454545 & 0.545455 & -10 & 127 & -137 \\
 56 & 2^3 7^1 & \text{N} & \text{N} & 9 & 4 & 1.5555556 & 0.464286 & 0.535714 & -1 & 136 & -137 \\
 57 & 3^1 19^1 & \text{Y} & \text{N} & 5 & 0 & 1.0000000 & 0.473684 & 0.526316 & 4 & 141 & -137 \\
 58 & 2^1 29^1 & \text{Y} & \text{N} & 5 & 0 & 1.0000000 & 0.482759 & 0.517241 & 9 & 146 & -137 \\
 59 & 59^1 & \text{Y} & \text{Y} & -2 & 0 & 1.0000000 & 0.474576 & 0.525424 & 7 & 146 & -139 \\
 60 & 2^2 3^1 5^1 & \text{N} & \text{N} & 30 & 14 & 1.1666667 & 0.483333 & 0.516667 & 37 & 176 & -139 \\
 61 & 61^1 & \text{Y} & \text{Y} & -2 & 0 & 1.0000000 & 0.475410 & 0.524590 & 35 & 176 & -141 \\
 62 & 2^1 31^1 & \text{Y} & \text{N} & 5 & 0 & 1.0000000 & 0.483871 & 0.516129 & 40 & 181 & -141 \\
 63 & 3^2 7^1 & \text{N} & \text{N} & -7 & 2 & 1.2857143 & 0.476190 & 0.523810 & 33 & 181 & -148 \\
 64 & 2^6 & \text{N} & \text{Y} & 2 & 0 & 3.5000000 & 0.484375 & 0.515625 & 35 & 183 & -148 \\
 65 & 5^1 13^1 & \text{Y} & \text{N} & 5 & 0 & 1.0000000 & 0.492308 & 0.507692 & 40 & 188 & -148 \\
 66 & 2^1 3^1 11^1 & \text{Y} & \text{N} & -16 & 0 & 1.0000000 & 0.484848 & 0.515152 & 24 & 188 & -164 \\
 67 & 67^1 & \text{Y} & \text{Y} & -2 & 0 & 1.0000000 & 0.477612 & 0.522388 & 22 & 188 & -166 \\
 68 & 2^2 17^1 & \text{N} & \text{N} & -7 & 2 & 1.2857143 & 0.470588 & 0.529412 & 15 & 188 & -173 \\
 69 & 3^1 23^1 & \text{Y} & \text{N} & 5 & 0 & 1.0000000 & 0.478261 & 0.521739 & 20 & 193 & -173 \\
 70 & 2^1 5^1 7^1 & \text{Y} & \text{N} & -16 & 0 & 1.0000000 & 0.471429 & 0.528571 & 4 & 193 & -189 \\
 71 & 71^1 & \text{Y} & \text{Y} & -2 & 0 & 1.0000000 & 0.464789 & 0.535211 & 2 & 193 & -191 \\
 72 & 2^3 3^2 & \text{N} & \text{N} & -23 & 18 & 1.4782609 & 0.458333 & 0.541667 & -21 & 193 & -214 \\
 73 & 73^1 & \text{Y} & \text{Y} & -2 & 0 & 1.0000000 & 0.452055 & 0.547945 & -23 & 193 & -216 \\
 74 & 2^1 37^1 & \text{Y} & \text{N} & 5 & 0 & 1.0000000 & 0.459459 & 0.540541 & -18 & 198 & -216 \\
 75 & 3^1 5^2 & \text{N} & \text{N} & -7 & 2 & 1.2857143 & 0.453333 & 0.546667 & -25 & 198 & -223 \\
 76 & 2^2 19^1 & \text{N} & \text{N} & -7 & 2 & 1.2857143 & 0.447368 & 0.552632 & -32 & 198 & -230 \\
 77 & 7^1 11^1 & \text{Y} & \text{N} & 5 & 0 & 1.0000000 & 0.454545 & 0.545455 & -27 & 203 & -230 \\
 78 & 2^1 3^1 13^1 & \text{Y} & \text{N} & -16 & 0 & 1.0000000 & 0.448718 & 0.551282 & -43 & 203 & -246 \\
 79 & 79^1 & \text{Y} & \text{Y} & -2 & 0 & 1.0000000 & 0.443038 & 0.556962 & -45 & 203 & -248 \\
 80 & 2^4 5^1 & \text{N} & \text{N} & -11 & 6 & 1.8181818 & 0.437500 & 0.562500 & -56 & 203 & -259 \\
 81 & 3^4 & \text{N} & \text{Y} & 2 & 0 & 2.5000000 & 0.444444 & 0.555556 & -54 & 205 & -259 \\
 82 & 2^1 41^1 & \text{Y} & \text{N} & 5 & 0 & 1.0000000 & 0.451220 & 0.548780 & -49 & 210 & -259 \\
 83 & 83^1 & \text{Y} & \text{Y} & -2 & 0 & 1.0000000 & 0.445783 & 0.554217 & -51 & 210 & -261 \\
 84 & 2^2 3^1 7^1 & \text{N} & \text{N} & 30 & 14 & 1.1666667 & 0.452381 & 0.547619 & -21 & 240 & -261 \\
 85 & 5^1 17^1 & \text{Y} & \text{N} & 5 & 0 & 1.0000000 & 0.458824 & 0.541176 & -16 & 245 & -261 \\
 86 & 2^1 43^1 & \text{Y} & \text{N} & 5 & 0 & 1.0000000 & 0.465116 & 0.534884 & -11 & 250 & -261 \\
 87 & 3^1 29^1 & \text{Y} & \text{N} & 5 & 0 & 1.0000000 & 0.471264 & 0.528736 & -6 & 255 & -261 \\
 88 & 2^3 11^1 & \text{N} & \text{N} & 9 & 4 & 1.5555556 & 0.477273 & 0.522727 & 3 & 264 & -261 \\
 89 & 89^1 & \text{Y} & \text{Y} & -2 & 0 & 1.0000000 & 0.471910 & 0.528090 & 1 & 264 & -263 \\
 90 & 2^1 3^2 5^1 & \text{N} & \text{N} & 30 & 14 & 1.1666667 & 0.477778 & 0.522222 & 31 & 294 & -263 \\
 91 & 7^1 13^1 & \text{Y} & \text{N} & 5 & 0 & 1.0000000 & 0.483516 & 0.516484 & 36 & 299 & -263 \\
 92 & 2^2 23^1 & \text{N} & \text{N} & -7 & 2 & 1.2857143 & 0.478261 & 0.521739 & 29 & 299 & -270 \\
 93 & 3^1 31^1 & \text{Y} & \text{N} & 5 & 0 & 1.0000000 & 0.483871 & 0.516129 & 34 & 304 & -270 \\
 94 & 2^1 47^1 & \text{Y} & \text{N} & 5 & 0 & 1.0000000 & 0.489362 & 0.510638 & 39 & 309 & -270 \\
 95 & 5^1 19^1 & \text{Y} & \text{N} & 5 & 0 & 1.0000000 & 0.494737 & 0.505263 & 44 & 314 & -270 \\
 96 & 2^5 3^1 & \text{N} & \text{N} & 13 & 8 & 2.0769231 & 0.500000 & 0.500000 & 57 & 327 & -270 \\
 97 & 97^1 & \text{Y} & \text{Y} & -2 & 0 & 1.0000000 & 0.494845 & 0.505155 & 55 & 327 & -272 \\
 98 & 2^1 7^2 & \text{N} & \text{N} & -7 & 2 & 1.2857143 & 0.489796 & 0.510204 & 48 & 327 & -279 \\
 99 & 3^2 11^1 & \text{N} & \text{N} & -7 & 2 & 1.2857143 & 0.484848 & 0.515152 & 41 & 327 & -286 \\
 100 & 2^2 5^2 & \text{N} & \text{N} & 14 & 9 & 1.3571429 & 0.490000 & 0.510000 & 55 & 341 & -286 \\
 101 & 101^1 & \text{Y} & \text{Y} & -2 & 0 & 1.0000000 & 0.485149 & 0.514851 & 53 & 341 & -288 \\
 102 & 2^1 3^1 17^1 & \text{Y} & \text{N} & -16 & 0 & 1.0000000 & 0.480392 & 0.519608 & 37 & 341 & -304 \\
 103 & 103^1 & \text{Y} & \text{Y} & -2 & 0 & 1.0000000 & 0.475728 & 0.524272 & 35 & 341 & -306 \\
 104 & 2^3 13^1 & \text{N} & \text{N} & 9 & 4 & 1.5555556 & 0.480769 & 0.519231 & 44 & 350 & -306 \\
 105 & 3^1 5^1 7^1 & \text{Y} & \text{N} & -16 & 0 & 1.0000000 & 0.476190 & 0.523810 & 28 & 350 & -322 \\
 106 & 2^1 53^1 & \text{Y} & \text{N} & 5 & 0 & 1.0000000 & 0.481132 & 0.518868 & 33 & 355 & -322 \\
 107 & 107^1 & \text{Y} & \text{Y} & -2 & 0 & 1.0000000 & 0.476636 & 0.523364 & 31 & 355 & -324 \\
 108 & 2^2 3^3 & \text{N} & \text{N} & -23 & 18 & 1.4782609 & 0.472222 & 0.527778 & 8 & 355 & -347 \\
 109 & 109^1 & \text{Y} & \text{Y} & -2 & 0 & 1.0000000 & 0.467890 & 0.532110 & 6 & 355 & -349 \\
 110 & 2^1 5^1 11^1 & \text{Y} & \text{N} & -16 & 0 & 1.0000000 & 0.463636 & 0.536364 & -10 & 355 & -365 \\
 111 & 3^1 37^1 & \text{Y} & \text{N} & 5 & 0 & 1.0000000 & 0.468468 & 0.531532 & -5 & 360 & -365 \\
 112 & 2^4 7^1 & \text{N} & \text{N} & -11 & 6 & 1.8181818 & 0.464286 & 0.535714 & -16 & 360 & -376 \\
 113 & 113^1 & \text{Y} & \text{Y} & -2 & 0 & 1.0000000 & 0.460177 & 0.539823 & -18 & 360 & -378 \\
 114 & 2^1 3^1 19^1 & \text{Y} & \text{N} & -16 & 0 & 1.0000000 & 0.456140 & 0.543860 & -34 & 360 & -394 \\
 115 & 5^1 23^1 & \text{Y} & \text{N} & 5 & 0 & 1.0000000 & 0.460870 & 0.539130 & -29 & 365 & -394 \\
 116 & 2^2 29^1 & \text{N} & \text{N} & -7 & 2 & 1.2857143 & 0.456897 & 0.543103 & -36 & 365 & -401 \\
 117 & 3^2 13^1 & \text{N} & \text{N} & -7 & 2 & 1.2857143 & 0.452991 & 0.547009 & -43 & 365 & -408 \\
 118 & 2^1 59^1 & \text{Y} & \text{N} & 5 & 0 & 1.0000000 & 0.457627 & 0.542373 & -38 & 370 & -408 \\
 119 & 7^1 17^1 & \text{Y} & \text{N} & 5 & 0 & 1.0000000 & 0.462185 & 0.537815 & -33 & 375 & -408 \\
 120 & 2^3 3^1 5^1 & \text{N} & \text{N} & -48 & 32 & 1.3333333 & 0.458333 & 0.541667 & -81 & 375 & -456 \\
 121 & 11^2 & \text{N} & \text{Y} & 2 & 0 & 1.5000000 & 0.462810 & 0.537190 & -79 & 377 & -456 \\
 122 & 2^1 61^1 & \text{Y} & \text{N} & 5 & 0 & 1.0000000 & 0.467213 & 0.532787 & -74 & 382 & -456 \\
 123 & 3^1 41^1 & \text{Y} & \text{N} & 5 & 0 & 1.0000000 & 0.471545 & 0.528455 & -69 & 387 & -456 \\
 124 & 2^2 31^1 & \text{N} & \text{N} & -7 & 2 & 1.2857143 & 0.467742 & 0.532258 & -76 & 387 & -463 \\ 
\end{array}
}
\end{equation*}

\end{table} 


\newpage
\begin{table}[h!]

\centering

\tiny
\begin{equation*}
\boxed{
\begin{array}{cc|cc|ccc|cc|ccc}
 n & \mathbf{Primes} & \mathbf{Sqfree} & \mathbf{PPower} & g^{-1}(n) & 
 \lambda(n) g^{-1}(n) - \widehat{f}_1(n) & 
 \frac{\sum_{d|n} C_{\Omega(d)}(d)}{|g^{-1}(n)|} & 
 \mathcal{L}_{+}(n) & \mathcal{L}_{-}(n) & 
 G^{-1}(n) & G^{-1}_{+}(n) & G^{-1}_{-}(n) \\ \hline 
 125 & 5^3 & \text{N} & \text{Y} & -2 & 0 & 2.0000000 & 0.464000 & 0.536000 & -78 & 387 & -465 \\
 126 & 2^1 3^2 7^1 & \text{N} & \text{N} & 30 & 14 & 1.1666667 & 0.468254 & 0.531746 & -48 & 417 & -465 \\
 127 & 127^1 & \text{Y} & \text{Y} & -2 & 0 & 1.0000000 & 0.464567 & 0.535433 & -50 & 417 & -467 \\
 128 & 2^7 & \text{N} & \text{Y} & -2 & 0 & 4.0000000 & 0.460938 & 0.539062 & -52 & 417 & -469 \\
 129 & 3^1 43^1 & \text{Y} & \text{N} & 5 & 0 & 1.0000000 & 0.465116 & 0.534884 & -47 & 422 & -469 \\
 130 & 2^1 5^1 13^1 & \text{Y} & \text{N} & -16 & 0 & 1.0000000 & 0.461538 & 0.538462 & -63 & 422 & -485 \\
 131 & 131^1 & \text{Y} & \text{Y} & -2 & 0 & 1.0000000 & 0.458015 & 0.541985 & -65 & 422 & -487 \\
 132 & 2^2 3^1 11^1 & \text{N} & \text{N} & 30 & 14 & 1.1666667 & 0.462121 & 0.537879 & -35 & 452 & -487 \\
 133 & 7^1 19^1 & \text{Y} & \text{N} & 5 & 0 & 1.0000000 & 0.466165 & 0.533835 & -30 & 457 & -487 \\
 134 & 2^1 67^1 & \text{Y} & \text{N} & 5 & 0 & 1.0000000 & 0.470149 & 0.529851 & -25 & 462 & -487 \\
 135 & 3^3 5^1 & \text{N} & \text{N} & 9 & 4 & 1.5555556 & 0.474074 & 0.525926 & -16 & 471 & -487 \\
 136 & 2^3 17^1 & \text{N} & \text{N} & 9 & 4 & 1.5555556 & 0.477941 & 0.522059 & -7 & 480 & -487 \\
 137 & 137^1 & \text{Y} & \text{Y} & -2 & 0 & 1.0000000 & 0.474453 & 0.525547 & -9 & 480 & -489 \\
 138 & 2^1 3^1 23^1 & \text{Y} & \text{N} & -16 & 0 & 1.0000000 & 0.471014 & 0.528986 & -25 & 480 & -505 \\
 139 & 139^1 & \text{Y} & \text{Y} & -2 & 0 & 1.0000000 & 0.467626 & 0.532374 & -27 & 480 & -507 \\
 140 & 2^2 5^1 7^1 & \text{N} & \text{N} & 30 & 14 & 1.1666667 & 0.471429 & 0.528571 & 3 & 510 & -507 \\
 141 & 3^1 47^1 & \text{Y} & \text{N} & 5 & 0 & 1.0000000 & 0.475177 & 0.524823 & 8 & 515 & -507 \\
 142 & 2^1 71^1 & \text{Y} & \text{N} & 5 & 0 & 1.0000000 & 0.478873 & 0.521127 & 13 & 520 & -507 \\
 143 & 11^1 13^1 & \text{Y} & \text{N} & 5 & 0 & 1.0000000 & 0.482517 & 0.517483 & 18 & 525 & -507 \\
 144 & 2^4 3^2 & \text{N} & \text{N} & 34 & 29 & 1.6176471 & 0.486111 & 0.513889 & 52 & 559 & -507 \\
 145 & 5^1 29^1 & \text{Y} & \text{N} & 5 & 0 & 1.0000000 & 0.489655 & 0.510345 & 57 & 564 & -507 \\
 146 & 2^1 73^1 & \text{Y} & \text{N} & 5 & 0 & 1.0000000 & 0.493151 & 0.506849 & 62 & 569 & -507 \\
 147 & 3^1 7^2 & \text{N} & \text{N} & -7 & 2 & 1.2857143 & 0.489796 & 0.510204 & 55 & 569 & -514 \\
 148 & 2^2 37^1 & \text{N} & \text{N} & -7 & 2 & 1.2857143 & 0.486486 & 0.513514 & 48 & 569 & -521 \\
 149 & 149^1 & \text{Y} & \text{Y} & -2 & 0 & 1.0000000 & 0.483221 & 0.516779 & 46 & 569 & -523 \\
 150 & 2^1 3^1 5^2 & \text{N} & \text{N} & 30 & 14 & 1.1666667 & 0.486667 & 0.513333 & 76 & 599 & -523 \\
 151 & 151^1 & \text{Y} & \text{Y} & -2 & 0 & 1.0000000 & 0.483444 & 0.516556 & 74 & 599 & -525 \\
 152 & 2^3 19^1 & \text{N} & \text{N} & 9 & 4 & 1.5555556 & 0.486842 & 0.513158 & 83 & 608 & -525 \\
 153 & 3^2 17^1 & \text{N} & \text{N} & -7 & 2 & 1.2857143 & 0.483660 & 0.516340 & 76 & 608 & -532 \\
 154 & 2^1 7^1 11^1 & \text{Y} & \text{N} & -16 & 0 & 1.0000000 & 0.480519 & 0.519481 & 60 & 608 & -548 \\
 155 & 5^1 31^1 & \text{Y} & \text{N} & 5 & 0 & 1.0000000 & 0.483871 & 0.516129 & 65 & 613 & -548 \\
 156 & 2^2 3^1 13^1 & \text{N} & \text{N} & 30 & 14 & 1.1666667 & 0.487179 & 0.512821 & 95 & 643 & -548 \\
 157 & 157^1 & \text{Y} & \text{Y} & -2 & 0 & 1.0000000 & 0.484076 & 0.515924 & 93 & 643 & -550 \\
 158 & 2^1 79^1 & \text{Y} & \text{N} & 5 & 0 & 1.0000000 & 0.487342 & 0.512658 & 98 & 648 & -550 \\
 159 & 3^1 53^1 & \text{Y} & \text{N} & 5 & 0 & 1.0000000 & 0.490566 & 0.509434 & 103 & 653 & -550 \\
 160 & 2^5 5^1 & \text{N} & \text{N} & 13 & 8 & 2.0769231 & 0.493750 & 0.506250 & 116 & 666 & -550 \\
 161 & 7^1 23^1 & \text{Y} & \text{N} & 5 & 0 & 1.0000000 & 0.496894 & 0.503106 & 121 & 671 & -550 \\
 162 & 2^1 3^4 & \text{N} & \text{N} & -11 & 6 & 1.8181818 & 0.493827 & 0.506173 & 110 & 671 & -561 \\
 163 & 163^1 & \text{Y} & \text{Y} & -2 & 0 & 1.0000000 & 0.490798 & 0.509202 & 108 & 671 & -563 \\
 164 & 2^2 41^1 & \text{N} & \text{N} & -7 & 2 & 1.2857143 & 0.487805 & 0.512195 & 101 & 671 & -570 \\
 165 & 3^1 5^1 11^1 & \text{Y} & \text{N} & -16 & 0 & 1.0000000 & 0.484848 & 0.515152 & 85 & 671 & -586 \\
 166 & 2^1 83^1 & \text{Y} & \text{N} & 5 & 0 & 1.0000000 & 0.487952 & 0.512048 & 90 & 676 & -586 \\
 167 & 167^1 & \text{Y} & \text{Y} & -2 & 0 & 1.0000000 & 0.485030 & 0.514970 & 88 & 676 & -588 \\
 168 & 2^3 3^1 7^1 & \text{N} & \text{N} & -48 & 32 & 1.3333333 & 0.482143 & 0.517857 & 40 & 676 & -636 \\
 169 & 13^2 & \text{N} & \text{Y} & 2 & 0 & 1.5000000 & 0.485207 & 0.514793 & 42 & 678 & -636 \\
 170 & 2^1 5^1 17^1 & \text{Y} & \text{N} & -16 & 0 & 1.0000000 & 0.482353 & 0.517647 & 26 & 678 & -652 \\
 171 & 3^2 19^1 & \text{N} & \text{N} & -7 & 2 & 1.2857143 & 0.479532 & 0.520468 & 19 & 678 & -659 \\
 172 & 2^2 43^1 & \text{N} & \text{N} & -7 & 2 & 1.2857143 & 0.476744 & 0.523256 & 12 & 678 & -666 \\
 173 & 173^1 & \text{Y} & \text{Y} & -2 & 0 & 1.0000000 & 0.473988 & 0.526012 & 10 & 678 & -668 \\
 174 & 2^1 3^1 29^1 & \text{Y} & \text{N} & -16 & 0 & 1.0000000 & 0.471264 & 0.528736 & -6 & 678 & -684 \\
 175 & 5^2 7^1 & \text{N} & \text{N} & -7 & 2 & 1.2857143 & 0.468571 & 0.531429 & -13 & 678 & -691 \\
 176 & 2^4 11^1 & \text{N} & \text{N} & -11 & 6 & 1.8181818 & 0.465909 & 0.534091 & -24 & 678 & -702 \\
 177 & 3^1 59^1 & \text{Y} & \text{N} & 5 & 0 & 1.0000000 & 0.468927 & 0.531073 & -19 & 683 & -702 \\
 178 & 2^1 89^1 & \text{Y} & \text{N} & 5 & 0 & 1.0000000 & 0.471910 & 0.528090 & -14 & 688 & -702 \\
 179 & 179^1 & \text{Y} & \text{Y} & -2 & 0 & 1.0000000 & 0.469274 & 0.530726 & -16 & 688 & -704 \\
 180 & 2^2 3^2 5^1 & \text{N} & \text{N} & -74 & 58 & 1.2162162 & 0.466667 & 0.533333 & -90 & 688 & -778 \\
 181 & 181^1 & \text{Y} & \text{Y} & -2 & 0 & 1.0000000 & 0.464088 & 0.535912 & -92 & 688 & -780 \\
 182 & 2^1 7^1 13^1 & \text{Y} & \text{N} & -16 & 0 & 1.0000000 & 0.461538 & 0.538462 & -108 & 688 & -796 \\
 183 & 3^1 61^1 & \text{Y} & \text{N} & 5 & 0 & 1.0000000 & 0.464481 & 0.535519 & -103 & 693 & -796 \\
 184 & 2^3 23^1 & \text{N} & \text{N} & 9 & 4 & 1.5555556 & 0.467391 & 0.532609 & -94 & 702 & -796 \\
 185 & 5^1 37^1 & \text{Y} & \text{N} & 5 & 0 & 1.0000000 & 0.470270 & 0.529730 & -89 & 707 & -796 \\
 186 & 2^1 3^1 31^1 & \text{Y} & \text{N} & -16 & 0 & 1.0000000 & 0.467742 & 0.532258 & -105 & 707 & -812 \\
 187 & 11^1 17^1 & \text{Y} & \text{N} & 5 & 0 & 1.0000000 & 0.470588 & 0.529412 & -100 & 712 & -812 \\
 188 & 2^2 47^1 & \text{N} & \text{N} & -7 & 2 & 1.2857143 & 0.468085 & 0.531915 & -107 & 712 & -819 \\
 189 & 3^3 7^1 & \text{N} & \text{N} & 9 & 4 & 1.5555556 & 0.470899 & 0.529101 & -98 & 721 & -819 \\
 190 & 2^1 5^1 19^1 & \text{Y} & \text{N} & -16 & 0 & 1.0000000 & 0.468421 & 0.531579 & -114 & 721 & -835 \\
 191 & 191^1 & \text{Y} & \text{Y} & -2 & 0 & 1.0000000 & 0.465969 & 0.534031 & -116 & 721 & -837 \\
 192 & 2^6 3^1 & \text{N} & \text{N} & -15 & 10 & 2.3333333 & 0.463542 & 0.536458 & -131 & 721 & -852 \\
 193 & 193^1 & \text{Y} & \text{Y} & -2 & 0 & 1.0000000 & 0.461140 & 0.538860 & -133 & 721 & -854 \\
 194 & 2^1 97^1 & \text{Y} & \text{N} & 5 & 0 & 1.0000000 & 0.463918 & 0.536082 & -128 & 726 & -854 \\
 195 & 3^1 5^1 13^1 & \text{Y} & \text{N} & -16 & 0 & 1.0000000 & 0.461538 & 0.538462 & -144 & 726 & -870 \\
 196 & 2^2 7^2 & \text{N} & \text{N} & 14 & 9 & 1.3571429 & 0.464286 & 0.535714 & -130 & 740 & -870 \\
 197 & 197^1 & \text{Y} & \text{Y} & -2 & 0 & 1.0000000 & 0.461929 & 0.538071 & -132 & 740 & -872 \\
 198 & 2^1 3^2 11^1 & \text{N} & \text{N} & 30 & 14 & 1.1666667 & 0.464646 & 0.535354 & -102 & 770 & -872 \\
 199 & 199^1 & \text{Y} & \text{Y} & -2 & 0 & 1.0000000 & 0.462312 & 0.537688 & -104 & 770 & -874 \\
 200 & 2^3 5^2 & \text{N} & \text{N} & -23 & 18 & 1.4782609 & 0.460000 & 0.540000 & -127 & 770 & -897 \\ 
\end{array}
}
\end{equation*}

\end{table} 

\newpage
\begin{table}[h!]

\centering

\tiny
\begin{equation*}
\boxed{
\begin{array}{cc|cc|ccc|cc|ccc}
 n & \mathbf{Primes} & \mathbf{Sqfree} & \mathbf{PPower} & g^{-1}(n) & 
 \lambda(n) g^{-1}(n) - \widehat{f}_1(n) & 
 \frac{\sum_{d|n} C_{\Omega(d)}(d)}{|g^{-1}(n)|} & 
 \mathcal{L}_{+}(n) & \mathcal{L}_{-}(n) & 
 G^{-1}(n) & G^{-1}_{+}(n) & G^{-1}_{-}(n) \\ \hline 
 201 & 3^1 67^1 & \text{Y} & \text{N} & 5 & 0 & 1.0000000 & 0.462687 & 0.537313 & -122 & 775 & -897 \\
 202 & 2^1 101^1 & \text{Y} & \text{N} & 5 & 0 & 1.0000000 & 0.465347 & 0.534653 & -117 & 780 & -897 \\
 203 & 7^1 29^1 & \text{Y} & \text{N} & 5 & 0 & 1.0000000 & 0.467980 & 0.532020 & -112 & 785 & -897 \\
 204 & 2^2 3^1 17^1 & \text{N} & \text{N} & 30 & 14 & 1.1666667 & 0.470588 & 0.529412 & -82 & 815 & -897 \\
 205 & 5^1 41^1 & \text{Y} & \text{N} & 5 & 0 & 1.0000000 & 0.473171 & 0.526829 & -77 & 820 & -897 \\
 206 & 2^1 103^1 & \text{Y} & \text{N} & 5 & 0 & 1.0000000 & 0.475728 & 0.524272 & -72 & 825 & -897 \\
 207 & 3^2 23^1 & \text{N} & \text{N} & -7 & 2 & 1.2857143 & 0.473430 & 0.526570 & -79 & 825 & -904 \\
 208 & 2^4 13^1 & \text{N} & \text{N} & -11 & 6 & 1.8181818 & 0.471154 & 0.528846 & -90 & 825 & -915 \\
 209 & 11^1 19^1 & \text{Y} & \text{N} & 5 & 0 & 1.0000000 & 0.473684 & 0.526316 & -85 & 830 & -915 \\
 210 & 2^1 3^1 5^1 7^1 & \text{Y} & \text{N} & 65 & 0 & 1.0000000 & 0.476190 & 0.523810 & -20 & 895 & -915 \\
 211 & 211^1 & \text{Y} & \text{Y} & -2 & 0 & 1.0000000 & 0.473934 & 0.526066 & -22 & 895 & -917 \\
 212 & 2^2 53^1 & \text{N} & \text{N} & -7 & 2 & 1.2857143 & 0.471698 & 0.528302 & -29 & 895 & -924 \\
 213 & 3^1 71^1 & \text{Y} & \text{N} & 5 & 0 & 1.0000000 & 0.474178 & 0.525822 & -24 & 900 & -924 \\
 214 & 2^1 107^1 & \text{Y} & \text{N} & 5 & 0 & 1.0000000 & 0.476636 & 0.523364 & -19 & 905 & -924 \\
 215 & 5^1 43^1 & \text{Y} & \text{N} & 5 & 0 & 1.0000000 & 0.479070 & 0.520930 & -14 & 910 & -924 \\
 216 & 2^3 3^3 & \text{N} & \text{N} & 46 & 41 & 1.5000000 & 0.481481 & 0.518519 & 32 & 956 & -924 \\
 217 & 7^1 31^1 & \text{Y} & \text{N} & 5 & 0 & 1.0000000 & 0.483871 & 0.516129 & 37 & 961 & -924 \\
 218 & 2^1 109^1 & \text{Y} & \text{N} & 5 & 0 & 1.0000000 & 0.486239 & 0.513761 & 42 & 966 & -924 \\
 219 & 3^1 73^1 & \text{Y} & \text{N} & 5 & 0 & 1.0000000 & 0.488584 & 0.511416 & 47 & 971 & -924 \\
 220 & 2^2 5^1 11^1 & \text{N} & \text{N} & 30 & 14 & 1.1666667 & 0.490909 & 0.509091 & 77 & 1001 & -924 \\
 221 & 13^1 17^1 & \text{Y} & \text{N} & 5 & 0 & 1.0000000 & 0.493213 & 0.506787 & 82 & 1006 & -924 \\
 222 & 2^1 3^1 37^1 & \text{Y} & \text{N} & -16 & 0 & 1.0000000 & 0.490991 & 0.509009 & 66 & 1006 & -940 \\
 223 & 223^1 & \text{Y} & \text{Y} & -2 & 0 & 1.0000000 & 0.488789 & 0.511211 & 64 & 1006 & -942 \\
 224 & 2^5 7^1 & \text{N} & \text{N} & 13 & 8 & 2.0769231 & 0.491071 & 0.508929 & 77 & 1019 & -942 \\
 225 & 3^2 5^2 & \text{N} & \text{N} & 14 & 9 & 1.3571429 & 0.493333 & 0.506667 & 91 & 1033 & -942 \\
 226 & 2^1 113^1 & \text{Y} & \text{N} & 5 & 0 & 1.0000000 & 0.495575 & 0.504425 & 96 & 1038 & -942 \\
 227 & 227^1 & \text{Y} & \text{Y} & -2 & 0 & 1.0000000 & 0.493392 & 0.506608 & 94 & 1038 & -944 \\
 228 & 2^2 3^1 19^1 & \text{N} & \text{N} & 30 & 14 & 1.1666667 & 0.495614 & 0.504386 & 124 & 1068 & -944 \\
 229 & 229^1 & \text{Y} & \text{Y} & -2 & 0 & 1.0000000 & 0.493450 & 0.506550 & 122 & 1068 & -946 \\
 230 & 2^1 5^1 23^1 & \text{Y} & \text{N} & -16 & 0 & 1.0000000 & 0.491304 & 0.508696 & 106 & 1068 & -962 \\
 231 & 3^1 7^1 11^1 & \text{Y} & \text{N} & -16 & 0 & 1.0000000 & 0.489177 & 0.510823 & 90 & 1068 & -978 \\
 232 & 2^3 29^1 & \text{N} & \text{N} & 9 & 4 & 1.5555556 & 0.491379 & 0.508621 & 99 & 1077 & -978 \\
 233 & 233^1 & \text{Y} & \text{Y} & -2 & 0 & 1.0000000 & 0.489270 & 0.510730 & 97 & 1077 & -980 \\
 234 & 2^1 3^2 13^1 & \text{N} & \text{N} & 30 & 14 & 1.1666667 & 0.491453 & 0.508547 & 127 & 1107 & -980 \\
 235 & 5^1 47^1 & \text{Y} & \text{N} & 5 & 0 & 1.0000000 & 0.493617 & 0.506383 & 132 & 1112 & -980 \\
 236 & 2^2 59^1 & \text{N} & \text{N} & -7 & 2 & 1.2857143 & 0.491525 & 0.508475 & 125 & 1112 & -987 \\
 237 & 3^1 79^1 & \text{Y} & \text{N} & 5 & 0 & 1.0000000 & 0.493671 & 0.506329 & 130 & 1117 & -987 \\
 238 & 2^1 7^1 17^1 & \text{Y} & \text{N} & -16 & 0 & 1.0000000 & 0.491597 & 0.508403 & 114 & 1117 & -1003 \\
 239 & 239^1 & \text{Y} & \text{Y} & -2 & 0 & 1.0000000 & 0.489540 & 0.510460 & 112 & 1117 & -1005 \\
 240 & 2^4 3^1 5^1 & \text{N} & \text{N} & 70 & 54 & 1.5000000 & 0.491667 & 0.508333 & 182 & 1187 & -1005 \\
 241 & 241^1 & \text{Y} & \text{Y} & -2 & 0 & 1.0000000 & 0.489627 & 0.510373 & 180 & 1187 & -1007 \\
 242 & 2^1 11^2 & \text{N} & \text{N} & -7 & 2 & 1.2857143 & 0.487603 & 0.512397 & 173 & 1187 & -1014 \\
 243 & 3^5 & \text{N} & \text{Y} & -2 & 0 & 3.0000000 & 0.485597 & 0.514403 & 171 & 1187 & -1016 \\
 244 & 2^2 61^1 & \text{N} & \text{N} & -7 & 2 & 1.2857143 & 0.483607 & 0.516393 & 164 & 1187 & -1023 \\
 245 & 5^1 7^2 & \text{N} & \text{N} & -7 & 2 & 1.2857143 & 0.481633 & 0.518367 & 157 & 1187 & -1030 \\
 246 & 2^1 3^1 41^1 & \text{Y} & \text{N} & -16 & 0 & 1.0000000 & 0.479675 & 0.520325 & 141 & 1187 & -1046 \\
 247 & 13^1 19^1 & \text{Y} & \text{N} & 5 & 0 & 1.0000000 & 0.481781 & 0.518219 & 146 & 1192 & -1046 \\
 248 & 2^3 31^1 & \text{N} & \text{N} & 9 & 4 & 1.5555556 & 0.483871 & 0.516129 & 155 & 1201 & -1046 \\
 249 & 3^1 83^1 & \text{Y} & \text{N} & 5 & 0 & 1.0000000 & 0.485944 & 0.514056 & 160 & 1206 & -1046 \\
 250 & 2^1 5^3 & \text{N} & \text{N} & 9 & 4 & 1.5555556 & 0.488000 & 0.512000 & 169 & 1215 & -1046 \\
 251 & 251^1 & \text{Y} & \text{Y} & -2 & 0 & 1.0000000 & 0.486056 & 0.513944 & 167 & 1215 & -1048 \\
 252 & 2^2 3^2 7^1 & \text{N} & \text{N} & -74 & 58 & 1.2162162 & 0.484127 & 0.515873 & 93 & 1215 & -1122 \\
 253 & 11^1 23^1 & \text{Y} & \text{N} & 5 & 0 & 1.0000000 & 0.486166 & 0.513834 & 98 & 1220 & -1122 \\
 254 & 2^1 127^1 & \text{Y} & \text{N} & 5 & 0 & 1.0000000 & 0.488189 & 0.511811 & 103 & 1225 & -1122 \\
 255 & 3^1 5^1 17^1 & \text{Y} & \text{N} & -16 & 0 & 1.0000000 & 0.486275 & 0.513725 & 87 & 1225 & -1138 \\
 256 & 2^8 & \text{N} & \text{Y} & 2 & 0 & 4.5000000 & 0.488281 & 0.511719 & 89 & 1227 & -1138 \\
 257 & 257^1 & \text{Y} & \text{Y} & -2 & 0 & 1.0000000 & 0.486381 & 0.513619 & 87 & 1227 & -1140 \\
 258 & 2^1 3^1 43^1 & \text{Y} & \text{N} & -16 & 0 & 1.0000000 & 0.484496 & 0.515504 & 71 & 1227 & -1156 \\
 259 & 7^1 37^1 & \text{Y} & \text{N} & 5 & 0 & 1.0000000 & 0.486486 & 0.513514 & 76 & 1232 & -1156 \\
 260 & 2^2 5^1 13^1 & \text{N} & \text{N} & 30 & 14 & 1.1666667 & 0.488462 & 0.511538 & 106 & 1262 & -1156 \\
 261 & 3^2 29^1 & \text{N} & \text{N} & -7 & 2 & 1.2857143 & 0.486590 & 0.513410 & 99 & 1262 & -1163 \\
 262 & 2^1 131^1 & \text{Y} & \text{N} & 5 & 0 & 1.0000000 & 0.488550 & 0.511450 & 104 & 1267 & -1163 \\
 263 & 263^1 & \text{Y} & \text{Y} & -2 & 0 & 1.0000000 & 0.486692 & 0.513308 & 102 & 1267 & -1165 \\
 264 & 2^3 3^1 11^1 & \text{N} & \text{N} & -48 & 32 & 1.3333333 & 0.484848 & 0.515152 & 54 & 1267 & -1213 \\
 265 & 5^1 53^1 & \text{Y} & \text{N} & 5 & 0 & 1.0000000 & 0.486792 & 0.513208 & 59 & 1272 & -1213 \\
 266 & 2^1 7^1 19^1 & \text{Y} & \text{N} & -16 & 0 & 1.0000000 & 0.484962 & 0.515038 & 43 & 1272 & -1229 \\
 267 & 3^1 89^1 & \text{Y} & \text{N} & 5 & 0 & 1.0000000 & 0.486891 & 0.513109 & 48 & 1277 & -1229 \\
 268 & 2^2 67^1 & \text{N} & \text{N} & -7 & 2 & 1.2857143 & 0.485075 & 0.514925 & 41 & 1277 & -1236 \\
 269 & 269^1 & \text{Y} & \text{Y} & -2 & 0 & 1.0000000 & 0.483271 & 0.516729 & 39 & 1277 & -1238 \\
 270 & 2^1 3^3 5^1 & \text{N} & \text{N} & -48 & 32 & 1.3333333 & 0.481481 & 0.518519 & -9 & 1277 & -1286 \\
 271 & 271^1 & \text{Y} & \text{Y} & -2 & 0 & 1.0000000 & 0.479705 & 0.520295 & -11 & 1277 & -1288 \\
 272 & 2^4 17^1 & \text{N} & \text{N} & -11 & 6 & 1.8181818 & 0.477941 & 0.522059 & -22 & 1277 & -1299 \\
 273 & 3^1 7^1 13^1 & \text{Y} & \text{N} & -16 & 0 & 1.0000000 & 0.476190 & 0.523810 & -38 & 1277 & -1315 \\
 274 & 2^1 137^1 & \text{Y} & \text{N} & 5 & 0 & 1.0000000 & 0.478102 & 0.521898 & -33 & 1282 & -1315 \\
 275 & 5^2 11^1 & \text{N} & \text{N} & -7 & 2 & 1.2857143 & 0.476364 & 0.523636 & -40 & 1282 & -1322 \\
 276 & 2^2 3^1 23^1 & \text{N} & \text{N} & 30 & 14 & 1.1666667 & 0.478261 & 0.521739 & -10 & 1312 & -1322 \\
 277 & 277^1 & \text{Y} & \text{Y} & -2 & 0 & 1.0000000 & 0.476534 & 0.523466 & -12 & 1312 & -1324 \\ 
\end{array}
}
\end{equation*}

\end{table} 

\newpage
\begin{table}[h!]

\centering

\tiny
\begin{equation*}
\boxed{
\begin{array}{cc|cc|ccc|cc|ccc}
 n & \mathbf{Primes} & \mathbf{Sqfree} & \mathbf{PPower} & g^{-1}(n) & 
 \lambda(n) g^{-1}(n) - \widehat{f}_1(n) & 
 \frac{\sum_{d|n} C_{\Omega(d)}(d)}{|g^{-1}(n)|} & 
 \mathcal{L}_{+}(n) & \mathcal{L}_{-}(n) & 
 G^{-1}(n) & G^{-1}_{+}(n) & G^{-1}_{-}(n) \\ \hline 
 278 & 2^1 139^1 & \text{Y} & \text{N} & 5 & 0 & 1.0000000 & 0.478417 & 0.521583 & -7 & 1317 & -1324 \\
 279 & 3^2 31^1 & \text{N} & \text{N} & -7 & 2 & 1.2857143 & 0.476703 & 0.523297 & -14 & 1317 & -1331 \\
 280 & 2^3 5^1 7^1 & \text{N} & \text{N} & -48 & 32 & 1.3333333 & 0.475000 & 0.525000 & -62 & 1317 & -1379 \\
 281 & 281^1 & \text{Y} & \text{Y} & -2 & 0 & 1.0000000 & 0.473310 & 0.526690 & -64 & 1317 & -1381 \\
 282 & 2^1 3^1 47^1 & \text{Y} & \text{N} & -16 & 0 & 1.0000000 & 0.471631 & 0.528369 & -80 & 1317 & -1397 \\
 283 & 283^1 & \text{Y} & \text{Y} & -2 & 0 & 1.0000000 & 0.469965 & 0.530035 & -82 & 1317 & -1399 \\
 284 & 2^2 71^1 & \text{N} & \text{N} & -7 & 2 & 1.2857143 & 0.468310 & 0.531690 & -89 & 1317 & -1406 \\
 285 & 3^1 5^1 19^1 & \text{Y} & \text{N} & -16 & 0 & 1.0000000 & 0.466667 & 0.533333 & -105 & 1317 & -1422 \\
 286 & 2^1 11^1 13^1 & \text{Y} & \text{N} & -16 & 0 & 1.0000000 & 0.465035 & 0.534965 & -121 & 1317 & -1438 \\
 287 & 7^1 41^1 & \text{Y} & \text{N} & 5 & 0 & 1.0000000 & 0.466899 & 0.533101 & -116 & 1322 & -1438 \\
 288 & 2^5 3^2 & \text{N} & \text{N} & -47 & 42 & 1.7659574 & 0.465278 & 0.534722 & -163 & 1322 & -1485 \\
 289 & 17^2 & \text{N} & \text{Y} & 2 & 0 & 1.5000000 & 0.467128 & 0.532872 & -161 & 1324 & -1485 \\
 290 & 2^1 5^1 29^1 & \text{Y} & \text{N} & -16 & 0 & 1.0000000 & 0.465517 & 0.534483 & -177 & 1324 & -1501 \\
 291 & 3^1 97^1 & \text{Y} & \text{N} & 5 & 0 & 1.0000000 & 0.467354 & 0.532646 & -172 & 1329 & -1501 \\
 292 & 2^2 73^1 & \text{N} & \text{N} & -7 & 2 & 1.2857143 & 0.465753 & 0.534247 & -179 & 1329 & -1508 \\
 293 & 293^1 & \text{Y} & \text{Y} & -2 & 0 & 1.0000000 & 0.464164 & 0.535836 & -181 & 1329 & -1510 \\
 294 & 2^1 3^1 7^2 & \text{N} & \text{N} & 30 & 14 & 1.1666667 & 0.465986 & 0.534014 & -151 & 1359 & -1510 \\
 295 & 5^1 59^1 & \text{Y} & \text{N} & 5 & 0 & 1.0000000 & 0.467797 & 0.532203 & -146 & 1364 & -1510 \\
 296 & 2^3 37^1 & \text{N} & \text{N} & 9 & 4 & 1.5555556 & 0.469595 & 0.530405 & -137 & 1373 & -1510 \\
 297 & 3^3 11^1 & \text{N} & \text{N} & 9 & 4 & 1.5555556 & 0.471380 & 0.528620 & -128 & 1382 & -1510 \\
 298 & 2^1 149^1 & \text{Y} & \text{N} & 5 & 0 & 1.0000000 & 0.473154 & 0.526846 & -123 & 1387 & -1510 \\
 299 & 13^1 23^1 & \text{Y} & \text{N} & 5 & 0 & 1.0000000 & 0.474916 & 0.525084 & -118 & 1392 & -1510 \\
 300 & 2^2 3^1 5^2 & \text{N} & \text{N} & -74 & 58 & 1.2162162 & 0.473333 & 0.526667 & -192 & 1392 & -1584 \\
 301 & 7^1 43^1 & \text{Y} & \text{N} & 5 & 0 & 1.0000000 & 0.475083 & 0.524917 & -187 & 1397 & -1584 \\
 302 & 2^1 151^1 & \text{Y} & \text{N} & 5 & 0 & 1.0000000 & 0.476821 & 0.523179 & -182 & 1402 & -1584 \\
 303 & 3^1 101^1 & \text{Y} & \text{N} & 5 & 0 & 1.0000000 & 0.478548 & 0.521452 & -177 & 1407 & -1584 \\
 304 & 2^4 19^1 & \text{N} & \text{N} & -11 & 6 & 1.8181818 & 0.476974 & 0.523026 & -188 & 1407 & -1595 \\
 305 & 5^1 61^1 & \text{Y} & \text{N} & 5 & 0 & 1.0000000 & 0.478689 & 0.521311 & -183 & 1412 & -1595 \\
 306 & 2^1 3^2 17^1 & \text{N} & \text{N} & 30 & 14 & 1.1666667 & 0.480392 & 0.519608 & -153 & 1442 & -1595 \\
 307 & 307^1 & \text{Y} & \text{Y} & -2 & 0 & 1.0000000 & 0.478827 & 0.521173 & -155 & 1442 & -1597 \\
 308 & 2^2 7^1 11^1 & \text{N} & \text{N} & 30 & 14 & 1.1666667 & 0.480519 & 0.519481 & -125 & 1472 & -1597 \\
 309 & 3^1 103^1 & \text{Y} & \text{N} & 5 & 0 & 1.0000000 & 0.482201 & 0.517799 & -120 & 1477 & -1597 \\
 310 & 2^1 5^1 31^1 & \text{Y} & \text{N} & -16 & 0 & 1.0000000 & 0.480645 & 0.519355 & -136 & 1477 & -1613 \\
 311 & 311^1 & \text{Y} & \text{Y} & -2 & 0 & 1.0000000 & 0.479100 & 0.520900 & -138 & 1477 & -1615 \\
 312 & 2^3 3^1 13^1 & \text{N} & \text{N} & -48 & 32 & 1.3333333 & 0.477564 & 0.522436 & -186 & 1477 & -1663 \\
 313 & 313^1 & \text{Y} & \text{Y} & -2 & 0 & 1.0000000 & 0.476038 & 0.523962 & -188 & 1477 & -1665 \\
 314 & 2^1 157^1 & \text{Y} & \text{N} & 5 & 0 & 1.0000000 & 0.477707 & 0.522293 & -183 & 1482 & -1665 \\
 315 & 3^2 5^1 7^1 & \text{N} & \text{N} & 30 & 14 & 1.1666667 & 0.479365 & 0.520635 & -153 & 1512 & -1665 \\
 316 & 2^2 79^1 & \text{N} & \text{N} & -7 & 2 & 1.2857143 & 0.477848 & 0.522152 & -160 & 1512 & -1672 \\
 317 & 317^1 & \text{Y} & \text{Y} & -2 & 0 & 1.0000000 & 0.476341 & 0.523659 & -162 & 1512 & -1674 \\
 318 & 2^1 3^1 53^1 & \text{Y} & \text{N} & -16 & 0 & 1.0000000 & 0.474843 & 0.525157 & -178 & 1512 & -1690 \\
 319 & 11^1 29^1 & \text{Y} & \text{N} & 5 & 0 & 1.0000000 & 0.476489 & 0.523511 & -173 & 1517 & -1690 \\
 320 & 2^6 5^1 & \text{N} & \text{N} & -15 & 10 & 2.3333333 & 0.475000 & 0.525000 & -188 & 1517 & -1705 \\
 321 & 3^1 107^1 & \text{Y} & \text{N} & 5 & 0 & 1.0000000 & 0.476636 & 0.523364 & -183 & 1522 & -1705 \\
 322 & 2^1 7^1 23^1 & \text{Y} & \text{N} & -16 & 0 & 1.0000000 & 0.475155 & 0.524845 & -199 & 1522 & -1721 \\
 323 & 17^1 19^1 & \text{Y} & \text{N} & 5 & 0 & 1.0000000 & 0.476780 & 0.523220 & -194 & 1527 & -1721 \\
 324 & 2^2 3^4 & \text{N} & \text{N} & 34 & 29 & 1.6176471 & 0.478395 & 0.521605 & -160 & 1561 & -1721 \\
 325 & 5^2 13^1 & \text{N} & \text{N} & -7 & 2 & 1.2857143 & 0.476923 & 0.523077 & -167 & 1561 & -1728 \\
 326 & 2^1 163^1 & \text{Y} & \text{N} & 5 & 0 & 1.0000000 & 0.478528 & 0.521472 & -162 & 1566 & -1728 \\
 327 & 3^1 109^1 & \text{Y} & \text{N} & 5 & 0 & 1.0000000 & 0.480122 & 0.519878 & -157 & 1571 & -1728 \\
 328 & 2^3 41^1 & \text{N} & \text{N} & 9 & 4 & 1.5555556 & 0.481707 & 0.518293 & -148 & 1580 & -1728 \\
 329 & 7^1 47^1 & \text{Y} & \text{N} & 5 & 0 & 1.0000000 & 0.483283 & 0.516717 & -143 & 1585 & -1728 \\
 330 & 2^1 3^1 5^1 11^1 & \text{Y} & \text{N} & 65 & 0 & 1.0000000 & 0.484848 & 0.515152 & -78 & 1650 & -1728 \\
 331 & 331^1 & \text{Y} & \text{Y} & -2 & 0 & 1.0000000 & 0.483384 & 0.516616 & -80 & 1650 & -1730 \\
 332 & 2^2 83^1 & \text{N} & \text{N} & -7 & 2 & 1.2857143 & 0.481928 & 0.518072 & -87 & 1650 & -1737 \\
 333 & 3^2 37^1 & \text{N} & \text{N} & -7 & 2 & 1.2857143 & 0.480480 & 0.519520 & -94 & 1650 & -1744 \\
 334 & 2^1 167^1 & \text{Y} & \text{N} & 5 & 0 & 1.0000000 & 0.482036 & 0.517964 & -89 & 1655 & -1744 \\
 335 & 5^1 67^1 & \text{Y} & \text{N} & 5 & 0 & 1.0000000 & 0.483582 & 0.516418 & -84 & 1660 & -1744 \\
 336 & 2^4 3^1 7^1 & \text{N} & \text{N} & 70 & 54 & 1.5000000 & 0.485119 & 0.514881 & -14 & 1730 & -1744 \\
 337 & 337^1 & \text{Y} & \text{Y} & -2 & 0 & 1.0000000 & 0.483680 & 0.516320 & -16 & 1730 & -1746 \\
 338 & 2^1 13^2 & \text{N} & \text{N} & -7 & 2 & 1.2857143 & 0.482249 & 0.517751 & -23 & 1730 & -1753 \\
 339 & 3^1 113^1 & \text{Y} & \text{N} & 5 & 0 & 1.0000000 & 0.483776 & 0.516224 & -18 & 1735 & -1753 \\
 340 & 2^2 5^1 17^1 & \text{N} & \text{N} & 30 & 14 & 1.1666667 & 0.485294 & 0.514706 & 12 & 1765 & -1753 \\
 341 & 11^1 31^1 & \text{Y} & \text{N} & 5 & 0 & 1.0000000 & 0.486804 & 0.513196 & 17 & 1770 & -1753 \\
 342 & 2^1 3^2 19^1 & \text{N} & \text{N} & 30 & 14 & 1.1666667 & 0.488304 & 0.511696 & 47 & 1800 & -1753 \\
 343 & 7^3 & \text{N} & \text{Y} & -2 & 0 & 2.0000000 & 0.486880 & 0.513120 & 45 & 1800 & -1755 \\
 344 & 2^3 43^1 & \text{N} & \text{N} & 9 & 4 & 1.5555556 & 0.488372 & 0.511628 & 54 & 1809 & -1755 \\
 345 & 3^1 5^1 23^1 & \text{Y} & \text{N} & -16 & 0 & 1.0000000 & 0.486957 & 0.513043 & 38 & 1809 & -1771 \\
 346 & 2^1 173^1 & \text{Y} & \text{N} & 5 & 0 & 1.0000000 & 0.488439 & 0.511561 & 43 & 1814 & -1771 \\
 347 & 347^1 & \text{Y} & \text{Y} & -2 & 0 & 1.0000000 & 0.487032 & 0.512968 & 41 & 1814 & -1773 \\
 348 & 2^2 3^1 29^1 & \text{N} & \text{N} & 30 & 14 & 1.1666667 & 0.488506 & 0.511494 & 71 & 1844 & -1773 \\
 349 & 349^1 & \text{Y} & \text{Y} & -2 & 0 & 1.0000000 & 0.487106 & 0.512894 & 69 & 1844 & -1775 \\
 350 & 2^1 5^2 7^1 & \text{N} & \text{N} & 30 & 14 & 1.1666667 & 0.488571 & 0.511429 & 99 & 1874 & -1775 \\ 
\end{array}
}
\end{equation*}

\end{table} 

\newpage
\begin{table}[h!]

\centering
\tiny
\begin{equation*}
\boxed{
\begin{array}{cc|cc|ccc|cc|ccc}
 n & \mathbf{Primes} & \mathbf{Sqfree} & \mathbf{PPower} & g^{-1}(n) & 
 \lambda(n) g^{-1}(n) - \widehat{f}_1(n) & 
 \frac{\sum_{d|n} C_{\Omega(d)}(d)}{|g^{-1}(n)|} & 
 \mathcal{L}_{+}(n) & \mathcal{L}_{-}(n) & 
 G^{-1}(n) & G^{-1}_{+}(n) & G^{-1}_{-}(n) \\ \hline 
 351 & 3^3 13^1 & \text{N} & \text{N} & 9 & 4 & 1.5555556 & 0.490028 & 0.509972 & 108 & 1883 & -1775 \\
 352 & 2^5 11^1 & \text{N} & \text{N} & 13 & 8 & 2.0769231 & 0.491477 & 0.508523 & 121 & 1896 & -1775 \\
 353 & 353^1 & \text{Y} & \text{Y} & -2 & 0 & 1.0000000 & 0.490085 & 0.509915 & 119 & 1896 & -1777 \\
 354 & 2^1 3^1 59^1 & \text{Y} & \text{N} & -16 & 0 & 1.0000000 & 0.488701 & 0.511299 & 103 & 1896 & -1793 \\
 355 & 5^1 71^1 & \text{Y} & \text{N} & 5 & 0 & 1.0000000 & 0.490141 & 0.509859 & 108 & 1901 & -1793 \\
 356 & 2^2 89^1 & \text{N} & \text{N} & -7 & 2 & 1.2857143 & 0.488764 & 0.511236 & 101 & 1901 & -1800 \\
 357 & 3^1 7^1 17^1 & \text{Y} & \text{N} & -16 & 0 & 1.0000000 & 0.487395 & 0.512605 & 85 & 1901 & -1816 \\
 358 & 2^1 179^1 & \text{Y} & \text{N} & 5 & 0 & 1.0000000 & 0.488827 & 0.511173 & 90 & 1906 & -1816 \\
 359 & 359^1 & \text{Y} & \text{Y} & -2 & 0 & 1.0000000 & 0.487465 & 0.512535 & 88 & 1906 & -1818 \\
 360 & 2^3 3^2 5^1 & \text{N} & \text{N} & 145 & 129 & 1.3034483 & 0.488889 & 0.511111 & 233 & 2051 & -1818 \\
 361 & 19^2 & \text{N} & \text{Y} & 2 & 0 & 1.5000000 & 0.490305 & 0.509695 & 235 & 2053 & -1818 \\
 362 & 2^1 181^1 & \text{Y} & \text{N} & 5 & 0 & 1.0000000 & 0.491713 & 0.508287 & 240 & 2058 & -1818 \\
 363 & 3^1 11^2 & \text{N} & \text{N} & -7 & 2 & 1.2857143 & 0.490358 & 0.509642 & 233 & 2058 & -1825 \\
 364 & 2^2 7^1 13^1 & \text{N} & \text{N} & 30 & 14 & 1.1666667 & 0.491758 & 0.508242 & 263 & 2088 & -1825 \\
 365 & 5^1 73^1 & \text{Y} & \text{N} & 5 & 0 & 1.0000000 & 0.493151 & 0.506849 & 268 & 2093 & -1825 \\
 366 & 2^1 3^1 61^1 & \text{Y} & \text{N} & -16 & 0 & 1.0000000 & 0.491803 & 0.508197 & 252 & 2093 & -1841 \\
 367 & 367^1 & \text{Y} & \text{Y} & -2 & 0 & 1.0000000 & 0.490463 & 0.509537 & 250 & 2093 & -1843 \\
 368 & 2^4 23^1 & \text{N} & \text{N} & -11 & 6 & 1.8181818 & 0.489130 & 0.510870 & 239 & 2093 & -1854 \\
 369 & 3^2 41^1 & \text{N} & \text{N} & -7 & 2 & 1.2857143 & 0.487805 & 0.512195 & 232 & 2093 & -1861 \\
 370 & 2^1 5^1 37^1 & \text{Y} & \text{N} & -16 & 0 & 1.0000000 & 0.486486 & 0.513514 & 216 & 2093 & -1877 \\
 371 & 7^1 53^1 & \text{Y} & \text{N} & 5 & 0 & 1.0000000 & 0.487871 & 0.512129 & 221 & 2098 & -1877 \\
 372 & 2^2 3^1 31^1 & \text{N} & \text{N} & 30 & 14 & 1.1666667 & 0.489247 & 0.510753 & 251 & 2128 & -1877 \\
 373 & 373^1 & \text{Y} & \text{Y} & -2 & 0 & 1.0000000 & 0.487936 & 0.512064 & 249 & 2128 & -1879 \\
 374 & 2^1 11^1 17^1 & \text{Y} & \text{N} & -16 & 0 & 1.0000000 & 0.486631 & 0.513369 & 233 & 2128 & -1895 \\
 375 & 3^1 5^3 & \text{N} & \text{N} & 9 & 4 & 1.5555556 & 0.488000 & 0.512000 & 242 & 2137 & -1895 \\
 376 & 2^3 47^1 & \text{N} & \text{N} & 9 & 4 & 1.5555556 & 0.489362 & 0.510638 & 251 & 2146 & -1895 \\
 377 & 13^1 29^1 & \text{Y} & \text{N} & 5 & 0 & 1.0000000 & 0.490716 & 0.509284 & 256 & 2151 & -1895 \\
 378 & 2^1 3^3 7^1 & \text{N} & \text{N} & -48 & 32 & 1.3333333 & 0.489418 & 0.510582 & 208 & 2151 & -1943 \\
 379 & 379^1 & \text{Y} & \text{Y} & -2 & 0 & 1.0000000 & 0.488127 & 0.511873 & 206 & 2151 & -1945 \\
 380 & 2^2 5^1 19^1 & \text{N} & \text{N} & 30 & 14 & 1.1666667 & 0.489474 & 0.510526 & 236 & 2181 & -1945 \\
 381 & 3^1 127^1 & \text{Y} & \text{N} & 5 & 0 & 1.0000000 & 0.490814 & 0.509186 & 241 & 2186 & -1945 \\
 382 & 2^1 191^1 & \text{Y} & \text{N} & 5 & 0 & 1.0000000 & 0.492147 & 0.507853 & 246 & 2191 & -1945 \\
 383 & 383^1 & \text{Y} & \text{Y} & -2 & 0 & 1.0000000 & 0.490862 & 0.509138 & 244 & 2191 & -1947 \\
 384 & 2^7 3^1 & \text{N} & \text{N} & 17 & 12 & 2.5882353 & 0.492188 & 0.507812 & 261 & 2208 & -1947 \\
 385 & 5^1 7^1 11^1 & \text{Y} & \text{N} & -16 & 0 & 1.0000000 & 0.490909 & 0.509091 & 245 & 2208 & -1963 \\
 386 & 2^1 193^1 & \text{Y} & \text{N} & 5 & 0 & 1.0000000 & 0.492228 & 0.507772 & 250 & 2213 & -1963 \\
 387 & 3^2 43^1 & \text{N} & \text{N} & -7 & 2 & 1.2857143 & 0.490956 & 0.509044 & 243 & 2213 & -1970 \\
 388 & 2^2 97^1 & \text{N} & \text{N} & -7 & 2 & 1.2857143 & 0.489691 & 0.510309 & 236 & 2213 & -1977 \\
 389 & 389^1 & \text{Y} & \text{Y} & -2 & 0 & 1.0000000 & 0.488432 & 0.511568 & 234 & 2213 & -1979 \\
 390 & 2^1 3^1 5^1 13^1 & \text{Y} & \text{N} & 65 & 0 & 1.0000000 & 0.489744 & 0.510256 & 299 & 2278 & -1979 \\
 391 & 17^1 23^1 & \text{Y} & \text{N} & 5 & 0 & 1.0000000 & 0.491049 & 0.508951 & 304 & 2283 & -1979 \\
 392 & 2^3 7^2 & \text{N} & \text{N} & -23 & 18 & 1.4782609 & 0.489796 & 0.510204 & 281 & 2283 & -2002 \\
 393 & 3^1 131^1 & \text{Y} & \text{N} & 5 & 0 & 1.0000000 & 0.491094 & 0.508906 & 286 & 2288 & -2002 \\
 394 & 2^1 197^1 & \text{Y} & \text{N} & 5 & 0 & 1.0000000 & 0.492386 & 0.507614 & 291 & 2293 & -2002 \\
 395 & 5^1 79^1 & \text{Y} & \text{N} & 5 & 0 & 1.0000000 & 0.493671 & 0.506329 & 296 & 2298 & -2002 \\
 396 & 2^2 3^2 11^1 & \text{N} & \text{N} & -74 & 58 & 1.2162162 & 0.492424 & 0.507576 & 222 & 2298 & -2076 \\
 397 & 397^1 & \text{Y} & \text{Y} & -2 & 0 & 1.0000000 & 0.491184 & 0.508816 & 220 & 2298 & -2078 \\
 398 & 2^1 199^1 & \text{Y} & \text{N} & 5 & 0 & 1.0000000 & 0.492462 & 0.507538 & 225 & 2303 & -2078 \\
 399 & 3^1 7^1 19^1 & \text{Y} & \text{N} & -16 & 0 & 1.0000000 & 0.491228 & 0.508772 & 209 & 2303 & -2094 \\
 400 & 2^4 5^2 & \text{N} & \text{N} & 34 & 29 & 1.6176471 & 0.492500 & 0.507500 & 243 & 2337 & -2094 \\
 401 & 401^1 & \text{Y} & \text{Y} & -2 & 0 & 1.0000000 & 0.491272 & 0.508728 & 241 & 2337 & -2096 \\
 402 & 2^1 3^1 67^1 & \text{Y} & \text{N} & -16 & 0 & 1.0000000 & 0.490050 & 0.509950 & 225 & 2337 & -2112 \\
 403 & 13^1 31^1 & \text{Y} & \text{N} & 5 & 0 & 1.0000000 & 0.491315 & 0.508685 & 230 & 2342 & -2112 \\
 404 & 2^2 101^1 & \text{N} & \text{N} & -7 & 2 & 1.2857143 & 0.490099 & 0.509901 & 223 & 2342 & -2119 \\
 405 & 3^4 5^1 & \text{N} & \text{N} & -11 & 6 & 1.8181818 & 0.488889 & 0.511111 & 212 & 2342 & -2130 \\
 406 & 2^1 7^1 29^1 & \text{Y} & \text{N} & -16 & 0 & 1.0000000 & 0.487685 & 0.512315 & 196 & 2342 & -2146 \\
 407 & 11^1 37^1 & \text{Y} & \text{N} & 5 & 0 & 1.0000000 & 0.488943 & 0.511057 & 201 & 2347 & -2146 \\
 408 & 2^3 3^1 17^1 & \text{N} & \text{N} & -48 & 32 & 1.3333333 & 0.487745 & 0.512255 & 153 & 2347 & -2194 \\
 409 & 409^1 & \text{Y} & \text{Y} & -2 & 0 & 1.0000000 & 0.486553 & 0.513447 & 151 & 2347 & -2196 \\
 410 & 2^1 5^1 41^1 & \text{Y} & \text{N} & -16 & 0 & 1.0000000 & 0.485366 & 0.514634 & 135 & 2347 & -2212 \\
 411 & 3^1 137^1 & \text{Y} & \text{N} & 5 & 0 & 1.0000000 & 0.486618 & 0.513382 & 140 & 2352 & -2212 \\
 412 & 2^2 103^1 & \text{N} & \text{N} & -7 & 2 & 1.2857143 & 0.485437 & 0.514563 & 133 & 2352 & -2219 \\
 413 & 7^1 59^1 & \text{Y} & \text{N} & 5 & 0 & 1.0000000 & 0.486683 & 0.513317 & 138 & 2357 & -2219 \\
 414 & 2^1 3^2 23^1 & \text{N} & \text{N} & 30 & 14 & 1.1666667 & 0.487923 & 0.512077 & 168 & 2387 & -2219 \\
 415 & 5^1 83^1 & \text{Y} & \text{N} & 5 & 0 & 1.0000000 & 0.489157 & 0.510843 & 173 & 2392 & -2219 \\
 416 & 2^5 13^1 & \text{N} & \text{N} & 13 & 8 & 2.0769231 & 0.490385 & 0.509615 & 186 & 2405 & -2219 \\
 417 & 3^1 139^1 & \text{Y} & \text{N} & 5 & 0 & 1.0000000 & 0.491607 & 0.508393 & 191 & 2410 & -2219 \\
 418 & 2^1 11^1 19^1 & \text{Y} & \text{N} & -16 & 0 & 1.0000000 & 0.490431 & 0.509569 & 175 & 2410 & -2235 \\
 419 & 419^1 & \text{Y} & \text{Y} & -2 & 0 & 1.0000000 & 0.489260 & 0.510740 & 173 & 2410 & -2237 \\
 420 & 2^2 3^1 5^1 7^1 & \text{N} & \text{N} & -155 & 90 & 1.1032258 & 0.488095 & 0.511905 & 18 & 2410 & -2392 \\
 421 & 421^1 & \text{Y} & \text{Y} & -2 & 0 & 1.0000000 & 0.486936 & 0.513064 & 16 & 2410 & -2394 \\
 422 & 2^1 211^1 & \text{Y} & \text{N} & 5 & 0 & 1.0000000 & 0.488152 & 0.511848 & 21 & 2415 & -2394 \\
 423 & 3^2 47^1 & \text{N} & \text{N} & -7 & 2 & 1.2857143 & 0.486998 & 0.513002 & 14 & 2415 & -2401 \\
 424 & 2^3 53^1 & \text{N} & \text{N} & 9 & 4 & 1.5555556 & 0.488208 & 0.511792 & 23 & 2424 & -2401 \\
 425 & 5^2 17^1 & \text{N} & \text{N} & -7 & 2 & 1.2857143 & 0.487059 & 0.512941 & 16 & 2424 & -2408 \\ 
\end{array}
}
\end{equation*}

\end{table} 

\newpage

\begin{table}[h!]
\label{table_conjecture_Mertens_ginvSeq_approx_values_LastPage} 

\centering
\tiny
\begin{equation*}
\boxed{
\begin{array}{cc|cc|ccc|cc|ccc}
 n & \mathbf{Primes} & \mathbf{Sqfree} & \mathbf{PPower} & g^{-1}(n) & 
 \lambda(n) g^{-1}(n) - \widehat{f}_1(n) & 
 \frac{\sum_{d|n} C_{\Omega(d)}(d)}{|g^{-1}(n)|} & 
 \mathcal{L}_{+}(n) & \mathcal{L}_{-}(n) & 
 G^{-1}(n) & G^{-1}_{+}(n) & G^{-1}_{-}(n) \\ \hline 
 426 & 2^1 3^1 71^1 & \text{Y} & \text{N} & -16 & 0 & 1.0000000 & 0.485915 & 0.514085 & 0 & 2424 & -2424 \\
 427 & 7^1 61^1 & \text{Y} & \text{N} & 5 & 0 & 1.0000000 & 0.487119 & 0.512881 & 5 & 2429 & -2424 \\
 428 & 2^2 107^1 & \text{N} & \text{N} & -7 & 2 & 1.2857143 & 0.485981 & 0.514019 & -2 & 2429 & -2431 \\
 429 & 3^1 11^1 13^1 & \text{Y} & \text{N} & -16 & 0 & 1.0000000 & 0.484848 & 0.515152 & -18 & 2429 & -2447 \\
 430 & 2^1 5^1 43^1 & \text{Y} & \text{N} & -16 & 0 & 1.0000000 & 0.483721 & 0.516279 & -34 & 2429 & -2463 \\
 431 & 431^1 & \text{Y} & \text{Y} & -2 & 0 & 1.0000000 & 0.482599 & 0.517401 & -36 & 2429 & -2465 \\
 432 & 2^4 3^3 & \text{N} & \text{N} & -80 & 75 & 1.5625000 & 0.481481 & 0.518519 & -116 & 2429 & -2545 \\
 433 & 433^1 & \text{Y} & \text{Y} & -2 & 0 & 1.0000000 & 0.480370 & 0.519630 & -118 & 2429 & -2547 \\
 434 & 2^1 7^1 31^1 & \text{Y} & \text{N} & -16 & 0 & 1.0000000 & 0.479263 & 0.520737 & -134 & 2429 & -2563 \\
 435 & 3^1 5^1 29^1 & \text{Y} & \text{N} & -16 & 0 & 1.0000000 & 0.478161 & 0.521839 & -150 & 2429 & -2579 \\
 436 & 2^2 109^1 & \text{N} & \text{N} & -7 & 2 & 1.2857143 & 0.477064 & 0.522936 & -157 & 2429 & -2586 \\
 437 & 19^1 23^1 & \text{Y} & \text{N} & 5 & 0 & 1.0000000 & 0.478261 & 0.521739 & -152 & 2434 & -2586 \\
 438 & 2^1 3^1 73^1 & \text{Y} & \text{N} & -16 & 0 & 1.0000000 & 0.477169 & 0.522831 & -168 & 2434 & -2602 \\
 439 & 439^1 & \text{Y} & \text{Y} & -2 & 0 & 1.0000000 & 0.476082 & 0.523918 & -170 & 2434 & -2604 \\
 440 & 2^3 5^1 11^1 & \text{N} & \text{N} & -48 & 32 & 1.3333333 & 0.475000 & 0.525000 & -218 & 2434 & -2652 \\
 441 & 3^2 7^2 & \text{N} & \text{N} & 14 & 9 & 1.3571429 & 0.476190 & 0.523810 & -204 & 2448 & -2652 \\
 442 & 2^1 13^1 17^1 & \text{Y} & \text{N} & -16 & 0 & 1.0000000 & 0.475113 & 0.524887 & -220 & 2448 & -2668 \\
 443 & 443^1 & \text{Y} & \text{Y} & -2 & 0 & 1.0000000 & 0.474041 & 0.525959 & -222 & 2448 & -2670 \\
 444 & 2^2 3^1 37^1 & \text{N} & \text{N} & 30 & 14 & 1.1666667 & 0.475225 & 0.524775 & -192 & 2478 & -2670 \\
 445 & 5^1 89^1 & \text{Y} & \text{N} & 5 & 0 & 1.0000000 & 0.476404 & 0.523596 & -187 & 2483 & -2670 \\
 446 & 2^1 223^1 & \text{Y} & \text{N} & 5 & 0 & 1.0000000 & 0.477578 & 0.522422 & -182 & 2488 & -2670 \\
 447 & 3^1 149^1 & \text{Y} & \text{N} & 5 & 0 & 1.0000000 & 0.478747 & 0.521253 & -177 & 2493 & -2670 \\
 448 & 2^6 7^1 & \text{N} & \text{N} & -15 & 10 & 2.3333333 & 0.477679 & 0.522321 & -192 & 2493 & -2685 \\
 449 & 449^1 & \text{Y} & \text{Y} & -2 & 0 & 1.0000000 & 0.476615 & 0.523385 & -194 & 2493 & -2687 \\
 450 & 2^1 3^2 5^2 & \text{N} & \text{N} & -74 & 58 & 1.2162162 & 0.475556 & 0.524444 & -268 & 2493 & -2761 \\
 451 & 11^1 41^1 & \text{Y} & \text{N} & 5 & 0 & 1.0000000 & 0.476718 & 0.523282 & -263 & 2498 & -2761 \\
 452 & 2^2 113^1 & \text{N} & \text{N} & -7 & 2 & 1.2857143 & 0.475664 & 0.524336 & -270 & 2498 & -2768 \\
 453 & 3^1 151^1 & \text{Y} & \text{N} & 5 & 0 & 1.0000000 & 0.476821 & 0.523179 & -265 & 2503 & -2768 \\
 454 & 2^1 227^1 & \text{Y} & \text{N} & 5 & 0 & 1.0000000 & 0.477974 & 0.522026 & -260 & 2508 & -2768 \\
 455 & 5^1 7^1 13^1 & \text{Y} & \text{N} & -16 & 0 & 1.0000000 & 0.476923 & 0.523077 & -276 & 2508 & -2784 \\
 456 & 2^3 3^1 19^1 & \text{N} & \text{N} & -48 & 32 & 1.3333333 & 0.475877 & 0.524123 & -324 & 2508 & -2832 \\
 457 & 457^1 & \text{Y} & \text{Y} & -2 & 0 & 1.0000000 & 0.474836 & 0.525164 & -326 & 2508 & -2834 \\
 458 & 2^1 229^1 & \text{Y} & \text{N} & 5 & 0 & 1.0000000 & 0.475983 & 0.524017 & -321 & 2513 & -2834 \\
 459 & 3^3 17^1 & \text{N} & \text{N} & 9 & 4 & 1.5555556 & 0.477124 & 0.522876 & -312 & 2522 & -2834 \\
 460 & 2^2 5^1 23^1 & \text{N} & \text{N} & 30 & 14 & 1.1666667 & 0.478261 & 0.521739 & -282 & 2552 & -2834 \\
 461 & 461^1 & \text{Y} & \text{Y} & -2 & 0 & 1.0000000 & 0.477223 & 0.522777 & -284 & 2552 & -2836 \\
 462 & 2^1 3^1 7^1 11^1 & \text{Y} & \text{N} & 65 & 0 & 1.0000000 & 0.478355 & 0.521645 & -219 & 2617 & -2836 \\
 463 & 463^1 & \text{Y} & \text{Y} & -2 & 0 & 1.0000000 & 0.477322 & 0.522678 & -221 & 2617 & -2838 \\
 464 & 2^4 29^1 & \text{N} & \text{N} & -11 & 6 & 1.8181818 & 0.476293 & 0.523707 & -232 & 2617 & -2849 \\
 465 & 3^1 5^1 31^1 & \text{Y} & \text{N} & -16 & 0 & 1.0000000 & 0.475269 & 0.524731 & -248 & 2617 & -2865 \\
 466 & 2^1 233^1 & \text{Y} & \text{N} & 5 & 0 & 1.0000000 & 0.476395 & 0.523605 & -243 & 2622 & -2865 \\
 467 & 467^1 & \text{Y} & \text{Y} & -2 & 0 & 1.0000000 & 0.475375 & 0.524625 & -245 & 2622 & -2867 \\
 468 & 2^2 3^2 13^1 & \text{N} & \text{N} & -74 & 58 & 1.2162162 & 0.474359 & 0.525641 & -319 & 2622 & -2941 \\
 469 & 7^1 67^1 & \text{Y} & \text{N} & 5 & 0 & 1.0000000 & 0.475480 & 0.524520 & -314 & 2627 & -2941 \\
 470 & 2^1 5^1 47^1 & \text{Y} & \text{N} & -16 & 0 & 1.0000000 & 0.474468 & 0.525532 & -330 & 2627 & -2957 \\
 471 & 3^1 157^1 & \text{Y} & \text{N} & 5 & 0 & 1.0000000 & 0.475584 & 0.524416 & -325 & 2632 & -2957 \\
 472 & 2^3 59^1 & \text{N} & \text{N} & 9 & 4 & 1.5555556 & 0.476695 & 0.523305 & -316 & 2641 & -2957 \\
 473 & 11^1 43^1 & \text{Y} & \text{N} & 5 & 0 & 1.0000000 & 0.477801 & 0.522199 & -311 & 2646 & -2957 \\
 474 & 2^1 3^1 79^1 & \text{Y} & \text{N} & -16 & 0 & 1.0000000 & 0.476793 & 0.523207 & -327 & 2646 & -2973 \\
 475 & 5^2 19^1 & \text{N} & \text{N} & -7 & 2 & 1.2857143 & 0.475789 & 0.524211 & -334 & 2646 & -2980 \\
 476 & 2^2 7^1 17^1 & \text{N} & \text{N} & 30 & 14 & 1.1666667 & 0.476891 & 0.523109 & -304 & 2676 & -2980 \\
 477 & 3^2 53^1 & \text{N} & \text{N} & -7 & 2 & 1.2857143 & 0.475891 & 0.524109 & -311 & 2676 & -2987 \\
 478 & 2^1 239^1 & \text{Y} & \text{N} & 5 & 0 & 1.0000000 & 0.476987 & 0.523013 & -306 & 2681 & -2987 \\
 479 & 479^1 & \text{Y} & \text{Y} & -2 & 0 & 1.0000000 & 0.475992 & 0.524008 & -308 & 2681 & -2989 \\
 480 & 2^5 3^1 5^1 & \text{N} & \text{N} & -96 & 80 & 1.6666667 & 0.475000 & 0.525000 & -404 & 2681 & -3085 \\
 481 & 13^1 37^1 & \text{Y} & \text{N} & 5 & 0 & 1.0000000 & 0.476091 & 0.523909 & -399 & 2686 & -3085 \\
 482 & 2^1 241^1 & \text{Y} & \text{N} & 5 & 0 & 1.0000000 & 0.477178 & 0.522822 & -394 & 2691 & -3085 \\
 483 & 3^1 7^1 23^1 & \text{Y} & \text{N} & -16 & 0 & 1.0000000 & 0.476190 & 0.523810 & -410 & 2691 & -3101 \\
 484 & 2^2 11^2 & \text{N} & \text{N} & 14 & 9 & 1.3571429 & 0.477273 & 0.522727 & -396 & 2705 & -3101 \\
 485 & 5^1 97^1 & \text{Y} & \text{N} & 5 & 0 & 1.0000000 & 0.478351 & 0.521649 & -391 & 2710 & -3101 \\
 486 & 2^1 3^5 & \text{N} & \text{N} & 13 & 8 & 2.0769231 & 0.479424 & 0.520576 & -378 & 2723 & -3101 \\
 487 & 487^1 & \text{Y} & \text{Y} & -2 & 0 & 1.0000000 & 0.478439 & 0.521561 & -380 & 2723 & -3103 \\
 488 & 2^3 61^1 & \text{N} & \text{N} & 9 & 4 & 1.5555556 & 0.479508 & 0.520492 & -371 & 2732 & -3103 \\
 489 & 3^1 163^1 & \text{Y} & \text{N} & 5 & 0 & 1.0000000 & 0.480573 & 0.519427 & -366 & 2737 & -3103 \\
 490 & 2^1 5^1 7^2 & \text{N} & \text{N} & 30 & 14 & 1.1666667 & 0.481633 & 0.518367 & -336 & 2767 & -3103 \\
 491 & 491^1 & \text{Y} & \text{Y} & -2 & 0 & 1.0000000 & 0.480652 & 0.519348 & -338 & 2767 & -3105 \\
 492 & 2^2 3^1 41^1 & \text{N} & \text{N} & 30 & 14 & 1.1666667 & 0.481707 & 0.518293 & -308 & 2797 & -3105 \\
 493 & 17^1 29^1 & \text{Y} & \text{N} & 5 & 0 & 1.0000000 & 0.482759 & 0.517241 & -303 & 2802 & -3105 \\
 494 & 2^1 13^1 19^1 & \text{Y} & \text{N} & -16 & 0 & 1.0000000 & 0.481781 & 0.518219 & -319 & 2802 & -3121 \\
 495 & 3^2 5^1 11^1 & \text{N} & \text{N} & 30 & 14 & 1.1666667 & 0.482828 & 0.517172 & -289 & 2832 & -3121 \\
 496 & 2^4 31^1 & \text{N} & \text{N} & -11 & 6 & 1.8181818 & 0.481855 & 0.518145 & -300 & 2832 & -3132 \\
 497 & 7^1 71^1 & \text{Y} & \text{N} & 5 & 0 & 1.0000000 & 0.482897 & 0.517103 & -295 & 2837 & -3132 \\
 498 & 2^1 3^1 83^1 & \text{Y} & \text{N} & -16 & 0 & 1.0000000 & 0.481928 & 0.518072 & -311 & 2837 & -3148 \\
 499 & 499^1 & \text{Y} & \text{Y} & -2 & 0 & 1.0000000 & 0.480962 & 0.519038 & -313 & 2837 & -3150 \\
 500 & 2^2 5^3 & \text{N} & \text{N} & -23 & 18 & 1.4782609 & 0.480000 & 0.520000 & -336 & 2837 & -3173 \\  
\end{array}
}
\end{equation*}

\end{table} 

%\NBRef{A03-2020-04026}
%\NBRef{A04-2020-04026}

\newpage
\setcounter{section}{0}
\renewcommand{\thesection}{Appendix \Alph{section}}
\renewcommand{\thesubsection}{\Alph{section}.\arabic{subsection}}

\end{document}
