\documentclass[11pt,reqno,a4letter]{article} 

\usepackage{amsmath,amssymb,amsfonts,amscd}

\usepackage{url}
\usepackage{hyperref}
\usepackage[usenames,dvipsnames]{xcolor}
\hypersetup{
    colorlinks,
    linkcolor={black!63!darkgray},
    citecolor={blue!70!white},
    urlcolor={blue!80!white}
}
\usepackage{graphicx} 

\newcommand{\hlocalref}[1]{\hyperref[#1]{\ref{#1}}}

\usepackage{datetime} 
\usepackage{cancel}
\usepackage{soul} 
\usepackage{subcaption}
\captionsetup{format=hang,labelfont={bf},textfont={small,it}} 
\numberwithin{equation}{section} 
\numberwithin{figure}{section}
\numberwithin{table}{section}

\usepackage{tocloft}
\usepackage{framed} 

\usepackage{enumitem}
\setlist[itemize]{leftmargin=0.65in}

\usepackage{changepage}
\usepackage{rotating,adjustbox}

\usepackage{diagbox}
\newcommand{\trianglenk}[2]{$\diagbox{#1}{#2}$}
\newcommand{\trianglenkII}[2]{\diagbox{#1}{#2}}

\let\citep\cite

\newcommand{\undersetbrace}[2]{\underset{\displaystyle{#1}}{\underbrace{#2}}}

\newcommand{\gkpSI}[2]{\ensuremath{\genfrac{\lbrack}{\rbrack}{0pt}{}{#1}{#2}}} 
\newcommand{\gkpSII}[2]{\ensuremath{\genfrac{\lbrace}{\rbrace}{0pt}{}{#1}{#2}}}
\newcommand{\cf}{\textit{cf.\ }} 
\newcommand{\Iverson}[1]{\ensuremath{\left[#1\right]_{\delta}}} 
\newcommand{\floor}[1]{\left\lfloor #1 \right\rfloor} 
\newcommand{\ceiling}[1]{\left\lceil #1 \right\rceil} 
\newcommand{\e}[1]{e\left(#1\right)} 
\newcommand{\seqnum}[1]{\href{http://oeis.org/#1}{\color{ProcessBlue}{\underline{#1}}}}

\usepackage{upgreek,dsfont,amssymb}
\renewcommand{\chi}{\upchi}
\newcommand{\ChiFunc}[1]{\ensuremath{\chi_{\{#1\}}}}
\newcommand{\OneFunc}[1]{\ensuremath{\mathds{1}_{#1}}}

%\usepackage{mathabx}
\makeatletter
\newcommand*\rel@kern[1]{\kern#1\dimexpr\macc@kerna}
\newcommand*\widebar[1]{%
  \begingroup
  \def\mathaccent##1##2{%
    \rel@kern{0.8}%
    \overline{\rel@kern{-0.8}\macc@nucleus\rel@kern{0.2}}%
    \rel@kern{-0.2}%
  }%
  \macc@depth\@ne
  \let\math@bgroup\@empty \let\math@egroup\macc@set@skewchar
  \mathsurround\z@ \frozen@everymath{\mathgroup\macc@group\relax}%
  \macc@set@skewchar\relax
  \let\mathaccentV\macc@nested@a
  \macc@nested@a\relax111{#1}%
  \endgroup
}

\usepackage{MnSymbol}
\newcommand{\gkpEII}[2]{\ensuremath{\genfrac{\llangle}{\rrangle}{0pt}{}{#1}{#2}}}

\usepackage{ifthen}
\newcommand{\Hn}[2]{
     \ifthenelse{\equal{#2}{1}}{H_{#1}}{H_{#1}^{\left(#2\right)}}
}

\newcommand{\Floor}[2]{\ensuremath{\left\lfloor \frac{#1}{#2} \right\rfloor}}
\newcommand{\Ceiling}[2]{\ensuremath{\left\lceil \frac{#1}{#2} \right\rceil}}

\DeclareMathOperator{\DGF}{DGF} 
\DeclareMathOperator{\ds}{ds} 
\DeclareMathOperator{\Id}{Id}
\DeclareMathOperator{\fg}{fg}
\DeclareMathOperator{\Div}{div}
\DeclareMathOperator{\rpp}{rpp}
\DeclareMathOperator{\logll}{\ell\ell}

\title{
       \LARGE{
       Exact formulas for partial sums of the M\"obius function expressed by 
       partial sums weighted by the Liouville lambda function
       } 
}
\author{{\Large Maxie Dion Schmidt} \\[0.1cm]  
        {\normalsize \href{mailto:maxieds@gmail.com}{maxieds@gmail.com}} \\[0.025cm] 
        {\normalsize \href{mailto:mschmidt34@gatech.edu}{mschmidt34@gatech.edu}} \\[0.025cm] 
        {\normalsize Georgia Institute of Technology} \\[0.025cm] 
        {\normalsize School of Mathematics} 
} 

%\date{\small\underline{Last Revised:} \today \ @\ \hhmmsstime{} \ -- \ Compiled with \LaTeX2e} 

\usepackage{amsthm} 

\theoremstyle{plain} 
\newtheorem{theorem}{Theorem}
\newtheorem{conjecture}[theorem]{Conjecture}
\newtheorem{claim}[theorem]{Claim}
\newtheorem{prop}[theorem]{Proposition}
\newtheorem{lemma}[theorem]{Lemma}
\newtheorem{cor}[theorem]{Corollary}
\numberwithin{theorem}{section}
\newtheorem*{conjecture*}{Conjecture}

\theoremstyle{definition} 
\newtheorem{example}[theorem]{Example}
\newtheorem{remark}[theorem]{Remark}
\newtheorem{definition}[theorem]{Definition}
\newtheorem{notation}[theorem]{Notation}
\newtheorem{question}[theorem]{Question}
\newtheorem{discussion}[theorem]{Discussion}
\newtheorem{facts}[theorem]{Facts}
\newtheorem{summary}[theorem]{Summary}
\newtheorem{heuristic}[theorem]{Heuristic}
\newtheorem{observation}[theorem]{Observation}
\newtheorem{ansatz}[theorem]{Ansatz}

\renewcommand{\arraystretch}{1.25} 
\usepackage[total={7in, 9.5in},tmargin=0.75in,headsep=8pt,footskip=30pt,footnotesep=0.5in]{geometry}

\usepackage{fancyhdr}
\pagestyle{empty}
\pagestyle{fancy}
\fancyhead[RO,RE]{Maxie Dion Schmidt -- \today} 
\fancyhead[LO,LE]{}
\fancyheadoffset{0.005\textwidth} 

\setlength{\parindent}{0em}
\setlength{\parskip}{2.5em} 

\renewcommand{\thefootnote}{\textbf{\arabic{footnote}}}

\renewcommand{\Re}{\operatorname{Re}}
\renewcommand{\Im}{\operatorname{Im}}

\usepackage{tikz}
\usetikzlibrary{shapes,arrows}

\usepackage{longtable}
\usepackage{arydshln} 
\usepackage[symbols,nogroupskip,nomain,automake=true,nonumberlist,section=section]{glossaries-extra}
\usepackage{glossary-mcols} 

\defglsdisplayfirst[main]{#1#4\protect\footnote{#2}}

%%%%%%%%%%%%

\providecommand{\glossarytoctitle}{\glossaryname}
\setlength{\glsdescwidth}{0.7\textwidth}

\newglossarystyle{glossstyleSymbol}{%
\renewenvironment{theglossary}%
 {\begin{longtable}{lp{\glsdescwidth}}}%
 {\end{longtable}}%
 \setlength{\parskip}{3.5pt}
 \renewcommand{\glsgroupskip}{}
\renewcommand*\glspostdescription{\dotfill}
\renewcommand*{\glossaryheader}{%
 \bfseries Symbols & \bfseries Definition
 \\\endhead}%
 \renewcommand*{\glsgroupheading}[1]{}%
  \renewcommand{\glossentry}[2]{%
    \glstarget{##1}{\glossentrysymbol{##1}} &
    \glossentrydesc{##1} \tabularnewline
  }%
  \renewcommand*{\glspostdescription}{}
  \renewcommand{\glossarymark}[1]{}
}

\setglossarystyle{glossstyleSymbol}
\makeglossaries

%%%%%%%%%%%%

\newglossaryentry{fCvlg}{
    symbol={\ensuremath{f \ast g}},
    sort={fg},
    description={The Dirichlet convolution of any two arithmetic functions 
    $f$ and $g$ at $n$ is defined to be 
    the divisor sum $(f \ast g)(n) := \sum\limits_{d|n} f(d) g\left(\frac{n}{d}\right)$ 
    for $n \geq 1$. 
    },
    type={symbols},
    name={Dirichlet convolution}
    }
\newglossaryentry{coeffExtraction}{
    symbol={\ensuremath{[q^n] F(q)}},
    sort={coeffExtraction},
    description={The coefficient of $q^n$ in the power series expansion of $F(q)$ about zero when 
    $F(q)$ is treated as the ordinary generating function (OGF) of a sequence, $\{f_n\}_{n \geq 0}$. 
    Namely, for integers $n \geq 0$ we define $[q^n] F(q) = f_n$ whenever 
    $F(q) := \sum\limits_{n \geq 0} f_n q^n$. },
    type={symbols},
    name={Series coefficient extraction}
    }
\newglossaryentry{MoebiusMuFunc}{
    symbol={\ensuremath{\mu(n),M(x)}},
    sort={MoebiusMuFunc},
    description={The M\"obius function defined such that $\mu^2(n)$ is the indicator function of the 
                 squarefree integers $n \geq 1$ where 
                 $\mu(n) = (-1)^{\omega(n)}$ whenever $n$ is squarefree. 
                 The Mertens function is the summatory function defined for all integers 
                 $x \geq 1$ by the partial sums $M(x) := \sum\limits_{n \leq x} \mu(n)$.
                 },
    type={symbols},
    name={M\"obius function}
    }
\newglossaryentry{Iverson}{
    symbol={\ensuremath{\Iverson{n=k}},\ensuremath{\Iverson{\mathtt{cond}}}},
    sort={Iverson},
    description={The symbol $\Iverson{n=k}$ is a synonym for $\delta_{n,k}$ 
                 which is one if and only if $n = k$, and is zero otherwise. 
                 For Boolean-valued conditions, \texttt{cond}, the symbol $\Iverson{\mathtt{cond}}$ 
                 evaluates to one precisely when \texttt{cond} is true or to zero otherwise.},
    type={symbols},
    name={Iverson's convention}
    }
\newglossaryentry{epsilonN}{
    symbol={\ensuremath{\varepsilon(n)}},
    sort={epsilonN},
    description={The multiplicative identity with respect to Dirichlet convolution, $\varepsilon(n) := \delta_{n,1}$, 
                 defined such that for any arithmetic function $f$ we have that 
                 $f \ast \varepsilon = \varepsilon \ast f = f$ where the operation 
                 $\ast$ denotes Dirichlet convolution. },
    type={symbols},
    name={Dirichlet multiplicative identity}
    }
\newglossaryentry{Zetas}{
    symbol={\ensuremath{\zeta(s)}},
    sort={Zetas},
    description={The Riemann zeta function is defined by 
                 $\zeta(s) := \sum\limits_{n \geq 1} n^{-s}$ when $\Re(s) > 1$, 
                 and by analytic continuation to any $s \in \mathbb{C}$ with the exception of a 
                 simple pole at $s = 1$ of residue one.},
    type={symbols},
    name={Riemann zeta function}
    }
\newglossaryentry{fInvn}{
     symbol={\ensuremath{f^{-1}(n)}},
    sort={fInvn},
    description={
     The Dirichlet inverse $f^{-1}$ of an arithmetic function $f$ exists 
     if and only if $f(1) \neq 0$. 
     The Dirichlet inverse of any $f$ such that $f(1) \neq 0$ 
     is defined recursively by 
     $f^{-1}(n) = -\frac{1}{f(1)} \times \sum\limits_{\substack{d|n \\ d>1}} f(d) f^{-1}\left(\frac{n}{d}\right)$ 
     for $n \geq 2$ with $f^{-1}(1) = f(1)^{-1}$. 
     When it exists, this inverse function 
     is unique and satisfies  $f^{-1} \ast f = f \ast f^{-1} = \varepsilon$.},
    type={symbols},
    name={Dirichlet inverse of $f$}
    }
\newglossaryentry{CkngInvAuxFunc}{
    symbol={$C_k(n),C_{\Omega}(n)$},
    sort={CkngInvAuxFunc},
    description={The first sequence is defined recursively for integers $n \geq 1$ and $k \geq 0$ as follows: 
                 \[
                 C_k(n) := \begin{cases} 
                      \delta_{n,1}, & \text{ if $k = 0$; } \\ 
                      \sum\limits_{d|n} \omega(d) C_{k-1}\left(\frac{n}{d}\right), & \text{ if $k \geq 1$. } 
                      \end{cases} 
                 \]
                 It represents the multiple ($k$-fold) convolution of the function $\omega(n)$ 
                 with itself. 
                 The function $C_{\Omega}(n) := C_{\Omega(n)}(n)$ has the DGF 
                 $(1-P(s))^{-1}$ for $\Re(s) > 1$. 
                 },
    type={symbols},
    name={Dirichlet inverse component functions}
    }
\newglossaryentry{gInvn}{
    symbol={$g(n),G(x),|G|(x)$},
    sort={gInvn},
    description={The Dirichlet inverse function, $g(n) = (\omega+1)^{-1}(n)$, has the 
                 summatory function $G(x) := \sum\limits_{n \leq x} g(n)$ for $x \geq 1$. 
                 We define the partial sums of the unsigned inverse function to be 
                 $|G|(x) := \sum_{n \leq x} |g(n)|$ for $x \geq 1$. },
    type={symbols},
    name={Key Dirichlet inverse functions}
    }
\newglossaryentry{PikPiHatkx}{
    symbol={$\pi_k(x),\widehat{\pi}_k(x)$},
    sort={PikPiHatkx},
    description={For integers $k \geq 1$, the 
                 function $\pi_k(x)$ denotes the number of 
                 $2 \leq n \leq x$ with 
                 exactly $k$ distinct prime factors: $\pi_k(x) := \#\{2 \leq n \leq x: \omega(n) = k\}$. 
                 Similarly, the function 
                 $\widehat{\pi}_k(x) := \#\{2 \leq n \leq x: \Omega(n) = k\}$ for $x \geq 2$ and fixed $k \geq 1$. },
    type={symbols},
    name={Distinct prime counting functions}
    }   
\newglossaryentry{primeOmegaFunctions}{
    symbol={$\omega(n)$,$\Omega(n)$}, 
    sort={OmegaPrimeOmegaFunctions},
    description={We define the strongly additive function 
                 $\omega(n) := \sum\limits_{p|n} 1$ and the completely additive function 
                 $\Omega(n) := \sum\limits_{p^{\alpha} || n} \alpha$. This means that if the prime 
                 factorization of any $n \geq 2$ is 
                 given by $n := p_1^{\alpha_1} \times \cdots \times p_r^{\alpha_r}$ 
                 with $p_i \neq p_j$ for all $i \neq j$, 
                 then $\omega(n) = r$ and $\Omega(n) = \alpha_1 + \cdots + \alpha_r$. 
                 We set $\omega(1) = \Omega(1) = 0$ by convention.},
    type={symbols},
    name={Prime omega functions}
    }
\newglossaryentry{LiouvilleLambdaFunc}{
     symbol={$\lambda(n), L(x)$}, 
    sort={LiouvilleLambdaFunc},
    description={The Liouville lambda function is the completely multiplicative function defined by 
                 $\lambda(n) := (-1)^{\Omega(n)}$. 
                 Its summatory function is defined by the partial sums 
                 $L(x) := \sum\limits_{n \leq x} \lambda(n)$ for $x \geq 1$. 
                 },
    type={symbols},
    name={Liouville lambda function}
    }
\newglossaryentry{QxSummatoryFunc}{
    symbol={$Q(x)$},
    sort={QxSummatoryFunc},
    description={For $x \geq 1$, we define $Q(x)$ to be the summatory function indicating the number of 
                 squarefree integers $n \leq x$. }, 
    type={symbols},
    name={Summatory function of the squarefree integers}
    }
\newglossaryentry{AApproxSimGGLLRelations}{
    symbol={$\gg,\ll,\asymp,\sim$},
    sort={AApproxSimGGLLRelations},
    description={
                 For functions $A,B$, the notation $A \ll B$ implies that $A = O(B)$. 
                 Similarly, for $B \geq 0$ the notation $A \gg B$ implies that $B = O(A)$. 
                 When we have that $A, B \geq 0$, $A \ll B$ and $B \ll A$, we write $A \asymp B$. 
                 Two arithmetic functions $A(x), B(x)$ satisfy the relation $A \sim B$ if 
                 $\lim_{x \rightarrow \infty} \frac{A(x)}{B(x)} = 1$. },
    type={symbols},
    name={Asymptotic relation symbols}
    }
\newglossaryentry{chiPrimeP}{
    symbol={$\chi_{\mathbb{P}}(n), P(s)$},
    sort={chiPrimeP},
    description={The indicator function of the primes equals one if and only if 
                 $n \in \mathbb{Z}^{+}$ is prime and is defined to be 
                 zero-valued otherwise. 
                 For any $s \in \mathbb{C}$ such that $\Re(s) > 1$, 
                 we define the prime zeta function to be the 
                 Dirichlet generating function (DGF) defined by 
                 $P(s) = \sum\limits_{n \geq 1} \frac{\chi_{\mathbb{P}}(n)}{n^s}$. 
                 The function $P(s)$ has an analytic continuation to the half-plane 
                 $\Re(s) > 0$ with the exception of $s = 1$ through the formula 
                 $P(s) = \sum\limits_{k \geq 1} \frac{\mu(k)}{k} \log\zeta(ks)$. The DGF $P(s)$ 
                 poles at the reciprocal of each positive integer and a natural boundary 
                 at the line $\Re(s) = 0$. },
    type={symbols},
    name={Prime set indicator function}
    }
\newglossaryentry{WLambertWFunction}{
    symbol={$W(x)$},
    sort={WLambertWFunction},
    description={For $x,y \in [0, \infty)$, we write that $x = W(y)$ if and only if $xe^{x} = y$. 
                 This function denotes the principal branch of the multi-valued Lambert $W$ function 
                 taken over the non-negative reals. },
    type={symbols},
    name={Lambert $W$-Function}
    }
\newglossaryentry{GGTildeFHatGHatBivariateFunctions}{
    symbol={$\mathcal{G}(z),\widetilde{\mathcal{G}}(z)$; $\widehat{F}(s, z),\widehat{\mathcal{G}}(z)$},
    sort={GGTildeFHatGHatBivariateFunctions},
    description={The functions $\mathcal{G}(z)$ and $\widetilde{\mathcal{G}}(z)$ are defined for 
                 $0 \leq |z| \leq R < 2$ on page 
                 \pageref{subSection_TheKnownDistsOfThePrimeOmegaFunctions_IntroResults_v1} of 
                 Appendix \hlocalref{subSection_TheKnownDistsOfThePrimeOmegaFunctions_IntroResults_v1}. 
                 The related constructions 
                 used to motivate the definitions of 
                 $\widehat{F}(s, z)$ and $\widehat{\mathcal{G}}(z)$ are defined 
                 by the infinite products given on pages 
                 \pageref{prop_HatAzx_ModSummatoryFuncExps_RelatedToCkn} and 
                 \pageref{theorem_CnkSpCasesScaledSummatoryFuncs} of 
                 Section \hlocalref{subSection_Section4_AnalyticPrerequisiteProofsOfUniformBoundsOnCertainPartialSumTypes_v1}, 
                 respectively. },
    type={symbols},
    name={Bivariate DGFs}
    }
\newglossaryentry{GammaIncompleteGamma}{
    symbol={$\Gamma(a, z)$},
    sort={GammaIncompleteGamma},
    description={The incomplete gamma function is defined as $\Gamma(a, z) := \int_z^{\infty} t^{a-1} e^{-t} dt$ 
		 by continuation for $a \in \mathbb{R}$ and $|\operatorname{arg}(z)| < \pi$. }, 
    type={symbols},
    name={Incomplete gamma function}
    }
\newglossaryentry{NormalCDFFunc}{
    symbol={$\Phi(z)$},
    sort={NormalCDFFunc},
    description={For $z \in \mathbb{R}$, we take the cumulative density function 
                 of the standard normal distribution to be denoted by 
                 $\Phi(z) := \frac{1}{\sqrt{2\pi}} \times \int\limits_{-\infty}^{z} e^{-\frac{t^2}{2}} dt$. 
                },
    type={symbols},
    name={Asymptotic relation symbol}
    }

\glsaddall[types={symbols}]

\allowdisplaybreaks 

\begin{document} 

\maketitle
\newcommand{\runtitle}{New exact formulas for partial sums of the M\"obius function}
\lhead{Maxie Dion Schmidt}
\rhead{\runtitle}

\begin{abstract} 
\noindent  
The Mertens function, $M(x) := \sum_{n \leq x} \mu(n)$, is 
defined as the summatory function of the classical M\"obius function for $x \geq 1$.
The Dirichlet inverse function $g(n) := (\omega+\mathds{1})^{-1}(n)$
is defined in terms of the shifted strongly additive function $\omega(n)$ that counts the 
number of distinct prime factors of $n$ without multiplicity. 
Discrete convolutions of the partial sums of $g(n)$ with the prime counting function 
provide new exact formulas for $M(x)$ that are weighted sums of the Liouville function 
involving $|g(n)|$ for $n \leq x$. 
We study the distribution of the unsigned function $|g(n)|$ whose 
Dirichlet generating function (DGF) is $\zeta(2s)^{-1}(1-P(s))^{-1}$ 
through the auxiliary unsigned sequence $C_{\Omega}(n)$ whose DGF is given by 
$(1-P(s))^{-1}$ for $\Re(s) > 1$ where $P(s) = \sum_p p^{-s}$ is the 
prime zeta function. 
We prove formulas for the average order of both 
$\log C_{\Omega}(n)$ and $\log |g(n)|$ and conjecture a central limit theorem 
for the distribution of their values over $n \leq x$ as $x \rightarrow \infty$. 

\bigskip 
\noindent
\textbf{Keywords and Phrases:} {\it M\"obius function; Mertens function; 
                                    Dirichlet inverse; Liouville lambda function; prime omega function; 
                                    prime counting function; Dirichlet generating function; 
                                    prime zeta function. } \\ 
% 11-XX			Number theory
%    11A25  	Arithmetic functions; related numbers; inversion formulas
%    11Y70  	Values of arithmetic functions; tables
%    11-04  	Software, source code, etc. for problems pertaining to number theory
% 11Nxx		Multiplicative number theory
%    11N05  	Distribution of primes
%    11N37  	Asymptotic results on arithmetic functions
%    11N56  	Rate of growth of arithmetic functions
%    11N60  	Distribution functions associated with additive and positive multiplicative functions
%    11N64  	Other results on the distribution of values or the characterization of arithmetic functions
\textbf{Math Subject Classifications (2010):} {\it 11N37; 11A25; 11N60; 11N64; and 11-04. } 
\end{abstract} 

\newpage
\renewcommand{\contentsname}{Table of Contents}
\setcounter{tocdepth}{2}
\tableofcontents

\newpage
\section{Introduction} 
\label{subSection_MertensMxClassical_Intro} 
\label{example_InvertingARecRelForMx_Intro}

The Mertens function is the summatory function of $\mu(n)$ defined by the partial sums 
\cite[\seqnum{A008683}; \seqnum{A002321}]{OEIS} 
\begin{align} 
M(x) & = \sum_{n \leq x} \mu(n), \text{ for } x \geq 1. 
\end{align} 
The partial sums of the Liouville lambda function are defined by 
\cite[\seqnum{A008836}; \seqnum{A002819}]{OEIS}
\begin{equation}
\label{eqn_LxSummatoryFuncDef_v1}
L(x) := \sum\limits_{n \leq x} \lambda(n), \text{ for } x \geq 1. 
\end{equation}
The Mertens function is related to the partial sums in 
\eqref{eqn_LxSummatoryFuncDef_v1} 
via the relation \cite{HUMPHRIES-JNT-2013,LEHMAN-1960} 
\begin{equation}
\label{eqn_MxInTermsOfLx_v1} 
M(x) = \sum_{d \leq \sqrt{x}} \mu(d) L\left(\Floor{x}{d^2}\right), \text{ for } x \geq 1.
\end{equation}
For any arithmetic functions $f$ and $h$, we define their Dirichlet convolution at $n \geq 1$ by 
\[
(f \ast h)(n) := \sum_{d|n} f(d) h\left(\frac{n}{d}\right).
\]
The arithmetic function $f$ has a unique inverse with respect to Dirichlet convolution, 
denoted by $f^{-1}$, that satisfies $(f \ast f^{-1})(n) = (f^{-1} \ast f)(n) = \delta_{n,1}$ 
if and only if $f(1) \neq 0$. 
We fix the notation for the Dirichlet inverse function \cite[\seqnum{A341444}]{OEIS} 
\begin{equation}
\label{eqn_gInvn_def_v1}
g(n) := (\omega + \mathds{1})^{-1}(n), \text{ for } n \geq 1, 
\end{equation}
where $\omega(n)$ is the strongly additive function that 
counts the number of distinct prime factors of $n$ without multiplicity. 
We use the notation $|g(n)|$ to denote the absolute value of $g(n)$ where the sign of $g(n)$ is 
given by $\lambda(n)$ for all $n \geq 1$ 
(see Proposition \hlocalref{prop_SignageDirInvsOfPosBddArithmeticFuncs_v1}). 
We define the partial sums $G(x)$ for integers $x \geq 1$ as follows \cite[\seqnum{A341472}]{OEIS}: 
\begin{equation}
\label{eqn_GInvx_PartialSumForms_v1} 
G(x) := \sum_{n \leq x} g(n) = \sum_{n \leq x} \lambda(n) |g(n)|. 
\end{equation} 

\begin{theorem} 
\label{prop_Mx_SBP_IntegralFormula} 
\begin{subequations}
For all $x \geq 1$ 
\begin{align} 
\label{prop_Mx_SBP_IntegralFormula_PartA} 
M(x) & = G(x) + \sum_{1 \leq k \leq x} |g(k)| \pi\left(\Floor{x}{k}\right) \lambda(k), \\ 
\label{prop_Mx_SBP_IntegralFormula_PartB} 
M(x) & = G(x) + 
     \sum_{1 \leq k \leq \frac{x}{2}} \left(
     \pi\left(\Floor{x}{k}\right) - \pi\left(\Floor{x}{k+1}\right) 
	\right) G(k), \\ 
\label{eqn_RmkInitialConnectionOfMxToGInvx_ProvedByInversion_v1} 
M(x) & = G(x) + \sum_{p \leq x} G\left(\Floor{x}{p}\right). 
\end{align} 
\end{subequations}
\end{theorem}

The relation in \eqref{eqn_MxInTermsOfLx_v1} 
gives an exact expression for $M(x)$ with summands involving $L(x)$ that are oscillatory. 
In contrast, the exact expansions for the Mertens function given in 
Theorem \hlocalref{prop_Mx_SBP_IntegralFormula} 
express $M(x)$ as finite sums over $\lambda(n)$ with weight coefficients that are unsigned. 
For $n \geq 2$, let the function 
$\mathcal{E}[n] \vdash (\alpha_1, \alpha_2, \ldots, \alpha_r)$ denote the unordered 
partition of exponents for which 
$n = p_1^{\alpha_1} \times \cdots \times p_r^{\alpha_r}$ is the factorization of 
$n$ into powers of distinct primes. 
For any $n_1,n_2 \geq 2$ we have that 
\begin{equation}
\label{eqn_FactSymmPropertyOfgn_v1} 
\mathcal{E}[n_1] = \mathcal{E}[n_2] \implies g(n_1) = g(n_2). 
\end{equation}
The property of the symmetry of the distinct values of $|g(n)|$ with respect to the 
prime factorizations of $n \geq 2$ in \eqref{eqn_FactSymmPropertyOfgn_v1} 
shows that \`{a} priori the unsigned weights on $\lambda(n)$ in 
the new formulas from the theorem are comparatively easier to work with than the known 
exact expressions for $M(x)$ in terms of $L(x)$. 
Stating tight bounds on the distribution of 
$L(x)$ is a problem that is equally as difficult 
as understanding the properties of $M(x)$ well at large $x$ or 
along infinite subsequences (\cf \cite{MR2877066,MR3779960}). 

An exact expression for $g(n)$ is given by 
(see Lemma \hlocalref{lemma_AnExactFormulaFor_gInvByMobiusInv_v1} and 
Corollary \hlocalref{lemma_AbsValueOf_gInvn_FornSquareFree_v1}) 
\begin{equation}
\label{eqn_gInvn_ExactDivisorSumFormula_WithSgnWeight_v1} 
\lambda(n) g(n) = \sum_{d|n} \mu^2\left(\frac{n}{d}\right) C_{\Omega}(d), n \geq 1. 
\end{equation}
The sequence $\lambda(n) C_{\Omega}(n)$ has the 
Dirichlet generating function (DGF) $(1 + P(s))^{-1}$ and 
$C_{\Omega}(n)$ has the DGF $(1-P(s))^{-1}$ for $\Re(s) > 1$ 
where $P(s) := \sum_p p^{-s}$ is the prime zeta function. 
The function $C_{\Omega}(n)$ was considered in 
\cite{FROBERG-1968} with an exact formula given by 
\begin{equation}
\label{eqn_proof_tag_hInvn_ExactNestedSumFormula_CombInterpetIdent_v3}
C_{\Omega}(n) = \begin{cases}
     1, & \text{if $n = 1$; } \\ 
     (\Omega(n))! \times \prod\limits_{p^{\alpha}||n} \frac{1}{\alpha!}, & \text{if $n \geq 2$. }
     \end{cases}
\end{equation} 
The focus of the article is on studying statistics of the unsigned functions 
$C_{\Omega}(n)$ and $|g(n)|$ and their partial sums. 
The new formulas for $M(x)$ given in 
Theorem \hlocalref{prop_Mx_SBP_IntegralFormula} 
provide a window from which we can view classically  
difficult problems about asymptotics for this function partially in terms of the 
properties of the auxiliary unsigned functions and their distributions. 

Define the function 
\[
\widehat{G}(z) := \frac{1}{\Gamma(1+z)} \times \left(\frac{6e^{-P(2)}}{\pi^2}\right)^z, 
     \text{ for } 0 \leq |z| \leq 9.
\]

\begin{theorem} 
\label{cor_SummatoryFuncsOfUnsignedSeqs_v2} 
For all sufficiently large $x$, there is an absolute constant $A_0 > 0$ such that 
uniformly for $1 \leq k \leq \frac{3}{2} \log\log x$  
\begin{align*} 
\sum_{\substack{n \leq x \\ \Omega(n)=k}} C_{\Omega}(n) & = 
     \frac{A_0 \sqrt{2\pi} x}{\log x} \times  
     \widehat{G}\left(\frac{2k-2}{\log\log x}\right) 
     \frac{k (\log\log x)^{2k-\frac{3}{2}}}{2^{k-1} (2k-3)!!} \left( 
     1 + O\left(\frac{1}{\log\log x}\right)\right). 
\end{align*} 
\end{theorem} 

In the last theorem, we use $(2n-1)!!$ to denote the 
double factorial function \cite[\seqnum{A001147}]{OEIS}. 
This result combined with \eqref{eqn_proof_tag_hInvn_ExactNestedSumFormula_CombInterpetIdent_v3} 
leads to a proof of the following average order formula: 

\begin{theorem} 
\label{lemma_HatCAstxSum_ExactFormulaWithError_v1} 
There is an absolute constant $B_0 > 0$ such that 
\[
\frac{1}{n} \times \sum_{k \leq n} \log C_{\Omega}(k) = 
     B_0 \cdot (\log\log n)(\log\log\log n) \left(1 + 
     O\left(\frac{1}{\log\log n}\right)\right), 
     \text{ as } n \rightarrow \infty. 
\] 
\end{theorem} 

\begin{conjecture*}
For any fixed $z > 0$ there is an absolute constant $D_0 > 0$ so that 
as $x \rightarrow \infty$ 
\begin{align*} 
\frac{1}{x} \times \#\left\{3 \leq n \leq x: -z \leq |g(n)| - 
     \frac{1}{n} \times \sum_{k \leq n} |g(k)| \leq z\right\} & = 
	\Phi\left(\frac{\log\left(\frac{\pi^2 |z|}{6}\right)-B_0 \cdot (\log\log x) (\log\log\log x)}{ 
	D_0 \cdot (\log\log x)(\log\log\log x)}\right) + o(1).
\end{align*} 
\end{conjecture*}

The article is organized into sections that prove our new results for each of the functions 
$C_{\Omega}(n)$, $g(n)$ and $|g(n)|$, and then establish the proofs of the 
exact formulas for $M(x)$ stated in 
Theorem \hlocalref{prop_Mx_SBP_IntegralFormula}. 
The appendix sections provide a glossary of notation and 
supplementary material on topics that can be separated from the 
organization of the main sections of the article. 

\section{An asymptotic formula for certain partial sums}

The formula in 
Theorem \hlocalref{prop_HatAzx_ModSummatoryFuncExps_RelatedToCkn} 
is used to prove the results in 
Section \hlocalref{Section_NewFormulasForgInvn_v1} which then in turn 
lead to proofs of the results on the unsigned inverse function $|g(n)|$ in 
Section \hlocalref{Section_NewFormulasForgInvn_v2}. 
The function is $C_{\Omega}(n)$ defined in equation 
\eqref{eqn_proof_tag_hInvn_ExactNestedSumFormula_CombInterpetIdent_v3} 
of the introduction (see 
Section \hlocalref{Section_NewFormulasForgInvn_v1}).
 
\begin{definition}
\label{def_BivariateDGF_HatFsz_AndRelatedFuncs_v1}
Let the bivariate DGF $\widehat{F}(s, z)$ be defined 
for $\Re(s) > 1$ and $|z| < R$ for any fixed $R > 0$ by 
\[
\widehat{F}(s, z) := \exp\left(-z P(s)\right) 
     \times \prod_p \left(1 - \frac{1}{p^s}\right)^{z} = 
     \exp\left(-z P(s)\right) \times \zeta(s)^{-z}. 
\]
For $|z| \leq 9$, we define the function 
\[
\widehat{G}(z) := \frac{\widehat{F}(2, z)}{\Gamma(1+z)} = 
     \frac{1}{\Gamma(1 + z)} \times \left(\frac{6e^{-P(2)}}{\pi^2}\right)^z.
\]
\end{definition}

The Dirichlet generating function $\widehat{F}(s, z)$ can be studied through the 
Selberg-Delange method presented in Tenenbaum \cite[\S II.6.1]{TENENBAUM-PROBNUMT-METHODS}. 
The next theorem is proved in \cite[\S 7.4]{MV}.

\begin{theorem}
\label{theorem_MV_SelbergDelange}
Let $|z| \leq R$ for some fixed $R > 0$. 
Suppose that the series 
\[
\sum_{m \geq 1} \frac{|b_z(m)| (\log m)^{2R+1}}{m}, 
\]
is uniformly bounded for any $|z| \leq R$. For $\Re(s) \geq 1$ and $|z| \leq R$, let the 
DGF expansion 
\[
F(s, z) = \sum_{m \geq 1} \frac{b_z(m)}{m^s}. 
\]
Let the coefficients $\{a_z(n)\}_{n \geq 1}$ be defined by the relation 
\[
\zeta(s)^z F(s, z) = \sum_{n \geq 1} \frac{a_z(n)}{n^s}. 
\]
For any $x \geq 2$
\[
\sum_{n \leq x} a_z(n) = \frac{F(1, z)}{\Gamma(z)} x (\log x)^{z-1} + O_z\left(x (\log x)^{\Re(z) - 2}\right). 
\]
\end{theorem}

\begin{theorem} 
\label{prop_HatAzx_ModSummatoryFuncExps_RelatedToCkn} 
For all sufficiently large $x \geq 2$ and $|z| \leq 3$ 
\[
\sum_{n \leq x} \frac{C_{\Omega}(n)}{(\Omega(n))!} (-1)^{\omega(n)} z^{2\Omega(n)} = 
     \frac{x \widehat{F}(2, z^2)}{\Gamma(z^2)} (\log x)^{z^2-1} + 
     O_{z}\left(x (\log x)^{\Re(z^2) - 2}\right). 
\]
\end{theorem} 
\begin{proof} 
It follows from \eqref{eqn_proof_tag_hInvn_ExactNestedSumFormula_CombInterpetIdent_v3} that 
we can generate exponentially scaled forms of the function $C_{\Omega}(n)$ by 
a product identity of the following form: 
\begin{align} 
\label{eqn_proof_tag_EPExpzPs_v2}
\sum_{n \geq 1} \frac{C_{\Omega}(n)}{(\Omega(n))!} \cdot 
     \frac{(-1)^{\omega(n)} z^{\Omega(n)}}{n^s} & = \prod_p \left(1 + \sum_{r \geq 1} 
     \frac{z^{\Omega(p^r)}}{r! p^{rs}}\right)^{-1} 
     = \exp\left(-z P(s)\right), \text{ for } \Re(s) > 1. 
\end{align} 
This Euler product type expansion is similar in construction to the parameterized bivariate 
DGFs defined in \cite[\S 7.4]{MV} \cite[\cf \S II.6.1]{TENENBAUM-PROBNUMT-METHODS}.
Let the function $F(s, z) \equiv F^{\ast}(s, z)$ from 
Definition \hlocalref{def_BivariateDGF_HatFsz_AndRelatedFuncs_v1} 
be defined by 
\[
F^{\ast}(s, z) := \widehat{F}(s+1, z) = \sum_{m \geq 1} \frac{b_z^{\ast}(m)}{m^s}. 
\]
Then by \eqref{eqn_proof_tag_EPExpzPs_v2} we see that the series 
\[
\sum_{m \geq 1} \frac{|b_{z^2}^{\ast}(m)| (\log m)^{2R+1}}{m}, 
\]
is uniformly bounded for any $|z| \leq R \leq 3$. 
The conclusion follows from 
Theorem \hlocalref{theorem_MV_SelbergDelange} 
applied to the coefficients in the DGF expansion of 
$F^{\ast}(s, w) \zeta(s)^{w}$ when $s := 1$ and $w := z^2$. 
\end{proof} 

\section{Properties of the function $C_{\Omega}(n)$} 
\label{Section_NewFormulasForgInvn_v1} 

The function $C_{\Omega}(n)$ is key to understanding the 
unsigned inverse sequence $|g(n)|$. In this section, we define $C_{\Omega}(n)$ 
precisely and explore its properties. 

\begin{definition}
We define the following bivariate sequence for integers $n \geq 1$ and $k \geq 0$: 
\begin{align} 
\label{eqn_CknFuncDef_v2} 
C_k(n) := \begin{cases} 
     \varepsilon(n), & \text{ if $k = 0$; } \\ 
     \sum\limits_{d|n} \omega(d) C_{k-1}\left(\frac{n}{d}\right), & \text{ if $k \geq 1$. } 
     \end{cases} 
\end{align} 
Using the notation for iterated convolution in 
Bateman and Diamond \cite[Def.~ 2.3; \S 2]{ANT-BATEMAN-DIAMOND}, we have 
$C_0(n) \equiv \omega^{\ast 0}(n)$ and $C_k(n) \equiv \omega^{\ast k}(n)$ for 
integers $k \geq 1$ and $n \geq 1$. 
The special case of \eqref{eqn_CknFuncDef_v2} where 
$k := \Omega(n)$ occurs frequently in the next sections of the 
article. To avoid cumbersome notation when referring to this common function variant, we suppress the 
duplicate index $n$ by writing $C_{\Omega}(n) := C_{\Omega(n)}(n)$ \cite[\seqnum{A008480}]{OEIS}. 
\end{definition}

\begin{remark}
By recursively expanding the definition of $C_k(n)$ 
at any fixed $n \geq 2$, we see that 
we can form a chain of at most $\Omega(n)$ iterated (or nested) divisor sums by 
unfolding the definition of \eqref{eqn_CknFuncDef_v2} inductively. 
By the same argument, we see that at fixed $n$, the function 
$C_k(n)$ is non-zero only possibly for 
$1 \leq k \leq \Omega(n)$ when $n \geq 2$. 
We see by 
\eqref{eqn_proof_tag_hInvn_ExactNestedSumFormula_CombInterpetIdent_v3} 
that $C_{\Omega}(n) \leq (\Omega(n))!$ for all $n \geq 1$ with 
equality precisely at the squarefree integers so that 
$(\Omega(n))! = (\omega(n))!$ whenever $\mu^2(n) = 1$. 
\end{remark}

\subsection{Uniform asymptotics for partial sums}
\label{subSection_Section4_AnalyticPrerequisiteProofsOfUniformBoundsOnCertainPartialSumTypes_v1} 

Recall that we have defined the function 
$\widehat{G}(z) := \left(\frac{6e^{-P(2)}}{\pi^2}\right)^z \times \Gamma(1+z)^{-1}$ 
for $0 \leq |z| \leq 9$ in 
Definition \hlocalref{def_BivariateDGF_HatFsz_AndRelatedFuncs_v1}. 
In the next theorem statement, note that for integers $n \geq 0$ the double factorial function 
is defined by the expansion $(2n)! = (2n-1)!! 2^n n!$. 

\begin{theorem} 
\label{theorem_CnkSpCasesScaledSummatoryFuncs} 
As $x \rightarrow \infty$, uniformly for $1 \leq k \leq \frac{3}{2}\log\log x$ 
\[
\sum_{\substack{n \leq x \\ \Omega(n) = k}} (-1)^{\omega(n)} C_{\Omega}(n) = 
     -\widehat{G}\left(\frac{2k-2}{\log\log x}\right) \frac{x}{\log x} \cdot 
     \frac{k (\log\log x)^{2k-2}}{2^{k-1} (2k-3)!!} \left( 
     1 + O\left(\frac{k}{(\log\log x)^2}\right)\right). 
\]
\end{theorem} 
\begin{proof} 
For integers $x,k \geq 1$, let 
\begin{equation}
\label{eqn_DefHatCkomegax_proof_tag_v1}
\widehat{C}_{k,\omega}(x) := \sum_{\substack{n \leq x \\ \Omega(n) = k}} 
     (-1)^{\omega(n)} C_{\Omega}(n).
\end{equation}
When $k = 1$, we have that $\Omega(n) = \omega(n)$ for all 
$n \leq x$ such that $\Omega(n) = k$. 
The positive integers $n$ that satisfy this requirement are precisely the primes $p \leq x$. 
The formula is satisfied as 
\begin{equation}
\label{eqn_proof_tag_HatCOmegax_kEQ1_PrimeSum_v1} 
\widehat{C}_{1,\omega}(x) = 
     \sum_{p \leq x} (-1)^{\omega(p)} C_{\Omega}(p) = -\sum_{p \leq x} 1 = 
     - \frac{x}{\log x} \left(1 + O\left(\frac{1}{\log x}\right)\right). 
\end{equation}
For $|z| \leq 3$ and integers $x \geq 2$, let 
\[
\widehat{A}_{z}(x) := 
     \sum_{n \leq x} \frac{C_{\Omega}(n)}{(\Omega(n))!} (-1)^{\omega(n)} z^{2\Omega(n)}. 
\]
For $2 \leq k \leq \frac{3}{2}\log\log x$, we will apply the error estimate from 
Theorem \hlocalref{prop_HatAzx_ModSummatoryFuncExps_RelatedToCkn} with 
$r := \sqrt{\frac{2k-2}{\log\log x}}$ to
\[
\frac{\widehat{C}_{k,\omega}(x)}{k!} = \frac{1}{2\pi\imath} \times \int_{|v|=r} 
     \frac{\widehat{A}_{v}(x)}{v^{2k+1}} dv. 
\]
The error in this formula contributes terms that are bounded by 
\begin{align*} 
\left\lvert x (\log x)^{-(\Re(v^2)+2)} v^{-(2k+1)} \right\rvert & \ll 
     \left\lvert x (\log x)^{-(r^2+2)} r^{-(2k+1)} \right\rvert 
     \ll \frac{x}{(\log x)^{2+\frac{2k-1}{\log\log x}}} \cdot 
     \frac{(\log\log x)^{2k+1}}{(2k-1)^{2k+1}} \\ 
     & \ll \frac{x}{\log x} \cdot \frac{(\log\log x)^{2k-6}}{(2k-2)!} = 
	\frac{x}{\log x} \cdot \frac{(\log\log x)^{2k-6}}{2^k (2k-3)!! (k-1)!}, 
     \text{ as } x \rightarrow \infty. 
\end{align*} 
We next find the main term for the coefficients 
of the following contour integral when 
$r^2 \in \left[0, \frac{3}{2}\right)$. 
To find the main term, we perform the change of variable 
$v \mapsto -\imath v$ to see that 
\begin{align} 
\label{eqn_WideTildeArx_CountourIntDef_v1} 
\frac{\widehat{C}_{k,\omega}(x)}{k!} = 
     \frac{(-1)^{k} x}{2\pi\imath (\log x)} 
     \times \int_{|v|=r} \frac{\widehat{F}(2, -v^2) 
     (\log x)^{-v^2}}{\Gamma(1-v^2) v^{2k-1}} dv + 
     O\left(\frac{x}{\log x} \cdot \frac{(\log\log x)^{2k-6}}{2^k (2k-3)!! (k-1)!}\right). 
\end{align} 
The main term of $\widehat{C}_{k,\omega}(x)$ in 
\eqref{eqn_WideTildeArx_CountourIntDef_v1}
is then given by $-\frac{x \cdot k!}{\log x} \times I_k(r, x)$, where we define 
\begin{align*}
I_k(r, x) & = \frac{1}{2\pi\imath} \times \int_{|v|=r} 
     \frac{\widehat{G}(v^2) (\log x)^{v^2}}{v^{2k-1}} dv \\ 
     & =: I_{1,k}(r, x) - I_{2,k}(r, x). 
\end{align*}
With $r = \sqrt{\frac{2k-2}{\log\log x}}$, the 
first component integral is defined to be 
\begin{align*}
I_{1,k}(r, x) & := \frac{\widehat{G}(r^2)}{2\pi\imath} \times \int_{|v|=r} 
     \frac{(\log x)^{v^2}}{v^{2k-1}} dv = \widehat{G}(r^2) \times \frac{(\log\log x)^{2k-2}}{(2k-2)!}. 
\end{align*}
The second integral, $I_{2,k}(r, x)$, corresponds to an error term in the approximation. 
This component function is defined by 
\[
I_{2,k}(r, x) := \frac{1}{2\pi\imath} \times \int_{|v|=r} 
     \left(\widehat{G}(v^2) - \widehat{G}(r^2)\right) 
     \frac{(\log x)^{v^2}}{v^{2k-1}} dv. 
\]
Integrating by parts shows that \cite[\cf Thm.\ 7.19; \S 7.4]{MV} 
\[
I_{2,k}(r, x) := \frac{1}{2\pi\imath} \times \int_{|v|=r} 
     \left(\widehat{G}(v^2) - \widehat{G}(r^2) - 
     \widehat{G}^{\prime}(r^2)(v^2-4r^2)\right) 
     (\log x)^{v^2} v^{-2k} dv. 
\]
We find that 
\[
\left\lvert \widehat{G}(v^2) - \widehat{G}(r^2) - 
     \widehat{G}^{\prime}(r^2)(v^2-4r^2) \right\rvert = 
     \left\lvert \int_{r}^{v} 2w (v^2-4w^2) \widehat{G}^{\prime\prime}(w^2) dw \right\rvert 
     \ll r^2 |v^2-4r^2|^2. 
\]
With the parameterization $v = re^{2\pi\imath\theta}$ for 
$2\theta \in \left[-\frac{1}{2}, \frac{1}{2}\right]$, we obtain 
\[
|I_{2,k}(r, x)| \ll r^{6-2k} \times 
     \int_{-\frac{1}{4}}^{\frac{1}{4}} (\sin 2\pi\theta)^4 e^{(2k-2) \cos(4\pi\theta)} d\theta. 
\]
Since $|\sin x| \leq |x|$ for all $|x| < 1$ and $\cos(4\pi\theta) \leq 1 - 8\theta^2$ if 
$-\frac{1}{4} \leq \theta \leq \frac{1}{4}$, the next bounds hold for 
$1 \leq k \leq \frac{3}{2} \log\log x$. 
\begin{align*}
|I_{2,k}(r, x)| & \ll r^{6-2k} e^{2k-2} \times \int_0^{\infty} \theta^4 e^{-8(2k-2) \theta^2} d\theta \\ 
     & \ll \frac{r^{6-2k} e^{2k-2}}{(2k-2)^{\frac{5}{2}}} = 
     \frac{(\log\log x)^{2k-6} e^{2k-2}}{(2k-2)^{2k-\frac{5}{2}}} 
     \ll 
     \frac{k (\log\log x)^{2k-6}}{(2k-2)!}. 
\end{align*}
Finally, whenever $1 \leq k \leq \frac{3}{2} \log\log x$ 
\[
1 = \widehat{G}(0) \geq \widehat{G}\left(\frac{2k-2}{\log\log x}\right) = 
     \frac{\exp\left(-\frac{(2k-2) P(2)}{\log\log x}\right) \zeta(2)^{-\frac{2k-2}{\log\log x}}}{ 
     \Gamma\left(1+\frac{2k-2}{\log\log x}\right)}  
     \geq \widehat{G}(3) \approx 0.00964223. 
\]
In particular, the function 
$\widehat{G}\left(\frac{2k-2}{\log\log x}\right) \gg 1$ for 
all $1 \leq k \leq \frac{3}{2} \log\log x$. 
\end{proof} 

\begin{proof}[Proof of Theorem \hlocalref{cor_SummatoryFuncsOfUnsignedSeqs_v2}]  
Suppose that $\hat{h}(t)$ and $\sum_{n \leq t} \ell(n)$ are 
piecewise smooth and differentiable functions of $t$ on $\mathbb{R}^{+}$. 
The next formulas follow from Abel summation and integration by parts. 
\begin{subequations}
\begin{align} 
\label{eqn_AbelSummationIBPReverseFormula_stmt_v1} 
     \sum_{n \leq x} \ell(n) \hat{h}(n) & = 
     \left(\sum_{n \leq t} \ell(n)\right) \hat{h}(t) \Biggr\rvert_1^x - 
     \int_{1}^{x} \left(\sum_{n \leq t} \ell(n)\right) \hat{h}^{\prime}(t) dt \\ 
\label{eqn_AbelSummationIBPReverseFormula_stmt_v2}
     & = 
     \int_1^{x} \frac{d}{dt}\left[\sum_{n \leq t} \ell(n)\right] \hat{h}(t) dt
\end{align} 
\end{subequations}
For integers $x,k \geq 1$, let the summatory function $\widehat{C}_{k,\omega}(x)$ 
be defined as in \eqref{eqn_DefHatCkomegax_proof_tag_v1} and let the unsigned partial sums 
\[
\widehat{C}_k(x) := \sum_{\substack{n \leq x \\ \Omega(n) = k}} C_{\Omega}(n).
\]
Since $1 \leq k \leq \frac{3}{2} \log\log x$, we have that 
\[
\widehat{C}_{k,\omega}(x) = 
     \sum_{\substack{n \leq x \\ \Omega(n)=k}} (-1)^{\omega(n)} C_{\Omega}(n) = 
     \sum_{n \leq x} (-1)^{\omega(n)} \Iverson{\omega(n) \leq \frac{3}{2} \log\log x} \times 
     C_{\Omega}(n) \Iverson{\Omega(n) = k}. 
\]
By the proof of Lemma \hlocalref{cor_AsymptoticsForSignedSumsOfomegan_v1} 
in the appendix section, we have 
that as $t \rightarrow \infty$ 
\begin{align} 
\label{eqn_ProofTag_LAsttSummatoryFuncAsymptotics_v1}
L_{\ast}(t) & := \sum_{\substack{n \leq t \\ \omega(n) \leq \frac{3}{2} \log\log t}} 
     (-1)^{\omega(n)} 
     = \frac{(-1)^{\floor{\log\log t}} t}{A_0 \sqrt{2\pi \log\log t}}\left(1 + 
     O\left(\frac{1}{\sqrt{\log\log t}}\right)\right). 
\end{align} 
Except for $t$ within a subset of $\left(e^e, \infty\right)$ with measure zero on which 
$L_{\ast}(t)$ may change sign, the main term of the derivative of this summatory function 
is approximated by 
\[
L_{\ast}^{\prime}(t) = \frac{(-1)^{\floor{\log\log t}}}{A_0 \sqrt{2\pi \log\log t}}\left(
     1 + O\left(\frac{1}{\sqrt{\log\log t}}\right)\right), 
     \text{ a.e.\ for } t > e^e. 
\]
We apply the formula from \eqref{eqn_AbelSummationIBPReverseFormula_stmt_v2}  
to deduce that whenever $1 \leq k \leq \frac{3}{2} \log\log x$ as $x \rightarrow \infty$  
\begin{align*} 
     \widehat{C}_{k,\omega}(x) & = 
     \sum_{j=1}^{\log\log x-1} \frac{(-1)^{j+1}}{A_0\sqrt{2\pi}} \times \int_{e^{e^j}}^{e^{e^{j+1}}} 
     \frac{C_{\Omega}(t) \Iverson{\Omega(t) = k}}{\sqrt{\log\log t}} \left(1 + 
     O\left(\frac{1}{\sqrt{\log\log t}}\right)\right) dt \\ 
     & = -\int_1^{\frac{\log\log x}{2}} \int_{e^{e^{2s-1}}}^{e^{e^{2s}}} 
     \frac{2 C_{\Omega}(t) \Iverson{\Omega(t) = k}}{A_0 \sqrt{2\pi \log\log t}} 
     \left(1 + O\left(\frac{1}{\sqrt{\log\log t}}\right)\right) dt ds \\ 
     & \phantom{=-\ } +
     \frac{1}{A_0 \sqrt{2\pi}} \times \int_{x^{e^{-1}}}^x 
     \frac{C_{\Omega}(t) \Iverson{\Omega(t) = k}}{\sqrt{\log\log t}} 
     \left(1 + O\left(\frac{1}{\sqrt{\log\log t}}\right)\right) dt. 
\end{align*} 
For large $x$, $(\log\log t)^{-\frac{1}{2}}$ is continuous and monotone decreasing for $t$ on 
$\left[x^{e^{-1}}, x\right]$ with 
\[
\frac{1}{\sqrt{\log\log x}} - \frac{1}{\sqrt{\log\log\left(x^{e^{-1}}\right)}} = 
     O\left(\frac{1}{(\log x) \sqrt{\log\log x}}\right), 
\]
Then we have 
\begin{equation} 
\label{eqn_ProofTag_HatCkx_Asymptotics_v1_v0}
     -A_0 \sqrt{2\pi} x (\log x) \sqrt{\log\log x} \times \widehat{C}_{k,\omega}^{\prime}(x) = 
     \left(\widehat{C}_k(x) - \widehat{C}_k\left(x^{e^{-1}}\right)\right)(1+o(1)) - 
     x (\log x) \widehat{C}_k^{\prime}(x). 
\end{equation} 
For $1 \leq k < \frac{3}{2} \log\log x$, we expect the integers $n \leq x$ 
such that $\omega(n) = \Omega(n) = k$ to satisfy 
\[
\widehat{C}_k(x) \gg \sum_{n \leq x} \Iverson{\Omega(n) = k} \asymp 
     \frac{x}{\log x} \times \frac{(\log\log x)^{k-1}}{(k-1)!}. 
\]
We conclude that 
$\widehat{C}_k\left(x^{e^{-1}}\right) = o\left(\widehat{C}_k(x)\right)$ for large $x$. 
The solution to \eqref{eqn_ProofTag_HatCkx_Asymptotics_v1_v0} is of the form 
\[
\widehat{C}_k(x) = -A_0\sqrt{2\pi}(\log x) \times \left(\int_3^x 
     \frac{\sqrt{\log\log t}}{\log t} \times \widehat{C}_{k,\omega}^{\prime}(t) dt\right)(1+o(1)). 
\]
When we integrate by parts and apply 
Theorem \hlocalref{theorem_CnkSpCasesScaledSummatoryFuncs}, we find 
\begin{align*}
\widehat{C}_k(x) & = -A_0 \sqrt{2\pi} \sqrt{\log\log x} \times \widehat{C}_{k,\omega}(x) + 
     O\left(\frac{x}{\sqrt{\log x}} \times \int_3^x \frac{\sqrt{\log\log t} \times 
     \widehat{C}_{k,\omega}(t)}{t^2 (\log t)^{\frac{5}{2}}} dt\right) \\ 
     & = -A_0 \sqrt{2\pi} \sqrt{\log\log x} \times \widehat{C}_{k,\omega}(x) + 
     O\left(\frac{kx}{\sqrt{\log x} \left(\frac{3}{2}\right)^{2k} 2^k (2k-3)!!} \times 
     \Gamma\left(2k-\frac{1}{2}, \frac{3}{2}\log\log x\right)\right). 
\end{align*} 
If $1 \leq k \leq \frac{3}{2} \log\log x$ such that $\rho \in (0, 1)$ in 
Proposition \hlocalref{prop_IncGammaLambdaTypeBounds_v1} of the appendix, the proposition and 
Theorem \hlocalref{theorem_CnkSpCasesScaledSummatoryFuncs} 
imply the conclusion. 
\end{proof}

\subsection{Average order and variance}
\label{subSection_AvgOrdersOfTheUnsignedSequences} 

\begin{proof}[Proof of Theorem \hlocalref{lemma_HatCAstxSum_ExactFormulaWithError_v1}]  
We first use \eqref{eqn_proof_tag_hInvn_ExactNestedSumFormula_CombInterpetIdent_v3} to 
see that there is an absolute constant $P_0 > 0$ 
such that 
\begin{align}
\label{eqn_proof_tag_SumLogCOmegan_P0_exp_v1}
\sum_{k \geq 1} \sum_{\substack{n \leq x \\ \Omega(n)=k}} \log C_{\Omega}(n) & = 
     \sum_{k \geq 1} P_0 \times \#\{n \leq x: \Omega(n)=k\} \times \log(k!). 
\end{align}
For $x \geq 3$, consider the following partial sums:
\[
L_{\Omega}(x) := \sum_{1 \leq k \leq \frac{3}{2}\log\log x} 
	\sum_{\substack{n \leq x \\ \Omega(n)=k}} \log C_{\Omega}(n). 
\]
For any $z \geq 0$, we cite the following known form of 
Binet's formula for the log-gamma function \cite[\S 5.9(i)]{NISTHB}: 
\[
\log z! = \left(z+\frac{1}{2}\right)\log(1+z) - z + O(1). 
\]
Then provided that \eqref{eqn_proof_tag_SumLogCOmegan_P0_exp_v1} holds, 
there is an absolute constant $B_0^{\ast} > 0$ such that 
\begin{align}
\label{eqn_proof_tag_SumLogCOmegan_P0_exp_v2}
L_{\Omega}(x) & = \sum_{1 \leq k \leq \frac{3}{2}\log\log x} 
	\frac{B_0^{\ast} x (\log\log x)^{k-1}}{(\log x) (k-1)!} \left(
	\left(k+\frac{1}{2}\right) \log(1+k) - k\right)\left(1 + 
	O\left(\frac{1}{\log\log x}\right)\right). 
\end{align}
The right-hand-side of \eqref{eqn_proof_tag_SumLogCOmegan_P0_exp_v2} can be 
approximated by Abel summation using the functions 
\[
A_x(u) := \frac{B_0^{\ast} x \Gamma\left(u, \log\log x\right)}{\Gamma\left(u\right)}; 
     f(u) := \frac{(2u+1)}{2} \log\left(1 + u\right) - \frac{(2u+1)}{2}, 
     f^{\prime}(u) = \log\left(1 + u\right) - \frac{1}{2\left(1 + u\right)}. 
\]
Then we have by 
Proposition \hlocalref{prop_IncGammaLambdaTypeBounds_v1} that for some absolute constant $B_0 > 0$ 
\begin{align*}
L_{\Omega}(x) & = A_x\left(\frac{3}{2}\log\log x\right) f\left(\frac{3}{2}\log\log x\right) - 
	\int_0^{\frac{3}{2}} A_x(\alpha \log\log x) f^{\prime}(\alpha \log\log x) d\alpha \\ 
	& = 
	B_0x (\log\log x)(\log\log\log x) \left(1 + O\left(\frac{1}{\log\log x}\right)\right). 
\end{align*}
It suffices to show 
\begin{equation}
\label{eqn_proof_tag_PartialSumsOver_HatCkx_EquivCond_v2}
\sum_{\substack{n \leq x \\ \Omega(n) \geq \frac{3}{2} \log\log x}} 
	\log C_{\Omega}(n)= o\left(x \log\log x\right), 
	\text{ as } x \rightarrow \infty. 
\end{equation}
Because $r-1-r\log r \approx -0.108198$ when $r := \frac{3}{2}$ 
and $\log C_{\Omega}(n) \ll \Omega(n) \log \Omega(n)$, 
we can argue using Rankin's method as in \cite[Thm.~7.20; \S 7.4]{MV} that 
\eqref{eqn_proof_tag_PartialSumsOver_HatCkx_EquivCond_v2} holds. 
\end{proof} 

\begin{remark}
In contrast with the result we proved in 
Theorem \hlocalref{lemma_HatCAstxSum_ExactFormulaWithError_v1} above, 
we notice that 
Theorem \hlocalref{cor_SummatoryFuncsOfUnsignedSeqs_v2} can be used to show that there are 
absolute constants $C_0,a > 0$ and $b$ such that for all sufficiently large $x$ 
\[
\frac{1}{x} \times \sum_{n \leq x} C_{\Omega}(n) \sim C_0 x^{ax+b}. 
\]
In particular, \emph{Mathematica} is able to sum a mean value form of the 
asymptotics from Theorem \hlocalref{cor_SummatoryFuncsOfUnsignedSeqs_v2} 
over the uniform range $1 \leq k \leq \frac{3}{2} \log\log x$. 
This procedure yields the main term of the average order of this function. 
\end{remark}

\begin{prop}
\label{prop_VarianceStat_for_COmegan_v1}
\label{prop_COmeganFunc_Variance_v1}
For $x \geq 3$, there is an absolute constant $D_0 > 0$ such that the 
variance of $\log C_{\Omega}(x)$ is 
\[
\sqrt{\frac{1}{x} \times \sum_{n \leq x} \log^2 C_{\Omega}(n)} = 
	D_0 \cdot \sqrt{x} (\log\log x) (\log\log\log x) \left( 
	1 + O\left(\frac{1}{\log\log x}\right)\right). 
\]
\end{prop}
\begin{proof}
Suppose that $n \geq 3$. 
We have a well-known identity follows from an application of the 
Newton-Girard identities relating elementary symmetric polynomials 
to power sum polynomials in the form of 
\[
S_{2,\Omega}(n) := \sum_{k \leq n} \log^2 C_{\Omega}(k) - 
     \left(\sum_{k \leq n} \log C_{\Omega}(k)\right)^2 = 
     2 \times \sum_{1 \leq j < k \leq n} \log C_{\Omega}(j) \cdot \log C_{\Omega}(k). 
\]
Let the respective unscaled first and second moment sums for this function 
be denoted by 
\begin{align*}
E_{\Omega}(n) & := \sum_{k \leq n} \log C_{\Omega}(k), \\ 
V_{\Omega}(n) & := \sum_{k \leq n} \log^2 C_{\Omega}(k). 
\end{align*}
The expansion on the right-hand-side of the first identity is rewritten as 
\begin{align*}
S_{2,\Omega}(n) & = V_{\Omega}(n) - E_{\Omega}(n)^2 = 
     2 \times \sum_{1 \leq j < n} \log C_{\Omega}(j) \left(
     E_{\Omega}(n) - E_{\Omega}(j)\right). 
\end{align*} 
The conclusion follows by 
Theorem \hlocalref{lemma_HatCAstxSum_ExactFormulaWithError_v1}, 
Abel summation and the mean value theorem. 
\end{proof}

\section{Properties of the function $g(n)$} 
\label{Section_NewFormulasForgInvn_v2} 

In this section, we explore and enumerate several key properties of the inverse function 
$g(n)$. The partial sums of this sequence yield the new formulas for $M(x)$ stated in 
Theorem \hlocalref{prop_Mx_SBP_IntegralFormula} proved in 
Section \hlocalref{Section_KeyApplications_NewExactFormulasForMx_FullSectionLabel} below. 

\begin{definition}
\label{def_gn_and_Absgn_v2} 
For integers $n \geq 1$, we define the Dirichlet inverse function 
taken with respect to the operation of Dirichlet convolution to be 
\[
g(n) = (\omega + \mathds{1})^{-1}(n), \text{ for } n \geq 1. 
\]
The function $|g(n)|$ denotes the unsigned inverse function. 
\end{definition}

We briefly motivate the definition of $g(n)$ given in 
Definition \hlocalref{def_gn_and_Absgn_v2} using the next argument.

\begin{remark} 
Let $\chi_{\mathbb{P}}(n)$ denote the characteristic function of the primes, let 
$\varepsilon(n) = \delta_{n,1}$ be the multiplicative identity 
with respect to Dirichlet convolution, 
and denote by $\omega(n)$ the strongly additive function that counts the number of 
distinct prime factors of $n$ (without multiplicity). 
We can see using elementary methods that 
\begin{equation}
\label{eqn_AntiqueDivisorSumIdent} 
\chi_{\mathbb{P}} + \varepsilon = (\omega + \mathds{1}) \ast \mu. 
\end{equation} 
Namely, the result in \eqref{eqn_AntiqueDivisorSumIdent} follows by M\"obius inversion 
since $\mu \ast 1 = \varepsilon$ and 
\[
\omega(n) = \sum_{p|n} 1 = \sum_{d|n} \chi_{\mathbb{P}}(d), \text{ for } n \geq 1. 
\]
We recall the classic inversion theorem of summatory functions 
(or generalized convolutions) proved in 
\cite[\S 2.14]{APOSTOLANUMT} for any Dirichlet invertible arithmetic 
function $\alpha(n)$ as follows: 
\begin{equation}
\label{eqn_ApostolStmt_ClassicSummatoryFuncInvThm_v1} 
G(x) = \sum_{n \leq x} \alpha(n) F\left(\frac{x}{n}\right) \implies 
     F(x) = \sum_{n \leq x} \alpha^{-1}(n) G\left(\frac{x}{n}\right), 
     \text{ for } x \geq 1. 
\end{equation}
Hence, to express the new formulas for $M(x)$, which forms the partial sums of $\mu(n)$, 
we may consider the inversion of the right-hand-side form of the partial sums 
\[
\pi(x) + 1 = \sum_{n \leq x} \left(\chi_{\mathbb{P}} + \varepsilon\right)(n) = 
	\sum_{n \leq x} (\omega + \mathds{1}) \ast \mu(n), 
	\text{ for } x \geq 1. 
\]
Theorem \hlocalref{theorem_SummatoryFuncsOfDirCvls} in 
Section \hlocalref{subSection_KeyApplications_NewExactFormulasForMx} 
provides more expansions of the inversion of partial sums of this type 
(in analog to equation \eqref{eqn_ApostolStmt_ClassicSummatoryFuncInvThm_v1} above). 
\end{remark}

\subsection{Signedness}
\label{Section_PrelimProofs_Config} 
\label{subSection_ProofOfSignednessOfgInvn_v1} 

\begin{prop}
\label{prop_SignageDirInvsOfPosBddArithmeticFuncs_v1} 
The sign of the function $g(n)$ is $\lambda(n)$ for all $n \geq 1$. 
\end{prop} 
\begin{proof} 
The series $D_f(s) := \sum_{n \geq 1} f(n) n^{-s}$ defines the 
Dirichlet generating function (DGF) of any 
arithmetic function $f$ which is convergent for all $s \in \mathbb{C}$ satisfying 
$\Re(s) > \sigma_f$ where $\sigma_f$ is the abscissa of convergence of the series. 
Recall that $D_{\mathds{1}}(s) = \zeta(s)$, $D_{\mu}(s) = \zeta(s)^{-1}$ and 
$D_{\omega}(s) = P(s) \zeta(s)$ for $\Re(s) > 1$. 
By \eqref{eqn_AntiqueDivisorSumIdent} and the fact that whenever $f(1) \neq 0$, 
the DGF of $f^{-1}(n)$ is $D_f(s)^{-1}$, we have 
\begin{align} 
\label{eqn_DGF_of_gInvn} 
D_{(\omega+1)^{-1}}(s) = \frac{1}{\zeta(s) (1+P(s))}, \text{ for } \Re(s) > 1. 
\end{align} 
It follows that $(\omega + 1)^{-1}(n) = (h^{-1} \ast \mu)(n)$ for 
$h := \chi_{\mathbb{P}} + \varepsilon$. 
We first show that $\operatorname{sgn}(h^{-1}) = \lambda$. 
This observation then implies that 
$\operatorname{sgn}(h^{-1} \ast \mu) = \lambda$. 

We recover exactly that \cite[\cf \S 2]{FROBERG-1968} 
\begin{equation} 
\notag 
h^{-1}(n) = \begin{cases} 
     1, & n = 1; \\ 
     \lambda(n) (\Omega(n))! \times \prod\limits_{p^{\alpha} || n} \frac{1}{\alpha!}, & n \geq 2. 
     \end{cases}
\end{equation} 
In particular, by expanding the DGF of 
$h^{-1}$ formally in powers of $P(s)$ (where $|P(s)| < 1$ whenever $\Re(s) \geq 2$) 
we count that 
\begin{align}
\notag
\frac{1}{1+P(s)} & = \sum_{n \geq 1} \frac{h^{-1}(n)}{n^s} = \sum_{k \geq 0} (-1)^k P(s)^k, \\ 
\notag
     & = 
     1 + \sum_{\substack{n \geq 2 \\ n =p_1^{\alpha_1}p_2^{\alpha_2} \times \cdots \times p_k^{\alpha_k}}} 
     \frac{(-1)^{\alpha_1+\alpha_2+\cdots+\alpha_k}}{n^s} \times 
     \binom{\alpha_1+\alpha_2+\cdots+\alpha_k}{\alpha_1,\alpha_2,\ldots,\alpha_k}, \\ 
\label{eqn_COmeganMultinomExp_as_DGFSeries_v2}
     & = 
     1 + \sum_{\substack{n \geq 2 \\ n =p_1^{\alpha_1}p_2^{\alpha_2} \times \cdots \times p_k^{\alpha_k}}} 
     \frac{\lambda(n)}{n^s} \times \binom{\Omega(n)}{\alpha_1,\alpha_2,\ldots,\alpha_k}. 
\end{align}
Since $\lambda$ is completely multiplicative we have that 
$\lambda\left(\frac{n}{d}\right) \lambda(d) = \lambda(n)$ for all divisors 
$d|n$ when $n \geq 1$. We also know that $\mu(n) = \lambda(n)$ whenever $n$ is squarefree so that
\[
g(n) = (h^{-1} \ast \mu)(n) = \lambda(n) \times 
     \sum_{d|n} \mu^2\left(\frac{n}{d}\right) |h^{-1}(n)|, \text{ for } n \geq 1. 
     \qedhere 
\]
\end{proof} 

\subsection{Precise relations to $C_{\Omega}(n)$} 
\label{Section_InvFunc_PreciseExpsAndAsymptotics} 
\label{subSection_Relating_CknFuncs_to_gInvn} 

\begin{lemma} 
\label{lemma_AnExactFormulaFor_gInvByMobiusInv_v1} 
For all $n \geq 1$ 
\[
g(n) = \sum_{d|n} \mu\left(\frac{n}{d}\right) \lambda(d) C_{\Omega}(d). 
\]
\end{lemma}
\begin{proof} 
We first expand the recurrence relation for the Dirichlet inverse 
when $g(1) = g(1)^{-1} = 1$ as 
\begin{align} 
\label{eqn_proof_tag_gInvCvlOne_EQ_omegaCvlgInvCvl_v1} 
g(n) & = - \sum_{\substack{d|n \\ d>1}} (\omega(d) + 1) g\left(\frac{n}{d}\right) 
     \quad\implies\quad 
     (g \ast 1)(n) = -(\omega \ast g)(n). 
\end{align} 
We argue that for $1 \leq m \leq \Omega(n)$, we can inductively expand the 
implication on the right-hand-side of \eqref{eqn_proof_tag_gInvCvlOne_EQ_omegaCvlgInvCvl_v1} 
in the form of $(g \ast 1)(n) = F_m(n)$ where 
$F_m(n) := (-1)^{m} (C_m(-) \ast g)(n)$ so that 
\[
F_m(n) = - 
     \begin{cases} 
     (\omega \ast g)(n), & m = 1; \\ 
     \sum\limits_{\substack{d|n \\ d > 1}} F_{m-1}(d) \times \sum\limits_{\substack{r|\frac{n}{d} \\ r > 1}} 
     \omega(r) g\left(\frac{n}{dr}\right), & 2 \leq m \leq \Omega(n); \\ 
     0, & \text{otherwise.} 
     \end{cases} 
\]
When $m := \Omega(n)$, i.e., with the expansions 
in the previous equation taken to a maximal depth, we obtain the relation 
\begin{equation} 
\label{eqn_proof_tag_gInvCvlOne_EQ_omegaCvlgInvCvl_v2} 
(g \ast 1)(n) = (-1)^{\Omega(n)} C_{\Omega}(n) = \lambda(n) C_{\Omega}(n). 
\end{equation} 
The stated formula for $g(n)$ follows from 
\eqref{eqn_proof_tag_gInvCvlOne_EQ_omegaCvlgInvCvl_v2} 
by M\"obius inversion. 
\end{proof} 

\begin{cor} 
\label{lemma_AbsValueOf_gInvn_FornSquareFree_v1} 
For all $n \geq 1$ 
\begin{equation} 
\label{eqn_AbsValueOf_gInvn_FornSquareFree_v1} 
|g(n)| = \sum_{d|n} \mu^2\left(\frac{n}{d}\right) C_{\Omega}(d). 
\end{equation} 
\end{cor} 
\begin{proof} 
The result follows by applying 
Lemma \hlocalref{lemma_AnExactFormulaFor_gInvByMobiusInv_v1}, 
Proposition \hlocalref{prop_SignageDirInvsOfPosBddArithmeticFuncs_v1} and the 
complete multiplicativity of $\lambda(n)$.  
Since $\mu(n)$ is non-zero only at squarefree integers and since 
at any squarefree $d \geq 1$ we have $\mu(d) = (-1)^{\omega(d)} = \lambda(d)$, we have 
\begin{align*} 
|g(n)| & = \lambda(n) \times \sum_{d|n} \mu\left(\frac{n}{d}\right) \lambda(d) C_{\Omega}(d) \\ 
     & = \lambda(n^2) \times \sum_{d|n} \mu^2\left(\frac{n}{d}\right) C_{\Omega}(d). 
\end{align*} 
The leading term $\lambda(n^2) = 1$ for all $n \geq 1$ since the number of distinct 
prime factors (counting multiplicity) of any square integer is even. 
\end{proof} 

\begin{remark}
\label{remark_MiscConsequencesOfCorForFormulaOfUnsgInvnFunc_v2} 
We have the following remarks on consequences of 
Corollary \hlocalref{lemma_AbsValueOf_gInvn_FornSquareFree_v1}: 
\begin{itemize}[noitemsep,topsep=0pt,leftmargin=0.23in]
\item 
\begin{subequations}
Whenever $n \geq 1$ is squarefree 
\begin{equation}
|g(n)| = \sum_{d|n} C_{\Omega}(d). 
\end{equation}
Since all divisors of a squarefree integer are squarefree, 
for all squarefree integers $n \geq 1$, we have that 
\begin{equation}
\label{eqn_PropB_lemma_gInv_MxExample} 
|g(n)| = \sum_{m=0}^{\omega(n)} \binom{\omega(n)}{m} \times m!. 
\end{equation}
\end{subequations}
\item 
The formula in \eqref{eqn_AbsValueOf_gInvn_FornSquareFree_v1} shows that 
the DGF of the unsigned inverse function $|g(n)|$ 
is given by the meromorphic function 
$\frac{1}{\zeta(2s)(1-P(s))}$ for all $s \in \mathbb{C}$ with $\Re(s) > 1$. 
This DGF has a pole to the right of the line at $\Re(s) = 1$ 
which occurs for the unique real $\sigma \equiv \sigma_1 \approx 1.39943$ 
such that $P(\sigma) = 1$ on $(1, \infty)$. 
\end{itemize}
\end{remark}

\subsection{Average order and variance} 

\begin{theorem} 
\label{cor_ExpectationFormulaAbsgInvn_v2} 
As $n \rightarrow \infty$ 
\begin{align*} 
\frac{1}{n} \times \sum_{k \leq n} \log |g(k)| & = 
	\left(\frac{B_0}{2} \cdot (\log\log n)(\log\log\log n) - 
	\frac{1}{2} \log\left(\frac{\pi^2}{6}\right)\right)(1 + o(1)). 
\end{align*} 
\end{theorem} 
\begin{proof}
A classical formula for the number of squarefree integers $n \leq x$ shows that 
\cite[\S 18.6]{HARDYWRIGHT} \cite[\seqnum{A013928}]{OEIS} 
\[ 
Q(x) = \sum_{n \leq x} \mu^2(n) = \frac{6x}{\pi^2} + O\left(\sqrt{x}\right), 
     \text{\ as $x \rightarrow \infty$}. 
\]
Therefore, summing over the formula from 
\eqref{eqn_AbsValueOf_gInvn_FornSquareFree_v1}, we find that for large $n$  
\begin{align} 
\notag 
\frac{1}{n} \times \sum_{k \leq n} |g(k)| & = \frac{1}{n} \times \sum_{d \leq n} 
     C_{\Omega}(d) Q\left(\Floor{n}{d}\right) \\ 
\notag 
     & \sim \sum_{d \leq n} C_{\Omega}(d) \left(\frac{6}{d \cdot \pi^2} + O\left(\frac{1}{\sqrt{dn}}\right) 
     \right) \\ 
\label{eqn_proof_tag_gn_InTermsOf_COmegan_v1}
     & = \frac{6}{\pi^2} \left(\frac{1}{n} \times \sum_{k \leq n} C_{\Omega}(k) + \sum_{d<n} 
     \sum_{k \leq d} \frac{C_{\Omega}(k)}{d^2}\right) + O(1). 
\end{align} 
We claim that 
\begin{align}
\label{eqn_proof_tag_gn_InTermsOf_COmegan_v2}
|g(n)| - \frac{1}{n} \times \sum_{k \leq n} |g(k)| & \sim \frac{6}{\pi^2} C_{\Omega}(n), 
     \text{\ as\ } n \rightarrow \infty. 
\end{align} 
Let the backwards difference operator with respect to $x$ 
be defined for $x \geq 2$ and any arithmetic function $f$ by 
$\Delta_x[f] := f(x) - f(x-1)$. 
We see from \eqref{eqn_proof_tag_gn_InTermsOf_COmegan_v1} that 
\begin{align*} 
     |g(n)| & = \Delta_n\left[\sum_{k \leq n} g(k)\right]  
     \sim \frac{6}{\pi^2} \times 
     \Delta_n\left[\sum_{d \leq n} C_{\Omega}(d) \frac{n}{d}\right] \\ 
     & = \frac{6}{\pi^2}\left(C_{\Omega}(n) + \sum_{d < n} C_{\Omega}(d) \frac{n}{d} - 
     \sum_{d<n} C_{\Omega}(d) \frac{(n-1)}{d}\right) \\ 
     & \sim \frac{6}{\pi^2} C_{\Omega}(n) + \frac{1}{n-1} \times \sum_{k < n} |g(k)|, 
     \mathrm{\ as\ } n \rightarrow \infty. 
\end{align*} 
By taking the logarithm of \eqref{eqn_proof_tag_gn_InTermsOf_COmegan_v2}, we find that
\[
\frac{1}{n} \times \sum_{k \leq n} \log |g(k)| = 
	\frac{B_0}{2} \cdot (\log\log n) (\log\log\log n) - 
     \frac{1}{2}\log\left(\frac{\pi^2}{6}\right) + 
     O\left(\frac{1}{n^2} \times \sum_{k \leq n} \log |g(k)|\right). 
     \qedhere
\]
\end{proof} 

A similar argument to that given in the proof of 
Proposition \hlocalref{prop_COmeganFunc_Variance_v1} 
shows that the variance of $\log |g(n)|$ is given by 
\[
\sqrt{\frac{1}{n} \times \sum_{k \leq n} \log^2 |g(k)|} = 
     \frac{D_0}{\sqrt{2}} \cdot 
     \sqrt{n} (\log\log n) (\log\log\log n) (1+o(1)). 
\]

\section{Conjectures on limiting distributions for the unsigned sequences} 
\label{subSection_ErdosKacTheorem_Analogs} 

In this section, we motivate a conjecture that provides a limiting 
central limit type distribution for the function $\log C_{\Omega}(n)$. 
The relations between $C_{\Omega}(n)$ and $g(n)$ we proved in 
Section \hlocalref{subSection_Relating_CknFuncs_to_gInvn} then 
allow us to formulate a limiting central limit theorem for the distribution 
of the unsigned inverse sequence $|g(n)|$ under the assumption that 
the conjecture holds. For any $z \in (-\infty, \infty)$, the function 
$\Phi(z) = \frac{1}{\sqrt{2\pi}} \times \int_{-\infty}^{z} e^{-t^2/2} dt$ denotes the 
cumulative density function of a standard normal random variable. 

\begin{conjecture}
\label{conj_DetFormOfEKTypeThmForCOmegan_v1} 
For any real $z$ as $x \rightarrow \infty$ 
\begin{equation} 
\notag
\frac{1}{x} \times \#\left\{2 \leq n \leq x: 
	\frac{\log C_{\Omega}(n) - 
	B_0 \cdot (\log\log x) (\log\log\log x)}{D_0 \cdot (\log\log x)(\log\log\log x)} \leq z\right\} = 
     \Phi\left(z\right) + o(1). 
\end{equation}
\end{conjecture} 

Rigorous proofs of the conjectures in this section are outside of the scope of this manuscript. 
Limiting distributions of the probability weights on the log-multinomial distributions associated 
with the distinct values of $C_{\Omega}(n)$ on $n \leq x$ that may yield a useful probability 
model under which we can prove our conjectured convergence in distribution are discussed in 
\citep[\cf \S 1.2]{LOG-COMB-STRUCTS-BOOK}. 

\begin{prop}
\label{cor_CLT_VII} 
Suppose that Conjecture \hlocalref{conj_DetFormOfEKTypeThmForCOmegan_v1} is true. 
For any $z > 0$ as $x \rightarrow \infty$ 
\begin{align*} 
\frac{1}{x} \times \#\left\{3 \leq n \leq x: -z \leq |g(n)| - 
     \frac{1}{n} \times \sum_{k \leq n} |g(k)| \leq z\right\} & = 
	\Phi\left(\frac{\log\left(\frac{\pi^2 |z|}{6}\right)-B_0 \cdot (\log\log x) (\log\log\log x)}{ 
	D_0 \cdot (\log\log x)(\log\log\log x)}\right) + o(1).
\end{align*} 
\end{prop} 
\begin{proof} 
 
The result follows from \eqref{eqn_proof_tag_gn_InTermsOf_COmegan_v2} 
as a re-normalization of Conjecture \hlocalref{conj_DetFormOfEKTypeThmForCOmegan_v1}. 
\end{proof} 

We observe that to cover the spread at the center of the right-hand-side distribution 
as $\Phi(w)$ for $0 < |w| \leq M$, the effective values of $z > 0$ in 
Proposition \hlocalref{cor_CLT_VII} depend on $x$ as 
\[
0 < z \leq \left(\frac{\Gamma(\log\log x + 1) (\log x)}{\sqrt{2\pi \log\log x}}\right)^{MD_0+B_0} 
     (1 + o(1)), \text{ as } x \rightarrow \infty. 
\]

\section{Proofs of the new exact formulas for $M(x)$} 
\label{Section_KeyApplications} 
\label{Section_KeyApplications_NewExactFormulasForMx_FullSectionLabel} 

In this section, we prove the formulas for $M(x)$ involving the partial sums 
of the function $g(n)$ stated in 
Theorem \hlocalref{prop_Mx_SBP_IntegralFormula}. 
These new formulas exactly identify the Mertens function with partial sums of 
positive unsigned arithmetic functions which are sign-weighted by $\lambda(n)$. 
Since the formulas in equations 
\eqref{prop_Mx_SBP_IntegralFormula_PartB} and 
\eqref{eqn_RmkInitialConnectionOfMxToGInvx_ProvedByInversion_v1} 
suggest that a more complete understanding of the 
asymptotics of the summatory function of $g(n)$ may yield insights into the behavior of 
$M(x)$, we take the time to explore its properties somewhat here as well. 

\subsection{Formulas relating $M(x)$ to the partial sums of $g(n)$} 
\label{subSection_KeyApplications_NewExactFormulasForMx} 

\begin{definition}
For any $x \geq 1$, let the partial sums of the Dirichlet convolution $r \ast h$ be defined by 
\begin{align*} 
S_{r \ast h}(x) & := \sum_{n \leq x} \sum_{d|n} r(d) h\left(\frac{n}{d}\right). 
\end{align*}
\end{definition}

\begin{theorem} 
\label{theorem_SummatoryFuncsOfDirCvls} 
Let $r,h: \mathbb{Z}^{+} \rightarrow \mathbb{C}$ be any arithmetic functions such that $r(1) \neq 0$. 
Suppose that $R(x) := \sum_{n \leq x} r(n)$, $H(x) := \sum_{n \leq x} h(n)$, and that 
$R^{-1}(x) := \sum_{n \leq x} r^{-1}(n)$ for $x \geq 1$. 
The following holds for all integers $x \geq 1$: 
\begin{align*} 
S_{r \ast h}(x) & \phantom{:}= \sum_{d=1}^x r(d) H\left(\Floor{x}{d}\right) \\ 
S_{r \ast h}(x) & \phantom{:}= \sum_{k=1}^{x} H(k) \left(R\left(\Floor{x}{k}\right) - 
     R\left(\Floor{x}{k+1}\right)\right). 
\end{align*} 
Moreover, for any $x \geq 1$ 
\begin{align*} 
H(x) & = \sum_{j=1}^{x} S_{r \ast h}(j) \left(R^{-1}\left(\Floor{x}{j}\right) - 
     R^{-1}\left(\Floor{x}{j+1}\right)\right) \\ 
     & = \sum_{k=1}^{x} r^{-1}(k) S_{r \ast h}(x). 
\end{align*} 
\end{theorem} 

A key consequence of Theorem \hlocalref{theorem_SummatoryFuncsOfDirCvls} 
(proved in the appendix via matrix methods) 
in the special cases where $h(n) := \mu(n)$ for all $n \geq 1$ 
is stated as the next corollary. 

\begin{cor} 
\label{cor_CvlGAstMu} 
Suppose that $r$ is an arithmetic function such that 
$r(1) \neq 0$. Let the summatory function 
$\widetilde{R}(x) := \sum_{n \leq x} (r \ast \mu)(n)$. 
The Mertens function is expressed by the following 
partial sums for all $x \geq 1$: 
\[
M(x) = \sum_{k=1}^{x} \left(\sum_{j=\floor{\frac{x}{k+1}}+1}^{\floor{\frac{x}{k}}} r^{-1}(j)\right) 
     \widetilde{R}(k). 
\]
\end{cor} 

\begin{figure}[ht!]

\captionsetup{singlelinecheck=off}
\centering

\begin{subfigure}[t!]{0.9\textwidth}
\fbox{\includegraphics[width=\textwidth]{images/Figure3-CompBetweenMxAndGInvx-GInvFunctionSequenceCalculations.png}}
\captionsetup{justification=centering}
\caption{}
\end{subfigure}

\smallskip

\begin{subfigure}[t!]{0.9\textwidth}
\fbox{\includegraphics[width=\textwidth]{images/Figure6-RatiosOfGInvToMxAndScaledVersions-GInvFunctionSequenceCalculations.png}}
\captionsetup{justification=centering}
\caption{}
\end{subfigure}

\captionsetup{justification=centering}
\caption{} 
\label{figure_MxAndNewAuxPartialSums_Comparison_Intro_v2_v1} 

\end{figure} 

\begin{figure}[ht!]

\captionsetup{singlelinecheck=off}
\centering

\begin{subfigure}[t!]{0.9\textwidth}
\fbox{\includegraphics[width=\textwidth]{images/Figure4-ComponentsOfGInvx-GInvFunctionSequenceCalculations.png}}
\captionsetup{justification=centering}
\caption{}
\end{subfigure}

\smallskip

\begin{subfigure}[t!]{0.9\textwidth}
\fbox{\includegraphics[width=\textwidth]{images/Figure5-ScaledGInvWithBddEnvelopes-GInvFunctionSequenceCalculations.png}}
\captionsetup{justification=centering}
\caption{}
\end{subfigure}

\captionsetup{justification=centering}
\caption{}
\label{figure_MxAndNewAuxPartialSums_Comparison_Intro_v2_v2} 

\end{figure} 

\begin{definition}
\label{def_GInvAndGInvAbs_SummFuncs_v2}
\begin{subequations}
The summatory function of $g(n)$ is defined for all $x \geq 1$ by the partial sums 
\begin{equation}
G(x) := \sum_{n \leq x} g(n) = \sum_{n \leq x} \lambda(n) |g(n)|. 
\end{equation}
\end{subequations}
\end{definition}

Based on the convolution identity in \eqref{eqn_AntiqueDivisorSumIdent}, 
we prove the formulas in 
Theorem \hlocalref{prop_Mx_SBP_IntegralFormula} as special cases of 
Corollary \hlocalref{cor_CvlGAstMu} below. 
 
\begin{proof}[Proof of 
              \eqref{prop_Mx_SBP_IntegralFormula_PartA} and \eqref{prop_Mx_SBP_IntegralFormula_PartB} of 
              Theorem \hlocalref{prop_Mx_SBP_IntegralFormula}] 
By applying Theorem \hlocalref{theorem_SummatoryFuncsOfDirCvls} to 
equation \eqref{eqn_AntiqueDivisorSumIdent} we have that 
\begin{align} 
\notag
M(x) & = \sum_{k=1}^{x} \left(\pi\left(\Floor{x}{k}\right)+1\right) g(k) \\ 
\notag 
     & = G(x) + \sum_{k=1}^{\frac{x}{2}} \pi\left(\Floor{x}{k}\right) g(k) \\ 
\notag 
     & = G(x) + G\left(\Floor{x}{2}\right) + 
     \sum_{k=1}^{\frac{x}{2}-1} \left( 
     \pi\left(\Floor{x}{k}\right) - \pi\left(\Floor{x}{k+1}\right) 
	\right) G(k).
\end{align} 
The upper bound on the sum is truncated to $k \in \left[1, \frac{x}{2}\right]$ in the second equation 
above because $\pi(1) = 0$. 
The third formula above follows directly by summation by parts. 
\end{proof} 
\begin{proof}[Proof of \eqref{eqn_RmkInitialConnectionOfMxToGInvx_ProvedByInversion_v1} of 
	      Theorem \hlocalref{prop_Mx_SBP_IntegralFormula}]
Lemma \hlocalref{lemma_AnExactFormulaFor_gInvByMobiusInv_v1} shows that 
\[
G(x) = \sum_{d \leq x} \lambda(d) C_{\Omega}(d) M\left(\Floor{x}{d}\right). 
\]
The identity in \eqref{eqn_AntiqueDivisorSumIdent} implies 
$$\lambda(d) C_{\Omega}(d) = (g \ast 1)(d) = (\chi_{\mathbb{P}} + \varepsilon)^{-1}(d).$$ 
We recover the stated result by classical inversion of summatory functions. 
\end{proof}

\subsection{Plots and numerical experiments}

The plots shown in the figures in this section compare 
the values of $M(x)$ and $G(x)$ with scaled forms of related auxiliary partial sums: 
\begin{itemize}[noitemsep,topsep=0pt,leftmargin=0.23in]

\item In Figure \hlocalref{figure_MxAndNewAuxPartialSums_Comparison_Intro_v2_v1}, 
      we plot comparisons of $M(x)$ to scaled forms of $G(x)$ for $x \leq 5000$. The 
      absolute constant $C_2 := \zeta(2)$ and where the function 
      $Q(x) := \sum_{n \leq x} \mu^2(n)$ counts the number of squarefree integers $n \leq x$ for any 
      $x \geq 1$. In (a) the shift to the left on the $x$-axis of the former function 
      is compared and seen to be similar in shape to the magnitude of $M(x)$ on this initial subinterval. 
      It is unknown whether the similar shape and magnitude of these two functions persists for 
      larger $x$. 
      In (b) we have observed unusual reflections and symmetry between the two ratios plotted in the 
      figure. Note that we have numerically modified the plot values to shift the denominators of 
      $M(x)$ by one at each $x \leq 5000$ for which $M(x) = 0$ to highlight 
      the distinctive features of each ratio on the interval. 

\item In Figure \hlocalref{figure_MxAndNewAuxPartialSums_Comparison_Intro_v2_v2}, we compare 
      envelopes on the logarithmically scaled values of $\frac{G(x)}{x}$ to other variants of 
      the partial sums of $g(n)$ for $x \leq 4500$. 
      In (a) we define $G(x) := G_{+}(x) - G_{-}(x)$ where the functions 
      $G_{+}(x) > 0$ and $G_{-}(x) > 0$ for all $x \geq 1$. 
      That is, these signed component functions denote the unsigned contributions of only those summands 
      $|g(n)|$ over $n \leq x$ such that $\lambda(n) = \pm 1$, respectively. 
      The summatory function $Q(x) = \frac{6x}{\pi^2}(1+o(1))$ in (b) has the same definition as in 
      Figure \hlocalref{figure_MxAndNewAuxPartialSums_Comparison_Intro_v2_v1} above. 
      The second plot suggests that for large $x$ there is sufficient cancellation in the 
      signed summatory function so that 
      \[
      |G(x)| \ll \frac{|G|(x)}{(\log x) \sqrt{\log\log x}} = 
          \frac{1}{(\log x) \sqrt{\log\log x}} \times \sum_{n \leq x} |g(n)|.
      \]

\end{itemize}

\subsection{Local cancellation of 
	    the formulas for $M(x)$ involving $G(x)$ along a subsequence} 
\label{subSection_LocalCancellationOfGInvx} 

\begin{definition}
Suppose that $p_n$ denotes the $n^{th}$ prime for $n \geq 1$ 
\cite[\seqnum{A000040}]{OEIS}. 
The set of primorial integers is defined by  
\cite[\seqnum{A002110}]{OEIS} 
\[
\left\{n\#\right\}_{n \geq 1} = \left\{\prod_{k=1}^{n} p_k\right\}_{n \geq 1}. 
\]
\end{definition}

We expect that there is usually (almost always) 
a large amount cancellation between the successive 
values of the summatory function in 
\eqref{eqn_RmkInitialConnectionOfMxToGInvx_ProvedByInversion_v1}. 
Proposition \hlocalref{theorem_PrimorialSeqGInvCalcs_v1} 
demonstrates the phenomenon well along the infinite 
subsequence of the primorials $\{(4m+1)\#\}_{m \geq 1}$. 

\begin{prop}
\label{theorem_PrimorialSeqGInvCalcs_v1} 
As $m \rightarrow \infty$, each of the following holds: 
\begin{align} 
\tag{A} 
-G((4m+1)\#) & \asymp (4m+1)!, \\ 
\tag{B} 
G\left(\frac{(4m+1)\#}{p_k}\right) & \asymp (4m)!, 
     \text{ for any } 1 \leq k \leq 4m+1. 
\end{align} 
\end{prop}
\begin{proof} 
We have by \eqref{eqn_PropB_lemma_gInv_MxExample} 
that for all squarefree integers $n \geq 1$ 
\begin{align*} 
|g(n)| & = \sum_{j=0}^{\omega(n)} \binom{\omega(n)}{j} \times j! 
     = (\omega(n))! \times \sum_{j=0}^{\omega(n)} \frac{1}{j!} \\ 
     & = (\omega(n))! \times \left(e + O\left(\frac{1}{(\omega(n)+1)!}\right)\right). 
\end{align*} 
Let $m$ be a large positive integer. 
We obtain main terms of the form 
\begin{align} 
\label{eqn_proof_tag_GinvxLocalCancellation_v1} 
\sum_{\substack{n \leq (4m+1)\# \\ \omega(n)=\Omega(n)}} \lambda(n) |g(n)| 
     & = \sum_{0 \leq k \leq 4m+1} \binom{4m+1}{k} (-1)^{k} k! 
     \left(e + O\left(\frac{1}{(k+1)!}\right)\right) \\ 
\notag 
     & = -(4m+1)! + O\left(\frac{1}{4m+1}\right). 
\end{align} 
The formula for $C_{\Omega}(n)$ stated in 
\eqref{eqn_proof_tag_hInvn_ExactNestedSumFormula_CombInterpetIdent_v3} 
then implies the result in (A). 
Namely, this follows since the contributions from the summands of the inner 
summation on the right-hand-side of 
\eqref{eqn_proof_tag_GinvxLocalCancellation_v1} 
off of the squarefree integers 
are at most a bounded multiple of $(-1)^k k!$ when $\Omega(n) = k$. 
We can similarly derive that for any $1 \leq k \leq 4m+1$ 
\begin{align*}
G\left(\frac{(4m+1)\#}{p_k}\right) & \asymp \sum_{0 \leq k \leq 4m} \binom{4m}{k} (-1)^{k} k! 
     \left(e + O\left(\frac{1}{(k+1)!}\right)\right) = (4m)! + O\left(\frac{1}{4m+1}\right). 
     \qedhere 
\end{align*}
\end{proof}

\begin{remark}
\label{remark_LocalCancellationWithGxAlongThePrimorialsUnderTheRH} 
The Riemann hypothesis (RH) is equivalent to showing that 
\begin{equation} 
\label{eqn_MertensMx_RHEquivProblem_Stmt_intro} 
M(x) = O\left(x^{\frac{1}{2}+\epsilon}\right), \text{ for all } 0 < \epsilon < \frac{1}{2}.
\end{equation}
The RH requires that the sums of the leading constants with opposing signs 
on the asymptotic bounds for the functions from the lemma match. 
In particular, we have that 
\cite{DUSART-1999,DUSART-2010} 
\[
n\# \sim e^{\vartheta(p_n)} \asymp n^n (\log n)^n e^{-n(1+o(1))}, 
     \text{ as } n \rightarrow \infty. 
\]
The observation on the necessary cancellation in 
\eqref{eqn_RmkInitialConnectionOfMxToGInvx_ProvedByInversion_v1}
then follows from the fact that if we obtain a contrary result  
\[
\frac{M((4m+1)\#)}{\sqrt{(4m+1)\#}} \gg \left[(4m+1)\#\right]^{\delta_0}, 
     \text{ as } m \rightarrow \infty, 
\]
for some fixed $\delta_0 > 0$ 
(in contradiction to \eqref{eqn_MertensMx_RHEquivProblem_Stmt_intro} above). 
\end{remark}

\section{Conclusions}

We have identified a sequence, 
$\{g(n)\}_{n \geq 1}$, that is the Dirichlet inverse of the 
shifted strongly additive function $\omega(n)$. 
We showed that there is a natural 
combinatorial interpretation to the repetition of distinct values 
of $|g(n)|$ in terms of the configuration of the 
exponents in the prime factorization of any $n \geq 2$. 
The sign of $g(n)$ is given by $\lambda(n)$ for all $n \geq 1$. 
This leads to a new exact relations of the 
summatory function $G(x)$ to $M(x)$ and the classical partial sums $L(x)$. 
In the process, 
we have formalized a new perspective from which we might express 
our intuition about features of the distribution of $G(x)$ 
via the properties of its $\lambda(n)$-sign-weighted summands.
The new results proved within this article 
are significant in providing a new window through which we can view bounding $M(x)$ 
through asymptotics of the auxiliary unsigned sequences and their partial sums. 
The computational data generated in 
Table \hlocalref{table_conjecture_Mertens_ginvSeq_approx_values} of the appendix section 
is suggests numerically that the distribution of $G(x)$ is easier to work with 
than a direct treatment of $M(x)$ or $L(x)$. 

\section*{Acknowledgments}
\addcontentsline{toc}{section}{Acknowledgments}

We thank the following mathematicians for offering significant 
discussion, feedback and correspondence over the many drafts of this manuscript: 
Gerg\H{o} Nemes, Jeffrey Lagarias, Robert Vaughan, Steven J.~Miller, 
Paul Pollack and Bruce Reznick. 
The work on the article was supported in part by 
funding made available within the School of Mathematics at the 
Georgia Institute of Technology in 2020 and 2021. 

\renewcommand{\refname}{References} 
\addcontentsline{toc}{section}{References}
%\bibliography{glossaries-bibtex/thesis-references}{}
\bibliographystyle{plain}

\begin{thebibliography}{10}

\bibitem{APOSTOLANUMT}
T.~M. Apostol.
\newblock {\em Introduction to Analytic Number Theory}.
\newblock Springer--Verlag, 1976.

\bibitem{ANT-BATEMAN-DIAMOND}
P.~T. Bateman and H.~G. Diamond.
\newblock {\em Analytic Number Theory}.
\newblock World Scientific Publishing, 2004.

\bibitem{BILLINGSLY-CLT-PRIMEDIVFUNC}
P.~Billingsley.
\newblock On the central limit theorem for the prime divisor function.
\newblock {\em Amer. Math. Monthly}, 76(2):132--139, 1969.

\bibitem{DUSART-1999}
P.~Dusart.
\newblock The $k^{th}$ prime is greater than $k(\log k +\log\log k-1)$ for $k
  \geq 2$.
\newblock {\em Math. Comp.}, 68(225):411--415, 1999.

\bibitem{DUSART-2010}
P.~Dusart.
\newblock Estimates of some functions over primes without {R}.{H}, 2010.

\bibitem{ERDOS-KAC-REF}
P.~Erd{\H{o}}s and M.~Kac.
\newblock The {G}aussian errors in the theory of additive arithmetic functions.
\newblock {\em American Journal of Mathematics}, 62(1):738--742, 1940.

\bibitem{MR3779960}
N.~Frantzikinakis and B.~Host.
\newblock The logarithmic {S}arnak conjecture for ergodic weights.
\newblock {\em Ann. of Math. (2)}, 187(3):869--931, 2018.

\bibitem{FROBERG-1968}
C.~E. Fr{\"{o}}berg.
\newblock On the prime zeta function.
\newblock {\em BIT Numerical Mathematics}, 8:87--202, 1968.

\bibitem{MR2877066}
B.~Green and T.~Tao.
\newblock The {M}\"{o}bius function is strongly orthogonal to nilsequences.
\newblock {\em Ann. of Math. (2)}, 175(2):541--566, 2012.

\bibitem{HARDYWRIGHT}
G.~H. Hardy and E.~M. Wright, editors.
\newblock {\em An Introduction to the Theory of Numbers}.
\newblock Oxford University Press, 2008 (Sixth Edition).

\bibitem{HUMPHRIES-JNT-2013}
P.~Humphries.
\newblock The distribution of weighted sums of the {L}iouville function and
  {P}\'{o}lya's conjecture.
\newblock {\em J. Number Theory}, 133:545--582, 2013.

\bibitem{LEHMAN-1960}
R.~S. Lehman.
\newblock On {L}iouville's function.
\newblock {\em Math. Comput.}, 14:311--320, 1960.

\bibitem{MV}
H.~L. Montgomery and R.~C. Vaughan.
\newblock {\em Multiplicative Number Theory: I. Classical Theory}.
\newblock Cambridge, 2006.

\bibitem{NEMES2015C}
G.~Nemes.
\newblock The resurgence properties of the incomplete gamma function {II}.
\newblock {\em Stud. Appl. Math.}, 135(1):86--116, 2015.

\bibitem{NEMES2016}
G.~Nemes.
\newblock The resurgence properties of the incomplete gamma function {I}.
\newblock {\em Anal. Appl. (Singap.)}, 14(5):631--677, 2016.

\bibitem{NEMES2019}
G.~Nemes and A.~B.~Olde Daalhuis.
\newblock Asymptotic expansions for the incomplete gamma function in the
  transition regions.
\newblock {\em Math. Comp.}, 88(318):1805--1827, 2019.

\bibitem{NISTHB}
F.~W.~J. Olver, D.~W. Lozier, R.~F. Boisvert, and C.~W. Clark, editors.
\newblock {\em {NIST} Handbook of Mathematical Functions}.
\newblock Cambridge University Press, 2010.

\bibitem{LOG-COMB-STRUCTS-BOOK}
A.~D.~Barbour R.~Arratia and Simon Tavar{\'{e}}.
\newblock {\em Logarithmic Combinatorial Structures: A Probabilistic Approach}.
\newblock Preprint draft, 2002.

\bibitem{OEIS}
N.~J.~A. Sloane.
\newblock The {O}nline {E}ncyclopedia of {I}nteger {S}equences, 2021.
\newblock \url{http://oeis.org}.

\bibitem{TENENBAUM-PROBNUMT-METHODS}
G.~Tenenbaum.
\newblock {\em Introduction to Analytic and Probabilistic Number Theory}.
\newblock American Mathematical Society, third edition, 2015.

\bibitem{LUNE-DRESSLER}
J.~van~de Lune and R.~E. Dressler.
\newblock Some theorems concerning the number theoretic function $\omega(n)$.
\newblock {\em J. Reine Angew. Math.}, 1975(277):117--119, 1975.

\end{thebibliography}

%\bigskip\hrule\smallskip 
\newpage

\appendix
\cftaddtitleline{toc}{section}{Appendices on supplementary material}{}
\setcounter{section}{0} 
\renewcommand{\thesection}{\Alph{section}} 

\section{Glossary of notation and conventions}
\label{Section_NotationAndConventions}

The next listing provides a 
glossary of common notation, conventions and 
abbreviations used in the article. 

\renewcommand*{\glsclearpage}{}
\renewcommand{\glossarysection}[2][]{}
\printglossary[type={symbols},
               style={glossstyleSymbol},
               nogroupskip=true]

\section{The distributions of $\omega(n)$ and $\Omega(n)$} 
\label{subSection_TheKnownDistsOfThePrimeOmegaFunctions_IntroResults_v1} 

As $n \rightarrow \infty$, we have that 
$$\frac{1}{n} \times \sum_{k \leq n} \omega(k) = \log\log n + B_1 + o(1),$$ 
and 
$$\frac{1}{n} \times \sum_{k \leq n} \Omega(k) = \log\log n + B_2 + o(1),$$ for 
$B_1 \approx 0.261497$ and $B_2 \approx 1.03465$ 
absolute constants \cite[\S 22.10]{HARDYWRIGHT}. 
The next theorems reproduced from \cite[\S 7.4]{MV} bound the frequency of the 
number of $\omega(n)$ and $\Omega(n)$ over $n \leq x$ such that 
these functions diverge substantially from their average order. 
These results reflect a distinctively normal tendency 
of these strongly additive arithmetic functions towards their respective average orders 
(\cf \cite{ERDOS-KAC-REF,BILLINGSLY-CLT-PRIMEDIVFUNC} \cite[\S 7.4]{MV}). 

\begin{theorem} 
\label{theorem_MV_Thm7.20-init_stmt} 
For $x \geq 2$ and $r > 0$, let 
\begin{align*} 
A(x, r) & := \#\left\{n \leq x: \Omega(n) \leq r \log\log x\right\}, \\ 
B(x, r) & := \#\left\{n \leq x: \Omega(n) \geq r \log\log x\right\}. 
\end{align*} 
If $0 < r \leq 1$, then 
\[
A(x, r) \ll\phantom{_R} x (\log x)^{r-1 - r\log r}, \text{ as } x \rightarrow \infty. 
\]
If $1 \leq r \leq R < 2$, then 
\[
B(x, r) \ll_R x (\log x)^{r-1-r \log r}, \text{ as } x \rightarrow \infty. 
\]
\end{theorem} 

\begin{theorem}
\label{theorem_HatPi_ExtInTermsOfGz} 
For integers $k \geq 1$ and $x \geq 2$ 
$$\widehat{\pi}_k(x) := \#\{2 \leq n \leq x: \Omega(n)=k\}.$$ 
For $0 < R < 2$, uniformly for $1 \leq k \leq R \log\log x$ 
\[
\widehat{\pi}_k(x) = \frac{x}{\log x} \times \mathcal{G}\left(\frac{k-1}{\log\log x}\right) 
     \frac{(\log\log x)^{k-1}}{(k-1)!} \left(1 + O_R\left(\frac{k}{(\log\log x)^2}\right)\right), 
\]
where 
\[
\mathcal{G}(z) := \frac{1}{\Gamma(1+z)} \times 
	\prod_p \left(1-\frac{z}{p}\right)^{-1} \left(1-\frac{1}{p}\right)^z, 
	\text{ for } 0 \leq |z| < R. 
\]
\end{theorem} 

\begin{remark} 
\label{remark_MV_Pikx_FuncResultsAnnotated_v1} 
We can extend the work in \cite{MV} on the distribution of $\Omega(n)$ to obtain 
corresponding analogous results for the distribution of $\omega(n)$. 
For integers $k  \geq 1$ and $x \geq 2$, we define 
\[
\pi_k(x) := \#\{2 \leq n \leq x: \omega(n)=k\}.
\]
For $0 < R < 2$ and as $x \rightarrow \infty$ 
\begin{equation}
\label{eqn_Pikx_UniformAsymptoticsStmt_from_MV_v2} 
\pi_k(x) = \frac{x}{\log x} \times 
     \widetilde{\mathcal{G}}\left(\frac{k-1}{\log\log x}\right) 
     \frac{(\log\log x)^{k-1}}{(k-1)!} \left( 
     1 + O_R\left(\frac{k}{(\log\log x)^2}\right) 
     \right), 
\end{equation}
uniformly for $1 \leq k \leq R\log\log x$. 
The factors of the function $\widetilde{\mathcal{G}}(z)$ are 
defined by $\widetilde{\mathcal{G}}(z) := \widetilde{F}(1, z) \times \Gamma(1+z)^{-1}$ where 
\[
\widetilde{F}(s, z) := \prod_p \left(1 + \frac{z}{p^s-1}\right) \left(1 - \frac{1}{p^s}\right)^{z}, 
	\text{ for } \Re(s) > \frac{1}{2} \text{ and } |z| \leq R < 2. 
\]
Let the functions 
\begin{align*} 
C(x, r) & := \#\{n \leq x: \omega(n) \leq r \log\log x\}, \\ 
D(x, r) & := \#\{n \leq x: \omega(n) \geq r \log\log x\}. 
\end{align*} 
The following upper bounds hold as $x \rightarrow \infty$: 
\begin{align*} 
C(x, r) & \ll\phantom{_R} x (\log x)^{r - 1 - r \log r}, \text{ uniformly for } 0 < r \leq 1, \\ 
D(x, r) & \ll_R x (\log x)^{r - 1 - r \log r}, \text{ uniformly for } 1 \leq r \leq R < 2.
\end{align*} 
\end{remark} 

\section{Asymptotics of the incomplete gamma function} 
\label{subSection_OtherFactsAndResults} 

We cite the correspondence with Gerg\H{o} Nemes 
from the Alfr\'{e}d R\'{e}nyi Institute of Mathematics and his 
careful notes on the limiting asymptotics for the sums identified in this section. 
The communication of his proofs are adapted to establish the next few lemmas based on 
his work in \cite{NEMES2015C,NEMES2016,NEMES2019}. 

\begin{definition}[The incomplete gamma function]
The (upper) incomplete gamma function is defined by \cite[\S 8.4]{NISTHB} 
\[
\Gamma(a, z) = \int_{z}^{\infty} t^{a-1} e^{-t} dt, \text{ for } 
	a \in \mathbb{R} \text{ and } |\arg z| < \pi.  
\]
\end{definition}

The function $\Gamma(a, z)$ can be continued to an analytic function of $z$ on the 
universal covering of $\mathbb{C} \mathbin{\backslash} \{0\}$. 
For $a \in \mathbb{Z}^{+}$, the function $\Gamma(a, z)$ is an entire function of $z$. 

\begin{facts} 
\label{facts_ExpIntIncGammaFuncs} 
\begin{subequations}
The following properties hold \cite[\S 8.4; \S 8.11(i)]{NISTHB}: 
\begin{align} 
\label{eqn_IncompleteGamma_PropA} 
     \Gamma(a, z) & = (a-1)! e^{-z} \times \sum_{k=0}^{a-1} \frac{z^k}{k!}, \text{ for } 
     a \in \mathbb{Z}^{+} \text{ and } z \in \mathbb{C}, \\ 
\label{eqn_IncompleteGamma_PropB} 
\Gamma(a, z) & \sim z^{a-1} e^{-z}, \text{ for fixed } a \in \mathbb{R} 
     \text{ and } z > 0 \text{ as } z \rightarrow \infty. 
\end{align}
For $z > 0$, as $z \rightarrow \infty$ we have that \cite{NEMES2015C} 
\begin{equation} 
\label{eqn_IncompleteGamma_PropC}
\Gamma(z, z) = \sqrt{\frac{\pi}{2}} z^{z-\frac{1}{2}} e^{-z} + 
     O\left(z^{z-1} e^{-z}\right), 
\end{equation} 
The sequence $\{b_n(\rho)\}_{n \geq 0}$ satisfies $b_0(\rho) = 1$ and 
the following recurrence relation for $n \geq 1$: 
\[
b_n(\rho) = \rho(1-\rho) b_{n-1}^{\prime}(\rho) + \rho(2n-1) b_{n-1}(\rho). 
\]
If $z,a \rightarrow \infty$ with $z = \rho a$ for some $\rho > 1$ such that 
$(\rho - 1)^{-1} = o\left(\sqrt{|a|}\right)$, then \cite{NEMES2015C}
\begin{equation}
\label{eqn_IncompleteGamma_PropD}
\Gamma(a, z) \sim z^a e^{-z} \times \sum_{n \geq 0} \frac{(-a)^n b_n(\rho)}{(z-a)^{2n+1}}. 
\end{equation} 
\end{subequations}
\end{facts} 

\begin{prop}
\label{prop_IncGammaLambdaTypeBounds_v1}
Let $a,z,\rho$ be positive real parameters such that $z=\rho a$. 
If $\rho \in (0, 1)$, then as $z \rightarrow \infty$ 
\[
\Gamma(a, z) = \Gamma(a) + O_{\rho}\left(z^{a-1} e^{-z}\right). 
\]
If $\rho > 1$, then as 
$z \rightarrow \infty$ 
\[
\Gamma(a, z) = \frac{z^{a-1} e^{-z}}{1-\rho^{-1}} + O_{\rho}\left(z^{a-2} e^{-z}\right). 
\]
If $\rho > W(1) \approx 0.56714$, then as $z \rightarrow \infty$ 
\[
\Gamma(a, z e^{\pm\pi\imath}) = -e^{\pm \pi\imath a} \frac{z^{a-1} e^{z}}{1 + \rho^{-1}} + 
     O_{\rho}\left(z^{a-2} e^{z}\right). 
\]
\end{prop}

\begin{remark}
The first two estimates in the proposition 
are only useful when $\rho$ is bounded away from the transition point at one. 
We cannot write the last expansion above 
as $\Gamma(a, -z)$ directly unless $a \in \mathbb{Z}^{+}$ 
as the incomplete gamma function 
has a branch point at the origin with respect to its second variable. 
This function becomes a single-valued 
analytic function of its second input by continuation 
on the universal covering of $\mathbb{C} \mathbin{\backslash} \{0\}$. 
\end{remark}

\begin{proof}[Proof of Proposition \hlocalref{prop_IncGammaLambdaTypeBounds_v1}] 
The first asymptotic estimate follows directly from the following 
asymptotic series expansion that holds as $z \rightarrow \infty$ 
\cite[Eq.\ (2.1)]{NEMES2019}: 
\[
\Gamma(a, z) \sim \Gamma(a) + z^a e^{-z} \times \sum_{k \geq 0} 
     \frac{(-a)^k b_k(\rho)}{(z-a)^{2k+1}}. 
\]
Using the notation from \eqref{eqn_IncompleteGamma_PropD} and \cite{NEMES2016} 
\[
\Gamma(a, z) = \frac{z^{a-1} e^{-z}}{1-\rho^{-1}} + z^{a} e^{-z} R_1(a, \rho). 
\]
From the bounds in \cite[\S 3.1]{NEMES2016}, we have 
\[
\left\lvert z^{a} e^{-z} R_1(a, \rho) \right\rvert \leq 
     z^a e^{-z} \times \frac{a \cdot b_1(\rho)}{(z-a)^{3}} = 
     \frac{z^{a-2} e^{-z}}{(1-\rho^{-1})^{3}}
\]
The main and error terms in the previous equation can also be 
seen by applying the asymptotic series in 
\eqref{eqn_IncompleteGamma_PropD} directly. 

The proof of the third equation above follows from the asymptotics 
\cite[Eq.\ (1.1)]{NEMES2015C}
\[
\Gamma(-a, z) \sim z^{-a} e^{-z} \times \sum_{n \geq 0} \frac{a^n b_n(-\rho)}{(z+a)^{2n+1}}, 
\]
by setting $(a, z) \mapsto \left(a e^{\pm \pi\imath}, z e^{\pm \pi\imath}\right)$ so that 
$\rho = \frac{z}{a} > W(1)$. 
The restriction on the range of $\rho$ over which the third formula holds is made to ensure that 
the formula from the reference is valid at negative real $a$. 
\end{proof}

\begin{lemma}
\label{lemma_ConvenientIncGammaFuncTypePartialSumAsymptotics_v2}
As $x \rightarrow \infty$  
\begin{align*}
\frac{x}{\log x} \times \left\lvert \sum_{1 \leq k \leq \log\log x} 
     \frac{(-1)^k (\log\log x)^{k-1}}{(k-1)!} \right\rvert 
     & = \frac{x}{2\sqrt{2\pi \log\log x}} + O\left(\frac{x}{(\log\log x)^{\frac{3}{2}}}\right). 
\end{align*}
\end{lemma}
\begin{proof}
We have for $n \geq 1$ and any $t > 0$ by 
\eqref{eqn_IncompleteGamma_PropA} that 
\[
\sum_{1 \leq k \leq n} \frac{(-1)^k t^{k-1}}{(k-1)!} = -e^{-t} \times 
     \frac{\Gamma(n, -t)}{(n-1)!}. 
\]
Suppose that $t = n + \xi$ with $\xi = O(1)$. 
By the third formula 
in Proposition \hlocalref{prop_IncGammaLambdaTypeBounds_v1} 
with the parameters $(a, z, \lambda) \mapsto \left(n, t, 1 + \frac{\xi}{n}\right)$, 
we deduce that as $n,t \rightarrow \infty$. 
\begin{equation}
\label{eqn_ProofTag_lemma_ConvenientIncGammaFuncTypePartialSumAsymptotics_v2}
\Gamma(n, -t) = (-1)^{n+1} \times \frac{t^n e^{t}}{t+n} + 
     O\left(\frac{n t^n e^{t}}{(t+n)^3}\right) = 
     (-1)^{n+1} \times \frac{t^n e^t}{2n} + O\left(\frac{t^{n-1} e^t}{n}\right). 
\end{equation}
Accordingly, we see that 
\[
\sum_{1 \leq k \leq n} \frac{(-1)^k t^{k-1}}{(k-1)!} = 
     (-1)^{n} \times \frac{t^n}{2n!} + O\left(\frac{t^{n-1}}{n!}\right). 
\]
By the variant of Stirling's formula in \cite[\cf Eq.\ (5.11.8)]{NISTHB}, we have 
\[
n! = \Gamma(1 + t - \xi) = \sqrt{2\pi} t^{t-\xi+\frac{1}{2}} e^{-t} \left(1 + O\left(t^{-1}\right)\right) = 
     \sqrt{2\pi} t^{n+\frac{1}{2}} e^{-t} \left(1 + O\left(t^{-1}\right)\right). 
\]
Hence, as $n \rightarrow \infty$ with $t := n + \xi$ and $\xi = O(1)$, we obtain that 
\[
\sum_{k=1}^{n} \frac{(-1)^k t^{k-1}}{(k-1)!} = (-1)^n \times \frac{e^t}{2 \sqrt{2\pi t}} + 
     O\left(e^t t^{-\frac{3}{2}}\right). 
\]
The conclusion follows by taking $n := \floor{\log\log x}$ and $t := \log\log x$. 
\end{proof}

\begin{definition}
For $x \geq 1$, let the summatory function (\cf \cite{LUNE-DRESSLER}) 
\[
L_{\omega}(x) := \sum_{n \leq x} (-1)^{\omega(n)}. 
\]
\end{definition}

\begin{lemma} 
\label{cor_AsymptoticsForSignedSumsOfomegan_v1}
As $x \rightarrow \infty$, there is an absolute constant $A_0 > 0$ such that 
\[
L_{\omega}(x) = 
     \frac{(-1)^{\floor{\log\log x}} x}{A_0 \sqrt{2\pi \log\log x}} + 
     O\left(\frac{x}{\log\log x}\right). 
\]
\end{lemma}
\begin{proof}
An adaptation of the proof of 
Lemma \hlocalref{lemma_ConvenientIncGammaFuncTypePartialSumAsymptotics_v2} 
provides that for any $a \in \left(1, W(1)^{-1}\right)$ where 
$W(1)^{-1} \approx 1.76321$, the next partial sums satisfy 
\begin{align}
\notag 
\widehat{S}_a(x) & := 
     \frac{x}{\log x} \times \left\lvert \sum_{k=1}^{a \log\log x} \frac{(-1)^{k} (\log\log x)^{k-1}}{(k-1)!} 
     \right\rvert \\ 
\label{eqn_ConvenientIncGammaFuncTypePartialSumAsymptotics_va3} 
	& \phantom{:} = \frac{\sqrt{a} x}{\sqrt{2\pi}(a+1) a^{\{a\log\log x\}}} 
     \times \frac{(\log x)^{a-1-a\log a}}{\sqrt{\log\log x}} 
     \left(1 + O\left(\frac{1}{\log\log x}\right)\right). 
\end{align}
Here, we take $\{x\} = x - \floor{x} \in [0, 1)$ to denote the fractional part of $x$. 

Suppose that we set $a := \frac{3}{2}$ so that $a-1-a\log a \approx -0.108198$. 
Then we expand as  
\begin{align*}
L_{\omega}(x) & = 
     \sum_{k \leq \log\log x} 2 (-1)^{k} \pi_k(x) + 
     O\left(\widehat{S}_{\frac{3}{2}}(x) + 
     \#\left\{n \leq x: \omega(n) \geq \frac{3}{2}\log\log x\right\}\right). 
\end{align*} 
The justification for the above error term including $\widehat{S}_{\frac{3}{2}}(x)$ is that for 
$0 \leq z \leq \frac{3}{2}$ we can show that $\widetilde{\mathcal{G}}\left(z\right)$ is bounded. 
We apply the uniform asymptotics for $\pi_k(x)$ that hold as $x \rightarrow \infty$ when 
$1 \leq k \leq R \log\log x$ for $1 \leq R < 2$ from 
Remark \hlocalref{remark_MV_Pikx_FuncResultsAnnotated_v1} to evaluate the sums that provide the 
main term of the expansion in the previous equation. 
We have that $\widetilde{G}(0)=1$ and that for any 
$0 < |z| < 1$ the function $\widetilde{G}(z)$ is positive, monotone in $z$ and 
has an absolutely convergent series expansion in $z$ about zero. 
For integers $m \geq 1$, we see by induction that 
\[
\sum_{k \leq \log\log x} \frac{(-1)^k (k-1)^m (\log\log x)^{k-1-m}}{(k-1)!} = 
     \sum_{k \leq \log\log x} \frac{(-1)^{k+m} (\log\log x)^{k-1}}{(k-1)!} \left( 
     1 + O\left(\frac{1}{\log\log x}\right)\right). 
\]
We can argue 
by Lemma \hlocalref{lemma_ConvenientIncGammaFuncTypePartialSumAsymptotics_v2} 
and \eqref{eqn_ConvenientIncGammaFuncTypePartialSumAsymptotics_va3} 
that for all sufficiently large $x$ 
there is a limiting absolute constant $A_0 > 0$ such that 
\begin{align} 
\label{eqn_cor_AsymptoticsForSignedSumsOfomegan_v1_PfTag_v3} 
L_{\omega}(x) & = \frac{(-1)^{\floor{\log\log x}} x}{A_0 \sqrt{2\pi \log\log x}} + 
     O\left(E_{\omega}(x) + 
     \frac{x}{(\log x)^{0.108197} \sqrt{\log\log x}} + 
     \#\left\{n \leq x: \omega(x) \geq \frac{3}{2}\log\log x\right\}\right). 
\end{align} 
The error term in \eqref{eqn_cor_AsymptoticsForSignedSumsOfomegan_v1_PfTag_v3} 
is bounded as follows when $x \rightarrow \infty$ using Stirling's formula, 
\eqref{eqn_IncompleteGamma_PropA} and 
\eqref{eqn_IncompleteGamma_PropC}:  
\begin{align*} 
E_{\omega}(x) & \ll \frac{x}{\log x} \times 
     \sum_{1 \leq k \leq \log\log x} \frac{(\log\log x)^{k-2}}{(k-1)!} \\ 
     & = 
     \frac{x \Gamma(\log\log x, \log\log x)}{\Gamma(\log\log x + 1)} 
     = \frac{x}{2\log\log x} \left(1 + O\left(\frac{1}{\sqrt{\log\log x}}\right)\right). 
\end{align*}
Finally, by an application of the results in 
Remark \hlocalref{remark_MV_Pikx_FuncResultsAnnotated_v1}, the remaining term in 
the error estimate from \eqref{eqn_cor_AsymptoticsForSignedSumsOfomegan_v1_PfTag_v3} 
above satisfies 
\[
\#\left\{n \leq x: \omega(x) \geq \frac{3}{2}\log\log x\right\} \ll 
     \frac{x}{(\log x)^{0.108197}}. 
     \qedhere 
\] 
\end{proof}

\section{Inversion theorems for partial sums of Dirichlet convolutions}
\label{Section_PrelimProofs_Config} 
\label{subSection_PrelimProofs_Config_InversionTheorem}

We give a proof of the inversion type results in 
Theorem \hlocalref{theorem_SummatoryFuncsOfDirCvls} 
below by matrix methods. 
Related results on summations of Dirichlet convolutions and their 
functional inversions appear in 
\cite[\S 2.14; \S 3.10; \S 3.12; \cf \S 4.9, p.\ 95]{APOSTOLANUMT}. 

\begin{proof}[Proof of Theorem \hlocalref{theorem_SummatoryFuncsOfDirCvls}] 
\label{proofOf_theorem_SummatoryFuncsOfDirCvls} 
Let $h,r$ be arithmetic functions such that $r(1) \neq 0$. 
The following formulas hold for all $x \geq 1$: 
\begin{align} 
\notag 
S_{r \ast h}(x) & := \sum_{n=1}^{x} \sum_{d|n} r(n) h\left(\frac{n}{d}\right) = 
     \sum_{d=1}^x r(d) H\left(\floor{\frac{x}{d}}\right) \\ 
\label{eqn_proof_tag_PigAsthx_ExactSummationFormula_exp_v2} 
     & \phantom{:} = 
     \sum_{i=1}^x \left(R\left(\floor{\frac{x}{i}}\right) - R\left(\floor{\frac{x}{i+1}}\right)\right) H(i). 
\end{align} 
The first formula on the right-hand-side above is well known from the references. 
The second formula is justified directly using 
summation by parts as \cite[\S 2.10(ii)]{NISTHB} 
\begin{align*} 
S_{r \ast h}(x) & = \sum_{d=1}^x h(d) R\left(\floor{\frac{x}{d}}\right) \\ 
     & = \sum_{i \leq x} \left(\sum_{j \leq i} h(j)\right) \times 
     \left(R\left(\floor{\frac{x}{i}}\right) - 
     R\left(\floor{\frac{x}{i+1}}\right)\right). 
\end{align*} 
We form the invertible matrix of coefficients, denoted by $\hat{R}$ below, 
associated with the linear system defining $H(j)$ for all 
$1 \leq j \leq x$ in \eqref{eqn_proof_tag_PigAsthx_ExactSummationFormula_exp_v2} by setting 
\[
r_{x,j} := R\left(\floor{\frac{x}{j}}\right) - R\left(\floor{\frac{x}{j+1}}\right) 
     \equiv R_{x,j} - R_{x,j+1}, 
\] 
with 
\[
R_{x,j} := R\left(\Floor{x}{j}\right), \text{ for } 1 \leq j \leq x. 
\]
Since $r_{x,x} = R(1) = r(1) \neq 0$ for all $x \geq 1$ and $r_{x,j} = 0$ for all $j > x$, 
the matrix we have defined in this problem is lower triangular with a non-zero 
constant on its diagonals, and so is invertible. 
If we let $\hat{R} := (R_{x,j})$, then the next matrix is 
expressed by applying an invertible shift operation as 
\[
(r_{x,j}) = \hat{R} \left(I - U^{T}\right). 
\]
The square matrix $U$ of sufficiently large finite dimensions $N \times N$ for $N \geq x$ 
has $(i,j)^{th}$ entries for all $1 \leq i,j \leq N$ that are defined by 
$(U)_{i,j} = \delta_{i+1,j}$ so that 
\[
\left[\left(I - U^T\right)^{-1}\right]_{i,j} = \Iverson{j \leq i}. 
\]
We observe that 
\[
\Floor{x}{j} - \Floor{x-1}{j} = \begin{cases} 
     1, & \text{ if $j|x$; } \\ 
     0, & \text{ otherwise. } 
     \end{cases} 
\] 
The previous equation implies that 
\begin{equation} 
\label{eqn_proof_tag_FloorFuncDiffsOfSummatoryFuncs_v2} 
R\left(\floor{\frac{x}{j}}\right) - R\left(\floor{\frac{x-1}{j}}\right) = 
     \begin{cases} 
     r\left(\frac{x}{j}\right), & \text{ if $j | x$; } \\ 
     0, & \text{ otherwise. } 
     \end{cases}
\end{equation} 
We use the property in \eqref{eqn_proof_tag_FloorFuncDiffsOfSummatoryFuncs_v2} 
to shift the matrix $\hat{R}$, and then invert the result to obtain a matrix involving the 
Dirichlet inverse of $r$ as follows: 
\begin{align*} 
\left(\left(I-U^{T}\right) \hat{R}\right)^{-1} & = 
     \left(r\left(\frac{x}{j}\right) \Iverson{j|x}\right)^{-1} = 
     \left(r^{-1}\left(\frac{x}{j}\right) \Iverson{j|x}\right). 
\end{align*} 
Our target matrix in the inversion problem is defined by 
$$(r_{x,j}) = \left(I-U^{T}\right) \left(r\left(\frac{x}{j}\right) \Iverson{j|x}\right) \left(I-U^{T}\right)^{-1}.$$
We can express its inverse by a similarity transformation conjugated by shift operators in the form of 
\begin{align*} 
(r_{x,j})^{-1} & = \left(I-U^{T}\right)^{-1} \left(r^{-1}\left(\frac{x}{j}\right) 
     \Iverson{j|x}\right) \left(I-U^{T}\right) \\ 
     & = \left(\sum_{k=1}^{\floor{\frac{x}{j}}} r^{-1}(k)\right) \left(I-U^{T}\right) \\ 
     & = \left(\sum_{k=1}^{\floor{\frac{x}{j}}} r^{-1}(k) - \sum_{k=1}^{\floor{\frac{x}{j+1}}} r^{-1}(k)\right). 
\end{align*} 
The summatory function $H(x)$ is given exactly for any integers $x \geq 1$ 
by a vector product with the inverse matrix from the previous equation in the form of 
\begin{align*} 
H(x) & = \sum_{k=1}^x \left(\sum_{j=\floor{\frac{x}{k+1}}+1}^{\floor{\frac{x}{k}}} r^{-1}(j)\right) 
     \times S_{r \ast h}(k). 
\end{align*} 
We can prove a second inversion formula providing the coefficients of the summatory function 
$R^{-1}(j)$ for $1 \leq j \leq x$ from the last equation by adapting our argument to prove 
\eqref{eqn_proof_tag_PigAsthx_ExactSummationFormula_exp_v2} above. 
This leads to the alternate identity expressing $H(x)$ given by 
\[
H(x) = \sum_{k=1}^{x} r^{-1}(k) \times S_{r \ast h}\left(\Floor{x}{k}\right). 
     \qedhere 
\]
\end{proof} 

\clearpage 

\newpage
\section{Tables of computations involving $g(n)$ and its partial sums} 
\label{table_conjecture_Mertens_ginvSeq_approx_values}

\renewcommand{\arraystretch}{1.25}

\begin{table}[ht!]

\centering

\tiny
\begin{equation*}
\boxed{
\begin{array}{cc|cc|ccc|cc|cccc}
 n & n & \mathbf{Sqfree} & \mathbf{PPower} & g(n) & 
 \lambda(n) g(n) - \widehat{f}_1(n) & 
 \frac{\sum_{d|n} C_{\Omega}(d)}{|g(n)|} & 
 \mathcal{L}_{+}(n) & \mathcal{L}_{-}(n) & 
	G(n) & G_{+}(n) & G_{-}(n) & |G|(n) \\[0.15cm] \hline 
 1 & 1^1 & \text{Y} & \text{N} & 1 & 0 & 1.0000000 & 1.00000 & 0 & 1 & 1 & 0 & 1 \\
 2 & 2^1 & \text{Y} & \text{Y} & -2 & 0 & 1.0000000 & 0.500000 & 0.500000 & -1 & 1 & -2 & 3 \\
 3 & 3^1 & \text{Y} & \text{Y} & -2 & 0 & 1.0000000 & 0.333333 & 0.666667 & -3 & 1 & -4 & 5 \\
 4 & 2^2 & \text{N} & \text{Y} & 2 & 0 & 1.5000000 & 0.500000 & 0.500000 & -1 & 3 & -4 & 7 \\
 5 & 5^1 & \text{Y} & \text{Y} & -2 & 0 & 1.0000000 & 0.400000 & 0.600000 & -3 & 3 & -6 & 9 \\
 6 & 2^1 3^1 & \text{Y} & \text{N} & 5 & 0 & 1.0000000 & 0.500000 & 0.500000 & 2 & 8 & -6 & 14 \\
 7 & 7^1 & \text{Y} & \text{Y} & -2 & 0 & 1.0000000 & 0.428571 & 0.571429 & 0 & 8 & -8 & 16 \\
 8 & 2^3 & \text{N} & \text{Y} & -2 & 0 & 2.0000000 & 0.375000 & 0.625000 & -2 & 8 & -10 & 18 \\
 9 & 3^2 & \text{N} & \text{Y} & 2 & 0 & 1.5000000 & 0.444444 & 0.555556 & 0 & 10 & -10 & 20 \\
 10 & 2^1 5^1 & \text{Y} & \text{N} & 5 & 0 & 1.0000000 & 0.500000 & 0.500000 & 5 & 15 & -10 & 25 \\
 11 & 11^1 & \text{Y} & \text{Y} & -2 & 0 & 1.0000000 & 0.454545 & 0.545455 & 3 & 15 & -12 & 27 \\
 12 & 2^2 3^1 & \text{N} & \text{N} & -7 & 2 & 1.2857143 & 0.416667 & 0.583333 & -4 & 15 & -19 & 34 \\
 13 & 13^1 & \text{Y} & \text{Y} & -2 & 0 & 1.0000000 & 0.384615 & 0.615385 & -6 & 15 & -21 & 36 \\
 14 & 2^1 7^1 & \text{Y} & \text{N} & 5 & 0 & 1.0000000 & 0.428571 & 0.571429 & -1 & 20 & -21 & 41 \\
 15 & 3^1 5^1 & \text{Y} & \text{N} & 5 & 0 & 1.0000000 & 0.466667 & 0.533333 & 4 & 25 & -21 & 46 \\
 16 & 2^4 & \text{N} & \text{Y} & 2 & 0 & 2.5000000 & 0.500000 & 0.500000 & 6 & 27 & -21 & 48 \\
 17 & 17^1 & \text{Y} & \text{Y} & -2 & 0 & 1.0000000 & 0.470588 & 0.529412 & 4 & 27 & -23 & 50 \\
 18 & 2^1 3^2 & \text{N} & \text{N} & -7 & 2 & 1.2857143 & 0.444444 & 0.555556 & -3 & 27 & -30 & 57 \\
 19 & 19^1 & \text{Y} & \text{Y} & -2 & 0 & 1.0000000 & 0.421053 & 0.578947 & -5 & 27 & -32 & 59 \\
 20 & 2^2 5^1 & \text{N} & \text{N} & -7 & 2 & 1.2857143 & 0.400000 & 0.600000 & -12 & 27 & -39 & 66 \\
 21 & 3^1 7^1 & \text{Y} & \text{N} & 5 & 0 & 1.0000000 & 0.428571 & 0.571429 & -7 & 32 & -39 & 71 \\
 22 & 2^1 11^1 & \text{Y} & \text{N} & 5 & 0 & 1.0000000 & 0.454545 & 0.545455 & -2 & 37 & -39 & 76 \\
 23 & 23^1 & \text{Y} & \text{Y} & -2 & 0 & 1.0000000 & 0.434783 & 0.565217 & -4 & 37 & -41 & 78 \\
 24 & 2^3 3^1 & \text{N} & \text{N} & 9 & 4 & 1.5555556 & 0.458333 & 0.541667 & 5 & 46 & -41 & 87 \\
 25 & 5^2 & \text{N} & \text{Y} & 2 & 0 & 1.5000000 & 0.480000 & 0.520000 & 7 & 48 & -41 & 89 \\
 26 & 2^1 13^1 & \text{Y} & \text{N} & 5 & 0 & 1.0000000 & 0.500000 & 0.500000 & 12 & 53 & -41 & 94 \\
 27 & 3^3 & \text{N} & \text{Y} & -2 & 0 & 2.0000000 & 0.481481 & 0.518519 & 10 & 53 & -43 & 96 \\
 28 & 2^2 7^1 & \text{N} & \text{N} & -7 & 2 & 1.2857143 & 0.464286 & 0.535714 & 3 & 53 & -50 & 103 \\
 29 & 29^1 & \text{Y} & \text{Y} & -2 & 0 & 1.0000000 & 0.448276 & 0.551724 & 1 & 53 & -52 & 105 \\
 30 & 2^1 3^1 5^1 & \text{Y} & \text{N} & -16 & 0 & 1.0000000 & 0.433333 & 0.566667 & -15 & 53 & -68 & 121 \\
 31 & 31^1 & \text{Y} & \text{Y} & -2 & 0 & 1.0000000 & 0.419355 & 0.580645 & -17 & 53 & -70 & 123 \\
 32 & 2^5 & \text{N} & \text{Y} & -2 & 0 & 3.0000000 & 0.406250 & 0.593750 & -19 & 53 & -72 & 125 \\
 33 & 3^1 11^1 & \text{Y} & \text{N} & 5 & 0 & 1.0000000 & 0.424242 & 0.575758 & -14 & 58 & -72 & 130 \\
 34 & 2^1 17^1 & \text{Y} & \text{N} & 5 & 0 & 1.0000000 & 0.441176 & 0.558824 & -9 & 63 & -72 & 135 \\
 35 & 5^1 7^1 & \text{Y} & \text{N} & 5 & 0 & 1.0000000 & 0.457143 & 0.542857 & -4 & 68 & -72 & 140 \\
 36 & 2^2 3^2 & \text{N} & \text{N} & 14 & 9 & 1.3571429 & 0.472222 & 0.527778 & 10 & 82 & -72 & 154 \\
 37 & 37^1 & \text{Y} & \text{Y} & -2 & 0 & 1.0000000 & 0.459459 & 0.540541 & 8 & 82 & -74 & 156 \\
 38 & 2^1 19^1 & \text{Y} & \text{N} & 5 & 0 & 1.0000000 & 0.473684 & 0.526316 & 13 & 87 & -74 & 161 \\
 39 & 3^1 13^1 & \text{Y} & \text{N} & 5 & 0 & 1.0000000 & 0.487179 & 0.512821 & 18 & 92 & -74 & 166 \\
 40 & 2^3 5^1 & \text{N} & \text{N} & 9 & 4 & 1.5555556 & 0.500000 & 0.500000 & 27 & 101 & -74 & 175 \\
 41 & 41^1 & \text{Y} & \text{Y} & -2 & 0 & 1.0000000 & 0.487805 & 0.512195 & 25 & 101 & -76 & 177 \\
 42 & 2^1 3^1 7^1 & \text{Y} & \text{N} & -16 & 0 & 1.0000000 & 0.476190 & 0.523810 & 9 & 101 & -92 & 193 \\
 43 & 43^1 & \text{Y} & \text{Y} & -2 & 0 & 1.0000000 & 0.465116 & 0.534884 & 7 & 101 & -94 & 195 \\
 44 & 2^2 11^1 & \text{N} & \text{N} & -7 & 2 & 1.2857143 & 0.454545 & 0.545455 & 0 & 101 & -101 & 202 \\
 45 & 3^2 5^1 & \text{N} & \text{N} & -7 & 2 & 1.2857143 & 0.444444 & 0.555556 & -7 & 101 & -108 & 209 \\
 46 & 2^1 23^1 & \text{Y} & \text{N} & 5 & 0 & 1.0000000 & 0.456522 & 0.543478 & -2 & 106 & -108 & 214 \\
 47 & 47^1 & \text{Y} & \text{Y} & -2 & 0 & 1.0000000 & 0.446809 & 0.553191 & -4 & 106 & -110 & 216 \\
 48 & 2^4 3^1 & \text{N} & \text{N} & -11 & 6 & 1.8181818 & 0.437500 & 0.562500 & -15 & 106 & -121 & 227 \\
\end{array}
}
\end{equation*}

\hrule\medskip 

\captionsetup{singlelinecheck=off} 
\caption*{{\large{\rm \textbf{\rm \bf Table \thesection:} 
          Computations involving $g(n) \equiv (\omega+1)^{-1}(n)$ 
          and $G(x)$ for $1 \leq n \leq 500$.}} 
          \begin{itemize}[noitemsep,topsep=0pt,leftmargin=0.23in] 
          \item[$\blacktriangleright$] 
          The second column labeled $n$ provides the prime factorization of each 
	  $n$ so that the values of 
          $\omega(n)$ and $\Omega(n)$ are easily extracted. 
	  \item[$\blacktriangleright$]
          The next columns labeled \texttt{Sqfree} and \texttt{PPower}, respectively, 
          list inclusion of $n$ in the sets of squarefree integers and the prime powers. 
          \item[$\blacktriangleright$] 
          The next three columns provide the 
          explicit values of the inverse function $g(n)$ and compare its explicit value with other estimates. 
          For comparison, we define the function 
	  $\widehat{f}_1(n) := \sum_{k=0}^{\omega(n)} \binom{\omega(n)}{k} \times k!$. 
          \item[$\blacktriangleright$] 
          The next columns indicate properties of the summatory function of $g(n)$. 
          The notation for the (approximate) densities of the sign weight of $g(n)$ are defined as 
          $\mathcal{L}_{\pm}(x) := \frac{1}{n} \times \#\left\{n \leq x: \lambda(n) = \pm 1\right\}$. 
          \item[$\blacktriangleright$]
	  The next three 
          columns then show the sign weighted components to the signed summatory function, 
          $G(x) := \sum_{n \leq x} g(n)$, decomposed into its 
          respective positive and negative magnitude sum contributions: $G(x) = G_{+}(x) + G_{-}(x)$ where 
          $G_{+}(x) > 0$ and $G_{-}(x) < 0$ for all $x \geq 1$. 
          The rightmost column of the table provides the partial sums of the absolute value of the unsigned inverse sequence, 
          $|G|(n) := \sum_{k \leq n} |g(k)|$. 
          \end{itemize} 
          } 
\clearpage 

\end{table}

\newpage
\begin{table}[ht]

\centering

\tiny
\begin{equation*}
\boxed{
\begin{array}{cc|cc|ccc|cc|cccc}
 n & n & \mathbf{Sqfree} & \mathbf{PPower} & g(n) & 
 \lambda(n) g(n) - \widehat{f}_1(n) & 
 \frac{\sum_{d|n} C_{\Omega}(d)}{|g(n)|} & 
 \mathcal{L}_{+}(n) & \mathcal{L}_{-}(n) & 
 G(n) & G_{+}(n) & G_{-}(n) & |G|(n) \\[0.15cm] \hline 
 49 & 7^2 & \text{N} & \text{Y} & 2 & 0 & 1.5000000 & 0.448980 & 0.551020 & -13 & 108 & -121 & 229 \\
 50 & 2^1 5^2 & \text{N} & \text{N} & -7 & 2 & 1.2857143 & 0.440000 & 0.560000 & -20 & 108 & -128 & 236 \\
 51 & 3^1 17^1 & \text{Y} & \text{N} & 5 & 0 & 1.0000000 & 0.450980 & 0.549020 & -15 & 113 & -128 & 241 \\
 52 & 2^2 13^1 & \text{N} & \text{N} & -7 & 2 & 1.2857143 & 0.442308 & 0.557692 & -22 & 113 & -135 & 248 \\
 53 & 53^1 & \text{Y} & \text{Y} & -2 & 0 & 1.0000000 & 0.433962 & 0.566038 & -24 & 113 & -137 & 250 \\
 54 & 2^1 3^3 & \text{N} & \text{N} & 9 & 4 & 1.5555556 & 0.444444 & 0.555556 & -15 & 122 & -137 & 259 \\
 55 & 5^1 11^1 & \text{Y} & \text{N} & 5 & 0 & 1.0000000 & 0.454545 & 0.545455 & -10 & 127 & -137 & 264 \\
 56 & 2^3 7^1 & \text{N} & \text{N} & 9 & 4 & 1.5555556 & 0.464286 & 0.535714 & -1 & 136 & -137 & 273 \\
 57 & 3^1 19^1 & \text{Y} & \text{N} & 5 & 0 & 1.0000000 & 0.473684 & 0.526316 & 4 & 141 & -137 & 278 \\
 58 & 2^1 29^1 & \text{Y} & \text{N} & 5 & 0 & 1.0000000 & 0.482759 & 0.517241 & 9 & 146 & -137 & 283 \\
 59 & 59^1 & \text{Y} & \text{Y} & -2 & 0 & 1.0000000 & 0.474576 & 0.525424 & 7 & 146 & -139 & 285 \\
 60 & 2^2 3^1 5^1 & \text{N} & \text{N} & 30 & 14 & 1.1666667 & 0.483333 & 0.516667 & 37 & 176 & -139 & 315 \\
 61 & 61^1 & \text{Y} & \text{Y} & -2 & 0 & 1.0000000 & 0.475410 & 0.524590 & 35 & 176 & -141 & 317 \\
 62 & 2^1 31^1 & \text{Y} & \text{N} & 5 & 0 & 1.0000000 & 0.483871 & 0.516129 & 40 & 181 & -141 & 322 \\
 63 & 3^2 7^1 & \text{N} & \text{N} & -7 & 2 & 1.2857143 & 0.476190 & 0.523810 & 33 & 181 & -148 & 329 \\
 64 & 2^6 & \text{N} & \text{Y} & 2 & 0 & 3.5000000 & 0.484375 & 0.515625 & 35 & 183 & -148 & 331 \\
 65 & 5^1 13^1 & \text{Y} & \text{N} & 5 & 0 & 1.0000000 & 0.492308 & 0.507692 & 40 & 188 & -148 & 336 \\
 66 & 2^1 3^1 11^1 & \text{Y} & \text{N} & -16 & 0 & 1.0000000 & 0.484848 & 0.515152 & 24 & 188 & -164 & 352 \\
 67 & 67^1 & \text{Y} & \text{Y} & -2 & 0 & 1.0000000 & 0.477612 & 0.522388 & 22 & 188 & -166 & 354 \\
 68 & 2^2 17^1 & \text{N} & \text{N} & -7 & 2 & 1.2857143 & 0.470588 & 0.529412 & 15 & 188 & -173 & 361 \\
 69 & 3^1 23^1 & \text{Y} & \text{N} & 5 & 0 & 1.0000000 & 0.478261 & 0.521739 & 20 & 193 & -173 & 366 \\
 70 & 2^1 5^1 7^1 & \text{Y} & \text{N} & -16 & 0 & 1.0000000 & 0.471429 & 0.528571 & 4 & 193 & -189 & 382 \\
 71 & 71^1 & \text{Y} & \text{Y} & -2 & 0 & 1.0000000 & 0.464789 & 0.535211 & 2 & 193 & -191 & 384 \\
 72 & 2^3 3^2 & \text{N} & \text{N} & -23 & 18 & 1.4782609 & 0.458333 & 0.541667 & -21 & 193 & -214 & 407 \\
 73 & 73^1 & \text{Y} & \text{Y} & -2 & 0 & 1.0000000 & 0.452055 & 0.547945 & -23 & 193 & -216 & 409 \\
 74 & 2^1 37^1 & \text{Y} & \text{N} & 5 & 0 & 1.0000000 & 0.459459 & 0.540541 & -18 & 198 & -216 & 414 \\
 75 & 3^1 5^2 & \text{N} & \text{N} & -7 & 2 & 1.2857143 & 0.453333 & 0.546667 & -25 & 198 & -223 & 421 \\
 76 & 2^2 19^1 & \text{N} & \text{N} & -7 & 2 & 1.2857143 & 0.447368 & 0.552632 & -32 & 198 & -230 & 428 \\
 77 & 7^1 11^1 & \text{Y} & \text{N} & 5 & 0 & 1.0000000 & 0.454545 & 0.545455 & -27 & 203 & -230 & 433 \\
 78 & 2^1 3^1 13^1 & \text{Y} & \text{N} & -16 & 0 & 1.0000000 & 0.448718 & 0.551282 & -43 & 203 & -246 & 449 \\
 79 & 79^1 & \text{Y} & \text{Y} & -2 & 0 & 1.0000000 & 0.443038 & 0.556962 & -45 & 203 & -248 & 451 \\
 80 & 2^4 5^1 & \text{N} & \text{N} & -11 & 6 & 1.8181818 & 0.437500 & 0.562500 & -56 & 203 & -259 & 462 \\
 81 & 3^4 & \text{N} & \text{Y} & 2 & 0 & 2.5000000 & 0.444444 & 0.555556 & -54 & 205 & -259 & 464 \\
 82 & 2^1 41^1 & \text{Y} & \text{N} & 5 & 0 & 1.0000000 & 0.451220 & 0.548780 & -49 & 210 & -259 & 469 \\
 83 & 83^1 & \text{Y} & \text{Y} & -2 & 0 & 1.0000000 & 0.445783 & 0.554217 & -51 & 210 & -261 & 471 \\
 84 & 2^2 3^1 7^1 & \text{N} & \text{N} & 30 & 14 & 1.1666667 & 0.452381 & 0.547619 & -21 & 240 & -261 & 501 \\
 85 & 5^1 17^1 & \text{Y} & \text{N} & 5 & 0 & 1.0000000 & 0.458824 & 0.541176 & -16 & 245 & -261 & 506 \\
 86 & 2^1 43^1 & \text{Y} & \text{N} & 5 & 0 & 1.0000000 & 0.465116 & 0.534884 & -11 & 250 & -261 & 511 \\
 87 & 3^1 29^1 & \text{Y} & \text{N} & 5 & 0 & 1.0000000 & 0.471264 & 0.528736 & -6 & 255 & -261 & 516 \\
 88 & 2^3 11^1 & \text{N} & \text{N} & 9 & 4 & 1.5555556 & 0.477273 & 0.522727 & 3 & 264 & -261 & 525 \\
 89 & 89^1 & \text{Y} & \text{Y} & -2 & 0 & 1.0000000 & 0.471910 & 0.528090 & 1 & 264 & -263 & 527 \\
 90 & 2^1 3^2 5^1 & \text{N} & \text{N} & 30 & 14 & 1.1666667 & 0.477778 & 0.522222 & 31 & 294 & -263 & 557 \\
 91 & 7^1 13^1 & \text{Y} & \text{N} & 5 & 0 & 1.0000000 & 0.483516 & 0.516484 & 36 & 299 & -263 & 562 \\
 92 & 2^2 23^1 & \text{N} & \text{N} & -7 & 2 & 1.2857143 & 0.478261 & 0.521739 & 29 & 299 & -270 & 569 \\
 93 & 3^1 31^1 & \text{Y} & \text{N} & 5 & 0 & 1.0000000 & 0.483871 & 0.516129 & 34 & 304 & -270 & 574 \\
 94 & 2^1 47^1 & \text{Y} & \text{N} & 5 & 0 & 1.0000000 & 0.489362 & 0.510638 & 39 & 309 & -270 & 579 \\
 95 & 5^1 19^1 & \text{Y} & \text{N} & 5 & 0 & 1.0000000 & 0.494737 & 0.505263 & 44 & 314 & -270 & 584 \\
 96 & 2^5 3^1 & \text{N} & \text{N} & 13 & 8 & 2.0769231 & 0.500000 & 0.500000 & 57 & 327 & -270 & 597 \\
 97 & 97^1 & \text{Y} & \text{Y} & -2 & 0 & 1.0000000 & 0.494845 & 0.505155 & 55 & 327 & -272 & 599 \\
 98 & 2^1 7^2 & \text{N} & \text{N} & -7 & 2 & 1.2857143 & 0.489796 & 0.510204 & 48 & 327 & -279 & 606 \\
 99 & 3^2 11^1 & \text{N} & \text{N} & -7 & 2 & 1.2857143 & 0.484848 & 0.515152 & 41 & 327 & -286 & 613 \\
 100 & 2^2 5^2 & \text{N} & \text{N} & 14 & 9 & 1.3571429 & 0.490000 & 0.510000 & 55 & 341 & -286 & 627 \\
 101 & 101^1 & \text{Y} & \text{Y} & -2 & 0 & 1.0000000 & 0.485149 & 0.514851 & 53 & 341 & -288 & 629 \\
 102 & 2^1 3^1 17^1 & \text{Y} & \text{N} & -16 & 0 & 1.0000000 & 0.480392 & 0.519608 & 37 & 341 & -304 & 645 \\
 103 & 103^1 & \text{Y} & \text{Y} & -2 & 0 & 1.0000000 & 0.475728 & 0.524272 & 35 & 341 & -306 & 647 \\
 104 & 2^3 13^1 & \text{N} & \text{N} & 9 & 4 & 1.5555556 & 0.480769 & 0.519231 & 44 & 350 & -306 & 656 \\
 105 & 3^1 5^1 7^1 & \text{Y} & \text{N} & -16 & 0 & 1.0000000 & 0.476190 & 0.523810 & 28 & 350 & -322 & 672 \\
 106 & 2^1 53^1 & \text{Y} & \text{N} & 5 & 0 & 1.0000000 & 0.481132 & 0.518868 & 33 & 355 & -322 & 677 \\
 107 & 107^1 & \text{Y} & \text{Y} & -2 & 0 & 1.0000000 & 0.476636 & 0.523364 & 31 & 355 & -324 & 679 \\
 108 & 2^2 3^3 & \text{N} & \text{N} & -23 & 18 & 1.4782609 & 0.472222 & 0.527778 & 8 & 355 & -347 & 702 \\
 109 & 109^1 & \text{Y} & \text{Y} & -2 & 0 & 1.0000000 & 0.467890 & 0.532110 & 6 & 355 & -349 & 704 \\
 110 & 2^1 5^1 11^1 & \text{Y} & \text{N} & -16 & 0 & 1.0000000 & 0.463636 & 0.536364 & -10 & 355 & -365 & 720 \\
 111 & 3^1 37^1 & \text{Y} & \text{N} & 5 & 0 & 1.0000000 & 0.468468 & 0.531532 & -5 & 360 & -365 & 725 \\
 112 & 2^4 7^1 & \text{N} & \text{N} & -11 & 6 & 1.8181818 & 0.464286 & 0.535714 & -16 & 360 & -376 & 736 \\
 113 & 113^1 & \text{Y} & \text{Y} & -2 & 0 & 1.0000000 & 0.460177 & 0.539823 & -18 & 360 & -378 & 738 \\
 114 & 2^1 3^1 19^1 & \text{Y} & \text{N} & -16 & 0 & 1.0000000 & 0.456140 & 0.543860 & -34 & 360 & -394 & 754 \\
 115 & 5^1 23^1 & \text{Y} & \text{N} & 5 & 0 & 1.0000000 & 0.460870 & 0.539130 & -29 & 365 & -394 & 759 \\
 116 & 2^2 29^1 & \text{N} & \text{N} & -7 & 2 & 1.2857143 & 0.456897 & 0.543103 & -36 & 365 & -401 & 766 \\
 117 & 3^2 13^1 & \text{N} & \text{N} & -7 & 2 & 1.2857143 & 0.452991 & 0.547009 & -43 & 365 & -408 & 773 \\
 118 & 2^1 59^1 & \text{Y} & \text{N} & 5 & 0 & 1.0000000 & 0.457627 & 0.542373 & -38 & 370 & -408 & 778 \\
 119 & 7^1 17^1 & \text{Y} & \text{N} & 5 & 0 & 1.0000000 & 0.462185 & 0.537815 & -33 & 375 & -408 & 783 \\
 120 & 2^3 3^1 5^1 & \text{N} & \text{N} & -48 & 32 & 1.3333333 & 0.458333 & 0.541667 & -81 & 375 & -456 & 831 \\
 121 & 11^2 & \text{N} & \text{Y} & 2 & 0 & 1.5000000 & 0.462810 & 0.537190 & -79 & 377 & -456 & 833 \\
 122 & 2^1 61^1 & \text{Y} & \text{N} & 5 & 0 & 1.0000000 & 0.467213 & 0.532787 & -74 & 382 & -456 & 838 \\
 123 & 3^1 41^1 & \text{Y} & \text{N} & 5 & 0 & 1.0000000 & 0.471545 & 0.528455 & -69 & 387 & -456 & 843 \\
 124 & 2^2 31^1 & \text{N} & \text{N} & -7 & 2 & 1.2857143 & 0.467742 & 0.532258 & -76 & 387 & -463 & 850 \\
\end{array}
}
\end{equation*}
\clearpage 

\end{table} 


\newpage
\begin{table}[ht]

\centering

\tiny
\begin{equation*}
\boxed{
\begin{array}{cc|cc|ccc|cc|cccc}
 n & n & \mathbf{Sqfree} & \mathbf{PPower} & g(n) & 
 \lambda(n) g(n) - \widehat{f}_1(n) & 
 \frac{\sum_{d|n} C_{\Omega}(d)}{|g(n)|} & 
 \mathcal{L}_{+}(n) & \mathcal{L}_{-}(n) & 
 G(n) & G_{+}(n) & G_{-}(n) & |G|(n) \\[0.15cm] \hline 
 125 & 5^3 & \text{N} & \text{Y} & -2 & 0 & 2.0000000 & 0.464000 & 0.536000 & -78 & 387 & -465 & 852 \\
 126 & 2^1 3^2 7^1 & \text{N} & \text{N} & 30 & 14 & 1.1666667 & 0.468254 & 0.531746 & -48 & 417 & -465 & 882 \\
 127 & 127^1 & \text{Y} & \text{Y} & -2 & 0 & 1.0000000 & 0.464567 & 0.535433 & -50 & 417 & -467 & 884 \\
 128 & 2^7 & \text{N} & \text{Y} & -2 & 0 & 4.0000000 & 0.460938 & 0.539062 & -52 & 417 & -469 & 886 \\
 129 & 3^1 43^1 & \text{Y} & \text{N} & 5 & 0 & 1.0000000 & 0.465116 & 0.534884 & -47 & 422 & -469 & 891 \\
 130 & 2^1 5^1 13^1 & \text{Y} & \text{N} & -16 & 0 & 1.0000000 & 0.461538 & 0.538462 & -63 & 422 & -485 & 907 \\
 131 & 131^1 & \text{Y} & \text{Y} & -2 & 0 & 1.0000000 & 0.458015 & 0.541985 & -65 & 422 & -487 & 909 \\
 132 & 2^2 3^1 11^1 & \text{N} & \text{N} & 30 & 14 & 1.1666667 & 0.462121 & 0.537879 & -35 & 452 & -487 & 939 \\
 133 & 7^1 19^1 & \text{Y} & \text{N} & 5 & 0 & 1.0000000 & 0.466165 & 0.533835 & -30 & 457 & -487 & 944 \\
 134 & 2^1 67^1 & \text{Y} & \text{N} & 5 & 0 & 1.0000000 & 0.470149 & 0.529851 & -25 & 462 & -487 & 949 \\
 135 & 3^3 5^1 & \text{N} & \text{N} & 9 & 4 & 1.5555556 & 0.474074 & 0.525926 & -16 & 471 & -487 & 958 \\
 136 & 2^3 17^1 & \text{N} & \text{N} & 9 & 4 & 1.5555556 & 0.477941 & 0.522059 & -7 & 480 & -487 & 967 \\
 137 & 137^1 & \text{Y} & \text{Y} & -2 & 0 & 1.0000000 & 0.474453 & 0.525547 & -9 & 480 & -489 & 969 \\
 138 & 2^1 3^1 23^1 & \text{Y} & \text{N} & -16 & 0 & 1.0000000 & 0.471014 & 0.528986 & -25 & 480 & -505 & 985 \\
 139 & 139^1 & \text{Y} & \text{Y} & -2 & 0 & 1.0000000 & 0.467626 & 0.532374 & -27 & 480 & -507 & 987 \\
 140 & 2^2 5^1 7^1 & \text{N} & \text{N} & 30 & 14 & 1.1666667 & 0.471429 & 0.528571 & 3 & 510 & -507 & 1017 \\
 141 & 3^1 47^1 & \text{Y} & \text{N} & 5 & 0 & 1.0000000 & 0.475177 & 0.524823 & 8 & 515 & -507 & 1022 \\
 142 & 2^1 71^1 & \text{Y} & \text{N} & 5 & 0 & 1.0000000 & 0.478873 & 0.521127 & 13 & 520 & -507 & 1027 \\
 143 & 11^1 13^1 & \text{Y} & \text{N} & 5 & 0 & 1.0000000 & 0.482517 & 0.517483 & 18 & 525 & -507 & 1032 \\
 144 & 2^4 3^2 & \text{N} & \text{N} & 34 & 29 & 1.6176471 & 0.486111 & 0.513889 & 52 & 559 & -507 & 1066 \\
 145 & 5^1 29^1 & \text{Y} & \text{N} & 5 & 0 & 1.0000000 & 0.489655 & 0.510345 & 57 & 564 & -507 & 1071 \\
 146 & 2^1 73^1 & \text{Y} & \text{N} & 5 & 0 & 1.0000000 & 0.493151 & 0.506849 & 62 & 569 & -507 & 1076 \\
 147 & 3^1 7^2 & \text{N} & \text{N} & -7 & 2 & 1.2857143 & 0.489796 & 0.510204 & 55 & 569 & -514 & 1083 \\
 148 & 2^2 37^1 & \text{N} & \text{N} & -7 & 2 & 1.2857143 & 0.486486 & 0.513514 & 48 & 569 & -521 & 1090 \\
 149 & 149^1 & \text{Y} & \text{Y} & -2 & 0 & 1.0000000 & 0.483221 & 0.516779 & 46 & 569 & -523 & 1092 \\
 150 & 2^1 3^1 5^2 & \text{N} & \text{N} & 30 & 14 & 1.1666667 & 0.486667 & 0.513333 & 76 & 599 & -523 & 1122 \\
 151 & 151^1 & \text{Y} & \text{Y} & -2 & 0 & 1.0000000 & 0.483444 & 0.516556 & 74 & 599 & -525 & 1124 \\
 152 & 2^3 19^1 & \text{N} & \text{N} & 9 & 4 & 1.5555556 & 0.486842 & 0.513158 & 83 & 608 & -525 & 1133 \\
 153 & 3^2 17^1 & \text{N} & \text{N} & -7 & 2 & 1.2857143 & 0.483660 & 0.516340 & 76 & 608 & -532 & 1140 \\
 154 & 2^1 7^1 11^1 & \text{Y} & \text{N} & -16 & 0 & 1.0000000 & 0.480519 & 0.519481 & 60 & 608 & -548 & 1156 \\
 155 & 5^1 31^1 & \text{Y} & \text{N} & 5 & 0 & 1.0000000 & 0.483871 & 0.516129 & 65 & 613 & -548 & 1161 \\
 156 & 2^2 3^1 13^1 & \text{N} & \text{N} & 30 & 14 & 1.1666667 & 0.487179 & 0.512821 & 95 & 643 & -548 & 1191 \\
 157 & 157^1 & \text{Y} & \text{Y} & -2 & 0 & 1.0000000 & 0.484076 & 0.515924 & 93 & 643 & -550 & 1193 \\
 158 & 2^1 79^1 & \text{Y} & \text{N} & 5 & 0 & 1.0000000 & 0.487342 & 0.512658 & 98 & 648 & -550 & 1198 \\
 159 & 3^1 53^1 & \text{Y} & \text{N} & 5 & 0 & 1.0000000 & 0.490566 & 0.509434 & 103 & 653 & -550 & 1203 \\
 160 & 2^5 5^1 & \text{N} & \text{N} & 13 & 8 & 2.0769231 & 0.493750 & 0.506250 & 116 & 666 & -550 & 1216 \\
 161 & 7^1 23^1 & \text{Y} & \text{N} & 5 & 0 & 1.0000000 & 0.496894 & 0.503106 & 121 & 671 & -550 & 1221 \\
 162 & 2^1 3^4 & \text{N} & \text{N} & -11 & 6 & 1.8181818 & 0.493827 & 0.506173 & 110 & 671 & -561 & 1232 \\
 163 & 163^1 & \text{Y} & \text{Y} & -2 & 0 & 1.0000000 & 0.490798 & 0.509202 & 108 & 671 & -563 & 1234 \\
 164 & 2^2 41^1 & \text{N} & \text{N} & -7 & 2 & 1.2857143 & 0.487805 & 0.512195 & 101 & 671 & -570 & 1241 \\
 165 & 3^1 5^1 11^1 & \text{Y} & \text{N} & -16 & 0 & 1.0000000 & 0.484848 & 0.515152 & 85 & 671 & -586 & 1257 \\
 166 & 2^1 83^1 & \text{Y} & \text{N} & 5 & 0 & 1.0000000 & 0.487952 & 0.512048 & 90 & 676 & -586 & 1262 \\
 167 & 167^1 & \text{Y} & \text{Y} & -2 & 0 & 1.0000000 & 0.485030 & 0.514970 & 88 & 676 & -588 & 1264 \\
 168 & 2^3 3^1 7^1 & \text{N} & \text{N} & -48 & 32 & 1.3333333 & 0.482143 & 0.517857 & 40 & 676 & -636 & 1312 \\
 169 & 13^2 & \text{N} & \text{Y} & 2 & 0 & 1.5000000 & 0.485207 & 0.514793 & 42 & 678 & -636 & 1314 \\
 170 & 2^1 5^1 17^1 & \text{Y} & \text{N} & -16 & 0 & 1.0000000 & 0.482353 & 0.517647 & 26 & 678 & -652 & 1330 \\
 171 & 3^2 19^1 & \text{N} & \text{N} & -7 & 2 & 1.2857143 & 0.479532 & 0.520468 & 19 & 678 & -659 & 1337 \\
 172 & 2^2 43^1 & \text{N} & \text{N} & -7 & 2 & 1.2857143 & 0.476744 & 0.523256 & 12 & 678 & -666 & 1344 \\
 173 & 173^1 & \text{Y} & \text{Y} & -2 & 0 & 1.0000000 & 0.473988 & 0.526012 & 10 & 678 & -668 & 1346 \\
 174 & 2^1 3^1 29^1 & \text{Y} & \text{N} & -16 & 0 & 1.0000000 & 0.471264 & 0.528736 & -6 & 678 & -684 & 1362 \\
 175 & 5^2 7^1 & \text{N} & \text{N} & -7 & 2 & 1.2857143 & 0.468571 & 0.531429 & -13 & 678 & -691 & 1369 \\
 176 & 2^4 11^1 & \text{N} & \text{N} & -11 & 6 & 1.8181818 & 0.465909 & 0.534091 & -24 & 678 & -702 & 1380 \\
 177 & 3^1 59^1 & \text{Y} & \text{N} & 5 & 0 & 1.0000000 & 0.468927 & 0.531073 & -19 & 683 & -702 & 1385 \\
 178 & 2^1 89^1 & \text{Y} & \text{N} & 5 & 0 & 1.0000000 & 0.471910 & 0.528090 & -14 & 688 & -702 & 1390 \\
 179 & 179^1 & \text{Y} & \text{Y} & -2 & 0 & 1.0000000 & 0.469274 & 0.530726 & -16 & 688 & -704 & 1392 \\
 180 & 2^2 3^2 5^1 & \text{N} & \text{N} & -74 & 58 & 1.2162162 & 0.466667 & 0.533333 & -90 & 688 & -778 & 1466 \\
 181 & 181^1 & \text{Y} & \text{Y} & -2 & 0 & 1.0000000 & 0.464088 & 0.535912 & -92 & 688 & -780 & 1468 \\
 182 & 2^1 7^1 13^1 & \text{Y} & \text{N} & -16 & 0 & 1.0000000 & 0.461538 & 0.538462 & -108 & 688 & -796 & 1484 \\
 183 & 3^1 61^1 & \text{Y} & \text{N} & 5 & 0 & 1.0000000 & 0.464481 & 0.535519 & -103 & 693 & -796 & 1489 \\
 184 & 2^3 23^1 & \text{N} & \text{N} & 9 & 4 & 1.5555556 & 0.467391 & 0.532609 & -94 & 702 & -796 & 1498 \\
 185 & 5^1 37^1 & \text{Y} & \text{N} & 5 & 0 & 1.0000000 & 0.470270 & 0.529730 & -89 & 707 & -796 & 1503 \\
 186 & 2^1 3^1 31^1 & \text{Y} & \text{N} & -16 & 0 & 1.0000000 & 0.467742 & 0.532258 & -105 & 707 & -812 & 1519 \\
 187 & 11^1 17^1 & \text{Y} & \text{N} & 5 & 0 & 1.0000000 & 0.470588 & 0.529412 & -100 & 712 & -812 & 1524 \\
 188 & 2^2 47^1 & \text{N} & \text{N} & -7 & 2 & 1.2857143 & 0.468085 & 0.531915 & -107 & 712 & -819 & 1531 \\
 189 & 3^3 7^1 & \text{N} & \text{N} & 9 & 4 & 1.5555556 & 0.470899 & 0.529101 & -98 & 721 & -819 & 1540 \\
 190 & 2^1 5^1 19^1 & \text{Y} & \text{N} & -16 & 0 & 1.0000000 & 0.468421 & 0.531579 & -114 & 721 & -835 & 1556 \\
 191 & 191^1 & \text{Y} & \text{Y} & -2 & 0 & 1.0000000 & 0.465969 & 0.534031 & -116 & 721 & -837 & 1558 \\
 192 & 2^6 3^1 & \text{N} & \text{N} & -15 & 10 & 2.3333333 & 0.463542 & 0.536458 & -131 & 721 & -852 & 1573 \\
 193 & 193^1 & \text{Y} & \text{Y} & -2 & 0 & 1.0000000 & 0.461140 & 0.538860 & -133 & 721 & -854 & 1575 \\
 194 & 2^1 97^1 & \text{Y} & \text{N} & 5 & 0 & 1.0000000 & 0.463918 & 0.536082 & -128 & 726 & -854 & 1580 \\
 195 & 3^1 5^1 13^1 & \text{Y} & \text{N} & -16 & 0 & 1.0000000 & 0.461538 & 0.538462 & -144 & 726 & -870 & 1596 \\
 196 & 2^2 7^2 & \text{N} & \text{N} & 14 & 9 & 1.3571429 & 0.464286 & 0.535714 & -130 & 740 & -870 & 1610 \\
 197 & 197^1 & \text{Y} & \text{Y} & -2 & 0 & 1.0000000 & 0.461929 & 0.538071 & -132 & 740 & -872 & 1612 \\
 198 & 2^1 3^2 11^1 & \text{N} & \text{N} & 30 & 14 & 1.1666667 & 0.464646 & 0.535354 & -102 & 770 & -872 & 1642 \\
 199 & 199^1 & \text{Y} & \text{Y} & -2 & 0 & 1.0000000 & 0.462312 & 0.537688 & -104 & 770 & -874 & 1644 \\
 200 & 2^3 5^2 & \text{N} & \text{N} & -23 & 18 & 1.4782609 & 0.460000 & 0.540000 & -127 & 770 & -897 & 1667 \\
\end{array}
}
\end{equation*}
\clearpage 

\end{table} 

\newpage
\begin{table}[ht]

\centering

\tiny
\begin{equation*}
\boxed{
\begin{array}{cc|cc|ccc|cc|cccc}
 n & n & \mathbf{Sqfree} & \mathbf{PPower} & g(n) & 
 \lambda(n) g(n) - \widehat{f}_1(n) & 
 \frac{\sum_{d|n} C_{\Omega}(d)}{|g(n)|} & 
 \mathcal{L}_{+}(n) & \mathcal{L}_{-}(n) & 
 G(n) & G_{+}(n) & G_{-}(n) & |G|(n) \\[0.15cm] \hline 
 201 & 3^1 67^1 & \text{Y} & \text{N} & 5 & 0 & 1.0000000 & 0.462687 & 0.537313 & -122 & 775 & -897 & 1672 \\
 202 & 2^1 101^1 & \text{Y} & \text{N} & 5 & 0 & 1.0000000 & 0.465347 & 0.534653 & -117 & 780 & -897 & 1677 \\
 203 & 7^1 29^1 & \text{Y} & \text{N} & 5 & 0 & 1.0000000 & 0.467980 & 0.532020 & -112 & 785 & -897 & 1682 \\
 204 & 2^2 3^1 17^1 & \text{N} & \text{N} & 30 & 14 & 1.1666667 & 0.470588 & 0.529412 & -82 & 815 & -897 & 1712 \\
 205 & 5^1 41^1 & \text{Y} & \text{N} & 5 & 0 & 1.0000000 & 0.473171 & 0.526829 & -77 & 820 & -897 & 1717 \\
 206 & 2^1 103^1 & \text{Y} & \text{N} & 5 & 0 & 1.0000000 & 0.475728 & 0.524272 & -72 & 825 & -897 & 1722 \\
 207 & 3^2 23^1 & \text{N} & \text{N} & -7 & 2 & 1.2857143 & 0.473430 & 0.526570 & -79 & 825 & -904 & 1729 \\
 208 & 2^4 13^1 & \text{N} & \text{N} & -11 & 6 & 1.8181818 & 0.471154 & 0.528846 & -90 & 825 & -915 & 1740 \\
 209 & 11^1 19^1 & \text{Y} & \text{N} & 5 & 0 & 1.0000000 & 0.473684 & 0.526316 & -85 & 830 & -915 & 1745 \\
 210 & 2^1 3^1 5^1 7^1 & \text{Y} & \text{N} & 65 & 0 & 1.0000000 & 0.476190 & 0.523810 & -20 & 895 & -915 & 1810 \\
 211 & 211^1 & \text{Y} & \text{Y} & -2 & 0 & 1.0000000 & 0.473934 & 0.526066 & -22 & 895 & -917 & 1812 \\
 212 & 2^2 53^1 & \text{N} & \text{N} & -7 & 2 & 1.2857143 & 0.471698 & 0.528302 & -29 & 895 & -924 & 1819 \\
 213 & 3^1 71^1 & \text{Y} & \text{N} & 5 & 0 & 1.0000000 & 0.474178 & 0.525822 & -24 & 900 & -924 & 1824 \\
 214 & 2^1 107^1 & \text{Y} & \text{N} & 5 & 0 & 1.0000000 & 0.476636 & 0.523364 & -19 & 905 & -924 & 1829 \\
 215 & 5^1 43^1 & \text{Y} & \text{N} & 5 & 0 & 1.0000000 & 0.479070 & 0.520930 & -14 & 910 & -924 & 1834 \\
 216 & 2^3 3^3 & \text{N} & \text{N} & 46 & 41 & 1.5000000 & 0.481481 & 0.518519 & 32 & 956 & -924 & 1880 \\
 217 & 7^1 31^1 & \text{Y} & \text{N} & 5 & 0 & 1.0000000 & 0.483871 & 0.516129 & 37 & 961 & -924 & 1885 \\
 218 & 2^1 109^1 & \text{Y} & \text{N} & 5 & 0 & 1.0000000 & 0.486239 & 0.513761 & 42 & 966 & -924 & 1890 \\
 219 & 3^1 73^1 & \text{Y} & \text{N} & 5 & 0 & 1.0000000 & 0.488584 & 0.511416 & 47 & 971 & -924 & 1895 \\
 220 & 2^2 5^1 11^1 & \text{N} & \text{N} & 30 & 14 & 1.1666667 & 0.490909 & 0.509091 & 77 & 1001 & -924 & 1925 \\
 221 & 13^1 17^1 & \text{Y} & \text{N} & 5 & 0 & 1.0000000 & 0.493213 & 0.506787 & 82 & 1006 & -924 & 1930 \\
 222 & 2^1 3^1 37^1 & \text{Y} & \text{N} & -16 & 0 & 1.0000000 & 0.490991 & 0.509009 & 66 & 1006 & -940 & 1946 \\
 223 & 223^1 & \text{Y} & \text{Y} & -2 & 0 & 1.0000000 & 0.488789 & 0.511211 & 64 & 1006 & -942 & 1948 \\
 224 & 2^5 7^1 & \text{N} & \text{N} & 13 & 8 & 2.0769231 & 0.491071 & 0.508929 & 77 & 1019 & -942 & 1961 \\
 225 & 3^2 5^2 & \text{N} & \text{N} & 14 & 9 & 1.3571429 & 0.493333 & 0.506667 & 91 & 1033 & -942 & 1975 \\
 226 & 2^1 113^1 & \text{Y} & \text{N} & 5 & 0 & 1.0000000 & 0.495575 & 0.504425 & 96 & 1038 & -942 & 1980 \\
 227 & 227^1 & \text{Y} & \text{Y} & -2 & 0 & 1.0000000 & 0.493392 & 0.506608 & 94 & 1038 & -944 & 1982 \\
 228 & 2^2 3^1 19^1 & \text{N} & \text{N} & 30 & 14 & 1.1666667 & 0.495614 & 0.504386 & 124 & 1068 & -944 & 2012 \\
 229 & 229^1 & \text{Y} & \text{Y} & -2 & 0 & 1.0000000 & 0.493450 & 0.506550 & 122 & 1068 & -946 & 2014 \\
 230 & 2^1 5^1 23^1 & \text{Y} & \text{N} & -16 & 0 & 1.0000000 & 0.491304 & 0.508696 & 106 & 1068 & -962 & 2030 \\
 231 & 3^1 7^1 11^1 & \text{Y} & \text{N} & -16 & 0 & 1.0000000 & 0.489177 & 0.510823 & 90 & 1068 & -978 & 2046 \\
 232 & 2^3 29^1 & \text{N} & \text{N} & 9 & 4 & 1.5555556 & 0.491379 & 0.508621 & 99 & 1077 & -978 & 2055 \\
 233 & 233^1 & \text{Y} & \text{Y} & -2 & 0 & 1.0000000 & 0.489270 & 0.510730 & 97 & 1077 & -980 & 2057 \\
 234 & 2^1 3^2 13^1 & \text{N} & \text{N} & 30 & 14 & 1.1666667 & 0.491453 & 0.508547 & 127 & 1107 & -980 & 2087 \\
 235 & 5^1 47^1 & \text{Y} & \text{N} & 5 & 0 & 1.0000000 & 0.493617 & 0.506383 & 132 & 1112 & -980 & 2092 \\
 236 & 2^2 59^1 & \text{N} & \text{N} & -7 & 2 & 1.2857143 & 0.491525 & 0.508475 & 125 & 1112 & -987 & 2099 \\
 237 & 3^1 79^1 & \text{Y} & \text{N} & 5 & 0 & 1.0000000 & 0.493671 & 0.506329 & 130 & 1117 & -987 & 2104 \\
 238 & 2^1 7^1 17^1 & \text{Y} & \text{N} & -16 & 0 & 1.0000000 & 0.491597 & 0.508403 & 114 & 1117 & -1003 & 2120 \\
 239 & 239^1 & \text{Y} & \text{Y} & -2 & 0 & 1.0000000 & 0.489540 & 0.510460 & 112 & 1117 & -1005 & 2122 \\
 240 & 2^4 3^1 5^1 & \text{N} & \text{N} & 70 & 54 & 1.5000000 & 0.491667 & 0.508333 & 182 & 1187 & -1005 & 2192 \\
 241 & 241^1 & \text{Y} & \text{Y} & -2 & 0 & 1.0000000 & 0.489627 & 0.510373 & 180 & 1187 & -1007 & 2194 \\
 242 & 2^1 11^2 & \text{N} & \text{N} & -7 & 2 & 1.2857143 & 0.487603 & 0.512397 & 173 & 1187 & -1014 & 2201 \\
 243 & 3^5 & \text{N} & \text{Y} & -2 & 0 & 3.0000000 & 0.485597 & 0.514403 & 171 & 1187 & -1016 & 2203 \\
 244 & 2^2 61^1 & \text{N} & \text{N} & -7 & 2 & 1.2857143 & 0.483607 & 0.516393 & 164 & 1187 & -1023 & 2210 \\
 245 & 5^1 7^2 & \text{N} & \text{N} & -7 & 2 & 1.2857143 & 0.481633 & 0.518367 & 157 & 1187 & -1030 & 2217 \\
 246 & 2^1 3^1 41^1 & \text{Y} & \text{N} & -16 & 0 & 1.0000000 & 0.479675 & 0.520325 & 141 & 1187 & -1046 & 2233 \\
 247 & 13^1 19^1 & \text{Y} & \text{N} & 5 & 0 & 1.0000000 & 0.481781 & 0.518219 & 146 & 1192 & -1046 & 2238 \\
 248 & 2^3 31^1 & \text{N} & \text{N} & 9 & 4 & 1.5555556 & 0.483871 & 0.516129 & 155 & 1201 & -1046 & 2247 \\
 249 & 3^1 83^1 & \text{Y} & \text{N} & 5 & 0 & 1.0000000 & 0.485944 & 0.514056 & 160 & 1206 & -1046 & 2252 \\
 250 & 2^1 5^3 & \text{N} & \text{N} & 9 & 4 & 1.5555556 & 0.488000 & 0.512000 & 169 & 1215 & -1046 & 2261 \\
 251 & 251^1 & \text{Y} & \text{Y} & -2 & 0 & 1.0000000 & 0.486056 & 0.513944 & 167 & 1215 & -1048 & 2263 \\
 252 & 2^2 3^2 7^1 & \text{N} & \text{N} & -74 & 58 & 1.2162162 & 0.484127 & 0.515873 & 93 & 1215 & -1122 & 2337 \\
 253 & 11^1 23^1 & \text{Y} & \text{N} & 5 & 0 & 1.0000000 & 0.486166 & 0.513834 & 98 & 1220 & -1122 & 2342 \\
 254 & 2^1 127^1 & \text{Y} & \text{N} & 5 & 0 & 1.0000000 & 0.488189 & 0.511811 & 103 & 1225 & -1122 & 2347 \\
 255 & 3^1 5^1 17^1 & \text{Y} & \text{N} & -16 & 0 & 1.0000000 & 0.486275 & 0.513725 & 87 & 1225 & -1138 & 2363 \\
 256 & 2^8 & \text{N} & \text{Y} & 2 & 0 & 4.5000000 & 0.488281 & 0.511719 & 89 & 1227 & -1138 & 2365 \\
 257 & 257^1 & \text{Y} & \text{Y} & -2 & 0 & 1.0000000 & 0.486381 & 0.513619 & 87 & 1227 & -1140 & 2367 \\
 258 & 2^1 3^1 43^1 & \text{Y} & \text{N} & -16 & 0 & 1.0000000 & 0.484496 & 0.515504 & 71 & 1227 & -1156 & 2383 \\
 259 & 7^1 37^1 & \text{Y} & \text{N} & 5 & 0 & 1.0000000 & 0.486486 & 0.513514 & 76 & 1232 & -1156 & 2388 \\
 260 & 2^2 5^1 13^1 & \text{N} & \text{N} & 30 & 14 & 1.1666667 & 0.488462 & 0.511538 & 106 & 1262 & -1156 & 2418 \\
 261 & 3^2 29^1 & \text{N} & \text{N} & -7 & 2 & 1.2857143 & 0.486590 & 0.513410 & 99 & 1262 & -1163 & 2425 \\
 262 & 2^1 131^1 & \text{Y} & \text{N} & 5 & 0 & 1.0000000 & 0.488550 & 0.511450 & 104 & 1267 & -1163 & 2430 \\
 263 & 263^1 & \text{Y} & \text{Y} & -2 & 0 & 1.0000000 & 0.486692 & 0.513308 & 102 & 1267 & -1165 & 2432 \\
 264 & 2^3 3^1 11^1 & \text{N} & \text{N} & -48 & 32 & 1.3333333 & 0.484848 & 0.515152 & 54 & 1267 & -1213 & 2480 \\
 265 & 5^1 53^1 & \text{Y} & \text{N} & 5 & 0 & 1.0000000 & 0.486792 & 0.513208 & 59 & 1272 & -1213 & 2485 \\
 266 & 2^1 7^1 19^1 & \text{Y} & \text{N} & -16 & 0 & 1.0000000 & 0.484962 & 0.515038 & 43 & 1272 & -1229 & 2501 \\
 267 & 3^1 89^1 & \text{Y} & \text{N} & 5 & 0 & 1.0000000 & 0.486891 & 0.513109 & 48 & 1277 & -1229 & 2506 \\
 268 & 2^2 67^1 & \text{N} & \text{N} & -7 & 2 & 1.2857143 & 0.485075 & 0.514925 & 41 & 1277 & -1236 & 2513 \\
 269 & 269^1 & \text{Y} & \text{Y} & -2 & 0 & 1.0000000 & 0.483271 & 0.516729 & 39 & 1277 & -1238 & 2515 \\
 270 & 2^1 3^3 5^1 & \text{N} & \text{N} & -48 & 32 & 1.3333333 & 0.481481 & 0.518519 & -9 & 1277 & -1286 & 2563 \\
 271 & 271^1 & \text{Y} & \text{Y} & -2 & 0 & 1.0000000 & 0.479705 & 0.520295 & -11 & 1277 & -1288 & 2565 \\
 272 & 2^4 17^1 & \text{N} & \text{N} & -11 & 6 & 1.8181818 & 0.477941 & 0.522059 & -22 & 1277 & -1299 & 2576 \\
 273 & 3^1 7^1 13^1 & \text{Y} & \text{N} & -16 & 0 & 1.0000000 & 0.476190 & 0.523810 & -38 & 1277 & -1315 & 2592 \\
 274 & 2^1 137^1 & \text{Y} & \text{N} & 5 & 0 & 1.0000000 & 0.478102 & 0.521898 & -33 & 1282 & -1315 & 2597 \\
 275 & 5^2 11^1 & \text{N} & \text{N} & -7 & 2 & 1.2857143 & 0.476364 & 0.523636 & -40 & 1282 & -1322 & 2604 \\
\end{array}
}
\end{equation*}
\clearpage 

\end{table} 

\newpage
\begin{table}[ht]

\centering

\tiny
\begin{equation*}
\boxed{
\begin{array}{cc|cc|ccc|cc|cccc}
 n & n & \mathbf{Sqfree} & \mathbf{PPower} & g(n) & 
 \lambda(n) g(n) - \widehat{f}_1(n) & 
 \frac{\sum_{d|n} C_{\Omega}(d)}{|g(n)|} & 
 \mathcal{L}_{+}(n) & \mathcal{L}_{-}(n) & 
 G(n) & G_{+}(n) & G_{-}(n) & |G|(n) \\[0.15cm] \hline 
 276 & 2^2 3^1 23^1 & \text{N} & \text{N} & 30 & 14 & 1.1666667 & 0.478261 & 0.521739 & -10 & 1312 & -1322 & 2634 \\
 277 & 277^1 & \text{Y} & \text{Y} & -2 & 0 & 1.0000000 & 0.476534 & 0.523466 & -12 & 1312 & -1324 & 2636 \\
 278 & 2^1 139^1 & \text{Y} & \text{N} & 5 & 0 & 1.0000000 & 0.478417 & 0.521583 & -7 & 1317 & -1324 & 2641 \\
 279 & 3^2 31^1 & \text{N} & \text{N} & -7 & 2 & 1.2857143 & 0.476703 & 0.523297 & -14 & 1317 & -1331 & 2648 \\
 280 & 2^3 5^1 7^1 & \text{N} & \text{N} & -48 & 32 & 1.3333333 & 0.475000 & 0.525000 & -62 & 1317 & -1379 & 2696 \\
 281 & 281^1 & \text{Y} & \text{Y} & -2 & 0 & 1.0000000 & 0.473310 & 0.526690 & -64 & 1317 & -1381 & 2698 \\
 282 & 2^1 3^1 47^1 & \text{Y} & \text{N} & -16 & 0 & 1.0000000 & 0.471631 & 0.528369 & -80 & 1317 & -1397 & 2714 \\
 283 & 283^1 & \text{Y} & \text{Y} & -2 & 0 & 1.0000000 & 0.469965 & 0.530035 & -82 & 1317 & -1399 & 2716 \\
 284 & 2^2 71^1 & \text{N} & \text{N} & -7 & 2 & 1.2857143 & 0.468310 & 0.531690 & -89 & 1317 & -1406 & 2723 \\
 285 & 3^1 5^1 19^1 & \text{Y} & \text{N} & -16 & 0 & 1.0000000 & 0.466667 & 0.533333 & -105 & 1317 & -1422 & 2739 \\
 286 & 2^1 11^1 13^1 & \text{Y} & \text{N} & -16 & 0 & 1.0000000 & 0.465035 & 0.534965 & -121 & 1317 & -1438 & 2755 \\
 287 & 7^1 41^1 & \text{Y} & \text{N} & 5 & 0 & 1.0000000 & 0.466899 & 0.533101 & -116 & 1322 & -1438 & 2760 \\
 288 & 2^5 3^2 & \text{N} & \text{N} & -47 & 42 & 1.7659574 & 0.465278 & 0.534722 & -163 & 1322 & -1485 & 2807 \\
 289 & 17^2 & \text{N} & \text{Y} & 2 & 0 & 1.5000000 & 0.467128 & 0.532872 & -161 & 1324 & -1485 & 2809 \\
 290 & 2^1 5^1 29^1 & \text{Y} & \text{N} & -16 & 0 & 1.0000000 & 0.465517 & 0.534483 & -177 & 1324 & -1501 & 2825 \\
 291 & 3^1 97^1 & \text{Y} & \text{N} & 5 & 0 & 1.0000000 & 0.467354 & 0.532646 & -172 & 1329 & -1501 & 2830 \\
 292 & 2^2 73^1 & \text{N} & \text{N} & -7 & 2 & 1.2857143 & 0.465753 & 0.534247 & -179 & 1329 & -1508 & 2837 \\
 293 & 293^1 & \text{Y} & \text{Y} & -2 & 0 & 1.0000000 & 0.464164 & 0.535836 & -181 & 1329 & -1510 & 2839 \\
 294 & 2^1 3^1 7^2 & \text{N} & \text{N} & 30 & 14 & 1.1666667 & 0.465986 & 0.534014 & -151 & 1359 & -1510 & 2869 \\
 295 & 5^1 59^1 & \text{Y} & \text{N} & 5 & 0 & 1.0000000 & 0.467797 & 0.532203 & -146 & 1364 & -1510 & 2874 \\
 296 & 2^3 37^1 & \text{N} & \text{N} & 9 & 4 & 1.5555556 & 0.469595 & 0.530405 & -137 & 1373 & -1510 & 2883 \\
 297 & 3^3 11^1 & \text{N} & \text{N} & 9 & 4 & 1.5555556 & 0.471380 & 0.528620 & -128 & 1382 & -1510 & 2892 \\
 298 & 2^1 149^1 & \text{Y} & \text{N} & 5 & 0 & 1.0000000 & 0.473154 & 0.526846 & -123 & 1387 & -1510 & 2897 \\
 299 & 13^1 23^1 & \text{Y} & \text{N} & 5 & 0 & 1.0000000 & 0.474916 & 0.525084 & -118 & 1392 & -1510 & 2902 \\
 300 & 2^2 3^1 5^2 & \text{N} & \text{N} & -74 & 58 & 1.2162162 & 0.473333 & 0.526667 & -192 & 1392 & -1584 & 2976 \\
 301 & 7^1 43^1 & \text{Y} & \text{N} & 5 & 0 & 1.0000000 & 0.475083 & 0.524917 & -187 & 1397 & -1584 & 2981 \\
 302 & 2^1 151^1 & \text{Y} & \text{N} & 5 & 0 & 1.0000000 & 0.476821 & 0.523179 & -182 & 1402 & -1584 & 2986 \\
 303 & 3^1 101^1 & \text{Y} & \text{N} & 5 & 0 & 1.0000000 & 0.478548 & 0.521452 & -177 & 1407 & -1584 & 2991 \\
 304 & 2^4 19^1 & \text{N} & \text{N} & -11 & 6 & 1.8181818 & 0.476974 & 0.523026 & -188 & 1407 & -1595 & 3002 \\
 305 & 5^1 61^1 & \text{Y} & \text{N} & 5 & 0 & 1.0000000 & 0.478689 & 0.521311 & -183 & 1412 & -1595 & 3007 \\
 306 & 2^1 3^2 17^1 & \text{N} & \text{N} & 30 & 14 & 1.1666667 & 0.480392 & 0.519608 & -153 & 1442 & -1595 & 3037 \\
 307 & 307^1 & \text{Y} & \text{Y} & -2 & 0 & 1.0000000 & 0.478827 & 0.521173 & -155 & 1442 & -1597 & 3039 \\
 308 & 2^2 7^1 11^1 & \text{N} & \text{N} & 30 & 14 & 1.1666667 & 0.480519 & 0.519481 & -125 & 1472 & -1597 & 3069 \\
 309 & 3^1 103^1 & \text{Y} & \text{N} & 5 & 0 & 1.0000000 & 0.482201 & 0.517799 & -120 & 1477 & -1597 & 3074 \\
 310 & 2^1 5^1 31^1 & \text{Y} & \text{N} & -16 & 0 & 1.0000000 & 0.480645 & 0.519355 & -136 & 1477 & -1613 & 3090 \\
 311 & 311^1 & \text{Y} & \text{Y} & -2 & 0 & 1.0000000 & 0.479100 & 0.520900 & -138 & 1477 & -1615 & 3092 \\
 312 & 2^3 3^1 13^1 & \text{N} & \text{N} & -48 & 32 & 1.3333333 & 0.477564 & 0.522436 & -186 & 1477 & -1663 & 3140 \\
 313 & 313^1 & \text{Y} & \text{Y} & -2 & 0 & 1.0000000 & 0.476038 & 0.523962 & -188 & 1477 & -1665 & 3142 \\
 314 & 2^1 157^1 & \text{Y} & \text{N} & 5 & 0 & 1.0000000 & 0.477707 & 0.522293 & -183 & 1482 & -1665 & 3147 \\
 315 & 3^2 5^1 7^1 & \text{N} & \text{N} & 30 & 14 & 1.1666667 & 0.479365 & 0.520635 & -153 & 1512 & -1665 & 3177 \\
 316 & 2^2 79^1 & \text{N} & \text{N} & -7 & 2 & 1.2857143 & 0.477848 & 0.522152 & -160 & 1512 & -1672 & 3184 \\
 317 & 317^1 & \text{Y} & \text{Y} & -2 & 0 & 1.0000000 & 0.476341 & 0.523659 & -162 & 1512 & -1674 & 3186 \\
 318 & 2^1 3^1 53^1 & \text{Y} & \text{N} & -16 & 0 & 1.0000000 & 0.474843 & 0.525157 & -178 & 1512 & -1690 & 3202 \\
 319 & 11^1 29^1 & \text{Y} & \text{N} & 5 & 0 & 1.0000000 & 0.476489 & 0.523511 & -173 & 1517 & -1690 & 3207 \\
 320 & 2^6 5^1 & \text{N} & \text{N} & -15 & 10 & 2.3333333 & 0.475000 & 0.525000 & -188 & 1517 & -1705 & 3222 \\
 321 & 3^1 107^1 & \text{Y} & \text{N} & 5 & 0 & 1.0000000 & 0.476636 & 0.523364 & -183 & 1522 & -1705 & 3227 \\
 322 & 2^1 7^1 23^1 & \text{Y} & \text{N} & -16 & 0 & 1.0000000 & 0.475155 & 0.524845 & -199 & 1522 & -1721 & 3243 \\
 323 & 17^1 19^1 & \text{Y} & \text{N} & 5 & 0 & 1.0000000 & 0.476780 & 0.523220 & -194 & 1527 & -1721 & 3248 \\
 324 & 2^2 3^4 & \text{N} & \text{N} & 34 & 29 & 1.6176471 & 0.478395 & 0.521605 & -160 & 1561 & -1721 & 3282 \\
 325 & 5^2 13^1 & \text{N} & \text{N} & -7 & 2 & 1.2857143 & 0.476923 & 0.523077 & -167 & 1561 & -1728 & 3289 \\
 326 & 2^1 163^1 & \text{Y} & \text{N} & 5 & 0 & 1.0000000 & 0.478528 & 0.521472 & -162 & 1566 & -1728 & 3294 \\
 327 & 3^1 109^1 & \text{Y} & \text{N} & 5 & 0 & 1.0000000 & 0.480122 & 0.519878 & -157 & 1571 & -1728 & 3299 \\
 328 & 2^3 41^1 & \text{N} & \text{N} & 9 & 4 & 1.5555556 & 0.481707 & 0.518293 & -148 & 1580 & -1728 & 3308 \\
 329 & 7^1 47^1 & \text{Y} & \text{N} & 5 & 0 & 1.0000000 & 0.483283 & 0.516717 & -143 & 1585 & -1728 & 3313 \\
 330 & 2^1 3^1 5^1 11^1 & \text{Y} & \text{N} & 65 & 0 & 1.0000000 & 0.484848 & 0.515152 & -78 & 1650 & -1728 & 3378 \\
 331 & 331^1 & \text{Y} & \text{Y} & -2 & 0 & 1.0000000 & 0.483384 & 0.516616 & -80 & 1650 & -1730 & 3380 \\
 332 & 2^2 83^1 & \text{N} & \text{N} & -7 & 2 & 1.2857143 & 0.481928 & 0.518072 & -87 & 1650 & -1737 & 3387 \\
 333 & 3^2 37^1 & \text{N} & \text{N} & -7 & 2 & 1.2857143 & 0.480480 & 0.519520 & -94 & 1650 & -1744 & 3394 \\
 334 & 2^1 167^1 & \text{Y} & \text{N} & 5 & 0 & 1.0000000 & 0.482036 & 0.517964 & -89 & 1655 & -1744 & 3399 \\
 335 & 5^1 67^1 & \text{Y} & \text{N} & 5 & 0 & 1.0000000 & 0.483582 & 0.516418 & -84 & 1660 & -1744 & 3404 \\
 336 & 2^4 3^1 7^1 & \text{N} & \text{N} & 70 & 54 & 1.5000000 & 0.485119 & 0.514881 & -14 & 1730 & -1744 & 3474 \\
 337 & 337^1 & \text{Y} & \text{Y} & -2 & 0 & 1.0000000 & 0.483680 & 0.516320 & -16 & 1730 & -1746 & 3476 \\
 338 & 2^1 13^2 & \text{N} & \text{N} & -7 & 2 & 1.2857143 & 0.482249 & 0.517751 & -23 & 1730 & -1753 & 3483 \\
 339 & 3^1 113^1 & \text{Y} & \text{N} & 5 & 0 & 1.0000000 & 0.483776 & 0.516224 & -18 & 1735 & -1753 & 3488 \\
 340 & 2^2 5^1 17^1 & \text{N} & \text{N} & 30 & 14 & 1.1666667 & 0.485294 & 0.514706 & 12 & 1765 & -1753 & 3518 \\
 341 & 11^1 31^1 & \text{Y} & \text{N} & 5 & 0 & 1.0000000 & 0.486804 & 0.513196 & 17 & 1770 & -1753 & 3523 \\
 342 & 2^1 3^2 19^1 & \text{N} & \text{N} & 30 & 14 & 1.1666667 & 0.488304 & 0.511696 & 47 & 1800 & -1753 & 3553 \\
 343 & 7^3 & \text{N} & \text{Y} & -2 & 0 & 2.0000000 & 0.486880 & 0.513120 & 45 & 1800 & -1755 & 3555 \\
 344 & 2^3 43^1 & \text{N} & \text{N} & 9 & 4 & 1.5555556 & 0.488372 & 0.511628 & 54 & 1809 & -1755 & 3564 \\
 345 & 3^1 5^1 23^1 & \text{Y} & \text{N} & -16 & 0 & 1.0000000 & 0.486957 & 0.513043 & 38 & 1809 & -1771 & 3580 \\
 346 & 2^1 173^1 & \text{Y} & \text{N} & 5 & 0 & 1.0000000 & 0.488439 & 0.511561 & 43 & 1814 & -1771 & 3585 \\
 347 & 347^1 & \text{Y} & \text{Y} & -2 & 0 & 1.0000000 & 0.487032 & 0.512968 & 41 & 1814 & -1773 & 3587 \\
 348 & 2^2 3^1 29^1 & \text{N} & \text{N} & 30 & 14 & 1.1666667 & 0.488506 & 0.511494 & 71 & 1844 & -1773 & 3617 \\
 349 & 349^1 & \text{Y} & \text{Y} & -2 & 0 & 1.0000000 & 0.487106 & 0.512894 & 69 & 1844 & -1775 & 3619 \\
 350 & 2^1 5^2 7^1 & \text{N} & \text{N} & 30 & 14 & 1.1666667 & 0.488571 & 0.511429 & 99 & 1874 & -1775 & 3649 \\
\end{array}
}
\end{equation*}
\clearpage 

\end{table} 

\newpage
\begin{table}[ht]

\centering
\tiny
\begin{equation*}
\boxed{
\begin{array}{cc|cc|ccc|cc|cccc}
 n & n & \mathbf{Sqfree} & \mathbf{PPower} & g(n) & 
 \lambda(n) g(n) - \widehat{f}_1(n) & 
 \frac{\sum_{d|n} C_{\Omega}(d)}{|g(n)|} & 
 \mathcal{L}_{+}(n) & \mathcal{L}_{-}(n) & 
 G(n) & G_{+}(n) & G_{-}(n) & |G|(n) \\[0.15cm] \hline 
 351 & 3^3 13^1 & \text{N} & \text{N} & 9 & 4 & 1.5555556 & 0.490028 & 0.509972 & 108 & 1883 & -1775 & 3658 \\
 352 & 2^5 11^1 & \text{N} & \text{N} & 13 & 8 & 2.0769231 & 0.491477 & 0.508523 & 121 & 1896 & -1775 & 3671 \\
 353 & 353^1 & \text{Y} & \text{Y} & -2 & 0 & 1.0000000 & 0.490085 & 0.509915 & 119 & 1896 & -1777 & 3673 \\
 354 & 2^1 3^1 59^1 & \text{Y} & \text{N} & -16 & 0 & 1.0000000 & 0.488701 & 0.511299 & 103 & 1896 & -1793 & 3689 \\
 355 & 5^1 71^1 & \text{Y} & \text{N} & 5 & 0 & 1.0000000 & 0.490141 & 0.509859 & 108 & 1901 & -1793 & 3694 \\
 356 & 2^2 89^1 & \text{N} & \text{N} & -7 & 2 & 1.2857143 & 0.488764 & 0.511236 & 101 & 1901 & -1800 & 3701 \\
 357 & 3^1 7^1 17^1 & \text{Y} & \text{N} & -16 & 0 & 1.0000000 & 0.487395 & 0.512605 & 85 & 1901 & -1816 & 3717 \\
 358 & 2^1 179^1 & \text{Y} & \text{N} & 5 & 0 & 1.0000000 & 0.488827 & 0.511173 & 90 & 1906 & -1816 & 3722 \\
 359 & 359^1 & \text{Y} & \text{Y} & -2 & 0 & 1.0000000 & 0.487465 & 0.512535 & 88 & 1906 & -1818 & 3724 \\
 360 & 2^3 3^2 5^1 & \text{N} & \text{N} & 145 & 129 & 1.3034483 & 0.488889 & 0.511111 & 233 & 2051 & -1818 & 3869 \\
 361 & 19^2 & \text{N} & \text{Y} & 2 & 0 & 1.5000000 & 0.490305 & 0.509695 & 235 & 2053 & -1818 & 3871 \\
 362 & 2^1 181^1 & \text{Y} & \text{N} & 5 & 0 & 1.0000000 & 0.491713 & 0.508287 & 240 & 2058 & -1818 & 3876 \\
 363 & 3^1 11^2 & \text{N} & \text{N} & -7 & 2 & 1.2857143 & 0.490358 & 0.509642 & 233 & 2058 & -1825 & 3883 \\
 364 & 2^2 7^1 13^1 & \text{N} & \text{N} & 30 & 14 & 1.1666667 & 0.491758 & 0.508242 & 263 & 2088 & -1825 & 3913 \\
 365 & 5^1 73^1 & \text{Y} & \text{N} & 5 & 0 & 1.0000000 & 0.493151 & 0.506849 & 268 & 2093 & -1825 & 3918 \\
 366 & 2^1 3^1 61^1 & \text{Y} & \text{N} & -16 & 0 & 1.0000000 & 0.491803 & 0.508197 & 252 & 2093 & -1841 & 3934 \\
 367 & 367^1 & \text{Y} & \text{Y} & -2 & 0 & 1.0000000 & 0.490463 & 0.509537 & 250 & 2093 & -1843 & 3936 \\
 368 & 2^4 23^1 & \text{N} & \text{N} & -11 & 6 & 1.8181818 & 0.489130 & 0.510870 & 239 & 2093 & -1854 & 3947 \\
 369 & 3^2 41^1 & \text{N} & \text{N} & -7 & 2 & 1.2857143 & 0.487805 & 0.512195 & 232 & 2093 & -1861 & 3954 \\
 370 & 2^1 5^1 37^1 & \text{Y} & \text{N} & -16 & 0 & 1.0000000 & 0.486486 & 0.513514 & 216 & 2093 & -1877 & 3970 \\
 371 & 7^1 53^1 & \text{Y} & \text{N} & 5 & 0 & 1.0000000 & 0.487871 & 0.512129 & 221 & 2098 & -1877 & 3975 \\
 372 & 2^2 3^1 31^1 & \text{N} & \text{N} & 30 & 14 & 1.1666667 & 0.489247 & 0.510753 & 251 & 2128 & -1877 & 4005 \\
 373 & 373^1 & \text{Y} & \text{Y} & -2 & 0 & 1.0000000 & 0.487936 & 0.512064 & 249 & 2128 & -1879 & 4007 \\
 374 & 2^1 11^1 17^1 & \text{Y} & \text{N} & -16 & 0 & 1.0000000 & 0.486631 & 0.513369 & 233 & 2128 & -1895 & 4023 \\
 375 & 3^1 5^3 & \text{N} & \text{N} & 9 & 4 & 1.5555556 & 0.488000 & 0.512000 & 242 & 2137 & -1895 & 4032 \\
 376 & 2^3 47^1 & \text{N} & \text{N} & 9 & 4 & 1.5555556 & 0.489362 & 0.510638 & 251 & 2146 & -1895 & 4041 \\
 377 & 13^1 29^1 & \text{Y} & \text{N} & 5 & 0 & 1.0000000 & 0.490716 & 0.509284 & 256 & 2151 & -1895 & 4046 \\
 378 & 2^1 3^3 7^1 & \text{N} & \text{N} & -48 & 32 & 1.3333333 & 0.489418 & 0.510582 & 208 & 2151 & -1943 & 4094 \\
 379 & 379^1 & \text{Y} & \text{Y} & -2 & 0 & 1.0000000 & 0.488127 & 0.511873 & 206 & 2151 & -1945 & 4096 \\
 380 & 2^2 5^1 19^1 & \text{N} & \text{N} & 30 & 14 & 1.1666667 & 0.489474 & 0.510526 & 236 & 2181 & -1945 & 4126 \\
 381 & 3^1 127^1 & \text{Y} & \text{N} & 5 & 0 & 1.0000000 & 0.490814 & 0.509186 & 241 & 2186 & -1945 & 4131 \\
 382 & 2^1 191^1 & \text{Y} & \text{N} & 5 & 0 & 1.0000000 & 0.492147 & 0.507853 & 246 & 2191 & -1945 & 4136 \\
 383 & 383^1 & \text{Y} & \text{Y} & -2 & 0 & 1.0000000 & 0.490862 & 0.509138 & 244 & 2191 & -1947 & 4138 \\
 384 & 2^7 3^1 & \text{N} & \text{N} & 17 & 12 & 2.5882353 & 0.492188 & 0.507812 & 261 & 2208 & -1947 & 4155 \\
 385 & 5^1 7^1 11^1 & \text{Y} & \text{N} & -16 & 0 & 1.0000000 & 0.490909 & 0.509091 & 245 & 2208 & -1963 & 4171 \\
 386 & 2^1 193^1 & \text{Y} & \text{N} & 5 & 0 & 1.0000000 & 0.492228 & 0.507772 & 250 & 2213 & -1963 & 4176 \\
 387 & 3^2 43^1 & \text{N} & \text{N} & -7 & 2 & 1.2857143 & 0.490956 & 0.509044 & 243 & 2213 & -1970 & 4183 \\
 388 & 2^2 97^1 & \text{N} & \text{N} & -7 & 2 & 1.2857143 & 0.489691 & 0.510309 & 236 & 2213 & -1977 & 4190 \\
 389 & 389^1 & \text{Y} & \text{Y} & -2 & 0 & 1.0000000 & 0.488432 & 0.511568 & 234 & 2213 & -1979 & 4192 \\
 390 & 2^1 3^1 5^1 13^1 & \text{Y} & \text{N} & 65 & 0 & 1.0000000 & 0.489744 & 0.510256 & 299 & 2278 & -1979 & 4257 \\
 391 & 17^1 23^1 & \text{Y} & \text{N} & 5 & 0 & 1.0000000 & 0.491049 & 0.508951 & 304 & 2283 & -1979 & 4262 \\
 392 & 2^3 7^2 & \text{N} & \text{N} & -23 & 18 & 1.4782609 & 0.489796 & 0.510204 & 281 & 2283 & -2002 & 4285 \\
 393 & 3^1 131^1 & \text{Y} & \text{N} & 5 & 0 & 1.0000000 & 0.491094 & 0.508906 & 286 & 2288 & -2002 & 4290 \\
 394 & 2^1 197^1 & \text{Y} & \text{N} & 5 & 0 & 1.0000000 & 0.492386 & 0.507614 & 291 & 2293 & -2002 & 4295 \\
 395 & 5^1 79^1 & \text{Y} & \text{N} & 5 & 0 & 1.0000000 & 0.493671 & 0.506329 & 296 & 2298 & -2002 & 4300 \\
 396 & 2^2 3^2 11^1 & \text{N} & \text{N} & -74 & 58 & 1.2162162 & 0.492424 & 0.507576 & 222 & 2298 & -2076 & 4374 \\
 397 & 397^1 & \text{Y} & \text{Y} & -2 & 0 & 1.0000000 & 0.491184 & 0.508816 & 220 & 2298 & -2078 & 4376 \\
 398 & 2^1 199^1 & \text{Y} & \text{N} & 5 & 0 & 1.0000000 & 0.492462 & 0.507538 & 225 & 2303 & -2078 & 4381 \\
 399 & 3^1 7^1 19^1 & \text{Y} & \text{N} & -16 & 0 & 1.0000000 & 0.491228 & 0.508772 & 209 & 2303 & -2094 & 4397 \\
 400 & 2^4 5^2 & \text{N} & \text{N} & 34 & 29 & 1.6176471 & 0.492500 & 0.507500 & 243 & 2337 & -2094 & 4431 \\
 401 & 401^1 & \text{Y} & \text{Y} & -2 & 0 & 1.0000000 & 0.491272 & 0.508728 & 241 & 2337 & -2096 & 4433 \\
 402 & 2^1 3^1 67^1 & \text{Y} & \text{N} & -16 & 0 & 1.0000000 & 0.490050 & 0.509950 & 225 & 2337 & -2112 & 4449 \\
 403 & 13^1 31^1 & \text{Y} & \text{N} & 5 & 0 & 1.0000000 & 0.491315 & 0.508685 & 230 & 2342 & -2112 & 4454 \\
 404 & 2^2 101^1 & \text{N} & \text{N} & -7 & 2 & 1.2857143 & 0.490099 & 0.509901 & 223 & 2342 & -2119 & 4461 \\
 405 & 3^4 5^1 & \text{N} & \text{N} & -11 & 6 & 1.8181818 & 0.488889 & 0.511111 & 212 & 2342 & -2130 & 4472 \\
 406 & 2^1 7^1 29^1 & \text{Y} & \text{N} & -16 & 0 & 1.0000000 & 0.487685 & 0.512315 & 196 & 2342 & -2146 & 4488 \\
 407 & 11^1 37^1 & \text{Y} & \text{N} & 5 & 0 & 1.0000000 & 0.488943 & 0.511057 & 201 & 2347 & -2146 & 4493 \\
 408 & 2^3 3^1 17^1 & \text{N} & \text{N} & -48 & 32 & 1.3333333 & 0.487745 & 0.512255 & 153 & 2347 & -2194 & 4541 \\
 409 & 409^1 & \text{Y} & \text{Y} & -2 & 0 & 1.0000000 & 0.486553 & 0.513447 & 151 & 2347 & -2196 & 4543 \\
 410 & 2^1 5^1 41^1 & \text{Y} & \text{N} & -16 & 0 & 1.0000000 & 0.485366 & 0.514634 & 135 & 2347 & -2212 & 4559 \\
 411 & 3^1 137^1 & \text{Y} & \text{N} & 5 & 0 & 1.0000000 & 0.486618 & 0.513382 & 140 & 2352 & -2212 & 4564 \\
 412 & 2^2 103^1 & \text{N} & \text{N} & -7 & 2 & 1.2857143 & 0.485437 & 0.514563 & 133 & 2352 & -2219 & 4571 \\
 413 & 7^1 59^1 & \text{Y} & \text{N} & 5 & 0 & 1.0000000 & 0.486683 & 0.513317 & 138 & 2357 & -2219 & 4576 \\
 414 & 2^1 3^2 23^1 & \text{N} & \text{N} & 30 & 14 & 1.1666667 & 0.487923 & 0.512077 & 168 & 2387 & -2219 & 4606 \\
 415 & 5^1 83^1 & \text{Y} & \text{N} & 5 & 0 & 1.0000000 & 0.489157 & 0.510843 & 173 & 2392 & -2219 & 4611 \\
 416 & 2^5 13^1 & \text{N} & \text{N} & 13 & 8 & 2.0769231 & 0.490385 & 0.509615 & 186 & 2405 & -2219 & 4624 \\
 417 & 3^1 139^1 & \text{Y} & \text{N} & 5 & 0 & 1.0000000 & 0.491607 & 0.508393 & 191 & 2410 & -2219 & 4629 \\
 418 & 2^1 11^1 19^1 & \text{Y} & \text{N} & -16 & 0 & 1.0000000 & 0.490431 & 0.509569 & 175 & 2410 & -2235 & 4645 \\
 419 & 419^1 & \text{Y} & \text{Y} & -2 & 0 & 1.0000000 & 0.489260 & 0.510740 & 173 & 2410 & -2237 & 4647 \\
 420 & 2^2 3^1 5^1 7^1 & \text{N} & \text{N} & -155 & 90 & 1.1032258 & 0.488095 & 0.511905 & 18 & 2410 & -2392 & 4802 \\
 421 & 421^1 & \text{Y} & \text{Y} & -2 & 0 & 1.0000000 & 0.486936 & 0.513064 & 16 & 2410 & -2394 & 4804 \\
 422 & 2^1 211^1 & \text{Y} & \text{N} & 5 & 0 & 1.0000000 & 0.488152 & 0.511848 & 21 & 2415 & -2394 & 4809 \\
 423 & 3^2 47^1 & \text{N} & \text{N} & -7 & 2 & 1.2857143 & 0.486998 & 0.513002 & 14 & 2415 & -2401 & 4816 \\
 424 & 2^3 53^1 & \text{N} & \text{N} & 9 & 4 & 1.5555556 & 0.488208 & 0.511792 & 23 & 2424 & -2401 & 4825 \\
 425 & 5^2 17^1 & \text{N} & \text{N} & -7 & 2 & 1.2857143 & 0.487059 & 0.512941 & 16 & 2424 & -2408 & 4832 \\
\end{array}
}
\end{equation*}
\clearpage 

\end{table} 

\newpage

\begin{table}[ht]
\label{table_conjecture_Mertens_ginvSeq_approx_values_LastPage} 

\centering
\tiny
\begin{equation*}
\boxed{
\begin{array}{cc|cc|ccc|cc|cccc}
 n & n & \mathbf{Sqfree} & \mathbf{PPower} & g(n) & 
 \lambda(n) g(n) - \widehat{f}_1(n) & 
 \frac{\sum_{d|n} C_{\Omega}(d)}{|g(n)|} & 
 \mathcal{L}_{+}(n) & \mathcal{L}_{-}(n) & 
 G(n) & G_{+}(n) & G_{-}(n) & |G|(n) \\[0.15cm] \hline 
 426 & 2^1 3^1 71^1 & \text{Y} & \text{N} & -16 & 0 & 1.0000000 & 0.485915 & 0.514085 & 0 & 2424 & -2424 & 4848 \\
 427 & 7^1 61^1 & \text{Y} & \text{N} & 5 & 0 & 1.0000000 & 0.487119 & 0.512881 & 5 & 2429 & -2424 & 4853 \\
 428 & 2^2 107^1 & \text{N} & \text{N} & -7 & 2 & 1.2857143 & 0.485981 & 0.514019 & -2 & 2429 & -2431 & 4860 \\
 429 & 3^1 11^1 13^1 & \text{Y} & \text{N} & -16 & 0 & 1.0000000 & 0.484848 & 0.515152 & -18 & 2429 & -2447 & 4876 \\
 430 & 2^1 5^1 43^1 & \text{Y} & \text{N} & -16 & 0 & 1.0000000 & 0.483721 & 0.516279 & -34 & 2429 & -2463 & 4892 \\
 431 & 431^1 & \text{Y} & \text{Y} & -2 & 0 & 1.0000000 & 0.482599 & 0.517401 & -36 & 2429 & -2465 & 4894 \\
 432 & 2^4 3^3 & \text{N} & \text{N} & -80 & 75 & 1.5625000 & 0.481481 & 0.518519 & -116 & 2429 & -2545 & 4974 \\
 433 & 433^1 & \text{Y} & \text{Y} & -2 & 0 & 1.0000000 & 0.480370 & 0.519630 & -118 & 2429 & -2547 & 4976 \\
 434 & 2^1 7^1 31^1 & \text{Y} & \text{N} & -16 & 0 & 1.0000000 & 0.479263 & 0.520737 & -134 & 2429 & -2563 & 4992 \\
 435 & 3^1 5^1 29^1 & \text{Y} & \text{N} & -16 & 0 & 1.0000000 & 0.478161 & 0.521839 & -150 & 2429 & -2579 & 5008 \\
 436 & 2^2 109^1 & \text{N} & \text{N} & -7 & 2 & 1.2857143 & 0.477064 & 0.522936 & -157 & 2429 & -2586 & 5015 \\
 437 & 19^1 23^1 & \text{Y} & \text{N} & 5 & 0 & 1.0000000 & 0.478261 & 0.521739 & -152 & 2434 & -2586 & 5020 \\
 438 & 2^1 3^1 73^1 & \text{Y} & \text{N} & -16 & 0 & 1.0000000 & 0.477169 & 0.522831 & -168 & 2434 & -2602 & 5036 \\
 439 & 439^1 & \text{Y} & \text{Y} & -2 & 0 & 1.0000000 & 0.476082 & 0.523918 & -170 & 2434 & -2604 & 5038 \\
 440 & 2^3 5^1 11^1 & \text{N} & \text{N} & -48 & 32 & 1.3333333 & 0.475000 & 0.525000 & -218 & 2434 & -2652 & 5086 \\
 441 & 3^2 7^2 & \text{N} & \text{N} & 14 & 9 & 1.3571429 & 0.476190 & 0.523810 & -204 & 2448 & -2652 & 5100 \\
 442 & 2^1 13^1 17^1 & \text{Y} & \text{N} & -16 & 0 & 1.0000000 & 0.475113 & 0.524887 & -220 & 2448 & -2668 & 5116 \\
 443 & 443^1 & \text{Y} & \text{Y} & -2 & 0 & 1.0000000 & 0.474041 & 0.525959 & -222 & 2448 & -2670 & 5118 \\
 444 & 2^2 3^1 37^1 & \text{N} & \text{N} & 30 & 14 & 1.1666667 & 0.475225 & 0.524775 & -192 & 2478 & -2670 & 5148 \\
 445 & 5^1 89^1 & \text{Y} & \text{N} & 5 & 0 & 1.0000000 & 0.476404 & 0.523596 & -187 & 2483 & -2670 & 5153 \\
 446 & 2^1 223^1 & \text{Y} & \text{N} & 5 & 0 & 1.0000000 & 0.477578 & 0.522422 & -182 & 2488 & -2670 & 5158 \\
 447 & 3^1 149^1 & \text{Y} & \text{N} & 5 & 0 & 1.0000000 & 0.478747 & 0.521253 & -177 & 2493 & -2670 & 5163 \\
 448 & 2^6 7^1 & \text{N} & \text{N} & -15 & 10 & 2.3333333 & 0.477679 & 0.522321 & -192 & 2493 & -2685 & 5178 \\
 449 & 449^1 & \text{Y} & \text{Y} & -2 & 0 & 1.0000000 & 0.476615 & 0.523385 & -194 & 2493 & -2687 & 5180 \\
 450 & 2^1 3^2 5^2 & \text{N} & \text{N} & -74 & 58 & 1.2162162 & 0.475556 & 0.524444 & -268 & 2493 & -2761 & 5254 \\
 451 & 11^1 41^1 & \text{Y} & \text{N} & 5 & 0 & 1.0000000 & 0.476718 & 0.523282 & -263 & 2498 & -2761 & 5259 \\
 452 & 2^2 113^1 & \text{N} & \text{N} & -7 & 2 & 1.2857143 & 0.475664 & 0.524336 & -270 & 2498 & -2768 & 5266 \\
 453 & 3^1 151^1 & \text{Y} & \text{N} & 5 & 0 & 1.0000000 & 0.476821 & 0.523179 & -265 & 2503 & -2768 & 5271 \\
 454 & 2^1 227^1 & \text{Y} & \text{N} & 5 & 0 & 1.0000000 & 0.477974 & 0.522026 & -260 & 2508 & -2768 & 5276 \\
 455 & 5^1 7^1 13^1 & \text{Y} & \text{N} & -16 & 0 & 1.0000000 & 0.476923 & 0.523077 & -276 & 2508 & -2784 & 5292 \\
 456 & 2^3 3^1 19^1 & \text{N} & \text{N} & -48 & 32 & 1.3333333 & 0.475877 & 0.524123 & -324 & 2508 & -2832 & 5340 \\
 457 & 457^1 & \text{Y} & \text{Y} & -2 & 0 & 1.0000000 & 0.474836 & 0.525164 & -326 & 2508 & -2834 & 5342 \\
 458 & 2^1 229^1 & \text{Y} & \text{N} & 5 & 0 & 1.0000000 & 0.475983 & 0.524017 & -321 & 2513 & -2834 & 5347 \\
 459 & 3^3 17^1 & \text{N} & \text{N} & 9 & 4 & 1.5555556 & 0.477124 & 0.522876 & -312 & 2522 & -2834 & 5356 \\
 460 & 2^2 5^1 23^1 & \text{N} & \text{N} & 30 & 14 & 1.1666667 & 0.478261 & 0.521739 & -282 & 2552 & -2834 & 5386 \\
 461 & 461^1 & \text{Y} & \text{Y} & -2 & 0 & 1.0000000 & 0.477223 & 0.522777 & -284 & 2552 & -2836 & 5388 \\
 462 & 2^1 3^1 7^1 11^1 & \text{Y} & \text{N} & 65 & 0 & 1.0000000 & 0.478355 & 0.521645 & -219 & 2617 & -2836 & 5453 \\
 463 & 463^1 & \text{Y} & \text{Y} & -2 & 0 & 1.0000000 & 0.477322 & 0.522678 & -221 & 2617 & -2838 & 5455 \\
 464 & 2^4 29^1 & \text{N} & \text{N} & -11 & 6 & 1.8181818 & 0.476293 & 0.523707 & -232 & 2617 & -2849 & 5466 \\
 465 & 3^1 5^1 31^1 & \text{Y} & \text{N} & -16 & 0 & 1.0000000 & 0.475269 & 0.524731 & -248 & 2617 & -2865 & 5482 \\
 466 & 2^1 233^1 & \text{Y} & \text{N} & 5 & 0 & 1.0000000 & 0.476395 & 0.523605 & -243 & 2622 & -2865 & 5487 \\
 467 & 467^1 & \text{Y} & \text{Y} & -2 & 0 & 1.0000000 & 0.475375 & 0.524625 & -245 & 2622 & -2867 & 5489 \\
 468 & 2^2 3^2 13^1 & \text{N} & \text{N} & -74 & 58 & 1.2162162 & 0.474359 & 0.525641 & -319 & 2622 & -2941 & 5563 \\
 469 & 7^1 67^1 & \text{Y} & \text{N} & 5 & 0 & 1.0000000 & 0.475480 & 0.524520 & -314 & 2627 & -2941 & 5568 \\
 470 & 2^1 5^1 47^1 & \text{Y} & \text{N} & -16 & 0 & 1.0000000 & 0.474468 & 0.525532 & -330 & 2627 & -2957 & 5584 \\
 471 & 3^1 157^1 & \text{Y} & \text{N} & 5 & 0 & 1.0000000 & 0.475584 & 0.524416 & -325 & 2632 & -2957 & 5589 \\
 472 & 2^3 59^1 & \text{N} & \text{N} & 9 & 4 & 1.5555556 & 0.476695 & 0.523305 & -316 & 2641 & -2957 & 5598 \\
 473 & 11^1 43^1 & \text{Y} & \text{N} & 5 & 0 & 1.0000000 & 0.477801 & 0.522199 & -311 & 2646 & -2957 & 5603 \\
 474 & 2^1 3^1 79^1 & \text{Y} & \text{N} & -16 & 0 & 1.0000000 & 0.476793 & 0.523207 & -327 & 2646 & -2973 & 5619 \\
 475 & 5^2 19^1 & \text{N} & \text{N} & -7 & 2 & 1.2857143 & 0.475789 & 0.524211 & -334 & 2646 & -2980 & 5626 \\
 476 & 2^2 7^1 17^1 & \text{N} & \text{N} & 30 & 14 & 1.1666667 & 0.476891 & 0.523109 & -304 & 2676 & -2980 & 5656 \\
 477 & 3^2 53^1 & \text{N} & \text{N} & -7 & 2 & 1.2857143 & 0.475891 & 0.524109 & -311 & 2676 & -2987 & 5663 \\
 478 & 2^1 239^1 & \text{Y} & \text{N} & 5 & 0 & 1.0000000 & 0.476987 & 0.523013 & -306 & 2681 & -2987 & 5668 \\
 479 & 479^1 & \text{Y} & \text{Y} & -2 & 0 & 1.0000000 & 0.475992 & 0.524008 & -308 & 2681 & -2989 & 5670 \\
 480 & 2^5 3^1 5^1 & \text{N} & \text{N} & -96 & 80 & 1.6666667 & 0.475000 & 0.525000 & -404 & 2681 & -3085 & 5766 \\
 481 & 13^1 37^1 & \text{Y} & \text{N} & 5 & 0 & 1.0000000 & 0.476091 & 0.523909 & -399 & 2686 & -3085 & 5771 \\
 482 & 2^1 241^1 & \text{Y} & \text{N} & 5 & 0 & 1.0000000 & 0.477178 & 0.522822 & -394 & 2691 & -3085 & 5776 \\
 483 & 3^1 7^1 23^1 & \text{Y} & \text{N} & -16 & 0 & 1.0000000 & 0.476190 & 0.523810 & -410 & 2691 & -3101 & 5792 \\
 484 & 2^2 11^2 & \text{N} & \text{N} & 14 & 9 & 1.3571429 & 0.477273 & 0.522727 & -396 & 2705 & -3101 & 5806 \\
 485 & 5^1 97^1 & \text{Y} & \text{N} & 5 & 0 & 1.0000000 & 0.478351 & 0.521649 & -391 & 2710 & -3101 & 5811 \\
 486 & 2^1 3^5 & \text{N} & \text{N} & 13 & 8 & 2.0769231 & 0.479424 & 0.520576 & -378 & 2723 & -3101 & 5824 \\
 487 & 487^1 & \text{Y} & \text{Y} & -2 & 0 & 1.0000000 & 0.478439 & 0.521561 & -380 & 2723 & -3103 & 5826 \\
 488 & 2^3 61^1 & \text{N} & \text{N} & 9 & 4 & 1.5555556 & 0.479508 & 0.520492 & -371 & 2732 & -3103 & 5835 \\
 489 & 3^1 163^1 & \text{Y} & \text{N} & 5 & 0 & 1.0000000 & 0.480573 & 0.519427 & -366 & 2737 & -3103 & 5840 \\
 490 & 2^1 5^1 7^2 & \text{N} & \text{N} & 30 & 14 & 1.1666667 & 0.481633 & 0.518367 & -336 & 2767 & -3103 & 5870 \\
 491 & 491^1 & \text{Y} & \text{Y} & -2 & 0 & 1.0000000 & 0.480652 & 0.519348 & -338 & 2767 & -3105 & 5872 \\
 492 & 2^2 3^1 41^1 & \text{N} & \text{N} & 30 & 14 & 1.1666667 & 0.481707 & 0.518293 & -308 & 2797 & -3105 & 5902 \\
 493 & 17^1 29^1 & \text{Y} & \text{N} & 5 & 0 & 1.0000000 & 0.482759 & 0.517241 & -303 & 2802 & -3105 & 5907 \\
 494 & 2^1 13^1 19^1 & \text{Y} & \text{N} & -16 & 0 & 1.0000000 & 0.481781 & 0.518219 & -319 & 2802 & -3121 & 5923 \\
 495 & 3^2 5^1 11^1 & \text{N} & \text{N} & 30 & 14 & 1.1666667 & 0.482828 & 0.517172 & -289 & 2832 & -3121 & 5953 \\
 496 & 2^4 31^1 & \text{N} & \text{N} & -11 & 6 & 1.8181818 & 0.481855 & 0.518145 & -300 & 2832 & -3132 & 5964 \\
 497 & 7^1 71^1 & \text{Y} & \text{N} & 5 & 0 & 1.0000000 & 0.482897 & 0.517103 & -295 & 2837 & -3132 & 5969 \\
 498 & 2^1 3^1 83^1 & \text{Y} & \text{N} & -16 & 0 & 1.0000000 & 0.481928 & 0.518072 & -311 & 2837 & -3148 & 5985 \\
 499 & 499^1 & \text{Y} & \text{Y} & -2 & 0 & 1.0000000 & 0.480962 & 0.519038 & -313 & 2837 & -3150 & 5987 \\
 500 & 2^2 5^3 & \text{N} & \text{N} & -23 & 18 & 1.4782609 & 0.480000 & 0.520000 & -336 & 2837 & -3173 & 6010 \\
\end{array}
}
\end{equation*}

\end{table} 

\clearpage 

\end{document}
