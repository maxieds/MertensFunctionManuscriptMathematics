\documentclass[11pt,reqno,a4letter]{article} 

\usepackage{amsmath,amssymb,amsfonts,amscd}
\usepackage[hidelinks]{hyperref} 
\usepackage{url}
\usepackage[usenames,dvipsnames]{xcolor}
\hypersetup{
    colorlinks,
    linkcolor={black!63!darkgray},
    citecolor={blue!70!white},
    urlcolor={blue!80!white}
}

%\usepackage[normalem]{ulem}
\usepackage{graphicx} 
\usepackage{datetime} 
\usepackage{cancel}
\usepackage{soul} 
\usepackage{subcaption}
\captionsetup{format=hang,labelfont={bf},textfont={small,it}} 
\numberwithin{figure}{section}
\numberwithin{table}{section}

\usepackage{framed} 
%\usepackage{ulem}
%\usepackage[T1]{fontenc}
%\usepackage{pbsi}

\usepackage{enumitem}
\setlist[itemize]{leftmargin=0.65in}

\usepackage{changepage}
\usepackage{rotating,adjustbox}

\usepackage{diagbox}
\newcommand{\trianglenk}[2]{$\diagbox{#1}{#2}$}
\newcommand{\trianglenkII}[2]{\diagbox{#1}{#2}}

\let\citep\cite

\newcommand{\undersetbrace}[2]{\underset{\displaystyle{#1}}{\underbrace{#2}}}

\newcommand{\gkpSI}[2]{\ensuremath{\genfrac{\lbrack}{\rbrack}{0pt}{}{#1}{#2}}} 
\newcommand{\gkpSII}[2]{\ensuremath{\genfrac{\lbrace}{\rbrace}{0pt}{}{#1}{#2}}}
\newcommand{\cf}{\textit{cf.\ }} 
\newcommand{\Iverson}[1]{\ensuremath{\left[#1\right]_{\delta}}} 
\newcommand{\floor}[1]{\left\lfloor #1 \right\rfloor} 
\newcommand{\ceiling}[1]{\left\lceil #1 \right\rceil} 
\newcommand{\e}[1]{e\left(#1\right)} 
\newcommand{\seqnum}[1]{\href{http://oeis.org/#1}{\color{ProcessBlue}{\underline{#1}}}}

\usepackage{upgreek,dsfont,amssymb}
\renewcommand{\chi}{\upchi}
\newcommand{\ChiFunc}[1]{\ensuremath{\chi_{\{#1\}}}}
\newcommand{\OneFunc}[1]{\ensuremath{\mathds{1}_{#1}}}

%\usepackage{mathabx}
\makeatletter
\newcommand*\rel@kern[1]{\kern#1\dimexpr\macc@kerna}
\newcommand*\widebar[1]{%
  \begingroup
  \def\mathaccent##1##2{%
    \rel@kern{0.8}%
    \overline{\rel@kern{-0.8}\macc@nucleus\rel@kern{0.2}}%
    \rel@kern{-0.2}%
  }%
  \macc@depth\@ne
  \let\math@bgroup\@empty \let\math@egroup\macc@set@skewchar
  \mathsurround\z@ \frozen@everymath{\mathgroup\macc@group\relax}%
  \macc@set@skewchar\relax
  \let\mathaccentV\macc@nested@a
  \macc@nested@a\relax111{#1}%
  \endgroup
}

\usepackage{MnSymbol}
\newcommand{\gkpEII}[2]{\ensuremath{\genfrac{\llangle}{\rrangle}{0pt}{}{#1}{#2}}}

\usepackage{ifthen}
\newcommand{\Hn}[2]{
     \ifthenelse{\equal{#2}{1}}{H_{#1}}{H_{#1}^{\left(#2\right)}}
}

\newcommand{\Floor}[2]{\ensuremath{\left\lfloor \frac{#1}{#2} \right\rfloor}}
\newcommand{\Ceiling}[2]{\ensuremath{\left\lceil \frac{#1}{#2} \right\rceil}}

\DeclareMathOperator{\DGF}{DGF} 
\DeclareMathOperator{\ds}{ds} 
\DeclareMathOperator{\Id}{Id}
\DeclareMathOperator{\fg}{fg}
\DeclareMathOperator{\Div}{div}
\DeclareMathOperator{\rpp}{rpp}
\DeclareMathOperator{\logll}{\ell\ell}

\title{
       \LARGE{
       New characterizations of partial sums of the M\"obius function 
       } 
}
\author{{\Large Maxie Dion Schmidt} \\ 
        %{\normalsize \href{mailto:maxieds@gmail.com}{maxieds@gmail.com}} \\[0.1cm] 
        {\normalsize Georgia Institute of Technology} \\[0.025cm] 
        {\normalsize School of Mathematics} 
} 

%\date{\small\underline{Last Revised:} \today \ @\ \hhmmsstime{} \ -- \ Compiled with \LaTeX2e} 

\usepackage{amsthm} 

\theoremstyle{plain} 
\newtheorem{theorem}{Theorem}
\newtheorem{conjecture}[theorem]{Conjecture}
\newtheorem{claim}[theorem]{Claim}
\newtheorem{prop}[theorem]{Proposition}
\newtheorem{lemma}[theorem]{Lemma}
\newtheorem{cor}[theorem]{Corollary}
\numberwithin{theorem}{section}

\theoremstyle{definition} 
\newtheorem{example}[theorem]{Example}
\newtheorem{remark}[theorem]{Remark}
\newtheorem{definition}[theorem]{Definition}
\newtheorem{notation}[theorem]{Notation}
\newtheorem{question}[theorem]{Question}
\newtheorem{discussion}[theorem]{Discussion}
\newtheorem{facts}[theorem]{Facts}
\newtheorem{summary}[theorem]{Summary}
\newtheorem{heuristic}[theorem]{Heuristic}
\newtheorem{observation}[theorem]{Observation}
\newtheorem{ansatz}[theorem]{Ansatz}

\renewcommand{\arraystretch}{1.25} 
\usepackage[total={7in, 9.5in},tmargin=0.75in,headsep=8pt,footskip=30pt,footnotesep=0.5in]{geometry}

\usepackage{fancyhdr}
\pagestyle{empty}
\pagestyle{fancy}
\fancyhead[RO,RE]{Maxie Dion Schmidt -- \today} 
\fancyhead[LO,LE]{}
\fancyheadoffset{0.005\textwidth} 

\setlength{\parindent}{0em}
\setlength{\parskip}{2.5em} 

\renewcommand{\thefootnote}{\textbf{\arabic{footnote}}}

\renewcommand{\Re}{\operatorname{Re}}
\renewcommand{\Im}{\operatorname{Im}}

\usepackage{tikz}
\usetikzlibrary{shapes,arrows}

\usepackage{longtable}
\usepackage{arydshln} 
\usepackage[symbols,nogroupskip,nomain,automake=true,nonumberlist,section=section,toc]{glossaries-extra}
\usepackage{glossary-mcols}

\defglsdisplayfirst[main]{#1#4\protect\footnote{#2}}

%%%%%%%%%%%%

\providecommand{\glossarytoctitle}{\glossaryname}
\setlength{\glsdescwidth}{0.7\textwidth}

\newglossarystyle{glossstyleSymbol}{%
\renewenvironment{theglossary}%
 {\begin{longtable}{lp{\glsdescwidth}}}%
 {\end{longtable}}%
 \setlength{\parskip}{3.5pt}
 \renewcommand{\glsgroupskip}{}
\renewcommand*\glspostdescription{\dotfill}
\renewcommand*{\glossaryheader}{%
 \bfseries Symbols & \bfseries Definition
 \\\endhead}%
 \renewcommand*{\glsgroupheading}[1]{}%
  \renewcommand{\glossentry}[2]{%
    \glstarget{##1}{\glossentrysymbol{##1}} &
    \glossentrydesc{##1} \tabularnewline
  }%
  \renewcommand*{\glspostdescription}{}
  \renewcommand{\glossarymark}[1]{}
}

\setglossarystyle{glossstyleSymbol}
\makeglossaries

%%%%%%%%%%%%

\newglossaryentry{fCvlg}{
    symbol={\ensuremath{f \ast g}},
    sort={fg},
    description={The Dirichlet convolution of any two arithmetic functions 
    $f$ and $g$ at $n$ is defined to be 
    the divisor sum $(f \ast g)(n) := \sum\limits_{d|n} f(d) g\left(\frac{n}{d}\right)$ 
    for $n \geq 1$. 
    },
    type={symbols},
    name={Dirichlet convolution}
    }
\newglossaryentry{coeffExtraction}{
    symbol={\ensuremath{[q^n] F(q)}},
    sort={coeffExtraction},
    description={The coefficient of $q^n$ in the power series expansion of $F(q)$ about zero when 
    $F(q)$ is treated as the ordinary generating function (OGF) of a sequence, $\{f_n\}_{n \geq 0}$. 
    Namely, for integers $n \geq 0$ we define $[q^n] F(q) = f_n$ whenever 
    $F(q) := \sum\limits_{n \geq 0} f_n q^n$. },
    type={symbols},
    name={Series coefficient extraction}
    }
\newglossaryentry{MoebiusMuFunc}{
    symbol={\ensuremath{\mu(n),M(x)}},
    sort={MoebiusMuFunc},
    description={The M\"obius function defined such that $\mu^2(n)$ is the indicator function of the 
                 squarefree integers $n \geq 1$ where 
                 $\mu(n) = (-1)^{\omega(n)}$ whenever $n$ is squarefree. 
                 The Mertens function is the summatory function defined for all integers 
                 $x \geq 1$ by the partial sums $M(x) := \sum\limits_{n \leq x} \mu(n)$.
                 },
    type={symbols},
    name={M\"obius function}
    }
\newglossaryentry{Iverson}{
    symbol={\ensuremath{\Iverson{n=k}},\ensuremath{\Iverson{\mathtt{cond}}}},
    sort={Iverson},
    description={The symbol $\Iverson{n=k}$ is a synonym for $\delta_{n,k}$ 
                 which is one if and only if $n = k$, and is zero otherwise. 
                 For Boolean-valued conditions, \texttt{cond}, the symbol $\Iverson{\mathtt{cond}}$ 
                 evaluates to one precisely when \texttt{cond} is true or to zero otherwise.},
    type={symbols},
    name={Iverson's convention}
    }
\newglossaryentry{epsilonN}{
    symbol={\ensuremath{\varepsilon(n)}},
    sort={epsilonN},
    description={The multiplicative identity with respect to Dirichlet convolution, $\varepsilon(n) := \delta_{n,1}$, 
                 defined such that for any arithmetic function $f$ we have that 
                 $f \ast \varepsilon = \varepsilon \ast f = f$ where the operation 
                 $\ast$ denotes Dirichlet convolution. },
    type={symbols},
    name={Dirichlet multiplicative identity}
    }
\newglossaryentry{Zetas}{
    symbol={\ensuremath{\zeta(s)}},
    sort={Zetas},
    description={The Riemann zeta function is defined by 
                 $\zeta(s) := \sum\limits_{n \geq 1} \frac{1}{n^{s}}$ when $\Re(s) > 1$, 
                 and by analytic continuation to any $s \in \mathbb{C}$ with the exception of a 
                 simple pole at $s = 1$ of residue one.},
    type={symbols},
    name={Riemann zeta function}
    }
\newglossaryentry{fInvn}{
     symbol={\ensuremath{f^{-1}(n)}},
    sort={fInvn},
    description={
     The Dirichlet inverse $f^{-1}$ of an arithmetic function $f$ exists 
     if and only if $f(1) \neq 0$. 
     The Dirichlet inverse of any $f$ such that $f(1) \neq 0$ 
     is defined recursively by 
     $f^{-1}(n) = -\frac{1}{f(1)} \times \sum\limits_{\substack{d|n \\ d>1}} f(d) f^{-1}\left(\frac{n}{d}\right)$ 
     for $n \geq 2$ with $f^{-1}(1) = f(1)^{-1}$. 
     When it exists, this inverse function 
     is unique and satisfies  $f^{-1} \ast f = f \ast f^{-1} = \varepsilon$.},
    type={symbols},
    name={Dirichlet inverse of $f$}
    }
\newglossaryentry{CkngInvAuxFunc}{
    symbol={$C_k(n),C_{\Omega}(n)$},
    sort={CkngInvAuxFunc},
    description={The first sequence is defined recursively for integers $n \geq 1$ and $k \geq 0$ as follows: 
                 \[
                 C_k(n) := \begin{cases} 
                      \delta_{n,1}, & \text{ if $k = 0$; } \\ 
                      \sum\limits_{d|n} \omega(d) C_{k-1}\left(\frac{n}{d}\right), & \text{ if $k \geq 1$. } 
                      \end{cases} 
                 \]
                 It represents the multiple ($k$-fold) convolution of the function $\omega(n)$ 
                 with itself. 
                 The function $C_{\Omega}(n) := C_{\Omega(n)}(n)$ has the DGF 
                 $(1-P(s))^{-1}$ for $\Re(s) > 1$. 
                 },
    type={symbols},
    name={Dirichlet inverse component functions}
    }
\newglossaryentry{gInvn}{
    symbol={$g^{-1}(n),G^{-1}(x),|G^{-1}|(x)$},
    sort={gInvn},
    description={The Dirichlet inverse function, $g^{-1}(n) = (\omega+1)^{-1}(n)$, has the 
                 summatory function $G^{-1}(x) := \sum\limits_{n \leq x} g^{-1}(n)$ for $x \geq 1$. 
                 We define the partial sums of the unsigned inverse function to be 
                 $|G^{-1}|(x) := \sum_{n \leq x} |g^{-1}(n)|$ for $x \geq 1$. },
    type={symbols},
    name={Key Dirichlet inverse functions}
    }
\newglossaryentry{PikPiHatkx}{
    symbol={$\pi_k(x),\widehat{\pi}_k(x)$},
    sort={PikPiHatkx},
    description={For integers $k \geq 1$, the 
                 function $\pi_k(x)$ denotes the number of 
                 $2 \leq n \leq x$ with 
                 exactly $k$ distinct prime factors: $\pi_k(x) := \#\{2 \leq n \leq x: \omega(n) = k\}$. 
                 Similarly, the function 
                 $\widehat{\pi}_k(x) := \#\{2 \leq n \leq x: \Omega(n) = k\}$ for $x \geq 2$ and fixed $k \geq 1$. },
    type={symbols},
    name={Distinct prime counting functions}
    }   
\newglossaryentry{Nupn}{
    symbol={$\nu_p(n)$}, 
    sort={Nupn},
    description={The valuation function that extracts the maximal exponent of $p$ in the prime factorization of $n$, e.g., 
                 $\nu_p(n) = 0$ if $p \nmid n$ and $\nu_p(n) = \alpha$ if $p^{\alpha} || n$ 
                 for $p \geq 2$ prime, $\alpha \geq 1$ and $n \geq 2$.},
    type={symbols},
    name={Exponent extraction function}
    }
\newglossaryentry{primeOmegaFunctions}{
    symbol={$\omega(n)$,$\Omega(n)$}, 
    sort={OmegaPrimeOmegaFunctions},
    description={We define the strongly additive function 
                 $\omega(n) := \sum\limits_{p|n} 1$ and the completely additive function 
                 $\Omega(n) := \sum\limits_{p^{\alpha} || n} \alpha$. This means that if the prime 
                 factorization of any $n \geq 2$ is 
                 given by $n := p_1^{\alpha_1} \times \cdots \times p_r^{\alpha_r}$ 
                 with $p_i \neq p_j$ for all $i \neq j$, 
                 then $\omega(n) = r$ and $\Omega(n) = \alpha_1 + \cdots + \alpha_r$. 
                 We set $\omega(1) = \Omega(1) = 0$ by convention.},
    type={symbols},
    name={Prime omega functions}
    }
\newglossaryentry{LiouvilleLambdaFunc}{
     symbol={$\lambda(n), L(x)$}, 
    sort={LiouvilleLambdaFunc},
    description={The Liouville lambda function is the completely multiplicative function defined by 
                 $\lambda(n) := (-1)^{\Omega(n)}$. 
                 Its summatory function is defined by the partial sums 
                 $L(x) := \sum\limits_{n \leq x} \lambda(n)$ for $x \geq 1$. 
                 },
    type={symbols},
    name={Liouville lambda function}
    }
\newglossaryentry{QxSummatoryFunc}{
    symbol={$Q(x)$},
    sort={QxSummatoryFunc},
    description={For $x \geq 1$, we define $Q(x)$ to be the summatory function indicating the number of 
                 squarefree integers $n \leq x$. That is, $Q(x) := \sum\limits_{n \leq x} \mu^2(n)$ where 
                 $Q(x) = \frac{6x}{\pi^2} + O(\sqrt{x})$.}, 
    type={symbols},
    name={Summatory function of the squarefree integers}
    }
\newglossaryentry{ApproxAndSimRelations}{
    symbol={$\approx,\sim$},
    sort={ApproxAndSimRelations},
    description={We write that $f(x) \approx g(x)$ if $|f(x) - g(x)| \ll 1$ 
                 as $x \rightarrow \infty$. 
                 Two arithmetic functions $A(x), B(x)$ satisfy the relation $A \sim B$ if 
                 $\lim_{x \rightarrow \infty} \frac{A(x)}{B(x)} = 1.$ },
    type={symbols},
    name={Asymptotic relation symbol}
    }
\newglossaryentry{AApproxSimGGLLRelations}{
    symbol={$\gg,\ll,\asymp$},
    sort={AApproxSimGGLLRelations},
    description={
                 For functions $A,B$, the notation $A \ll B$ implies that $A = O(B)$. 
                 Similarly, for $B \geq 0$ the notation $A \gg B$ implies that $B = O(A)$. 
                 When we have that $A, B \geq 0$, $A \ll B$ and $B \ll A$, we write $A \asymp B$. },
    type={symbols},
    name={Asymptotic relation symbols}
    }
\newglossaryentry{NormalCDFFunc}{
    symbol={$\Phi(z),\mathcal{N}(0,1)$},
    sort={NormalCDFFunc},
    description={For $z \in \mathbb{R}$, we take the cumulative density function (CDF) 
                 of the standard normal distribution to be denoted by 
                 $\Phi(z) := \frac{1}{\sqrt{2\pi}} \times \int\limits_{-\infty}^{z} e^{-\frac{t^2}{2}} dt$. 
                 A random variable $Z$ whose values are distributed according to the CDF 
                 $\Phi(z) = \mathbb{P}[Z \leq z]$ 
                 has distribution denoted by $Z \sim \mathcal{N}(0, 1)$. },
    type={symbols},
    name={Asymptotic relation symbol}
    }

\newglossaryentry{chiPrimeP}{
    symbol={$\chi_{\mathbb{P}}(n), P(s)$},
    sort={chiPrimeP},
    description={The indicator function of the primes equals one if and only if 
                 $n \in \mathbb{Z}^{+}$ is prime and is defined to be 
                 zero-valued otherwise. 
                 For any $s \in \mathbb{C}$ such that $\Re(s) > 1$, 
                 we define the prime zeta function to be the 
                 Dirichlet generating function (DGF) defined by 
                 $P(s) = \sum\limits_{n \geq 1} \frac{\chi_{\mathbb{P}}(n)}{n^s}$. 
                 The function $P(s)$ has an analytic continuation to the half-plane 
                 $\Re(s) > 0$ with the exception of $s = 1$ through the formula 
                 $P(s) = \sum\limits_{k \geq 1} \frac{\mu(k)}{k} \log\zeta(ks)$. The DGF $P(s)$ 
                 poles at the reciprocal of each positive integer and a natural boundary 
                 at the line $\Re(s) = 0$. },
    type={symbols},
    name={Prime set indicator function}
    }
\newglossaryentry{WLambertWFunction}{
    symbol={$W(x)$},
    sort={WLambertWFunction},
    description={For $x,y \in [0, +\infty)$, we write that $x = W(y)$ if and only if $xe^{x} = y$. 
                 This function denotes the principal branch of the multi-valued Lambert $W$ function 
                 taken over the non-negative reals. },
    type={symbols},
    name={Lambert $W$-Function}
    }
\newglossaryentry{GGTildeFHatGHatBivariateFunctions}{
    symbol={$\mathcal{G}(z),\widetilde{\mathcal{G}}(z)$; $\widehat{F}(s, z),\widehat{\mathcal{G}}(z)$},
    sort={GGTildeFHatGHatBivariateFunctions},
    description={The functions $\mathcal{G}(z)$ and $\widetilde{\mathcal{G}}(z)$ are defined for 
                 $0 \leq |z| \leq R < 2$ on page 
                 \pageref{subSection_TheKnownDistsOfThePrimeOmegaFunctions_IntroResults_v1} of 
                 Section \ref{subSection_TheKnownDistsOfThePrimeOmegaFunctions_IntroResults_v1}. 
                 The related constructions 
                 used to motivate the definitions of 
                 $\widehat{F}(s, z)$ and $\widehat{\mathcal{G}}(z)$ are defined 
                 by the infinite products over the primes given on pages 
                 \pageref{prop_HatAzx_ModSummatoryFuncExps_RelatedToCkn} and 
                 \pageref{theorem_CnkSpCasesScaledSummatoryFuncs} of 
                 Section \ref{subSection_Section4_AnalyticPrerequisiteProofsOfUniformBoundsOnCertainPartialSumTypes_v1}, 
                 respectively. },
    type={symbols},
    name={Bivariate DGFs}
    }
\newglossaryentry{GammaIncompleteGamma}{
    symbol={$\Gamma(a, z)$},
    sort={GammaIncompleteGamma},
    description={The incomplete gamma function is defined as $\Gamma(a, z) := \int_z^{\infty} t^{a-1} e^{-t} dt$ 
                 by continuation for $a \in \mathbb{R}$ and $|\operatorname{arg}(z)| < \pi$. 
                 Asymptotics of this function as both $a, z \rightarrow \infty$ independently 
                 are discussed in the appendix. },
    type={symbols},
    name={Incomplete gamma function}
    }
\newglossaryentry{ErrorFunctionsErfErfiz}{
    symbol={$\operatorname{erf}(z),\operatorname{erfi}(z)$},
    sort={ErrorFunctionsErfErfiz},
    description={The function 
                 $\operatorname{erf}(z)$ denotes the (ordinary) error function. It is related to the CDF, $\Phi(z)$, of the 
                 standard normal distribution for any $z \in (-\infty, +\infty)$ through the 
                 relation $\Phi(z) = \frac{1}{2}\left(1+\operatorname{erf}\left(\frac{z}{\sqrt{2}}\right)\right)$. 
                 The imaginary error function is defined as 
                 $\operatorname{erfi}(z) = \operatorname{erf}(\imath z) := 
                  \frac{1}{\imath\sqrt{\pi}} \times \int_0^{\imath z} e^{t^2} dt$ for $z \in (-\infty, +\infty)$. 
                 %An asymptotic series for these two functions as $|z| \rightarrow +\infty$ appears inline on 
                 %page \pageref{cor_ExpectationFormulaAbsgInvn_v2} of 
                 %Section \ref{subSection_AvgOrdersOfTheUnsignedSequences}. 
                 },
    type={symbols},
    name={Error function variants}
    }

\glsaddall[types={symbols}]

\allowdisplaybreaks 

\begin{document} 

\maketitle

\begin{abstract} 
The Mertens function, $M(x) := \sum_{n \leq x} \mu(n)$, is 
defined as the summatory function of the classical M\"obius function for $x \geq 1$. 
The inverse function $g^{-1}(n) := (\omega+1)^{-1}(n)$
taken with respect to Dirichlet convolution is defined in terms of the 
strongly additive function $\omega(n)$ that counts the 
number of distinct prime factors of the integers $n \geq 2$ without multiplicity. 
For large $x$ and $n \leq x$, we associate a natural combinatorial 
significance to the magnitude of the distinct values of 
$|g^{-1}(n)|$ that depends directly on the exponent patterns in the 
prime factorizations of the integers $2 \leq n \leq x$ viewed as multisets. 
We conjecture two deterministic Erd\H{o}s-Kac theorem analogs for the distributions of each of the 
unsigned sequences $C_{\Omega}(n) := (\Omega(n))! \times \prod_{p^{\alpha}||n} (\alpha!)^{-1}$ and 
$|g^{-1}(n)|$ over $n \leq x$ as $x \rightarrow \infty$. 
Discrete convolutions of the summatory function 
$G^{-1}(x) := \sum_{n \leq x} \lambda(n) |g^{-1}(n)|$ with 
the prime counting function $\pi(x)$ determine 
exact formulas and new characterizations of the asymptotic behavior of $M(x)$. 
In this way, we prove another characteristic 
link of the Mertens function to the distribution of the partial sums  
$L(x) := \sum_{n \leq x} \lambda(n)$ and connect these two classical 
summatory functions with explicit probability distributions at large $x$. 

\bigskip 
\noindent
\textbf{Keywords and Phrases:} {\it M\"obius function; Mertens function; 
                                    Dirichlet inverse; Liouville lambda function; prime omega function; 
                                    prime counting function; Dirichlet generating function; 
                                    prime zeta function; 
                                    Erd\H{o}s-Kac theorem; strongly additive function. } \\ 
% 11-XX			Number theory
%    11A25  	Arithmetic functions; related numbers; inversion formulas
%    11Y70  	Values of arithmetic functions; tables
%    11-04  	Software, source code, etc. for problems pertaining to number theory
% 11Nxx		Multiplicative number theory
%    11N05  	Distribution of primes
%    11N37  	Asymptotic results on arithmetic functions
%    11N56  	Rate of growth of arithmetic functions
%    11N60  	Distribution functions associated with additive and positive multiplicative functions
%    11N64  	Other results on the distribution of values or the characterization of arithmetic functions
\textbf{Math Subject Classifications (MSC 2010):} {\it 11N37; 11A25; 11N60; 11N64; and 11-04. } 
\end{abstract} 

%\bigskip\hrule\medskip
%\begin{center}
%\begin{adjustwidth}{3.5cm}{}
%\emph{It is evident that the primes are randomly distributed but, \\ 
%        unfortunately, we do not know what 'random' means.} \\ 
%      \textbf{R. C. Vaughan} 
%\end{adjustwidth}
%\end{center}
%\medskip\hrule\bigskip

\newpage
\renewcommand{\contentsname}{Article Index}
\setcounter{tocdepth}{2}
\addcontentsline{toc}{section}{\nameref{Section_NotationAndConventions}}
\tableofcontents

\newpage
\section*{Notation and conventions}
\label{Section_NotationAndConventions}

The next listing provides a glossary of common notation, conventions and 
abbreviations employed throughout the article. 

\renewcommand*{\glsclearpage}{}
\renewcommand{\glossarysection}[2][]{}
\printglossary[type={symbols},
               style={glossstyleSymbol},
               nogroupskip=true]

\newpage
\section{Introduction} 
\label{subSection_MertensMxClassical_Intro} 

The \emph{Mertens function} is the summatory function of $\mu(n)$ defined for 
any positive integer $x \geq 1$ by the partial sums 
\begin{align*} 
M(x) & = \sum_{n \leq x} \mu(n), x \geq 1. 
\end{align*} 
The first several oscillatory values of $M(x)$ are calculated as follows 
\cite[\seqnum{A008683}; \seqnum{A002321}]{OEIS}: 
\[
\{M(x)\}_{x \geq 1} = \{1, 0, -1, -1, -2, -1, -2, -2, -2, -1, -2, -2, -3, -2, 
     -1, -1, -2, -2, -3, -3, -2, -1, -2, \ldots\}. 
\] 
The Mertens function is related 
to the partial sums of the Liouville lambda function, 
denoted by $L(x) := \sum\limits_{n \leq x} \lambda(n)$, 
via the relation \cite{HUMPHRIES-JNT-2013,LEHMAN-1960} 
\cite[\seqnum{A008836}; \seqnum{A002819}]{OEIS}
\[
L(x) = \sum_{d \leq \sqrt{x}} M\left(\Floor{x}{d^2}\right), x \geq 1. 
\] 
The main interpretation to take away from the article is 
the characterization of $M(x)$ through two primary 
new auxiliary unsigned sequences and their 
summatory functions, namely, the functions $C_{\Omega}(n)$ and $|g^{-1}(n)|$, and their 
partial sums. This characterization is formed by constructing the 
combinatorially motivated sequences related to the distribution of the primes 
by convolutions of the strongly additive function $\omega(n)$. 
The methods in this article stem from an initial 
curiosity about an elementary identity from 
the list of exercises in \cite[\S 2; \cf \S 11]{APOSTOLANUMT}. 
In particular, the indicator function of the primes is given by M\"obius inversion as the 
Dirichlet convolution $\chi_{\mathbb{P}} = \omega \ast \mu$. 
We form partial sums of 
$\chi_{\mathbb{P}} + \varepsilon = (\omega + 1) \ast \mu(n)$ over $n \leq x$ for 
any $x \geq 1$ and then apply classical inversion theorems to relate  
$M(x)$ to the partial sums of $g^{-1}(n) := (\omega+1)^{-1}(n)$ 
(see Theorem \ref{theorem_SummatoryFuncsOfDirCvls}; 
Corollary \ref{cor_CvlGAstMu}; and  
Corollary \ref{cor_Mx_gInvnPixk_formula}). 

\subsection{Motivation} 

There is a natural relationship between $g^{-1}(n)$ and the auxiliary function 
$C_{\Omega}(n)$, or the $\Omega(n)$-fold Dirichlet convolution of $\omega(n)$ 
with itself evaluated at $n$, which we prove by elementary methods in 
Section \ref{Section_InvFunc_PreciseExpsAndAsymptotics}. 
These identities inspire the deep connection between the 
unsigned inverse function, $|g^{-1}(n)|$, 
and the resulting additive prime counting combinatorics we find in 
Section \ref{subSection_AConnectionToDistOfThePrimes}. 
In the sense of pulling in the property of the strong additivity of $\omega(n)$ into our 
constructions, the new results stated within this article diverge from the proofs 
typified by previous analytic and combinatorial 
methods to bound $M(x)$ from the references. 
The function $C_{\Omega}(n)$ is considered under alternate notation 
by Fr\"oberg (circa 1968) in his work on the series expansions of the 
\emph{prime zeta function}, $P(s)$, 
i.e., the prime sums defined as the 
Dirichlet generating function (or DGF) of $\chi_{\mathbb{P}}(n)$. 
The clear connection of the function $C_{\Omega}(n)$ to $M(x)$ 
is unique to our work to establish the 
properties of this auxiliary sequence. 
References to uniform asymptotics for restricted partial sums of 
$C_{\Omega}(n)$ and the conjectured features of the limiting distribution 
of this function are missing in surrounding literature 
(see Corollary \ref{cor_SummatoryFuncsOfUnsignedSeqs_v2}; 
Proposition \ref{lemma_HatCAstxSum_ExactFormulaWithError_v1}; and 
Conjecture \ref{conj_DetFormOfEKTypeThmForCOmegan_v1}). 

The signed inverse sequence $g^{-1}(n)$ and its partial sums defined by 
$G^{-1}(x) := \sum_{n \leq x} g^{-1}(n)$ are linked to 
canonical examples of strongly and completely additive functions, 
e.g., in relation to $\omega(n)$ and $\Omega(n)$, respectively. 
The definitions of the sequences we formulate, and the 
proof methods given in the spirit of Montgomery and Vaughan's work, 
allow us to reconcile the property of strong additivity with the signed 
partial sums of a multiplicative function. 
We leverage the connection of $C_{\Omega}(n)$ and $|g^{-1}(n)|$ with 
additivity to obtain the results proved 
in Section \ref{Section_NewFormulasForgInvn}.
We reformulate the proofs of the results in \cite[\S 7.4; \S 2.4]{MV} 
that apply traditional analytic methods to formulate limiting asymptotics and to 
prove an Erd\H{o}s-Kac theorem analog characterizing key properties of the 
distribution of the completely additive 
function $\Omega(n)$. Adaptations of the key ideas 
from the exposition in the reference 
provide a foundation for analytic proofs of several limiting 
properties of, asymptotic formulae for restricted partial sums involving, and in part the 
conjectured deterministic Erd\H{o}s-Kac type theorems for both 
$C_{\Omega}(n)$ and $|g^{-1}(n)|$. 

In the process, we also formalize a probabilistic perspective from which to express 
our intuition about features of the distribution of $G^{-1}(x)$ 
via the properties of its $\lambda(n)$-sign-weighted summands. 
That is, since we prove that $\operatorname{sgn}(g^{-1}(n)) = \lambda(n)$ for all $n \geq 1$ in 
Proposition \ref{prop_SignageDirInvsOfPosBddArithmeticFuncs_v1}, 
the partial sums defined by $G^{-1}(x)$ are precisely related to the properties of 
$|g^{-1}(n)|$ and asymptotics for $L(x)$. 
Our new results then relate the 
distribution of $L(x)$, explicitly identified 
probability distributions, and $M(x)$ as $x \rightarrow \infty$. 
Stating tight bounds on the properties of the distribution of 
$L(x)$ is still viewed as a problem that is equally as difficult 
as understanding the properties of $M(x)$ well at large $x$ or along infinite subsequences. 

Our characterizations of $M(x)$ by the summatory function of the signed 
inverse sequence, $G^{-1}(x)$, 
is suggestive of new approaches to bounding the Mertens function. 
These results motivate future work to state upper (and possibly lower) bounds 
on $M(x)$ in terms of the additive combinatorial properties of the repeated distinct 
values of the sign weighted summands of $G^{-1}(x)$. 
We also expect that an outline of the method behind the collective proofs we 
provide with respect to the Mertens function case can be generalized to identify 
associated additive functions with the same role of $\omega(n)$ in this paper. 
Such generalizations can then be used to 
express asymptotics for partial sums of other Dirichlet inverse functions. 

\subsection{Preliminaries on the Mertens function}
\label{subSection_Intro_Mx_properties} 

An approach to evaluating the 
behavior of $M(x)$ for large $x \rightarrow \infty$ considers an 
inverse Mellin transform of the reciprocal of the Riemann zeta function given by 
\[
\frac{1}{\zeta(s)} = \prod_{p} \left(1 - \frac{1}{p^s}\right) = 
     s \times \int_1^{\infty} \frac{M(x)}{x^{s+1}} dx, \text{ for } \Re(s) > 1. 
\]
In particular, we obtain the following contour integral representation of $M(x)$ for $x \geq 1$: 
\[
M(x) = \lim_{T \rightarrow \infty}\ \frac{1}{2\pi\imath} \times \int_{T-\imath\infty}^{T+\imath\infty} 
     \frac{x^s}{s \zeta(s)} ds. 
\] 
The previous formulas lead to the 
exact expression of $M(x)$ for any $x > 0$ 
given by the next theorem. 
\nocite{TITCHMARSH} 

\begin{theorem}[Titchmarsh] 
\label{theorem_MxMellinTransformInvFormula} 
Assuming the Riemann Hypothesis (RH), there exists an infinite sequence 
$\{T_k\}_{k \geq 1}$ satisfying $k \leq T_k \leq k+1$ for each integer $k \geq 1$ 
such that for any real $x > 0$ 
\[
M(x) = \lim_{k \rightarrow \infty} 
     \sum_{\substack{\rho: \zeta(\rho) = 0 \\ 0 < |\Im(\rho)| < T_k}} 
     \frac{x^{\rho}}{\rho \zeta^{\prime}(\rho)} + 
     \sum_{n \geq 1} \frac{(-1)^{n-1}}{n (2n)! \zeta(2n+1)} 
     \left(\frac{2\pi}{x}\right)^{2n} + 
     \frac{\mu(x)}{2} \Iverson{x \in \mathbb{Z}^{+}} - 2. 
\] 
\end{theorem} 

An unconditional bound on the Mertens function due to Walfisz (circa 1963) 
states that there is an absolute constant $C_1 > 0$ such that 
$$M(x) \ll x \times \exp\left(-C_1 \log^{\frac{3}{5}}(x) 
  (\log\log x)^{-\frac{1}{5}}\right).$$ 
Under the assumption of the RH, Soundararajan and Humphries, respectively, 
improved estimates bounding $M(x)$ from above for large $x$ in the 
following forms 
\cite{SOUND-MERTENS-ANNALS,HUMPHRIES-JNT-2013}: 
\begin{align*} 
M(x) & \ll \sqrt{x} \times \exp\left(\sqrt{\log x} (\log\log x)^{14}\right), \\ 
M(x) & \ll \sqrt{x} \times \exp\left( 
     \sqrt{\log x} (\log\log x)^{\frac{5}{2}+\epsilon}\right), 
     \text{ for all } \epsilon > 0. 
\end{align*} 
The RH is equivalent to showing that 
\begin{equation} 
\label{eqn_MertensMx_RHEquivProblem_Stmt_intro} 
M(x) = O\left(x^{\frac{1}{2}+\epsilon}\right), \text{ for all } 0 < \epsilon < \frac{1}{2}.
\end{equation}
There is a rich history to the original statement of the \emph{Mertens conjecture} which 
asserts that $|M(x)| < C_2 \sqrt{x}$ for some absolute constant $C_2 > 0$. 
The conjecture was first verified by F.~Mertens himself for $C_2 = 1$ and all $x < 10^{4}$ 
without the benefit of modern computation. 
Since its beginnings in 1897, the Mertens conjecture was disproved by computational methods involving 
non-trivial simple zeta function zeros with comparatively small imaginary parts in the 
famous paper by Odlyzko and te Riele \cite{ODLYZ-TRIELE}. 

More recent attempts 
at bounding $M(x)$ naturally consider determining the rates at which the function 
$M(x) x^{-\frac{1}{2}}$ grows with or without bound along infinite 
subsequences, e.g., considering the asymptotics of the function 
in the limit supremum and limit infimum senses. 
It is verified by computation 
that \cite[\cf \S 4.1]{PRIMEREC} 
\cite[\cf \seqnum{A051400}; \seqnum{A051401}]{OEIS} 
\[
\limsup_{x\rightarrow\infty} \frac{M(x)}{\sqrt{x}} > 1.060\ \qquad (\text{more recently } \geq 1.826054), 
\] 
and 
\[ 
\liminf_{x\rightarrow\infty} \frac{M(x)}{\sqrt{x}} < -1.009\ \qquad (\text{more recently } \leq -1.837625). 
\] 
Based on the work by Odlyzko and te Riele, it is widely believed that 
these limiting bounds evaluate to $\pm \infty$, respectively 
\cite{ODLYZ-TRIELE,MREVISITED,ORDER-MERTENSFN,HURST-2017}. 
A conjecture due to Gonek asserts that 
$M(x)$ in fact satisfies \cite{NG-MERTENS}
$$\limsup_{x \rightarrow \infty} \frac{|M(x)|}{\sqrt{x} (\log\log\log x)^{\frac{5}{4}}} = C_3,$$ 
for $C_3 > 0$ an absolute constant. 

\subsection{A concrete new approach to characterizing $M(x)$} 

\subsubsection{Summatory functions of Dirichlet convolutions of arithmetic functions} 

We prove the formulas in the next inversion theorem by matrix methods in 
Section \ref{subSection_PrelimProofs_Config_InversionTheorem}. 

\begin{theorem}[Partial sums of Dirichlet convolutions and their inversions] 
\label{theorem_SummatoryFuncsOfDirCvls} 
Let $r,h: \mathbb{Z}^{+} \rightarrow \mathbb{C}$ be any arithmetic functions such that $r(1) \neq 0$. 
Suppose that $R(x) := \sum_{n \leq x} r(n)$ and $H(x) := \sum_{n \leq x} h(n)$ denote the summatory 
functions of $r$ and $h$, respectively, and that $R^{-1}(x) := \sum_{n \leq x} r^{-1}(n)$ 
denotes the summatory function of the 
Dirichlet inverse of $r$ for any $x \geq 1$. We have that the following exact expressions hold 
for all integers $x \geq 1$: 
\begin{align*} 
\pi_{r \ast h}(x) & := \sum_{n \leq x} \sum_{d|n} r(d) h\left(\frac{n}{d}\right) \\ 
     & \phantom{:}= \sum_{d \leq x} r(d) H\left(\Floor{x}{d}\right) \\ 
     & \phantom{:}= \sum_{k=1}^{x} H(k) \left(R\left(\Floor{x}{k}\right) - 
     R\left(\Floor{x}{k+1}\right)\right). 
\end{align*} 
Moreover, for any $x \geq 1$ we have 
\begin{align*} 
H(x) & = \sum_{j=1}^{x} \pi_{r \ast h}(j) \left(R^{-1}\left(\Floor{x}{j}\right) - 
     R^{-1}\left(\Floor{x}{j+1}\right)\right) \\ 
     & = \sum_{k=1}^{x} r^{-1}(k) \pi_{r \ast h}\left(\Floor{x}{k}\right). 
\end{align*} 
\end{theorem} 

Key consequences of Theorem \ref{theorem_SummatoryFuncsOfDirCvls} 
as it applies to $M(x)$ in the special cases where $h(n) := \mu(n)$ for all $n \geq 1$ 
are stated in the next two corollaries. 

\begin{cor}[Applications of M\"obius inversion] 
\label{cor_CvlGAstMu} 
Suppose that $r$ is an arithmetic function such that 
$r(1) \neq 0$. Define the summatory function of 
the convolution of $r$ with $\mu$ by $\widetilde{R}(x) := \sum_{n \leq x} (r \ast \mu)(n)$. 
Then the Mertens function is expressed by the partial sums 
\[
M(x) = \sum_{k=1}^{x} \left(\sum_{j=\floor{\frac{x}{k+1}}+1}^{\floor{\frac{x}{k}}} r^{-1}(j)\right) 
     \widetilde{R}(k), \forall x \geq 1. 
\]
\end{cor} 

\begin{cor}[Key Identity] 
\label{cor_Mx_gInvnPixk_formula} 
We have that for all $x \geq 1$ 
\begin{equation} 
\label{eqn_Mx_gInvnPixk_formula} 
M(x) = \sum_{k=1}^{x} (\omega+1)^{-1}(k) \left(\pi\left(\Floor{x}{k}\right) + 1\right). 
\end{equation} 
\end{cor} 

\subsubsection{An exact expression for $M(x)$ via strongly additive functions} 
\label{example_InvertingARecRelForMx_Intro}

We fix the notation for the Dirichlet invertible function $g(n) := \omega(n) + 1$ and define its 
inverse with respect to Dirichlet convolution by $g^{-1}(n)$. 
We compute the first several values of this sequence as follows \cite[\seqnum{A341444}]{OEIS}: 
\[
\{g^{-1}(n)\}_{n \geq 1} = \{1, -2, -2, 2, -2, 5, -2, -2, 2, 5, -2, -7, -2, 5, 5, 2, -2, -7, -2, 
     -7, 5, 5, -2, 9, \ldots \}. 
\] 
There is not a simple 
direct recursion between the distinct values of $g^{-1}(n)$ that holds for all $n \geq 1$. 
However, the next observation is suggestive of the quasi-periodicity of the distribution of 
distinct values of this inverse function over $n \geq 2$. 

\begin{observation}[Additive symmetry in $g^{-1}(n)$ from the prime factorizations of $n \leq x$] 
\label{heuristic_SymmetryIngInvFuncs} 
Suppose that $n_1, n_2 \geq 2$ are such that their factorizations into distinct primes are 
given by $n_1 = p_1^{\alpha_1} \times \cdots \times p_r^{\alpha_r}$ and 
$n_2 = q_1^{\beta_1} \times \cdots \times q_s^{\beta_s}$. 
If $r = s$ and $\{\alpha_1, \ldots, \alpha_r\} \equiv \{\beta_1, \ldots, \beta_r\}$ 
as multisets of the prime exponents, 
then $g^{-1}(n_1) = g^{-1}(n_2)$. For example, $g^{-1}$ has the same values on the squarefree integers 
with exactly one, two, three (and so on) prime factors, or at all $n$ of the form 
$n = p_1p_2^2p_3^4$ for $p_1$, $p_2$ and $p_3$ distinct primes. 
Hence, there is an essentially additive structure underneath the sequence 
$\{g^{-1}(n)\}_{n \geq 2}$. 
\end{observation} 

\begin{prop} 
\label{lemma_gInv_MxExample} 
We have the following properties of the 
Dirichlet inverse function $g^{-1}(n)$: 
\begin{itemize} 

\item[(A)] For all $n \geq 1$, $\operatorname{sgn}(g^{-1}(n)) = \lambda(n)$; 
\item[(B)] For squarefree integers $n \geq 1$, we have that 
     \[
     |g^{-1}(n)| = \sum_{m=0}^{\omega(n)} \binom{\omega(n)}{m} \times m!; 
     \]
\item[(C)] If $n \geq 1$ and $\Omega(n) = k$ for some $k \geq 0$, then 
     \[
     2 \leq |g^{-1}(n)| \leq \sum_{j=0}^{k} \binom{k}{j} \times j!. 
     \]
\end{itemize} 
\end{prop} 

The signedness property in (A) is proved precisely in 
Proposition \ref{prop_SignageDirInvsOfPosBddArithmeticFuncs_v1}. 
A proof of (B) follows from 
Lemma \ref{lemma_AnExactFormulaFor_gInvByMobiusInv_v1}. 
The realization that the beautiful and remarkably simple combinatorial form of property (B) 
in Proposition \ref{lemma_gInv_MxExample} holds for all squarefree integers 
motivates our pursuit of simpler formulas for the inverse function $g^{-1}(n)$ 
through the sums of the auxiliary sequence 
$C_{\Omega}(n)$, that is defined in Section \ref{Section_InvFunc_PreciseExpsAndAsymptotics}. 
We observe a familiar formula for $g^{-1}(n)$ 
on an asymptotically dense infinite subset of integers (with density $\frac{6}{\pi^2}$), i.e., 
that holds for all squarefree $n \geq 2$, and then seek 
to extrapolate by conjecturing there are in fact 
regular properties of the distribution of this sequence when viewed 
more generally over the positive integers. 

An exact expression for $g^{-1}(n)$ is given by 
\[
g^{-1}(n) = \lambda(n) \times \sum_{d|n} \mu^2\left(\frac{n}{d}\right) C_{\Omega}(d), n \geq 1,  
\]
where the sequence $\lambda(n) C_{\Omega}(n)$ has the DGF $(1 + P(s))^{-1}$ and 
$C_{\Omega}(n)$ has DGF $(1-P(s))^{-1}$ for $\Re(s) > 1$ 
(see Proposition \ref{prop_SignageDirInvsOfPosBddArithmeticFuncs_v1}). 
The function $C_{\Omega}(n)$ was considered in 
\cite{FROBERG-1968} with an exact formula 
given by \cite[\cf \S 3]{CLT-RANDOM-ORDERED-FACTS-2011} 
\[
C_{\Omega}(n) = \begin{cases}
     1, & \text{if $n = 1$; } \\ 
     (\Omega(n))! \times \prod\limits_{p^{\alpha}||n} \frac{1}{\alpha!}, & \text{if $n \geq 2$. }
     \end{cases}
\]
In Corollary \ref{cor_SummatoryFuncsOfUnsignedSeqs_v2}, 
we use the result proved in 
Theorem \ref{theorem_CnkSpCasesScaledSummatoryFuncs} 
to show that uniformly for $1 \leq k \leq \frac{3}{2} \log\log x$ there is an absolute 
constant $A_0 > 0$ such that 
\[
\sum_{\substack{n \leq x \\ \Omega(n)=k}} C_{\Omega}(n) = 
     \frac{A_0 \sqrt{2\pi} x}{\log x} \times 
     \widehat{G}\left(\frac{k-1}{\log\log x}\right) 
     \frac{(\log\log x)^{k-\frac{1}{2}}}{(k-1)!} \left( 
     1 + O\left(\frac{1}{\log\log x}\right)\right), 
     \mathrm{\ as\ } x \rightarrow \infty, 
\]
where $\widehat{G}(z) := \frac{\zeta(2)^{-z}}{\Gamma(1+z) (1+P(2)z)}$ for 
$0 \leq |z| < P(2)^{-1}$. 
In Proposition \ref{lemma_HatCAstxSum_ExactFormulaWithError_v1}, 
we use an adaptation of the asymptotics for the summations 
proved in the appendix section of this article
combined with the form of \emph{Rankin's method} from \cite[Thm.~7.20]{MV} to show that 
there is an absolute constant $B_0 > 0$ such that 
\[
\frac{1}{n} \times \sum_{k \leq n} C_{\Omega}(k) = 
     B_0 (\log n) \sqrt{\log\log n} 
     \left(1 + O\left(\frac{1}{\log\log n}\right)\right), 
     \mathrm{\ as\ } n \rightarrow \infty. 
\]
In Corollary \ref{cor_ExpectationFormulaAbsgInvn_v2}, we prove that 
the average order of $|g^{-1}(n)|$ is 
\[
\frac{1}{n} \times \sum_{k \leq n} |g^{-1}(k)| = 
     \frac{6B_0 (\log n)^2 \sqrt{\log\log n}}{\pi^2} 
     \left(1 + O\left(\frac{1}{\log\log n}\right)\right), 
     \mathrm{\ as\ } n \rightarrow \infty. 
\]
In Section \ref{subSection_ErdosKacTheorem_Analogs}, 
we conjecture the forms of two variants of the Erd\H{o}s-Kac theorem 
that characterize the distribution of $C_{\Omega}(n)$.
The first proposed deterministic form of the 
theorem stated in Conjecture \ref{conj_DetFormOfEKTypeThmForCOmegan_v1} 
leads the conclusion of the following result for any fixed $Y > 0$ 
which holds uniformly for all $-Y \leq y \leq Y$ with  
$\mu_x(C) := \log\log x - 
 \log\left(\frac{\sqrt{2\pi}A_0}{\zeta(2)(1+P(2))}\right)$ and 
$\sigma_x(C) := \sqrt{\log\log x}$ 
(see Corollary \ref{cor_CLT_VII}): 
\begin{align*}
\frac{1}{x} \times & \#\left\{3 \leq n \leq x: 
     \frac{|g^{-1}(n)|}{(\log n) \sqrt{\log\log n}} - 
     \frac{6}{\pi^2 n (\log n) \sqrt{\log\log n}} 
     \times \sum_{k \leq n} |g^{-1}(k)| \leq y\right\} \\ 
     & = 
     \Phi\left(\frac{\frac{\pi^2 y}{6}-\mu_x(C)}{\sigma_x(C)}\right) + 
     o(1), \text{ as } x \rightarrow \infty. 
\end{align*}
We also conjecture that for any real $y$, as $x \rightarrow \infty$ 
\begin{align*} 
\frac{1}{x} \times \#\left\{2 \leq n \leq x: |g^{-1}(n)| - 
     \frac{6}{\pi^2 n} \times \sum_{k \leq n} |g^{-1}(k)| \leq y\right\} & = 
     \Phi\left(\frac{\frac{\pi^2 y}{6}- B_0(\log x)\sqrt{\log\log x}}{ 
     D_0\sqrt{x}(\log x)\sqrt{\log\log x}}\right) + o(1), 
\end{align*} 
where $D_0 > 0$ is an absolute constant. 
By Proposition \ref{lemma_gInv_MxExample} and since 
\[
\lim_{x \rightarrow \infty} \frac{(\log\log x)^{\log\log x}}{\sqrt{x}} = 0, 
\]
we see that if the result in the last equation holds, the unsigned inverse function 
$|g^{-1}(n)|$ is centered at its average order for nearly 
every sufficiently large $n$ with only a negligible error in the approximation 
(\cf Section \ref{subSection_TheKnownDistsOfThePrimeOmegaFunctions_IntroResults_v1}). 
The regularity and quasi-periodicity we alluded 
to in the previous remarks are then 
quantifiable insomuch as $|g^{-1}(n)|$ 
usually tends to a scaled multiple of its average order 
with a non-central normal tendency 
(provided that Conjecture \ref{conj_DetFormOfEKTypeThmForCOmegan_v1} holds). 

\subsubsection{Formulas illustrating the new characterizations of $M(x)$} 

Let the partial sums 
$G^{-1}(x) := \sum_{n \leq x} g^{-1}(n)$ for integers $x \geq 1$ 
\cite[\seqnum{A341472}]{OEIS}. 
We prove that (see Proposition \ref{prop_Mx_SBP_IntegralFormula}) 
\begin{align} 
\label{eqn_Mx_gInvnPixk_formula_v2} 
M(x) & = G^{-1}(x) + 
     \sum_{k=1}^{\frac{x}{2}} G^{-1}(k) \left( 
     \pi\left(\Floor{x}{k}\right) - \pi\left(\Floor{x}{k+1}\right) 
     \right), x \geq 1, 
\end{align} 
and that (\cf Section \ref{subSection_Relating_CknFuncs_to_gInvn}) 
\[
M(x) = G^{-1}(x) + \sum_{p \leq x} G^{-1}\left(\Floor{x}{p}\right), x \geq 1. 
\]
These formulas 
imply that we can establish asymptotic bounds on 
$M(x)$ along infinite subsequences
by sharply bounding the summatory function $G^{-1}(x)$ along these points. 
Suppose that the unsigned partial sums are defined by 
\[
|G^{-1}|(x) := \sum_{n \leq x} |g^{-1}(n)|, x \geq 1. 
\]
We have an identification of $G^{-1}(x)$ with $L(x)$ given by 
\begin{align*}
G^{-1}(x) & = 
     L(x)|g^{-1}(x)| - \sum_{n < x} 
     L(n) \left(\left\lvert g^{-1}(n+1)\right\rvert - \left\lvert g^{-1}(n)\right\rvert\right), \\ 
     & \sim \sum_{n \leq x} \lambda(n) \left(\int_{n-1}^{n} \frac{d}{dt} |G^{-1}|(t) dt\right), 
\end{align*} 
where the distribution of $|g^{-1}(n)|$ is characterized by the conjectured results in 
Corollary \ref{cor_CLT_VII}. 
In Section \ref{subSection_AsymptoticsOfGinvx}, 
we use the analytic methods due to H.~Davenport and H.~Heilbronn to prove that for 
$\sigma_1 \approx 1.39943$, the unique solution to $P(\sigma) = 1$ on 
$(1, \infty)$, we have 
\[
\limsup_{x \rightarrow \infty} \frac{\log\left\lvert G^{-1} \right\rvert(x)}{\log x} \geq \sigma_1. 
\]
Thus, for any $\epsilon > 0$ there are arbitrarily large $x$ such that 
(see Corollary \ref{cor_Vaughan_LimSupLowerBounds_On_GInvx_AtLarge_x_v2}) 
\[
|G^{-1}|(x) > x^{\sigma_1-\epsilon}. 
\]
These bounds on the partial sums with unsigned inverse function summands provide some local 
information on $G^{-1}(x)$ through its connection with $|G^{-1}|(x)$ as expanded in the equation above 
(see Remark \ref{remark_AbsGInvx_ProspectsOfTheNewBVBoundProofs_v2}). 
Nonetheless, we still expect substantial local cancellation in the terms involving 
$G^{-1}(x)$ in our new formulas for $M(x)$ at almost every large $x$ 
(see Section \ref{subSection_LocalCancellationOfGInvx}). 

\section{Initial elementary proofs of new results} 
\label{Section_PrelimProofs_Config} 

\subsection{Establishing the summatory function properties and inversion identities} 
\label{subSection_PrelimProofs_Config_InversionTheorem}

We give a proof of the inversion type results in 
Theorem \ref{theorem_SummatoryFuncsOfDirCvls} 
by matrix methods in this subsection. 
Related results on summations of Dirichlet convolutions and their inversion appear in 
\cite[\S 2.14; \S 3.10; \S 3.12; \cf \S 4.9, p.\ 95]{APOSTOLANUMT}. 
It is similarly not difficult to establish the identity 
\[
\sum_{n \leq x} h(n) (q \ast r)(n) = 
     \sum_{n \leq x} q(n) \times \sum_{k \leq \Floor{x}{n}} r(k) h(kn). 
\]

\begin{proof}[Proof of Theorem \ref{theorem_SummatoryFuncsOfDirCvls}] 
\label{proofOf_theorem_SummatoryFuncsOfDirCvls} 
Let $h,r$ be arithmetic functions such that $r(1) \neq 0$. 
Let the summatory functions of $h$, $r$ and $r^{-1}$, 
respectively, be denoted by $H(x) = \sum_{n \leq x} h(n)$, $R(x) = \sum_{n \leq x} r(n)$, 
and $R^{-1}(x) = \sum_{n \leq x} r^{-1}(n)$. 
We define $\pi_{r \ast h}(x)$ to be the summatory function of the 
Dirichlet convolution of $r$ with $h$. 
The following formulas hold for all $x \geq 1$: 
\begin{align} 
\notag 
\pi_{r \ast h}(x) & := \sum_{n=1}^{x} \sum_{d|n} r(n) h\left(\frac{n}{d}\right) = 
     \sum_{d=1}^x r(d) H\left(\floor{\frac{x}{d}}\right) \\ 
\label{eqn_proof_tag_PigAsthx_ExactSummationFormula_exp_v2} 
     & = \sum_{i=1}^x \left(R\left(\floor{\frac{x}{i}}\right) - R\left(\floor{\frac{x}{i+1}}\right)\right) H(i). 
\end{align} 
The first formula above is well known from the references cited above. 
The second formula is justified directly using 
summation by parts as \cite[\S 2.10(ii)]{NISTHB} 
\begin{align*} 
\pi_{r \ast h}(x) & = \sum_{d=1}^x h(d) R\left(\floor{\frac{x}{d}}\right) \\ 
     & = \sum_{i \leq x} \left(\sum_{j \leq i} h(j)\right) \times 
     \left(R\left(\floor{\frac{x}{i}}\right) - 
     R\left(\floor{\frac{x}{i+1}}\right)\right). 
\end{align*} 
We form the invertible matrix of coefficients, denoted by $\hat{R}$ below, 
associated with the linear system defining $H(j)$ for all 
$1 \leq j \leq x$ in \eqref{eqn_proof_tag_PigAsthx_ExactSummationFormula_exp_v2} by setting 
\[
r_{x,j} := R\left(\floor{\frac{x}{j}}\right) - R\left(\floor{\frac{x}{j+1}}\right) 
     \equiv R_{x,j} - R_{x,j+1}, 
\] 
with 
\[
R_{x,j} := R\left(\Floor{x}{j}\right), \text{ for } 1 \leq j \leq x. 
\]
Since $r_{x,x} = R(1) = r(1) \neq 0$ for all $x \geq 1$ and $r_{x,j} = 0$ for all $j > x$, 
the matrix we have defined in this problem is lower triangular with a non-zero 
constant on its diagonals, and so is invertible. 
If we let $\hat{R} := (R_{x,j})$, then the next matrix is 
expressed by applying an invertible shift operation as 
\[
(r_{x,j}) = \hat{R} \left(I - U^{T}\right). 
\]
The square matrix $U$ of sufficiently large finite dimensions $N \times N$ for $N \geq x$ 
has $(i,j)^{th}$ entries for all $1 \leq i,j \leq N$ that are defined by 
$(U)_{i,j} = \delta_{i+1,j}$ so that 
\[
\left[(I - U^T)^{-1}\right]_{i,j} = \Iverson{j \leq i}. 
\]
We observe that 
\[
\Floor{x}{j} - \Floor{x-1}{j} = \begin{cases} 
     1, & \text{ if $j|x$; } \\ 
     0, & \text{ otherwise. } 
     \end{cases} 
\] 
The previous equation implies that 
\begin{equation} 
\label{eqn_proof_tag_FloorFuncDiffsOfSummatoryFuncs_v2} 
R\left(\floor{\frac{x}{j}}\right) - R\left(\floor{\frac{x-1}{j}}\right) = 
     \begin{cases} 
     r\left(\frac{x}{j}\right), & \text{ if $j | x$; } \\ 
     0, & \text{ otherwise. } 
     \end{cases}
\end{equation} 
We use the property in \eqref{eqn_proof_tag_FloorFuncDiffsOfSummatoryFuncs_v2} 
to shift the matrix $\hat{R}$, and then invert the result to obtain a matrix involving the 
Dirichlet inverse of $r$ as follows: 
\begin{align*} 
\left(\left(I-U^{T}\right) \hat{R}\right)^{-1} & = 
     \left(r\left(\frac{x}{j}\right) \Iverson{j|x}\right)^{-1} = 
     \left(r^{-1}\left(\frac{x}{j}\right) \Iverson{j|x}\right). 
\end{align*} 
Our target matrix in the inversion problem is defined by 
$$(r_{x,j}) = \left(I-U^{T}\right) \left(r\left(\frac{x}{j}\right) \Iverson{j|x}\right) \left(I-U^{T}\right)^{-1}.$$
We can express its inverse by a similarity transformation conjugated by shift operators in the form of 
\begin{align*} 
(r_{x,j})^{-1} & = \left(I-U^{T}\right)^{-1} \left(r^{-1}\left(\frac{x}{j}\right) 
     \Iverson{j|x}\right) \left(I-U^{T}\right) \\ 
     & = \left(\sum_{k=1}^{\floor{\frac{x}{j}}} r^{-1}(k)\right) (I-U^{T}) \\ 
     & = \left(\sum_{k=1}^{\floor{\frac{x}{j}}} r^{-1}(k) - \sum_{k=1}^{\floor{\frac{x}{j+1}}} r^{-1}(k)\right). 
\end{align*} 
The summatory function $H(x)$ is given exactly for any integers $x \geq 1$ 
by a vector product with the inverse matrix from the previous equation in the form of 
\begin{align*} 
H(x) & = \sum_{k=1}^x \left(\sum_{j=\floor{\frac{x}{k+1}}+1}^{\floor{\frac{x}{k}}} r^{-1}(j)\right) 
     \times \pi_{r \ast h}(k). 
\end{align*} 
We can prove a second inversion formula providing the coefficients of the summatory function 
$R^{-1}(j)$ for $1 \leq j \leq x$ from the last equation by adapting our argument to prove 
\eqref{eqn_proof_tag_PigAsthx_ExactSummationFormula_exp_v2} above. 
This leads to the alternate identity expressing $H(x)$ given by 
\[
H(x) = \sum_{k=1}^{x} r^{-1}(k) \times \pi_{r \ast h}\left(\Floor{x}{k}\right). 
     \qedhere 
\]
\end{proof} 

\subsection{Proving the characteristic signedness property of $g^{-1}(n)$} 

Let $\chi_{\mathbb{P}}(n)$ denote the characteristic function of the primes, let 
$\varepsilon(n) = \delta_{n,1}$ be the multiplicative identity with respect to Dirichlet convolution, 
and denote by $\omega(n)$ the strongly additive function that counts the number of 
distinct prime factors of $n$ (without multiplicity). We can see using 
elementary methods that 
\begin{equation}
\label{eqn_AntiqueDivisorSumIdent} 
\chi_{\mathbb{P}} + \varepsilon = (\omega + 1) \ast \mu. 
\end{equation} 
Namely, since $\mu \ast 1 = \varepsilon$ and 
\[
\omega(n) = \sum_{p|n} 1 = \sum_{d|n} \chi_{\mathbb{P}}(d), \text{ for } n \geq 1, 
\]
the result in \eqref{eqn_AntiqueDivisorSumIdent} follows by M\"obius inversion. 
When combined with Corollary \ref{cor_CvlGAstMu}, 
this convolution identity yields the key exact 
formula for $M(x)$ stated in \eqref{eqn_Mx_gInvnPixk_formula} of 
Corollary \ref{cor_Mx_gInvnPixk_formula}. 
Notice that the shift by one in the form of $(\omega + 1) \ast \mu$ in the right-hand-side 
convolution in \eqref{eqn_AntiqueDivisorSumIdent} above is 
selected so that the resulting arithmetic function we convolve with $\mu(n)$ in constructing these 
summatory functions is Dirichlet invertible with $(\omega + 1)(1) \neq 0$ where 
$\omega(1) := 0$ (by convention). 

\begin{prop}[The sign of $g^{-1}(n)$]
\label{prop_SignageDirInvsOfPosBddArithmeticFuncs_v1} 
For any arithmetic function $r(n)$, let the operator 
$\operatorname{sgn}(r(n)) = \frac{r(n)}{|r(n)| + \Iverson{r(n) = 0}} \in \{0, \pm 1\}$ 
provide the signedness of the arithmetic function $h$ at any $n \geq 1$, or the mapping of 
$r(n)$ onto $\pm 1$ to indicate its positivity (negativity) or 
otherwise onto zero if the function $r$ vanishes at $n$. 
We have that $\operatorname{sgn}(g^{-1}(n)) = \lambda(n)$ for all $n \geq 1$. 
\end{prop} 
\begin{proof} 
The series $D_f(s) := \sum_{n \geq 1} f(n) n^{-s}$ defines the 
Dirichlet generating function (DGF) of any 
arithmetic function $f$ which is convergent for all $s \in \mathbb{C}$ satisfying 
$\Re(s) > \sigma_f$ where $\sigma_f$ is the abscissa of convergence of the series. 
Recall that $D_1(s) = \zeta(s)$, $D_{\mu}(s) = \zeta(s)^{-1}$ and 
$D_{\omega}(s) = P(s) \zeta(s)$ for $\Re(s) > 1$, where $P(s) := \sum_{n \geq 1} \chi_{\mathbb{P}}(n) n^{-s}$ 
denotes the \emph{prime zeta function} defined in the glossary on 
page \pageref{Section_NotationAndConventions} (\cf \cite{FROBERG-1968}). 
By \eqref{eqn_AntiqueDivisorSumIdent} and the fact that whenever $f(1) \neq 0$, 
the DGF of $f^{-1}(n)$ is $D_f(s)^{-1}$ for $\Re(s) > \sigma_f$, we have that 
\begin{align} 
\label{eqn_DGF_of_gInvn} 
D_{(\omega+1)^{-1}}(s) = \frac{1}{\zeta(s) (1+P(s))}, \Re(s) > 1. 
\end{align} 
It follows that $(\omega + 1)^{-1}(n) = (h^{-1} \ast \mu)(n)$ if we take 
$h := \chi_{\mathbb{P}} + \varepsilon$. 
We first show that $\operatorname{sgn}(h^{-1}) = \lambda$. 
This observation implies that 
$\operatorname{sgn}(h^{-1} \ast \mu) = \lambda$ through the next arguments. 

First, by a combinatorial argument related to multinomial coefficient expansions of these sums, 
we recover exactly that \cite[\cf \S 2]{FROBERG-1968}\footnote{
     Beginning in Section \ref{Section_InvFunc_PreciseExpsAndAsymptotics}, we adopt the alternate 
     notation for the Dirichlet inverse function $h^{-1}(n)$ employed in this proof given by 
     $C_{\Omega}(n)$. See also the remarks following the conclusion of this proof on the 
     function $C_k(n)$. 
} 
\begin{equation} 
\label{eqn_proof_tag_hInvn_ExactNestedSumFormula_CombInterpetIdent_v3} 
h^{-1}(n) = \begin{cases} 
     1, & n = 1; \\ 
     \lambda(n) (\Omega(n))! \times \prod\limits_{p^{\alpha} || n} \frac{1}{\alpha!}, & n \geq 2. 
     \end{cases}
\end{equation} 
In particular, by expanding the DGF of $h^{-1}$ in powers of $P(s)$, where $|P(s)| < 1$ whenever 
$\Re(s) > 1$, we count that 
\begin{align*}
\frac{1}{1+P(s)} & = \sum_{n \geq 1} \frac{h^{-1}(n)}{n^s} = \sum_{k \geq 0} (-1)^k P(s)^k \\ 
     & = 
     1 + \sum_{\substack{n \geq 2 \\ n =p_1^{\alpha_1}p_2^{\alpha_2} \times \cdots \times p_k^{\alpha_k}}} 
     \frac{(-1)^{\alpha_1+\alpha_2+\cdots+\alpha_k}}{n^s} \times 
     \binom{\alpha_1+\alpha_2+\cdots+\alpha_k}{\alpha_1,\alpha_2,\ldots,\alpha_k} \\ 
     & = 
     1 + \sum_{\substack{n \geq 2 \\ n =p_1^{\alpha_1}p_2^{\alpha_2} \times \cdots \times p_k^{\alpha_k}}} 
     \frac{\lambda(n)}{n^s} \times \binom{\Omega(n)}{\alpha_1,\alpha_2,\ldots,\alpha_k}. 
\end{align*}
Since $\lambda$ is completely multiplicative we have that 
$\lambda\left(\frac{n}{d}\right) \lambda(d) = \lambda(n)$ for all divisors 
$d|n$ when $n \geq 1$. We also know that $\mu(n) = \lambda(n)$ whenever $n$ is squarefree, 
so that we obtain the following results: 
\[
g^{-1}(n) = (h^{-1} \ast \mu)(n) = \lambda(n) \times \sum_{d|n} \mu^2\left(\frac{n}{d}\right) |h^{-1}(n)|, n \geq 1. 
     \qedhere 
\]
\end{proof} 

The conclusion of the proof of 
Proposition \ref{prop_SignageDirInvsOfPosBddArithmeticFuncs_v1} 
implies the stronger result that 
\[
g^{-1}(n) = \lambda(n) \times \sum_{d|n} \mu^2\left(\frac{n}{d}\right) C_{\Omega}(d).  
\]
We have adopted the notation that for $n \geq 2$, 
$C_{\Omega}(n) \equiv |h^{-1}(n)| = (\Omega(n))! \times \prod_{p^{\alpha} || n} (\alpha!)^{-1}$, 
where the same function, $C_0(n)$, is taken to be one for $n := 1$ 
(see Section \ref{Section_InvFunc_PreciseExpsAndAsymptotics}). 
%We see that the scaled functions $f_1(n) := \frac{C_{\Omega}(n)}{(\Omega(n))!}$ and 
%$f_2(n) := \frac{\lambda(n) C_{\Omega}(n)}{(\Omega(n))!}$ are both multiplicative. 

\subsection{The distributions of $\omega(n)$ and $\Omega(n)$} 
\label{subSection_TheKnownDistsOfThePrimeOmegaFunctions_IntroResults_v1} 

The next theorems reproduced from \cite[\S 7.4]{MV} characterize the relative 
scarcity of the distributions of $\omega(n)$ and $\Omega(n)$ for $n \leq x$ such that 
$\omega(n),\Omega(n) < \log\log x$ and $\omega(n),\Omega(n) > \log\log x$. 
Since $\frac{1}{n} \times \sum_{k \leq n} \omega(k) = \log\log n + B_1 + o(1)$ and 
$\frac{1}{n} \times \sum_{k \leq n} \Omega(k) = \log\log n + B_2 + o(1)$ for 
$B_1 \approx 0.261497$ and $B_2 \approx 1.03465$ 
absolute constants in each case \cite[\S 22.10]{HARDYWRIGHT}, 
these results imply a distinctively regular tendency 
of these strongly additive arithmetic functions towards their respective average orders. 

\begin{theorem}[Upper bounds on exceptional values of $\Omega(n)$ for large $n$] 
\label{theorem_MV_Thm7.20-init_stmt} 
For $x \geq 2$ and $r > 0$, let 
\begin{align*} 
A(x, r) & := \#\left\{n \leq x: \Omega(n) \leq r \log\log x\right\}, \\ 
B(x, r) & := \#\left\{n \leq x: \Omega(n) \geq r \log\log x\right\}. 
\end{align*} 
If $0 < r \leq 1$, then 
\[
A(x, r) \ll\phantom{_R} x (\log x)^{r-1 - r\log r}, \text{ as } x \rightarrow \infty. 
\]
If $1 \leq r \leq R < 2$, then 
\[
B(x, r) \ll_R x (\log x)^{r-1-r \log r}, \text{ as } x \rightarrow \infty. 
\]
\end{theorem} 

Theorem \ref{theorem_MV_Thm7.21-init_stmt} is a special case analog of the 
Erd\H{o}s-Kac theorem for the normally distributed values of 
$\frac{\omega(n) - \log\log n}{\sqrt{\log\log n}}$ over $n \leq x$ as 
$x \rightarrow \infty$ \cite[\cf Thm.\ 7.21]{MV} \cite[\cf \S 1.7]{IWANIEC-KOWALSKI}. 

\begin{theorem}
\label{theorem_MV_Thm7.21-init_stmt} 
We have that as $x \rightarrow \infty$ 
\[
\#\left\{3 \leq n \leq x: \Omega(n) \leq \log\log n \right\} = 
     \frac{x}{2} + O\left(\frac{x}{\sqrt{\log\log x}}\right). 
\]
\end{theorem} 

\begin{theorem}[Montgomery and Vaughan]
\label{theorem_HatPi_ExtInTermsOfGz} 
Recall that for integers $k \geq 1$ and $x \geq 2$ we have defined 
$$\widehat{\pi}_k(x) := \#\{2 \leq n \leq x: \Omega(n)=k\}.$$ 
For $0 < R < 2$ we have uniformly for all $1 \leq k \leq R \log\log x$ that 
\[
\widehat{\pi}_k(x) = \frac{x}{\log x} \times \mathcal{G}\left(\frac{k-1}{\log\log x}\right) 
     \frac{(\log\log x)^{k-1}}{(k-1)!} \left(1 + O_R\left(\frac{k}{(\log\log x)^2}\right)\right), 
\]
where 
\[
\mathcal{G}(z) := \frac{1}{\Gamma(1+z)} \times 
     \prod_p \left(1-\frac{z}{p}\right)^{-1} \left(1-\frac{1}{p}\right)^z, 0 \leq |z| < R. 
\]
\end{theorem} 

\begin{remark} 
\label{remark_MV_Pikx_FuncResultsAnnotated_v1} 
We can extend the work in \cite{MV} on the distribution of $\Omega(n)$ to obtain 
corresponding analogous results for the distribution of $\omega(n)$. 
For $0 < R < 2$ we have that as $x \rightarrow \infty$ 
\begin{equation}
\label{eqn_Pikx_UniformAsymptoticsStmt_from_MV_v2} 
\pi_k(x) = \frac{x}{\log x} \times 
     \widetilde{\mathcal{G}}\left(\frac{k-1}{\log\log x}\right) 
     \frac{(\log\log x)^{k-1}}{(k-1)!} \left( 
     1 + O_R\left(\frac{k}{(\log\log x)^2}\right) 
     \right), 
\end{equation}
uniformly for any $1 \leq k \leq R\log\log x$. 
The factors of the function $\widetilde{\mathcal{G}}(z)$ used to express these bounds are 
defined by $\widetilde{\mathcal{G}}(z) := \widetilde{F}(1, z) \times \Gamma(1+z)^{-1}$ where 
\[
\widetilde{F}(s, z) := \prod_p \left(1 + \frac{z}{p^s-1}\right) \left(1 - \frac{1}{p^s}\right)^{z}, 
     \Re(s) > \frac{1}{2}, |z| \leq R < 2. 
\]
Let the functions 
\begin{align*} 
C(x, r) & := \#\{n \leq x: \omega(n) \leq r \log\log x\}, \\ 
D(x, r) & := \#\{n \leq x: \omega(n) \geq r \log\log x\}. 
\end{align*} 
We have the following upper bounds that hold as $x \rightarrow \infty$: 
\begin{align*} 
C(x, r) & \ll\phantom{_R} x (\log x)^{r - 1 - r \log r}, \text{ uniformly for } 0 < r \leq 1, \\ 
D(x, r) & \ll_R x (\log x)^{r - 1 - r \log r}, \text{ uniformly for } 1 \leq r \leq R < 2.
\end{align*} 
\end{remark} 

\section{Auxiliary sequences related to the inverse function $g^{-1}(n)$} 
\label{Section_InvFunc_PreciseExpsAndAsymptotics} 

The computational data given as Table \ref{table_conjecture_Mertens_ginvSeq_approx_values} 
in the second appendix section is intended to 
provide clear insight into the significance of the few characteristic formulas for  
$g^{-1}(n)$ proved in this section. The table provides illustrative 
numerical data by examining the first cases of $1 \leq n \leq 500$ with 
\emph{Mathematica} and \emph{SageMath} 
\cite{SCHMIDT-MERTENS-COMPUTATIONS}. 

\subsection{Definitions and properties of triangular component function sequences} 

We define the following bivariate sequence for integers $n \geq 1$ and $k \geq 0$: 
\begin{align} 
\label{eqn_CknFuncDef_v2} 
C_k(n) := \begin{cases} 
     \varepsilon(n), & \text{ if $k = 0$; } \\ 
     \sum\limits_{d|n} \omega(d) C_{k-1}\left(\frac{n}{d}\right), & \text{ if $k \geq 1$. } 
     \end{cases} 
\end{align} 
Using the more standardized definitions in \cite[\S 2]{ANT-BATEMAN-DIAMOND}, we can alternately 
identify the $k$-fold convolution of $\omega$ with itself in the following notation: 
$C_0(n) \equiv \omega^{0\ast}(n)$ and $C_k(n) \equiv \omega^{k\ast}(n)$ for 
integers $k \geq 1$ and $n \geq 1$. 
The special case of \eqref{eqn_CknFuncDef_v2} where 
$k := \Omega(n)$ occurs frequently in the next sections of the 
article. To avoid cumbersome notation when referring to this common function variant, we suppress the 
double appearance of the index $n$ by writing $C_{\Omega}(n) := C_{\Omega(n)}(n)$ instead. 

By recursively expanding the definition of $C_k(n)$ 
at any fixed $n \geq 2$, we see that 
we can form a chain of at most $\Omega(n)$ iterated (or nested) divisor sums by 
unfolding the definition of \eqref{eqn_CknFuncDef_v2} inductively. 
By the same argument, we see that at fixed $n$, the function 
$C_k(n)$ is non-zero only possibly when 
$1 \leq k \leq \Omega(n)$ whenever $n \geq 2$. 
A sequence of signed semi-diagonals of the functions $C_k(n)$ begins as follows 
\cite[\seqnum{A008480}]{OEIS}: 
\[
\{\lambda(n) C_{\Omega}(n) \}_{n \geq 1} = \{
     1, -1, -1, 1, -1, 2, -1, -1, 1, 2, -1, -3, -1, 2, 2, 1, -1, -3, -1, 
     -3, 2, 2, -1, 4, 1, 2, \ldots \}. 
\]
We see by 
\eqref{eqn_proof_tag_hInvn_ExactNestedSumFormula_CombInterpetIdent_v3} 
that $C_{\Omega}(n) \leq (\Omega(n))!$ for all $n \geq 1$ with 
equality precisely at the squarefree integers so that 
$(\Omega(n))! = (\omega(n))!$. 

\subsection{Formulas relating the unsigned $C_{\Omega}(n)$ to $g^{-1}(n)$} 
\label{subSection_Relating_CknFuncs_to_gInvn} 

\begin{remark}[Motivation for proving the next elementary results] 
The formula exactly expanding $C_{\Omega}(n)$ by finite products in 
\eqref{eqn_proof_tag_hInvn_ExactNestedSumFormula_CombInterpetIdent_v3} (using the prior 
alternate notation of $h^{-1}(n)$ for this function) shows that its values are  
determined completely by the \emph{exponents} alone 
in the prime factorization of any $n \geq 2$. 
We use the next lemma to write the inverse function $g^{-1}(n)$ we are interested in studying 
as a Dirichlet convolution of the auxiliary function, $C_{\Omega}(n)$, with the square of 
the M\"obius function, $\mu^2(n) = |\mu(n)|$. This result then allows us to see that up to 
the leading sign weight by $\lambda(n)$ on the values of this function, there is an 
essentially additive structure beneath its distinct values $g^{-1}(n)$ for $n \leq x$ 
(see Section \ref{subSection_AConnectionToDistOfThePrimes}). 
The formula that connects $g^{-1}(n)$ to the convolutions defined by $C_k(n)$ 
when $k := \Omega(n)$ in the previous 
subsection is not trivial to identify without the M\"obius inversion procedure we 
outline in the next proof. 
\end{remark}

\begin{lemma} 
\label{lemma_AnExactFormulaFor_gInvByMobiusInv_v1} 
For all $n \geq 1$, we have that 
\[
g^{-1}(n) = \sum_{d|n} \mu\left(\frac{n}{d}\right) \lambda(d) C_{\Omega}(d). 
\]
\end{lemma}
\begin{proof} 
We first expand the recurrence relation for the Dirichlet inverse 
when $g^{-1}(1) = g(1)^{-1} = 1$ as 
\begin{align} 
\label{eqn_proof_tag_gInvCvlOne_EQ_omegaCvlgInvCvl_v1} 
g^{-1}(n) & = - \sum_{\substack{d|n \\ d>1}} (\omega(d) + 1) g^{-1}\left(\frac{n}{d}\right) 
     \quad\implies\quad 
     (g^{-1} \ast 1)(n) = -(\omega \ast g^{-1})(n). 
\end{align} 
We argue that for $1 \leq m \leq \Omega(n)$, we can inductively expand the 
implication on the right-hand-side of \eqref{eqn_proof_tag_gInvCvlOne_EQ_omegaCvlgInvCvl_v1} 
in the form of $(g^{-1} \ast 1)(n) = F_m(n)$ where 
$F_m(n) := (-1)^{m} (C_m(-) \ast g^{-1})(n)$, so that 
\[
F_m(n) = - 
     \begin{cases} 
     (\omega \ast g^{-1})(n), & m = 1; \\ 
     \sum\limits_{\substack{d|n \\ d > 1}} F_{m-1}(d) \times \sum\limits_{\substack{r|\frac{n}{d} \\ r > 1}} 
     \omega(r) g^{-1}\left(\frac{n}{dr}\right), & 2 \leq m \leq \Omega(n); \\ 
     0, & \text{otherwise.} 
     \end{cases} 
\]
When $m := \Omega(n)$, i.e., with the expansions 
in the previous equation taken to a maximal depth, we obtain the relation 
\begin{equation} 
\label{eqn_proof_tag_gInvCvlOne_EQ_omegaCvlgInvCvl_v2} 
(g^{-1} \ast 1)(n) = (-1)^{\Omega(n)} C_{\Omega}(n) = \lambda(n) C_{\Omega}(n). 
\end{equation} 
The stated formula for $g^{-1}(n)$ follows from 
\eqref{eqn_proof_tag_gInvCvlOne_EQ_omegaCvlgInvCvl_v2} 
by M\"obius inversion. 
\end{proof} 

\begin{cor} 
\label{lemma_AbsValueOf_gInvn_FornSquareFree_v1} 
For all positive integers $n \geq 1$, we have that 
\begin{equation} 
\label{eqn_AbsValueOf_gInvn_FornSquareFree_v1} 
|g^{-1}(n)| = \sum_{d|n} \mu^2\left(\frac{n}{d}\right) C_{\Omega}(d). 
\end{equation} 
\end{cor} 
\begin{proof} 
By applying 
Lemma \ref{lemma_AnExactFormulaFor_gInvByMobiusInv_v1}, 
Proposition \ref{prop_SignageDirInvsOfPosBddArithmeticFuncs_v1} and the 
complete multiplicativity of $\lambda(n)$, 
we easily obtain the stated result. 
In particular, since $\mu(n)$ is non-zero only at squarefree integers and since 
at any squarefree $d \geq 1$ we have $\mu(d) = (-1)^{\omega(d)} = \lambda(d)$, 
Lemma \ref{lemma_AnExactFormulaFor_gInvByMobiusInv_v1} and 
Proposition \ref{prop_SignageDirInvsOfPosBddArithmeticFuncs_v1} imply that  
\begin{align*} 
|g^{-1}(n)| & = \lambda(n) \times \sum_{d|n} \mu\left(\frac{n}{d}\right) \lambda(d) C_{\Omega}(d) \\ 
     %& = \sum_{d|n} \mu^2\left(\frac{n}{d}\right) \lambda\left(\frac{n}{d}\right) 
     %\lambda(nd) C_{\Omega}(d) \\ 
     & = \lambda(n^2) \times \sum_{d|n} \mu^2\left(\frac{n}{d}\right) C_{\Omega}(d). 
\end{align*} 
We see that 
that $\lambda(n^2) = +1$ for all $n \geq 1$ since the number of distinct 
prime factors (counting multiplicity) of any square integer is even. 
\end{proof} 

\begin{remark} 
\label{remark_COmegann_AtSquareFreeIntegers_v3} 
The identification of an exact formula for $g^{-1}(n)$ using 
Lemma \ref{lemma_AnExactFormulaFor_gInvByMobiusInv_v1} 
implies both of the next results when $n$ is squarefree. 
It also is suggestive of more regularity beneath the 
distribution of $|g^{-1}(n)|$ which we quantify with 
precise statements in the conjectures given in 
Section \ref{subSection_ErdosKacTheorem_Analogs}. 
In particular, since $C_{\Omega}(n) = |h^{-1}(n)|$ using the original 
notation from the proof of 
Proposition \ref{prop_SignageDirInvsOfPosBddArithmeticFuncs_v1}, we can see that 
$C_{\Omega}(n) = (\omega(n))!$ for all squarefree $n \geq 1$. 
We also have that whenever $n \geq 1$ is squarefree 
\[
|g^{-1}(n)| = \sum_{d|n} C_{\Omega}(d). 
\]
Since all divisors of a squarefree integer are squarefree, 
a proof of part (B) of Proposition \ref{lemma_gInv_MxExample} 
follows by an elementary counting argument as an immediate consequence 
of the previous equation. 
\end{remark}

\begin{remark} 
Lemma \ref{lemma_AnExactFormulaFor_gInvByMobiusInv_v1} 
shows that the summatory 
function of this sequence satisfies 
\[
G^{-1}(x) = \sum_{d \leq x} \lambda(d) C_{\Omega}(d) M\left(\Floor{x}{d}\right). 
\]
Equation \eqref{eqn_AntiqueDivisorSumIdent} implies that 
$$\lambda(d) C_{\Omega}(d) = (g^{-1} \ast 1)(d) = (\chi_{\mathbb{P}} + \varepsilon)^{-1}(d).$$ 
We recover by inversion that 
\begin{equation}
\label{eqn_RmkInitialConnectionOfMxToGInvx_ProvedByInversion_v1} 
M(x) = G^{-1}(x) + \sum_{p \leq x} G^{-1}\left(\Floor{x}{p}\right), x \geq 1. 
\end{equation}
The proof of Corollary \ref{cor_ExpectationFormulaAbsgInvn_v2} (below) 
shows that 
\[
\sum_{n \leq x} |g^{-1}(n)| = \sum_{d \leq x} C_{\Omega}(d) Q\left(\Floor{x}{d}\right), x \geq 1, 
\]
where $Q(x) := \sum_{n \leq x} \mu^2(n)$ counts the number of squarefree $n \leq x$. 
\end{remark} 

\subsection{Combinatorial connections to the distribution of the primes} 
\label{subSection_AConnectionToDistOfThePrimes} 

The combinatorial properties of $g^{-1}(n)$ are deeply tied to the distribution of the primes 
$p \leq n$ as $n \rightarrow \infty$. 
The magnitudes of and spacings between the primes $p \leq n$ restricts the 
repeating of these distinct sequence values. 
We can see that the following 
is still clear about the relation of the weight functions $|g^{-1}(n)|$ to the 
distribution of the primes: 
The value of $|g^{-1}(n)|$ is entirely dependent only on the pattern of the exponents 
(viewed as multisets) of the distinct prime factors of $n \geq 2$, rather than on the 
prime factor weights themselves 
(\cf Observation \ref{heuristic_SymmetryIngInvFuncs}). 
This property shows that $|g^{-1}(n)|$ has an inherently additive, rather than 
multiplicative, structure underneath the distribution of its distinct values over $n \leq x$. 

\begin{example} 
There is a natural extremal behavior of $|g^{-1}(n)|$ 
with respect to the distinct values of $\Omega(n)$ 
at squarefree integers and prime powers. For integers 
$k \geq 1$ we define the 
infinite sets $\overline{M}_k$ and $\underline{m}_k$ to correspond to the maximal (minimal) sets of 
positive integers such that 
\begin{align*} 
\overline{M}_k & := \left\{n \geq 2: |g^{-1}(n)| = \underset{{\substack{j \geq 2 \\ \Omega(j) = k}}}{\operatorname{sup}} 
     |g^{-1}(j)|\right\} \subseteq \mathbb{Z}^{+}, \\  
\underline{m}_k & := \left\{n \geq 2: |g^{-1}(n)| = \underset{{\substack{j \geq 2 \\ \Omega(j) = k}}}{\operatorname{inf}} 
     |g^{-1}(j)|\right\} \subseteq \mathbb{Z}^{+}. 
\end{align*} 
Any element of $\overline{M}_k$ is squarefree and any element of 
$\underline{m}_k$ is a prime power. 
Moreover, for any fixed $k \geq 1$ 
we have that for any $N_k \in \overline{M}_k$ and $n_k \in \underline{m}_k$
\[
(-1)^{k} g^{-1}(N_k) = \sum_{j=0}^{k} \binom{k}{j} \times j!, 
     \quad \mathrm{\ and\ } \quad 
     (-1)^{k} g^{-1}(n_k) = 2, 
\]
where $\lambda(N_k) = \lambda(n_k) = (-1)^{k}$. 
\end{example}

\begin{remark} 
The formula for the function $h^{-1}(n) = (g^{-1} \ast 1)(n)$ defined in the proof of 
Proposition \ref{prop_SignageDirInvsOfPosBddArithmeticFuncs_v1} shows that we can express 
$g^{-1}(n)$ in terms of symmetric polynomials in the 
exponents of the prime factorization of $n$. 
For $n \geq 2$ and $0 \leq k \leq \omega(n)$ let 
\[
\widehat{e}_k(n) := [z^k] \prod_{p|n} (1 + z \nu_p(n)) = [z^k] \prod_{p^{\alpha} || n} (1 + \alpha z). 
\]
Then we can prove using 
\eqref{eqn_proof_tag_hInvn_ExactNestedSumFormula_CombInterpetIdent_v3} and 
\eqref{eqn_AbsValueOf_gInvn_FornSquareFree_v1} that the following formula holds: 
\[
g^{-1}(n) = h^{-1}(n) \times \sum_{k=0}^{\omega(n)} \binom{\Omega(n)}{k}^{-1} 
     \frac{\widehat{e}_k(n)}{k!}, n \geq 2. 
\]
The key combinatorial formula for 
$h^{-1}(n) = \lambda(n) (\Omega(n))! \times \prod_{p^{\alpha} || n} (\alpha !)^{-1}$ 
suggests additional patterns and regularity in the contributions of the distinct sign weighted 
terms in the summands of $G^{-1}(x)$\footnote{ 
     This sequence is also considered using a different motivation based on the DGFs 
     $(1\pm P(s))^{-1}$ in \cite[\S 2]{FROBERG-1968}. 
}. 
Sections \ref{subSection_AsymptoticsOfGinvx} and 
\ref{subSection_LocalCancellationOfGInvx} 
discuss limiting asymptotic properties and local cancellation in the formula for $M(x)$ 
from \eqref{eqn_RmkInitialConnectionOfMxToGInvx_ProvedByInversion_v1} 
that is expanded exactly through the auxiliary sums $G^{-1}(x)$ as above. 
\end{remark}

\section{The distributions of $C_{\Omega}(n)$ and $|g^{-1}(n)|$ and their partial sums} 
\label{Section_NewFormulasForgInvn} 

We observed an intuition in the introduction that the relation of the unsigned auxiliary 
functions, $|g^{-1}(n)|$ and $C_{\Omega}(n)$, to the canonically additive functions 
$\omega(n)$ and $\Omega(n)$ leads to the regular properties 
illustrated in Table \ref{table_conjecture_Mertens_ginvSeq_approx_values}. 
Each of $\omega(n)$ and $\Omega(n)$ satisfies 
an Erd\H{o}s-Kac type theorem that provides a central limiting 
distribution for each of these functions over 
$n \leq x$ as $x \rightarrow \infty$ 
\cite{ERDOS-KAC-REF,BILLINGSLY-CLT-PRIMEDIVFUNC,RENYI-TURAN} 
(\cf \cite{CLT-RANDOM-ORDERED-FACTS-2011}). 
In the remainder of this section, we use analytic methods, primarily in the spirit of 
\cite[\S 7.4]{MV}, to prove and conjecture new 
properties that characterize the distributions of the auxiliary functions. 

\subsection{Analytic proofs extending bivariate DGF methods involving additive functions} 
\label{subSection_Section4_AnalyticPrerequisiteProofsOfUniformBoundsOnCertainPartialSumTypes_v1} 

Theorem \ref{prop_HatAzx_ModSummatoryFuncExps_RelatedToCkn} 
proves a core bound on the partial sums of certain sign weighted arithmetic 
functions which are parameterized in the powers $z^{\Omega(n)}$ of a complex-valued 
indeterminate $z$. 
We use this bound to prove uniform asymptotics for the partial sums, 
$\sum_{n \leq x} (-1)^{\omega(n)} C_{\Omega}(n)$, uniformly 
along only those values of $n \leq x$ with 
$\Omega(n) = k$ for $1 \leq k \leq \frac{3}{2}\log\log x$ when $x$ is large in 
Theorem \ref{theorem_CnkSpCasesScaledSummatoryFuncs}. 
At the conclusion of this subsection of the article, we use an argument 
involving Abel summation with the partial sums of $\lambda_{\ast}(n) := (-1)^{\omega(n)}$ to 
turn the uniform asymptotics for the signed sums into bounds we will need on the 
corresponding unsigned sums of the same functions along $n \leq x$ such that 
$\Omega(n) = k$ for $k$ within our uniform ranges 
(see Lemma \ref{cor_AsymptoticsForSignedSumsOfomegan_v1} and the conclusion in 
Corollary \ref{cor_SummatoryFuncsOfUnsignedSeqs_v2}). 
The arguments given in the next few proofs are  
new and technical while mimicking as closely as possible the 
spirit of the proofs we cite inline from the references \cite{MV,TENENBAUM-PROBNUMT-METHODS}. 

\begin{theorem} 
\label{prop_HatAzx_ModSummatoryFuncExps_RelatedToCkn} 
Let the bivariate DGF $\widehat{F}(s, z)$ be defined in terms of the prime zeta function, $P(s)$,  
for $\Re(s) > 1$ and $|z| < |P(s)|^{-1}$ by 
\[
\widehat{F}(s, z) := \frac{1}{1+P(s) z} 
     \times \prod_p \left(1 - \frac{1}{p^s}\right)^{z}. 
\]
The partial sums of the coefficients of the DGF 
$\widehat{F}(s, z) \zeta(s)^{z}$ are given by 
\[
\widehat{A}_z(x) := \sum_{n \leq x} (-1)^{\omega(n)} 
     C_{\Omega}(n) z^{\Omega(n)}, x \geq 1. 
\]
We have for all sufficiently large $x \geq 2$ and any $|z|< P(2)^{-1} \approx 2.21118$ that
\[
\widehat{A}_z(x) = \frac{x \widehat{F}(2, z)}{\Gamma(z)} (\log x)^{z-1} + 
     O_{z}\left(x (\log x)^{\Re(z) - 2}\right). 
\]
\end{theorem} 
\begin{proof} 
It follows from \eqref{eqn_proof_tag_hInvn_ExactNestedSumFormula_CombInterpetIdent_v3} that 
we can generate exponentially scaled forms of the function $C_{\Omega}(n)$ by 
a product identity of the following form: 
\begin{align*} 
\sum_{n \geq 1} \frac{C_{\Omega}(n)}{(\Omega(n))!} \cdot 
     \frac{(-1)^{\omega(n)} z^{\Omega(n)}}{n^s} & = \prod_p \left(1 + \sum_{r \geq 1} 
     \frac{z^{\Omega(p^r)}}{r! p^{rs}}\right)^{-1} 
     = \exp\left(-z P(s)\right), \text{ for } \Re(s) > 1 \wedge \Re(P(s)z) > -1. 
\end{align*} 
This Euler type product expansion is similar in construction to the parameterized bivariate 
DGFs defined in \cite[\S 7.4]{MV} \cite[\cf \S II.6.1]{TENENBAUM-PROBNUMT-METHODS}.
By computing a termwise Laplace transform applied to the right-hand-side of the 
previous equation, we obtain that 
\begin{align*} 
\sum_{n \geq 1} \frac{C_{\Omega}(n) (-1)^{\omega(n)} z^{\Omega(n)}}{n^s} & = 
     \int_0^{\infty} e^{-t} \exp\left(-tz P(s)\right) dt = \frac{1}{1 + P(s) z}, 
     \text{ for } \Re(s) > 1 \wedge \Re(P(s)z) > -1. 
\end{align*} 
It follows from the Euler product representation of $\zeta(s)$, which is convergent for any 
$\Re(s) > 1$, that 
\[
\sum_{n \geq 1} \frac{(-1)^{\omega(n)} C_{\Omega}(n) z^{\Omega(n)}}{n^s} = 
     \widehat{F}(s, z) \zeta(s)^{z}, \text{ for } \Re(s) > 1 \wedge |z| < |P(s)|^{-1}. 
\]
The bivariate DGF $\widehat{F}(s, z)$ is an analytic function of $s$ for all $\Re(s) > 1$ 
whenever the parameter $|z| < |P(s)|^{-1}$. 
If the sequence $\{b_z(n)\}_{n \geq 1}$ indexes the coefficients in 
the DGF expansion of $\widehat{F}(s, z) \zeta(s)^{z}$, then the series 
\[
\left\lvert \sum_{n \geq 1} \frac{b_z(n) (\log n)^{2R+1}}{n^s} \right\rvert < +\infty. 
\]
Moreover, the series in the last equation is uniformly bounded for all $\Re(s) \geq 2$ and 
$|z| \leq R < |P(s)|^{-1}$. This fact follows by repeated 
termwise differentiation of the series for the original function 
$\ceiling{2R+1}$ times with respect to $s$. 

For fixed $0 < |z| < 2$, let the sequence $\{d_z(n)\}_{n \geq 1}$ be generated as the coefficients of the DGF 
$$\zeta(s)^{z} = \sum_{n \geq 1} \frac{d_z(n)}{n^s}, \text{ for } \Re(s) > 1.$$ The corresponding 
summatory function of $d_z(n)$ is defined by $D_z(x) := \sum\limits_{n \leq x} d_z(n)$. 
The theorem proved by careful contour integration in 
\cite[Thm.\ 7.17; \S 7.4]{MV} shows that for any $0 < |z| < 2$ 
and all integers $x \geq 2$ we have 
\[
D_z(x) = \frac{x (\log x)^{z-1}}{\Gamma(z)} + O_z\left(x (\log x)^{\Re(z)-2}\right). 
\]
Set 
$b_z(n) := (-1)^{\omega(n)} C_{\Omega}(n) z^{\Omega(n)}$, define the convolution 
$a_z(n) := \sum\limits_{d|n} b_z(d) d_z\left(\frac{n}{d}\right)$, 
and take its partial sums to be  
$A_z(x) := \sum\limits_{n \leq x} a_z(n)$. 
Then we have that 
\begin{align} 
\notag 
A_z(x) & = \sum_{m \leq \frac{x}{2}} b_z(m) D_z\left(\frac{x}{m}\right) + 
     \sum_{\frac{x}{2} < m \leq x} b_z(m) \\ 
\label{eqn_proof_tag_Azx_FullTermsFormulaSum_v1} 
     & = \frac{x}{\Gamma(z)} \times \sum_{m \leq \frac{x}{2}} 
     \frac{b_z(m)}{m} \log\left(\frac{x}{m}\right)^{z-1} + 
     O\left(\sum_{m \leq x} \frac{x |b_z(m)|}{m} \times
     \log\left(\frac{2x}{m}\right)^{\Re(z) - 2}\right). 
\end{align} 
We can sum the coefficients $\frac{b_z(m)}{m}$ 
for integers $m \leq u$ when $u$ is taken sufficiently large as follows: 
\begin{align*} 
\sum_{m \leq u} \frac{b_z(m)}{m^2} \times m & = \left(\widehat{F}(2, z) + 
     O_z\left(u^{-2}\right)\right) u - \int_1^{u} 
     \left(\widehat{F}(2, z) + O_z\left(t^{-2}\right)\right) dt 
     = \widehat{F}(2, z) + O_z\left(u^{-1}\right). 
\end{align*} 
Suppose that $0 < |z| \leq R < P(2)^{-1}$. 
For large $x$, the error term in \eqref{eqn_proof_tag_Azx_FullTermsFormulaSum_v1} satisfies 
\begin{align*} 
\sum_{m \leq x} \frac{x |b_z(m)|}{m} 
     \log\left(\frac{2x}{m}\right)^{\Re(z) - 2} & \ll 
     x (\log x)^{\Re(z) - 2} \times \sum_{m \leq \sqrt{x}} \frac{|b_z(m)|}{m} \\ 
     & \phantom{\ll x\ } + 
     x (\log x)^{-(R+2)} \times \sum_{m > \sqrt{x}} \frac{|b_z(m)|}{m} (\log m)^{2R}, \\ 
     & = O_z\left(x (\log x)^{\Re(z) - 2}\right), 
\end{align*} 
whenever $0 < |z| \leq R$. 
When $m \leq \sqrt{x}$ we have that 
\[
\log\left(\frac{x}{m}\right)^{z-1} = (\log x)^{z-1} + 
     O\left((\log m) (\log x)^{\Re(z) - 2}\right). 
\]
A related upper bound is obtained for the left-hand-side of the previous equation when 
$\sqrt{x} < m < x$ and $0 < |z| < R$. 
The combined sum over the interval $m \leq \frac{x}{2}$ corresponds to bounding the 
sum components when $0 < |z| \leq R$ by 
\begin{align*} 
\sum_{m \leq \frac{x}{2}} b_z(m) D_z\left(\frac{x}{m}\right) & = \frac{x}{\Gamma(z)} (\log x)^{z-1} \times 
     \sum_{m \leq \frac{x}{2}} \frac{b_z(m)}{m} \\ 
     & \phantom{=\quad\ } + 
     O_R\left(x (\log x)^{\Re(z)-2} \times \sum_{m \leq \sqrt{x}} \frac{|b_z(m)| \log m}{m} + 
     x (\log x)^{R-1} \times \sum_{m > \sqrt{x}} \frac{|b_z(m)|}{m}\right) \\ 
     & = \frac{x \widehat{F}(2, z)}{\Gamma(z)} (\log x)^{z-1} + O_R\left( 
     x (\log x)^{\Re(z)-2} \times \sum_{m \geq 1} \frac{b_z(m) (\log m)^{2R+1}}{m^2} 
     \right) \\ 
     & = \frac{x \widehat{F}(2, z)}{\Gamma(z)} (\log x)^{z-1} + O_{R}\left( 
     x (\log x)^{\Re(z)-2}\right). 
     \qedhere  
\end{align*} 
\end{proof} 

\begin{theorem} 
\label{theorem_CnkSpCasesScaledSummatoryFuncs} 
For all large $x \geq 3$ and integers $k \geq 1$, let 
\[
\widehat{C}_{k,\ast}(x) := \sum_{\substack{n \leq x \\ \Omega(n) = k}} 
     (-1)^{\omega(n)} C_k(n) 
\]
Let $\widehat{G}(z) := \widehat{F}(2, z) \times \Gamma(1+z)^{-1}$ when 
$0 \leq |z| < P(2)^{-1}$ and where $\widehat{F}(s, z)$ is defined as 
in Theorem \ref{prop_HatAzx_ModSummatoryFuncExps_RelatedToCkn}. 
As $x \rightarrow \infty$, we have uniformly for any $1 \leq k \leq 2\log\log x$ that 
\[
\widehat{C}_{k,\ast}(x) = -\widehat{G}\left(\frac{k-1}{\log\log x}\right) \frac{x}{\log x} \cdot 
     \frac{(\log\log x)^{k-1}}{(k-1)!} \left( 
     1 + O\left(\frac{k}{(\log\log x)^2}\right)\right). 
\]
\end{theorem} 
\begin{proof} 
When $k = 1$, we have that $\Omega(n) = \omega(n)$ for all $n \leq x$ such that $\Omega(n) = k$. 
The positive integers $n$ that satisfy this requirement are precisely the primes $p \leq x$. 
Hence, the formula is satisfied as 
\[
\sum_{p \leq x} (-1)^{\omega(p)} C_1(p) = -\sum_{p \leq x} 1 = 
     - \frac{x}{\log x} \left(1 + O\left(\frac{1}{\log x}\right)\right). 
\]
Since $O\left((\log x)^{-1}\right) = O\left((\log\log x)^{-2}\right)$ as 
$x \rightarrow \infty$, we obtain the required error term for the bound at $k = 1$. 

For $2 \leq k \leq 2\log\log x$, we will apply the error estimate from 
Theorem \ref{prop_HatAzx_ModSummatoryFuncExps_RelatedToCkn} with 
$r := \frac{k-1}{\log\log x}$ in the formula 
\[
\widehat{C}_{k,\ast}(x) = \frac{(-1)^{k+1}}{2\pi\imath} \times \int_{|v|=r} 
     \frac{\widehat{A}_{-v}(x)}{v^{k+1}} dv. 
\]
The error in the formula 
contributes terms that are bounded by 
\begin{align*} 
\left\lvert x (\log x)^{-(\Re(v)+2)} v^{-(k+1)} \right\rvert & \ll 
     \left\lvert x (\log x)^{-(r+2)} r^{-(k+1)} \right\rvert 
     \ll \frac{x}{(\log x)^{2-\frac{k-1}{\log\log x}}} \cdot 
     \frac{(\log\log x)^{k}}{(k-1)^{k}} \\ 
     & \ll \frac{x}{(\log x)^2} \cdot \frac{(\log\log x)^{k+1}}{(k-1)^{\frac{1}{2}} (k-1)!} 
     \ll \frac{x}{\log x} \cdot \frac{k (\log\log x)^{k-5}}{(k-1)!}, 
     \text{ as } x \rightarrow \infty. 
\end{align*} 
We next find the main term for the coefficients 
of the following contour integral when 
$r \in [0, z_{\max}] \subseteq \left[0, P(2)^{-1}\right)$: 
\begin{align} 
\label{eqn_WideTildeArx_CountourIntDef_v1} 
\widehat{C}_{k,\ast}(x) \sim  
     \frac{(-1)^{k} x}{\log x} 
     \times \int_{|v|=r} \frac{(\log x)^{-v} \zeta(2)^{v}}{\Gamma(1 - v) 
     v^{k} (1 - P(2) v)} dv. 
\end{align} 
The main term of $\widehat{C}_{k,\ast}(x)$ 
is given by $-\frac{x}{\log x} \times I_k(r, x)$, where we define 
\begin{align*}
I_k(r, x) & = \frac{1}{2\pi\imath} \times \int_{|v|=r} 
     \frac{\widehat{G}(v) (\log x)^{v}}{v^k} dv \\ 
     & =: I_{1,k}(r, x) + I_{2,k}(r, x). 
\end{align*}
Taking $r = \frac{k-1}{\log\log x}$, the 
first of the component integrals is defined to be 
\begin{align*}
I_{1,k}(r, x) & := \frac{\widehat{G}(r)}{2\pi\imath} \times \int_{|v|=r} 
     \frac{(\log x)^{v}}{v^k} dv = \widehat{G}(r) \times \frac{(\log\log x)^{k-1}}{(k-1)!}. 
\end{align*}
The second integral, $I_{2,k}(r, x)$, corresponds to another error term in our approximation. 
This component function is defined by 
\[
I_{2,k}(r, x) := \frac{1}{2\pi\imath} \times \int_{|v|=r} 
     \left(\widehat{G}(v) - \widehat{G}(r)\right) 
     \frac{(\log x)^{v}}{v^k} dv. 
\]
Integrating by parts shows that \cite[\cf Thm.\ 7.19; \S 7.4]{MV} 
\[
\frac{(r-v)}{2\pi\imath} \times \int_{|v|=r} (\log x)^v v^{-k} dv = 0, 
\]
so that integrating by parts once again we have 
\[
I_{2,k}(r, x) := \frac{1}{2\pi\imath} \times \int_{|v|=r} 
     \left(\widehat{G}(v) - \widehat{G}(r) - 
     \widehat{G}^{\prime}(r)(v-r)\right) 
     (\log x)^{v} v^{-k} dv. 
\]
We find that 
\[
\widehat{G}(v) - \widehat{G}(r) - \widehat{G}^{\prime}(r)(v-r) = 
     \int_{r}^{v} (v-w) \widehat{G}^{\prime\prime}(w) dw 
     \ll |v-r|^2. 
\]
With the parameterization $v = re^{2\pi\imath\theta}$ for 
$\theta \in \left[-\frac{1}{2}, \frac{1}{2}\right]$ 
(again selecting $r := \frac{k-1}{\log\log x}$), we obtain 
\[
|I_{2,k}(r, x)| \ll r^{3-k} \times 
     \int_{-\frac{1}{2}}^{\frac{1}{2}} (\sin \pi\theta)^2 e^{(k-1) \cos(2\pi\theta)} d\theta. 
\]
Since $|\sin x| \leq |x|$ for all $|x| < 1$ and $\cos(2\pi\theta) \leq 1 - 8\theta^2$ if 
$-\frac{1}{2} \leq \theta \leq \frac{1}{2}$, we arrive at the next bounds 
at any $1 \leq k \leq 2\log\log x$ when $r = \frac{k-1}{\log\log x}$. 
\begin{align*}
|I_{2,k}(r, x)| & \ll r^{3-k} e^{k-1} \times \int_0^{\infty} \theta^2 e^{-8(k-1) \theta^2} d\theta \\ 
     & \ll \frac{r^{3-k} e^{k-1}}{(k-1)^{\frac{3}{2}}} = 
     \frac{(\log\log x)^{k-3} e^{k-1}}{(k-1)^{k-\frac{3}{2}}} 
     \ll 
     \frac{k (\log\log x)^{k-3}}{(k-1)!}. 
\end{align*}
Finally, whenever $1 \leq k \leq 2\log\log x$ we have 
\[
1 = \widehat{G}(0) \geq \widehat{G}\left(\frac{k-1}{\log\log x}\right) = 
     \frac{1}{\Gamma\left(1+\frac{k-1}{\log\log x}\right)} \times 
     \frac{\zeta(2)^{\frac{1-k}{\log\log x}}}{\left(1+\frac{P(2)(k-1)}{\log\log x}\right)} 
     \geq \widehat{G}(2) \approx 0.097027. 
\]
In particular, the function 
$\widehat{G}\left(\frac{k-1}{\log\log x}\right) \gg 1$ for 
all $1 \leq k \leq 2\log\log x$. 
This in turn implies the result of the theorem. 
\end{proof} 

\begin{lemma} 
\label{cor_AsymptoticsForSignedSumsOfomegan_v1}
As $x \rightarrow \infty$, there is an absolute constant $A_0 > 0$ such that 
\[
L_{\omega}(x) := \sum_{n \leq x} (-1)^{\omega(n)} = 
     \frac{(-1)^{\floor{\log\log x}} x}{A_0 \sqrt{2\pi \log\log x}} + 
     O\left(\frac{x}{\log\log x}\right). 
\]
\end{lemma}
\begin{proof}
An adaptation of the proof of Lemma \ref{lemma_ConvenientIncGammaFuncTypePartialSumAsymptotics_v2} 
from the appendix provides that for any $a \in (1, 1.76321) \subset \left(1, W(1)^{-1}\right)$ 
\begin{align}
\notag 
S_a(x) := 
     \frac{x}{\log x} & \times \left\lvert \sum_{k=1}^{\floor{a\log\log x}} \frac{(-1)^{k} (\log\log x)^{k-1}}{(k-1)!} 
     \right\rvert \\ 
\label{eqn_ConvenientIncGammaFuncTypePartialSumAsymptotics_va3} 
     & = \frac{\sqrt{a} x}{\sqrt{2\pi}(a+1) a^{\{a\log\log x\}}} 
     \times \frac{(\log x)^{a-1-a\log a}}{\sqrt{\log\log x}} 
     \left(1 + O\left(\frac{1}{\log\log x}\right)\right). 
\end{align}
Here, we define $\{x\} = x - \floor{x} \in [0, 1)$ to be the \emph{fractional part} of $x$. 
Suppose that we take $a := \frac{3}{2}$ so that $a-1-a\log a \approx -0.108198$. 
We can then define and expand the next partial sums as 
\begin{align*}
L_{\omega}(x) := & \sum_{n \leq x} (-1)^{\omega(n)} = 
     \sum_{k \leq \log\log x} 2 (-1)^{k} \pi_k(x) + 
     O\left(S_{\frac{3}{2}}(x) + 
     \#\left\{n \leq x: \omega(n) \geq \frac{3}{2}\log\log x\right\}\right). 
\end{align*} 
The justification for the error term including $S_{\frac{3}{2}}(x)$ is that for 
$1 \leq k < \frac{3}{2}\log\log x$, we can show that 
$\widetilde{\mathcal{G}}\left(\frac{k-1}{\log\log x}\right) \asymp 1$ where the function 
$\widetilde{\mathcal{G}}\left(\frac{k-1}{\log\log x}\right)$ is monotone for $k$ within 
each of the two disjoint intervals $[1, \log\log x] \bigcup \left(\log\log x, \frac{3}{2}\log\log x\right]$. 
Moreover, we can show that for any $1 < k \leq \log\log x$, 
the function $\widetilde{\mathcal{G}}\left(\frac{k-1}{\log\log x}\right)$ from 
Remark \ref{remark_MV_Pikx_FuncResultsAnnotated_v1} is decreasing in $k$ for $1 \leq k \leq \log\log x$ 
with $\widetilde{\mathcal{G}}(0) = 1$. It also satisfies the 
following inequalities for $k$ taken within the same range:  
\[ 
\widetilde{\mathcal{G}}\left(\frac{k-1}{\log\log x}\right) \geq 
     \widetilde{\mathcal{G}}\left(1-\frac{1}{\log\log x}\right) \geq 
     \widetilde{\mathcal{G}}(1) = 1. 
\]
We apply the uniform asymptotics for $\pi_k(x)$ that hold as $x \rightarrow \infty$ when 
$1 \leq k \leq R \log\log x$ for $1 \leq R < 2$ from 
Remark \ref{remark_MV_Pikx_FuncResultsAnnotated_v1}. We then see 
by Lemma \ref{lemma_ConvenientIncGammaFuncTypePartialSumAsymptotics_v2} 
and \eqref{eqn_ConvenientIncGammaFuncTypePartialSumAsymptotics_va3} 
that for all sufficiently large $x$ 
there is some absolute constant $A_0 > 0$ such that 
\begin{align*}
L_{\omega}(x) & = \frac{(-1)^{\floor{\log\log x}} x}{A_0 \sqrt{2\pi \log\log x}} + 
     O\left(E_{\omega}(x) + 
     \frac{x}{(\log x)^{0.108198} \sqrt{\log\log x}} + 
     \#\left\{n \leq x: \omega(x) \geq \frac{3}{2}\log\log x\right\}\right). 
\end{align*} 
The error term in the previous equation 
is bounded by the next sum as $x \rightarrow \infty$. 
In particular, the following estimate is obtained from Stirling's formula, and 
equations \eqref{eqn_IncompleteGamma_PropA} and 
\eqref{eqn_IncompleteGamma_PropC} from the appendix: 
\begin{align*} 
E_{\omega}(x) & \ll \frac{x}{\log x} \times 
     \sum_{1 \leq k \leq \log\log x} \frac{(\log\log x)^{k-2}}{(k-1)!} \\ 
     & = 
     \frac{x \Gamma(\log\log x, \log\log x)}{\Gamma(\log\log x + 1)} 
     \sim \frac{x}{2\log\log x} \left(1 + O\left(\frac{1}{\sqrt{\log\log x}}\right)\right). 
\end{align*}
By an application of the second set of results in 
Remark \ref{remark_MV_Pikx_FuncResultsAnnotated_v1}, we finally see that 
\[
\#\left\{n \leq x: \omega(x) \geq \frac{3}{2}\log\log x\right\} \ll 
     \frac{x}{(\log x)^{0.108198}}. 
     \qedhere 
\] 
Hence, we have obtained a correct main and error term on the partial sums $L_{\omega}(x)$. 
\end{proof}

\begin{cor} 
\label{cor_SummatoryFuncsOfUnsignedSeqs_v2} 
We have uniformly for $1 \leq k \leq \frac{3}{2} \log\log x$ 
that at all sufficiently large $x$ 
\begin{align*} 
\widehat{C}_k(x) := 
     \sum_{\substack{n \leq x \\ \Omega(n) = k}} C_{\Omega}(n) & 
     = A_0 \sqrt{2\pi} x 
     \widehat{G}\left(\frac{k-1}{\log\log x}\right) 
     \frac{(\log\log x)^{k-\frac{1}{2}}}{(k-1)!} \left( 
     1 + O\left(\frac{1}{\log\log x}\right)\right). 
\end{align*} 
\end{cor} 
\begin{proof} 
Suppose that $\hat{h}(t)$ and $\sum_{n \leq t} \lambda_{\ast}(n)$ are 
piecewise smooth and differentiable functions of $t$ on $\mathbb{R}^{+}$. 
The next integral formulas result by 
Abel summation and integration by parts. 
\begin{subequations}
\begin{align} 
\label{eqn_AbelSummationIBPReverseFormula_stmt_v1} 
     \sum_{n \leq x} \lambda_{\ast}(n) \hat{h}(n) & = 
     \left(\sum_{n \leq x} \lambda_{\ast}(n)\right) \hat{h}(x) - 
     \int_{1}^{x} \left(\sum_{n \leq t} \lambda_{\ast}(n)\right) \hat{h}^{\prime}(t) dt \\ 
\label{eqn_AbelSummationIBPReverseFormula_stmt_v2}
     & \sim 
     \int_1^{x} \frac{d}{dt}\left[\sum_{n \leq t} \lambda_{\ast}(n)\right] \hat{h}(t) dt
\end{align} 
\end{subequations}
We transform our previous results for the partial sums of 
$(-1)^{\omega(n)} C_{\Omega}(n)$ such that $\Omega(n) = k$ from 
Theorem \ref{theorem_CnkSpCasesScaledSummatoryFuncs} to approximate 
the corresponding partial sums of only the unsigned function $C_{\Omega}(n)$. 
In particular, since $1 \leq k \leq \frac{3}{2} \log\log x$, we have that 
\[
\widehat{C}_{k,\ast}(x) = 
     \sum_{\substack{n \leq x \\ \Omega(n)=k}} (-1)^{\omega(n)} C_{\Omega}(n) = 
     \sum_{n \leq x} (-1)^{\omega(n)} \Iverson{\omega(n) \leq \frac{3}{2} \log\log x} \times 
     C_{\Omega}(n) \Iverson{\Omega(n) = k}. 
\]
By the proof of Lemma \ref{cor_AsymptoticsForSignedSumsOfomegan_v1}, we have 
that as $t \rightarrow \infty$ 
\begin{align} 
\label{eqn_ProofTag_LAsttSummatoryFuncAsymptotics_v1}
L_{\ast}(t) & := \sum_{\substack{n \leq t \\ \omega(n) \leq \frac{3}{2} \log\log t}} 
     (-1)^{\omega(n)} 
     = \frac{(-1)^{\floor{\log\log t}} t}{A_0 \sqrt{2\pi \log\log t}}\left(1 + 
     O\left(\frac{1}{\sqrt{\log\log t}}\right)\right). 
\end{align} 
Except for $t$ within a subset of $(0, \infty)$ of measure zero on which 
$L_{\ast}(t)$ changes sign, the main term of the derivative of this summatory function 
is approximated almost everywhere by 
\[
L_{\ast}^{\prime}(t) \sim \frac{(-1)^{\floor{\log\log t}}}{A_0 \sqrt{2\pi \log\log t}}. 
\]
We apply the formula from \eqref{eqn_AbelSummationIBPReverseFormula_stmt_v2},  
to deduce that as $x \rightarrow \infty$ whenever $1 \leq k \leq \frac{3}{2} \log\log x$ 
\begin{align*} 
     \widehat{C}_{k,\ast}(x) & \sim 
     \sum_{j=1}^{\log\log x-1} \frac{2 \cdot (-1)^{j+1}}{A_0\sqrt{2\pi}} \times \int_{e^{e^j}}^{e^{e^{j+1}}} 
     \frac{C_{\Omega(t)}(t) \Iverson{\Omega(t) = k}}{\sqrt{\log\log t}} dt \\ 
     & \sim -\int_1^{\frac{\log\log x}{2}} \int_{e^{e^{2s-1}}}^{e^{e^{2s}}} 
     \frac{2 C_{\Omega(t)}(t) \Iverson{\Omega(t) = k}}{A_0 \sqrt{2\pi \log\log t}} dt ds +
     \frac{1}{A_0 \sqrt{2\pi}} \times \int_{e^e}^x 
     \frac{C_{\Omega(t)}(t) \Iverson{\Omega(t) = k}}{\sqrt{\log\log t}} dt. 
\end{align*} 
For large $x$, $(\log\log t)^{-\frac{1}{2}}$ is continuous and monotone decreasing on 
$\left[x^{e^{-1}}, x\right]$ with 
\[
\frac{1}{\sqrt{\log\log x}} - \frac{1}{\sqrt{\log\log\left(x^{e^{-1}}\right)}} = 
     O\left(\frac{1}{(\log x) \sqrt{\log\log x}}\right), 
\]
Hence, we have that 
\begin{equation} 
\label{eqn_ProofTag_HatCkx_Asymptotics_v1_v0}
     -A_0 \sqrt{2\pi} x (\log x) \sqrt{\log\log x} \widehat{C}_{k,\ast}^{\prime}(x) = 
     \left(\widehat{C}_k(x) - \widehat{C}_k\left(x^{e^{-1}}\right)\right)(1+o(1)) - 
     x (\log x) \widehat{C}_k^{\prime}(x). 
\end{equation} 
For $1 \leq k < \frac{3}{2} \log\log x$, we expect contributions from the squarefree integers $n \leq x$ 
such that $\omega(n) = \Omega(n) = k$ to be on the order of 
\[
\widehat{C}_k^{\prime}(x) \gg \widehat{\pi}_k(x) \asymp 
     \frac{x}{\log x} \times \frac{(\log\log x)^{k-1}}{(k-1)!}. 
\]
The argument used to justify the last equation is that 
\[
\widehat{C}_k^{\prime}(x) \gg 
     \int_{n-1}^{n} \frac{d}{dt} \widehat{C}_k(t) dt \gg \sum_{n \leq x} \Iverson{\Omega(n) = k}. 
\]
We conclude that 
$\widehat{C}_k\left(x^{e^{-1}}\right) = o\left(\widehat{C}_k(x)\right)$ at sufficiently large $x$. 
Equation \eqref{eqn_ProofTag_HatCkx_Asymptotics_v1_v0} 
becomes an ordinary differential equation for $\widehat{C}_k(x)$ after this 
observation. Its solution has the form 
\[
\widehat{C}_k(x) = A_0\sqrt{2\pi}(\log x) \times \int_3^x 
     \frac{\sqrt{\log\log t}}{\log t} \widehat{C}_{k,\ast}^{\prime}(t) dt + 
     O(\log x). 
\]
When we integrate by parts and apply the result from 
Theorem \ref{theorem_CnkSpCasesScaledSummatoryFuncs}, we find that 
\begin{align*}
\widehat{C}_k(x) & = \frac{\sqrt{\log\log x}}{\log x} \widehat{C}_{k,\ast}(x) + 
     O\left(x \times \int_3^x \frac{\sqrt{\log\log t} \widehat{C}_{k,\ast}(t)}{t^2 (\log t)^2} dt\right) \\ 
     & = \frac{\sqrt{\log\log x}}{\log x} \widehat{C}_{k,\ast}(x) + 
     O\left(\frac{x}{2^k} \times \Gamma\left(k+\frac{1}{2}, 2\log\log x\right)\right). 
\end{align*} 
Finally, whenever we assume that $1 \leq k \leq \frac{3}{2} \log\log x$ such that $\lambda > 1$ in 
Proposition \ref{prop_IncGammaLambdaTypeBounds_v1} 
(\cf Facts \ref{facts_ExpIntIncGammaFuncs} for $k$ of 
substantially lesser order in $x$ than this upper bound), 
Theorem \ref{theorem_CnkSpCasesScaledSummatoryFuncs} 
implies the conclusion of our corollary. 
\end{proof}

\subsection{Average orders of the unsigned sequences}
\label{subSection_AvgOrdersOfTheUnsignedSequences} 

In the next subsection 
(see Section \ref{subSection_ErdosKacTheorem_Analogs}), 
we conjecture that there are explicitly defined 
probability measures that underlie the distributions of the distinct values of 
the functions $C_{\Omega}(n)$ and $|g^{-1}(n)|$ for $n \leq x$ as $x \rightarrow \infty$. 
These results rely on asymptotics for the first moments, e.g., the 
respective average orders, of these two functions. 
We prove asymptotic formulae for the main and error terms of the average order of these 
two key unsigned sequences within this subsection. 
Namely, we state and prove the results in 
Proposition \ref{lemma_HatCAstxSum_ExactFormulaWithError_v1} and 
Corollary \ref{cor_ExpectationFormulaAbsgInvn_v2} below. 
The proof of the former proposition requires the uniform asymptotics we proved in 
Section \ref{subSection_Section4_AnalyticPrerequisiteProofsOfUniformBoundsOnCertainPartialSumTypes_v1} 
along with an adaptation of Rankin's method from \cite[\S 7.4]{MV} to 
bound error terms for partial sums taken in the ranges of $n \leq x$ for $\Omega(n) = k$ 
outside of the uniform ranges for $k$. 

\begin{prop} 
\label{lemma_HatCAstxSum_ExactFormulaWithError_v1} 
There is an absolute constant $B_0 > 0$ such that 
as $n \rightarrow \infty$ 
\[
\frac{1}{n} \times \sum_{k \leq n} C_{\Omega}(k) = 
B_0 (\log n) \sqrt{\log\log n}\left(1 + O\left(\frac{1}{\log\log n}\right)\right). 
\] 
\end{prop} 
\begin{proof} 
By Corollary \ref{cor_SummatoryFuncsOfUnsignedSeqs_v2} and 
Proposition \ref{prop_IncGammaLambdaTypeBounds_v1} 
when $\lambda = \frac{2}{3}$, we have that 
\begin{align} 
\notag 
\sum_{k=1}^{\frac{3}{2} \log\log x} \sum_{\substack{n \leq x \\ \Omega(n) = k}} C_{\Omega}(n) & \asymp 
     \sum_{k=1}^{\frac{3}{2} \log\log x} \frac{x (\log\log x)^{k-\frac{1}{2}}}{(k-1)!} 
     \left(1 + O\left(\frac{1}{\log\log x}\right)\right) \\ 
\notag 
     & = \frac{x (\log x) \sqrt{\log\log x} \Gamma\left(\frac{3}{2} \log\log x, \log\log x\right)}{ 
     \Gamma\left(\frac{3}{2} \log\log x\right)} 
     \left(1 + O\left(\frac{1}{\log\log x}\right)\right) \\ 
\notag 
     & = 
     x(\log x)\sqrt{\log\log x} \left(1 + O\left(\frac{1}{\log\log x}\right)\right). 
\end{align}
For real $0 \leq z \leq 2$, the function $\widehat{G}(z)$ is monotone in 
$z$ with $\widehat{G}(0) = 1$ and $\widehat{G}(2) \approx 0.303964$. 
Then we see that there is an absolute constant $B_0 > 0$ such that 
\begin{align} 
\notag 
\frac{1}{x} \times \sum_{k=1}^{\frac{3}{2} \log\log x} 
     \sum_{\substack{n \leq x \\ \Omega(n) = k}} C_{\Omega}(n) & = 
     B_0 (\log x) \sqrt{\log\log x} \left(1 + O\left(\frac{1}{\log\log x}\right)\right). 
\end{align} 
We claim that 
\begin{align} 
\notag 
\frac{1}{x} \times \sum_{n \leq x} C_{\Omega}(n) & = \frac{1}{x} \times 
     \sum_{k \geq 1} \sum_{\substack{n \leq x \\ \Omega(n) = k}} C_{\Omega}(n) \\ 
\notag 
     & = 
     \frac{1}{x} \times \sum_{k=1}^{\frac{3}{2} \log\log x} 
     \sum_{\substack{n \leq x \\ \Omega(n) = k}} 
     C_{\Omega}(n) (1+o(1)), 
     \text{ as } x \rightarrow \infty. 
\end{align} 
To prove the claim it suffices to show that 
\begin{equation} 
\label{eqn_proof_tag_PartialSumsOver_HatCkx_EquivCond_v2} 
\frac{1}{x} \times 
     \sum\limits_{\substack{n \leq x \\ \Omega(n) \geq \frac{3}{2} \log\log x}} C_{\Omega}(n)
     = o\left((\log x) \sqrt{\log\log x}\right). 
\end{equation} 
We proved in Theorem \ref{prop_HatAzx_ModSummatoryFuncExps_RelatedToCkn} 
that for all sufficiently large $x$ and $|z| < P(2)^{-1}$ 
\[
\sum_{n \leq x} (-1)^{\omega(n)} C_{\Omega}(n) z^{\Omega(n)} = 
     \frac{x \widehat{F}(2, z)}{\Gamma(z)} (\log x)^{z-1} + O\left( 
     x (\log x)^{\Re(z)-2}\right). 
\]
By Lemma \ref{cor_AsymptoticsForSignedSumsOfomegan_v1}, 
we have that the summatory function 
\[
\sum_{n \leq x} (-1)^{\omega(n)} = 
     \frac{(-1)^{\floor{\log\log x}} x}{A_0 \sqrt{2\pi \log\log x}} 
     \left(1+O\left(\frac{1}{\sqrt{\log\log x}}\right)\right), 
\]
where $\frac{d}{dx}\left[\frac{x}{\sqrt{\log\log x}}\right] = \frac{1}{\sqrt{\log\log x}} + o(1)$. 
We can argue as in the proof of Corollary \ref{cor_SummatoryFuncsOfUnsignedSeqs_v2} 
that whenever $0 < |z| < P(2)^{-1}$ with $x$ sufficiently large we have 
\begin{align}
\notag
\sum_{n \leq x} C_{\Omega}(n) z^{\Omega(n)} & \ll 
     \frac{\widehat{F}(2, z) x (\log x) \sqrt{\log\log x}}{\Gamma(z)} \times 
     \frac{\partial}{\partial x}\left[x (\log x)^{z-1}\right] \\ 
\label{eqn_COmegannzPowOmeganLLRelation_v1} 
     & \ll 
     \frac{\widehat{F}(2, z) x \sqrt{\log\log x}}{\Gamma(z)} (\log x)^{z}. 
\end{align}
For large $x$ and any fixed $0 < r < P(2)^{-1}$, we define 
\[
\widehat{B}(x, r) := \sum_{\substack{n \leq x \\ \Omega(n) \geq r\log\log x}} 
     C_{\Omega}(n). 
\]
We adapt the proof from the reference \cite[\cf Thm.\ 7.20; \S 7.4]{MV} by 
applying \eqref{eqn_COmegannzPowOmeganLLRelation_v1} when $1 \leq r < P(2)^{-1}$. 
Since $r \widehat{F}(2, r) = \frac{r \zeta(2)^{-r}}{1+P(2)r} \ll 1$ 
for $r \in [1, P(2)^{-1})$, and similarly since we have that 
$\frac{1}{\Gamma(1+r)} \gg 1$ for $r$ within the same range, 
we find that 
\[
x \sqrt{\log\log x} (\log x)^{r} \gg \sum_{\substack{n \leq x \\ \Omega(n) \geq r\log\log x}} 
     C_{\Omega}(n) r^{\Omega(n)} \gg 
     \sum_{\substack{n \leq x \\ \Omega(n) \geq r\log\log x}} 
     C_{\Omega}(n) r^{r \log\log x}. 
\]
This implies that for $r := \frac{3}{2}$ we have 
\begin{equation}
\label{eqn_BHatxrUpperBound_v1}
\widehat{B}(x, r) \ll x (\log x)^{r-r\log r} \sqrt{\log\log x} = 
     O\left(x (\log x)^{0.891802} \sqrt{\log\log x}\right)
\end{equation}
We evaluate the limiting asymptotics of the sums 
\begin{align*}
S_2(x) & := \frac{1}{x} \times 
     \sum_{k \geq \frac{3}{2} \log\log x} \sum_{\substack{n \leq x \\ \Omega(n)=k}} 
     C_{\Omega}(n) \ll \frac{1}{x} \times \widehat{B}\left(x, \frac{3}{2}\right) = 
     O\left((\log x)^{0.891802} \sqrt{\log\log x}\right), 
     \text{ as } x \rightarrow \infty. 
\end{align*} 
This implies that 
\eqref{eqn_proof_tag_PartialSumsOver_HatCkx_EquivCond_v2} holds. 
\end{proof} 

\begin{cor}
\label{cor_ExpectationFormulaAbsgInvn_v2} 
We have that as $n \rightarrow \infty$ 
\begin{align*} 
\frac{1}{n} \times \sum_{k \leq n} |g^{-1}(k)| & = 
     \frac{6B_0 (\log n)^2 \sqrt{\log\log n}}{\pi^2} 
     \left(1 + O\left(\frac{1}{\log\log n}\right)\right). 
\end{align*} 
\end{cor} 
\begin{proof} 
As $|z| \rightarrow \infty$, the \emph{imaginary error function}, 
$\operatorname{erfi}(z)$, has the following asymptotic expansion 
\cite[\S 7.12]{NISTHB}: 
\begin{equation}
\label{eqn_Erfix_KnownAsymptoticSeries_v1}
\operatorname{erfi}(z) := \frac{2}{\sqrt{\pi} \imath} \times \int_0^{\imath z} e^{t^2} dt = 
     \frac{e^{z^2}}{\sqrt{\pi}} \left(\frac{1}{z} + \frac{1}{2z^3} + 
     \frac{3}{4z^5} + \frac{15}{8z^7} + O\left(\frac{1}{z^{9}}\right)\right). 
\end{equation}
We use the formula from Proposition \ref{lemma_HatCAstxSum_ExactFormulaWithError_v1} 
to sum the average order of $C_{\Omega}(n)$.
The proposition and error terms obtained from \eqref{eqn_Erfix_KnownAsymptoticSeries_v1} 
imply that for all sufficiently large $t \rightarrow \infty$ 
\begin{align*} 
\int \frac{\sum_{n \leq t} C_{\Omega}(n)}{t^2} dt & = 
     B_0 (\log t)^2 \sqrt{\log\log t} - \frac{1}{4} \sqrt{\frac{\pi}{2}} 
     \operatorname{erfi}\left(\sqrt{2\log\log t}\right) \\ 
     & = 
     B_0 (\log t)^2 \sqrt{\log\log t} \left(1 + O\left(\frac{1}{\log\log t}\right)\right). 
\end{align*} 
A classical formula for the summatory function that counts the 
number of \emph{squarefree} integers $n \leq x$ shows that this function satisfies 
\cite[\S 18.6]{HARDYWRIGHT} \cite[\seqnum{A013928}]{OEIS} 
\[ 
Q(x) = \sum_{n \leq x} \mu^2(n) = \frac{6x}{\pi^2} + O\left(\sqrt{x}\right), 
     \text{\ as $x \rightarrow \infty$}. 
\]
Therefore, summing over the formula from 
\eqref{eqn_AbsValueOf_gInvn_FornSquareFree_v1} in 
Section \ref{subSection_Relating_CknFuncs_to_gInvn}, we find that  
\begin{align} 
\notag 
\frac{1}{n} \times \sum_{k \leq n} |g^{-1}(k)| & = \frac{1}{n} \times \sum_{d \leq n} 
     C_{\Omega}(d) Q\left(\Floor{n}{d}\right) \\ 
\notag 
     & \sim \sum_{d \leq n} C_{\Omega}(d) \left[\frac{6}{d \cdot \pi^2} + O\left(\frac{1}{\sqrt{dn}}\right) 
     \right] \\ 
\notag 
     & = \frac{6}{\pi^2} \left[\frac{1}{n} \times \sum_{k \leq n} C_{\Omega}(k) + \sum_{d<n} 
     \sum_{k \leq d} \frac{C_{\Omega}(k)}{d^2}\right] + O(1). 
\end{align} 
The latter sum in the previous equation forms the main term. 
\end{proof} 

\subsection{Erd\H{o}s-Kac theorem analogs for the distributions of the unsigned functions} 
\label{subSection_ErdosKacTheorem_Analogs} 

It is not difficult to prove that 
\begin{equation*} 
%\label{eqn_ProofTag_AnsatzRVDiffFromMeanBound_v0} 
\sum_{\substack{n \leq x \\ \Omega(n)=k}} 
     \frac{C_{\Omega}(n)}{(\log n) \sqrt{\log\log n}} = 
     \frac{A_0 \sqrt{2\pi} x}{\log x} \times \widehat{G}\left(\frac{k-1}{\log\log x}\right) 
     \frac{(\log\log x)^{k-1}}{(k-1)!}\left(1 + O\left(\frac{1}{\log\log x}\right)\right), 
     \text{ as } x \rightarrow \infty
\end{equation*} 
A modified set of proof mechanics that draw upon the methods in 
\cite[Thm.~7.21; \S 7.4]{MV} suggest that the first result in (A) of the 
next conjecture should hold. 
The average order of $C_{\Omega}(n)$ is given by 
Proposition \ref{lemma_HatCAstxSum_ExactFormulaWithError_v1}. 
We can show that there is an absolute constant $D_0 > 0$ such that the 
variance type sums 
\begin{align*}
\frac{1}{n} \times \left(\sum_{k \leq n} C_{\Omega}(k)^2 - 
     \left(\sum_{k \leq n} C_{\Omega}(k)\right)^2\right) & = 
     \frac{2}{n} \times \sum_{1 \leq j < k \leq n} C_{\Omega}(j) C_{\Omega}(k), \\ 
     & = 
     D_0^2 n (\log n)^2 (\log\log n) (1 + o(1)), \text{ as } n \rightarrow \infty. 
\end{align*}
This perspective leads to the second conjectured result in part (B) below. 

\begin{conjecture}[Deterministic form of the Erd\H{o}s-Kac theorem analog for $C_{\Omega}(n)$]
\label{conj_DetFormOfEKTypeThmForCOmegan_v1} 
For sufficiently large $x$, let the mean and variance parameter analogs be defined by 
\[
\mu_x(C) := \log\log x - 
     \log\left(\sqrt{2\pi}A_0 \widehat{G}(1)\right), 
     \quad \mathrm{\ and\ } \quad 
     \sigma_x(C) := \sqrt{\log\log x}. 
\]
We have for any $z \in (-\infty, +\infty)$ that 
\begin{equation} 
\tag{A} 
\frac{1}{x} \times \#\left\{2 \leq n \leq x: 
     \frac{\frac{C_{\Omega}(n)}{(\log n)\sqrt{\log\log n}} - 
     \mu_x(C)}{\sigma_x(C)} \leq z\right\} = 
     \Phi\left(z\right) + o(1), 
     \mathrm{\ as\ } x \rightarrow \infty. 
\end{equation} 
Similarly, for any real $z$ we have that 
\begin{equation} 
\tag{B} 
\frac{1}{x} \times \#\left\{2 \leq n \leq x: 
     \frac{C_{\Omega}(n) - B_0(\log x)\sqrt{\log\log x}}{ 
     D_0\sqrt{x}(\log x)\sqrt{\log\log x}} \leq z\right\} = 
     \Phi\left(z\right) + o(1), 
     \mathrm{\ as\ } x \rightarrow \infty
\end{equation}
\end{conjecture} 

\begin{cor} 
\label{cor_CLT_VII} 
Suppose that Conjecture \ref{conj_DetFormOfEKTypeThmForCOmegan_v1} is true and that 
$\mu_x(C)$ and $\sigma_x(C)$ are defined as in 
the conjecture for sufficiently large $x$. 
Let $Y > 0$. 
We have uniformly for all $-Y \leq y \leq Y$ 
that as $x \rightarrow \infty$ 
\begin{align*} 
\frac{1}{x} \times \#\left\{2 \leq n \leq x: \frac{|g^{-1}(n)|}{(\log n) \sqrt{\log\log n}} - 
     \frac{6}{\pi^2 n (\log n) \sqrt{\log\log n}} \times \sum_{k \leq n} |g^{-1}(k)| \leq y\right\} & = 
     \Phi\left(\frac{\frac{\pi^2 y}{6}-\mu_x(C)}{\sigma_x(C)}\right) + o(1). 
\end{align*} 
Moreover, we have that for any real $y$, as $x \rightarrow \infty$ 
\begin{align*} 
\frac{1}{x} \times \#\left\{2 \leq n \leq x: |g^{-1}(n)| - 
     \frac{6}{\pi^2 n} \times \sum_{k \leq n} |g^{-1}(k)| \leq y\right\} & = 
     \Phi\left(\frac{\frac{\pi^2 y}{6}- B_0(\log x)\sqrt{\log\log x}}{ 
     D_0\sqrt{x}(\log x)\sqrt{\log\log x}}\right) + o(1).
\end{align*} 
\end{cor} 
\begin{proof} 
We claim that 
\begin{align*} 
|g^{-1}(n)| - \frac{6}{\pi^2 n} \times \sum_{k \leq n} |g^{-1}(k)| & \sim \frac{6}{\pi^2} C_{\Omega}(n), 
     \text{\ as\ } n \rightarrow \infty. 
\end{align*} 
As in the proof of Corollary \ref{cor_ExpectationFormulaAbsgInvn_v2}, 
we obtain that 
\begin{align*} 
\frac{1}{x} \times \sum_{n \leq x} |g^{-1}(n)| & = 
     \frac{6}{\pi^2} \left(\frac{1}{x} \times \sum_{n \leq x} C_{\Omega}(n) + \sum_{d<x} 
     \sum_{k \leq d} \frac{C_{\Omega}(k)}{d^2}\right) + O(1). 
\end{align*} 
Let the \emph{backwards difference operator} with respect to $x$ 
be defined for $x \geq 2$ and any arithmetic function $f$ as 
$\Delta_x(f(x)) := f(x) - f(x-1)$. 
We see that for large $n$ 
\begin{align*} 
     |g^{-1}(n)| & = \Delta_n\left(\sum_{k \leq n} g^{-1}(k)\right)  
     \sim \frac{6}{\pi^2} \times 
     \Delta_n\left(\sum_{d \leq n} C_{\Omega}(d) \cdot \frac{n}{d}\right) \\ 
     & = \frac{6}{\pi^2}\left(C_{\Omega}(n) + \sum_{d < n} C_{\Omega}(d) \frac{n}{d} - 
     \sum_{d<n} C_{\Omega}(d) \frac{(n-1)}{d}\right) \\ 
     & \sim \frac{6}{\pi^2} \left(C_{\Omega}(n) + \frac{1}{n-1} \times \sum_{k < n} |g^{-1}(k)|\right), 
     \mathrm{\ as\ } n \rightarrow \infty. 
\end{align*} 
Since $\frac{1}{n-1} \times \sum_{k < n} |g^{-1}(k)| \sim \frac{1}{n} \times \sum_{k \leq n} |g^{-1}(k)|$ 
for all sufficiently large $n$, 
the results follow by a re-normalization of Conjecture \ref{conj_DetFormOfEKTypeThmForCOmegan_v1}. 
\end{proof} 

\section{New formulas and limiting relations characterizing $M(x)$} 
\label{Section_KeyApplications} 

\subsection{Formulas relating $M(x)$ to the summatory function $G^{-1}(x)$} 

\begin{prop} 
\label{prop_Mx_SBP_IntegralFormula} 
For all sufficiently large $x$, we have that 
\begin{align} 
\label{eqn_pf_tag_v2-restated_v2} 
M(x) & = G^{-1}(x) + 
     \sum_{k=1}^{\frac{x}{2}} G^{-1}(k) \left(
     \pi\left(\Floor{x}{k}\right) - \pi\left(\Floor{x}{k+1}\right) 
     \right). 
\end{align} 
\end{prop} 
\begin{proof} 
We know by applying Corollary \ref{cor_Mx_gInvnPixk_formula} that 
\begin{align} 
\notag
M(x) & = \sum_{k=1}^{x} g^{-1}(k) \left(\pi\left(\Floor{x}{k}\right)+1\right) \\ 
\notag 
     & = G^{-1}(x) + \sum_{k=1}^{\frac{x}{2}} g^{-1}(k) \pi\left(\Floor{x}{k}\right) \\ 
\notag 
     & = G^{-1}(x) + G^{-1}\left(\Floor{x}{2}\right) + 
     \sum_{k=1}^{\frac{x}{2}-1} G^{-1}(k) \left( 
     \pi\left(\Floor{x}{k}\right) - \pi\left(\Floor{x}{k+1}\right) 
     \right).
\end{align} 
The upper bound on the sum is truncated to $k \in \left[1, \frac{x}{2}\right]$ in the second equation 
above due to the fact that $\pi(1) = 0$. 
The third formula above follows directly by (ordinary) summation by parts. 
\end{proof} 

By the result from \eqref{eqn_RmkInitialConnectionOfMxToGInvx_ProvedByInversion_v1} 
proved in Section \ref{subSection_Relating_CknFuncs_to_gInvn}, 
we recall that 
\[
M(x) = G^{-1}(x) + \sum_{p \leq x} G^{-1}\left(\Floor{x}{p}\right), 
     \text{ for } x \geq 1. 
\]
Summation by parts implies that we can also express $G^{-1}(x)$ in terms 
of the summatory function $L(x)$ and differences of the unsigned sequence 
whose distribution is given by 
Corollary \ref{cor_CLT_VII}. That is, we have  
\[
G^{-1}(x) 
     = \sum_{n \leq x} \lambda(n) |g^{-1}(n)| 
     = L(x)|g^{-1}(x)| - \sum_{n < x} 
     L(n) \left(|g^{-1}(n+1)| - |g^{-1}(n)|\right), 
     \text{ for } x \geq 1. 
\]

\subsection{Asymptotics of partial sums of the unsigned inverse sequence} 
\label{subSection_AsymptoticsOfGinvx} 

The proofs in this subsection are credited to correspondence with 
Professor R.~C.~Vaughan and his suggestions 
about approaches to upper bounds on $|G^{-1}|(x)$ that are attained 
along infinite subsequences as $x \rightarrow \infty$. 
The ideas at the crux of the proof of the next theorem 
are found in the references by Davenport and Heilbronn \cite{DAVHEIL-1936A,DAVHEIL-1936B}. 
They are known to date back to the work of 
Harald Bohr \cite[\cf \S 11]{TITCHMARSH}. 

\begin{theorem}
\label{theorem_VaughanGrowthOfGInvxAndZerosOfPrimeZetaFunc_v1}
Let $\sigma_1$ denote the unique solution to the equation 
$P(\sigma) = 1$ for $\sigma > 1$. 
There are complex $s$ with $\Re(s)$ arbitrarily close to $\sigma_1$
such that $1-P(s) = 0$. 
\end{theorem}
\begin{proof}
The function $P(\sigma)$ is decreasing on $(1, \infty)$, 
tends to $+\infty$ as $\sigma \rightarrow 1^{+}$, and tends to zero as 
$\sigma \rightarrow \infty$. Thus we find that the equation $P(\sigma) = 1$ 
has a unique solution for $\sigma > 1$, which we denote by 
$\sigma = \sigma_1 \approx 1.39943$. 
Let $\delta > 0$ be chosen small enough that $|1-P(z)| > 0$ for all 
$z$ such that $|z-\sigma_1| = \delta$. Set 
\[
\eta = \min_{\substack{z \in \mathbb{C} \\ |z-\sigma_1|=\delta}} |1-P(z)|. 
\]
Since $P(z)$ is continuous whenever $\Re(z) > 1$, we have that 
$\eta > 0$. 
Let $X \geq 2$ be a sufficiently large integer so that 
\[
\sum_{p > X} p^{\delta-\sigma_1} < \frac{\eta}{4}. 
\]
Kronecker's theorem provides a fixed $t$ such that the 
following inequality holds \cite[\S XXIII]{HARDYWRIGHT}: 
\[
\max_{2 < p \leq X} \min_{n \in \mathbb{Z}} \left\lvert 
     \frac{t \log p}{2\pi} - n - \frac{1}{2} \right\rvert < \delta\eta. 
\]
Thus we have that 
\[
\sum_{p > 2} p^{\delta-\sigma_1} \left\lvert p^{\imath t} + 1 \right\rvert < 
     \frac{\eta}{2}. 
\]
Hence, for all $z$ such that $|z-\sigma_1|=\delta$, we have 
\[
\left\lvert P(z+\imath t) + P(z) \right\rvert < \frac{\eta}{2}. 
\]
We apply Rouch\'{e}'s theorem to see that the functions 
$1-P(z)$ and $1-P(z) + P(z+\imath t) + P(z)$ have the same number of zeros in 
the disk $\mathcal{D}_{\delta} = \{z \in \mathbb{C}: |z-\sigma_1| < \delta\}$. 
Since $1-P(z)$ has at least one zero within $\mathcal{D}_{\delta}$, we must have that 
$1+P(w)$ has at least one zero with $|w-\sigma_1-\imath t| < \delta$. 
Since we can take $\delta$ as small as necessary, 
there are zeros of the function $1+P(s)$ that are arbitrarily close to the 
line $s = \sigma_1$. 
\end{proof}

\begin{cor}
\label{cor_Vaughan_LimSupLowerBounds_On_GInvx_AtLarge_x_v2} 
Suppose that the partial sums of the unsigned inverse sequence are defined as follows: 
\[
|G^{-1}|(x) := \sum_{n \leq x} |g^{-1}(n)|, x \geq 1. 
\]
Let $\sigma_1 > 1$ be defined as in 
Theorem \ref{theorem_VaughanGrowthOfGInvxAndZerosOfPrimeZetaFunc_v1}. 
For any $\epsilon > 0$, there are arbitrarily large $x$ such that 
\[
|G^{-1}|(x) > x^{\sigma_1-\epsilon}. 
\]
\end{cor}
\begin{proof} 
Since the DGF of the function $C_{\Omega}(n)$ is given by $(1-P(s))^{-1}$ for $\Re(s) > 1$,  
we have that 
\[
D_{|g^{-1}|}(s) := \sum_{n \geq 1} \frac{|g^{-1}(n)|}{n^s} = \frac{1}{\zeta(s)(1-P(s))}, 
     \text{ for } \Re(s) > 1. 
\]
Theorem \ref{theorem_VaughanGrowthOfGInvxAndZerosOfPrimeZetaFunc_v1} implies that 
$D_{g^{-1}}(s)$ has singularities $s \in \mathbb{C}$ such that 
the $\Re(s)$ are arbitrarily close to $\sigma_1$. 
By applying \cite[Cor.~1.2; \S 1.2]{MV}, we have that any Dirichlet series 
is locally uniformly convergent in its half-plane of convergence, 
e.g., for $\Re(s) > \sigma_c$, and is hence analytic in this half-plane. 
It follows that the abscissa of convergence of 
$D_{g^{-1}}(s)$ is given by $\sigma_c \geq \sigma_1 > 1$. In particular, 
the abscissa of convergence of this DGF cannot be smaller than $\sigma_1$. 
The theorem proved in \cite[Thm.~1.3; \S 1.2]{MV} shows that 
\[
\limsup_{x \rightarrow \infty} \frac{\log |G^{-1}|(x)}{\log x} = \sigma_c \geq \sigma_1. 
     \qedhere 
\]
\end{proof}

\begin{remark}[Implications for new bounds on $M(x)$]
\label{remark_AbsGInvx_ProspectsOfTheNewBVBoundProofs_v2} 
Notice that for any $x \geq 1$ we can for the signed partial sums of $g^{-1}(n)$ as  
\begin{align*} 
G^{-1}(x) & = \sum_{n \leq x} \lambda(n) |g^{-1}(n)| \sim \sum_{n \leq x} \lambda(n) \left( 
     \int_{n-1}^{n} \frac{d}{dt} |G^{-1}|(t) dt \right). 
\end{align*} 
Hence, we note that it is worthwhile to attempt to extract more precise information about the 
asymptotics of this summatory function, and its characterization of $M(x)$ in 
\eqref{eqn_RmkInitialConnectionOfMxToGInvx_ProvedByInversion_v1}, 
based on limit-supremum type bounds of the type in 
Corollary \ref{cor_Vaughan_LimSupLowerBounds_On_GInvx_AtLarge_x_v2} along infinite 
subsequences of positive integers. 
An important motivating open problem is to resolve whether it is the case that
\[
\limsup_{x \rightarrow \infty} \frac{|M(x)|}{\sqrt{x}} = +\infty, 
\]
and if so, to determine the rate with which the normalized Mertens function becomes unbounded.  
Extensions of the bounds we have proved in this subsection then formulate 
one concrete new approach to this famous problem. 
\end{remark}

\subsection{Local cancellation of $G^{-1}(x)$ in the new formulas for $M(x)$} 
\label{subSection_LocalCancellationOfGInvx} 

\begin{lemma}
\label{theorem_PrimorialSeqGInvCalcs_v1} 
Suppose that $p_n$ denotes the $n^{th}$ prime for $n \geq 1$ 
\cite[\seqnum{A000040}]{OEIS}. 
Let $\mathcal{P}_{\#}$ denote the set of positive primorial integers given by  
\cite[\seqnum{A002110}]{OEIS} 
\[
\mathcal{P}_{\#} = \left\{n\#\right\}_{n \geq 1} = \left\{\prod_{k=1}^{n} p_k : n \geq 1\right\} = 
     \{2, 6, 30, 210, 2310, 30030, \ldots\}. 
\]
As $m \rightarrow \infty$ we have that 
\begin{align} 
\tag{A} 
-G^{-1}((4m+1)\#) & \asymp (4m+1)!, \\ 
\tag{B} 
G^{-1}\left(\frac{(4m+1)\#}{p_k}\right) & \asymp (4m)!, 
     \mathrm{\ for\ any\ } 1 \leq k \leq 4m+1. 
\end{align} 
\end{lemma}
\begin{proof} 
We have by part (B) of Proposition \ref{lemma_gInv_MxExample} 
that for all squarefree integers $n \geq 1$ 
\begin{align*} 
|g^{-1}(n)| & = \sum_{j=0}^{\omega(n)} \binom{\omega(n)}{j} \times j! 
     = (\omega(n))! \times \sum_{j=0}^{\omega(n)} \frac{1}{j!} \\ 
     & = (\omega(n))! \times \left(e + O\left(\frac{1}{(\omega(n)+1)!}\right)\right). 
\end{align*} 
Let $m$ be a large positive integer. 
We obtain main terms of the form 
\begin{align*} 
\sum_{\substack{n \leq (4m+1)\# \\ \omega(n)=\Omega(n)}} \lambda(n) |g^{-1}(n)| 
     & = \sum_{0 \leq k \leq 4m+1} \binom{4m+1}{k} (-1)^{k} k! 
     \left(e + O\left(\frac{1}{(k+1)!}\right)\right) \\ 
     & = -(4m+1)! + O(1). 
\end{align*} 
The formula for $C_{\Omega}(n)$ stated in 
\eqref{eqn_proof_tag_hInvn_ExactNestedSumFormula_CombInterpetIdent_v3} 
then implies the result in (A). 
We can similarly derive for any $1 \leq k \leq 4m+1$ that 
\begin{align*}
G^{-1}\left(\frac{(4m+1)\#}{p_k}\right) & \asymp \sum_{0 \leq k \leq 4m} \binom{4m}{k} (-1)^{k} k! 
     \left(e + O\left(\frac{1}{(k+1)!}\right)\right) \asymp (4m)!. 
     \qedhere 
\end{align*}
\end{proof}

\begin{remark}
Even though we get comparatively large order growth of 
$|G^{-1}|(x) \geq |G^{-1}(x)|$ infinitely often, 
we should expect that there is usually (almost always) 
a large cancellation between the successive 
values of this summatory function in the form of 
\eqref{eqn_RmkInitialConnectionOfMxToGInvx_ProvedByInversion_v1}. 
Lemma \ref{theorem_PrimorialSeqGInvCalcs_v1} 
demonstrates the phenomenon well along the infinite 
subsequence of large $x$ taken along the primorials, or 
the integers $x = (4m+1)\#$ 
that are precisely the product of the first $4m+1$ primes for $m \geq 1$. 
In particular, we have that 
\cite{DUSART-1999,DUSART-2010} 
\[
n\# \sim e^{\vartheta(p_n)} \asymp n^n (\log n)^n e^{-n(1+o(1))}, 
     \text{ as } n \rightarrow \infty. 
\]
The RH then requires that the sums of the leading constants with opposing signs 
on the asymptotics for the functions from the lemma match. 
Indeed, this observation follows from the fact that if we obtain a contrary result, 
equation \eqref{eqn_RmkInitialConnectionOfMxToGInvx_ProvedByInversion_v1} would imply that 
\[
\frac{M((4m+1)\#)}{\sqrt{(4m+1)\#}} \gg \left[(4m+1)\#\right]^{\delta_0}, 
     \text{ as } m \rightarrow \infty, 
\]
for some fixed $\delta_0 > 0$ 
(\cf equation \eqref{eqn_MertensMx_RHEquivProblem_Stmt_intro} of the introduction).
\end{remark}

\section{Conclusions}

We have identified a new sequence, 
$\{g^{-1}(n)\}_{n \geq 1}$, that is the Dirichlet inverse of the 
shifted strongly additive function $\omega(n)$. 
Section \ref{subSection_AConnectionToDistOfThePrimes}, 
shows that there is a natural combinatorial interpretation to the 
distribution of distinct values 
of $|g^{-1}(n)|$ for $n \leq x$ involving the distribution of the 
primes $p \leq x$ at large $x$. 
In particular, the magnitude of $g^{-1}(n)$ depends only on the pattern of 
the exponents of the prime factorization of $n$. 
The sign of $g^{-1}(n)$ is given by $\lambda(n)$ for all $n \geq 1$. 
This leads to a new relations of the 
summatory function $G^{-1}(x)$, which characterizes the distribution of $M(x)$, 
to the distribution of the classical summatory function $L(x)$. 

We emphasize that our new work on the Mertens function proved within this article 
is significant in providing a new window through which we can view bounding $M(x)$ 
through asymptotics of auxiliary sequences and partial sums. 
The computational data generated in 
Table \ref{table_conjecture_Mertens_ginvSeq_approx_values} of the appendix section 
indicates numerically that the distribution of $G^{-1}(x)$ is easier to work with 
than that of $M(x)$ or $L(x)$. 
The additively combinatorial 
relation of the distinct (and repetition of) values of $|g^{-1}(n)|$ 
for $n \leq x$ are suggestive towards bounding main terms for $G^{-1}(x)$ along 
infinite subsequences in future work. 

\section*{Acknowledgments}
\addcontentsline{toc}{section}{Acknowledgments}

We thank the following professors that offered 
discussion, feedback and correspondence while the article was being actively written: 
Gerg\H{o} Nemes, Robert Vaughan, Jeffrey Lagarias, Steven J.~Miller, 
Paul Pollack and Bruce Reznick. 
The work on the article was supported in part by 
funding made available within the School of Mathematics at the 
Georgia Institute of Technology in 2020 and 2021. 
Without this combined support, the article would not have been possible.

\renewcommand{\refname}{References} 
\addcontentsline{toc}{section}{References}
%\bibliography{glossaries-bibtex/thesis-references}{}
\bibliographystyle{plain}

\begin{thebibliography}{10}

\bibitem{APOSTOLANUMT}
T.~M. Apostol.
\newblock {\em Introduction to Analytic Number Theory}.
\newblock Springer--Verlag, 1976.

\bibitem{ANT-BATEMAN-DIAMOND}
P.~T. Bateman and H.~G. Diamond.
\newblock {\em Analytic Number Theory}.
\newblock World Scientific Publishing, 2004.

\bibitem{BILLINGSLY-CLT-PRIMEDIVFUNC}
P.~Billingsley.
\newblock On the central limit theorem for the prime divisor function.
\newblock {\em Amer. Math. Monthly}, 76(2):132--139, 1969.

\bibitem{DAVHEIL-1936A}
H.~Davenport and H.~Heilbronn.
\newblock On the zeros of certain {D}irichlet series {I}.
\newblock {\em J. London Math. Soc.}, 11:181--185, 1936.

\bibitem{DAVHEIL-1936B}
H.~Davenport and H.~Heilbronn.
\newblock On the zeros of certain {D}irichlet series {II}.
\newblock {\em J. London Math. Soc.}, 11:307--312, 1936.

\bibitem{DUSART-1999}
P.~Dusart.
\newblock The $k^{th}$ prime is greater than $k(\log k +\log\log k-1)$ for $k
  \geq 2$.
\newblock {\em Math. Comp.}, 68(225):411--415, 1999.

\bibitem{DUSART-2010}
P.~Dusart.
\newblock Estimates of some functions over primes without {R}.{H}, 2010.

\bibitem{ERDOS-KAC-REF}
P.~Erd{\H{o}}s and M.~Kac.
\newblock The {G}aussian errors in the theory of additive arithmetic functions.
\newblock {\em American Journal of Mathematics}, 62(1):738--742, 1940.

\bibitem{FROBERG-1968}
C.~E. Fr{\"{o}}berg.
\newblock On the prime zeta function.
\newblock {\em BIT Numerical Mathematics}, 8:87--202, 1968.

\bibitem{HARDYWRIGHT}
G.~H. Hardy and E.~M. Wright.
\newblock {\em An Introduction to the Theory of Numbers}.
\newblock Oxford University Press, 2008 (Sixth Edition).

\bibitem{HUMPHRIES-JNT-2013}
P.~Humphries.
\newblock The distribution of weighted sums of the {L}iouville function and
  {P}\'{o}lya's conjecture.
\newblock {\em J. Number Theory}, 133:545--582, 2013.

\bibitem{HURST-2017}
G.~Hurst.
\newblock Computations of the {M}ertens function and improved bounds on the
  {M}ertens conjecture.
\newblock {\em Math. Comp.}, 87:1013--1028, 2018.

\bibitem{CLT-RANDOM-ORDERED-FACTS-2011}
H.~Hwang and S.~Janson.
\newblock A central limit theorem for random ordered factorizations of
  integers.
\newblock {\em Electron. J. Probab.}, 16(12):347--361, 2011.

\bibitem{IWANIEC-KOWALSKI}
H.~Iwaniec and E.~Kowalski.
\newblock {\em Analytic Number Theory}, volume~53.
\newblock AMS Colloquium Publications, 2004.

\bibitem{MREVISITED}
T.~Kotnik and H.~te~Riele.
\newblock The {M}ertens conjecture revisited.
\newblock {\em Algorithmic Number Theory}, $7^{th}$ International Symposium,
  2006.

\bibitem{ORDER-MERTENSFN}
T.~Kotnik and J.~van~de Lune.
\newblock On the order of the {M}ertens function.
\newblock {\em Exp. Math.}, 2004.

\bibitem{LEHMAN-1960}
R.~S. Lehman.
\newblock On {L}iouville's function.
\newblock {\em Math. Comput.}, 14:311--320, 1960.

\bibitem{MV}
H.~L. Montgomery and R.~C. Vaughan.
\newblock {\em Multiplicative Number Theory: I. Classical Theory}.
\newblock Cambridge, 2006.

\bibitem{NEMES2015C}
G.~Nemes.
\newblock The resurgence properties of the incomplete gamma function {II}.
\newblock {\em Stud. Appl. Math.}, 135(1):86--116, 2015.

\bibitem{NEMES2016}
G.~Nemes.
\newblock The resurgence properties of the incomplete gamma function {I}.
\newblock {\em Anal. Appl. (Singap.)}, 14(5):631--677, 2016.

\bibitem{NEMES2019}
G.~Nemes and A.~B.~Olde Daalhuis.
\newblock Asymptotic expansions for the incomplete gamma function in the
  transition regions.
\newblock {\em Math. Comp.}, 88(318):1805--1827, 2019.

\bibitem{NG-MERTENS}
N.~Ng.
\newblock The distribution of the summatory function of the {M}{\'{o}}bius
  function.
\newblock {\em Proc. London Math. Soc.}, 89(3):361--389, 2004.

\bibitem{ODLYZ-TRIELE}
A.~M. Odlyzko and H.~J.~J. te~Riele.
\newblock Disproof of the {M}ertens conjecture.
\newblock {\em J. Reine Angew. Math.}, 1985.

\bibitem{NISTHB}
F.~W.~J. Olver, D.~W. Lozier, R.~F. Boisvert, and C.~W. Clark, editors.
\newblock {\em {NIST} Handbook of Mathematical Functions}.
\newblock Cambridge University Press, 2010.

\bibitem{RENYI-TURAN}
A.~Renyi and P.~Turan.
\newblock On a theorem of {E}rd{\H{o}}s-{K}ac.
\newblock {\em Acta Arithmetica}, 4(1):71--84, 1958.

\bibitem{PRIMEREC}
P.~Ribenboim.
\newblock {\em The new book of prime number records}.
\newblock Springer, 1996.

\bibitem{SCHMIDT-MERTENS-COMPUTATIONS}
M.~D. Schmidt.
\newblock {S}age{M}ath and {M}athematica software for computations with the
  {M}ertens function, 2021.
\newblock \url{https://github.com/maxieds/MertensFunctionComputations}.

\bibitem{OEIS}
N.~J.~A. Sloane.
\newblock The {O}nline {E}ncyclopedia of {I}nteger {S}equences, 2021.
\newblock \url{http://oeis.org}.

\bibitem{SOUND-MERTENS-ANNALS}
K.~Soundararajan.
\newblock Partial sums of the {M}{\"{o}}bius function.
\newblock {\em J. Reine Angew. Math.}, 2009(631):141--152, 2009.

\bibitem{TENENBAUM-PROBNUMT-METHODS}
G.~Tenenbaum.
\newblock {\em Introduction to Analytic and Probabilistic Number Theory}.
\newblock American Mathematical Society, third edition, 2015.

\bibitem{TITCHMARSH}
E.~C. Titchmarsh.
\newblock {\em The theory of the {R}iemann zeta function}.
\newblock Oxford University Press, second edition, 1986.

\end{thebibliography}

\setcounter{section}{0} 
\renewcommand{\thesection}{\Alph{section}} 

\newpage
\section{Appendix: Asymptotic formulas for partial sums} 
\label{subSection_OtherFactsAndResults} 

We appreciate the correspondence with Gerg\H{o} Nemes 
from the Alfr\'{e}d R\'{e}nyi Institute of Mathematics and his 
careful notes on the limiting asymptotics for the sums identified in this section. 
We have adapted the communication of his proofs to establish the next few lemmas based on his 
recent work in the references \cite{NEMES2015C,NEMES2016,NEMES2019}. 

\begin{facts}[The incomplete gamma function] 
\label{facts_ExpIntIncGammaFuncs} 
\begin{subequations}
The (upper) \emph{incomplete gamma function} is defined by \cite[\S 8.4]{NISTHB} 
\[
\Gamma(a, z) = \int_{z}^{\infty} t^{a-1} e^{-t} dt, a \in \mathbb{R}, |\arg z| < \pi.  
\]
The function $\Gamma(a, z)$ can be continued to an analytic function of $z$ on the 
universal covering of $\mathbb{C} \mathbin{\backslash} \{0\}$. 
For $a \in \mathbb{Z}^{+}$, the function $\Gamma(a, z)$ is an entire function of $z$. 
The following properties of $\Gamma(a, z)$ hold \cite[\S 8.4; \S 8.11(i)]{NISTHB}: 
\begin{align} 
\label{eqn_IncompleteGamma_PropA} 
\Gamma(a, z) & = (a-1)! e^{-z} \times \sum_{k=0}^{a-1} \frac{z^k}{k!}, \mathrm{\ for\ } 
     a \in \mathbb{Z}^{+}, z \in \mathbb{C}, \\ 
\label{eqn_IncompleteGamma_PropB} 
\Gamma(a, z) & \sim z^{a-1} e^{-z}, \mathrm{\ for\ fixed\ } a \in \mathbb{C}, 
     \mathrm{\ as\ } z \rightarrow +\infty. 
\end{align}
Moreover, for real $z > 0$, as $z \rightarrow +\infty$ we have that \cite{NEMES2015C} 
\begin{equation} 
\label{eqn_IncompleteGamma_PropC}
\Gamma(z, z) = \sqrt{\frac{\pi}{2}} z^{z-\frac{1}{2}} e^{-z} + 
     O\left(z^{z-1} e^{-z}\right), 
\end{equation} 
If $z,a \rightarrow \infty$ with $z = \lambda a$ for some $\lambda > 1$ such that 
$(\lambda - 1)^{-1} = o\left(\sqrt{|a|}\right)$, then \cite{NEMES2015C}
\begin{equation}
\label{eqn_IncompleteGamma_PropD}
\Gamma(a, z) \sim z^a e^{-z} \times \sum_{n \geq 0} \frac{(-a)^n b_n(\lambda)}{(z-a)^{2n+1}}. 
\end{equation} 
The sequence $b_n(\lambda)$ satisfies the characteristic recurrence relation that 
$b_0(\lambda) = 1$ and\footnote{
     An exact formula for $b_n(\lambda)$ is given in terms of the 
     \emph{second-order Eulerian number triangle} 
     \cite[\seqnum{A008517}]{OEIS} as follows: 
     \[
          b_n(\lambda) = \sum_{k=0}^{n} \gkpEII{n}{k} \lambda^{k+1}. 
     \]
}
\[
b_n(\lambda) = \lambda(1-\lambda) b_{n-1}^{\prime}(\lambda) + \lambda(2n-1) b_{n-1}(\lambda), n \geq 1. 
\]
\end{subequations}
\end{facts} 

\begin{prop}
\label{prop_IncGammaLambdaTypeBounds_v1}
Let $a,z,\lambda$ be positive real parameters such that $z=\lambda a$. 
If $\lambda \in (0, 1)$, then as $z \rightarrow \infty$ 
\[
\Gamma(a, z) = \Gamma(a) + O_{\lambda}\left(z^{a-1} e^{-z}\right). 
\]
If $\lambda > 1$, then as 
$z \rightarrow \infty$ 
\[
\Gamma(a, z) = \frac{z^{a-1} e^{-z}}{1-\lambda^{-1}} + O_{\lambda}\left(z^{a-2} e^{-z}\right). 
\]
If $\lambda > 0.567142 > W(1)$ where $W(x)$ denotes the principal branch of the 
Lambert $W$-function for $x \geq 0$, 
then as $z \rightarrow \infty$ 
\[
\Gamma(a, z e^{\pm\pi\imath}) = -e^{\pm \pi\imath a} \frac{z^{a-1} e^{z}}{1 + \lambda^{-1}} + 
     O_{\lambda}\left(z^{a-2} e^{z}\right). 
\]
\end{prop}
Note that the first two estimates are only useful when $\lambda$ is bounded away from the 
transition point at $1$. 
We cannot write the last expansion above 
as $\Gamma(a, -z)$ directly unless $a \in \mathbb{Z}^{+}$ 
as the incomplete gamma function 
has a branch point at the origin with respect to its second variable. 
This function becomes a single-valued 
analytic function of its second input by continuation 
on the universal covering of $\mathbb{C} \mathbin{\backslash} \{0\}$. 
\begin{proof}
The first asymptotic estimate follows directly from the following 
asymptotic series expansion that holds as $z \rightarrow +\infty$ 
\cite[Eq.\ (2.1)]{NEMES2019}: 
\[
\Gamma(a, z) \sim \Gamma(a) + z^a e^{-z} \times \sum_{k \geq 0} 
     \frac{(-a)^k b_k(\lambda)}{(z-a)^{2k+1}}. 
\]
Using the notation from \eqref{eqn_IncompleteGamma_PropD} and \cite{NEMES2016}, 
we have that 
\[
\Gamma(a, z) = \frac{z^{a-1} e^{-z}}{1-\lambda^{-1}} + z^{a} e^{-z} R_1(a, \lambda). 
\]
From the bounds in \cite[\S 3.1]{NEMES2016}, we have that 
\[
\left\lvert z^{a} e^{-z} R_1(a, \lambda) \right\rvert \leq 
     z^a e^{-z} \times \frac{a \cdot b_1(\lambda)}{(z-a)^{3}} = 
     \frac{z^{a-2} e^{-z}}{(1-\lambda^{-1})^{3}}
\]
The main and error terms in the previous equation can also be 
seen by applying the asymptotic series in 
\eqref{eqn_IncompleteGamma_PropD} directly. 

The proof of the third equation above follows from the following asymptotics 
\cite[Eq.\ (1.1)]{NEMES2015C}
\[
\Gamma(-a, z) \sim z^{-a} e^{-z} \times \sum_{n \geq 0} \frac{a^n b_n(-\lambda)}{(z+a)^{2n+1}}, 
\]
by setting $(a, z) \mapsto \left(a e^{\pm \pi\imath}, z e^{\pm \pi\imath}\right)$ so that 
$\lambda = \frac{z}{a} > 0.567142 > W(1)$. 
The restriction on the range of $\lambda$ over which the third formula holds is made to ensure that 
the last formula from the reference is valid at negative real $a$. 
\end{proof}

\begin{lemma}
\label{lemma_ConvenientIncGammaFuncTypePartialSumAsymptotics_v2}
For $x \rightarrow +\infty$, we have that 
\begin{align*}
S_1(x) & := \frac{x}{\log x} \times \left\lvert \sum_{1 \leq k \leq \floor{\log\log x}} 
     \frac{(-1)^k (\log\log x)^{k-1}}{(k-1)!} \right\rvert 
     = \frac{x}{2\sqrt{2\pi \log\log x}} + O\left(\frac{x}{(\log\log x)^{\frac{3}{2}}}\right). 
\end{align*}
\end{lemma}
\begin{proof}
We have for $n \geq 1$ and any $t > 0$ by 
\eqref{eqn_IncompleteGamma_PropA} that 
\[
\sum_{1 \leq k \leq n} \frac{(-1)^k t^{k-1}}{(k-1)!} = -e^{-t} \times 
     \frac{\Gamma(n, -t)}{(n-1)!}. 
\]
Suppose that $t = n + \xi$ with $\xi = O(1)$, e.g., so we can 
formally take the floor of the input $n$ to truncate the last sum. 
By the third formula 
in Proposition \ref{prop_IncGammaLambdaTypeBounds_v1} 
with the parameters $(a, z, \lambda) \mapsto \left(n, t, 1 + \frac{\xi}{n}\right)$, 
we deduce that as $n,t \rightarrow +\infty$. 
\begin{equation}
\label{eqn_ProofTag_lemma_ConvenientIncGammaFuncTypePartialSumAsymptotics_v2}
\Gamma(n, -t) = (-1)^{n+1} \times \frac{t^n e^{t}}{t+n} + 
     O\left(\frac{n t^n e^{t}}{(t+n)^3}\right) = 
     (-1)^{n+1} \frac{t^n e^t}{2n} + O\left(\frac{t^{n-1} e^t}{n}\right). 
\end{equation}
Accordingly, we see that 
\[
\sum_{1 \leq k \leq n} \frac{(-1)^k t^{k-1}}{(k-1)!} = 
     (-1)^{n} \frac{t^n}{2n!} + O\left(\frac{t^{n-1}}{n!}\right). 
\]
By the variant of Stirling's formula in \cite[\cf Eq.\ (5.11.8)]{NISTHB}, we have 
\[
n! = \Gamma(1 + t - \xi) = \sqrt{2\pi} t^{t-\xi+\frac{1}{2}} e^{-t} \left(1 + O\left(t^{-1}\right)\right) = 
     \sqrt{2\pi} t^{n+\frac{1}{2}} e^{-t} \left(1 + O\left(t^{-1}\right)\right). 
\]
Hence, as $n \rightarrow +\infty$ with $t := n + \xi$ and $\xi = O(1)$, we obtain that 
\[
\sum_{k=1}^{n} \frac{(-1)^k t^{k-1}}{(k-1)!} = (-1)^n \frac{e^t}{2 \sqrt{2\pi t}} + 
     O\left(e^t t^{-\frac{3}{2}}\right). 
\]
The conclusion follows by taking $n := \floor{\log\log x}$, 
$t := \log\log x$ and applying the triangle inequality 
to obtain the result. 
\end{proof}

\newpage
\section{Table: Computations involving $g^{-1}(n)$ and $G^{-1}(n)$} 
\label{table_conjecture_Mertens_ginvSeq_approx_values}

\begin{table}[ht!]

\centering

\tiny
\begin{equation*}
\boxed{
\begin{array}{cc|cc|ccc|cc|cccc}
 n & \mathbf{Primes} & \mathbf{Sqfree} & \mathbf{PPower} & g^{-1}(n) & 
 \lambda(n) g^{-1}(n) - \widehat{f}_1(n) & 
 \frac{\sum_{d|n} C_{\Omega}(d)}{|g^{-1}(n)|} & 
 \mathcal{L}_{+}(n) & \mathcal{L}_{-}(n) & 
 G^{-1}(n) & G^{-1}_{+}(n) & G^{-1}_{-}(n) & |G^{-1}|(n) \\ \hline 
 1 & 1^1 & \text{Y} & \text{N} & 1 & 0 & 1.0000000 & 1.00000 & 0 & 1 & 1 & 0 & 1 \\
 2 & 2^1 & \text{Y} & \text{Y} & -2 & 0 & 1.0000000 & 0.500000 & 0.500000 & -1 & 1 & -2 & 3 \\
 3 & 3^1 & \text{Y} & \text{Y} & -2 & 0 & 1.0000000 & 0.333333 & 0.666667 & -3 & 1 & -4 & 5 \\
 4 & 2^2 & \text{N} & \text{Y} & 2 & 0 & 1.5000000 & 0.500000 & 0.500000 & -1 & 3 & -4 & 7 \\
 5 & 5^1 & \text{Y} & \text{Y} & -2 & 0 & 1.0000000 & 0.400000 & 0.600000 & -3 & 3 & -6 & 9 \\
 6 & 2^1 3^1 & \text{Y} & \text{N} & 5 & 0 & 1.0000000 & 0.500000 & 0.500000 & 2 & 8 & -6 & 14 \\
 7 & 7^1 & \text{Y} & \text{Y} & -2 & 0 & 1.0000000 & 0.428571 & 0.571429 & 0 & 8 & -8 & 16 \\
 8 & 2^3 & \text{N} & \text{Y} & -2 & 0 & 2.0000000 & 0.375000 & 0.625000 & -2 & 8 & -10 & 18 \\
 9 & 3^2 & \text{N} & \text{Y} & 2 & 0 & 1.5000000 & 0.444444 & 0.555556 & 0 & 10 & -10 & 20 \\
 10 & 2^1 5^1 & \text{Y} & \text{N} & 5 & 0 & 1.0000000 & 0.500000 & 0.500000 & 5 & 15 & -10 & 25 \\
 11 & 11^1 & \text{Y} & \text{Y} & -2 & 0 & 1.0000000 & 0.454545 & 0.545455 & 3 & 15 & -12 & 27 \\
 12 & 2^2 3^1 & \text{N} & \text{N} & -7 & 2 & 1.2857143 & 0.416667 & 0.583333 & -4 & 15 & -19 & 34 \\
 13 & 13^1 & \text{Y} & \text{Y} & -2 & 0 & 1.0000000 & 0.384615 & 0.615385 & -6 & 15 & -21 & 36 \\
 14 & 2^1 7^1 & \text{Y} & \text{N} & 5 & 0 & 1.0000000 & 0.428571 & 0.571429 & -1 & 20 & -21 & 41 \\
 15 & 3^1 5^1 & \text{Y} & \text{N} & 5 & 0 & 1.0000000 & 0.466667 & 0.533333 & 4 & 25 & -21 & 46 \\
 16 & 2^4 & \text{N} & \text{Y} & 2 & 0 & 2.5000000 & 0.500000 & 0.500000 & 6 & 27 & -21 & 48 \\
 17 & 17^1 & \text{Y} & \text{Y} & -2 & 0 & 1.0000000 & 0.470588 & 0.529412 & 4 & 27 & -23 & 50 \\
 18 & 2^1 3^2 & \text{N} & \text{N} & -7 & 2 & 1.2857143 & 0.444444 & 0.555556 & -3 & 27 & -30 & 57 \\
 19 & 19^1 & \text{Y} & \text{Y} & -2 & 0 & 1.0000000 & 0.421053 & 0.578947 & -5 & 27 & -32 & 59 \\
 20 & 2^2 5^1 & \text{N} & \text{N} & -7 & 2 & 1.2857143 & 0.400000 & 0.600000 & -12 & 27 & -39 & 66 \\
 21 & 3^1 7^1 & \text{Y} & \text{N} & 5 & 0 & 1.0000000 & 0.428571 & 0.571429 & -7 & 32 & -39 & 71 \\
 22 & 2^1 11^1 & \text{Y} & \text{N} & 5 & 0 & 1.0000000 & 0.454545 & 0.545455 & -2 & 37 & -39 & 76 \\
 23 & 23^1 & \text{Y} & \text{Y} & -2 & 0 & 1.0000000 & 0.434783 & 0.565217 & -4 & 37 & -41 & 78 \\
 24 & 2^3 3^1 & \text{N} & \text{N} & 9 & 4 & 1.5555556 & 0.458333 & 0.541667 & 5 & 46 & -41 & 87 \\
 25 & 5^2 & \text{N} & \text{Y} & 2 & 0 & 1.5000000 & 0.480000 & 0.520000 & 7 & 48 & -41 & 89 \\
 26 & 2^1 13^1 & \text{Y} & \text{N} & 5 & 0 & 1.0000000 & 0.500000 & 0.500000 & 12 & 53 & -41 & 94 \\
 27 & 3^3 & \text{N} & \text{Y} & -2 & 0 & 2.0000000 & 0.481481 & 0.518519 & 10 & 53 & -43 & 96 \\
 28 & 2^2 7^1 & \text{N} & \text{N} & -7 & 2 & 1.2857143 & 0.464286 & 0.535714 & 3 & 53 & -50 & 103 \\
 29 & 29^1 & \text{Y} & \text{Y} & -2 & 0 & 1.0000000 & 0.448276 & 0.551724 & 1 & 53 & -52 & 105 \\
 30 & 2^1 3^1 5^1 & \text{Y} & \text{N} & -16 & 0 & 1.0000000 & 0.433333 & 0.566667 & -15 & 53 & -68 & 121 \\
 31 & 31^1 & \text{Y} & \text{Y} & -2 & 0 & 1.0000000 & 0.419355 & 0.580645 & -17 & 53 & -70 & 123 \\
 32 & 2^5 & \text{N} & \text{Y} & -2 & 0 & 3.0000000 & 0.406250 & 0.593750 & -19 & 53 & -72 & 125 \\
 33 & 3^1 11^1 & \text{Y} & \text{N} & 5 & 0 & 1.0000000 & 0.424242 & 0.575758 & -14 & 58 & -72 & 130 \\
 34 & 2^1 17^1 & \text{Y} & \text{N} & 5 & 0 & 1.0000000 & 0.441176 & 0.558824 & -9 & 63 & -72 & 135 \\
 35 & 5^1 7^1 & \text{Y} & \text{N} & 5 & 0 & 1.0000000 & 0.457143 & 0.542857 & -4 & 68 & -72 & 140 \\
 36 & 2^2 3^2 & \text{N} & \text{N} & 14 & 9 & 1.3571429 & 0.472222 & 0.527778 & 10 & 82 & -72 & 154 \\
 37 & 37^1 & \text{Y} & \text{Y} & -2 & 0 & 1.0000000 & 0.459459 & 0.540541 & 8 & 82 & -74 & 156 \\
 38 & 2^1 19^1 & \text{Y} & \text{N} & 5 & 0 & 1.0000000 & 0.473684 & 0.526316 & 13 & 87 & -74 & 161 \\
 39 & 3^1 13^1 & \text{Y} & \text{N} & 5 & 0 & 1.0000000 & 0.487179 & 0.512821 & 18 & 92 & -74 & 166 \\
 40 & 2^3 5^1 & \text{N} & \text{N} & 9 & 4 & 1.5555556 & 0.500000 & 0.500000 & 27 & 101 & -74 & 175 \\
 41 & 41^1 & \text{Y} & \text{Y} & -2 & 0 & 1.0000000 & 0.487805 & 0.512195 & 25 & 101 & -76 & 177 \\
 42 & 2^1 3^1 7^1 & \text{Y} & \text{N} & -16 & 0 & 1.0000000 & 0.476190 & 0.523810 & 9 & 101 & -92 & 193 \\
 43 & 43^1 & \text{Y} & \text{Y} & -2 & 0 & 1.0000000 & 0.465116 & 0.534884 & 7 & 101 & -94 & 195 \\
 44 & 2^2 11^1 & \text{N} & \text{N} & -7 & 2 & 1.2857143 & 0.454545 & 0.545455 & 0 & 101 & -101 & 202 \\
 45 & 3^2 5^1 & \text{N} & \text{N} & -7 & 2 & 1.2857143 & 0.444444 & 0.555556 & -7 & 101 & -108 & 209 \\
 46 & 2^1 23^1 & \text{Y} & \text{N} & 5 & 0 & 1.0000000 & 0.456522 & 0.543478 & -2 & 106 & -108 & 214 \\
 47 & 47^1 & \text{Y} & \text{Y} & -2 & 0 & 1.0000000 & 0.446809 & 0.553191 & -4 & 106 & -110 & 216 \\
 48 & 2^4 3^1 & \text{N} & \text{N} & -11 & 6 & 1.8181818 & 0.437500 & 0.562500 & -15 & 106 & -121 & 227 \\
\end{array}
}
\end{equation*}

\hrule\smallskip 

\captionsetup{singlelinecheck=off} 
\caption*{{\large{\rm \textbf{\rm \bf Table \thesection:} 
          Computations involving $g^{-1}(n) \equiv (\omega+1)^{-1}(n)$ 
          and $G^{-1}(x)$ for $1 \leq n \leq 500$.}} 
          \begin{itemize}[noitemsep,topsep=0pt,leftmargin=0.23in] 
          \item[$\blacktriangleright$] 
          The column labeled \texttt{Primes} provides the prime factorization of each $n$ so that the values of 
          $\omega(n)$ and $\Omega(n)$ are easily extracted. 
          The columns labeled \texttt{Sqfree} and \texttt{PPower}, respectively, 
          list inclusion of $n$ in the sets of squarefree integers and the prime powers. 
          \item[$\blacktriangleright$] 
          The next three columns provide the 
          explicit values of the inverse function $g^{-1}(n)$ and compare its explicit value with other estimates. 
          We define the function $\widehat{f}_1(n) := \sum_{k=0}^{\omega(n)} \binom{\omega(n)}{k} \times k!$. 
          \item[$\blacktriangleright$] 
          The last columns indicate properties of the summatory function of $g^{-1}(n)$. 
          The notation for the (approximate) densities of the sign weight of $g^{-1}(n)$ is defined as 
          $\mathcal{L}_{\pm}(x) := \frac{1}{n} \times \#\left\{n \leq x: \lambda(n) = \pm 1\right\}$. 
          The last three 
          columns then show the sign weighted components to the signed summatory function, 
          $G^{-1}(x) := \sum_{n \leq x} g^{-1}(n)$, decomposed into its 
          respective positive and negative magnitude sum contributions: $G^{-1}(x) = G^{-1}_{+}(x) + G^{-1}_{-}(x)$ where 
          $G^{-1}_{+}(x) > 0$ and $G^{-1}_{-}(x) < 0$ for all $x \geq 1$. 
          That is, the component functions $G^{-1}_{\pm}(x)$ displayed in these second to last two columns 
          of the table correspond to the summatory function $G^{-1}(x)$ with summands that are 
          positive and negative, respectively. 
          The final column of the table provides the partial sums of the absolute value of the unsigned inverse sequence, 
          $|G^{-1}|(n) := \sum_{k \leq n} |g^{-1}(k)|$. 
          \end{itemize} 
          } 
\clearpage 

\end{table}

\newpage
\begin{table}[ht]

\centering

\tiny
\begin{equation*}
\boxed{
\begin{array}{cc|cc|ccc|cc|cccc}
 n & \mathbf{Primes} & \mathbf{Sqfree} & \mathbf{PPower} & g^{-1}(n) & 
 \lambda(n) g^{-1}(n) - \widehat{f}_1(n) & 
 \frac{\sum_{d|n} C_{\Omega}(d)}{|g^{-1}(n)|} & 
 \mathcal{L}_{+}(n) & \mathcal{L}_{-}(n) & 
 G^{-1}(n) & G^{-1}_{+}(n) & G^{-1}_{-}(n) & |G^{-1}|(n) \\ \hline 
 49 & 7^2 & \text{N} & \text{Y} & 2 & 0 & 1.5000000 & 0.448980 & 0.551020 & -13 & 108 & -121 & 229 \\
 50 & 2^1 5^2 & \text{N} & \text{N} & -7 & 2 & 1.2857143 & 0.440000 & 0.560000 & -20 & 108 & -128 & 236 \\
 51 & 3^1 17^1 & \text{Y} & \text{N} & 5 & 0 & 1.0000000 & 0.450980 & 0.549020 & -15 & 113 & -128 & 241 \\
 52 & 2^2 13^1 & \text{N} & \text{N} & -7 & 2 & 1.2857143 & 0.442308 & 0.557692 & -22 & 113 & -135 & 248 \\
 53 & 53^1 & \text{Y} & \text{Y} & -2 & 0 & 1.0000000 & 0.433962 & 0.566038 & -24 & 113 & -137 & 250 \\
 54 & 2^1 3^3 & \text{N} & \text{N} & 9 & 4 & 1.5555556 & 0.444444 & 0.555556 & -15 & 122 & -137 & 259 \\
 55 & 5^1 11^1 & \text{Y} & \text{N} & 5 & 0 & 1.0000000 & 0.454545 & 0.545455 & -10 & 127 & -137 & 264 \\
 56 & 2^3 7^1 & \text{N} & \text{N} & 9 & 4 & 1.5555556 & 0.464286 & 0.535714 & -1 & 136 & -137 & 273 \\
 57 & 3^1 19^1 & \text{Y} & \text{N} & 5 & 0 & 1.0000000 & 0.473684 & 0.526316 & 4 & 141 & -137 & 278 \\
 58 & 2^1 29^1 & \text{Y} & \text{N} & 5 & 0 & 1.0000000 & 0.482759 & 0.517241 & 9 & 146 & -137 & 283 \\
 59 & 59^1 & \text{Y} & \text{Y} & -2 & 0 & 1.0000000 & 0.474576 & 0.525424 & 7 & 146 & -139 & 285 \\
 60 & 2^2 3^1 5^1 & \text{N} & \text{N} & 30 & 14 & 1.1666667 & 0.483333 & 0.516667 & 37 & 176 & -139 & 315 \\
 61 & 61^1 & \text{Y} & \text{Y} & -2 & 0 & 1.0000000 & 0.475410 & 0.524590 & 35 & 176 & -141 & 317 \\
 62 & 2^1 31^1 & \text{Y} & \text{N} & 5 & 0 & 1.0000000 & 0.483871 & 0.516129 & 40 & 181 & -141 & 322 \\
 63 & 3^2 7^1 & \text{N} & \text{N} & -7 & 2 & 1.2857143 & 0.476190 & 0.523810 & 33 & 181 & -148 & 329 \\
 64 & 2^6 & \text{N} & \text{Y} & 2 & 0 & 3.5000000 & 0.484375 & 0.515625 & 35 & 183 & -148 & 331 \\
 65 & 5^1 13^1 & \text{Y} & \text{N} & 5 & 0 & 1.0000000 & 0.492308 & 0.507692 & 40 & 188 & -148 & 336 \\
 66 & 2^1 3^1 11^1 & \text{Y} & \text{N} & -16 & 0 & 1.0000000 & 0.484848 & 0.515152 & 24 & 188 & -164 & 352 \\
 67 & 67^1 & \text{Y} & \text{Y} & -2 & 0 & 1.0000000 & 0.477612 & 0.522388 & 22 & 188 & -166 & 354 \\
 68 & 2^2 17^1 & \text{N} & \text{N} & -7 & 2 & 1.2857143 & 0.470588 & 0.529412 & 15 & 188 & -173 & 361 \\
 69 & 3^1 23^1 & \text{Y} & \text{N} & 5 & 0 & 1.0000000 & 0.478261 & 0.521739 & 20 & 193 & -173 & 366 \\
 70 & 2^1 5^1 7^1 & \text{Y} & \text{N} & -16 & 0 & 1.0000000 & 0.471429 & 0.528571 & 4 & 193 & -189 & 382 \\
 71 & 71^1 & \text{Y} & \text{Y} & -2 & 0 & 1.0000000 & 0.464789 & 0.535211 & 2 & 193 & -191 & 384 \\
 72 & 2^3 3^2 & \text{N} & \text{N} & -23 & 18 & 1.4782609 & 0.458333 & 0.541667 & -21 & 193 & -214 & 407 \\
 73 & 73^1 & \text{Y} & \text{Y} & -2 & 0 & 1.0000000 & 0.452055 & 0.547945 & -23 & 193 & -216 & 409 \\
 74 & 2^1 37^1 & \text{Y} & \text{N} & 5 & 0 & 1.0000000 & 0.459459 & 0.540541 & -18 & 198 & -216 & 414 \\
 75 & 3^1 5^2 & \text{N} & \text{N} & -7 & 2 & 1.2857143 & 0.453333 & 0.546667 & -25 & 198 & -223 & 421 \\
 76 & 2^2 19^1 & \text{N} & \text{N} & -7 & 2 & 1.2857143 & 0.447368 & 0.552632 & -32 & 198 & -230 & 428 \\
 77 & 7^1 11^1 & \text{Y} & \text{N} & 5 & 0 & 1.0000000 & 0.454545 & 0.545455 & -27 & 203 & -230 & 433 \\
 78 & 2^1 3^1 13^1 & \text{Y} & \text{N} & -16 & 0 & 1.0000000 & 0.448718 & 0.551282 & -43 & 203 & -246 & 449 \\
 79 & 79^1 & \text{Y} & \text{Y} & -2 & 0 & 1.0000000 & 0.443038 & 0.556962 & -45 & 203 & -248 & 451 \\
 80 & 2^4 5^1 & \text{N} & \text{N} & -11 & 6 & 1.8181818 & 0.437500 & 0.562500 & -56 & 203 & -259 & 462 \\
 81 & 3^4 & \text{N} & \text{Y} & 2 & 0 & 2.5000000 & 0.444444 & 0.555556 & -54 & 205 & -259 & 464 \\
 82 & 2^1 41^1 & \text{Y} & \text{N} & 5 & 0 & 1.0000000 & 0.451220 & 0.548780 & -49 & 210 & -259 & 469 \\
 83 & 83^1 & \text{Y} & \text{Y} & -2 & 0 & 1.0000000 & 0.445783 & 0.554217 & -51 & 210 & -261 & 471 \\
 84 & 2^2 3^1 7^1 & \text{N} & \text{N} & 30 & 14 & 1.1666667 & 0.452381 & 0.547619 & -21 & 240 & -261 & 501 \\
 85 & 5^1 17^1 & \text{Y} & \text{N} & 5 & 0 & 1.0000000 & 0.458824 & 0.541176 & -16 & 245 & -261 & 506 \\
 86 & 2^1 43^1 & \text{Y} & \text{N} & 5 & 0 & 1.0000000 & 0.465116 & 0.534884 & -11 & 250 & -261 & 511 \\
 87 & 3^1 29^1 & \text{Y} & \text{N} & 5 & 0 & 1.0000000 & 0.471264 & 0.528736 & -6 & 255 & -261 & 516 \\
 88 & 2^3 11^1 & \text{N} & \text{N} & 9 & 4 & 1.5555556 & 0.477273 & 0.522727 & 3 & 264 & -261 & 525 \\
 89 & 89^1 & \text{Y} & \text{Y} & -2 & 0 & 1.0000000 & 0.471910 & 0.528090 & 1 & 264 & -263 & 527 \\
 90 & 2^1 3^2 5^1 & \text{N} & \text{N} & 30 & 14 & 1.1666667 & 0.477778 & 0.522222 & 31 & 294 & -263 & 557 \\
 91 & 7^1 13^1 & \text{Y} & \text{N} & 5 & 0 & 1.0000000 & 0.483516 & 0.516484 & 36 & 299 & -263 & 562 \\
 92 & 2^2 23^1 & \text{N} & \text{N} & -7 & 2 & 1.2857143 & 0.478261 & 0.521739 & 29 & 299 & -270 & 569 \\
 93 & 3^1 31^1 & \text{Y} & \text{N} & 5 & 0 & 1.0000000 & 0.483871 & 0.516129 & 34 & 304 & -270 & 574 \\
 94 & 2^1 47^1 & \text{Y} & \text{N} & 5 & 0 & 1.0000000 & 0.489362 & 0.510638 & 39 & 309 & -270 & 579 \\
 95 & 5^1 19^1 & \text{Y} & \text{N} & 5 & 0 & 1.0000000 & 0.494737 & 0.505263 & 44 & 314 & -270 & 584 \\
 96 & 2^5 3^1 & \text{N} & \text{N} & 13 & 8 & 2.0769231 & 0.500000 & 0.500000 & 57 & 327 & -270 & 597 \\
 97 & 97^1 & \text{Y} & \text{Y} & -2 & 0 & 1.0000000 & 0.494845 & 0.505155 & 55 & 327 & -272 & 599 \\
 98 & 2^1 7^2 & \text{N} & \text{N} & -7 & 2 & 1.2857143 & 0.489796 & 0.510204 & 48 & 327 & -279 & 606 \\
 99 & 3^2 11^1 & \text{N} & \text{N} & -7 & 2 & 1.2857143 & 0.484848 & 0.515152 & 41 & 327 & -286 & 613 \\
 100 & 2^2 5^2 & \text{N} & \text{N} & 14 & 9 & 1.3571429 & 0.490000 & 0.510000 & 55 & 341 & -286 & 627 \\
 101 & 101^1 & \text{Y} & \text{Y} & -2 & 0 & 1.0000000 & 0.485149 & 0.514851 & 53 & 341 & -288 & 629 \\
 102 & 2^1 3^1 17^1 & \text{Y} & \text{N} & -16 & 0 & 1.0000000 & 0.480392 & 0.519608 & 37 & 341 & -304 & 645 \\
 103 & 103^1 & \text{Y} & \text{Y} & -2 & 0 & 1.0000000 & 0.475728 & 0.524272 & 35 & 341 & -306 & 647 \\
 104 & 2^3 13^1 & \text{N} & \text{N} & 9 & 4 & 1.5555556 & 0.480769 & 0.519231 & 44 & 350 & -306 & 656 \\
 105 & 3^1 5^1 7^1 & \text{Y} & \text{N} & -16 & 0 & 1.0000000 & 0.476190 & 0.523810 & 28 & 350 & -322 & 672 \\
 106 & 2^1 53^1 & \text{Y} & \text{N} & 5 & 0 & 1.0000000 & 0.481132 & 0.518868 & 33 & 355 & -322 & 677 \\
 107 & 107^1 & \text{Y} & \text{Y} & -2 & 0 & 1.0000000 & 0.476636 & 0.523364 & 31 & 355 & -324 & 679 \\
 108 & 2^2 3^3 & \text{N} & \text{N} & -23 & 18 & 1.4782609 & 0.472222 & 0.527778 & 8 & 355 & -347 & 702 \\
 109 & 109^1 & \text{Y} & \text{Y} & -2 & 0 & 1.0000000 & 0.467890 & 0.532110 & 6 & 355 & -349 & 704 \\
 110 & 2^1 5^1 11^1 & \text{Y} & \text{N} & -16 & 0 & 1.0000000 & 0.463636 & 0.536364 & -10 & 355 & -365 & 720 \\
 111 & 3^1 37^1 & \text{Y} & \text{N} & 5 & 0 & 1.0000000 & 0.468468 & 0.531532 & -5 & 360 & -365 & 725 \\
 112 & 2^4 7^1 & \text{N} & \text{N} & -11 & 6 & 1.8181818 & 0.464286 & 0.535714 & -16 & 360 & -376 & 736 \\
 113 & 113^1 & \text{Y} & \text{Y} & -2 & 0 & 1.0000000 & 0.460177 & 0.539823 & -18 & 360 & -378 & 738 \\
 114 & 2^1 3^1 19^1 & \text{Y} & \text{N} & -16 & 0 & 1.0000000 & 0.456140 & 0.543860 & -34 & 360 & -394 & 754 \\
 115 & 5^1 23^1 & \text{Y} & \text{N} & 5 & 0 & 1.0000000 & 0.460870 & 0.539130 & -29 & 365 & -394 & 759 \\
 116 & 2^2 29^1 & \text{N} & \text{N} & -7 & 2 & 1.2857143 & 0.456897 & 0.543103 & -36 & 365 & -401 & 766 \\
 117 & 3^2 13^1 & \text{N} & \text{N} & -7 & 2 & 1.2857143 & 0.452991 & 0.547009 & -43 & 365 & -408 & 773 \\
 118 & 2^1 59^1 & \text{Y} & \text{N} & 5 & 0 & 1.0000000 & 0.457627 & 0.542373 & -38 & 370 & -408 & 778 \\
 119 & 7^1 17^1 & \text{Y} & \text{N} & 5 & 0 & 1.0000000 & 0.462185 & 0.537815 & -33 & 375 & -408 & 783 \\
 120 & 2^3 3^1 5^1 & \text{N} & \text{N} & -48 & 32 & 1.3333333 & 0.458333 & 0.541667 & -81 & 375 & -456 & 831 \\
 121 & 11^2 & \text{N} & \text{Y} & 2 & 0 & 1.5000000 & 0.462810 & 0.537190 & -79 & 377 & -456 & 833 \\
 122 & 2^1 61^1 & \text{Y} & \text{N} & 5 & 0 & 1.0000000 & 0.467213 & 0.532787 & -74 & 382 & -456 & 838 \\
 123 & 3^1 41^1 & \text{Y} & \text{N} & 5 & 0 & 1.0000000 & 0.471545 & 0.528455 & -69 & 387 & -456 & 843 \\
 124 & 2^2 31^1 & \text{N} & \text{N} & -7 & 2 & 1.2857143 & 0.467742 & 0.532258 & -76 & 387 & -463 & 850 \\
\end{array}
}
\end{equation*}
\clearpage 

\end{table} 


\newpage
\begin{table}[ht]

\centering

\tiny
\begin{equation*}
\boxed{
\begin{array}{cc|cc|ccc|cc|cccc}
 n & \mathbf{Primes} & \mathbf{Sqfree} & \mathbf{PPower} & g^{-1}(n) & 
 \lambda(n) g^{-1}(n) - \widehat{f}_1(n) & 
 \frac{\sum_{d|n} C_{\Omega}(d)}{|g^{-1}(n)|} & 
 \mathcal{L}_{+}(n) & \mathcal{L}_{-}(n) & 
 G^{-1}(n) & G^{-1}_{+}(n) & G^{-1}_{-}(n) & |G^{-1}|(n) \\ \hline 
 125 & 5^3 & \text{N} & \text{Y} & -2 & 0 & 2.0000000 & 0.464000 & 0.536000 & -78 & 387 & -465 & 852 \\
 126 & 2^1 3^2 7^1 & \text{N} & \text{N} & 30 & 14 & 1.1666667 & 0.468254 & 0.531746 & -48 & 417 & -465 & 882 \\
 127 & 127^1 & \text{Y} & \text{Y} & -2 & 0 & 1.0000000 & 0.464567 & 0.535433 & -50 & 417 & -467 & 884 \\
 128 & 2^7 & \text{N} & \text{Y} & -2 & 0 & 4.0000000 & 0.460938 & 0.539062 & -52 & 417 & -469 & 886 \\
 129 & 3^1 43^1 & \text{Y} & \text{N} & 5 & 0 & 1.0000000 & 0.465116 & 0.534884 & -47 & 422 & -469 & 891 \\
 130 & 2^1 5^1 13^1 & \text{Y} & \text{N} & -16 & 0 & 1.0000000 & 0.461538 & 0.538462 & -63 & 422 & -485 & 907 \\
 131 & 131^1 & \text{Y} & \text{Y} & -2 & 0 & 1.0000000 & 0.458015 & 0.541985 & -65 & 422 & -487 & 909 \\
 132 & 2^2 3^1 11^1 & \text{N} & \text{N} & 30 & 14 & 1.1666667 & 0.462121 & 0.537879 & -35 & 452 & -487 & 939 \\
 133 & 7^1 19^1 & \text{Y} & \text{N} & 5 & 0 & 1.0000000 & 0.466165 & 0.533835 & -30 & 457 & -487 & 944 \\
 134 & 2^1 67^1 & \text{Y} & \text{N} & 5 & 0 & 1.0000000 & 0.470149 & 0.529851 & -25 & 462 & -487 & 949 \\
 135 & 3^3 5^1 & \text{N} & \text{N} & 9 & 4 & 1.5555556 & 0.474074 & 0.525926 & -16 & 471 & -487 & 958 \\
 136 & 2^3 17^1 & \text{N} & \text{N} & 9 & 4 & 1.5555556 & 0.477941 & 0.522059 & -7 & 480 & -487 & 967 \\
 137 & 137^1 & \text{Y} & \text{Y} & -2 & 0 & 1.0000000 & 0.474453 & 0.525547 & -9 & 480 & -489 & 969 \\
 138 & 2^1 3^1 23^1 & \text{Y} & \text{N} & -16 & 0 & 1.0000000 & 0.471014 & 0.528986 & -25 & 480 & -505 & 985 \\
 139 & 139^1 & \text{Y} & \text{Y} & -2 & 0 & 1.0000000 & 0.467626 & 0.532374 & -27 & 480 & -507 & 987 \\
 140 & 2^2 5^1 7^1 & \text{N} & \text{N} & 30 & 14 & 1.1666667 & 0.471429 & 0.528571 & 3 & 510 & -507 & 1017 \\
 141 & 3^1 47^1 & \text{Y} & \text{N} & 5 & 0 & 1.0000000 & 0.475177 & 0.524823 & 8 & 515 & -507 & 1022 \\
 142 & 2^1 71^1 & \text{Y} & \text{N} & 5 & 0 & 1.0000000 & 0.478873 & 0.521127 & 13 & 520 & -507 & 1027 \\
 143 & 11^1 13^1 & \text{Y} & \text{N} & 5 & 0 & 1.0000000 & 0.482517 & 0.517483 & 18 & 525 & -507 & 1032 \\
 144 & 2^4 3^2 & \text{N} & \text{N} & 34 & 29 & 1.6176471 & 0.486111 & 0.513889 & 52 & 559 & -507 & 1066 \\
 145 & 5^1 29^1 & \text{Y} & \text{N} & 5 & 0 & 1.0000000 & 0.489655 & 0.510345 & 57 & 564 & -507 & 1071 \\
 146 & 2^1 73^1 & \text{Y} & \text{N} & 5 & 0 & 1.0000000 & 0.493151 & 0.506849 & 62 & 569 & -507 & 1076 \\
 147 & 3^1 7^2 & \text{N} & \text{N} & -7 & 2 & 1.2857143 & 0.489796 & 0.510204 & 55 & 569 & -514 & 1083 \\
 148 & 2^2 37^1 & \text{N} & \text{N} & -7 & 2 & 1.2857143 & 0.486486 & 0.513514 & 48 & 569 & -521 & 1090 \\
 149 & 149^1 & \text{Y} & \text{Y} & -2 & 0 & 1.0000000 & 0.483221 & 0.516779 & 46 & 569 & -523 & 1092 \\
 150 & 2^1 3^1 5^2 & \text{N} & \text{N} & 30 & 14 & 1.1666667 & 0.486667 & 0.513333 & 76 & 599 & -523 & 1122 \\
 151 & 151^1 & \text{Y} & \text{Y} & -2 & 0 & 1.0000000 & 0.483444 & 0.516556 & 74 & 599 & -525 & 1124 \\
 152 & 2^3 19^1 & \text{N} & \text{N} & 9 & 4 & 1.5555556 & 0.486842 & 0.513158 & 83 & 608 & -525 & 1133 \\
 153 & 3^2 17^1 & \text{N} & \text{N} & -7 & 2 & 1.2857143 & 0.483660 & 0.516340 & 76 & 608 & -532 & 1140 \\
 154 & 2^1 7^1 11^1 & \text{Y} & \text{N} & -16 & 0 & 1.0000000 & 0.480519 & 0.519481 & 60 & 608 & -548 & 1156 \\
 155 & 5^1 31^1 & \text{Y} & \text{N} & 5 & 0 & 1.0000000 & 0.483871 & 0.516129 & 65 & 613 & -548 & 1161 \\
 156 & 2^2 3^1 13^1 & \text{N} & \text{N} & 30 & 14 & 1.1666667 & 0.487179 & 0.512821 & 95 & 643 & -548 & 1191 \\
 157 & 157^1 & \text{Y} & \text{Y} & -2 & 0 & 1.0000000 & 0.484076 & 0.515924 & 93 & 643 & -550 & 1193 \\
 158 & 2^1 79^1 & \text{Y} & \text{N} & 5 & 0 & 1.0000000 & 0.487342 & 0.512658 & 98 & 648 & -550 & 1198 \\
 159 & 3^1 53^1 & \text{Y} & \text{N} & 5 & 0 & 1.0000000 & 0.490566 & 0.509434 & 103 & 653 & -550 & 1203 \\
 160 & 2^5 5^1 & \text{N} & \text{N} & 13 & 8 & 2.0769231 & 0.493750 & 0.506250 & 116 & 666 & -550 & 1216 \\
 161 & 7^1 23^1 & \text{Y} & \text{N} & 5 & 0 & 1.0000000 & 0.496894 & 0.503106 & 121 & 671 & -550 & 1221 \\
 162 & 2^1 3^4 & \text{N} & \text{N} & -11 & 6 & 1.8181818 & 0.493827 & 0.506173 & 110 & 671 & -561 & 1232 \\
 163 & 163^1 & \text{Y} & \text{Y} & -2 & 0 & 1.0000000 & 0.490798 & 0.509202 & 108 & 671 & -563 & 1234 \\
 164 & 2^2 41^1 & \text{N} & \text{N} & -7 & 2 & 1.2857143 & 0.487805 & 0.512195 & 101 & 671 & -570 & 1241 \\
 165 & 3^1 5^1 11^1 & \text{Y} & \text{N} & -16 & 0 & 1.0000000 & 0.484848 & 0.515152 & 85 & 671 & -586 & 1257 \\
 166 & 2^1 83^1 & \text{Y} & \text{N} & 5 & 0 & 1.0000000 & 0.487952 & 0.512048 & 90 & 676 & -586 & 1262 \\
 167 & 167^1 & \text{Y} & \text{Y} & -2 & 0 & 1.0000000 & 0.485030 & 0.514970 & 88 & 676 & -588 & 1264 \\
 168 & 2^3 3^1 7^1 & \text{N} & \text{N} & -48 & 32 & 1.3333333 & 0.482143 & 0.517857 & 40 & 676 & -636 & 1312 \\
 169 & 13^2 & \text{N} & \text{Y} & 2 & 0 & 1.5000000 & 0.485207 & 0.514793 & 42 & 678 & -636 & 1314 \\
 170 & 2^1 5^1 17^1 & \text{Y} & \text{N} & -16 & 0 & 1.0000000 & 0.482353 & 0.517647 & 26 & 678 & -652 & 1330 \\
 171 & 3^2 19^1 & \text{N} & \text{N} & -7 & 2 & 1.2857143 & 0.479532 & 0.520468 & 19 & 678 & -659 & 1337 \\
 172 & 2^2 43^1 & \text{N} & \text{N} & -7 & 2 & 1.2857143 & 0.476744 & 0.523256 & 12 & 678 & -666 & 1344 \\
 173 & 173^1 & \text{Y} & \text{Y} & -2 & 0 & 1.0000000 & 0.473988 & 0.526012 & 10 & 678 & -668 & 1346 \\
 174 & 2^1 3^1 29^1 & \text{Y} & \text{N} & -16 & 0 & 1.0000000 & 0.471264 & 0.528736 & -6 & 678 & -684 & 1362 \\
 175 & 5^2 7^1 & \text{N} & \text{N} & -7 & 2 & 1.2857143 & 0.468571 & 0.531429 & -13 & 678 & -691 & 1369 \\
 176 & 2^4 11^1 & \text{N} & \text{N} & -11 & 6 & 1.8181818 & 0.465909 & 0.534091 & -24 & 678 & -702 & 1380 \\
 177 & 3^1 59^1 & \text{Y} & \text{N} & 5 & 0 & 1.0000000 & 0.468927 & 0.531073 & -19 & 683 & -702 & 1385 \\
 178 & 2^1 89^1 & \text{Y} & \text{N} & 5 & 0 & 1.0000000 & 0.471910 & 0.528090 & -14 & 688 & -702 & 1390 \\
 179 & 179^1 & \text{Y} & \text{Y} & -2 & 0 & 1.0000000 & 0.469274 & 0.530726 & -16 & 688 & -704 & 1392 \\
 180 & 2^2 3^2 5^1 & \text{N} & \text{N} & -74 & 58 & 1.2162162 & 0.466667 & 0.533333 & -90 & 688 & -778 & 1466 \\
 181 & 181^1 & \text{Y} & \text{Y} & -2 & 0 & 1.0000000 & 0.464088 & 0.535912 & -92 & 688 & -780 & 1468 \\
 182 & 2^1 7^1 13^1 & \text{Y} & \text{N} & -16 & 0 & 1.0000000 & 0.461538 & 0.538462 & -108 & 688 & -796 & 1484 \\
 183 & 3^1 61^1 & \text{Y} & \text{N} & 5 & 0 & 1.0000000 & 0.464481 & 0.535519 & -103 & 693 & -796 & 1489 \\
 184 & 2^3 23^1 & \text{N} & \text{N} & 9 & 4 & 1.5555556 & 0.467391 & 0.532609 & -94 & 702 & -796 & 1498 \\
 185 & 5^1 37^1 & \text{Y} & \text{N} & 5 & 0 & 1.0000000 & 0.470270 & 0.529730 & -89 & 707 & -796 & 1503 \\
 186 & 2^1 3^1 31^1 & \text{Y} & \text{N} & -16 & 0 & 1.0000000 & 0.467742 & 0.532258 & -105 & 707 & -812 & 1519 \\
 187 & 11^1 17^1 & \text{Y} & \text{N} & 5 & 0 & 1.0000000 & 0.470588 & 0.529412 & -100 & 712 & -812 & 1524 \\
 188 & 2^2 47^1 & \text{N} & \text{N} & -7 & 2 & 1.2857143 & 0.468085 & 0.531915 & -107 & 712 & -819 & 1531 \\
 189 & 3^3 7^1 & \text{N} & \text{N} & 9 & 4 & 1.5555556 & 0.470899 & 0.529101 & -98 & 721 & -819 & 1540 \\
 190 & 2^1 5^1 19^1 & \text{Y} & \text{N} & -16 & 0 & 1.0000000 & 0.468421 & 0.531579 & -114 & 721 & -835 & 1556 \\
 191 & 191^1 & \text{Y} & \text{Y} & -2 & 0 & 1.0000000 & 0.465969 & 0.534031 & -116 & 721 & -837 & 1558 \\
 192 & 2^6 3^1 & \text{N} & \text{N} & -15 & 10 & 2.3333333 & 0.463542 & 0.536458 & -131 & 721 & -852 & 1573 \\
 193 & 193^1 & \text{Y} & \text{Y} & -2 & 0 & 1.0000000 & 0.461140 & 0.538860 & -133 & 721 & -854 & 1575 \\
 194 & 2^1 97^1 & \text{Y} & \text{N} & 5 & 0 & 1.0000000 & 0.463918 & 0.536082 & -128 & 726 & -854 & 1580 \\
 195 & 3^1 5^1 13^1 & \text{Y} & \text{N} & -16 & 0 & 1.0000000 & 0.461538 & 0.538462 & -144 & 726 & -870 & 1596 \\
 196 & 2^2 7^2 & \text{N} & \text{N} & 14 & 9 & 1.3571429 & 0.464286 & 0.535714 & -130 & 740 & -870 & 1610 \\
 197 & 197^1 & \text{Y} & \text{Y} & -2 & 0 & 1.0000000 & 0.461929 & 0.538071 & -132 & 740 & -872 & 1612 \\
 198 & 2^1 3^2 11^1 & \text{N} & \text{N} & 30 & 14 & 1.1666667 & 0.464646 & 0.535354 & -102 & 770 & -872 & 1642 \\
 199 & 199^1 & \text{Y} & \text{Y} & -2 & 0 & 1.0000000 & 0.462312 & 0.537688 & -104 & 770 & -874 & 1644 \\
 200 & 2^3 5^2 & \text{N} & \text{N} & -23 & 18 & 1.4782609 & 0.460000 & 0.540000 & -127 & 770 & -897 & 1667 \\
\end{array}
}
\end{equation*}
\clearpage 

\end{table} 

\newpage
\begin{table}[ht]

\centering

\tiny
\begin{equation*}
\boxed{
\begin{array}{cc|cc|ccc|cc|cccc}
 n & \mathbf{Primes} & \mathbf{Sqfree} & \mathbf{PPower} & g^{-1}(n) & 
 \lambda(n) g^{-1}(n) - \widehat{f}_1(n) & 
 \frac{\sum_{d|n} C_{\Omega}(d)}{|g^{-1}(n)|} & 
 \mathcal{L}_{+}(n) & \mathcal{L}_{-}(n) & 
 G^{-1}(n) & G^{-1}_{+}(n) & G^{-1}_{-}(n) & |G^{-1}|(n) \\ \hline 
 201 & 3^1 67^1 & \text{Y} & \text{N} & 5 & 0 & 1.0000000 & 0.462687 & 0.537313 & -122 & 775 & -897 & 1672 \\
 202 & 2^1 101^1 & \text{Y} & \text{N} & 5 & 0 & 1.0000000 & 0.465347 & 0.534653 & -117 & 780 & -897 & 1677 \\
 203 & 7^1 29^1 & \text{Y} & \text{N} & 5 & 0 & 1.0000000 & 0.467980 & 0.532020 & -112 & 785 & -897 & 1682 \\
 204 & 2^2 3^1 17^1 & \text{N} & \text{N} & 30 & 14 & 1.1666667 & 0.470588 & 0.529412 & -82 & 815 & -897 & 1712 \\
 205 & 5^1 41^1 & \text{Y} & \text{N} & 5 & 0 & 1.0000000 & 0.473171 & 0.526829 & -77 & 820 & -897 & 1717 \\
 206 & 2^1 103^1 & \text{Y} & \text{N} & 5 & 0 & 1.0000000 & 0.475728 & 0.524272 & -72 & 825 & -897 & 1722 \\
 207 & 3^2 23^1 & \text{N} & \text{N} & -7 & 2 & 1.2857143 & 0.473430 & 0.526570 & -79 & 825 & -904 & 1729 \\
 208 & 2^4 13^1 & \text{N} & \text{N} & -11 & 6 & 1.8181818 & 0.471154 & 0.528846 & -90 & 825 & -915 & 1740 \\
 209 & 11^1 19^1 & \text{Y} & \text{N} & 5 & 0 & 1.0000000 & 0.473684 & 0.526316 & -85 & 830 & -915 & 1745 \\
 210 & 2^1 3^1 5^1 7^1 & \text{Y} & \text{N} & 65 & 0 & 1.0000000 & 0.476190 & 0.523810 & -20 & 895 & -915 & 1810 \\
 211 & 211^1 & \text{Y} & \text{Y} & -2 & 0 & 1.0000000 & 0.473934 & 0.526066 & -22 & 895 & -917 & 1812 \\
 212 & 2^2 53^1 & \text{N} & \text{N} & -7 & 2 & 1.2857143 & 0.471698 & 0.528302 & -29 & 895 & -924 & 1819 \\
 213 & 3^1 71^1 & \text{Y} & \text{N} & 5 & 0 & 1.0000000 & 0.474178 & 0.525822 & -24 & 900 & -924 & 1824 \\
 214 & 2^1 107^1 & \text{Y} & \text{N} & 5 & 0 & 1.0000000 & 0.476636 & 0.523364 & -19 & 905 & -924 & 1829 \\
 215 & 5^1 43^1 & \text{Y} & \text{N} & 5 & 0 & 1.0000000 & 0.479070 & 0.520930 & -14 & 910 & -924 & 1834 \\
 216 & 2^3 3^3 & \text{N} & \text{N} & 46 & 41 & 1.5000000 & 0.481481 & 0.518519 & 32 & 956 & -924 & 1880 \\
 217 & 7^1 31^1 & \text{Y} & \text{N} & 5 & 0 & 1.0000000 & 0.483871 & 0.516129 & 37 & 961 & -924 & 1885 \\
 218 & 2^1 109^1 & \text{Y} & \text{N} & 5 & 0 & 1.0000000 & 0.486239 & 0.513761 & 42 & 966 & -924 & 1890 \\
 219 & 3^1 73^1 & \text{Y} & \text{N} & 5 & 0 & 1.0000000 & 0.488584 & 0.511416 & 47 & 971 & -924 & 1895 \\
 220 & 2^2 5^1 11^1 & \text{N} & \text{N} & 30 & 14 & 1.1666667 & 0.490909 & 0.509091 & 77 & 1001 & -924 & 1925 \\
 221 & 13^1 17^1 & \text{Y} & \text{N} & 5 & 0 & 1.0000000 & 0.493213 & 0.506787 & 82 & 1006 & -924 & 1930 \\
 222 & 2^1 3^1 37^1 & \text{Y} & \text{N} & -16 & 0 & 1.0000000 & 0.490991 & 0.509009 & 66 & 1006 & -940 & 1946 \\
 223 & 223^1 & \text{Y} & \text{Y} & -2 & 0 & 1.0000000 & 0.488789 & 0.511211 & 64 & 1006 & -942 & 1948 \\
 224 & 2^5 7^1 & \text{N} & \text{N} & 13 & 8 & 2.0769231 & 0.491071 & 0.508929 & 77 & 1019 & -942 & 1961 \\
 225 & 3^2 5^2 & \text{N} & \text{N} & 14 & 9 & 1.3571429 & 0.493333 & 0.506667 & 91 & 1033 & -942 & 1975 \\
 226 & 2^1 113^1 & \text{Y} & \text{N} & 5 & 0 & 1.0000000 & 0.495575 & 0.504425 & 96 & 1038 & -942 & 1980 \\
 227 & 227^1 & \text{Y} & \text{Y} & -2 & 0 & 1.0000000 & 0.493392 & 0.506608 & 94 & 1038 & -944 & 1982 \\
 228 & 2^2 3^1 19^1 & \text{N} & \text{N} & 30 & 14 & 1.1666667 & 0.495614 & 0.504386 & 124 & 1068 & -944 & 2012 \\
 229 & 229^1 & \text{Y} & \text{Y} & -2 & 0 & 1.0000000 & 0.493450 & 0.506550 & 122 & 1068 & -946 & 2014 \\
 230 & 2^1 5^1 23^1 & \text{Y} & \text{N} & -16 & 0 & 1.0000000 & 0.491304 & 0.508696 & 106 & 1068 & -962 & 2030 \\
 231 & 3^1 7^1 11^1 & \text{Y} & \text{N} & -16 & 0 & 1.0000000 & 0.489177 & 0.510823 & 90 & 1068 & -978 & 2046 \\
 232 & 2^3 29^1 & \text{N} & \text{N} & 9 & 4 & 1.5555556 & 0.491379 & 0.508621 & 99 & 1077 & -978 & 2055 \\
 233 & 233^1 & \text{Y} & \text{Y} & -2 & 0 & 1.0000000 & 0.489270 & 0.510730 & 97 & 1077 & -980 & 2057 \\
 234 & 2^1 3^2 13^1 & \text{N} & \text{N} & 30 & 14 & 1.1666667 & 0.491453 & 0.508547 & 127 & 1107 & -980 & 2087 \\
 235 & 5^1 47^1 & \text{Y} & \text{N} & 5 & 0 & 1.0000000 & 0.493617 & 0.506383 & 132 & 1112 & -980 & 2092 \\
 236 & 2^2 59^1 & \text{N} & \text{N} & -7 & 2 & 1.2857143 & 0.491525 & 0.508475 & 125 & 1112 & -987 & 2099 \\
 237 & 3^1 79^1 & \text{Y} & \text{N} & 5 & 0 & 1.0000000 & 0.493671 & 0.506329 & 130 & 1117 & -987 & 2104 \\
 238 & 2^1 7^1 17^1 & \text{Y} & \text{N} & -16 & 0 & 1.0000000 & 0.491597 & 0.508403 & 114 & 1117 & -1003 & 2120 \\
 239 & 239^1 & \text{Y} & \text{Y} & -2 & 0 & 1.0000000 & 0.489540 & 0.510460 & 112 & 1117 & -1005 & 2122 \\
 240 & 2^4 3^1 5^1 & \text{N} & \text{N} & 70 & 54 & 1.5000000 & 0.491667 & 0.508333 & 182 & 1187 & -1005 & 2192 \\
 241 & 241^1 & \text{Y} & \text{Y} & -2 & 0 & 1.0000000 & 0.489627 & 0.510373 & 180 & 1187 & -1007 & 2194 \\
 242 & 2^1 11^2 & \text{N} & \text{N} & -7 & 2 & 1.2857143 & 0.487603 & 0.512397 & 173 & 1187 & -1014 & 2201 \\
 243 & 3^5 & \text{N} & \text{Y} & -2 & 0 & 3.0000000 & 0.485597 & 0.514403 & 171 & 1187 & -1016 & 2203 \\
 244 & 2^2 61^1 & \text{N} & \text{N} & -7 & 2 & 1.2857143 & 0.483607 & 0.516393 & 164 & 1187 & -1023 & 2210 \\
 245 & 5^1 7^2 & \text{N} & \text{N} & -7 & 2 & 1.2857143 & 0.481633 & 0.518367 & 157 & 1187 & -1030 & 2217 \\
 246 & 2^1 3^1 41^1 & \text{Y} & \text{N} & -16 & 0 & 1.0000000 & 0.479675 & 0.520325 & 141 & 1187 & -1046 & 2233 \\
 247 & 13^1 19^1 & \text{Y} & \text{N} & 5 & 0 & 1.0000000 & 0.481781 & 0.518219 & 146 & 1192 & -1046 & 2238 \\
 248 & 2^3 31^1 & \text{N} & \text{N} & 9 & 4 & 1.5555556 & 0.483871 & 0.516129 & 155 & 1201 & -1046 & 2247 \\
 249 & 3^1 83^1 & \text{Y} & \text{N} & 5 & 0 & 1.0000000 & 0.485944 & 0.514056 & 160 & 1206 & -1046 & 2252 \\
 250 & 2^1 5^3 & \text{N} & \text{N} & 9 & 4 & 1.5555556 & 0.488000 & 0.512000 & 169 & 1215 & -1046 & 2261 \\
 251 & 251^1 & \text{Y} & \text{Y} & -2 & 0 & 1.0000000 & 0.486056 & 0.513944 & 167 & 1215 & -1048 & 2263 \\
 252 & 2^2 3^2 7^1 & \text{N} & \text{N} & -74 & 58 & 1.2162162 & 0.484127 & 0.515873 & 93 & 1215 & -1122 & 2337 \\
 253 & 11^1 23^1 & \text{Y} & \text{N} & 5 & 0 & 1.0000000 & 0.486166 & 0.513834 & 98 & 1220 & -1122 & 2342 \\
 254 & 2^1 127^1 & \text{Y} & \text{N} & 5 & 0 & 1.0000000 & 0.488189 & 0.511811 & 103 & 1225 & -1122 & 2347 \\
 255 & 3^1 5^1 17^1 & \text{Y} & \text{N} & -16 & 0 & 1.0000000 & 0.486275 & 0.513725 & 87 & 1225 & -1138 & 2363 \\
 256 & 2^8 & \text{N} & \text{Y} & 2 & 0 & 4.5000000 & 0.488281 & 0.511719 & 89 & 1227 & -1138 & 2365 \\
 257 & 257^1 & \text{Y} & \text{Y} & -2 & 0 & 1.0000000 & 0.486381 & 0.513619 & 87 & 1227 & -1140 & 2367 \\
 258 & 2^1 3^1 43^1 & \text{Y} & \text{N} & -16 & 0 & 1.0000000 & 0.484496 & 0.515504 & 71 & 1227 & -1156 & 2383 \\
 259 & 7^1 37^1 & \text{Y} & \text{N} & 5 & 0 & 1.0000000 & 0.486486 & 0.513514 & 76 & 1232 & -1156 & 2388 \\
 260 & 2^2 5^1 13^1 & \text{N} & \text{N} & 30 & 14 & 1.1666667 & 0.488462 & 0.511538 & 106 & 1262 & -1156 & 2418 \\
 261 & 3^2 29^1 & \text{N} & \text{N} & -7 & 2 & 1.2857143 & 0.486590 & 0.513410 & 99 & 1262 & -1163 & 2425 \\
 262 & 2^1 131^1 & \text{Y} & \text{N} & 5 & 0 & 1.0000000 & 0.488550 & 0.511450 & 104 & 1267 & -1163 & 2430 \\
 263 & 263^1 & \text{Y} & \text{Y} & -2 & 0 & 1.0000000 & 0.486692 & 0.513308 & 102 & 1267 & -1165 & 2432 \\
 264 & 2^3 3^1 11^1 & \text{N} & \text{N} & -48 & 32 & 1.3333333 & 0.484848 & 0.515152 & 54 & 1267 & -1213 & 2480 \\
 265 & 5^1 53^1 & \text{Y} & \text{N} & 5 & 0 & 1.0000000 & 0.486792 & 0.513208 & 59 & 1272 & -1213 & 2485 \\
 266 & 2^1 7^1 19^1 & \text{Y} & \text{N} & -16 & 0 & 1.0000000 & 0.484962 & 0.515038 & 43 & 1272 & -1229 & 2501 \\
 267 & 3^1 89^1 & \text{Y} & \text{N} & 5 & 0 & 1.0000000 & 0.486891 & 0.513109 & 48 & 1277 & -1229 & 2506 \\
 268 & 2^2 67^1 & \text{N} & \text{N} & -7 & 2 & 1.2857143 & 0.485075 & 0.514925 & 41 & 1277 & -1236 & 2513 \\
 269 & 269^1 & \text{Y} & \text{Y} & -2 & 0 & 1.0000000 & 0.483271 & 0.516729 & 39 & 1277 & -1238 & 2515 \\
 270 & 2^1 3^3 5^1 & \text{N} & \text{N} & -48 & 32 & 1.3333333 & 0.481481 & 0.518519 & -9 & 1277 & -1286 & 2563 \\
 271 & 271^1 & \text{Y} & \text{Y} & -2 & 0 & 1.0000000 & 0.479705 & 0.520295 & -11 & 1277 & -1288 & 2565 \\
 272 & 2^4 17^1 & \text{N} & \text{N} & -11 & 6 & 1.8181818 & 0.477941 & 0.522059 & -22 & 1277 & -1299 & 2576 \\
 273 & 3^1 7^1 13^1 & \text{Y} & \text{N} & -16 & 0 & 1.0000000 & 0.476190 & 0.523810 & -38 & 1277 & -1315 & 2592 \\
 274 & 2^1 137^1 & \text{Y} & \text{N} & 5 & 0 & 1.0000000 & 0.478102 & 0.521898 & -33 & 1282 & -1315 & 2597 \\
 275 & 5^2 11^1 & \text{N} & \text{N} & -7 & 2 & 1.2857143 & 0.476364 & 0.523636 & -40 & 1282 & -1322 & 2604 \\
 276 & 2^2 3^1 23^1 & \text{N} & \text{N} & 30 & 14 & 1.1666667 & 0.478261 & 0.521739 & -10 & 1312 & -1322 & 2634 \\
 277 & 277^1 & \text{Y} & \text{Y} & -2 & 0 & 1.0000000 & 0.476534 & 0.523466 & -12 & 1312 & -1324 & 2636 \\
\end{array}
}
\end{equation*}
\clearpage 

\end{table} 

\newpage
\begin{table}[ht]

\centering

\tiny
\begin{equation*}
\boxed{
\begin{array}{cc|cc|ccc|cc|cccc}
 n & \mathbf{Primes} & \mathbf{Sqfree} & \mathbf{PPower} & g^{-1}(n) & 
 \lambda(n) g^{-1}(n) - \widehat{f}_1(n) & 
 \frac{\sum_{d|n} C_{\Omega}(d)}{|g^{-1}(n)|} & 
 \mathcal{L}_{+}(n) & \mathcal{L}_{-}(n) & 
 G^{-1}(n) & G^{-1}_{+}(n) & G^{-1}_{-}(n) & |G^{-1}|(n) \\ \hline 
 278 & 2^1 139^1 & \text{Y} & \text{N} & 5 & 0 & 1.0000000 & 0.478417 & 0.521583 & -7 & 1317 & -1324 & 2641 \\
 279 & 3^2 31^1 & \text{N} & \text{N} & -7 & 2 & 1.2857143 & 0.476703 & 0.523297 & -14 & 1317 & -1331 & 2648 \\
 280 & 2^3 5^1 7^1 & \text{N} & \text{N} & -48 & 32 & 1.3333333 & 0.475000 & 0.525000 & -62 & 1317 & -1379 & 2696 \\
 281 & 281^1 & \text{Y} & \text{Y} & -2 & 0 & 1.0000000 & 0.473310 & 0.526690 & -64 & 1317 & -1381 & 2698 \\
 282 & 2^1 3^1 47^1 & \text{Y} & \text{N} & -16 & 0 & 1.0000000 & 0.471631 & 0.528369 & -80 & 1317 & -1397 & 2714 \\
 283 & 283^1 & \text{Y} & \text{Y} & -2 & 0 & 1.0000000 & 0.469965 & 0.530035 & -82 & 1317 & -1399 & 2716 \\
 284 & 2^2 71^1 & \text{N} & \text{N} & -7 & 2 & 1.2857143 & 0.468310 & 0.531690 & -89 & 1317 & -1406 & 2723 \\
 285 & 3^1 5^1 19^1 & \text{Y} & \text{N} & -16 & 0 & 1.0000000 & 0.466667 & 0.533333 & -105 & 1317 & -1422 & 2739 \\
 286 & 2^1 11^1 13^1 & \text{Y} & \text{N} & -16 & 0 & 1.0000000 & 0.465035 & 0.534965 & -121 & 1317 & -1438 & 2755 \\
 287 & 7^1 41^1 & \text{Y} & \text{N} & 5 & 0 & 1.0000000 & 0.466899 & 0.533101 & -116 & 1322 & -1438 & 2760 \\
 288 & 2^5 3^2 & \text{N} & \text{N} & -47 & 42 & 1.7659574 & 0.465278 & 0.534722 & -163 & 1322 & -1485 & 2807 \\
 289 & 17^2 & \text{N} & \text{Y} & 2 & 0 & 1.5000000 & 0.467128 & 0.532872 & -161 & 1324 & -1485 & 2809 \\
 290 & 2^1 5^1 29^1 & \text{Y} & \text{N} & -16 & 0 & 1.0000000 & 0.465517 & 0.534483 & -177 & 1324 & -1501 & 2825 \\
 291 & 3^1 97^1 & \text{Y} & \text{N} & 5 & 0 & 1.0000000 & 0.467354 & 0.532646 & -172 & 1329 & -1501 & 2830 \\
 292 & 2^2 73^1 & \text{N} & \text{N} & -7 & 2 & 1.2857143 & 0.465753 & 0.534247 & -179 & 1329 & -1508 & 2837 \\
 293 & 293^1 & \text{Y} & \text{Y} & -2 & 0 & 1.0000000 & 0.464164 & 0.535836 & -181 & 1329 & -1510 & 2839 \\
 294 & 2^1 3^1 7^2 & \text{N} & \text{N} & 30 & 14 & 1.1666667 & 0.465986 & 0.534014 & -151 & 1359 & -1510 & 2869 \\
 295 & 5^1 59^1 & \text{Y} & \text{N} & 5 & 0 & 1.0000000 & 0.467797 & 0.532203 & -146 & 1364 & -1510 & 2874 \\
 296 & 2^3 37^1 & \text{N} & \text{N} & 9 & 4 & 1.5555556 & 0.469595 & 0.530405 & -137 & 1373 & -1510 & 2883 \\
 297 & 3^3 11^1 & \text{N} & \text{N} & 9 & 4 & 1.5555556 & 0.471380 & 0.528620 & -128 & 1382 & -1510 & 2892 \\
 298 & 2^1 149^1 & \text{Y} & \text{N} & 5 & 0 & 1.0000000 & 0.473154 & 0.526846 & -123 & 1387 & -1510 & 2897 \\
 299 & 13^1 23^1 & \text{Y} & \text{N} & 5 & 0 & 1.0000000 & 0.474916 & 0.525084 & -118 & 1392 & -1510 & 2902 \\
 300 & 2^2 3^1 5^2 & \text{N} & \text{N} & -74 & 58 & 1.2162162 & 0.473333 & 0.526667 & -192 & 1392 & -1584 & 2976 \\
 301 & 7^1 43^1 & \text{Y} & \text{N} & 5 & 0 & 1.0000000 & 0.475083 & 0.524917 & -187 & 1397 & -1584 & 2981 \\
 302 & 2^1 151^1 & \text{Y} & \text{N} & 5 & 0 & 1.0000000 & 0.476821 & 0.523179 & -182 & 1402 & -1584 & 2986 \\
 303 & 3^1 101^1 & \text{Y} & \text{N} & 5 & 0 & 1.0000000 & 0.478548 & 0.521452 & -177 & 1407 & -1584 & 2991 \\
 304 & 2^4 19^1 & \text{N} & \text{N} & -11 & 6 & 1.8181818 & 0.476974 & 0.523026 & -188 & 1407 & -1595 & 3002 \\
 305 & 5^1 61^1 & \text{Y} & \text{N} & 5 & 0 & 1.0000000 & 0.478689 & 0.521311 & -183 & 1412 & -1595 & 3007 \\
 306 & 2^1 3^2 17^1 & \text{N} & \text{N} & 30 & 14 & 1.1666667 & 0.480392 & 0.519608 & -153 & 1442 & -1595 & 3037 \\
 307 & 307^1 & \text{Y} & \text{Y} & -2 & 0 & 1.0000000 & 0.478827 & 0.521173 & -155 & 1442 & -1597 & 3039 \\
 308 & 2^2 7^1 11^1 & \text{N} & \text{N} & 30 & 14 & 1.1666667 & 0.480519 & 0.519481 & -125 & 1472 & -1597 & 3069 \\
 309 & 3^1 103^1 & \text{Y} & \text{N} & 5 & 0 & 1.0000000 & 0.482201 & 0.517799 & -120 & 1477 & -1597 & 3074 \\
 310 & 2^1 5^1 31^1 & \text{Y} & \text{N} & -16 & 0 & 1.0000000 & 0.480645 & 0.519355 & -136 & 1477 & -1613 & 3090 \\
 311 & 311^1 & \text{Y} & \text{Y} & -2 & 0 & 1.0000000 & 0.479100 & 0.520900 & -138 & 1477 & -1615 & 3092 \\
 312 & 2^3 3^1 13^1 & \text{N} & \text{N} & -48 & 32 & 1.3333333 & 0.477564 & 0.522436 & -186 & 1477 & -1663 & 3140 \\
 313 & 313^1 & \text{Y} & \text{Y} & -2 & 0 & 1.0000000 & 0.476038 & 0.523962 & -188 & 1477 & -1665 & 3142 \\
 314 & 2^1 157^1 & \text{Y} & \text{N} & 5 & 0 & 1.0000000 & 0.477707 & 0.522293 & -183 & 1482 & -1665 & 3147 \\
 315 & 3^2 5^1 7^1 & \text{N} & \text{N} & 30 & 14 & 1.1666667 & 0.479365 & 0.520635 & -153 & 1512 & -1665 & 3177 \\
 316 & 2^2 79^1 & \text{N} & \text{N} & -7 & 2 & 1.2857143 & 0.477848 & 0.522152 & -160 & 1512 & -1672 & 3184 \\
 317 & 317^1 & \text{Y} & \text{Y} & -2 & 0 & 1.0000000 & 0.476341 & 0.523659 & -162 & 1512 & -1674 & 3186 \\
 318 & 2^1 3^1 53^1 & \text{Y} & \text{N} & -16 & 0 & 1.0000000 & 0.474843 & 0.525157 & -178 & 1512 & -1690 & 3202 \\
 319 & 11^1 29^1 & \text{Y} & \text{N} & 5 & 0 & 1.0000000 & 0.476489 & 0.523511 & -173 & 1517 & -1690 & 3207 \\
 320 & 2^6 5^1 & \text{N} & \text{N} & -15 & 10 & 2.3333333 & 0.475000 & 0.525000 & -188 & 1517 & -1705 & 3222 \\
 321 & 3^1 107^1 & \text{Y} & \text{N} & 5 & 0 & 1.0000000 & 0.476636 & 0.523364 & -183 & 1522 & -1705 & 3227 \\
 322 & 2^1 7^1 23^1 & \text{Y} & \text{N} & -16 & 0 & 1.0000000 & 0.475155 & 0.524845 & -199 & 1522 & -1721 & 3243 \\
 323 & 17^1 19^1 & \text{Y} & \text{N} & 5 & 0 & 1.0000000 & 0.476780 & 0.523220 & -194 & 1527 & -1721 & 3248 \\
 324 & 2^2 3^4 & \text{N} & \text{N} & 34 & 29 & 1.6176471 & 0.478395 & 0.521605 & -160 & 1561 & -1721 & 3282 \\
 325 & 5^2 13^1 & \text{N} & \text{N} & -7 & 2 & 1.2857143 & 0.476923 & 0.523077 & -167 & 1561 & -1728 & 3289 \\
 326 & 2^1 163^1 & \text{Y} & \text{N} & 5 & 0 & 1.0000000 & 0.478528 & 0.521472 & -162 & 1566 & -1728 & 3294 \\
 327 & 3^1 109^1 & \text{Y} & \text{N} & 5 & 0 & 1.0000000 & 0.480122 & 0.519878 & -157 & 1571 & -1728 & 3299 \\
 328 & 2^3 41^1 & \text{N} & \text{N} & 9 & 4 & 1.5555556 & 0.481707 & 0.518293 & -148 & 1580 & -1728 & 3308 \\
 329 & 7^1 47^1 & \text{Y} & \text{N} & 5 & 0 & 1.0000000 & 0.483283 & 0.516717 & -143 & 1585 & -1728 & 3313 \\
 330 & 2^1 3^1 5^1 11^1 & \text{Y} & \text{N} & 65 & 0 & 1.0000000 & 0.484848 & 0.515152 & -78 & 1650 & -1728 & 3378 \\
 331 & 331^1 & \text{Y} & \text{Y} & -2 & 0 & 1.0000000 & 0.483384 & 0.516616 & -80 & 1650 & -1730 & 3380 \\
 332 & 2^2 83^1 & \text{N} & \text{N} & -7 & 2 & 1.2857143 & 0.481928 & 0.518072 & -87 & 1650 & -1737 & 3387 \\
 333 & 3^2 37^1 & \text{N} & \text{N} & -7 & 2 & 1.2857143 & 0.480480 & 0.519520 & -94 & 1650 & -1744 & 3394 \\
 334 & 2^1 167^1 & \text{Y} & \text{N} & 5 & 0 & 1.0000000 & 0.482036 & 0.517964 & -89 & 1655 & -1744 & 3399 \\
 335 & 5^1 67^1 & \text{Y} & \text{N} & 5 & 0 & 1.0000000 & 0.483582 & 0.516418 & -84 & 1660 & -1744 & 3404 \\
 336 & 2^4 3^1 7^1 & \text{N} & \text{N} & 70 & 54 & 1.5000000 & 0.485119 & 0.514881 & -14 & 1730 & -1744 & 3474 \\
 337 & 337^1 & \text{Y} & \text{Y} & -2 & 0 & 1.0000000 & 0.483680 & 0.516320 & -16 & 1730 & -1746 & 3476 \\
 338 & 2^1 13^2 & \text{N} & \text{N} & -7 & 2 & 1.2857143 & 0.482249 & 0.517751 & -23 & 1730 & -1753 & 3483 \\
 339 & 3^1 113^1 & \text{Y} & \text{N} & 5 & 0 & 1.0000000 & 0.483776 & 0.516224 & -18 & 1735 & -1753 & 3488 \\
 340 & 2^2 5^1 17^1 & \text{N} & \text{N} & 30 & 14 & 1.1666667 & 0.485294 & 0.514706 & 12 & 1765 & -1753 & 3518 \\
 341 & 11^1 31^1 & \text{Y} & \text{N} & 5 & 0 & 1.0000000 & 0.486804 & 0.513196 & 17 & 1770 & -1753 & 3523 \\
 342 & 2^1 3^2 19^1 & \text{N} & \text{N} & 30 & 14 & 1.1666667 & 0.488304 & 0.511696 & 47 & 1800 & -1753 & 3553 \\
 343 & 7^3 & \text{N} & \text{Y} & -2 & 0 & 2.0000000 & 0.486880 & 0.513120 & 45 & 1800 & -1755 & 3555 \\
 344 & 2^3 43^1 & \text{N} & \text{N} & 9 & 4 & 1.5555556 & 0.488372 & 0.511628 & 54 & 1809 & -1755 & 3564 \\
 345 & 3^1 5^1 23^1 & \text{Y} & \text{N} & -16 & 0 & 1.0000000 & 0.486957 & 0.513043 & 38 & 1809 & -1771 & 3580 \\
 346 & 2^1 173^1 & \text{Y} & \text{N} & 5 & 0 & 1.0000000 & 0.488439 & 0.511561 & 43 & 1814 & -1771 & 3585 \\
 347 & 347^1 & \text{Y} & \text{Y} & -2 & 0 & 1.0000000 & 0.487032 & 0.512968 & 41 & 1814 & -1773 & 3587 \\
 348 & 2^2 3^1 29^1 & \text{N} & \text{N} & 30 & 14 & 1.1666667 & 0.488506 & 0.511494 & 71 & 1844 & -1773 & 3617 \\
 349 & 349^1 & \text{Y} & \text{Y} & -2 & 0 & 1.0000000 & 0.487106 & 0.512894 & 69 & 1844 & -1775 & 3619 \\
 350 & 2^1 5^2 7^1 & \text{N} & \text{N} & 30 & 14 & 1.1666667 & 0.488571 & 0.511429 & 99 & 1874 & -1775 & 3649 \\
\end{array}
}
\end{equation*}
\clearpage 

\end{table} 

\newpage
\begin{table}[ht]

\centering
\tiny
\begin{equation*}
\boxed{
\begin{array}{cc|cc|ccc|cc|cccc}
 n & \mathbf{Primes} & \mathbf{Sqfree} & \mathbf{PPower} & g^{-1}(n) & 
 \lambda(n) g^{-1}(n) - \widehat{f}_1(n) & 
 \frac{\sum_{d|n} C_{\Omega}(d)}{|g^{-1}(n)|} & 
 \mathcal{L}_{+}(n) & \mathcal{L}_{-}(n) & 
 G^{-1}(n) & G^{-1}_{+}(n) & G^{-1}_{-}(n) & |G^{-1}|(n) \\ \hline 
 351 & 3^3 13^1 & \text{N} & \text{N} & 9 & 4 & 1.5555556 & 0.490028 & 0.509972 & 108 & 1883 & -1775 & 3658 \\
 352 & 2^5 11^1 & \text{N} & \text{N} & 13 & 8 & 2.0769231 & 0.491477 & 0.508523 & 121 & 1896 & -1775 & 3671 \\
 353 & 353^1 & \text{Y} & \text{Y} & -2 & 0 & 1.0000000 & 0.490085 & 0.509915 & 119 & 1896 & -1777 & 3673 \\
 354 & 2^1 3^1 59^1 & \text{Y} & \text{N} & -16 & 0 & 1.0000000 & 0.488701 & 0.511299 & 103 & 1896 & -1793 & 3689 \\
 355 & 5^1 71^1 & \text{Y} & \text{N} & 5 & 0 & 1.0000000 & 0.490141 & 0.509859 & 108 & 1901 & -1793 & 3694 \\
 356 & 2^2 89^1 & \text{N} & \text{N} & -7 & 2 & 1.2857143 & 0.488764 & 0.511236 & 101 & 1901 & -1800 & 3701 \\
 357 & 3^1 7^1 17^1 & \text{Y} & \text{N} & -16 & 0 & 1.0000000 & 0.487395 & 0.512605 & 85 & 1901 & -1816 & 3717 \\
 358 & 2^1 179^1 & \text{Y} & \text{N} & 5 & 0 & 1.0000000 & 0.488827 & 0.511173 & 90 & 1906 & -1816 & 3722 \\
 359 & 359^1 & \text{Y} & \text{Y} & -2 & 0 & 1.0000000 & 0.487465 & 0.512535 & 88 & 1906 & -1818 & 3724 \\
 360 & 2^3 3^2 5^1 & \text{N} & \text{N} & 145 & 129 & 1.3034483 & 0.488889 & 0.511111 & 233 & 2051 & -1818 & 3869 \\
 361 & 19^2 & \text{N} & \text{Y} & 2 & 0 & 1.5000000 & 0.490305 & 0.509695 & 235 & 2053 & -1818 & 3871 \\
 362 & 2^1 181^1 & \text{Y} & \text{N} & 5 & 0 & 1.0000000 & 0.491713 & 0.508287 & 240 & 2058 & -1818 & 3876 \\
 363 & 3^1 11^2 & \text{N} & \text{N} & -7 & 2 & 1.2857143 & 0.490358 & 0.509642 & 233 & 2058 & -1825 & 3883 \\
 364 & 2^2 7^1 13^1 & \text{N} & \text{N} & 30 & 14 & 1.1666667 & 0.491758 & 0.508242 & 263 & 2088 & -1825 & 3913 \\
 365 & 5^1 73^1 & \text{Y} & \text{N} & 5 & 0 & 1.0000000 & 0.493151 & 0.506849 & 268 & 2093 & -1825 & 3918 \\
 366 & 2^1 3^1 61^1 & \text{Y} & \text{N} & -16 & 0 & 1.0000000 & 0.491803 & 0.508197 & 252 & 2093 & -1841 & 3934 \\
 367 & 367^1 & \text{Y} & \text{Y} & -2 & 0 & 1.0000000 & 0.490463 & 0.509537 & 250 & 2093 & -1843 & 3936 \\
 368 & 2^4 23^1 & \text{N} & \text{N} & -11 & 6 & 1.8181818 & 0.489130 & 0.510870 & 239 & 2093 & -1854 & 3947 \\
 369 & 3^2 41^1 & \text{N} & \text{N} & -7 & 2 & 1.2857143 & 0.487805 & 0.512195 & 232 & 2093 & -1861 & 3954 \\
 370 & 2^1 5^1 37^1 & \text{Y} & \text{N} & -16 & 0 & 1.0000000 & 0.486486 & 0.513514 & 216 & 2093 & -1877 & 3970 \\
 371 & 7^1 53^1 & \text{Y} & \text{N} & 5 & 0 & 1.0000000 & 0.487871 & 0.512129 & 221 & 2098 & -1877 & 3975 \\
 372 & 2^2 3^1 31^1 & \text{N} & \text{N} & 30 & 14 & 1.1666667 & 0.489247 & 0.510753 & 251 & 2128 & -1877 & 4005 \\
 373 & 373^1 & \text{Y} & \text{Y} & -2 & 0 & 1.0000000 & 0.487936 & 0.512064 & 249 & 2128 & -1879 & 4007 \\
 374 & 2^1 11^1 17^1 & \text{Y} & \text{N} & -16 & 0 & 1.0000000 & 0.486631 & 0.513369 & 233 & 2128 & -1895 & 4023 \\
 375 & 3^1 5^3 & \text{N} & \text{N} & 9 & 4 & 1.5555556 & 0.488000 & 0.512000 & 242 & 2137 & -1895 & 4032 \\
 376 & 2^3 47^1 & \text{N} & \text{N} & 9 & 4 & 1.5555556 & 0.489362 & 0.510638 & 251 & 2146 & -1895 & 4041 \\
 377 & 13^1 29^1 & \text{Y} & \text{N} & 5 & 0 & 1.0000000 & 0.490716 & 0.509284 & 256 & 2151 & -1895 & 4046 \\
 378 & 2^1 3^3 7^1 & \text{N} & \text{N} & -48 & 32 & 1.3333333 & 0.489418 & 0.510582 & 208 & 2151 & -1943 & 4094 \\
 379 & 379^1 & \text{Y} & \text{Y} & -2 & 0 & 1.0000000 & 0.488127 & 0.511873 & 206 & 2151 & -1945 & 4096 \\
 380 & 2^2 5^1 19^1 & \text{N} & \text{N} & 30 & 14 & 1.1666667 & 0.489474 & 0.510526 & 236 & 2181 & -1945 & 4126 \\
 381 & 3^1 127^1 & \text{Y} & \text{N} & 5 & 0 & 1.0000000 & 0.490814 & 0.509186 & 241 & 2186 & -1945 & 4131 \\
 382 & 2^1 191^1 & \text{Y} & \text{N} & 5 & 0 & 1.0000000 & 0.492147 & 0.507853 & 246 & 2191 & -1945 & 4136 \\
 383 & 383^1 & \text{Y} & \text{Y} & -2 & 0 & 1.0000000 & 0.490862 & 0.509138 & 244 & 2191 & -1947 & 4138 \\
 384 & 2^7 3^1 & \text{N} & \text{N} & 17 & 12 & 2.5882353 & 0.492188 & 0.507812 & 261 & 2208 & -1947 & 4155 \\
 385 & 5^1 7^1 11^1 & \text{Y} & \text{N} & -16 & 0 & 1.0000000 & 0.490909 & 0.509091 & 245 & 2208 & -1963 & 4171 \\
 386 & 2^1 193^1 & \text{Y} & \text{N} & 5 & 0 & 1.0000000 & 0.492228 & 0.507772 & 250 & 2213 & -1963 & 4176 \\
 387 & 3^2 43^1 & \text{N} & \text{N} & -7 & 2 & 1.2857143 & 0.490956 & 0.509044 & 243 & 2213 & -1970 & 4183 \\
 388 & 2^2 97^1 & \text{N} & \text{N} & -7 & 2 & 1.2857143 & 0.489691 & 0.510309 & 236 & 2213 & -1977 & 4190 \\
 389 & 389^1 & \text{Y} & \text{Y} & -2 & 0 & 1.0000000 & 0.488432 & 0.511568 & 234 & 2213 & -1979 & 4192 \\
 390 & 2^1 3^1 5^1 13^1 & \text{Y} & \text{N} & 65 & 0 & 1.0000000 & 0.489744 & 0.510256 & 299 & 2278 & -1979 & 4257 \\
 391 & 17^1 23^1 & \text{Y} & \text{N} & 5 & 0 & 1.0000000 & 0.491049 & 0.508951 & 304 & 2283 & -1979 & 4262 \\
 392 & 2^3 7^2 & \text{N} & \text{N} & -23 & 18 & 1.4782609 & 0.489796 & 0.510204 & 281 & 2283 & -2002 & 4285 \\
 393 & 3^1 131^1 & \text{Y} & \text{N} & 5 & 0 & 1.0000000 & 0.491094 & 0.508906 & 286 & 2288 & -2002 & 4290 \\
 394 & 2^1 197^1 & \text{Y} & \text{N} & 5 & 0 & 1.0000000 & 0.492386 & 0.507614 & 291 & 2293 & -2002 & 4295 \\
 395 & 5^1 79^1 & \text{Y} & \text{N} & 5 & 0 & 1.0000000 & 0.493671 & 0.506329 & 296 & 2298 & -2002 & 4300 \\
 396 & 2^2 3^2 11^1 & \text{N} & \text{N} & -74 & 58 & 1.2162162 & 0.492424 & 0.507576 & 222 & 2298 & -2076 & 4374 \\
 397 & 397^1 & \text{Y} & \text{Y} & -2 & 0 & 1.0000000 & 0.491184 & 0.508816 & 220 & 2298 & -2078 & 4376 \\
 398 & 2^1 199^1 & \text{Y} & \text{N} & 5 & 0 & 1.0000000 & 0.492462 & 0.507538 & 225 & 2303 & -2078 & 4381 \\
 399 & 3^1 7^1 19^1 & \text{Y} & \text{N} & -16 & 0 & 1.0000000 & 0.491228 & 0.508772 & 209 & 2303 & -2094 & 4397 \\
 400 & 2^4 5^2 & \text{N} & \text{N} & 34 & 29 & 1.6176471 & 0.492500 & 0.507500 & 243 & 2337 & -2094 & 4431 \\
 401 & 401^1 & \text{Y} & \text{Y} & -2 & 0 & 1.0000000 & 0.491272 & 0.508728 & 241 & 2337 & -2096 & 4433 \\
 402 & 2^1 3^1 67^1 & \text{Y} & \text{N} & -16 & 0 & 1.0000000 & 0.490050 & 0.509950 & 225 & 2337 & -2112 & 4449 \\
 403 & 13^1 31^1 & \text{Y} & \text{N} & 5 & 0 & 1.0000000 & 0.491315 & 0.508685 & 230 & 2342 & -2112 & 4454 \\
 404 & 2^2 101^1 & \text{N} & \text{N} & -7 & 2 & 1.2857143 & 0.490099 & 0.509901 & 223 & 2342 & -2119 & 4461 \\
 405 & 3^4 5^1 & \text{N} & \text{N} & -11 & 6 & 1.8181818 & 0.488889 & 0.511111 & 212 & 2342 & -2130 & 4472 \\
 406 & 2^1 7^1 29^1 & \text{Y} & \text{N} & -16 & 0 & 1.0000000 & 0.487685 & 0.512315 & 196 & 2342 & -2146 & 4488 \\
 407 & 11^1 37^1 & \text{Y} & \text{N} & 5 & 0 & 1.0000000 & 0.488943 & 0.511057 & 201 & 2347 & -2146 & 4493 \\
 408 & 2^3 3^1 17^1 & \text{N} & \text{N} & -48 & 32 & 1.3333333 & 0.487745 & 0.512255 & 153 & 2347 & -2194 & 4541 \\
 409 & 409^1 & \text{Y} & \text{Y} & -2 & 0 & 1.0000000 & 0.486553 & 0.513447 & 151 & 2347 & -2196 & 4543 \\
 410 & 2^1 5^1 41^1 & \text{Y} & \text{N} & -16 & 0 & 1.0000000 & 0.485366 & 0.514634 & 135 & 2347 & -2212 & 4559 \\
 411 & 3^1 137^1 & \text{Y} & \text{N} & 5 & 0 & 1.0000000 & 0.486618 & 0.513382 & 140 & 2352 & -2212 & 4564 \\
 412 & 2^2 103^1 & \text{N} & \text{N} & -7 & 2 & 1.2857143 & 0.485437 & 0.514563 & 133 & 2352 & -2219 & 4571 \\
 413 & 7^1 59^1 & \text{Y} & \text{N} & 5 & 0 & 1.0000000 & 0.486683 & 0.513317 & 138 & 2357 & -2219 & 4576 \\
 414 & 2^1 3^2 23^1 & \text{N} & \text{N} & 30 & 14 & 1.1666667 & 0.487923 & 0.512077 & 168 & 2387 & -2219 & 4606 \\
 415 & 5^1 83^1 & \text{Y} & \text{N} & 5 & 0 & 1.0000000 & 0.489157 & 0.510843 & 173 & 2392 & -2219 & 4611 \\
 416 & 2^5 13^1 & \text{N} & \text{N} & 13 & 8 & 2.0769231 & 0.490385 & 0.509615 & 186 & 2405 & -2219 & 4624 \\
 417 & 3^1 139^1 & \text{Y} & \text{N} & 5 & 0 & 1.0000000 & 0.491607 & 0.508393 & 191 & 2410 & -2219 & 4629 \\
 418 & 2^1 11^1 19^1 & \text{Y} & \text{N} & -16 & 0 & 1.0000000 & 0.490431 & 0.509569 & 175 & 2410 & -2235 & 4645 \\
 419 & 419^1 & \text{Y} & \text{Y} & -2 & 0 & 1.0000000 & 0.489260 & 0.510740 & 173 & 2410 & -2237 & 4647 \\
 420 & 2^2 3^1 5^1 7^1 & \text{N} & \text{N} & -155 & 90 & 1.1032258 & 0.488095 & 0.511905 & 18 & 2410 & -2392 & 4802 \\
 421 & 421^1 & \text{Y} & \text{Y} & -2 & 0 & 1.0000000 & 0.486936 & 0.513064 & 16 & 2410 & -2394 & 4804 \\
 422 & 2^1 211^1 & \text{Y} & \text{N} & 5 & 0 & 1.0000000 & 0.488152 & 0.511848 & 21 & 2415 & -2394 & 4809 \\
 423 & 3^2 47^1 & \text{N} & \text{N} & -7 & 2 & 1.2857143 & 0.486998 & 0.513002 & 14 & 2415 & -2401 & 4816 \\
 424 & 2^3 53^1 & \text{N} & \text{N} & 9 & 4 & 1.5555556 & 0.488208 & 0.511792 & 23 & 2424 & -2401 & 4825 \\
 425 & 5^2 17^1 & \text{N} & \text{N} & -7 & 2 & 1.2857143 & 0.487059 & 0.512941 & 16 & 2424 & -2408 & 4832 \\
\end{array}
}
\end{equation*}
\clearpage 

\end{table} 

\newpage

\begin{table}[ht]
\label{table_conjecture_Mertens_ginvSeq_approx_values_LastPage} 

\centering
\tiny
\begin{equation*}
\boxed{
\begin{array}{cc|cc|ccc|cc|cccc}
 n & \mathbf{Primes} & \mathbf{Sqfree} & \mathbf{PPower} & g^{-1}(n) & 
 \lambda(n) g^{-1}(n) - \widehat{f}_1(n) & 
 \frac{\sum_{d|n} C_{\Omega}(d)}{|g^{-1}(n)|} & 
 \mathcal{L}_{+}(n) & \mathcal{L}_{-}(n) & 
 G^{-1}(n) & G^{-1}_{+}(n) & G^{-1}_{-}(n) & |G^{-1}|(n) \\ \hline 
 426 & 2^1 3^1 71^1 & \text{Y} & \text{N} & -16 & 0 & 1.0000000 & 0.485915 & 0.514085 & 0 & 2424 & -2424 & 4848 \\
 427 & 7^1 61^1 & \text{Y} & \text{N} & 5 & 0 & 1.0000000 & 0.487119 & 0.512881 & 5 & 2429 & -2424 & 4853 \\
 428 & 2^2 107^1 & \text{N} & \text{N} & -7 & 2 & 1.2857143 & 0.485981 & 0.514019 & -2 & 2429 & -2431 & 4860 \\
 429 & 3^1 11^1 13^1 & \text{Y} & \text{N} & -16 & 0 & 1.0000000 & 0.484848 & 0.515152 & -18 & 2429 & -2447 & 4876 \\
 430 & 2^1 5^1 43^1 & \text{Y} & \text{N} & -16 & 0 & 1.0000000 & 0.483721 & 0.516279 & -34 & 2429 & -2463 & 4892 \\
 431 & 431^1 & \text{Y} & \text{Y} & -2 & 0 & 1.0000000 & 0.482599 & 0.517401 & -36 & 2429 & -2465 & 4894 \\
 432 & 2^4 3^3 & \text{N} & \text{N} & -80 & 75 & 1.5625000 & 0.481481 & 0.518519 & -116 & 2429 & -2545 & 4974 \\
 433 & 433^1 & \text{Y} & \text{Y} & -2 & 0 & 1.0000000 & 0.480370 & 0.519630 & -118 & 2429 & -2547 & 4976 \\
 434 & 2^1 7^1 31^1 & \text{Y} & \text{N} & -16 & 0 & 1.0000000 & 0.479263 & 0.520737 & -134 & 2429 & -2563 & 4992 \\
 435 & 3^1 5^1 29^1 & \text{Y} & \text{N} & -16 & 0 & 1.0000000 & 0.478161 & 0.521839 & -150 & 2429 & -2579 & 5008 \\
 436 & 2^2 109^1 & \text{N} & \text{N} & -7 & 2 & 1.2857143 & 0.477064 & 0.522936 & -157 & 2429 & -2586 & 5015 \\
 437 & 19^1 23^1 & \text{Y} & \text{N} & 5 & 0 & 1.0000000 & 0.478261 & 0.521739 & -152 & 2434 & -2586 & 5020 \\
 438 & 2^1 3^1 73^1 & \text{Y} & \text{N} & -16 & 0 & 1.0000000 & 0.477169 & 0.522831 & -168 & 2434 & -2602 & 5036 \\
 439 & 439^1 & \text{Y} & \text{Y} & -2 & 0 & 1.0000000 & 0.476082 & 0.523918 & -170 & 2434 & -2604 & 5038 \\
 440 & 2^3 5^1 11^1 & \text{N} & \text{N} & -48 & 32 & 1.3333333 & 0.475000 & 0.525000 & -218 & 2434 & -2652 & 5086 \\
 441 & 3^2 7^2 & \text{N} & \text{N} & 14 & 9 & 1.3571429 & 0.476190 & 0.523810 & -204 & 2448 & -2652 & 5100 \\
 442 & 2^1 13^1 17^1 & \text{Y} & \text{N} & -16 & 0 & 1.0000000 & 0.475113 & 0.524887 & -220 & 2448 & -2668 & 5116 \\
 443 & 443^1 & \text{Y} & \text{Y} & -2 & 0 & 1.0000000 & 0.474041 & 0.525959 & -222 & 2448 & -2670 & 5118 \\
 444 & 2^2 3^1 37^1 & \text{N} & \text{N} & 30 & 14 & 1.1666667 & 0.475225 & 0.524775 & -192 & 2478 & -2670 & 5148 \\
 445 & 5^1 89^1 & \text{Y} & \text{N} & 5 & 0 & 1.0000000 & 0.476404 & 0.523596 & -187 & 2483 & -2670 & 5153 \\
 446 & 2^1 223^1 & \text{Y} & \text{N} & 5 & 0 & 1.0000000 & 0.477578 & 0.522422 & -182 & 2488 & -2670 & 5158 \\
 447 & 3^1 149^1 & \text{Y} & \text{N} & 5 & 0 & 1.0000000 & 0.478747 & 0.521253 & -177 & 2493 & -2670 & 5163 \\
 448 & 2^6 7^1 & \text{N} & \text{N} & -15 & 10 & 2.3333333 & 0.477679 & 0.522321 & -192 & 2493 & -2685 & 5178 \\
 449 & 449^1 & \text{Y} & \text{Y} & -2 & 0 & 1.0000000 & 0.476615 & 0.523385 & -194 & 2493 & -2687 & 5180 \\
 450 & 2^1 3^2 5^2 & \text{N} & \text{N} & -74 & 58 & 1.2162162 & 0.475556 & 0.524444 & -268 & 2493 & -2761 & 5254 \\
 451 & 11^1 41^1 & \text{Y} & \text{N} & 5 & 0 & 1.0000000 & 0.476718 & 0.523282 & -263 & 2498 & -2761 & 5259 \\
 452 & 2^2 113^1 & \text{N} & \text{N} & -7 & 2 & 1.2857143 & 0.475664 & 0.524336 & -270 & 2498 & -2768 & 5266 \\
 453 & 3^1 151^1 & \text{Y} & \text{N} & 5 & 0 & 1.0000000 & 0.476821 & 0.523179 & -265 & 2503 & -2768 & 5271 \\
 454 & 2^1 227^1 & \text{Y} & \text{N} & 5 & 0 & 1.0000000 & 0.477974 & 0.522026 & -260 & 2508 & -2768 & 5276 \\
 455 & 5^1 7^1 13^1 & \text{Y} & \text{N} & -16 & 0 & 1.0000000 & 0.476923 & 0.523077 & -276 & 2508 & -2784 & 5292 \\
 456 & 2^3 3^1 19^1 & \text{N} & \text{N} & -48 & 32 & 1.3333333 & 0.475877 & 0.524123 & -324 & 2508 & -2832 & 5340 \\
 457 & 457^1 & \text{Y} & \text{Y} & -2 & 0 & 1.0000000 & 0.474836 & 0.525164 & -326 & 2508 & -2834 & 5342 \\
 458 & 2^1 229^1 & \text{Y} & \text{N} & 5 & 0 & 1.0000000 & 0.475983 & 0.524017 & -321 & 2513 & -2834 & 5347 \\
 459 & 3^3 17^1 & \text{N} & \text{N} & 9 & 4 & 1.5555556 & 0.477124 & 0.522876 & -312 & 2522 & -2834 & 5356 \\
 460 & 2^2 5^1 23^1 & \text{N} & \text{N} & 30 & 14 & 1.1666667 & 0.478261 & 0.521739 & -282 & 2552 & -2834 & 5386 \\
 461 & 461^1 & \text{Y} & \text{Y} & -2 & 0 & 1.0000000 & 0.477223 & 0.522777 & -284 & 2552 & -2836 & 5388 \\
 462 & 2^1 3^1 7^1 11^1 & \text{Y} & \text{N} & 65 & 0 & 1.0000000 & 0.478355 & 0.521645 & -219 & 2617 & -2836 & 5453 \\
 463 & 463^1 & \text{Y} & \text{Y} & -2 & 0 & 1.0000000 & 0.477322 & 0.522678 & -221 & 2617 & -2838 & 5455 \\
 464 & 2^4 29^1 & \text{N} & \text{N} & -11 & 6 & 1.8181818 & 0.476293 & 0.523707 & -232 & 2617 & -2849 & 5466 \\
 465 & 3^1 5^1 31^1 & \text{Y} & \text{N} & -16 & 0 & 1.0000000 & 0.475269 & 0.524731 & -248 & 2617 & -2865 & 5482 \\
 466 & 2^1 233^1 & \text{Y} & \text{N} & 5 & 0 & 1.0000000 & 0.476395 & 0.523605 & -243 & 2622 & -2865 & 5487 \\
 467 & 467^1 & \text{Y} & \text{Y} & -2 & 0 & 1.0000000 & 0.475375 & 0.524625 & -245 & 2622 & -2867 & 5489 \\
 468 & 2^2 3^2 13^1 & \text{N} & \text{N} & -74 & 58 & 1.2162162 & 0.474359 & 0.525641 & -319 & 2622 & -2941 & 5563 \\
 469 & 7^1 67^1 & \text{Y} & \text{N} & 5 & 0 & 1.0000000 & 0.475480 & 0.524520 & -314 & 2627 & -2941 & 5568 \\
 470 & 2^1 5^1 47^1 & \text{Y} & \text{N} & -16 & 0 & 1.0000000 & 0.474468 & 0.525532 & -330 & 2627 & -2957 & 5584 \\
 471 & 3^1 157^1 & \text{Y} & \text{N} & 5 & 0 & 1.0000000 & 0.475584 & 0.524416 & -325 & 2632 & -2957 & 5589 \\
 472 & 2^3 59^1 & \text{N} & \text{N} & 9 & 4 & 1.5555556 & 0.476695 & 0.523305 & -316 & 2641 & -2957 & 5598 \\
 473 & 11^1 43^1 & \text{Y} & \text{N} & 5 & 0 & 1.0000000 & 0.477801 & 0.522199 & -311 & 2646 & -2957 & 5603 \\
 474 & 2^1 3^1 79^1 & \text{Y} & \text{N} & -16 & 0 & 1.0000000 & 0.476793 & 0.523207 & -327 & 2646 & -2973 & 5619 \\
 475 & 5^2 19^1 & \text{N} & \text{N} & -7 & 2 & 1.2857143 & 0.475789 & 0.524211 & -334 & 2646 & -2980 & 5626 \\
 476 & 2^2 7^1 17^1 & \text{N} & \text{N} & 30 & 14 & 1.1666667 & 0.476891 & 0.523109 & -304 & 2676 & -2980 & 5656 \\
 477 & 3^2 53^1 & \text{N} & \text{N} & -7 & 2 & 1.2857143 & 0.475891 & 0.524109 & -311 & 2676 & -2987 & 5663 \\
 478 & 2^1 239^1 & \text{Y} & \text{N} & 5 & 0 & 1.0000000 & 0.476987 & 0.523013 & -306 & 2681 & -2987 & 5668 \\
 479 & 479^1 & \text{Y} & \text{Y} & -2 & 0 & 1.0000000 & 0.475992 & 0.524008 & -308 & 2681 & -2989 & 5670 \\
 480 & 2^5 3^1 5^1 & \text{N} & \text{N} & -96 & 80 & 1.6666667 & 0.475000 & 0.525000 & -404 & 2681 & -3085 & 5766 \\
 481 & 13^1 37^1 & \text{Y} & \text{N} & 5 & 0 & 1.0000000 & 0.476091 & 0.523909 & -399 & 2686 & -3085 & 5771 \\
 482 & 2^1 241^1 & \text{Y} & \text{N} & 5 & 0 & 1.0000000 & 0.477178 & 0.522822 & -394 & 2691 & -3085 & 5776 \\
 483 & 3^1 7^1 23^1 & \text{Y} & \text{N} & -16 & 0 & 1.0000000 & 0.476190 & 0.523810 & -410 & 2691 & -3101 & 5792 \\
 484 & 2^2 11^2 & \text{N} & \text{N} & 14 & 9 & 1.3571429 & 0.477273 & 0.522727 & -396 & 2705 & -3101 & 5806 \\
 485 & 5^1 97^1 & \text{Y} & \text{N} & 5 & 0 & 1.0000000 & 0.478351 & 0.521649 & -391 & 2710 & -3101 & 5811 \\
 486 & 2^1 3^5 & \text{N} & \text{N} & 13 & 8 & 2.0769231 & 0.479424 & 0.520576 & -378 & 2723 & -3101 & 5824 \\
 487 & 487^1 & \text{Y} & \text{Y} & -2 & 0 & 1.0000000 & 0.478439 & 0.521561 & -380 & 2723 & -3103 & 5826 \\
 488 & 2^3 61^1 & \text{N} & \text{N} & 9 & 4 & 1.5555556 & 0.479508 & 0.520492 & -371 & 2732 & -3103 & 5835 \\
 489 & 3^1 163^1 & \text{Y} & \text{N} & 5 & 0 & 1.0000000 & 0.480573 & 0.519427 & -366 & 2737 & -3103 & 5840 \\
 490 & 2^1 5^1 7^2 & \text{N} & \text{N} & 30 & 14 & 1.1666667 & 0.481633 & 0.518367 & -336 & 2767 & -3103 & 5870 \\
 491 & 491^1 & \text{Y} & \text{Y} & -2 & 0 & 1.0000000 & 0.480652 & 0.519348 & -338 & 2767 & -3105 & 5872 \\
 492 & 2^2 3^1 41^1 & \text{N} & \text{N} & 30 & 14 & 1.1666667 & 0.481707 & 0.518293 & -308 & 2797 & -3105 & 5902 \\
 493 & 17^1 29^1 & \text{Y} & \text{N} & 5 & 0 & 1.0000000 & 0.482759 & 0.517241 & -303 & 2802 & -3105 & 5907 \\
 494 & 2^1 13^1 19^1 & \text{Y} & \text{N} & -16 & 0 & 1.0000000 & 0.481781 & 0.518219 & -319 & 2802 & -3121 & 5923 \\
 495 & 3^2 5^1 11^1 & \text{N} & \text{N} & 30 & 14 & 1.1666667 & 0.482828 & 0.517172 & -289 & 2832 & -3121 & 5953 \\
 496 & 2^4 31^1 & \text{N} & \text{N} & -11 & 6 & 1.8181818 & 0.481855 & 0.518145 & -300 & 2832 & -3132 & 5964 \\
 497 & 7^1 71^1 & \text{Y} & \text{N} & 5 & 0 & 1.0000000 & 0.482897 & 0.517103 & -295 & 2837 & -3132 & 5969 \\
 498 & 2^1 3^1 83^1 & \text{Y} & \text{N} & -16 & 0 & 1.0000000 & 0.481928 & 0.518072 & -311 & 2837 & -3148 & 5985 \\
 499 & 499^1 & \text{Y} & \text{Y} & -2 & 0 & 1.0000000 & 0.480962 & 0.519038 & -313 & 2837 & -3150 & 5987 \\
 500 & 2^2 5^3 & \text{N} & \text{N} & -23 & 18 & 1.4782609 & 0.480000 & 0.520000 & -336 & 2837 & -3173 & 6010 \\
\end{array}
}
\end{equation*}

\end{table} 

\clearpage 

\end{document}
