\documentclass[11pt,reqno,a4letter]{article} 

\usepackage{amsmath,amssymb,amsfonts,amscd}
\usepackage[hidelinks]{hyperref} 
\usepackage{url}
\usepackage[usenames,dvipsnames]{xcolor}
\hypersetup{
    colorlinks,
    linkcolor={green!63!darkgray},
    citecolor={blue!70!white},
    urlcolor={blue!80!white}
}

\usepackage[normalem]{ulem}
\usepackage{graphicx} 
\usepackage{datetime} 
\usepackage{cancel}
\usepackage{subcaption}
\captionsetup{format=hang,labelfont={bf},textfont={small,it}} 
\numberwithin{figure}{section}
\numberwithin{table}{section}

%\usepackage{stmaryrd,tikzsymbols,mathabx,wasysym} 
\usepackage{framed} 
\usepackage{ulem}
\usepackage[T1]{fontenc}
\usepackage{pbsi}


\usepackage{enumitem}
\setlist[itemize]{leftmargin=0.65in}

\usepackage{rotating,adjustbox}

\usepackage{diagbox}
\newcommand{\trianglenk}[2]{$\diagbox{#1}{#2}$}
\newcommand{\trianglenkII}[2]{\diagbox{#1}{#2}}

\let\citep\cite

\newcommand{\undersetbrace}[2]{\underset{\displaystyle{#1}}{\underbrace{#2}}}

\newcommand{\gkpSI}[2]{\ensuremath{\genfrac{\lbrack}{\rbrack}{0pt}{}{#1}{#2}}} 
\newcommand{\gkpSII}[2]{\ensuremath{\genfrac{\lbrace}{\rbrace}{0pt}{}{#1}{#2}}}
\newcommand{\cf}{\textit{cf.\ }} 
\newcommand{\Iverson}[1]{\ensuremath{\left[#1\right]_{\delta}}} 
\newcommand{\floor}[1]{\left\lfloor #1 \right\rfloor} 
\newcommand{\ceiling}[1]{\left\lceil #1 \right\rceil} 
\newcommand{\e}[1]{e\left(#1\right)} 
\newcommand{\seqnum}[1]{\href{http://oeis.org/#1}{\color{ProcessBlue}{\underline{#1}}}}

\usepackage{upgreek,dsfont,amssymb}
\renewcommand{\chi}{\upchi}
\newcommand{\ChiFunc}[1]{\ensuremath{\chi_{\{#1\}}}}
\newcommand{\OneFunc}[1]{\ensuremath{\mathds{1}_{#1}}}

\usepackage{ifthen}
\newcommand{\Hn}[2]{
     \ifthenelse{\equal{#2}{1}}{H_{#1}}{H_{#1}^{\left(#2\right)}}
}

\newcommand{\Floor}[2]{\ensuremath{\left\lfloor \frac{#1}{#2} \right\rfloor}}
\newcommand{\Ceiling}[2]{\ensuremath{\left\lceil \frac{#1}{#2} \right\rceil}}

\DeclareMathOperator{\DGF}{DGF} 
\DeclareMathOperator{\ds}{ds} 
\DeclareMathOperator{\Id}{Id}
\DeclareMathOperator{\fg}{fg}
\DeclareMathOperator{\Div}{div}
\DeclareMathOperator{\rpp}{rpp}
\DeclareMathOperator{\logll}{\ell\ell}

\title{
       \LARGE{
       Lower bounds on the summatory function of the M\"obius function along infinite subsequences 
       } 
}
\author{{\Large Maxie Dion Schmidt} \\ 
        %{\normalsize \href{mailto:maxieds@gmail.com}{maxieds@gmail.com}} \\[0.1cm] 
        {\normalsize Georgia Institute of Technology} \\[0.025cm] 
        {\normalsize School of Mathematics} 
} 

\date{\small\underline{Last Revised:} \today \ @\ \hhmmsstime{} \ -- \ Compiled with \LaTeX2e} 

%\usepackage[amsmath,thmmarks,framed]{ntheorem}
%\usepackage{framed} 
%\definecolor{thmborder}{rgb}{0.419601,0.4941,0.55294} 
%\renewcommand*\FrameCommand{{\color{thmborder}\vrule width 5pt \hspace{10pt}}}

\usepackage{amsthm} 

\theoremstyle{plain} 
\newtheorem{theorem}{Theorem}
\newtheorem{conjecture}[theorem]{Conjecture}
\newtheorem{claim}[theorem]{Claim}
\newtheorem{prop}[theorem]{Proposition}
\newtheorem{lemma}[theorem]{Lemma}
\newtheorem{cor}[theorem]{Corollary}
\numberwithin{theorem}{section}

\theoremstyle{definition} 
\newtheorem{example}[theorem]{Example}
\newtheorem{remark}[theorem]{Remark}
\newtheorem{definition}[theorem]{Definition}
\newtheorem{notation}[theorem]{Notation}
\newtheorem{question}[theorem]{Question}
\newtheorem{discussion}[theorem]{Discussion}
\newtheorem{facts}[theorem]{Facts}
\newtheorem{summary}[theorem]{Summary}
\newtheorem{heuristic}[theorem]{Heuristic}

\renewcommand{\arraystretch}{1.25} 

\setlength{\textheight}{9in}
\setlength{\topmargin}{-.18in}
\setlength{\textwidth}{7.65in} 
\setlength{\evensidemargin}{-0.25in} 
\setlength{\oddsidemargin}{-0.25in} 
\setlength{\headsep}{8pt} 
%\setlength{\footskip}{10pt} 

\usepackage{geometry}
%\newgeometry{top=0.65in, bottom=18mm, left=15mm, right=15mm, outer=2in, heightrounded, marginparwidth=1.5in, marginparsep=0.15in}
\newgeometry{top=0.65in, bottom=16mm, left=15mm, right=15mm, heightrounded, marginparwidth=0in, marginparsep=0.15in}

\usepackage{fancyhdr}
\pagestyle{empty}
\pagestyle{fancy}
\fancyhead[RO,RE]{Maxie Dion Schmidt -- \today} 
\fancyhead[LO,LE]{}
\fancyheadoffset{0.005\textwidth} 

\setlength{\parindent}{0in}
\setlength{\parskip}{2cm} 

\renewcommand{\thefootnote}{\textbf{\ [--\Alph{footnote}}--]}
\makeatletter
\@addtoreset{footnote}{section}
\makeatother

%\usepackage{marginnote,todonotes}
%\colorlet{NBRefColor}{RoyalBlue!73} 
%\newcommand{\NBRef}[1]{
%     \todo[linecolor=green!85!white,backgroundcolor=orange!50!white,bordercolor=blue!30!black,textcolor=cyan!15!black,shadow,size=\small,fancyline]{
%     \color{NBRefColor}{\textbf{#1}
%     }
%     }
%}
\newcommand{\NBRef}[1]{}  

\newcommand{\SuccSim}[0]{\overset{_{\scriptsize{\blacktriangle}}}{\succsim}} 
\newcommand{\PrecSim}[0]{\overset{_{\scriptsize{\blacktriangle}}}{\precsim}} 
\renewcommand{\SuccSim}[0]{\ensuremath{\gg}} 
\renewcommand{\PrecSim}[0]{\ensuremath{\ll}} 

\renewcommand{\Re}{\operatorname{Re}}
\renewcommand{\Im}{\operatorname{Im}}

\input{glossaries-bibtex/PreambleGlossaries-Mertens}

\usepackage{tikz}
\usetikzlibrary{shapes,arrows}

\usepackage{enumitem} 

\allowdisplaybreaks 

\begin{document} 

\maketitle

\begin{abstract} 
The Mertens function, $M(x) = \sum_{n \leq x} \mu(n)$, is classically 
defined as the summatory function of the M\"obius function $\mu(n)$. 
The Mertens conjecture states that $|M(x)| < C \cdot \sqrt{x}$ with come absolute $C > 0$ for all 
$x \geq 1$. 
This classical conjecture has a well-known disproof due to 
Odlyzko and t\'{e} Riele by computation of 
non-trivial zeta function zeros in conjunction with integral formulas expressing $M(x)$. 
We prove the unboundedness of $|M(x)| / \sqrt{x}$ using new methods by showing that 
$$\limsup_{x \rightarrow \infty} \frac{|M(x)| \sqrt{\log\log x} \cdot (\log\log\log x)^{2}}{ 
  \sqrt{x} \cdot (\log x)^{\frac{1}{4}}} \geq 0.106408.$$ 
There is a distinct stylistic 
flavor and new element of combinatorial analysis to our proof 
combined with the standard methods from analytic, additive and elementary number theory. 
This stylistic tendency distinguishes 
our methods from other proofs of established upper, rather than lower, bounds on $M(x)$. 

\bigskip 
\noindent
\textbf{Keywords and Phrases:} {\it M\"obius function; Mertens function; summatory function; 
                                    Dirichlet inverse; Liouville lambda function; prime omega function; 
                                    prime counting functions; Dirichlet generating function; 
                                    asymptotic lower bounds; Mertens conjecture. } \\ 
% 11-XX			Number theory
%    11A25  	Arithmetic functions; related numbers; inversion formulas
%    11Y70  	Values of arithmetic functions; tables
%    11-04  	Software, source code, etc. for problems pertaining to number theory
% 11Nxx		Multiplicative number theory
%    11N05  	Distribution of primes
%    11N37  	Asymptotic results on arithmetic functions
%    11N56  	Rate of growth of arithmetic functions
%    11N60  	Distribution functions associated with additive and positive multiplicative functions
%    11N64  	Other results on the distribution of values or the characterization of arithmetic functions
\textbf{Math Subject Classifications (MSC 2010):} {\it 11N37; 11A25; 11N60; and 11N64. } 
\end{abstract} 

%\bigskip\hrule\bigskip

\newpage
%\section{Reference on abbreviations, special notation and other conventions} 
\label{Appendix_Glossary_NotationConvs}
     \vskip 0in
     \printglossary[type={symbols},
                    title={Glossary of special notation and conventions},
                    style={glossstyleSymbol},
                    nogroupskip=true]


%\newpage
%\setcounter{tocdepth}{2}
%\renewcommand{\contentsname}{Listing of major sections and topics} 
%\tableofcontents 

\newpage
\section{Introduction} 
\label{subSection_MertensMxClassical_Intro} 

\subsection{Definitions} 

We define the \emph{M\"obius function} to be the signed indicator function 
of the squarefree integers in the form of \cite[\seqnum{A008683}]{OEIS} 
\[
\mu(n) = \begin{cases} 
     1, & \text{if $n = 1$; } \\ 
     (-1)^{\omega(n)}, & \text{if $\omega(n) = \Omega(n)$ and $n \geq 2$; } \\ 
     0, & \text{otherwise.} 
     \end{cases} 
\]
There are many variants and special properties of the M\"obius function 
and its generalizations in other non-classical constructs \cite[\cf \S 2]{HANDBOOKNT-2004}. 
One crucial role of the classical $\mu(n)$ is that the function forms an inversion relation 
for arithmetic functions convolved with one through \emph{M\"obius inversion}: 
\[
g(n) = (f \ast 1)(n) \iff f(n) = (g \ast \mu)(n), \forall n \geq 1. 
\]
The \emph{Mertens function}, or summatory function of $\mu(n)$, is defined on the 
positive integers as 
\begin{align*} 
M(x) & = \sum_{n \leq x} \mu(n), x \geq 1. 
\end{align*} 
The sequence of slow growing oscillatory values of this 
summatory function begins as follows \cite[\seqnum{A002321}]{OEIS}: 
\[
\{M(x)\}_{x \geq 1} = \{1, 0, -1, -1, -2, -1, -2, -2, -2, -1, -2, -2, -3, -2, 
     -1, -1, -2, -2, -3, -3, -2, -1, -2, \ldots\}. 
\]
Clearly, a positive integer $n \geq 1$ is \emph{squarefree}, or contains no (prime power) divisors which are 
squares, if and only if $\mu^2(n) = 1$. 
A related summatory function which counts the 
number of \emph{squarefree} integers $n \leq x$ then satisfies 
\cite[\S 18.6]{HARDYWRIGHT} \cite[\seqnum{A013928}]{OEIS} 
\[ 
Q(x) = \sum_{n \leq x} \mu^2(n) \sim \frac{6x}{\pi^2} + O\left(\sqrt{x}\right). 
\] 
It is known that the asymptotic density of the positively versus negatively 
weighted sets of squarefree numbers characterized by the sign of the 
M\"obius function are in fact equal as $x \rightarrow \infty$: 
\[
\mu_{+}(x) = \frac{\#\{1 \leq n \leq x: \mu(n) = +1\}}{x} \overset{\mathbb{E}}{\sim} 
     \mu_{-}(x) = \frac{\#\{1 \leq n \leq x: \mu(n) = -1\}}{x} 
     \xrightarrow{x \rightarrow \infty} \frac{3}{\pi^2}. 
\]

\subsection{Properties} 

A conventional approach to evaluating the limiting asymptotic 
behavior of $M(x)$ for large $x \rightarrow \infty$ results by considering an 
inverse Mellin transformation over the reciprocal of the Riemann zeta function. 
In particular, since 
\[
\frac{1}{\zeta(s)} = \prod_{p} \left(1 - \frac{1}{p^s}\right) = 
     s \cdot \int_1^{\infty} \frac{M(x)}{x^{s+1}} dx, \Re(s) > 1, 
\]
we obtain that 
\[
M(x) = \lim_{T \rightarrow \infty}\ \frac{1}{2\pi\imath} \int_{T-\imath\infty}^{T+\imath\infty} 
     \frac{x^s}{s \cdot \zeta(s)} ds. 
\] 
The previous two representations lead us to the 
exact expression for $M(x)$ for any real $x > 0$ 
given by the next theorem due to Titchmarsh. 
\nocite{TITCHMARSH} 

\begin{theorem}[Analytic Formula for $M(x)$] 
\label{theorem_MxMellinTransformInvFormula} 
Assuming the Riemann Hypothesis (RH), there exists an infinite sequence 
$\{T_k\}_{k \geq 1}$ satisfying $k \leq T_k \leq k+1$ for each $k$ 
such that for any real $x > 0$ 
\[
M(x) = \lim_{k \rightarrow \infty} 
     \sum_{\substack{\rho: \zeta(\rho) = 0 \\ |\Im(\rho)| < T_k}} 
     \frac{x^{\rho}}{\rho \cdot \zeta^{\prime}(\rho)} - 2 + 
     \sum_{n \geq 1} \frac{(-1)^{n-1}}{n \cdot (2n)! \zeta(2n+1)} 
     \left(\frac{2\pi}{x}\right)^{2n} + 
     \frac{\mu(x)}{2} \Iverson{x \in \mathbb{Z}^{+}}. 
\] 
\end{theorem} 

A historical unconditional bound on the Mertens function due to Walfisz (circa 1963) 
states that there is an absolute constant $C > 0$ such that 
$$M(x) \ll x \cdot \exp\left(-C \cdot \log^{3/5}(x) 
  (\log\log x)^{-3/5}\right).$$ 
Under the assumption of the RH, Soundararajan more recently proved new updated estimates 
bounding $M(x)$ from above for large $x$ in the following forms \cite{SOUND-MERTENS-ANNALS}: 
\begin{align*} 
M(x) & \ll \sqrt{x} \cdot \exp\left(\log^{1/2}(x) (\log\log x)^{14}\right), \\ 
M(x) & = O\left(\sqrt{x} \cdot \exp\left( 
     \log^{1/2}(x) (\log\log x)^{5/2+\epsilon}\right)\right),\ 
     \forall \epsilon > 0. 
\end{align*} 

\subsection{Conjectures on boundedness and limiting behavior} 

The RH is equivalent to showing that 
$M(x) = O\left(x^{\frac{1}{2}+\varepsilon}\right)$ for any 
$0 < \varepsilon < \frac{1}{2}$. 
There is a rich history to the original statement of the \emph{Mertens conjecture} which 
asserts that 
\[ 
|M(x)| < C \cdot \sqrt{x},\ \text{ for some absolute constant $C > 0$. }
\] 
The conjecture was first verified by Mertens for $C = 1$ and all $x < 10000$. 
Since its beginnings in 1897, the Mertens conjecture has been disproven by computation 
of non-trivial simple zeta function zeros with comparitively small imaginary parts in a famous paper by 
Odlyzko and t\'{e} Riele from the early 1980's \cite{ODLYZ-TRIELE}. 
Since the truth of the conjecture would have implied the RH, more recent attempts 
at bounding $M(x)$ naturally consider determining the rates at which the function 
$M(x) / \sqrt{x}$ grows with or without bound along infinite 
subsequences, e.g., considering the asymptotics of the function in the limit supremum and 
limit infimum senses. 

One of the most famous still unanswered questions about the Mertens 
function concerns whether $|M(x)| / \sqrt{x}$ actually grows without bound on the 
natural numbers. A precise statement of this 
problem is to produce an unconditional proof of whether 
$\limsup_{x \rightarrow \infty} M(x) / \sqrt{x} = +\infty$ and 
$\liminf_{x \rightarrow \infty} M(x) / \sqrt{x} = -\infty$, or 
equivalently whether there are infinite subsequences of natural numbers 
$\{x_1, x_2, x_3, \ldots\}$ such that the magnitude of 
$M(x_i) x_i^{-1/2}$ grows without bound towards either $\pm \infty$ 
along the subsequence. 
We cite that prior to this point it is only known by computation 
that \cite[\cf \S 4.1]{PRIMEREC} 
\cite[\cf \seqnum{A051400}; \seqnum{A051401}]{OEIS} 
\[
\limsup_{x\rightarrow\infty} \frac{M(x)}{\sqrt{x}} > 1.060\ \qquad (\text{now } \geq 1.826054), 
\] 
and 
\[ 
\liminf_{x\rightarrow\infty} \frac{M(x)}{\sqrt{x}} < -1.009\ \qquad (\text{now } \leq -1.837625). 
\] 
Based on work by Odlyzyko and t\'{e} Riele, it seems probable that 
each of these limits should evaluate to $\pm \infty$, respectively 
\cite{ODLYZ-TRIELE,MREVISITED,ORDER-MERTENSFN,HURST-2017}. 
Extensive computational evidence has produced 
a conjecture due to Gonek that in fact the limiting behavior of 
$M(x)$ satisfies \cite{NG-MERTENS}
$$\limsup_{x \rightarrow \infty} \frac{|M(x)|}{\sqrt{x} \cdot (\log\log\log x)^{5/4}} = O(1).$$ 

\newpage
\section{An overview of the core logical steps and components to the proof} 

We offer an initial step-by-step summary overview of the core components 
to our proof outlined in the next points. 
As our proof methodology is new and relies on non-standard elements compared to more 
traditional methods of bounding $M(x)$, we hope that this sketch of the logical components 
to this argument makes the article easier to parse. 

\begin{itemize} 

\item[\textbf{(1)}] We prove a matrix inversion formula relating the summatory 
           functions of an arithmetic function $f$ and its Dirichlet inverse $f^{-1}$ (for $f(1) \neq 0$). 
           See Theorem \ref{theorem_SummatoryFuncsOfDirCvls} in 
           Section \ref{Section_PrelimProofs_Config}.  
\item[\textbf{(2)}] This crucial step provides us with an exact formula for $M(x)$ in terms of $\pi(x)$, the prime counting function, and the 
           Dirichlet inverse of the shifted additive function $g(n) := \omega(n) + 1$. This 
           formula is stated in \eqref{eqn_Mx_gInvnPixk_formula}. 
           The link relating our new formula in 
           \eqref{eqn_Mx_gInvnPixk_formula} to canonical additive functions and their 
           distributions lends a recent distinguishing element to the 
           success of the methods in our proof. 
\item[\textbf{(3)}] We tighten bounds from a more recent result proved in 
            \cite[\S 7]{MV} providing uniform asymptotic formulas for the  
           summatory functions, $\widehat{\pi}_k(x)$, large $x \gg e$ and 
           $1 \leq k \leq \log\log x$ 
           (see Theorem \ref{theorem_GFs_SymmFuncs_SumsOfRecipOfPowsOfPrimes}). 
           We use this result to bound sums of the form 
           $\sum_{n \leq x} \lambda(n) f(n)$ from below for particular positive arithmetic 
           functions $f$ when $x$ is large. 
\item[\textbf{(4)}] We then turn to estimating the limiting 
           asymptotics of the quasi-periodic function, $|g^{-1}(n)|$, by proving several formulas bounding its 
           average order as $x \rightarrow \infty$ in 
           Section \ref{Section_InvFunc_PreciseExpsAndAsymptotics}. 
           We eventually use these estimates to prove a substantially unique new lower bound formulas 
           for the summatory function 
           $G^{-1}(x) := \sum_{n \leq x} g^{-1}(n) = \sum_{n \leq x} \lambda(n) |g^{-1}(n)|$ 
           along certain asymptotically large 
           infinite subsequences (see Theorem \ref{theorem_gInv_GeneralAsymptoticsForms}). 
\item[\textbf{(5)}] When we return to step \textbf{(2)} 
           with our new lower bounds at hand, we have a new unconditional proof of the 
           unboundedness of $\frac{|M(x)|}{\sqrt{x}}$ 
           along a very large increasing infinite subsequence of positive natural numbers. 
           What we recover is a quick, and rigorous, proof of 
           Theorem \ref{cor_ThePipeDreamResult_v1} given in 
           Section \ref{subSection_TheCoreResultProof}. 
           
\end{itemize} 

\newpage 
\section{A concrete new approach for bounding $M(x)$ from below} 

\subsection{Summatory functions of Dirichlet convolutions of arithmetic functions} 

\begin{theorem}[Summatory functions of Dirichlet convolutions] 
\label{theorem_SummatoryFuncsOfDirCvls} 
Let $f,h: \mathbb{Z}^{+} \rightarrow \mathbb{C}$ be any arithmetic functions such that $f(1) \neq 0$. 
Suppose that $F(x) := \sum_{n \leq x} f(n)$ and $H(x) := \sum_{n \leq x} h(n)$ denote the summatory 
functions of $f$ and $h$, respectively, and that $F^{-1}(x)$ denotes the summatory function of the 
Dirichlet inverse of $f$. Then we have the following exact expressions for the 
summatory function of $f \ast h$ for all integers $x \geq 1$: 
\begin{align*} 
\pi_{f \ast h}(x) & := \sum_{n \leq x} \sum_{d|n} f(d) h(n/d) \\ 
     & \phantom{:}= \sum_{d \leq x} f(d) H\left(\Floor{x}{d}\right) \\ 
     & \phantom{:}= \sum_{k=1}^{x} H(k) \left[F\left(\Floor{x}{k}\right) - 
     F\left(\Floor{x}{k+1}\right)\right]. 
\end{align*} 
Moreover, for all $x \geq 1$ we have that 
\begin{align*} 
H(x) & = \sum_{j=1}^{x} \pi_{f \ast h}(j) \left[F^{-1}\left(\Floor{x}{j}\right) - 
     F^{-1}\left(\Floor{x}{j+1}\right)\right] \\ 
     & = \sum_{n=1}^{x} f^{-1}(n) \pi_{f \ast h}\left(\Floor{x}{n}\right). 
\end{align*} 
\end{theorem} 

\begin{cor}[Convolutions Arising From M\"obius Inversion] 
\label{cor_CvlGAstMu} 
Suppose that $g$ is an arithmetic function on the positive integers such that 
$g(1) \neq 0$. Define the summatory function of 
the convolution of $g$ with $\mu$ by $\widetilde{G}(x) := \sum_{n \leq x} (g \ast \mu)(n)$. 
Then the Mertens function equals 
\[
M(x) = \sum_{k=1}^{x} \left(\sum_{j=\floor{\frac{x}{k+1}}+1}^{\floor{\frac{x}{k}}} g^{-1}(j)\right) 
     \widetilde{G}(k), \forall x \geq 1. 
\]
\end{cor} 

\begin{cor}[A motivating special case] 
\label{cor_Mx_gInvnPixk_formula} 
We have exactly that for all $x \geq 1$ 
\begin{equation} 
\label{eqn_Mx_gInvnPixk_formula} 
M(x) = \sum_{k=1}^{x} (\omega+1)^{-1}(k) \left[\pi\left(\Floor{x}{k}\right) + 1\right]. 
\end{equation} 
\end{cor} 

\subsection{An exact expression for $M(x)$ in terms of strongly additive functions} 
\label{example_InvertingARecRelForMx_Intro}

We fix the notation for the Dirichlet invertible function $g(n) := \omega(n) + 1$ and define its 
inverse with respect to Dirichlet convolution by $g^{-1}(n) = (\omega+1)^{-1}(n)$. 
We can compute the
Dirichlet inverse of $g(n)$ exactly for the first few sequence values as 
(see Table \ref{table_conjecture_Mertens_ginvSeq_approx_values} starting on page 
\pageref{table_conjecture_Mertens_ginvSeq_approx_values} of the appendix section) 
\[
\{g^{-1}(n)\}_{n \geq 1} = \{1, -2, -2, 2, -2, 5, -2, -2, 2, 5, -2, -7, -2, 5, 5, 2, -2, -7, -2, 
     -7, 5, 5, -2, 9, \ldots \}. 
\] 
The sign of these positive terms is given by 
$\operatorname{sgn}(g^{-1}(n)) = \frac{g^{-1}(n)}{|g^{-1}(n)|} = \lambda(n)$ for all $n \geq 1$ 
(see Proposition \ref{prop_SignageDirInvsOfPosBddArithmeticFuncs_v1}). 

There does not appear to be an easy, nor subtle 
direct recursion between the distinct values of $g^{-1}(n)$, except through auxiliary function sequences. 
The distribution of distinct sets of prime exponents is still fairly regular so that 
$\omega(n)$ and $\Omega(n)$ play a crucial role in the repitition of common values of 
$g^{-1}(n)$. 
The following observation is suggestive of the quasi-periodicity of the distribution of 
distinct values of $g^{-1}(n)$ over $n \geq 2$: 

\begin{heuristic}[Symmetry in $g^{-1}(n)$ in the prime factorizations of $n$] 
Suppose that $n_1, n_2 \geq 2$ are such that their factorizations into distinct primes are 
given by $n_1 = p_1^{\alpha_1} \cdots p_r^{\alpha_r}$ and $n_2 = q_1^{\beta_1} \cdots q_r^{\beta_r}$ 
for some $r \geq 1$. 
If $\{\alpha_1, \ldots, \alpha_r\} \equiv \{\beta_1, \ldots, \beta_r\}$ as multisets of prime exponents, 
then $g^{-1}(n_1) = g^{-1}(n_2)$. For example, $g^{-1}$ has the same values on the squarefree integers 
with exactly one, two, three, and so on prime factors.  
\end{heuristic} 

\NBRef{A01-2020-04-26}
\begin{conjecture}
\label{lemma_gInv_MxExample} 
We have the following properties characterizing the 
Dirichlet inverse function $g^{-1}(n)$: 
\begin{itemize} 

\item[\textbf{(A)}] $g^{-1}(1) = 1$; 
\item[\textbf{(B)}] For all $n \geq 1$, $\operatorname{sgn}(g^{-1}(n)) = \lambda(n)$; 
\item[\textbf{(C)}] For all squarefree integers $n \geq 1$, we have that 
     \[
     |g^{-1}(n)| = \sum_{m=0}^{\omega(n)} \binom{\omega(n)}{m} \cdot m!; 
     \]
\item[\textbf{(D)}] If $n \geq 2$ and $\Omega(n) = k$, then 
     \[
     2 \leq |g^{-1}(n)| \leq \sum_{m=0}^{k} \binom{k}{m} \cdot m!. 
     \]
\end{itemize} 
\end{conjecture} 

We illustrate parts (B)--(D) of the conjecture clearly using the computation of initial values of 
this inverse sequence in 
Table \ref{table_conjecture_Mertens_ginvSeq_approx_values}. 
A proof of (C) in fact follows from 
Lemma \ref{lemma_AnExactFormulaFor_gInvByMobiusInv_v1} 
stated on page \pageref{lemma_AnExactFormulaFor_gInvByMobiusInv_v1}. 
The realization that the beautiful and remarkably simple combinatorial form of property (C) 
in Conjecture \ref{lemma_gInv_MxExample} holds for all squarefree $n \geq 1$ 
motivates our pursuit of simpler formulas for the inverse functions $g^{-1}(n)$ 
through sums of auxiliary sequences of arithmetic functions 
(see Section \ref{Section_InvFunc_PreciseExpsAndAsymptotics}). 

With this in mind, for natural numbers $n \geq 1$ and $k \geq 0$, let 
\begin{align*} 
C_k(n) := \begin{cases} 
     \varepsilon(n) = \delta_{n,1}, & \text{ if $k = 0$; } \\ 
     \sum\limits_{d|n} \omega(d) C_{k-1}(n/d), & \text{ if $k \geq 1$. } 
     \end{cases} 
\end{align*} 
For any $n \geq 1$, we can prove that (see Lemma \ref{lemma_AnExactFormulaFor_gInvByMobiusInv_v1})
\begin{equation} 
\label{eqn_AnExactFormulaFor_gInvByMobiusInv_v2-Intro_v1} 
g^{-1}(n) = \lambda(n) \times \sum_{d|n} \mu^2\left(\frac{n}{d}\right) C_{\Omega(d)}(d). 
\end{equation} 
We prove that (see Proposition \ref{prop_Mx_SBP_IntegralFormula}) 
\[
M(x) \approx G^{-1}(x) - \sum_{k=1}^{x/2} G^{-1}(k) \cdot \frac{x}{k^2 \log(x/k)}. 
\]
The formula in \eqref{eqn_AnExactFormulaFor_gInvByMobiusInv_v2-Intro_v1} 
then implies that we can establish new \emph{lower bounds} on $M(x)$ along large infinite subsequences 
by bounding appropriate estimates of the summatory function $G^{-1}(x)$. 

\subsection{Uniform asymptotics from enumerative bivariate DGFs in Mongomery and Vaughan} 

\begin{theorem}[Montgomery and Vaughan]
\label{theorem_HatPi_ExtInTermsOfGz} 
Recall that we have defined 
$$\widehat{\pi}_k(x) := \#\{n \leq x: \Omega(n)=k\}.$$ 
For $R < 2$ we have that uniformly for all $1 \leq k \leq R \log\log x$ 
\[
\widehat{\pi}_k(x) = \mathcal{G}\left(\frac{k-1}{\log\log x}\right) \frac{x}{\log x} 
     \frac{(\log\log x)^{k-1}}{(k-1)!} \left[1 + O_R\left(\frac{k}{(\log\log x)^2}\right)\right], 
\]
where 
\[
\mathcal{G}(z) := \frac{1}{\Gamma(z+1)} \times 
     \prod_p \left(1-\frac{z}{p}\right)^{-1} \left(1-\frac{1}{p}\right)^z, 0 \leq |z| \leq R. 
\]
\end{theorem} 

The proof of the next result is combinatorially motivated in so much as it interprets 
lower bounds on a key infinite product factor of $\mathcal{G}(z)$ defined in 
Theorem \ref{theorem_HatPi_ExtInTermsOfGz} 
as corresponding to an ordinary generating function of certain 
homogeneous symmetric polynomials involving the primes. This interpretation allows us to recover the 
following uniform lower bounds on $\widehat{\pi}_k(x)$ as $x \rightarrow \infty$: 

\begin{theorem} 
\label{theorem_GFs_SymmFuncs_SumsOfRecipOfPowsOfPrimes} 
\label{cor_BoundsOnGz_FromMVBook_initial_stmt_v1} 
We have that for all sufficiently large $x \rightarrow \infty$ and 
$1 \leq k \leq \log\log x$ 
\[
\mathcal{G}\left(\frac{1-k}{\log\log x}\right) \gg 
     \frac{2^{\frac{3}{4}} (\log 2)^{\frac{1}{2}}}{x^{\frac{3}{4}} (\log x)^{\frac{1}{2}}} 
     \exp\left(-\frac{15}{16} (\log 2)^2\right). 
\]
Then for all large enough $x$ we have uniformly for $1 \leq k \leq \log\log x$ that 
\[
\widehat{\pi}_k(x) \gg \frac{\widehat{C}_0 x^{\frac{1}{4}}}{(\log x)^{\frac{3}{2}}} 
     \frac{(\log\log x)^{k-1}}{(k-1)!} \left[1 + 
     O\left(\frac{k}{(\log\log x)^3}\right)\right], 
\]
where the absolute constant is defined by 
$\widehat{C}_0 := 2^{\frac{3}{4}} (\log 2)^{\frac{1}{2}} 
 \exp\left(-\frac{15}{16} (\log 2)^2\right) \approx 0.892418$. 
\end{theorem} 

\subsubsection{Applications of the new uniform lower bound estimates} 

Our inspiration for the new bounds found in the last sections of this article allows us to 
approximate sums of certain bounded 
non-negative arithmetic functions weighted by the Liouville lambda function 
$\lambda(n)$ taken over all $n \leq x$ well from below as $x \rightarrow \infty$. 
In fact, in 
Section \ref{subsubSection_RoutineProofsNeededForMainBoundOnGInvxFunc} 
we prove the following lemma: 

\begin{lemma} 
\label{lemma_CLT_and_AbelSummation} 
Suppose that $f(n)$ is an arithmetic function defined 
such that $f(n) > 0$ for all $n > u_0$ where 
$f(n) \SuccSim \widehat{\tau}_{\ell}(n) > 0$ whenever $n > u_0$ 
as $n \rightarrow \infty$. Assume also that 
the bounding function $\widehat{\tau}_{\ell}(t)$ is a 
continuously differentiable function of $t$ for all 
large enough $t \gg u_0$.  
We define the $\lambda$-sign-scaled summatory function of $f$ as follows: 
\[
F_{\lambda}(x) := \sum_{\substack{u_0 < n \leq x}} \lambda(n) f(n). 
\]
Let the summatory weight function be defined as 
\begin{align*} 
A_{\Omega}(t) & := \sum_{k=1}^{\floor{\log\log t}} (-1)^k \widehat{\pi}_k(t). 
\end{align*} 
Suppose that $|A_{\Omega}(t)| \gg |A_{\Omega}^{(\ell)}(t)|$ as $t \rightarrow \infty$, the function 
$|A_{\Omega}^{(\ell)}(t)|$ is monotone increasing for $t \gg 2$ large, and that 
$\left\lvert \widehat{\tau}_{\ell}\left(\frac{\log\log x}{2}\right) - 
 \widehat{\tau}_{\ell}\left(\frac{\log\log x}{2} - \frac{1}{2}\right) \right\rvert = 
 O\left(\frac{\widehat{\tau}_{\ell}(x)}{\log\log x}\right)$ as $x \rightarrow \infty$. 
Then we have that for sufficiently large $x > e$ 
\begin{equation} 
\label{eqn_Flambdax_RHA_AbelSummationFormula_v1} 
|F_{\lambda}(x)| \SuccSim \left\lvert 
     \left\lvert A_{\Omega}^{(\ell)}(x) \widehat{\tau}_{\ell}(x) \right\rvert - 
     \int_{\frac{\log\log x}{2} - \frac{1}{2}}^{\frac{\log\log x}{2}} 
     \left\lvert A_{\Omega}^{(\ell)}\left(e^{e^{2t}}\right) 
     \widehat{\tau}_{\ell}^{\prime}\left(e^{e^{2t}}\right) 
     \right\rvert e^{e^{2t}} dt 
     \right\rvert.  
\end{equation} 
\end{lemma} 

\subsubsection{Remarks} 

We emphasize the relevant recency of the method demonstrated by 
Montgomery and Vaughan in constructing a proof of 
Theorem \ref{theorem_HatPi_ExtInTermsOfGz}. 
To the best of our knowledge, this textbook reference is 
one of the first clear-cut applications documenting something of a hybrid 
DGF-and-OGF approach to enumerating sequences of arithmetic functions 
and their summatory functions. The hybrid method is motivated by the fact that it 
does not require a direct appeal to 
traditional highly oscillatory DGF-only inversions and integral formulas 
involving the Riemmann zeta function. 
This newer interpretaion of certain bivariate DGFs 
offers a window into the best of both generating function series worlds: 
it combines an additive structure 
implicit to the coefficients indexed by a formal power series variable formed by 
multiplication of these structures, while coordinating the distinct DGF-best 
property of the multiplicativity of prime powers invoked 
by taking powers of a reciprocal Euler product. 

\subsection{Cracking the classical unboundedness barrier} 

In Section \ref{Section_KeyApplications}, 
we are able to state what forms a bridge between the results 
we carefully prove up to that point the article. 
What we obtain at the conclusion of the section 
is the next summary theorem that unconditionally 
resolves the classical question of the 
unboundedness of the scaled function Mertens function 
$q(x) := |M(x)| / \sqrt{x}$ in the limit supremum sense. 

\begin{theorem}[Unboundedness of the the Mertens function, $q(x)$] 
\label{cor_ThePipeDreamResult_v1} 
We have that 
\[
\limsup_{x \rightarrow \infty} \frac{|M(x)|}{\sqrt{x}} = +\infty. 
\]
\end{theorem} 

In particular, we show that 
$$\left\lvert M\left(e^{2e^{e^{2y+1}}}\right) \right\rvert e^{-e^{e^{2y+1}}} \gg 
  \frac{e^{\frac{1}{4} e^{2y+1}}}{e^{2y+1} \cdot (2y+1)^2}, \mathrm{\ as\ } 
  y \rightarrow \infty.$$ 
In establishing the rigorous proof of 
Theorem \ref{cor_ThePipeDreamResult_v1} 
based on our new methods, we not only show unboundedness of 
$q(x)$, but also set a minimal rate (along a large infinite subsequence) 
at which this form of the 
scaled Mertens function grows without bound. 

\newpage 
\section{Preliminary proofs of new results} 
\label{Section_PrelimProofs_Config} 

\subsection{Establishing the summatory function properties and inversion identities} 

We will first prove Theorem \ref{theorem_SummatoryFuncsOfDirCvls} 
using an intuitive construction by matrix methods. 
Related results on summations of Dirichlet convolutions appear in 
\cite[\S 2.14; \S 3.10; \S 3.12; \cf \S 4.9, p.\ 95]{APOSTOLANUMT}. 

\begin{proof}[Proof of Theorem \ref{theorem_SummatoryFuncsOfDirCvls}] 
\label{proofOf_theorem_SummatoryFuncsOfDirCvls} 
Let $h,g$ be arithmetic functions such that $g(1) \neq 0$. 
Denote the summatory functions of $h$ and $g$, 
respectively, by $H(x) = \sum_{n \leq x} h(n)$ and $G(x) = \sum_{n \leq x} g(n)$. 
We define $\pi_{g \ast h}(x)$ to be the summatory function of the 
Dirichlet convolution of $g$ with $h$. 
Then we have that the following formulas hold for all $x \geq 1$: 
\begin{align} 
\notag 
\pi_{g \ast h}(x) & := \sum_{n=1}^{x} \sum_{d|n} g(n) h(n/d) = \sum_{d=1}^x g(d) H\left(\floor{\frac{x}{d}}\right) \\ 
\label{eqn_proof_tag_PigAsthx_ExactSummationFormula_exp_v2} 
     & = \sum_{i=1}^x \left[G\left(\floor{\frac{x}{i}}\right) - G\left(\floor{\frac{x}{i+1}}\right)\right] H(i). 
\end{align} 
In particular, the first formula above is well known. The second formula is justified directly using 
summation by parts as\footnote{
     For any arithmetic functions, $u_n,v_n$, 
     with $U_j := u_1+u_2+\cdots+u_j$ for $j \geq 1$, we have that 
     \cite[\S 2.10(ii)]{NISTHB} 
     \[
     \sum_{j=1}^{n-1} u_j \cdot v_j = U_{n-1} v_n + 
          \sum_{j=1}^{n-1} U_j \left(v_j - v_{j+1}\right), n \geq 2. 
     \]
} 
\begin{align*} 
\pi_{g \ast h}(x) & = \sum_{d=1}^x h(d) G\left(\floor{\frac{x}{d}}\right) \\ 
     & = \sum_{i \leq x} \left(\sum_{j \leq i} h(j)\right) \times 
     \left[G\left(\floor{\frac{x}{i}}\right) - 
     G\left(\floor{\frac{x}{i+1}}\right)\right]. 
\end{align*} 
We next form the invertible matrix of coefficients associated with this linear system defining $H(j)$ for all 
$1 \leq j \leq x$ resulting from \eqref{eqn_proof_tag_PigAsthx_ExactSummationFormula_exp_v2} by defining 
\[
g_{x,j} := G\left(\floor{\frac{x}{j}}\right) - G\left(\floor{\frac{x}{j+1}}\right) \equiv G_{x,j} - G_{x,j+1}, 
\] 
where 
\[
G_{x,j} := G\left(\Floor{x}{j}\right), \forall 1 \leq j \leq x. 
\]
Since $g_{x,x} = G(1) = g(1)$ and $g_{x,j} = 0$ for all $j > x$, 
the matrix we must invert in this problem is lower triangular with a non-zero 
constant on its diagonals, and is hence invertible. 
Moreover, if we let $\hat{G} := (G_{x,j})$, then this matrix is 
expressable by an invertible shift operation as 
\[
(g_{x,j}) = \hat{G} (I - U^{T}). 
\]
Here, $U$ is a square matrix with sufficiently large finite dimensions 
whose $(i,j)^{th}$ entries are defined by $(U)_{i,j} = \delta_{i+1,j}$ such that 
\[
\left[(I - U^T)^{-1}\right]_{i,j} = \Iverson{j \leq i}. 
\]
It is a useful fact that if we take successive differences in $x$ of the 
floor of certain fractions, $\Floor{x}{j}$, in the form of 
\[
\Floor{x}{j} - \Floor{x-1}{j} = \begin{cases} 
     1, & \text{ if $j|x$; } \\ 
     0, & \text{ otherwise, } 
     \end{cases} 
\] 
for $1 \leq j \leq x$, we obtain non-zero differences at the indices $j$ taken precisely 
over the divisors of $x$. 
This implies that 
\begin{equation} 
\label{eqn_proof_tag_FloorFuncDiffsOfSummatoryFuncs_v2} 
G\left(\floor{\frac{x}{j}}\right) - G\left(\floor{\frac{x-1}{j}}\right) = 
     \begin{cases} 
     g\left(\frac{x}{j}\right), & \text{ if $j | x$; } \\ 
     0, & \text{ otherwise. } 
     \end{cases}
\end{equation} 
We use the last property in \eqref{eqn_proof_tag_FloorFuncDiffsOfSummatoryFuncs_v2} 
to shift the matrix $\hat{G}$, and then invert the result to obtain a matrix involving the 
Dirichlet inverse of $g$ in the following form: 
\begin{align*} 
\left[(I-U^{T}) \hat{G}\right]^{-1} & = \left(g\left(\frac{x}{j}\right) \Iverson{j|x}\right)^{-1} = 
     \left(g^{-1}\left(\frac{x}{j}\right) \Iverson{j|x}\right). 
\end{align*} 
Now we can express the inverse of our target matrix, 
$$(g_{x,j}) = (I-U^{T}) \left(g\left(\frac{x}{j}\right) \Iverson{j|x}\right) (I-U^{T})^{-1},$$
using a similarity transformation conjugated by shift operators as follows: 
\begin{align*} 
(g_{x,j})^{-1} & = (I-U^{T})^{-1} \left(g^{-1}\left(\frac{x}{j}\right) \Iverson{j|x}\right) (I-U^{T}) \\ 
     & = \left(\sum_{k=1}^{\floor{\frac{x}{j}}} g^{-1}(k)\right) (I-U^{T}) \\ 
     & = \left(\sum_{k=1}^{\floor{\frac{x}{j}}} g^{-1}(k) - \sum_{k=1}^{\floor{\frac{x}{j+1}}} g^{-1}(k)\right). 
\end{align*} 
Hence, the summatory function $H(x)$ is exactly expressed for any $x \geq 1$ 
by a vector product with the inverse matrix from the previous equation given by 
\begin{align*} 
H(x) & = \sum_{k=1}^x g_{x,k}^{-1} \cdot \pi_{g \ast h}(k) 
     = \sum_{k=1}^x \left(\sum_{j=\floor{\frac{x}{k+1}}+1}^{\floor{\frac{x}{k}}} g^{-1}(j)\right) 
     \cdot \pi_{g \ast h}(k). 
\end{align*} 
We can prove an inversion formula providing the coefficients of $G^{-1}(i)$ for $1 \leq i \leq x$ given 
by the last equation by adapting our argument to prove 
\eqref{eqn_proof_tag_PigAsthx_ExactSummationFormula_exp_v2} above. 
This leads to the identity that 
\[
H(x) = \sum_{k=1}^{x} g^{-1}(x) \pi_{g \ast h}\left(\Floor{x}{k}\right). 
     \qedhere 
\]
\end{proof} 

\subsection{Proving the characteristic signedness property of $g^{-1}(n)$} 

Let $\chi_{\mathbb{P}}$ denote the characteristic function of the primes, 
$\varepsilon(n) = \delta_{n,1}$ be the multiplicative identity with respect to Dirichlet convolution, 
and denote by $\omega(n)$ the strongly additive function that counts the number of 
distinct prime factors of $n$. Then we can easily prove using DGFs that 
\begin{equation}
\label{eqn_AntiqueDivisorSumIdent} 
\chi_{\mathbb{P}} + \varepsilon = (\omega + 1) \ast \mu. 
\end{equation} 
When combined with Corollary \ref{cor_CvlGAstMu} 
this convolution identity yields the exact 
formula for $M(x)$ stated in \eqref{eqn_Mx_gInvnPixk_formula} of 
Corollary \ref{cor_Mx_gInvnPixk_formula}. 

\begin{prop}[The signedness property of $g^{-1}(n)$]
\label{prop_SignageDirInvsOfPosBddArithmeticFuncs_v1} 
Let the operator 
$\operatorname{sgn}(h(n)) = \frac{h(n)}{|h(n)| + \Iverson{h(n) = 0}} \in \{0, \pm 1\}$ denote the sign 
of the arithmetic function $h$ at integers $n \geq 1$. 
For the Dirichlet invertible function, $g(n) := \omega(n) + 1$, 
we have that $\operatorname{sgn}(g^{-1}(n)) = \lambda(n)$ for all $n \geq 1$. 
\NBRef{A02-2020-04-26}
\end{prop} 
\begin{proof} 
The function $D_f(s) := \sum_{n \geq 1} f(n) n^{-s}$ denotes the 
\emph{Dirichlet generating function} (DGF) of any 
arithmetic function $f(n)$ which is convergent for all $s \in \mathbb{C}$ satisfying 
$\Re(s) > \sigma_f$ for $\sigma_f$ the abcissa of convergence of the series. 
Recall that $D_1(s) = \zeta(s)$, $D_{\mu}(s) = 1 / \zeta(s)$ and $D_{\omega}(s) = P(s) \zeta(s)$ for 
$\Re(s) > 1$. 
Then by \eqref{eqn_AntiqueDivisorSumIdent} and the known property that the DGF of $f^{-1}(n)$ is 
the reciprocal of the DGF of any arithmetic function $f$ such that $f(1) \neq 0$, 
we have for all $\Re(s) > 1$ that 
\begin{align} 
\label{eqn_DGF_of_gInvn} 
D_{(\omega+1)^{-1}}(s) = \frac{1}{(P(s)+1) \zeta(s)}. 
\end{align} 
It follows that $(\omega + 1)^{-1}(n) = (h^{-1} \ast \mu)(n)$ when we take 
$h := \chi_{\mathbb{P}} + \varepsilon$, i.e., where the function $h$ has the DGF 
$D_h(s) := P(s) + 1$ for $\Re(s) > 1$. 
We first show that $\operatorname{sgn}(h^{-1}) = \lambda$. 
This observation implies 
that $\operatorname{sgn}(h^{-1} \ast \mu) = \lambda$. The remainder of the proof fills in the 
precise details needed to make our claims rigorous. 

By the recurrence relation that defines the Dirichlet inverse function of any 
arithmetic function $h$ such that $h(1) = 1$, we have that \cite[\S 2.7]{APOSTOLANUMT} 
\begin{equation} 
\label{eqn_proof_tag_hInvn_ExactRecFormula_v1}
h^{-1}(n) = \begin{cases} 
            1, & n = 1; \\ 
            -\sum\limits_{\substack{d|n \\ d>1}} h(d) h^{-1}(n/d), & n \geq 2. 
            \end{cases} 
\end{equation} 
For $n \geq 2$, the summands in \eqref{eqn_proof_tag_hInvn_ExactRecFormula_v1} 
can be simply indexed over the primes $p|n$ given our definition of $h$ from above. 
This observation yields that we can inductively 
unfold these sums into nested divisor sums provided the depth of the 
expanded divisor sums does not exceed the 
capacity to index summations over the primes dividing $n$. Namely, notice that for $n \geq 2$ 
\begin{align*} 
h^{-1}(n) & = -\sum_{p|n} h^{-1}\left(\frac{n}{p}\right), && \text{\ if\ } \Omega(n) \geq 1 \\ 
     & = \sum_{p_1|n} \sum_{p_2|\frac{n}{p_1}} h^{-1}\left(\frac{n}{p_1p_2}\right), && \text{\ if\ } \Omega(n) \geq 2 \\ 
     & = -\sum_{p_1|n} \sum_{p_2|\frac{n}{p_1}} \sum_{p_3|\frac{n}{p_1p_2}} h^{-1}\left(\frac{n}{p_1p_2p_3}\right), 
     && \text{\ if\ } \Omega(n) \geq 3. 
\end{align*} 
Then by induction with $h^{-1}(1) = h(1) = 1$, we expand these 
nested divisor sums as above to the maximal possible depth as 
\begin{equation} 
\label{eqn_proof_tag_hInvn_ExactNestedSumFormula_v2} 
\lambda(n) \cdot h^{-1}(n) = \sum_{p_1|n} \sum_{p_2|\frac{n}{p_1}} \times \cdots \times 
     \sum_{p_{\Omega(n)}|\frac{n}{p_1p_2 \cdots p_{\Omega(n)-1}}} 1, n \geq 2. 
\end{equation} 
In fact, by a combinatorial argument we recover exactly that 
\begin{equation} 
\label{eqn_proof_tag_hInvn_ExactNestedSumFormula_CombInterpetIdent_v3} 
h^{-1}(n) = \lambda(n) \frac{(\alpha_1+\cdots+\alpha_{\omega(n)})!}{ 
     \alpha_1! \alpha_2! \cdots \alpha_{\omega(n)}!} = 
     \lambda(n) \binom{\Omega(n)}{\alpha_1,\alpha_2,\ldots,\alpha_{\omega(n)}}. 
\end{equation} 
So the property in \eqref{eqn_proof_tag_hInvn_ExactNestedSumFormula_v2} 
implies that the following property holds for all $n \geq 1$: 
\begin{equation} 
\notag 
\operatorname{sgn}(h^{-1}(n)) = \lambda(n). 
\end{equation} 
Since $\lambda$ is completely multiplicative we have that 
$\lambda\left(\frac{n}{d}\right) \lambda(d) = \lambda(n)$ for all 
$d|n$ and $n \geq 1$. Since we also know that $\mu(n) = \lambda(n)$ whenever $n$ is squarefree, 
we obtain that 
\[
g^{-1}(n) = (h^{-1} \ast \mu)(n) = \lambda(n) \times \sum_{d|n} \mu^2\left(\frac{n}{d}\right) |h^{-1}(n)|, n \geq 1. 
     \qedhere 
\]
\end{proof} 

\subsection{Statements of other facts and known limiting asymptotics} 
\label{subSection_OtherFactsAndResults} 

\begin{theorem}[Mertens theorem]
\label{theorem_Mertens_theorem} 
For all $x \geq 2$ we have that 
\[
P_1(x) := \sum_{p \leq x} \frac{1}{p} = \log\log x + B + o(1), 
     \mathrm{\ as\ } x \rightarrow \infty, 
\]
where 
$B \approx 0.2614972128476427837554$ 
is an absolute constant\footnote{ 
     Precisely, we have that the \emph{Mertens constant} is defined by 
     \cite[\seqnum{A077761}]{OEIS} 
     \[
     B = \gamma + \sum_{m \geq 2} \frac{\mu(m)}{m} \log\left[\zeta(m)\right]. 
     \]
}.
\end{theorem} 

\begin{cor}[Product form of Mertens theorem] 
\label{lemma_Gz_productTermV2} 
We have that for all sufficiently large $x \gg 2$ 
\[
\prod_{p \leq x} \left(1 - \frac{1}{p}\right) = \frac{e^{-\gamma}}{\log x}\left( 
     1 + o(1)\right), \mathrm{\ as\ } x \rightarrow \infty, 
\]
where the notation for the absolute constant $0 < B < 1$ coincides with the definition of 
Mertens constant from Theorem \ref{theorem_Mertens_theorem}. 
Hence, for any real $z \geq 0$ we obtain that 
\[
\prod_{p \leq x} \left(1 - \frac{1}{p}\right)^{z} \sim 
     \frac{e^{-\gamma z}}{(\log x)^{z}}, \mathrm{\ as\ } x \rightarrow \infty. 
\]
\end{cor} 

Proofs of Theorem \ref{theorem_Mertens_theorem} and 
Corollary \ref{lemma_Gz_productTermV2} are given in 
\cite[\S 22.7; \S 22.8]{HARDYWRIGHT}. 
We have a related analog of Corollary \ref{lemma_Gz_productTermV2} 
that is justified using the Euler product representation for the 
Riemann zeta function: 
\begin{align*} 
\prod_{p \leq x} \left(1 + \frac{1}{p}\right) & = \prod_{p \leq x} 
     \frac{\left(1 - p^{-2}\right)}{\left(1 - p^{-1}\right)} 
     = \zeta(2) e^{\gamma} (\log x) (1 + o(1)), 
     \mathrm{\ as\ } x \rightarrow \infty. 
\end{align*} 

\begin{facts}[Exponential integrals and the incomplete gamma function] 
\label{facts_ExpIntIncGammaFuncs} 
\begin{subequations}
Two variants of the \emph{exponential integral function} are defined by the 
integral next representations \cite[\S 8.19]{NISTHB}. 
\begin{align*} 
\operatorname{Ei}(x) & := \int_{-x}^{\infty} \frac{e^{-t}}{t} dt, x \in \mathbb{R} \\ 
E_1(z) & := \int_1^{\infty} \frac{e^{-tz}}{t} dt, \Re(z) \geq 0 
\end{align*} 
These functions are related by $\operatorname{Ei}(-kz) = -E_1(kz)$ for real $k, z > 0$. 
We have the following inequalities providing 
quasi-polynomial upper and lower bounds on $\operatorname{Ei}(\pm x)$ for all real $x > 0$: 
\begin{align}
\gamma + \log x - x \leq & \operatorname{Ei}(-x) \leq \gamma + \log x - x + \frac{x^2}{4}, \\ 
\notag 
1 + \gamma + \log x -\frac{3}{4} x \leq & \operatorname{Ei}(x) \phantom{-} \leq 
     1 + \gamma + \log x -\frac{3}{4} x + \frac{11}{36} x^2. 
\end{align}
The (upper) \emph{incomplete gamma function} is defined by \cite[\S 8.4]{NISTHB} 
\[
\Gamma(s, x) = \int_{x}^{\infty} t^{s-1} e^{-t} dt, \Re(s) > 0. 
\]
The following properties of $\Gamma(s, x)$ hold: 
\begin{align} 
\label{eqn_IncompleteGamma_PropA} 
\Gamma(s, x) & = (s-1)! \cdot e^{-x} \times \sum_{k=0}^{s-1} \frac{x^k}{k!}, s \in \mathbb{Z}^{+}, x > 0, \\ 
\label{eqn_IncompleteGamma_PropB} 
\Gamma(s, x) & \sim x^{s-1} \cdot e^{-x}, s > 0, \mathrm{\ as\ } x \rightarrow \infty. 
\end{align}
\end{subequations}
\end{facts} 

\newpage 
\section{Components to the asymptotic analysis of lower bounds for 
         sums of arithmetic functions weighted by $\lambda(n)$} 
\label{Section_MVCh7_GzBounds} 

\subsection{A discussion of the results proved by Montgomery and Vaughan} 
\label{subSection_MVPrereqResultStmts} 

\begin{remark}[Intuition and constructions in Theorem \ref{theorem_HatPi_ExtInTermsOfGz}] 
\label{remark_intuitionConstrIn_theorem_HatPi_ExtInTermsOfGz} 
For $|z| < 2$ and $\Re(s) > 1$, let 
\begin{equation} 
\label{eqn_IntuitionMVThm_FszFuncDef_v1} 
F(s, z) := \prod_{p} \left(1 - \frac{z}{p^s}\right)^{-1} \left(1 - \frac{1}{p^s}\right)^{z}, 
\end{equation} 
and define the DGF coefficients, $a_z(n)$ for $n \geq 1$, by the product 
\[
\zeta(s)^{z} \cdot F(s, z) := \sum_{n \geq 1} \frac{a_z(n)}{n^s}, \Re(s) > 1. 
\]
Suppose that $A_z(x) := \sum_{n \leq x} a_z(n)$ for $x \geq 1$. Then we obtain the next 
generating function like identity in $z$. \footnote{ 
     In fact, for any additive arithmetic function $a(n)$, 
     characterized by the property that 
     $a(n) = \sum_{p^{\alpha} || n} a(p^{\alpha})$ for all $n \geq 2$, we have that 
     \cite[\cf \S 1.7]{IWANIEC-KOWALSKI} 
     \[
     \sum_{n \geq 1} \frac{z^{a(n)}}{n^s} = \prod_p \left( 
          1 - \sum_{m \geq 1} \frac{z^{a(p^m)}}{p^{ms}}\right)^{-1}, \Re(s) > 1. 
     \]
}
\begin{equation} 
\label{eqn_remark_MV_AzxCoeffFormlaIntegral_v1} 
A_z(x) = \sum_{n \leq x} z^{\Omega(n)} = \sum_{k \geq 0} \widehat{\pi}_k(x) z^k 
\end{equation} 
Thus for $r < 2$, by Cauchy's integral formula we have 
\[
\widehat{\pi}_k(x) = \frac{1}{2\pi\imath} \int_{|z|=r} \frac{A_z(x)}{z^{k+1}} dz. 
\]
Selecting $r := \frac{k-1}{\log\log x}$ for $1 \leq k < 2\log\log x$ 
leads to the uniform asymptotic formulas for $\widehat{\pi}_k(x)$ given in 
Theorem \ref{theorem_HatPi_ExtInTermsOfGz}. 
Montgomery and Vaughan then consider individual analysis of the main and error 
terms for $A_z(x)$ to prove that 
\[
\widehat{\pi}_k(x) = \mathcal{G}\left(\frac{k-1}{\log\log x}\right) \frac{x}{\log x} \cdot 
     \frac{(\log\log x)^{k-1}}{(k-1)!} \left[1 + O\left(\frac{k}{(\log\log x)^2}\right)\right]. 
\]
%It is remarked that a more precise result can be stated as follows: 
%[
%\widehat{\pi}_k(x) =  \frac{x}{\log x} \times \left(\sum_{\nu=0}^{k-1} 
%     \frac{\mathcal{G}^{(\nu)}(0)}{\nu!}  
%     \frac{(\log\log x)^{k-1-\nu}}{(k-1-\nu)!} \right) \times 
%     \left[1 + O\left(\frac{k}{(\log\log x)^2}\right)\right]. 
%\]
We will require estimates of $A_{-z}(x)$ from below to form summatory functions 
that weight the terms of $\lambda(n)$ in our formulas in the next sections. 
\end{remark} 

\subsection{New uniform asymptotics based on refinements of Theorem \ref{theorem_HatPi_ExtInTermsOfGz}} 
\label{subSection_PartialPrimeProducts_Proofs} 

\begin{prop} 
\label{cor_PartialSumsOfReciprocalsOfPrimePowers} 
For real $s \geq 1$, let 
\[
P_s(x) := \sum_{p \leq x} p^{-s}, x \geq 2. 
\]
When $s := 1$, we have the asymptotic formula from Mertens theorem 
(see Theorem \ref{theorem_Mertens_theorem}). 
For all integers $s \geq 2$ 
there is an absolutely defined bounding function $\gamma_0(s, x)$ such that 
\[
\gamma_0(s, x) + o(1) \leq P_s(x), \mathrm{\ as\ } x \rightarrow \infty. 
\] 
It suffices to define the bound in the previous equation as as the 
quasi-polynomial function in $s$ and $x$ given by 
\begin{align*} 
\gamma_0(s, x) & = s\log\left(\frac{\log x}{\log 2}\right) - 
     s(s-1) \log\left(\frac{x}{2}\right) - 
     \frac{1}{4} s(s-1)^2 \log^2(2). 
\end{align*}
\end{prop} 
\NBRef{A05-2020-04-26} 
\begin{proof} 
Let $s > 1$ be real-valued. 
By Abel summation with the smooth summatory function 
$A(x) = \pi(x) \sim \frac{x}{\log x}$, and where 
our target function smooth function is $f(t) = t^{-s}$ so that 
$f^{\prime}(t) = -s \cdot t^{-(s+1)}$, we obtain that 
\begin{align*} 
P_s(x) & = \frac{1}{x^s \cdot \log x} + s \cdot \int_2^{x} \frac{dt}{t^s \log t} \\ 
     & = \operatorname{Ei}(-(s-1) \log x) - \operatorname{Ei}(-(s-1) \log 2) + o(1), 
     \mathrm{\ as\ } x \rightarrow \infty. 
\end{align*} 
Now using the inequalities in Facts \ref{facts_ExpIntIncGammaFuncs}, we obtain that the 
difference of the exponential integral functions is bounded above and below by 
\begin{align*} 
\frac{P_s(x)}{s} & \geq \log\left(\frac{\log x}{\log 2}\right) - (s-1) \log\left(\frac{x}{2}\right) - 
     \frac{1}{4} (s-1)^2 \log^2(2) \\ 
\frac{P_s(x)}{s} & \leq \log\left(\frac{\log x}{\log 2}\right) - (s-1) \log\left(\frac{x}{2}\right) + 
     \frac{1}{4} (s-1)^2 \log^2(x). 
     \qedhere 
\end{align*} 
\end{proof} 

We will first prove the stated form of the lower bound on 
$\left\lvert \mathcal{G}(-z) \right\rvert$ for $z := \frac{k-1}{\log\log x}$. 
Then we will discuss the technical adaptations to Montgomery and Vaughan's proof of 
Theorem \ref{theorem_HatPi_ExtInTermsOfGz} in 
Remark \ref{remark_TechAdjustments_theorem_HatPi_ExtInTermsOfGz_TO_GFs_SymmFuncs_SumsOfRecipOfPowsOfPrimes} 
to justify the new uniform asymptotic lower bounds on $\widehat{\pi}_k(x)$ for all 
$1 \leq k \leq \log\log x$. 

\NBRef{A06-2020-04-26} 
\begin{proof}[Proof of Theorem \ref{theorem_GFs_SymmFuncs_SumsOfRecipOfPowsOfPrimes}] 
\label{proofOf_theorem_GFs_SymmFuncs_SumsOfRecipOfPowsOfPrimes} 
For $0 \leq z < 2$ and integers $x \geq 2$, 
the right-hand-side of the following product is finite as $x \rightarrow \infty$: 
\[
\widehat{P}(z, x) := \prod_{p \leq x} \left(1 - \frac{z}{p}\right)^{-1}. 
\]
For fixed, finite $x \geq 2$ let 
\[
\mathbb{P}_x := \left\{n \geq 1: \mathrm{ all\ prime\ divisors\ } 
     p|n \mathrm{\ satisfy\ } p \leq x\right\}. 
\]
Then we can see as in the constructions from Montgomery and Vaughan sketeched in 
Remark \ref{remark_intuitionConstrIn_theorem_HatPi_ExtInTermsOfGz} that 
\begin{equation} 
\label{eqn_proof_tag_PHatFiniteTruncProdFactorOfGz_v2} 
\prod_{p \leq x} \left(1 - \frac{z}{p^s}\right)^{-1} = \sum_{n \in \mathbb{P}_x} 
     \frac{z^{\Omega(n)}}{n^s}, x \geq 2. 
\end{equation} 
By extending the argument in the proof given in 
\cite[\S 7.4]{MV}, we have that the formulas 
\[
A_{-z}(x) := \sum_{n \leq x} \lambda(n) z^{\Omega(n)} = 
     \sum_{k \geq 0} \widehat{\pi}_k(x) (-z)^k, 
\] 
If we let $a_n(z, x)$ be defined by the DGF 
\[
\sum_{n \geq 1} \frac{a_n(z, x)}{n^s} = \widehat{P}(z, x), 
\]
then we see that 
\[
\sum_{n \leq x} a_n(-z, x) = \sum_{n \leq x} \lambda(n) z^{\Omega(n)} = 
     \sum_{k=0}^{\log_2(x)} \widehat{\pi}_k(x) (-z)^k + 
     \sum_{k > \log_2(x)} e_k(x) (-z)^{k}. 
\]
This assertion if correct since the products of all non-negative integral powers of the 
primes $p \leq x$ generate the integers $\{1 \leq n \leq x\}$ as a subset. 
Thus we capture all of the relevant terms needed to express 
$(-1)^{k} \cdot \widehat{\pi}_k(x)$ 
via the Cauchy integral formula representation over $A_{-z}(x)$ by 
replacing the corresponding infinite product terms by 
$\widehat{P}(-z, x)$ in the definition of $\mathcal{G}(-z)$. 

Now we must argue that 
\[
\mathcal{G}(-z) \gg \prod_{p \leq x} \left(1 + \frac{z}{p}\right)^{-1} 
     \left(1 - \frac{1}{p}\right)^{-z}, \mathrm{\ for\ } 0 \leq z < 1, x \geq 2. 
\]
For $0 \leq z < 1$ and $x \geq 2$, we see that 
\begin{align*} 
\mathcal{G}(-z) & = \exp\left(-\sum_p \left[\log\left(1 + \frac{z}{p}\right) + 
     \log\left(1 - \frac{1}{p}\right)\right]\right) \\ 
     & \gg 
     \exp\left(-z \times \sum_{p > x} \left[
     \log\left(1 - \frac{1}{p}\right) + \frac{1}{p}\right] - 
     \sum_{p \leq x} \left[\log\left(1 + \frac{z}{p}\right) + 
     \log\left(1 - \frac{1}{p}\right)\right]\right) \\ 
     & = \widehat{P}(-z, x) \times \exp(|B-\gamma| z) \times 
     \exp\left(-z \times \sum_{p \leq x} \left[
     \log\left(1 - \frac{1}{p}\right) + \frac{1}{p}\right]\right) \\ 
     & \gg_z \widehat{P}(-z, x). 
\end{align*} 
We have for all integers $0 \leq k \leq m < +\infty$, and any sequence 
$\{f(n)\}_{n \geq 1}$ with bounded partial sums, that 
\cite[\S 2]{MACDONALD-SYMFUNCS} 
\begin{equation} 
\label{eqn_pf_tag_hSymmPolysGF} 
[z^k] \prod_{1 \leq i \leq m} (1-f(i) z)^{-1} = [z^k] \exp\left(\sum_{j \geq 1} 
     \left(\sum_{i=1}^m f(i)^j\right) \frac{z^j}{j}\right), |z| < 1. 
\end{equation} 
In our case we have that $f(i)$ denotes the reciprocal of the 
$i^{th}$ prime in the generating function expansion of 
\eqref{eqn_pf_tag_hSymmPolysGF}. 
It follows from Proposition \ref{cor_PartialSumsOfReciprocalsOfPrimePowers} that 
for real $0 \leq z < 1$ we obtain 
\begin{align} 
\notag 
\log\left[\prod_{p \leq x} \left(1+\frac{z}{p}\right)^{-1}\right] & \geq -(B + \log\log x) z + 
     \sum_{j \geq 2} \left[a(x) - b(x)(j-1) - c(x) (j-1)^2\right] (-z)^j \\ 
\label{eqn_proof_tag_PHatFiniteTruncProdFactorOfGz_v3} 
     & = -(B + \log\log x) z + a(x) \left(z + \frac{1}{1+z} - 1\right) \\ 
\notag 
     & \phantom{= -(B + \log\log x) z\ } + 
     b(x) \left( 
     1 - \frac{2}{1+z} + \frac{1}{(1+z)^2}\right) \\ 
\notag 
     & \phantom{= -(B + \log\log x) z\ } + 
     c(x) \left( 
     1 - \frac{4}{1+z} + \frac{5}{(1+z)^2} - \frac{2}{(1+z)^3}\right) \\ 
\notag 
     & =: \widehat{\mathcal{B}}(x; z). 
\end{align} 
The lower bounds formed by the functions 
$(a(x),b(x),c(x)) \equiv (a_{\ell}(x), b_{\ell}(x), c_{\ell}(x))$ in 
\eqref{eqn_proof_tag_PHatFiniteTruncProdFactorOfGz_v3} 
evaluated at $x$ are given by the corresponding lower bounds from 
Proposition \ref{cor_PartialSumsOfReciprocalsOfPrimePowers} as 
\begin{align*} 
(a_{\ell}(x), b_{\ell}(x), c_{\ell}(x)) & := 
     \left(\log\left(\frac{\log x}{\log 2}\right), 
     \log\left(\frac{x}{2}\right), \frac{1}{4} \log^2 2\right). 
\end{align*} 
We adjust the uniform bound parameter so that 
$$z \equiv z(k, x) = \frac{k-1}{\log\log x} \in \left[0, 1\right),$$ 
whenever $1 \leq k \leq \log\log x$ 
in the notation of Theorem \ref{theorem_HatPi_ExtInTermsOfGz}. 
This implies that $(1+z)^{-1} \in \left(\frac{1}{2}, 1\right]$. 

The extremal values of the coefficients of $c_{\ell}(x)$ 
contribute the following constant factor to our lower bound: 
\[
\exp\left(c_{\ell}(x) \left[ 
     1 - \frac{4}{1+z} + \frac{5}{(1+z)^2} - \frac{2}{(1+z)^3} 
     \right]\right) \geq \exp\left(-\frac{15}{16} (\log 2)^2\right) 
     \approx 0.637357. 
\]
We next consider the coefficients of $b_{\ell}(x)$ in our product expansion: 
\[
\exp\left(b_{\ell}(x) \left[1 - \frac{2}{1+z} + \frac{1}{(1+z)^2}\right]\right) 
     \geq \left(\frac{x}{2}\right)^{-\frac{3}{4}}. 
\]
Lastly, we will bound the contributions to the product from the coefficients of 
$a_{\ell}(x)$ as follows: 
\begin{align*} 
\exp\left(-a_{\ell}(x) \left[1 - \frac{1}{1+z} + z\right]\right) & \geq 
     \sqrt{\frac{\log 2}{\log x}} 
     \left(\frac{\log x}{\log 2}\right)^{z} \\ 
     & \gg 
     \sqrt{\frac{\log 2}{\log x}} e^{k-1} \gg 
     \sqrt{\frac{\log 2}{\log x}}. 
\end{align*} 
In summary, we have arrived at a proof that 
as $x \rightarrow \infty$
\begin{align} 
\label{eqn_proof_tag_simpl_v1} 
\frac{e^{\gamma z}}{(\log x)^{-z}} \times \exp\left(\widehat{\mathcal{B}}(u, x; z)\right) & \gg 
     \frac{2^{\frac{3}{4}} (\log 2)^{\frac{1}{2}}}{x^{\frac{3}{4}} (\log x)^{\frac{1}{2}}} 
     \exp\left(-\frac{15}{16} (\log 2)^2\right) \times e^{(\gamma - B) z} \\ 
\notag 
     & \gg \frac{\widehat{C}_0}{x^{3/4} \cdot (\log x)^{1/2}}, 
\end{align} 
where the leading constant is numerically approximated by 
$\widehat{C}_0 := 2^{\frac{3}{4}} \sqrt{\log 2} 
 \exp\left(-\frac{15}{16} (\log 2)^2\right) \approx 0.892418$. 

Finally, to finish our proof of the new form of the lower bound on $\mathcal{G}(-z)$, 
we need to bound the reciprocal factor of $\Gamma(1-z)$. 
Since $z \equiv z(k, x) = \frac{k-1}{\log\log x}$ and 
$k \in [1, \log\log x]$, or again with $z \in [0, 1)$, 
we obtain for minimal $k$ and all large enough $x \gg 1$ that 
$\Gamma(1-z) = \Gamma(1) = 1$, and for $k$ towards the upper range of 
its interval that 
\[
\Gamma(1-z) \approx \Gamma\left(\frac{1}{\log\log x}\right) = 
     \frac{1}{\log\log x} \Gamma\left(1 + \frac{1}{\log\log x}\right) 
     \approx \frac{1}{\log\log x}. 
     \qedhere 
\]
\end{proof} 

\begin{remark}[Technical adjustments in the proof of Theorem \ref{theorem_GFs_SymmFuncs_SumsOfRecipOfPowsOfPrimes}] 
\label{remark_TechAdjustments_theorem_HatPi_ExtInTermsOfGz_TO_GFs_SymmFuncs_SumsOfRecipOfPowsOfPrimes} 
We now discuss the differences between our construction and that in 
the technical proof of Theorem \ref{theorem_HatPi_ExtInTermsOfGz} 
given by Montgomery and Vaughan when we bound $\mathcal{G}(-z)$ from below as in 
Theorem \ref{theorem_GFs_SymmFuncs_SumsOfRecipOfPowsOfPrimes}. 
The reference proves that for real $0 \leq z < 2$ 
\begin{equation} 
\label{eqn_MV_Azx_formula} 
A_{-z}(x) = -\frac{z F(1, -z)}{\Gamma(1-z)} \cdot x (\log x)^{-(z+1)} + 
     O\left(x (\log x)^{-\Re(z) - 2}\right). 
\end{equation}
Recall that for $r < 2$ we have by Cauchy's integral formula that 
\begin{equation} 
\label{eqn_MV7.61_CIF} 
(-1)^{k} \widehat{\pi}_k(x) = \frac{1}{2\pi\imath} \int_{|z|=r} 
     \frac{A_{-z}(x)}{z^{k+1}} dz. 
\end{equation} 
We first claim that uniformly for large $x$ and $1 \leq k \leq \log\log x$ we have 
\begin{equation} 
\label{eqn_proof_tag_HatPikx_BoundForGmz_v1} 
\widehat{\pi}_k(x) = \mathcal{G}\left(\frac{1-k}{\log\log x}\right) \times 
     \frac{x (\log\log x)^{k-1}}{(\log x) (k-1)!} 
     \left[1 + O\left(\frac{k}{(\log\log x)^3}\right)\right]. 
\end{equation} 
Then since we have proved in Theorem \ref{theorem_HatPi_ExtInTermsOfGz} 
above that 
\[
\left\lvert \mathcal{G}\left(\frac{1-k}{\log\log x}\right) \right\rvert \gg 
     \frac{\widehat{C}_0}{x^{3/4} (\log x)^{1/2}} \cdot 
     \frac{(k-1)}{\log\log x}, 
\]
the result in \eqref{eqn_proof_tag_HatPikx_BoundForGmz_v1} implies our 
stated uniform asymptotic bound. Namely, we obtain that 
\[
\widehat{\pi}_k(x) \gg \frac{\widehat{C}_0 x^{\frac{1}{4}}}{(\log x)^{\frac{3}{2}}} 
     \frac{(\log\log x)^{k-1}}{(k-1)!} \left[1 + 
     O\left(\frac{k}{(\log\log x)^2}\right)\right]. 
\]
We have to provide analogs to the two separate bounds corresponding to the error and 
main terms of our estimate according to 
\eqref{eqn_MV_Azx_formula} and \eqref{eqn_MV7.61_CIF}. 
The error term estimate is simpler, so we tackle it first in the next argument. 
The second part of our proof establishing the main term in 
\eqref{eqn_proof_tag_HatPikx_BoundForGmz_v1} 
requires us to duplicate and adjust significant parts of the 
fine-tuned reasoning given in the reference. \\ 
\textit{Error Term Bound.} 
To prove that the error term bound holds, we estimate that 
\begin{align} 
\notag 
\left\lvert \frac{1}{2\pi\imath} \int_{|z|=r} 
     \frac{x}{(\log x)^2} \frac{(\log x)^{-\Re(z)}}{z^{k+1}} \right\rvert & \ll 
     x (\log x)^{-(r+2)} r^{-(k+1)} 
     \ll \frac{x}{(\log x)^2} \frac{(\log\log x)^{k+1}}{e^{k-1} (k-1)^{k+1}} \\ 
\notag 
     & \ll \frac{x}{(\log x)^2} \frac{(\log\log x)^{k+1}}{e^{2(k-1)} (k-1)! (k-1)} 
     \ll \frac{x}{(\log x)^{2}} \frac{(\log\log x)^{k+1}}{(k-1)!} \\ 
\label{eqn_proof_tag_ErrorTermBounds_v1} 
     & \ll \frac{x}{\log x} \frac{(\log\log x)^{k-4}}{(k-1)!}. 
\end{align} 
Now we can calculate that for $0 \leq z < 1$ 
\begin{align*} 
\prod_p \left(1 + \frac{z}{p}\right)^{-1} \left(1 - \frac{1}{p}\right)^{-z} & = 
     \exp\left(-\sum_p \left[\log\left(1 + \frac{z}{p}\right) + z 
     \log\left(1 - \frac{1}{p}\right) \right]\right) \\ 
     & \sim \exp\left(-o(z) \times \sum_p \frac{1}{p^2}\right) \\ 
     & \gg \exp\left(-o(z) \frac{\pi^2}{6}\right) \gg_z 1. 
\end{align*} 
In other words, we have that 
$\left\lvert \mathcal{G}\left(\frac{1-k}{\log\log x}\right) \right\rvert \gg 1$. 
So the error term in \eqref{eqn_proof_tag_ErrorTermBounds_v1} 
is majorized by taking $O\left(\frac{k}{(\log\log x)^3}\right)$ as our 
upper bound. \\ 
\textit{Main Term Bounds.} 
We have to process a more complicated set of integral-based bounds to 
justify that the main term in 
\eqref{eqn_proof_tag_HatPikx_BoundForGmz_v1} 
holds as stated. Notice that the main term 
estimate corresponfing to \eqref{eqn_MV_Azx_formula} and \eqref{eqn_MV7.61_CIF} 
is given by $\frac{x}{\log x} I$, where 
\[
I := \frac{(-1)^{k-1}}{2\pi\imath} \int_{|z|=r} G(-z) (\log x)^{-z} z^{-k} dz. 
\]
In particular, we can write $I = I_1 + I_2$ where we define 
\begin{align*} 
I_1 & := \frac{(-1)^{k-1} G(-r)}{2\pi\imath} \int_{|z|=r} (\log x)^{-z} z^{-k} dz \\ 
    & \phantom{:}= \frac{G(-r) (\log\log x)^{k-1}}{(k-1)!} \\ 
I_2 & := \frac{(-1)^{k-1}}{2\pi\imath} \int_{|z|=r} (G(-z) - G(-r)) (\log x)^{-z} z^{-k} dz \\ 
    & \phantom{:}= \frac{(-1)^{k-1}}{2\pi\imath} \int_{|z|=r} (G(-z) - G(-r) + G^{\prime}(-r) (z+r)) 
    (\log x)^{-z} z^{-k} dz. 
\end{align*} 
We have by a power series expansion of $G^{\prime\prime}(-w)$ about $-z$ and integrating 
the resulting series termwise with respect to $w$ that 
\[
\left\lvert G(-z) - G(-r) + G^{\prime}(-r) (z+r) \right\rvert = 
     \left\lvert \int_{-r}^{z} (z+w) G^{\prime\prime}(-w) dw \right\rvert \ll 
     G^{\prime\prime}(-r) \times |z+r|^2 \ll |z+r|^2. 
\] 
Now we parameterize the curve in the contour for $I_2$ by writing 
$z = re^{2\pi\imath t}$ for $t \in [-1/2, 1/2]$. This leads to the bounds 
\begin{align*} 
|I_2| & = r^{3-k} \times \int_{-1/2}^{1/2} |e^{2\pi\imath t} + 1|^2 \cdot 
     (\log x)^{r e^{2\pi\imath t}} \cdot e^{2\pi\imath t} dt \\ 
     & \ll r^{3-k} \times \int_{-1/2}^{1/2} \sin^2(\pi t) \cdot 
     e^{(1-k) \cos(2\pi t)} dt. 
\end{align*} 
Whenever $|x| \leq 1$, we know that $|\sin x| \leq |x|$. 
We can construct bounds on $-\cos(2\pi t)$ for 
$t \in [-1/2, 1/2]$ by writing $\cos(2x) = 1 - 2\sin^2 x$ for $|x| < 1/2$. 
We have an alternating Taylor series expansions for the sine function that shows 
\begin{align*} 
1-2\sin^2(2\pi t) & \geq 1 - 2 \left(1 - \frac{\pi t}{3}\right)^2 \geq -1 - \frac{2\pi^2 t^2}{9} 
     \qquad \implies \\ 
-\cos(2\pi t) & \leq 1 + \frac{2\pi^2 t^2}{9} \leq \left(4 + \frac{2\pi^2}{9}\right) t^2 \leq 1 + 3t^2. 
\end{align*} 
So it follows that 
\begin{align*} 
|I_2| & \ll r^{3-k} e^{k-1} \times \left\lvert \int_0^{\infty} t^2 e^{3(k-1) t^2} dt 
     \right\rvert \\ 
     & \ll \frac{r^{3-k} e^{k-1}}{(k-1)^{3/2}} = \frac{(\log\log x)^{k-3} e^{k-1}}{(k-1)^{k-3/2}} \\ 
     & \ll \frac{k \cdot (\log\log x)^{k-3}}{(k-1)!}. 
\end{align*} 
Thus the contribution from the term $|I_2|$ can then be asborbed into the error term bound 
in \eqref{eqn_proof_tag_HatPikx_BoundForGmz_v1}. 
\end{remark} 

\subsection{The distribution of exceptional values of $\Omega(n)$} 

The next theorems reproduced from \cite[\S 7.4]{MV} characterize the relative 
scarcity of the distribution of the $\Omega(n)$ for $n \leq x$ such that 
$\Omega(n) > \log\log x$. The tendency of this canonical completely additive 
function to not deviate substantially from its average order is an exceptional 
property that allows us to prove asymptotic relations on summatory functions that 
are weighted by its parity without having to account for significant local 
oscillations when we average over a large interval. 

\begin{theorem}[Upper bounds on exceptional values of $\Omega(n)$ for large $n$] 
\label{theorem_MV_Thm7.20-init_stmt} 
Let 
\begin{align*} 
A(x, r) & := \#\left\{n \leq x: \Omega(n) \leq r \cdot \log\log x\right\}, \\ 
B(x, r) & := \#\left\{n \leq x: \Omega(n) \geq r \cdot \log\log x\right\}. 
\end{align*} 
If $0 < r \leq 1$ and $x \geq 2$, then 
\[
A(x, r) \ll x (\log x)^{r-1 - r\log r}, \text{ \ as\ } x \rightarrow \infty. 
\]
If $1 \leq r \leq R < 2$ and $x \geq 2$, then 
\[
B(x, r) \ll_R x \cdot (\log x)^{r-1-r \log r}, \text{ \ as\ } x \rightarrow \infty. 
\]
\end{theorem} 

Theorem \ref{theorem_MV_Thm7.21-init_stmt} is an analog to the 
celebrated Erd\"os-Kac theorem typically stated for the 
similarly normally distributed values of the $\omega(n)$ function over $n \leq x$ as 
$x \rightarrow \infty$. 

\begin{theorem}[Exact bounds on exceptional values of $\Omega(n)$ for large $n$] 
\label{theorem_MV_Thm7.21-init_stmt} 
We have that as $x \rightarrow \infty$ 
\[
\#\left\{3 \leq n \leq x: \Omega(n) - \log\log n \leq 0\right\} = 
     \frac{x}{2} + O\left(\frac{x}{\sqrt{\log\log x}}\right). 
\]
\end{theorem} 

\begin{remark} 
The key interpretation we need to take away from the statements 
of Theorem \ref{theorem_MV_Thm7.20-init_stmt} and 
Theorem \ref{theorem_MV_Thm7.21-init_stmt} 
is the result proved as the next corollary. 
The role of the parameter $R$ involved in stating the previous theorem 
is a critical bound as the scalar factor in the upper bound on $k \leq R\log\log x$ in 
Theorem \ref{theorem_HatPi_ExtInTermsOfGz} up to which our uniform bounds given by 
Theorem \ref{theorem_GFs_SymmFuncs_SumsOfRecipOfPowsOfPrimes} hold. 
In contrast, for $n \geq 2$ we can actually 
have contributions from values distributed throughout the range $1 \leq \Omega(n) \leq \log_2(n)$ 
infinitely often. 
It is then crucial that we can show that the main term in the asymptotic formulas we obtain 
for these summatory functions is captured by summing only over the truncated range of 
$k \in [1, \log\log x]$ where the uniform bounds hold. 
\end{remark} 

\begin{cor} 
\label{theorem_MV_Thm7.20} 
Using the notation for $A(x, r)$ and $B(x, r)$ from 
Theorem \ref{theorem_MV_Thm7.20-init_stmt}, 
we have that for $\delta > 0$, 
\[
o(1) \leq \left\lvert \frac{B(x, 1+\delta)}{A(x, 1)} \right\rvert \ll 2, 
     \mathrm{\ as\ } \delta \rightarrow 0^{+}, x \rightarrow \infty. 
\]
\end{cor} 
\begin{proof} 
The lower bound stated above should be clear. To show that the asymptotic 
upper bound is correct, we compute using Theorem \ref{theorem_MV_Thm7.20-init_stmt} and 
Theorem \ref{theorem_MV_Thm7.21-init_stmt} that 
\begin{align*} 
\left\lvert \frac{B(x, 1+\delta)}{A(x, 1)} \right\rvert & \ll 
     \left\lvert \frac{x \cdot (\log x)^{\delta - \delta\log(1+\delta)}}{ 
     O(1) + \frac{x}{2} + 
     O\left(\frac{x}{\sqrt{\log\log x}}\right)} \right\rvert 
     \sim 
     \left\lvert \frac{(\log x)^{\delta - \delta\log(1+\delta)}}{ 
     \frac{1}{2} + o(1)}\right\rvert 
     \xrightarrow{\delta \rightarrow 0^{+}} 2, 
\end{align*} 
as $x \rightarrow \infty$. Notice that since $\mathbb{E}[\Omega(n)] = \log\log n + B$, with $0 < B < 1$ the 
absolute constant from Mertens theorem, 
when we denote the range of $k > \log\log x$ as holding in the form of 
$k > (1 + \delta) \log\log x$ for $\delta > 0$ at large $x$, we can assume that 
$\delta \rightarrow 0^{+}$ as $x \rightarrow \infty$. 
In particular, this holds since $k > \log\log x$ implies that 
\[
\floor{\log\log x} + 1 \geq (1 + \delta) \log\log x \quad\implies\quad 
     \delta \leq \frac{1 + \left\{\log\log x\right\}}{\log\log x} = o(1), 
     \mathrm{\ as\ } x \rightarrow \infty. 
     \qedhere 
\]
\end{proof} 

\newpage
\section{Average case analysis of bounds on the Dirichlet inverse functions, $g^{-1}(n)$} 
\label{Section_InvFunc_PreciseExpsAndAsymptotics} 

The pages of tabular data given as Table \ref{table_conjecture_Mertens_ginvSeq_approx_values} 
in the appendix section (refer to 
page \pageref{table_conjecture_Mertens_ginvSeq_approx_values}) are intended to 
provide clear insight into why we arrived at the convenient approximations to 
$g^{-1}(n)$ proved in this section. The table provides illustrative 
numerical data by examining the approximate behavior 
at hand for the cases of $1 \leq n \leq 500$ with \emph{Mathematica}. 

\subsection{Definitions and basic properties of component function sequences} 

We define the following sequence for integers $n \geq 1, k \geq 0$: 
\begin{align} 
\label{eqn_CknFuncDef_v2} 
C_k(n) := \begin{cases} 
     \varepsilon(n), & \text{ if $k = 0$; } \\ 
     \sum\limits_{d|n} \omega(d) C_{k-1}(n/d), & \text{ if $k \geq 1$. } 
     \end{cases} 
\end{align} 
By recursively expanding the definition of $C_k(n)$ 
at any fixed $n \geq 2$, we see that 
we can form a chain of at most $\Omega(n)$ iterated (or nested) divisor sums by 
unfolding the definition of \eqref{eqn_CknFuncDef_v2} inductively. 
By the same argument, we see that at fixed $n$, the function 
$C_k(n)$ is seen to only ever possibly be non-zero for 
$k \leq \Omega(n)$ whenever $n \geq 2$. 
A sequence of relevant signed semi-diagonals of the functions $C_k(n)$ begins as 
\cite[\seqnum{A008480}]{OEIS} 
\[
\{\lambda(n) \cdot C_{\Omega(n)}(n) \}_{n \geq 1} \mapsto \{
     1, -1, -1, 1, -1, 2, -1, -1, 1, 2, -1, -3, -1, 2, 2, 1, -1, -3, -1, \
     -3, 2, 2, -1, 4, 1, 2, \ldots \}. 
\]

\begin{example}[Special cases of the functions $C_k(n)$ for small $k$] 
\label{example_SpCase_Ckn} 
We cite the following special cases which are verified by 
explicit computation using \eqref{eqn_CknFuncDef_v2} 
\cite[\seqnum{A066922}]{OEIS}\footnote{ 
     For all $n,k \geq 2$, we have the following recurrence 
     relation satisfied by $C_k(n)$ between successive values of $k$: 
     \begin{equation*}
     C_k(n) = \sum_{p|n} \sum_{d\rvert\frac{n}{p^{\nu_p(n)}}} \sum_{i=0}^{\nu_p(n)-1} 
          C_{k-1}\left(dp^i\right), n \geq 1. 
     \end{equation*}
}: 
\NBRef{A07-2020-04-26} 
\begin{align*} 
C_0(n) & = \delta_{n,1} \\ 
C_1(n) & = \omega(n) \\ 
C_2(n) & = d(n) \times \sum_{p|n} \frac{\nu_p(n)}{\nu_p(n)+1} - \gcd\left(\Omega(n), \omega(n)\right). 
\end{align*} 
\end{example} 

The connection between the auxiliary functions $C_k(n)$ and the inverse sequence $g^{-1}(n)$ is 
clarified precisely in Section \ref{subSection_Relating_CknFuncs_to_gInvn}. Before we can prove explicit 
bounds on $|g^{-1}(n)|$ through its relation to these functions, we will require a perspective 
on the lower asymptotic order of $C_k(n)$ for fixed $k$ when $n$ is large. 

\subsection{Uniform asymptotics of $C_k(n)$ for large all $n$ and fixed, uniformly bounded $k$} 

The next theorem formally proves a minimal growth rate of the class of functions 
$C_k(n)$ as functions of $k,n$ for limiting cases of $n$ large and fixed $k$. 
In the statement of the result that follows, we view $k$ as a fixed variable which is 
necessarily bounded in $n$, but is still taken as an independent parameter as 
we let $n \rightarrow \infty$. 

\begin{theorem}[Asymptotics of the functions $C_k(n)$] 
\label{theorem_Ckn_GeneralAsymptoticsForms} 
For $k := 0$, we have by definition that $C_0(n) = \delta_{n,1}$. 
For all sufficiently large $n > 1$ and any fixed $1 \leq k \leq \Omega(n)$ 
taken independently of $n$, 
we obtain that the asymptotic main term for the expected order of 
$C_k(n)$ is bounded uniformly from below as 
\[
\mathbb{E}[C_k(n)] \gg (\log\log n)^{2k-1}, \mathrm{\ as\ }n \rightarrow \infty. 
\]
\end{theorem} 
\NBRef{A08-2020-04-26} 
\begin{proof} 
\label{proofOf_theorem_Ckn_GeneralAsymptoticsForms} 
We prove our bounds by induction on $k$. 
We can see by Example \ref{example_SpCase_Ckn} that $C_1(n)$ 
satsfies the formula we must establish when $k := 1$ since $\mathbb{E}[\omega(n)] = \log\log n$. 
Suppose that $k \geq 2$ and let our inductive assumption provide that for all $1 \leq m < k$ 
\[
\mathbb{E}[C_m(n)] \gg (\log\log n)^{2m-1}. 
\] 
For all large $x \gg 2$, we cite that the summatory function of 
$\omega(n)$ satisfies \cite[\S 22.10]{HARDYWRIGHT} 
\[
\sum_{n \leq x} \omega(n) = x \log\log x + Bx + O\left(\frac{x}{\log x}\right). 
\]
Now using the recursive formula we used to define the sequences of $C_k(n)$ in 
\eqref{eqn_CknFuncDef_v2}, we have that as $n \rightarrow \infty$ 
\begin{align} 
\notag
\mathbb{E}[C_k(n)] & = \mathbb{E}\left[\sum_{d|n} \omega(n/d) C_{k-1}(d)\right] \\ 
\notag 
     & = \frac{1}{n} \times \sum_{d \leq n} C_{k-1}(d) \times \sum_{r=1}^{\Floor{n}{d}} \omega(r) \\ 
\notag 
     & \sim \sum_{d \leq n} C_{k-1}(d) \left[ 
     \frac{\log\log(n/d) \Iverson{d \leq \frac{n}{e}}}{d} + \frac{B}{d} + o(1)\right] \\ 
\label{eqn_proof_tag_ECkn_sum_steps_v3} 
     & \sim \sum_{d \leq \frac{n}{e}} \left[ 
     \sum_{m < d} \frac{\mathbb{E}[C_{k-1}(m)]}{m} \log\log\left(\frac{n}{m}\right) + 
     B \cdot \mathbb{E}[C_{k-1}(d)] + B \cdot \sum_{m < d} \frac{\mathbb{E}[C_{k-1}(m)]}{m} 
     \right] \\ 
\notag 
     & \gg B \times \sum_{d \leq \frac{n}{e}} \frac{\mathbb{E}[C_{k-1}(m)]}{m} \\ 
\notag 
     & \gg B \cdot (\log n) (\log\log n)^{2k-3}. 
\end{align} 
In transitioning from the previous step, we have used that 
$B \cdot (\log n) \gg (\log\log n)^2$ as $n \rightarrow \infty$. We have also used that for large 
$n$ and fixed $m$, by an asymptotic approximation to the incomplete gamma function 
we have that 
\[
\int_{e}^{n} \frac{(\log\log t)^m}{t} \sim (\log n) (\log\log n)^{m}, 
     \mathrm{\ as\ } n \rightarrow \infty. 
\]
Hence, the claim follows by mathematical induction for large $n \rightarrow \infty$ whenever 
$1 \leq k \leq \Omega(n)$. 
\end{proof} 

%\begin{remark}[A more accurate main term]
%In step \eqref{eqn_proof_tag_ECkn_sum_steps_v3} of the above proof, we can see that the main term 
%for the expectation of $C_k(n)$ actually corresponds to the double sum 
%\[
%M_k(n) := \sum_{d \leq \frac{n}{e}} 
%     \sum_{m < d} \frac{\mathbb{E}[C_{k-1}(m)]}{m} \log\log\left(\frac{n}{m}\right). 
%\]
%Since
%\[
%\log\log\left(\frac{n}{m}\right) = \log\log n - \frac{\log m}{\log n} + 
%     O\left(\frac{\log^2 m}{\log^2 n}\right), 
%\]
%we can approximate $M_k(n)$ as follows: 
%\begin{align*} 
%M_k(n) & \sim \int_e^n \left(
%     \int_e^t \frac{(\log s)^{2k-4} (\log\log s)^{k-1} (\log\log n)}{s \cdot (2k-4)!} ds 
%     \right) dt \\ 
%     & \gg \int_e^n \frac{(\log t)^{2k-3} (\log\log t)^{k-1} (\log\log n)}{t \cdot (2k-3)!} dt \\ 
%     & \sim \frac{(\log n)^{2k-2} (\log\log n)^k}{(2k-2)!}. 
%\end{align*} 
%This estimate is much closer to the asymptotically dominant behavior of 
%$\mathbb{E}[C_k(n)]$ as $n \rightarrow \infty$. 
%\end{remark}

\subsection{Relating the auxiliary functions $C_k(n)$ to formulas approximating $g^{-1}(n)$} 
\label{subSection_Relating_CknFuncs_to_gInvn} 

\begin{lemma}[An exact formula for $g^{-1}(n)$] 
\label{lemma_AnExactFormulaFor_gInvByMobiusInv_v1} 
For all $n \geq 1$, we have that 
\[
g^{-1}(n) = \sum_{d|n} \mu\left(\frac{n}{d}\right) \lambda(d) C_{\Omega(d)}(d). 
\]
\end{lemma}
\begin{proof} 
We first write out the standard recurrence relation for the Dirichlet inverse of 
$\omega+1$ as 
\begin{align} 
\label{eqn_proof_tag_gInvCvlOne_EQ_omegaCvlgInvCvl_v1} 
g^{-1}(n) & = - \sum_{\substack{d|n \\ d>1}} (\omega(d) + 1) g^{-1}(n/d) 
     \quad\implies\quad 
     (g^{-1} \ast 1)(n) = -(\omega \ast g^{-1})(n). 
\end{align} 
We argue that for $1 \leq m \leq \Omega(n)$, we can inductively expand the 
implication on the right-hand-side of \eqref{eqn_proof_tag_gInvCvlOne_EQ_omegaCvlgInvCvl_v1} 
in the form of $(g^{-1} \ast 1)(n) = F_m(n)$ where 
$F_m(n) := (-1)^{m} \cdot (C_m(-) \ast g^{-1})(n)$, or so that 
\[
F_m(n) = - 
     \begin{cases} 
     \sum\limits_{\substack{d|n \\ d > 1}} F_{m-1}(d) \times \sum\limits_{\substack{r|\frac{n}{d} \\ r > 1}} 
     \omega(r) g^{-1}\left(\frac{n}{dr}\right), & m \geq 2, \\ 
     (\omega \ast g^{-1})(n), & m = 1. 
     \end{cases} 
\]
Now by repeatedly expanding the right-hand-side of the previous equation, 
we find inductively that for $m := \Omega(n)$ 
\begin{equation} 
\label{eqn_proof_tag_gInvCvlOne_EQ_omegaCvlgInvCvl_v2} 
(g^{-1} \ast 1)(n) = (-1)^{\Omega(n)} C_{\Omega(n)}(n) = \lambda(n) C_{\Omega(n)}(n). 
\end{equation} 
The statement then follows from \eqref{eqn_proof_tag_gInvCvlOne_EQ_omegaCvlgInvCvl_v2} 
by M\"obius inversion applied to each side of the last equation. 
\end{proof} 

\begin{cor} 
\label{cor_AnExactFormulaFor_gInvByMobiusInv_nSqFree_v2} 
For all squarefree integers $n \geq 1$, we have that 
\begin{equation} 
\label{eqn_gInvnSqFreeN_exactDivSum_Formula} 
g^{-1}(n) = \lambda(n) \times \sum_{d|n} C_{\Omega(d)}(d). 
\end{equation} 
\end{cor} 
\begin{proof} 
Since $g^{-1}(1) = 1$, clearly the claim is true for $n = 1$. Suppose that $n \geq 2$ and that 
$n$ is squarefree. Then $n = p_1p_2 \cdots p_{\omega(n)}$ where $p_i$ is prime for all 
$1 \leq i \leq \omega(n)$. So since all divisors of any squarefree $n$ are necessarily also squarefree, 
we can transform the exact divisor sum guaranteed for all $n$ in 
Lemma \ref{lemma_AnExactFormulaFor_gInvByMobiusInv_v1} into a sum that partitions the divisors by 
their number of distinct prime factors: 
\begin{align*} 
g^{-1}(n) & = \sum_{i=0}^{\omega(n)} \sum_{\substack{d|n \\ \omega(d)=i}} (-1)^{\omega(n) - i} (-1)^{i} \cdot 
     C_{\Omega(d)}(d) \\ 
     & = \lambda(n) \times \sum_{i=0}^{\omega(n)} \sum_{\substack{d|n \\ \omega(d)=i}} C_{\Omega(d)}(d) \\ 
     & = \lambda(n) \times \sum_{d|n} C_{\Omega(d)}(d). 
\end{align*} 
The signed contributions in the first of the previous equations is 
justified by noting that $\lambda(n) = (-1)^{\omega(n)}$ 
whenever $n$ is squarefree, and that for $d \geq 1$
 squarefree we have the correspondence 
 $\omega(d) = k$ $\implies$ $\Omega(d) = k$ for $1 \leq k \leq \log_2(d)$. 
\end{proof} 

Since $C_{\Omega(n)}(n) = |h^{-1}(n)|$ using the notation defined in the the proof of 
Proposition \ref{prop_SignageDirInvsOfPosBddArithmeticFuncs_v1}, we can see that 
$C_{\Omega(n)}(n) = (\omega(n))!$ for squarefree $n \geq 1$. 
A proof of part (C) of Conjecture \ref{lemma_gInv_MxExample} then 
follows as an immediate consequence. 

\begin{lemma} 
\label{lemma_AbsValueOf_gInvn_FornSquareFree_v1} 
For all positive integers $n \geq 1$, we have that 
\begin{equation} 
\label{eqn_AbsValueOf_gInvn_FornSquareFree_v1} 
|g^{-1}(n)| = \sum_{d|n} \mu^2\left(\frac{n}{d}\right) C_{\Omega(d)}(d). 
\end{equation} 
\end{lemma} 
\begin{proof} 
By applying 
Lemma \ref{lemma_AnExactFormulaFor_gInvByMobiusInv_v1}, 
Proposition \ref{prop_SignageDirInvsOfPosBddArithmeticFuncs_v1} and the 
complete multiplicativity of $\lambda(n)$, 
we easily obtain the stated result. 
In particular, since $\mu(n)$ is non-zero only at squarefree integers and 
at any squarefree $d \geq 1$ we have $\mu(d) = (-1)^{\omega(d)} = \lambda(d)$, 
Lemma \ref{lemma_AnExactFormulaFor_gInvByMobiusInv_v1} implies 
\begin{align*} 
|g^{-1}(n)| & = \lambda(n) \times \sum_{d|n} \mu\left(\frac{n}{d}\right) \lambda(d) C_{\Omega(d)}(d) \\ 
     & = \sum_{d|n} \mu^2\left(\frac{n}{d}\right) \lambda\left(\frac{n}{d}\right) 
     \lambda(nd) C_{\Omega(d)}(d) \\ 
     & = \lambda(n^2) \times \sum_{d|n} \mu^2\left(\frac{n}{d}\right) C_{\Omega(d)}(d). 
\end{align*} 
In the last equation, we see that 
that $\lambda(n^2) = +1$ for all $n \geq 1$ since the number of distinct 
prime factors (counting multiplicity) of any square integer is even. 
\end{proof} 

Combined with the signedness property of $g^{-1}(n)$ guaranteed by 
Proposition \ref{prop_SignageDirInvsOfPosBddArithmeticFuncs_v1}, 
Lemma \ref{lemma_AbsValueOf_gInvn_FornSquareFree_v1} shows that the summatory 
function is expressed as 
\[
G^{-1}(x) = \sum_{d \leq x} \lambda(d) C_{\Omega(d)}(d) M\left(\Floor{x}{d}\right). 
\]
Since $\lambda(d) C_{\Omega(d)}(d) = (g^{-1} \ast 1)^{-1}(d) = (\chi_{\mathbb{P}} + \varepsilon)(d)$ 
where $\chi_{\mathbb{P}}$ denotes the characteristic function of the primes, we clearly 
obtain by inversion that 
\[
M(x) = G^{-1}(x) + \sum_{p \leq x} G^{-1}\left(\Floor{x}{p}\right), x \geq 1. 
\]

\begin{cor} 
\label{lemma_BddExpectationOfgInvn} 
We have that 
\[
\frac{6}{\pi^2} (\log n) (\log\log n) \ll 
     \mathbb{E}|g^{-1}(n)| \leq 
     \mathbb{E}\left[\sum_{d|n} C_{\Omega(d)}(d)\right]. 
\]
\end{cor} 
\begin{proof} 
To prove the lower bound, 
recall from the introduction that the summatory function of the 
squarefree integers is given by 
\[
Q(x) := \sum_{n \leq x} \mu^2(n) = \frac{6x}{\pi^2} + O(\sqrt{x}). 
\]
Then since $C_{\Omega(d)}(d) \geq 1$ for all $d \geq 1$, and since 
$\mathbb{E}[C_k(d)]$ is minimized when $k := 1$ according to 
Theorem \ref{theorem_Ckn_GeneralAsymptoticsForms}, 
we obtain by summing over 
\eqref{eqn_AbsValueOf_gInvn_FornSquareFree_v1} that 
\begin{align*} 
\frac{1}{x} \times \sum_{n \leq x} |g^{-1}(n)| & = \frac{1}{x} \times \sum_{d \leq x} 
     C_{\Omega(d)}(d) Q\left(\Floor{x}{d}\right) \\ 
     & \sim \sum_{d \leq x} C_{\Omega(d)}(d) \left[\frac{6}{d \cdot \pi^2} + O\left(\frac{1}{\sqrt{dx}}\right) 
     \right] \\ 
     & = \frac{6}{\pi^2} \left[\mathbb{E}[C_{\Omega(x)}(x)] + \sum_{d<x} 
     \frac{\mathbb{E}[C_{\Omega(d)}(d)]}{d}\right] + 
     O\left(\frac{1}{\sqrt{x}} \times \int_0^{x} t^{-1/2} dt\right) \\ 
     & \gg \frac{6}{\pi^2} \left[\sum_{e \leq d \leq x} 
     \frac{\log\log d}{d}\right] + O(1) \\ 
     & \sim \frac{6}{\pi^2} \times \int_{e}^{x} \frac{\log\log t}{t} dt + O(1) \\ 
     & \gg \frac{6}{\pi^2} (\log x) (\log\log x), \mathrm{\ as\ } x \rightarrow \infty. 
\end{align*} 
To prove the upper bound, notice that by 
Lemma \ref{lemma_AnExactFormulaFor_gInvByMobiusInv_v1} and 
Corollary \ref{cor_AnExactFormulaFor_gInvByMobiusInv_nSqFree_v2}, 
\[
|g^{-1}(n)| \leq \sum_{d|n} C_{\Omega(d)}(d), n \geq 1. 
\]
Now since both of the above quantities are positive for all $n \geq 1$, 
we clearly obtain the upper bound stated above when we average over $n \leq x$ 
for all large $x$. 
\end{proof} 

\subsubsection{A connection to the distribution of the primes} 

\begin{remark} 
The combinatorial complexity of $g^{-1}(n)$ is deeply tied to the distribution of the primes 
$p \leq n$ as $n \rightarrow \infty$. 
While the magnitudes and dispersion of the primes $p \leq x$ certainly restricts the 
repeating of these distinct sequence 
values we can see in the contributions to $G^{-1}(x)$, the following 
statement is still clear about the relation of the weights $|g^{-1}(n)|$ to the prime numbers: 
The value of $|g^{-1}(n)|$ is entirely dependent on the pattern of the \emph{exponents} 
(viewed as multisets) of the distinct prime factors of $n \geq 2$. 
The relation of the repitition of the distinct values 
of $|g^{-1}(n)|$ as weights to the signed summatory function $G^{-1}(x)$ makes a clear tie to 
$M(x)$ through Proposition \ref{prop_Mx_SBP_IntegralFormula} in the next section. 
Moreover, the complexity of the distribution of $G^{-1}(x)$ for large $x$ clearly has an intimate 
connection to the distribution of $\omega(n)$, $\Omega(n)$, and $\Omega(n) - \omega(n)$ and the role 
these functions play in describing the distinct exponent multisets of the prime factorization of 
$n \leq x$. 
\end{remark}

\begin{example}[Combinatorial significance to the distribution of $g^{-1}(n)$] 
We have a natural extremal behavior with respect to distinct values of $\Omega(n)$ 
corresponding to squarefree integers, and prime powers. Namely, if for $k \geq 1$ we define the 
infinite sets $M_k$ and $m_k$ to correspond to the maximal (minimal) positive integers such that 
\begin{align*} 
M_k & := \left\{n \geq 2: |g^{-1}(n)| = \underset{{\substack{j \geq 2 \\ \Omega(j) = k}}}{\operatorname{sup}} 
     |g^{-1}(j)|\right\}, \\  
m_k & := \left\{n \geq 2: |g^{-1}(n)| = \underset{{\substack{j \geq 2 \\ \Omega(j) = k}}}{\operatorname{inf}} 
     |g^{-1}(j)|\right\}, 
\end{align*} 
then any element of $M_k$ is squarefree and any element of $m_k$ is a prime power. 
In particular, we have that for any $N_k \in M_k$ and $n_k \in m_k$
\[
N_k = \sum_{j=0}^{k} \binom{k}{j} \cdot j!, \quad \mathrm{\ and\ } \quad n_k = 2 \cdot (-1)^{k}. 
\]
The formula for the function $h^{-1}(n) = (g^{-1} \ast 1)(n)$ defined in the proof of 
Proposition \ref{prop_SignageDirInvsOfPosBddArithmeticFuncs_v1} implies that we can express 
an exact formula for $g^{-1}(n)$ in terms of symmetric polynomials in the 
exponents of the prime factorization of $n$. 
Namely, for $n \geq 2$ let 
\[
\widehat{e}_k(n) := [z^k] \prod_{p|n} (1 + z \cdot \nu_p(n)) = [z^k] \prod_{p^{\alpha} || n} (1 + \alpha z), 
     0 \leq k \leq \omega(n). 
\]
Then we have essentially shown using 
\eqref{eqn_proof_tag_hInvn_ExactNestedSumFormula_CombInterpetIdent_v3} and 
\eqref{eqn_AbsValueOf_gInvn_FornSquareFree_v1} that we can expand 
\[
g^{-1}(n) = h^{-1}(n) \times \sum_{k=0}^{\omega(n)} \binom{\Omega(n)}{k}^{-1} 
     \frac{\widehat{e}_k(n)}{k!}, n \geq 2. 
\]
The combinatorial formula for 
$h^{-1}(n) = \lambda(n) \cdot (\Omega(n))! \times \prod_{p^{\alpha} || n} (\alpha !)^{-1}$ 
we derived in the proof of the key signedness proposition in 
Section \ref{Section_PrelimProofs_Config} 
suggests further patterns and more regularity in the contributions of the distinct weighted 
terms for $G^{-1}(x)$ when we sum over all of the distinct prime exponent patterns that factorize 
$n \leq x$. 
\end{example} 

\newpage
\section{Lower bounds for $M(x)$ along infinite subsequences} 
\label{Section_KeyApplications} 

\begin{prop} 
\label{prop_Mx_SBP_IntegralFormula} 
For all sufficiently large $x$, we have that 
\begin{align} 
\label{eqn_pf_tag_v2-restated_v2} 
M(x) & \approx G^{-1}(x) - x \cdot \int_1^{x/2} \frac{G^{-1}(t)}{t^2 \cdot \log(x/t)} dt. 
\end{align} 
\end{prop} 
\begin{proof} 
We know by applying Corollary \ref{cor_Mx_gInvnPixk_formula} that 
\begin{align} 
\notag
M(x) & = \sum_{k=1}^{x} g^{-1}(k) \left(\pi\left(\Floor{x}{k}\right)+1\right) \\ 
\label{eqn_proof_tag_MxFormulaInitSepTerms_v1} 
     & \approx G^{-1}(x) + \sum_{k=1}^{x} g^{-1}(k) \pi\left(\Floor{x}{k}\right), 
\end{align} 
We can replace the floored integer-valued arguments to $\pi(x)$ 
in \eqref{eqn_proof_tag_MxFormulaInitSepTerms_v1} using 
its approximation by the monotone non-decreasing asymptotic order of $\pi(x) \sim \frac{x}{\log x}$. 
We can always 
bound $$\frac{Ax}{\log x} \leq \pi(x) \leq \frac{Bx}{\log x},$$ for suitably defined 
absolute constants, $A,B > 0$ whenever $x \geq 2$. 
Therefore the approximation obtained by replacing $\pi(x)$ by the main term in its 
limting asymptotic formula is actually valid for all $x > 1$ up to at most 
a small constant difference. 

What we require to sum and simplify the right-hand-side terms from 
\eqref{eqn_proof_tag_MxFormulaInitSepTerms_v1} follows from the exact 
summation by parts formula. 
In particular, we argue that for sufficiently large 
$x \geq 3$ we can approximate\footnote{
     Since $\pi(1) = 0$, the actual range of summation corresponds to 
     $k \in \left[1, \frac{x}{2}\right]$. 
}
\begin{subequations}
\begin{align} 
\notag 
\sum_{k=1}^{x} g^{-1}(k) \pi(x/k) & = G^{-1}(x) \pi(1) - 
     \sum_{k=1}^{\min\left(\frac{x}{2}, x-1\right)} G^{-1}(k) \left[ 
     \pi\left(\frac{x}{k}\right) - \pi\left(\frac{x}{k+1}\right)\right] \\ 
\notag
     & = -\sum_{k=1}^{x/2} G^{-1}(k) \left[ 
     \pi\left(\frac{x}{k}\right) - \pi\left(\frac{x}{k+1}\right)\right] \\ 
\label{eqn_proof_tag_MxSumApprox_Step_v3} 
     & \approx -\sum_{k=1}^{x/2} G^{-1}(k) \left[ 
     \frac{x}{k \cdot \log(x/k)} - \frac{x}{(k+1) \cdot \log(x/k)}\right] \\ 
\label{eqn_proof_tag_MxSumApprox_Step_v4} 
     & \approx -\sum_{k=1}^{x/2} G^{-1}(k) \frac{x}{k^2 \cdot \log(x/k)}, 
     \mathrm{\ as\ } x \rightarrow \infty. 
\end{align} 
\end{subequations} 
Indeed, we can justify that step \eqref{eqn_proof_tag_MxSumApprox_Step_v3} is 
correct by writing 
\begin{align*} 
\frac{x}{(k+1) \log\left(\frac{x}{k+1}\right)} & = \frac{x}{k+1} \cdot 
     \frac{1}{\left[\log\left(\frac{x}{k}\right) + \log\left(1 - \frac{1}{k+1}\right)\right]} 
     = \frac{x}{(k+1) \log\left(\frac{x}{k}\right)} \cdot 
     \frac{1}{1 + \frac{\log\left(1 - \frac{1}{k+1}\right)}{\log x \left[ 
     1 - \frac{\log k}{\log x}\right]}} \\ 
     & \sim \frac{x}{(k+1) \log\left(\frac{x}{k}\right)}, \mathrm{\ as\ } x \rightarrow \infty. 
\end{align*} 
The correctness of the transition from 
step \eqref{eqn_proof_tag_MxSumApprox_Step_v3} to \eqref{eqn_proof_tag_MxSumApprox_Step_v4} is 
verified by seeing that for $\Re(s) > 1$, we have that 
\[
\left\lvert \sum_{k \geq 1} \frac{G^{-1}(k)}{k^{s+1}} \right\rvert = 
      \left\lvert \int_1^{\infty} \frac{G^{-1}(x)}{x^{s+1}} dx \right\rvert = 
      \left\lvert \frac{1}{s \cdot (P(s) + 1) \zeta(s)} \right\rvert < 
      \infty. 
\]
When $s := \frac{3}{2}$, we obtain that 
\[
0 \leq \left\lvert \sum_{k \geq 1} \frac{G^{-1}(k)}{k^{2} (k+1)} \right\rvert \leq 
     \left\lvert \sum_{k \geq 1} \frac{G^{-1}(k)}{k^{\frac{5}{2}}} \right\rvert < \infty. 
\]
Then the difference of the terms in forming the approximation in this step 
is bounded above and below by absolute constants as 
\[
\left\lvert \sum_{k=1}^{\frac{x}{2}} G^{-1}(k) \left[\frac{1}{k^2} - \frac{1}{k(k+1)}\right] \right\rvert \leq 
     \left\lvert\sum_{k=1}^{\frac{x}{2}} \frac{G^{-1}(k)}{k^2 (k+1)} \right\rvert = O(1). 
\]
For $x$ large enough the summand factor $\frac{x}{k^2 \cdot \log(x/k)}$ 
is monotonic as $k$ ranges over $k \in [1, x/2]$ in ascending order. Because this 
summand factor is a smooth function of $k$ (and $x$) where $G^{-1}(x)$ is 
a summatory function with jumps only in steps of the positive integers, we can finally approximate 
$M(x)$ for any finite $x \geq 2$ as follows: 
\[
M(x) \approx G^{-1}(x) - x \cdot \int_1^{x/2} \frac{G^{-1}(t)}{t^2 \cdot \log(x/t)} dt. 
\]
We will later only use unsigned lower bound approximations to this function in the next theorems so that 
the signedness of the summatory function term in the integral formula above 
does not require more restrictive attention in limiting cases as $x \rightarrow \infty$. 
\end{proof} 

\subsection{Establishing initial lower bounds on the summatory functions $G^{-1}(x)$} 
\label{Section_ProofOfValidityOfAverageOrderLowerBounds} 

Let the summatory function $G_E^{-1}(x)$ be defined for $x \geq 1$ by 
\begin{equation} 
\label{eqn_GEInvxSummatoryFuncDef_v1} 
G_E^{-1}(x) := \sum_{n \leq (\log x)^{5} (\log\log x)} \lambda(n) \times 
     \sum_{\substack{d|n \\ d > e}} \frac{(\log d)^{\frac{1}{4}}}{\log\log d}. 
\end{equation} 
The subscript of $E$ is a formality of 
notation that does not correspond to an actual parameter or any 
implicit dependence on $E$ in the function definition given above. 

\begin{theorem} 
\label{theorem_GInvxLowerBoundByGEInvx_v1} 
For almost all sufficiently large integers $x \rightarrow \infty$, we have that 
\[
|G^{-1}(x)| \gg |G_E^{-1}(x)|. 
\]
\end{theorem} 
\begin{proof} 
First, consider the following upper bound on $|G_E^{-1}(x)|$: 
\begin{align} 
\notag 
|G_E^{-1}(x)| & = \left\lvert \sum_{\substack{e \leq n \leq (\log x)^{5} (\log\log x)}} \lambda(n) \times 
     \sum_{\substack{d|n \\ d > e}} \frac{(\log d)^{\frac{1}{4}}}{\log\log d} 
     \right\rvert \\ 
\notag 
     & \ll \sum_{e < d \leq (\log x)^{5} (\log\log x)} \frac{(\log d)^{\frac{1}{4}}}{\log\log d} \cdot 
     \Floor{(\log x)^{5} (\log\log x)^{16}}{d} \\ 
\notag 
     & \ll (\log x)^{5} (\log\log x) \times 
     \int_{e}^{(\log x)^{5} (\log\log x)} \frac{(\log t)^{\frac{1}{4}}}{t \cdot \log\log t} dt \\ 
\notag 
     & = (\log x)^{5} (\log\log x) \times 
     \operatorname{Ei}\left(\frac{5}{4} \log\log\left((\log x)^{5} (\log\log x)\right)\right) \\ 
\label{eqn_proof_tag_AbsGEInvx_AsymptoticUpperBound_v1} 
     & \ll \frac{25}{64} \cdot (\log x)^{5} (\log\log x) (\log\log\log x)^2. 
\end{align} 
Next, we compute that for almost every sufficiently large $x \rightarrow \infty$: 
\begin{align*} 
\frac{|G^{-1}(x)|}{x} & = \frac{1}{x} \times \left\lvert 
     \sum_{\substack{d \leq x \\ \lambda(d)=+1}} |g^{-1}(d)| - 
     \sum_{\substack{d \leq x \\ \lambda(d)=-1}} |g^{-1}(d)| 
     \right\rvert 
     \gg \left\lvert 
     \mathbb{E}|g^{-1}(x)| - \frac{2}{x} \times \sum_{\substack{d \leq x \\ \lambda(d)=-1}} |g^{-1}(d)| 
     \right\rvert. 
\end{align*} 
Let the indeterminate summation in the previous equation be defined by 
\[
S_{-}(x) := \sum_{\substack{d \leq x \\ \lambda(d)=-1}} |g^{-1}(d)|. 
\]
We will find upper and lower bounds on this sum that show $\mathbb{E}|g^{-1}(x)| \gg \frac{S_{-}(x)}{x}$. 

For the positive summands of $S_{-}(x)$ 
to be at their largest, we require that for $d \geq 2$ 
\[
|g^{-1}(d)| = \sum_{j=0}^{\omega(d)} \binom{\omega(d)}{j} \cdot j!. 
\]
Then we have that 
\begin{equation} 
\label{eqn_proof_tag_SMinusx_LL_Form_v1} 
S_{-}(x) \ll \sum_{1 \leq k \leq \log_2(x)} \widehat{\pi}_k(x) \times \sum_{j=0}^{k} \binom{k}{j} j!. 
\end{equation} 
We can bound the summatory function terms by 
\begin{align*} 
\widehat{\pi}_k(x) \leq \frac{\widehat{\pi}_k(x) \cdot \pi_k(x)}{\#\left\{n \leq x: 
     \Omega(n) = \omega(n) \land \Omega(n) = k\right\}}. 
\end{align*} 
By an argument with conditional probabilities of sets, we then obtain 
\begin{align} 
\notag 
\#\left\{n \leq x: \Omega(n) = \omega(n) \land \Omega(n) = k\right\} & \geq \frac{1}{x} \cdot 
     \#\left\{n \leq x: n\mathrm{\ squarefree } \land \mu(n) = (-1)^{k}\right\} \times 
     \widehat{\pi}_k(x) \\ 
\notag 
     & = \frac{3}{\pi^2} \widehat{\pi}_k(x), \mathrm{\ as\ } x \rightarrow \infty. 
\end{align} 
So from \eqref{eqn_proof_tag_SMinusx_LL_Form_v1}, we obtain that 
\begin{equation} 
\label{eqn_proof_tag_SMinusx_LL_Form_v2} 
S_{-}(x) \ll \sum_{1 \leq k \leq \log_2(x)} \frac{\pi^2}{3} \pi_k(x) \times 
     \sum_{j=0}^{k} \binom{k}{j} j!. 
\end{equation} 
We weight by the known asymptotic formula for the summatory functions 
$\pi_k(x) \sim \frac{x}{\log x} \frac{(\log\log x)^{k-1}}{(k-1)!}(1+o(1))$ as $x \rightarrow \infty$ 
to find that 
\begin{align*} 
S_{-}(x) & \ll \frac{\pi^2}{3} \times \sum_{1 \leq k \leq \log_2(x)} \pi_k(x) \times \sum_{j=0}^{k} \binom{k}{j} j! \\ 
     & \ll \frac{\pi^2}{3} \times 
     \frac{x}{(\log x) (\log\log x)} \times \sum_{k \geq 1} k \cdot (\log\log x)^{k} \sum_{j=0}^{k} \frac{1}{j!} \\ 
     & \ll \frac{\pi^2}{3} \times \frac{ex}{(\log x) (\log\log x)} \times \sum_{k \geq 1} k \cdot (\log\log x)^{k} \\ 
     & \ll \frac{\pi^2}{3} \times \frac{ex}{(\log x) (\log\log x)^2}. 
\end{align*} 
Thus, over these choices bounding the $g^{-1}(d)$, we obtain that 
$\frac{S_{-}(x)}{x} = o(1)$ as $x \rightarrow \infty$. 

On the other hand, we can choose the summands to satisfy $|g^{-1}(d)| \geq 2$. 
We define the following densities for large $x \geq 2$ \cite[\cf \S 1]{TAO-VALUEPATTERNS}: 
\begin{align*} 
\mathcal{L}_{+}(x) & := \frac{1}{n} \cdot \#\{n \leq x: \lambda(n) = +1\} 
     \overset{\mathbb{E}}{\sim} \frac{1}{2} \\ 
\mathcal{L}_{-}(x) & := \frac{1}{n} \cdot \#\{n \leq x: \lambda(n) = -1\} 
     \overset{\mathbb{E}}{\sim} \frac{1}{2}. 
\end{align*} 
Now we see that 
\[
S_{-}(x) \gg 2x \cdot \min\left(\mathcal{L}_{-}(x), 1 - \mathcal{L}_{-}(x)\right). 
\]
This implies that $\frac{S_{-}(x)}{x} = O(1)$. In either of these extreme cases, we have by 
Corollary \ref{lemma_BddExpectationOfgInvn} that 
\[
\frac{|G^{-1}(x)|}{x} \gg \frac{6}{\pi^2} (\log x) (\log\log x). 
\]
Then naturally from \eqref{eqn_proof_tag_AbsGEInvx_AsymptoticUpperBound_v1} we 
have proved that as $x \rightarrow \infty$, 
$|G^{-1}(x)| \gg |G_E^{-1}(x)|$. 
\end{proof} 

\begin{remark} 
Note that the only cases of $x \geq 1$ we need to be wary of in the \emph{almost everywhere} clause 
to applying the statement of 
Theorem \ref{theorem_GInvxLowerBoundByGEInvx_v1} 
happen when $G^{-1}(x) = 0$. In these cases, the bounds we proved above cannot be conclusively shown to 
hold. This singularity in the distribution of $G^{-1}(x)$ can only occur when 
\[
G^{-1}(x) = \frac{g^{-1}(x)}{2} + 
     \frac{1}{2\pi\imath} \int_{c-\imath\infty}^{c+\imath\infty} 
     \frac{x^s}{s \cdot (P(s) + 1) \zeta(s)} ds = 0, \mathrm{\ for\ } c > 1. 
\] 
It suffices to assume that $G^{-1}(x) \neq 0$ on a dense subset of the integers for the 
bounds we need to prove 
Corllary \ref{cor_ThePipeDreamResult_v1} 
in the last subsection. In particular, we require that 
\[
\lim_{x \rightarrow \infty} \frac{1}{x} \cdot \#\{n \leq x: G^{-1}(n) \neq 0\} \geq \frac{1}{2}. 
\]
\end{remark} 

\begin{cor} 
\label{cor_ASemiForm_ForGInvx_v1} 
We have that for almost every sufficiently large $x$, that as $x \rightarrow \infty$ 
\begin{align*} 
\left\lvert G_E^{-1}(x) \right\rvert & \SuccSim 
     \frac{\widehat{C}_0}{2\sqrt{2\pi}} \times 
     \frac{(\log x)^{\frac{5}{4}}}{(\log\log x)^{\frac{1}{4}} \sqrt{\log\log\log x}} \times 
     \left\lvert \sum_{e < d \leq \log x} 
     \frac{\lambda(d) (\log d)^{\frac{1}{4}}}{d^{1/4} \cdot \log\log d} 
     \right\rvert. 
\end{align*} 
\end{cor} 
\NBRef{A10-2020.04-26} 
\begin{proof} 
Using the definition in \eqref{eqn_GEInvxSummatoryFuncDef_v1}, we obtain on average that\footnote{ 
     For any arithmetic functions $f,h$, we have that \cite[\cf \S 3.10; \S 3.12]{APOSTOLANUMT} 
     \[
     \sum_{n \leq x} h(n) \times \sum_{d|n} f(d) = \sum_{d \leq x} f(d) \times \sum_{n=1}^{\Floor{x}{d}} h(dn). 
     \] 
}
\begin{align*} 
\left\lvert G_E^{-1}(x) \right\rvert & = 
     \left\lvert \sum_{n \leq (\log x)^{5} (\log\log x)} \lambda(n) \times 
     \sum_{\substack{d|n \\ d > e}} \frac{\lambda(d) (\log d)^{\frac{1}{4}}}{\log\log d} \right\rvert \\ 
     & = \left\lvert \sum_{e < d \leq (\log x)^{5} (\log\log x)} 
     \frac{(\log d)^{\frac{1}{4}}}{\log\log d} \times 
     \sum_{n=1}^{\Floor{\log x}{d}} \lambda(dn) \right\rvert. 
\end{align*} 
We see that by complete additivity of $\Omega(n)$ 
(complete multiplicativity of $\lambda(n)$) that 
\begin{align*} 
\sum_{n=1}^{\Floor{x}{d}} \lambda(dn) & = \sum_{n=1}^{\Floor{x}{d}} \lambda(d) \times \lambda(n) 
     = \lambda(d) \times \sum_{n \leq \Floor{x}{d}} \lambda(n). 
\end{align*} 
From Theorem \ref{theorem_GFs_SymmFuncs_SumsOfRecipOfPowsOfPrimes} and 
Lemma \ref{lemma_lowerBoundsOnLambdaFuncParitySummFuncs}, 
we can establish that 
\begin{align} 
\label{eqn_proof_tag_GEInvxLowerBound_v1} 
\left\lvert \sum_{k \leq \log\log x} (-1)^k \cdot \widehat{\pi}_k(x) \right\rvert 
     & \gg \frac{\widehat{C}_0}{\sqrt{2\pi}} \cdot 
     \frac{x^{\frac{1}{4}}}{(\log x)^{\frac{1}{2}} \sqrt{\log\log x}} 
     =: \widehat{L}_0(x), \mathrm{\ as\ } x \rightarrow \infty. 
\end{align} 
The sign of the sum obtained by taking the right-hand-side of 
\eqref{eqn_proof_tag_GEInvxLowerBound_v1} without the 
absolute value operation is given by $(-1)^{1+\floor{\log\log x}}$. 
The precise formula for the 
limiting lower bound stated above for $\widehat{L}_0(x)$ is computed by symbolic summation 
in \emph{Mathematica} using the new bounds on $\widehat{\pi}_k(x)$ guaranteed by 
the theorem, and then by applying subsequent standard asymptotic estimates to the 
resulting formulas for large $x \rightarrow \infty$, e.g., 
in the form of \eqref{eqn_IncompleteGamma_PropB} and Stirling's formula. 
It follows that 
\begin{align} 
\label{eqn_proof_tag_GEInvxLowerBound_v2} 
|G_E^{-1}(x)| & \gg \left\lvert \sum_{e < d \leq (\log x)^{5} (\log\log x)} 
     \frac{\lambda(d) (\log d)^{\frac{1}{4}}}{\log\log d} \times 
     (-1)^{\floor{\log\log\left(\frac{(\log x)^{5} (\log\log x)}{d}\right)}} \cdot 
     \widehat{L}_0\left(\frac{(\log x)^{5} (\log\log x)}{d}\right) \right\rvert. 
\end{align} 
\textbf{Outline for the remainder of the proof.} 
We sketch the following steps remaining to prove our claimed lower bound on 
$|G_E^{-1}(x)|$: 
\begin{itemize}[itemsep=0pt,topsep=4pt,leftmargin=0.75in] 
\item[\textbf{(A)}] We identify an initial subinterval $\mathcal{R}_x$ where we can expect 
     constant sign term contributions resulting from the inputs to the function $\widehat{L}_0$ 
     involving both $d,x$ for $x$ large and $d$ on this smaller subinterval. 
\item[\textbf{(B)}] We factor out easily bounded terms from the expansion of the 
     monotone $\widehat{L}_0$ on this interval. 
\item[\textbf{(C)}] We define and determine additional asymptotic formulas we will 
     refer to in later sections for the resulting lower bounds on $|G_E^{-1}(x)|$ 
     that are formed by restricting the range of $d$ in 
     \eqref{eqn_proof_tag_GEInvxLowerBound_v2} to $\mathcal{R}_x$. 
\item[\textbf{(D)}] We argue 
     that the sums of oscillatory terms on the upper end of the deleted interval 
     for $d \in \left(e, (\log x)^{5} (\log\log x)\right] \setminus \mathcal{R}_x$ 
     cannot generate trivial bounds by cancellation with the new lower bounds. 
\end{itemize} 
\textbf{Part A.} 
We will simplify \eqref{eqn_proof_tag_GEInvxLowerBound_v2} by proving that there are 
ranges of consecutive integers over which we obtain essentially 
constant sign contributions from the 
function $\widehat{L}_0((\log x)^{5} (\log\log x) / d)$ as $x \rightarrow \infty$. 
In particular, consider that 
\begin{align*} 
\log\log\left(\frac{(\log x)^{5} (\log\log x)}{d}\right) & = 
     \log\log\left((\log x)^{5} (\log\log x)\right) \\ 
     & \phantom{=\ } + \log\left(1 - 
     \frac{\log d}{(\log x)^{5} (\log\log x) \log\left( 
     (\log x)^{5} (\log\log x)\right)}\right), 
     \mathrm{\ as\ } x \rightarrow \infty. 
\end{align*} 
If we take $d \in (e, \log x] =: \mathcal{R}_x$, we have that 
$$\frac{\log d}{(\log x)^{5} (\log\log x) \log\left( 
 (\log x)^{5} (\log\log x)\right)} = o(1) \rightarrow 0, \mathrm{\ as\ } 
 x \rightarrow \infty.$$  
For $d$ within $\mathcal{R}_x$, 
we expect that for almost every $x$ there are at most 
a handful of negligible cases of comparitively small order 
$d \leq d_{0,x}$ such that 
\[
\floor{\log\log\left(\frac{(\log x)^{5} (\log\log x)}{d}\right)} \sim 
     \floor{\log\log\left((\log x)^{5} (\log\log x)\right) + o(1)}, 
\]
changes in parity transitioning from $d = d_{0,x}-1$ to $d = d_{0,x}$. 
An argument making this assertion precise brings leads us to 
two primary cases that rely on the small-order distribution of the fractional parts 
$f_x := \left\{\log\log\left((\log x)^{5} (\log\log x)\right)\right\}$ within $[0, 1)$ for 
large $x \rightarrow \infty$ and any $\log d \in \mathcal{R}_x$: 
\begin{itemize}[itemsep=0pt,topsep=0pt,leftmargin=0.35in] 
\item[\textbf{(1)}] If the fractional part 
     $f_x = 0$, then 
     \begin{align*} 
     \floor{\log\log\left(\frac{(\log x)^{5} (\log\log x)}{d}\right)} & = 
          \floor{\log\log\left((\log x)^{5} (\log\log x)\right)} \\ 
          & \phantom{=\ } + 
          \floor{-\frac{\log d}{(\log x)^{5} (\log\log x) \log\left( 
          (\log x)^{5} (\log\log x)\right)}}. 
     \end{align*} 
     This implies that provided that 
     \[
     -1 \leq -\frac{\log d}{(\log x)^{5} (\log\log x) \log\left( 
          (\log x)^{5} (\log\log x)\right)} < 0, 
     \]
     we obtain a constant multplier as 
     $\operatorname{sgn}\left[\widehat{L}_0\left(\frac{(\log x)^{5} (\log\log x)}{d}\right)\right]$ 
     for $d \in \mathcal{R}_x$. 
     Since $d$ is positive and maximized at $\log x$, 
     this condition clearly happens whenever $x$ is sufficiently large. 
\item[\textbf{(2)}] If the fractional part $f_x \in (0, 1)$, then 
     \begin{align*} 
     & \floor{\log\log\left(\frac{(\log x)^{5} (\log\log x)}{d}\right)} = 
          \floor{\log\log\left((\log x)^{5} (\log\log x)\right)} \\ 
          & \phantom{\qquad =\ } + 
          \floor{\left\{\log\log\left((\log x)^{5} (\log\log x)\right)\right\} - 
          \frac{\log d}{(\log x)^{5} (\log\log x) \log\left( 
          (\log x)^{5} (\log\log x)\right)}}. 
     \end{align*} 
     Define shorthand notation for the function 
     $\mathcal{B}(x) := (\log x)^{5} (\log\log x) \log\left((\log x)^{5} (\log\log x)\right)$. 
     We require that 
     \begin{align*} 
     -1 & \leq f_x - \frac{\log d}{\mathcal{B}(x)} < 0 \iff 
          (1 + f_x) \cdot \mathcal{B}(x) \geq \log d > 0. 
     \end{align*} 
     This property is similarly clearly attained for $d \in \mathcal{R}_x$ 
     since $(1 + f_x) \cdot \mathcal{B}(x) \geq \mathcal{B}(x)$ 
     as $x \rightarrow \infty$ . 
\end{itemize} 
\textbf{Part B.} 
Provided that the sign term involving both $d$ and $x$ 
from \eqref{eqn_proof_tag_GEInvxLowerBound_v2} does not change for 
$d \in \mathcal{R}_x$, 
we can remove any oscillations in the sums due to sign changes in the monotonically 
decreasing function 
$\widehat{L}_0(d, x) := \widehat{L}_0\left((\log x)^{5} (\log\log x)/d\right)$. 
The function $\widehat{L}_0(d, x)$ is monotone decreasing 
in the variable $d$ for fixed $x$ as we sum along the 
subinterval $\mathcal{R}_x$ in ascending order. 
We can see that this function is decreasing 
in $d$ by computing its partial derivative and 
evaluating the asymptotic main terms as having a leading negative sign 
for all large $x$. 
Thus we determine that we should select $d := \log x$ in 
\eqref{eqn_proof_tag_GEInvxLowerBound_v2} to 
obtain a global lower bound on $|G_E^{-1}(x)|$ if we truncate the sum 
to range only over the subset of original indices $d \in \mathcal{R}_x$. \\ 
\textbf{Part C.} 
Let the magnitudes of the signed remainder term sums be 
defined for all sufficiently large $x$ by 
\[
R_E(x) := \left\lvert \sum_{\log x < d < \frac{(\log x)^{5} (\log\log x)}{e}} 
     \frac{\lambda(d) (\log d)^{\frac{1}{4}}}{\log\log d} \times 
     (-1)^{\floor{\log\log\left(\frac{(\log x)^{5} (\log\log x)}{d}\right)}} \cdot 
     \widehat{L}_0\left(\frac{(\log x)^{5} (\log\log x)}{d}\right) \right\rvert. 
\]
Set the function $T_E(x)$ to correspond to the 
easily factored dependence of the less simply integrable factors 
in $\widehat{L}_0(d, x)$ when we set $d := \log x$ on $\mathcal{R}_x$. 
This function is defined for all large enough $x$ as 
\begin{equation} 
\label{eqn_proof_tag_TExFuncDefAndBounds_v1} 
T_E(x) \gg \frac{1}{\log\left[(\log x)^{4} (\log\log x)\right]^{\frac{1}{2}} 
     \sqrt{\log\log\left[(\log x)^{4} (\log\log x)\right]}} \gg 
     \frac{1}{2 (\log\log x)^{\frac{1}{2}} \sqrt{\log\log\log x}}. 
\end{equation} 
Then in limiting cases the lower bounding function satisfies 
\begin{align} 
\notag 
S_{E,1}(x) & := \left\lvert \sum_{e < d \leq (\log x)^{5} (\log\log x)} 
     \frac{\lambda(d) (\log d)^{\frac{1}{4}}}{\log\log d} \times 
     (-1)^{\floor{\log\log\left(\frac{(\log x)^{5} (\log\log x)}{d}\right)}} 
     \widehat{L}_0\left(\frac{(\log x)^{5} (\log\log x)}{d}\right) 
     \right\rvert \\ 
\label{eqn_proof_tag_SE1xFuncExp_v1} 
     & \gg \frac{\widehat{C}_0}{\sqrt{2\pi}} \times 
     (\log x)^{\frac{5}{4}} (\log\log x)^{\frac{1}{4}} 
     T_E(x) \times 
     \left\lvert \sum_{e < d \leq \log x} 
     \frac{\lambda(d) (\log d)^{\frac{1}{4}}}{d^{1/4} \cdot \log\log d} 
     \right\rvert \\ 
\notag 
     & \gg 
     \frac{\widehat{C}_0}{2\sqrt{2\pi}} \times 
     \frac{(\log x)^{\frac{5}{4}}}{(\log\log x)^{\frac{1}{4}} \cdot 
     \sqrt{\log\log\log x}} \times 
     \left\lvert \sum_{e < d \leq \log x} 
     \frac{\lambda(d) (\log d)^{\frac{1}{4}}}{d^{1/4} \cdot \log\log d} 
     \right\rvert. 
\end{align} 
The formulas in 
\eqref{eqn_proof_tag_GEInvxLowerBound_v2} and \eqref{eqn_proof_tag_SE1xFuncExp_v1} 
imply the following lower bound by the triangle inequality 
that holds as $x \rightarrow \infty$: 
\begin{align} 
\label{eqn_proof_tag_GEInvxLowerBound_v3}
|G_E^{-1}(x)| & \gg 
     \Biggl\lvert S_{E,1}(x) - R_E(x) \Biggr\rvert \gg S_{E,1}(x), \mathrm{\ as\ } 
     x \rightarrow \infty. 
\end{align} 
We have claimed that we can in fact drop the sum terms over upper range of 
$d \notin \mathcal{R}_x$ and still 
obtain the asymptotic lower bound on $|G_E^{-1}(x)|$ stated in 
\eqref{eqn_proof_tag_GEInvxLowerBound_v3}. 
To justify this step in the proof, 
we will provide limiting lower bounds on $R_E(x)$ that show that the 
contribution from the deleted interval in absolute value exceeds the magnitude of the 
corresponding sums over $d \in \mathcal{R}_x$ defined by 
$S_{E,1}(x)$ when $x$ is large. \\ 
\textbf{Part D.} 
To obtain a lower bound on $R_E(x)$ lower bound, consider 
that since $\frac{(\log d)^{\frac{1}{4}}}{d^{1/4} \cdot \log\log d}$ 
is monotone decreasing for all large enough $d > e$, 
we obtain the smallest possible magnitude on the sum 
by alternating signs on consecutive terms in the sum. 
We can then bound the sum as $x \rightarrow \infty$ by 
\begin{align*} 
\frac{R_E(x)}{(\log x)^{\frac{5}{4}} (\log\log x)^{\frac{1}{4}}} & \gg 
     \left\lvert o(1) + 
     \sum_{\log x < d < \frac{(\log x)^{5} (\log\log x)}{2e}} 
     \left[ 
     \frac{\log(2d)^{1/4}}{(2d)^{1/4} \cdot \log\log(2d)} - 
     \frac{\log(2d+1)^{1/4}}{(2d+1)^{1/4} \log\log(2d+1)} 
     \right] \right\rvert \\ 
     & \approx \left\lvert 
     \sum_{\log x < d < \frac{(\log x)^{5} (\log\log x)}{2e}} 
     \frac{\log(2d)^{1/4}}{(2d)^{1/4} \log\log(2d)} \left[
     1 - 
     \frac{\left(1 + \frac{1}{2d \cdot \log(2d)}\right)^{1/4}}{ 
     \left(1 + \frac{1}{2d}\right)^{1/4} 
     \left(1 + \frac{1}{2d \cdot \log(2d))}\right)} 
     \right] \right\rvert. 
\end{align*} 
Expanding convergent binomial and geometric series expansions implies that 
\begin{align*} 
1 - \frac{\left(1 + \frac{1}{2d \cdot \log(2d)}\right)^{1/4}}{ 
    \left(1 + \frac{1}{2d}\right)^{1/4} 
    \left(1 + \frac{1}{2d \cdot \log(2d)}\right)} & = 
    1 - \left(1 - \frac{1}{8d} + O\left(\frac{1}{d^2}\right)\right) \times 
    \left(1 - \frac{3}{8d \log(2d)} + O\left(\frac{1}{d^2 \log^2(2d)}\right)\right) \\ 
    & = 
    \frac{1}{8d} + \frac{3}{8d \log(2d)} + O\left(\frac{1}{d^2 \log^2(2d)}\right). 
\end{align*} 
Hence, the significant terms in the inner terms of the last 
equation are bounded by 
\begin{align*} 
\frac{R_E(x)}{(\log x)^{\frac{5}{4}} (\log\log x)^{\frac{1}{4}}} & \gg 
     \left\lvert 
     \sum_{\log x < d < \frac{(\log x)^{5} (\log\log x)}{2e}} 
     O\left(\frac{\log(2d)^{1/4}}{(2d)^{5/4} \log\log(2d)}\right) 
     \right\rvert = 
     O\left(1\right). 
     \qedhere 
\end{align*} 
\end{proof} 

\subsubsection{A few more necessary results} 
\label{subsubSection_RoutineProofsNeededForMainBoundOnGInvxFunc} 

We now use the superscript and subscript notation of 
$(\ell)$ not to denote a formal parameter to 
the functions we define below, but instead to denote that these functions form 
\emph{lower bound} (rather than exact) 
approximations to other forms of the functions without the scripted $(\ell)$. 

\begin{lemma} 
\label{lemma_lowerBoundsOnLambdaFuncParitySummFuncs} 
Suppose that $\widehat{\pi}_k^{(\ell)}(x) = o\left(\widehat{\pi}_k(x)\right)$ where 
$\widehat{\pi}_k^{(\ell)}(x) \geq 1$ 
for all integers $1 \leq k \leq \log\log x$ as $x \rightarrow \infty$. 
Let the weighted summatory functions be defined as 
\begin{align*} 
A_{\Omega}^{(\ell)}(x) & := \sum_{k \leq \log\log x} (-1)^k \widehat{\pi}_k^{(\ell)}(x) \\ 
A_{\Omega}(x) & := \sum_{k \leq \log\log x} (-1)^k \widehat{\pi}_k(x). 
\end{align*} 
Futhermore, suppose that $|A_{\Omega}^{(\ell)}(x)|, |A_{\Omega}(x)| \nrightarrow 0$ as 
$x \rightarrow \infty$. 
Then for all sufficiently large $x$, we have that 
$$|A_{\Omega}(x)| \gg |A_{\Omega}^{(\ell)}(x)|.$$ 
\end{lemma} 
\begin{proof} 
We have by the first condition $\widehat{\pi}_k^{(\ell)}(x) = o\left(\widehat{\pi}_k(x)\right)$ that 
\begin{align*} 
\left\lvert \sum_{k \leq \log\log x} (-1)^{k} \widehat{\pi}_k(x) \left(1 - 
     \sup_{1 \leq k \leq \log\log x} \frac{\widehat{\pi}_k^{(\ell)}(x)}{\widehat{\pi}_k(x)}\right) 
     \right\rvert & \leq 
     \left\lvert A_{\Omega}(x) - A_{\Omega}^{(\ell)}(x) \right\rvert \\ 
\left\lvert \sum_{k \leq \log\log x} (-1)^{k} \widehat{\pi}_k(x) \left(1 + 
     \inf_{1 \leq k \leq \log\log x} \frac{\widehat{\pi}_k^{(\ell)}(x)}{\widehat{\pi}_k(x)}\right) 
     \right\rvert & \leq 
     \left\lvert A_{\Omega}(x) - A_{\Omega}^{(\ell)}(x) \right\rvert \\ 
\left\lvert \sum_{k \leq \log\log x} (-1)^{k} \widehat{\pi}_k(x) \left(1 - 
     \inf_{1 \leq k \leq \log\log x} \frac{\widehat{\pi}_k^{(\ell)}(x)}{\widehat{\pi}_k(x)}\right) 
     \right\rvert & \geq 
     \left\lvert A_{\Omega}(x) - A_{\Omega}^{(\ell)}(x) \right\rvert \\ 
\left\lvert \sum_{k \leq \log\log x} (-1)^{k} \widehat{\pi}_k(x) \left(1 + 
     \sup_{1 \leq k \leq \log\log x} \frac{\widehat{\pi}_k^{(\ell)}(x)}{\widehat{\pi}_k(x)}\right) 
     \right\rvert & \geq 
     \left\lvert A_{\Omega}(x) - A_{\Omega}^{(\ell)}(x) \right\rvert. 
\end{align*} 
This implies that 
\[
|A_{\Omega}(x)|(1+o(1)) \ll \left\lvert |A_{\Omega}(x)| \pm |A_{\Omega}^{(\ell)}(x)| \right\rvert \ll 
     |A_{\Omega}(x)|(1+o(1)), \mathrm{\ as\ } x \rightarrow \infty. 
\]
Because we have that 
$|A_{\Omega}^{(\ell)}(x)|, |A_{\Omega}(x)| \nrightarrow 0$, the previous equation shows that 
$|A_{\Omega}^{(\ell)}(x)|$ is bounded above and below by a constant times 
$|A_{\Omega}(x)|$. In other words, $|A_{\Omega}(x)| \gg |A_{\Omega}^{(\ell)}(x)|$ whenever 
$x$ is sufficiently large. 
\end{proof} 

\begin{proof}[Proof of Lemma \ref{lemma_CLT_and_AbelSummation}] 
We can form an accurate $C^{1}(\mathbb{R})$ approximation by the smoothness of 
$\widehat{\pi}_k^{(\ell)}(x)$ that allows us to apply the Abel summation formula using the summatory 
function $A_{\Omega}(t)$ for $t$ on any bounded connected subinterval of $[1, \infty)$. 
Namely, we see that 
\begin{align} 
\notag 
|F_{\lambda}(x)| & \gg 
     \left\lvert A_{\Omega}(x) f(x) - \int_{u_0}^{x} 
     A_{\Omega}(t) f^{\prime}(t) dt \right\rvert \\ 
\label{eqn_Flambdax_RHA_AbelSummationFormula_v2} 
     & \gg 
     \left\lvert |A_{\Omega}(x) f(x)| - \int_{u_0}^{x} 
     |A_{\Omega}(t) f^{\prime}(t)| dt \right\rvert \\ 
\notag 
     & \gg 
     \left\lvert |A_{\Omega}^{(\ell)}(x) \widehat{\tau}_{\ell}(x)| - \int_{u_0}^{x} 
     |A_{\Omega}(t) f^{\prime}(t)| dt \right\rvert. 
\end{align} 
The stated lower bound formula for $|F_{\lambda}(x)|$ in 
\eqref{eqn_Flambdax_RHA_AbelSummationFormula_v2} 
above is valid  whenever 
\[
0 \leq \left\lvert \frac{\displaystyle\sum\limits_{\log\log t < k \leq \frac{\log t}{\log 2}} 
     (-1)^k \widehat{\pi}_k(t)}{A_{\Omega}(t)}\right\rvert \ll 2, 
     \mathrm{\ as\ } t \rightarrow \infty, 
\]
Indeed, by Corollary \ref{theorem_MV_Thm7.20}, we have that 
the assertion above holds as $t \rightarrow \infty$. 
This property remarkably holds even when we should technically 
index over all $k \in [1, \log_2(x)]$ to obtain an exact formula for this 
summatory weight function given by 
$L(x) := \sum_{n \leq x} \lambda(n)$.  

Let the function 
\[
\widehat{I}_{\ell}(x) := \int_{\frac{\log\log x}{2} - \frac{1}{2}}^{\frac{\log\log x}{2}} 
     \left\lvert A_{\Omega}^{(\ell)}\left(e^{e^{2t}}\right) 
     \widehat{\tau}_{\ell}^{\prime}\left(e^{e^{2t}}\right) 
     \right\rvert e^{e^{2t}} dt. 
\]
We argue that two key properties of this function hold as $x \rightarrow \infty$: 
\begin{itemize}[itemsep=0pt,topsep=4pt,leftmargin=0.75in] 
\item[\textbf{(1)}] 
     $\left\lvert \int_{u_0}^{x} 
     |A_{\Omega}(t) f^{\prime}(t)| dt \right\rvert \gg \widehat{I}_{\ell}(x)$; and 
\item[\textbf{(2)}] 
     $\widehat{I}_{\ell}(x) = O\left(A_{\Omega}^{(\ell)}(\log\log x) 
      \widehat{\tau}_{\ell}(\log\log x)\right)$. 
\end{itemize} 
To prove property \textbf{(1)}, observe that by hypothesis since 
$|A_{\Omega}(x)| \gg |A_{\Omega}^{(\ell)}(x)|$ as $x \rightarrow \infty$, 
we have that 
\begin{align*} 
\int_{u_0}^{x} |A_{\Omega}(t) f^{\prime}(t)| dt & \gg 
     \int_{u_0}^{x} |A_{\Omega}(t) \widehat{\tau}_{\ell}^{\prime}(t)| dt \\ 
     & \gg \left\lvert \sum_{k=u_0}^{\log\log x} (-1)^{k} 
     \left\lvert A_{\Omega}\left(e^{e^{k}}\right) 
     \widehat{\tau}_{\ell}^{\prime}\left(e^{e^{k}}\right) \right\rvert \cdot \left( 
     e^{e^{k}} - e^{e^{k-1}}\right) \right\rvert \\ 
     & \gg \left\lvert \sum_{k=u_0}^{\frac{\log\log x}{2}} \left[ 
     \left\lvert A_{\Omega}\left(e^{e^{2k}}\right) 
     \widehat{\tau}_{\ell}^{\prime}\left(e^{e^{2k}}\right) \right\rvert \cdot e^{e^{2k}} - 
     \left\lvert A_{\Omega}\left(e^{e^{2k-1}}\right) 
     \widehat{\tau}_{\ell}^{\prime}\left(e^{e^{2k-1}}\right) \right\rvert \cdot e^{e^{2k-1}} 
     \right] \right\rvert \\ 
     & \gg 
     \int_{\frac{\log\log x}{2} - \frac{1}{2}}^{\frac{\log\log x}{2}} 
     \left\lvert A_{\Omega}\left(e^{e^{2t}}\right) 
     \widehat{\tau}_{\ell}^{\prime}\left(e^{e^{2t}}\right) 
     \right\rvert e^{e^{2t}} dt \\ 
     & \gg 
     \int_{\frac{\log\log x}{2} - \frac{1}{2}}^{\frac{\log\log x}{2}} 
     \left\lvert A_{\Omega}^{(\ell)}\left(e^{e^{2t}}\right) 
     \widehat{\tau}_{\ell}^{\prime}\left(e^{e^{2t}}\right) 
     \right\rvert e^{e^{2t}} dt. 
\end{align*} 
To prove property \textbf{(2)}, we see by the mean value theorem, the 
monotonicity of $|A_{\Omega}^{(\ell)}(x)|$ as $x \rightarrow \infty$, and the hypothesis 
$\left\lvert \widehat{\tau}_{\ell}\left(\frac{\log\log x}{2}\right) - 
 \widehat{\tau}_{\ell}\left(\frac{\log\log x}{2} - \frac{1}{2}\right) \right\rvert = 
 O\left(\frac{\widehat{\tau}_{\ell}(x)}{\log\log x}\right)$ as $x \rightarrow \infty$, that 
for some $c \in \left[\frac{\log\log x}{2} - \frac{1}{2}, \frac{\log\log x}{2}\right]$ we have 
\begin{align*} 
\widehat{I}_{\ell}(x) & = \left\lvert A_{\Omega}^{(\ell)}\left(e^{e^{2c}}\right) 
     \right\rvert e^{e^{2c}} \times \left\lvert 
     \widehat{\tau}_{\ell}\left(\frac{\log\log x}{2}\right) - 
     \widehat{\tau}_{\ell}\left(\frac{\log\log x}{2} - \frac{1}{2}\right) \right\rvert \\ 
     & = O\left(\log\log x \cdot A_{\Omega}^{(\ell)}\left(\log\log x\right) \times 
     \left\lvert 
     \widehat{\tau}_{\ell}\left(\frac{\log\log x}{2}\right) - 
     \widehat{\tau}_{\ell}\left(\frac{\log\log x}{2} - \frac{1}{2}\right) \right\rvert
     \right) \\ 
     & = O\left(A_{\Omega}^{(\ell)}(\log\log x) 
      \widehat{\tau}_{\ell}(\log\log x)\right). 
\end{align*} 
Combined with the last equation in \eqref{eqn_Flambdax_RHA_AbelSummationFormula_v2}, 
properties \textbf{(1)} and \textbf{(2)} imply the stated result. 
\end{proof} 

\begin{cor}[Conditions on our central bounding functions] 
\label{cor_CondsOnCentralBoundingFuncs_v3} 
Let the smooth bounding functions be defined for large $t \gg e$ as 
\begin{align*} 
\widehat{\tau}_{\ell}(t) & := \frac{(\log t)^{\frac{1}{4}}}{t^{\frac{1}{4}} \cdot (\log\log t)},  \\ 
A_{\Omega}^{(\ell)}(t) & := \frac{\widehat{C}_0}{\sqrt{2\pi}} \cdot 
     \frac{t^{\frac{1}{4}}}{(\log t)^{\frac{1}{2}} \sqrt{\log\log t}}. 
\end{align*} 
Then we have that as $x \rightarrow \infty$ 
\[
|G_E^{-1}(x)| \gg \frac{\widehat{C}_0}{2 \sqrt{2\pi}} \cdot 
     \frac{(\log x)^{5/4}}{(\log\log x)^{1/4} \sqrt{\log\log\log x}} \times 
     \left\lvert A_{\Omega}^{(\ell)}(\log x) \widehat{\tau}_{\ell}(\log x) - 
     \int_{\frac{\log\log\log x}{2} - \frac{1}{2}}^{\frac{\log\log\log x}{2}} 
     A_{\Omega}^{(\ell)}\left(e^{e^{2t}}\right) 
     \widehat{\tau}_{\ell}\left(e^{e^{2t}}\right) e^{e^{2t}} dt
     \right\rvert. 
\]
\end{cor} 
\begin{proof} 
By Corollary \ref{cor_ASemiForm_ForGInvx_v1}, we have that 
\begin{equation} 
\label{eqn_proof_tag_AbsGEInvLowerBoundByAbelSummation_v1} 
|G_E^{-1}(x)| \gg \frac{\widehat{C}_0}{2 \sqrt{2\pi}} \cdot 
     \frac{(\log x)^{5/4}}{(\log\log x)^{1/4} \sqrt{\log\log\log x}} \times 
     \left\lvert \sum_{e < d \leq \log x} \frac{\lambda(d) (\log d)^{1/4}}{ 
     d^{1/4} \cdot \log\log d} \right\rvert, 
     \mathrm{\ as\ } x \rightarrow \infty.  
\end{equation} 
The crux of the remainder of the proof boils down to checking hypotheses in 
Lemma \ref{lemma_lowerBoundsOnLambdaFuncParitySummFuncs} and 
Lemma \ref{lemma_CLT_and_AbelSummation}. 

We first apply Lemma \ref{lemma_lowerBoundsOnLambdaFuncParitySummFuncs} with 
the lower bound function resulting from 
Theorem \ref{theorem_GFs_SymmFuncs_SumsOfRecipOfPowsOfPrimes} as follows: 
\[
\widehat{\pi}_k^{(\ell)}(x) := 
     \frac{\widehat{C}_0 x^{\frac{1}{4}}}{(\log x)^{\frac{3}{2}}} 
     \frac{(\log\log x)^{k-1}}{(k-1)!}. 
\] 
This provides that the necessary hypotheses on the function 
$A_{\Omega}^{(\ell)}(t)$ required by 
Lemma \ref{lemma_CLT_and_AbelSummation} 
are satisfied according to the sums for the function 
approximated by \eqref{eqn_proof_tag_GEInvxLowerBound_v1} for large $t$. 

We next select the non-negative arithmetic function 
$f(d) := \frac{(\log d)^{1/4}}{d^{1/4} \cdot \log\log d}$ in applying 
Lemma \ref{lemma_CLT_and_AbelSummation}. In particular, we can take 
the function 
$\widehat{\tau}_{\ell}(t) := \frac{(\log t)^{1/4}}{t^{1/4} \cdot \log\log t}$, which is 
non-negative and monotone for all $t > e$. Furthermore, we compute that for large $x$ 
we have 
\begin{align*} 
\left\lvert \widehat{\tau}_{\ell}\left(\frac{x}{2}\right) - 
     \widehat{\tau}_{\ell}\left(\frac{x}{2} - \frac{1}{2}\right) \right\rvert & = 
     \widehat{\tau}_{\ell}\left(\frac{x}{2}\right) \times \left\lvert 1 - 
     \frac{\left(1 + \frac{1}{\log x} \cdot \log\left(1 - \frac{1}{x}\right)\right)^{1/4}}{ 
     \left(1 - \frac{1}{x}\right)^{1/4} \times 
     \log\left(1 + \frac{1}{\log x} \cdot \log\left(1 - \frac{1}{x}\right)\right)} 
     \right\rvert \\ 
     & =  \widehat{\tau}_{\ell}\left(\frac{x}{2}\right) \times \left\lvert 
     \frac{1}{4x} + \frac{3}{4x (\log x)} - \frac{1}{4 x^2 (\log x)^2} + 
     \frac{3}{16 x^2 (\log x)} + O\left(\frac{1}{x^3 (\log x)^2}\right) 
     \right\rvert \\ 
     & = O\left(\frac{\widehat{\tau}_{\ell}\left(x\right)}{x}\right). 
\end{align*} 
This shows that all of the requirements in 
Lemma \ref{lemma_CLT_and_AbelSummation} 
on our choice of $\widehat{\tau}_{\ell}(t)$ are also satisfied. 
So the stated result follows from 
\eqref{eqn_proof_tag_AbsGEInvLowerBoundByAbelSummation_v1} and 
Lemma \ref{lemma_CLT_and_AbelSummation}. 
\end{proof} 

\subsubsection{The proof of a central lower bound on the magnitude of $G_{E}^{-1}(x)$} 

The next central theorem is the last barrier required to prove 
Theorem \ref{cor_ThePipeDreamResult_v1} 
in the next subsection. 
Combined with Theorem \ref{theorem_GInvxLowerBoundByGEInvx_v1} 
proved in the last section, the new lower bounds we establish below provide us 
with a sufficient mechanism to bound the formula from 
Proposition \ref{prop_Mx_SBP_IntegralFormula}. 

\begin{theorem}[Asymptotics and bounds for the summatory function $G^{-1}(x)$] 
\label{theorem_gInv_GeneralAsymptoticsForms}
Let $C_{\ell,1} > 0$ be the absolute constant defined by 
\[
\widehat{C}_{\ell,1} = \frac{\widehat{C}_0^2}{32\pi} = 
     \frac{(\log 2) \cdot \exp\left(-\frac{15}{16} (\log 2)^2\right)}{8 \sqrt{2} \pi} 
     \approx 0.00792203.  
\]
We obtain the following limiting estimate for the bounding function 
$G_{E}^{-1}(x)$ as $x \rightarrow \infty$: 
\begin{align*} 
 & \left\lvert G_{E}^{-1}\left(x\right) \right\rvert
     \SuccSim 
     \frac{\left(8-e^{1/4}\right) \widehat{C}_{\ell,1} \cdot (\log x)^{5/4}}{ 
     \sqrt{\log\log x} \cdot (\log\log\log x)^2}. 
\end{align*} 
\end{theorem} 
\NBRef{A10-2020.04-26} 
\begin{proof} 
We can form a lower summatory function indicating the signed contributions over the distinct 
parity of $\Omega(n)$ for all $n \leq x$ as follows by applying 
\eqref{eqn_IncompleteGamma_PropA} and Stirling's approximation as already noted in the 
proof of Corollary \ref{cor_ASemiForm_ForGInvx_v1} given above: 
\begin{align} 
\label{proof_thm_GInvFunc_v0} 
\left\lvert A_{\Omega}^{(\ell)}(t) \right\rvert & = 
     \left\lvert \sum_{k \leq \log\log t} (-1)^k \widehat{\pi}_k(t) \right\rvert 
     \gg \frac{\widehat{C}_0}{\sqrt{2\pi}} \times 
     \frac{x^{\frac{1}{4}}}{(\log x)^{\frac{1}{2}} \sqrt{\log\log x}}, 
     \mathrm{\ as\ } t \rightarrow \infty. 
\end{align} 
We select the functions 
$\widehat{\tau}_0(t) := \frac{(\log t)^{1/4}}{t^{1/4} \cdot \log\log t}$ and 
$-\widehat{\tau}^{\prime}_0(t) \gg \frac{(\log t)^{1/4}}{4 t^{5/4} \cdot \log\log t}$ 
in the form of the next equation using 
the notation in Corollary \ref{cor_CondsOnCentralBoundingFuncs_v3}. 
\begin{align} 
\label{eqn_HatTauPrimet_summation_weight_func_exp_v2} 
-\widehat{\tau}_0^{\prime}(t) & = -\frac{d}{dt}\left[ 
     \frac{(\log t)^{\frac{1}{4}}}{t^{\frac{1}{4}} \cdot \log\log t} 
     \right] \SuccSim \frac{(\log t)^{1/4}}{4 t^{\frac{5}{4}} \cdot \log\log t} 
\end{align} 
Moreover, we have that we can select 
the initial form of the lower bound on the function $G_{E}^{-1}(x)$ to be defined as follows: 
\begin{align} 
\label{proof_thm_GInvFunc_v1} 
G_{E}^{-1}(x) & \gg \frac{\widehat{C}_0}{2 \sqrt{2\pi}} \cdot 
     \frac{(\log x)^{5/4}}{(\log\log x)^{1/4} \sqrt{\log\log\log x}} \times \\ 
\notag 
     & \phantom{:=\qquad\ } \times 
     \left\lvert A_{\Omega}^{(\ell)}(\log x) \widehat{\tau}_0(\log x) - 
     \int_{\frac{\log\log\log x}{2}-\frac{1}{2}}^{\frac{\log\log\log x}{2}} 
     \left\lvert 
     A_{\Omega}^{(\ell)}\left(e^{e^{2t}}\right) \widehat{\tau}_0^{\prime}\left(e^{e^{2t}}\right) 
     \right\rvert e^{e^{2t}} dt 
     \right\rvert. 
\end{align} 
We express the integrand function as the following function of $t$: 
\begin{equation} 
\label{eqn_proof_thm_GInvFunc_v3v2_approx} 
\widehat{I}_{\ell}(t) := \left\lvert 
     A_{\Omega}^{(\ell)}\left(e^{e^{2t}}\right) \widehat{\tau}_0^{\prime}\left(e^{e^{2t}}\right) 
     \right\rvert e^{e^{2t}} \gg \frac{\widehat{C}_0}{16 \sqrt{\pi}} \cdot \frac{e^{-t/2}}{t^{3/2}}. 
\end{equation} 
We find from the mean value theorem with the monotone function from 
\eqref{eqn_proof_thm_GInvFunc_v3v2_approx} that 
\begin{align} 
\label{eqn_proof_thm_GInvFunc_v4_approx} 
\frac{\widehat{C}_0}{2 \sqrt{2\pi}} \cdot 
     \frac{(\log x)^{5/4}}{(\log\log x)^{1/4} \sqrt{\log\log\log x}} \times 
     \int_{\frac{\log\log\log x}{2}-\frac{1}{2}}^{\frac{\log\log\log x}{2}} 
     \widehat{I}_{\ell}(t) dt & \ll 
     \frac{1}{2} \widehat{I}_{\ell}\left(\frac{\log\log\log x}{2}-\frac{1}{2}\right) \\ 
\notag 
     & = \frac{e^{1/4} \cdot \widehat{C}_{\ell,1} \cdot (\log x)^{5/4}}{ 
     \sqrt{\log\log x} \cdot (\log\log\log x)^2}. 
\end{align} 
%Similarly, by evaluating $\widehat{I}_{\ell}(t)$ at the 
%upper bound on the integral above we can conclude that 
%\begin{align} 
%\notag 
%\frac{\widehat{C}_0}{2 \sqrt{2\pi}} \cdot 
%     \frac{(\log x)^{5/4}}{(\log\log x)^{1/4} \sqrt{\log\log\log x}} \times 
%     \int_{\frac{\log\log\log x}{2}-\frac{1}{2}}^{\frac{\log\log\log x}{2}} 
%     \widehat{I}_{\ell}(t) dt & \gg 
%     \frac{1}{2} \widehat{I}_{\ell}\left(\frac{\log\log\log x}{2}\right) \\ 
%\notag 
%     & = \frac{\widehat{C}_{\ell,1} \cdot (\log x)^{5/4}}{ 
%     \sqrt{\log\log x} \cdot (\log\log\log x)^2}. 
%\end{align} 
Consider the following expansion for the leading term in 
the Abel summation formula from \eqref{proof_thm_GInvFunc_v1} for comparison with 
\eqref{eqn_proof_thm_GInvFunc_v4_approx}: 
\begin{align} 
\label{eqn_proof_thm_GInvFunc_v5_approx} 
\frac{\widehat{C}_0}{2 \sqrt{2\pi}} & \cdot 
     \frac{(\log x)^{5/4}}{(\log\log x)^{1/4} \sqrt{\log\log\log x}} \times 
     \left\lvert A_{\Omega}^{(\ell)}(\log x) \widehat{\tau}_0(\log x) \right\rvert 
     \gg 
     \frac{8 \widehat{C}_{\ell,1} \cdot (\log x)^{5/4}}{ 
     \sqrt{\log\log x} \cdot (\log\log\log x)^2}
\end{align} 
Hence, we conclude that we can take $\left\lvert G_{E}^{-1}\left(x\right) \right\rvert$ 
bounded below by the difference of terms in 
\eqref{eqn_proof_thm_GInvFunc_v5_approx} and 
\eqref{eqn_proof_thm_GInvFunc_v4_approx}. 
\end{proof} 

\subsection{Proof of the unboundedness of the scaled Mertens function}
\label{subSection_TheCoreResultProof} 

We finally address the main conclusion of our arguments given so far with the 
following proof: 

\begin{proof}[Proof of Theorem \ref{cor_ThePipeDreamResult_v1}] 
\label{proofOf_cor_ThePipeDreamResult_v1} 
We split the interval of integration from 
Proposition \ref{prop_Mx_SBP_IntegralFormula} 
over $t \in [1, x/2]$ into two disjoint subintervals: 
one that is easily bounded 
from $1 \leq t \leq \sqrt{x}$ 
and another that will conveniently give us our slow-growing tendency towards 
infinity along the subsequence when evaluated using 
Theorem \ref{theorem_gInv_GeneralAsymptoticsForms}. 
Given a fixed large infinitely tending $x$, we have some (at least one) point 
$x_0 \in \left[\sqrt{x}, \frac{x}{2}\right]$ defined such that 
$|G^{-1}(t)|$ is minimal and non-vanishing as 
\[
\left\lvert G^{-1}(x_0) \right\rvert := 
     \min_{\substack{\sqrt{x} \leq t \leq \frac{x}{2} \\ G^{-1}(t) \neq 0}} |G^{-1}(t)|. 
\]
We can then apply Proposition \ref{prop_Mx_SBP_IntegralFormula} to bound the function as 
follows: 
\begin{align} 
\notag 
\frac{|M(x)|}{\sqrt{x}} & = 
     \frac{1}{\sqrt{x}} \left\lvert G^{-1}(x) - x \cdot \int_1^{x/2} \frac{G^{-1}(t)}{ 
     t^2 \cdot \log(x/t)} dt \right\rvert \\ 
\label{eqn_MxGInvxLowerBound_stmt_v0} 
     & \gg 
     \left\lvert \left\lvert \frac{G^{-1}(x)}{\sqrt{x}} \right\rvert - \sqrt{x} 
     \int_1^{x/2} \frac{|G^{-1}(t)|}{ 
     t^2 \cdot \log(x/t)} dt \right\rvert \\ 
\notag 
     & \gg 
     \sqrt{x} \times \left\lvert \int_{\sqrt{x}}^{x/2} \frac{|G^{-1}(t)|}{ 
     t^2 \cdot \log(x/t)} dt \right\rvert \\ 
\notag 
     & \gg \left( 
     \min\limits_{\substack{\sqrt{x} \leq t \leq \frac{x}{2} \\ G^{-1}(t) \neq 0}} |G^{-1}(t)| 
     \right) \times \left\lvert \int_{\sqrt{x}}^{\frac{x}{2}} \frac{2 \sqrt{x}}{ 
     t^2 \cdot \log\left(x_0\right)} dt \right\rvert \\ 
\label{eqn_MxGInvxLowerBound_stmt_v1} 
     & \gg  
     \frac{2 \left\lvert G^{-1}(x_0) \right\rvert}{\log\left(x_0\right)}. 
\end{align} 
In the second to last step, we observe that $G^{-1}(x) = 0$ for $x$ on a set of 
asymptotic density \emph{at least} bounded below by $\frac{1}{2}$, so that our 
claim is accurate as the integral bound does not vanish at large $x$. 
 
To complete the logic to the bound we arrived at in \eqref{eqn_MxGInvxLowerBound_stmt_v1}, 
first observe that the difference of terms we have in 
\eqref{eqn_MxGInvxLowerBound_stmt_v0} corresponds to the first term having a 
bound from below of the form (see the proof of Theorem \ref{theorem_GInvxLowerBoundByGEInvx_v1}) 
\[
\frac{|G^{-1}(x)|}{\sqrt{x}} \gg \frac{6\sqrt{x}}{\pi^2} (\log x) (\log\log x), 
     \mathrm{\ for\ a.e.\ } x, \mathrm{\ as\ } x \rightarrow \infty. 
\]
Secondly, for the sake of argument, 
suppose that there is a smooth approximation for $|G^{-1}(t)|$ so that by the 
mean value theorem for 
some $c_0 \in [1, \sqrt{x}]$ and $c_1 \in \left[\sqrt{x}, \frac{x}{2}\right]$ we have 
\begin{align*} 
\sqrt{x} & \left\lvert \int_1^{x/2} \frac{|G^{-1}(t)|}{t^2 \cdot \log(x/t)} dt \right\rvert \\ 
     & \gg \left\lvert \frac{\sqrt{x} \cdot |G^{-1}(c_0)|}{c_0} \times 
     \left\lvert \int_1^{\sqrt{x}} \frac{dt}{t \log(x/t)} \right\rvert + 
     \sqrt{x} \cdot |G^{-1}(c_1)| \times 
     \left\lvert \int_{\sqrt{x}}^{x/2} \frac{dt}{t^2 \log(x)} \right\rvert \right\rvert \\ 
     & \gg \left\lvert \left(\min\limits_{\substack{1 \leq c \leq \sqrt{x} \\ G^{-1}(c) \neq 0}} 
     |G^{-1}(c)|\right) \times \log\log x + 
     \left(\min\limits_{\substack{\sqrt{x} \leq c^{\ast} \leq \frac{x}{2} \\ G^{-1}(c^{\ast}) \neq 0}} 
     |G^{-1}(c^{\ast})|\right) \times \left(\frac{1}{\log x} + o\left(\frac{1}{\log x}\right)\right) 
     \right\rvert. 
\end{align*} 
Since $G^{-1}(x)$ changes stepwise 
only at $x \in \mathbb{Z}^{+}$, what we in fact exactly arrive at 
is a close variant of this mean value theorem type observation. 

By Theorem \ref{theorem_GInvxLowerBoundByGEInvx_v1}, the result in 
\eqref{eqn_MxGInvxLowerBound_stmt_v1} implies that 
\begin{align} 
\label{eqn_MxGInvxLowerBound_stmt_v2} 
\frac{|M(x)|}{\sqrt{x}} & \gg \frac{2 \left\lvert G_{E}^{-1}(x_0) \right\rvert}{\log\left(x_0\right)}. 
\end{align} 
Define the infinite increasing subsequence, 
$\{x_{0,y}\}_{y \geq Y_0}$, by $x_{0,y} := e^{2e^{e^{2y+1}}}$ for sequence indices 
starting at some sufficiently 
large finite integer $Y_0 \gg 1$. 
We can verify that for sufficiently large $y \rightarrow \infty$, this infinitely 
tending subsequence is well defined as $\hat{x}_{0,y+1} > \hat{x}_{0,y}$ whenever $y \geq Y_0$. 
When we assume that $x \mapsto x_{0,y}$ is taken along this subsequence, 
we can transform the bound in the last 
equation into a statement about a lower bound for $|M(x)| / \sqrt{x}$ 
along an infinitely tending subsequence by 
applying Theorem \ref{theorem_gInv_GeneralAsymptoticsForms} to 
\eqref{eqn_MxGInvxLowerBound_stmt_v2} in the following form: 
\begin{align} 
\label{eqn_MxGInvxLowerBound_stmt_v3} 
\frac{|M(x_{0,y})|}{\sqrt{x_{0,y}}} & \gg 
     \frac{2 \left(8-e^{1/4}\right) \cdot \widehat{C}_{\ell,1} \cdot 
     (\log \sqrt{x_{0,y}})^{\frac{1}{4}}}{ 
     (\log\log \sqrt{x_{0,y}})^{\frac{1}{2}} (\log\log\log \sqrt{x_{0,y}})^2}, 
     \mathrm{\ as\ } y \rightarrow \infty. 
\end{align} 
Finally, we evaluate the following limit to conclude unboundedness 
where $\sqrt{x_{0,y}} \rightarrow +\infty$ as $y \rightarrow +\infty$: 
\[
\lim_{x \rightarrow \infty} \left[\frac{(\log x)^{\frac{1}{4}}}{ 
     (\log\log x)^{\frac{1}{2}} (\log\log\log x)^2}  
     \right] = +\infty. 
\]
There is a small, but nonetheless insightful point to explain about a 
technicality in stating \eqref{eqn_MxGInvxLowerBound_stmt_v3}. 
Namely, we are not asserting that 
$|M(x)| / \sqrt{x}$ grows unbounded along the precise subsequence of 
$x \mapsto x_{0,y}$ itself as $y \rightarrow \infty$. 
Rather, we are asserting that the unboundedness of this function 
can be witnessed along some subsequence whose points are taken within a 
large interval window for 
$\hat{x}_{0,y} \in \left[\sqrt{x_{0,y}}, \frac{x_{0,y}}{2}\right]$ as 
$y \rightarrow \infty$. 
We choose to state the lower bound given on the right-hand-side of 
\eqref{eqn_MxGInvxLowerBound_stmt_v3} using the 
monotonicity of the lower bound on $|G_E^{-1}(x)|$ we proved in 
Theorem \ref{theorem_gInv_GeneralAsymptoticsForms} 
with $\hat{x}_{0,y} \geq \sqrt{x_{0,y}}$. 
\end{proof} 

\newpage 
\renewcommand{\refname}{References} 
\bibliography{glossaries-bibtex/thesis-references}{}
\bibliographystyle{plain}

\newpage
\setcounter{section}{0} 
\renewcommand{\thesection}{T.\arabic{section}} 

\section{Table: The Dirichlet inverse function $g^{-1}(n)$ and the 
         distribution of its summatory function} 
\label{table_conjecture_Mertens_ginvSeq_approx_values}

\begin{table}[h!]

\centering

\tiny
\begin{equation*}
\boxed{
\begin{array}{cc|cc|ccc|cc|ccc}
 n & \mathbf{Primes} & \mathbf{Sqfree} & \mathbf{PPower} & g^{-1}(n) & 
 \lambda(n) g^{-1}(n) - \widehat{f}_1(n) & 
 \frac{\sum_{d|n} C_{\Omega(d)}(d)}{|g^{-1}(n)|} & 
 \mathcal{L}_{+}(n) & \mathcal{L}_{-}(n) & 
 G^{-1}(n) & G^{-1}_{+}(n) & G^{-1}_{-}(n) \\ \hline 
1 & 1^1 & \text{Y} & \text{N} & 1 & 0 & 1.0000000 & 1.000000 & 0.000000 & 1 & 1 & 0 \\
 2 & 2^1 & \text{Y} & \text{Y} & -2 & 0 & 1.0000000 & 0.500000 & 0.500000 & -1 & 1 & -2 \\
 3 & 3^1 & \text{Y} & \text{Y} & -2 & 0 & 1.0000000 & 0.333333 & 0.666667 & -3 & 1 & -4 \\
 4 & 2^2 & \text{N} & \text{Y} & 2 & 0 & 1.5000000 & 0.500000 & 0.500000 & -1 & 3 & -4 \\
 5 & 5^1 & \text{Y} & \text{Y} & -2 & 0 & 1.0000000 & 0.400000 & 0.600000 & -3 & 3 & -6 \\
 6 & 2^1 3^1 & \text{Y} & \text{N} & 5 & 0 & 1.0000000 & 0.500000 & 0.500000 & 2 & 8 & -6 \\
 7 & 7^1 & \text{Y} & \text{Y} & -2 & 0 & 1.0000000 & 0.428571 & 0.571429 & 0 & 8 & -8 \\
 8 & 2^3 & \text{N} & \text{Y} & -2 & 0 & 2.0000000 & 0.375000 & 0.625000 & -2 & 8 & -10 \\
 9 & 3^2 & \text{N} & \text{Y} & 2 & 0 & 1.5000000 & 0.444444 & 0.555556 & 0 & 10 & -10 \\
 10 & 2^1 5^1 & \text{Y} & \text{N} & 5 & 0 & 1.0000000 & 0.500000 & 0.500000 & 5 & 15 & -10 \\
 11 & 11^1 & \text{Y} & \text{Y} & -2 & 0 & 1.0000000 & 0.454545 & 0.545455 & 3 & 15 & -12 \\
 12 & 2^2 3^1 & \text{N} & \text{N} & -7 & 2 & 1.2857143 & 0.416667 & 0.583333 & -4 & 15 & -19 \\
 13 & 13^1 & \text{Y} & \text{Y} & -2 & 0 & 1.0000000 & 0.384615 & 0.615385 & -6 & 15 & -21 \\
 14 & 2^1 7^1 & \text{Y} & \text{N} & 5 & 0 & 1.0000000 & 0.428571 & 0.571429 & -1 & 20 & -21 \\
 15 & 3^1 5^1 & \text{Y} & \text{N} & 5 & 0 & 1.0000000 & 0.466667 & 0.533333 & 4 & 25 & -21 \\
 16 & 2^4 & \text{N} & \text{Y} & 2 & 0 & 2.5000000 & 0.500000 & 0.500000 & 6 & 27 & -21 \\
 17 & 17^1 & \text{Y} & \text{Y} & -2 & 0 & 1.0000000 & 0.470588 & 0.529412 & 4 & 27 & -23 \\
 18 & 2^1 3^2 & \text{N} & \text{N} & -7 & 2 & 1.2857143 & 0.444444 & 0.555556 & -3 & 27 & -30 \\
 19 & 19^1 & \text{Y} & \text{Y} & -2 & 0 & 1.0000000 & 0.421053 & 0.578947 & -5 & 27 & -32 \\
 20 & 2^2 5^1 & \text{N} & \text{N} & -7 & 2 & 1.2857143 & 0.400000 & 0.600000 & -12 & 27 & -39 \\
 21 & 3^1 7^1 & \text{Y} & \text{N} & 5 & 0 & 1.0000000 & 0.428571 & 0.571429 & -7 & 32 & -39 \\
 22 & 2^1 11^1 & \text{Y} & \text{N} & 5 & 0 & 1.0000000 & 0.454545 & 0.545455 & -2 & 37 & -39 \\
 23 & 23^1 & \text{Y} & \text{Y} & -2 & 0 & 1.0000000 & 0.434783 & 0.565217 & -4 & 37 & -41 \\
 24 & 2^3 3^1 & \text{N} & \text{N} & 9 & 4 & 1.5555556 & 0.458333 & 0.541667 & 5 & 46 & -41 \\
 25 & 5^2 & \text{N} & \text{Y} & 2 & 0 & 1.5000000 & 0.480000 & 0.520000 & 7 & 48 & -41 \\
 26 & 2^1 13^1 & \text{Y} & \text{N} & 5 & 0 & 1.0000000 & 0.500000 & 0.500000 & 12 & 53 & -41 \\
 27 & 3^3 & \text{N} & \text{Y} & -2 & 0 & 2.0000000 & 0.481481 & 0.518519 & 10 & 53 & -43 \\
 28 & 2^2 7^1 & \text{N} & \text{N} & -7 & 2 & 1.2857143 & 0.464286 & 0.535714 & 3 & 53 & -50 \\
 29 & 29^1 & \text{Y} & \text{Y} & -2 & 0 & 1.0000000 & 0.448276 & 0.551724 & 1 & 53 & -52 \\
 30 & 2^1 3^1 5^1 & \text{Y} & \text{N} & -16 & 0 & 1.0000000 & 0.433333 & 0.566667 & -15 & 53 & -68 \\
 31 & 31^1 & \text{Y} & \text{Y} & -2 & 0 & 1.0000000 & 0.419355 & 0.580645 & -17 & 53 & -70 \\
 32 & 2^5 & \text{N} & \text{Y} & -2 & 0 & 3.0000000 & 0.406250 & 0.593750 & -19 & 53 & -72 \\
 33 & 3^1 11^1 & \text{Y} & \text{N} & 5 & 0 & 1.0000000 & 0.424242 & 0.575758 & -14 & 58 & -72 \\
 34 & 2^1 17^1 & \text{Y} & \text{N} & 5 & 0 & 1.0000000 & 0.441176 & 0.558824 & -9 & 63 & -72 \\
 35 & 5^1 7^1 & \text{Y} & \text{N} & 5 & 0 & 1.0000000 & 0.457143 & 0.542857 & -4 & 68 & -72 \\
 36 & 2^2 3^2 & \text{N} & \text{N} & 14 & 9 & 1.3571429 & 0.472222 & 0.527778 & 10 & 82 & -72 \\
 37 & 37^1 & \text{Y} & \text{Y} & -2 & 0 & 1.0000000 & 0.459459 & 0.540541 & 8 & 82 & -74 \\
 38 & 2^1 19^1 & \text{Y} & \text{N} & 5 & 0 & 1.0000000 & 0.473684 & 0.526316 & 13 & 87 & -74 \\
 39 & 3^1 13^1 & \text{Y} & \text{N} & 5 & 0 & 1.0000000 & 0.487179 & 0.512821 & 18 & 92 & -74 \\
 40 & 2^3 5^1 & \text{N} & \text{N} & 9 & 4 & 1.5555556 & 0.500000 & 0.500000 & 27 & 101 & -74 \\
 41 & 41^1 & \text{Y} & \text{Y} & -2 & 0 & 1.0000000 & 0.487805 & 0.512195 & 25 & 101 & -76 \\
 42 & 2^1 3^1 7^1 & \text{Y} & \text{N} & -16 & 0 & 1.0000000 & 0.476190 & 0.523810 & 9 & 101 & -92 \\
 43 & 43^1 & \text{Y} & \text{Y} & -2 & 0 & 1.0000000 & 0.465116 & 0.534884 & 7 & 101 & -94 \\
 44 & 2^2 11^1 & \text{N} & \text{N} & -7 & 2 & 1.2857143 & 0.454545 & 0.545455 & 0 & 101 & -101 \\
 45 & 3^2 5^1 & \text{N} & \text{N} & -7 & 2 & 1.2857143 & 0.444444 & 0.555556 & -7 & 101 & -108 \\
 46 & 2^1 23^1 & \text{Y} & \text{N} & 5 & 0 & 1.0000000 & 0.456522 & 0.543478 & -2 & 106 & -108 \\
 47 & 47^1 & \text{Y} & \text{Y} & -2 & 0 & 1.0000000 & 0.446809 & 0.553191 & -4 & 106 & -110 \\
 48 & 2^4 3^1 & \text{N} & \text{N} & -11 & 6 & 1.8181818 & 0.437500 & 0.562500 & -15 & 106 & -121 \\ 
\end{array}
}
\end{equation*}

\bigskip\hrule\smallskip 

\captionsetup{singlelinecheck=off} 
\caption*{\textbf{\rm \bf Table \thesection:} 
          \textbf{Computations with $\mathbf{g^{-1}(n) \equiv (\omega+1)^{-1}(n)}$ 
          for $\mathbf{1 \leq n \leq 500}$.} 
          \begin{itemize}[noitemsep,topsep=0pt,leftmargin=0.23in] 
          \item[$\blacktriangleright$] 
          The column labeled \texttt{Primes} provides the prime factorization of each $n$ so that the values of 
          $\omega(n)$ and $\Omega(n)$ are easily extracted. 
          The columns labeled \texttt{Sqfree} and \texttt{PPower}, respectively, 
          list inclusion of $n$ in the sets of squarefree integers and the prime powers. 
          \item[$\blacktriangleright$] 
          The next three columns provide the 
          explicit values of the inverse function $g^{-1}(n)$ and compare its explicit value with other estimates. 
          We define the function $\widehat{f}_1(n) := \sum_{k=0}^{\omega(n)} \binom{\omega(n)}{k} \cdot k!$. 
          \item[$\blacktriangleright$] 
          The last several columns indicate properties of the summatory function of $g^{-1}(n)$. 
          The notation for the densities of the sign weight of $g^{-1}(n)$ is defined as 
          $\mathcal{L}_{\pm}(x) := \frac{1}{n} \cdot \#\left\{n \leq x: \lambda(n) = \pm 1\right\}$. 
          The last three 
          columns then show the explicit components to the signed summatory function, 
          $G^{-1}(x) := \sum_{n \leq x} g^{-1}(n)$, decomposed into its 
          respective positive and negative magnitude sum contributions: $G^{-1}(x) = G^{-1}_{+}(x) + G^{-1}_{-}(x)$ where 
          $G^{-1}_{+}(x) > 0$ and $G^{-1}_{-}(x) < 0$ for all $x \geq 1$. 
          \end{itemize} 
          } 

\end{table}

\newpage
\begin{table}[h!]

\centering

\tiny
\begin{equation*}
\boxed{
\begin{array}{cc|cc|ccc|cc|ccc}
 n & \mathbf{Primes} & \mathbf{Sqfree} & \mathbf{PPower} & g^{-1}(n) & 
 \lambda(n) g^{-1}(n) - \widehat{f}_1(n) & 
 \frac{\sum_{d|n} C_{\Omega(d)}(d)}{|g^{-1}(n)|} & 
 \mathcal{L}_{+}(n) & \mathcal{L}_{-}(n) & 
 G^{-1}(n) & G^{-1}_{+}(n) & G^{-1}_{-}(n) \\ \hline 
 49 & 7^2 & \text{N} & \text{Y} & 2 & 0 & 1.5000000 & 0.448980 & 0.551020 & -13 & 108 & -121 \\
 50 & 2^1 5^2 & \text{N} & \text{N} & -7 & 2 & 1.2857143 & 0.440000 & 0.560000 & -20 & 108 & -128 \\
 51 & 3^1 17^1 & \text{Y} & \text{N} & 5 & 0 & 1.0000000 & 0.450980 & 0.549020 & -15 & 113 & -128 \\
 52 & 2^2 13^1 & \text{N} & \text{N} & -7 & 2 & 1.2857143 & 0.442308 & 0.557692 & -22 & 113 & -135 \\
 53 & 53^1 & \text{Y} & \text{Y} & -2 & 0 & 1.0000000 & 0.433962 & 0.566038 & -24 & 113 & -137 \\
 54 & 2^1 3^3 & \text{N} & \text{N} & 9 & 4 & 1.5555556 & 0.444444 & 0.555556 & -15 & 122 & -137 \\
 55 & 5^1 11^1 & \text{Y} & \text{N} & 5 & 0 & 1.0000000 & 0.454545 & 0.545455 & -10 & 127 & -137 \\
 56 & 2^3 7^1 & \text{N} & \text{N} & 9 & 4 & 1.5555556 & 0.464286 & 0.535714 & -1 & 136 & -137 \\
 57 & 3^1 19^1 & \text{Y} & \text{N} & 5 & 0 & 1.0000000 & 0.473684 & 0.526316 & 4 & 141 & -137 \\
 58 & 2^1 29^1 & \text{Y} & \text{N} & 5 & 0 & 1.0000000 & 0.482759 & 0.517241 & 9 & 146 & -137 \\
 59 & 59^1 & \text{Y} & \text{Y} & -2 & 0 & 1.0000000 & 0.474576 & 0.525424 & 7 & 146 & -139 \\
 60 & 2^2 3^1 5^1 & \text{N} & \text{N} & 30 & 14 & 1.1666667 & 0.483333 & 0.516667 & 37 & 176 & -139 \\
 61 & 61^1 & \text{Y} & \text{Y} & -2 & 0 & 1.0000000 & 0.475410 & 0.524590 & 35 & 176 & -141 \\
 62 & 2^1 31^1 & \text{Y} & \text{N} & 5 & 0 & 1.0000000 & 0.483871 & 0.516129 & 40 & 181 & -141 \\
 63 & 3^2 7^1 & \text{N} & \text{N} & -7 & 2 & 1.2857143 & 0.476190 & 0.523810 & 33 & 181 & -148 \\
 64 & 2^6 & \text{N} & \text{Y} & 2 & 0 & 3.5000000 & 0.484375 & 0.515625 & 35 & 183 & -148 \\
 65 & 5^1 13^1 & \text{Y} & \text{N} & 5 & 0 & 1.0000000 & 0.492308 & 0.507692 & 40 & 188 & -148 \\
 66 & 2^1 3^1 11^1 & \text{Y} & \text{N} & -16 & 0 & 1.0000000 & 0.484848 & 0.515152 & 24 & 188 & -164 \\
 67 & 67^1 & \text{Y} & \text{Y} & -2 & 0 & 1.0000000 & 0.477612 & 0.522388 & 22 & 188 & -166 \\
 68 & 2^2 17^1 & \text{N} & \text{N} & -7 & 2 & 1.2857143 & 0.470588 & 0.529412 & 15 & 188 & -173 \\
 69 & 3^1 23^1 & \text{Y} & \text{N} & 5 & 0 & 1.0000000 & 0.478261 & 0.521739 & 20 & 193 & -173 \\
 70 & 2^1 5^1 7^1 & \text{Y} & \text{N} & -16 & 0 & 1.0000000 & 0.471429 & 0.528571 & 4 & 193 & -189 \\
 71 & 71^1 & \text{Y} & \text{Y} & -2 & 0 & 1.0000000 & 0.464789 & 0.535211 & 2 & 193 & -191 \\
 72 & 2^3 3^2 & \text{N} & \text{N} & -23 & 18 & 1.4782609 & 0.458333 & 0.541667 & -21 & 193 & -214 \\
 73 & 73^1 & \text{Y} & \text{Y} & -2 & 0 & 1.0000000 & 0.452055 & 0.547945 & -23 & 193 & -216 \\
 74 & 2^1 37^1 & \text{Y} & \text{N} & 5 & 0 & 1.0000000 & 0.459459 & 0.540541 & -18 & 198 & -216 \\
 75 & 3^1 5^2 & \text{N} & \text{N} & -7 & 2 & 1.2857143 & 0.453333 & 0.546667 & -25 & 198 & -223 \\
 76 & 2^2 19^1 & \text{N} & \text{N} & -7 & 2 & 1.2857143 & 0.447368 & 0.552632 & -32 & 198 & -230 \\
 77 & 7^1 11^1 & \text{Y} & \text{N} & 5 & 0 & 1.0000000 & 0.454545 & 0.545455 & -27 & 203 & -230 \\
 78 & 2^1 3^1 13^1 & \text{Y} & \text{N} & -16 & 0 & 1.0000000 & 0.448718 & 0.551282 & -43 & 203 & -246 \\
 79 & 79^1 & \text{Y} & \text{Y} & -2 & 0 & 1.0000000 & 0.443038 & 0.556962 & -45 & 203 & -248 \\
 80 & 2^4 5^1 & \text{N} & \text{N} & -11 & 6 & 1.8181818 & 0.437500 & 0.562500 & -56 & 203 & -259 \\
 81 & 3^4 & \text{N} & \text{Y} & 2 & 0 & 2.5000000 & 0.444444 & 0.555556 & -54 & 205 & -259 \\
 82 & 2^1 41^1 & \text{Y} & \text{N} & 5 & 0 & 1.0000000 & 0.451220 & 0.548780 & -49 & 210 & -259 \\
 83 & 83^1 & \text{Y} & \text{Y} & -2 & 0 & 1.0000000 & 0.445783 & 0.554217 & -51 & 210 & -261 \\
 84 & 2^2 3^1 7^1 & \text{N} & \text{N} & 30 & 14 & 1.1666667 & 0.452381 & 0.547619 & -21 & 240 & -261 \\
 85 & 5^1 17^1 & \text{Y} & \text{N} & 5 & 0 & 1.0000000 & 0.458824 & 0.541176 & -16 & 245 & -261 \\
 86 & 2^1 43^1 & \text{Y} & \text{N} & 5 & 0 & 1.0000000 & 0.465116 & 0.534884 & -11 & 250 & -261 \\
 87 & 3^1 29^1 & \text{Y} & \text{N} & 5 & 0 & 1.0000000 & 0.471264 & 0.528736 & -6 & 255 & -261 \\
 88 & 2^3 11^1 & \text{N} & \text{N} & 9 & 4 & 1.5555556 & 0.477273 & 0.522727 & 3 & 264 & -261 \\
 89 & 89^1 & \text{Y} & \text{Y} & -2 & 0 & 1.0000000 & 0.471910 & 0.528090 & 1 & 264 & -263 \\
 90 & 2^1 3^2 5^1 & \text{N} & \text{N} & 30 & 14 & 1.1666667 & 0.477778 & 0.522222 & 31 & 294 & -263 \\
 91 & 7^1 13^1 & \text{Y} & \text{N} & 5 & 0 & 1.0000000 & 0.483516 & 0.516484 & 36 & 299 & -263 \\
 92 & 2^2 23^1 & \text{N} & \text{N} & -7 & 2 & 1.2857143 & 0.478261 & 0.521739 & 29 & 299 & -270 \\
 93 & 3^1 31^1 & \text{Y} & \text{N} & 5 & 0 & 1.0000000 & 0.483871 & 0.516129 & 34 & 304 & -270 \\
 94 & 2^1 47^1 & \text{Y} & \text{N} & 5 & 0 & 1.0000000 & 0.489362 & 0.510638 & 39 & 309 & -270 \\
 95 & 5^1 19^1 & \text{Y} & \text{N} & 5 & 0 & 1.0000000 & 0.494737 & 0.505263 & 44 & 314 & -270 \\
 96 & 2^5 3^1 & \text{N} & \text{N} & 13 & 8 & 2.0769231 & 0.500000 & 0.500000 & 57 & 327 & -270 \\
 97 & 97^1 & \text{Y} & \text{Y} & -2 & 0 & 1.0000000 & 0.494845 & 0.505155 & 55 & 327 & -272 \\
 98 & 2^1 7^2 & \text{N} & \text{N} & -7 & 2 & 1.2857143 & 0.489796 & 0.510204 & 48 & 327 & -279 \\
 99 & 3^2 11^1 & \text{N} & \text{N} & -7 & 2 & 1.2857143 & 0.484848 & 0.515152 & 41 & 327 & -286 \\
 100 & 2^2 5^2 & \text{N} & \text{N} & 14 & 9 & 1.3571429 & 0.490000 & 0.510000 & 55 & 341 & -286 \\
 101 & 101^1 & \text{Y} & \text{Y} & -2 & 0 & 1.0000000 & 0.485149 & 0.514851 & 53 & 341 & -288 \\
 102 & 2^1 3^1 17^1 & \text{Y} & \text{N} & -16 & 0 & 1.0000000 & 0.480392 & 0.519608 & 37 & 341 & -304 \\
 103 & 103^1 & \text{Y} & \text{Y} & -2 & 0 & 1.0000000 & 0.475728 & 0.524272 & 35 & 341 & -306 \\
 104 & 2^3 13^1 & \text{N} & \text{N} & 9 & 4 & 1.5555556 & 0.480769 & 0.519231 & 44 & 350 & -306 \\
 105 & 3^1 5^1 7^1 & \text{Y} & \text{N} & -16 & 0 & 1.0000000 & 0.476190 & 0.523810 & 28 & 350 & -322 \\
 106 & 2^1 53^1 & \text{Y} & \text{N} & 5 & 0 & 1.0000000 & 0.481132 & 0.518868 & 33 & 355 & -322 \\
 107 & 107^1 & \text{Y} & \text{Y} & -2 & 0 & 1.0000000 & 0.476636 & 0.523364 & 31 & 355 & -324 \\
 108 & 2^2 3^3 & \text{N} & \text{N} & -23 & 18 & 1.4782609 & 0.472222 & 0.527778 & 8 & 355 & -347 \\
 109 & 109^1 & \text{Y} & \text{Y} & -2 & 0 & 1.0000000 & 0.467890 & 0.532110 & 6 & 355 & -349 \\
 110 & 2^1 5^1 11^1 & \text{Y} & \text{N} & -16 & 0 & 1.0000000 & 0.463636 & 0.536364 & -10 & 355 & -365 \\
 111 & 3^1 37^1 & \text{Y} & \text{N} & 5 & 0 & 1.0000000 & 0.468468 & 0.531532 & -5 & 360 & -365 \\
 112 & 2^4 7^1 & \text{N} & \text{N} & -11 & 6 & 1.8181818 & 0.464286 & 0.535714 & -16 & 360 & -376 \\
 113 & 113^1 & \text{Y} & \text{Y} & -2 & 0 & 1.0000000 & 0.460177 & 0.539823 & -18 & 360 & -378 \\
 114 & 2^1 3^1 19^1 & \text{Y} & \text{N} & -16 & 0 & 1.0000000 & 0.456140 & 0.543860 & -34 & 360 & -394 \\
 115 & 5^1 23^1 & \text{Y} & \text{N} & 5 & 0 & 1.0000000 & 0.460870 & 0.539130 & -29 & 365 & -394 \\
 116 & 2^2 29^1 & \text{N} & \text{N} & -7 & 2 & 1.2857143 & 0.456897 & 0.543103 & -36 & 365 & -401 \\
 117 & 3^2 13^1 & \text{N} & \text{N} & -7 & 2 & 1.2857143 & 0.452991 & 0.547009 & -43 & 365 & -408 \\
 118 & 2^1 59^1 & \text{Y} & \text{N} & 5 & 0 & 1.0000000 & 0.457627 & 0.542373 & -38 & 370 & -408 \\
 119 & 7^1 17^1 & \text{Y} & \text{N} & 5 & 0 & 1.0000000 & 0.462185 & 0.537815 & -33 & 375 & -408 \\
 120 & 2^3 3^1 5^1 & \text{N} & \text{N} & -48 & 32 & 1.3333333 & 0.458333 & 0.541667 & -81 & 375 & -456 \\
 121 & 11^2 & \text{N} & \text{Y} & 2 & 0 & 1.5000000 & 0.462810 & 0.537190 & -79 & 377 & -456 \\
 122 & 2^1 61^1 & \text{Y} & \text{N} & 5 & 0 & 1.0000000 & 0.467213 & 0.532787 & -74 & 382 & -456 \\
 123 & 3^1 41^1 & \text{Y} & \text{N} & 5 & 0 & 1.0000000 & 0.471545 & 0.528455 & -69 & 387 & -456 \\
 124 & 2^2 31^1 & \text{N} & \text{N} & -7 & 2 & 1.2857143 & 0.467742 & 0.532258 & -76 & 387 & -463 \\ 
\end{array}
}
\end{equation*}

\end{table} 


\newpage
\begin{table}[h!]

\centering

\tiny
\begin{equation*}
\boxed{
\begin{array}{cc|cc|ccc|cc|ccc}
 n & \mathbf{Primes} & \mathbf{Sqfree} & \mathbf{PPower} & g^{-1}(n) & 
 \lambda(n) g^{-1}(n) - \widehat{f}_1(n) & 
 \frac{\sum_{d|n} C_{\Omega(d)}(d)}{|g^{-1}(n)|} & 
 \mathcal{L}_{+}(n) & \mathcal{L}_{-}(n) & 
 G^{-1}(n) & G^{-1}_{+}(n) & G^{-1}_{-}(n) \\ \hline 
 125 & 5^3 & \text{N} & \text{Y} & -2 & 0 & 2.0000000 & 0.464000 & 0.536000 & -78 & 387 & -465 \\
 126 & 2^1 3^2 7^1 & \text{N} & \text{N} & 30 & 14 & 1.1666667 & 0.468254 & 0.531746 & -48 & 417 & -465 \\
 127 & 127^1 & \text{Y} & \text{Y} & -2 & 0 & 1.0000000 & 0.464567 & 0.535433 & -50 & 417 & -467 \\
 128 & 2^7 & \text{N} & \text{Y} & -2 & 0 & 4.0000000 & 0.460938 & 0.539062 & -52 & 417 & -469 \\
 129 & 3^1 43^1 & \text{Y} & \text{N} & 5 & 0 & 1.0000000 & 0.465116 & 0.534884 & -47 & 422 & -469 \\
 130 & 2^1 5^1 13^1 & \text{Y} & \text{N} & -16 & 0 & 1.0000000 & 0.461538 & 0.538462 & -63 & 422 & -485 \\
 131 & 131^1 & \text{Y} & \text{Y} & -2 & 0 & 1.0000000 & 0.458015 & 0.541985 & -65 & 422 & -487 \\
 132 & 2^2 3^1 11^1 & \text{N} & \text{N} & 30 & 14 & 1.1666667 & 0.462121 & 0.537879 & -35 & 452 & -487 \\
 133 & 7^1 19^1 & \text{Y} & \text{N} & 5 & 0 & 1.0000000 & 0.466165 & 0.533835 & -30 & 457 & -487 \\
 134 & 2^1 67^1 & \text{Y} & \text{N} & 5 & 0 & 1.0000000 & 0.470149 & 0.529851 & -25 & 462 & -487 \\
 135 & 3^3 5^1 & \text{N} & \text{N} & 9 & 4 & 1.5555556 & 0.474074 & 0.525926 & -16 & 471 & -487 \\
 136 & 2^3 17^1 & \text{N} & \text{N} & 9 & 4 & 1.5555556 & 0.477941 & 0.522059 & -7 & 480 & -487 \\
 137 & 137^1 & \text{Y} & \text{Y} & -2 & 0 & 1.0000000 & 0.474453 & 0.525547 & -9 & 480 & -489 \\
 138 & 2^1 3^1 23^1 & \text{Y} & \text{N} & -16 & 0 & 1.0000000 & 0.471014 & 0.528986 & -25 & 480 & -505 \\
 139 & 139^1 & \text{Y} & \text{Y} & -2 & 0 & 1.0000000 & 0.467626 & 0.532374 & -27 & 480 & -507 \\
 140 & 2^2 5^1 7^1 & \text{N} & \text{N} & 30 & 14 & 1.1666667 & 0.471429 & 0.528571 & 3 & 510 & -507 \\
 141 & 3^1 47^1 & \text{Y} & \text{N} & 5 & 0 & 1.0000000 & 0.475177 & 0.524823 & 8 & 515 & -507 \\
 142 & 2^1 71^1 & \text{Y} & \text{N} & 5 & 0 & 1.0000000 & 0.478873 & 0.521127 & 13 & 520 & -507 \\
 143 & 11^1 13^1 & \text{Y} & \text{N} & 5 & 0 & 1.0000000 & 0.482517 & 0.517483 & 18 & 525 & -507 \\
 144 & 2^4 3^2 & \text{N} & \text{N} & 34 & 29 & 1.6176471 & 0.486111 & 0.513889 & 52 & 559 & -507 \\
 145 & 5^1 29^1 & \text{Y} & \text{N} & 5 & 0 & 1.0000000 & 0.489655 & 0.510345 & 57 & 564 & -507 \\
 146 & 2^1 73^1 & \text{Y} & \text{N} & 5 & 0 & 1.0000000 & 0.493151 & 0.506849 & 62 & 569 & -507 \\
 147 & 3^1 7^2 & \text{N} & \text{N} & -7 & 2 & 1.2857143 & 0.489796 & 0.510204 & 55 & 569 & -514 \\
 148 & 2^2 37^1 & \text{N} & \text{N} & -7 & 2 & 1.2857143 & 0.486486 & 0.513514 & 48 & 569 & -521 \\
 149 & 149^1 & \text{Y} & \text{Y} & -2 & 0 & 1.0000000 & 0.483221 & 0.516779 & 46 & 569 & -523 \\
 150 & 2^1 3^1 5^2 & \text{N} & \text{N} & 30 & 14 & 1.1666667 & 0.486667 & 0.513333 & 76 & 599 & -523 \\
 151 & 151^1 & \text{Y} & \text{Y} & -2 & 0 & 1.0000000 & 0.483444 & 0.516556 & 74 & 599 & -525 \\
 152 & 2^3 19^1 & \text{N} & \text{N} & 9 & 4 & 1.5555556 & 0.486842 & 0.513158 & 83 & 608 & -525 \\
 153 & 3^2 17^1 & \text{N} & \text{N} & -7 & 2 & 1.2857143 & 0.483660 & 0.516340 & 76 & 608 & -532 \\
 154 & 2^1 7^1 11^1 & \text{Y} & \text{N} & -16 & 0 & 1.0000000 & 0.480519 & 0.519481 & 60 & 608 & -548 \\
 155 & 5^1 31^1 & \text{Y} & \text{N} & 5 & 0 & 1.0000000 & 0.483871 & 0.516129 & 65 & 613 & -548 \\
 156 & 2^2 3^1 13^1 & \text{N} & \text{N} & 30 & 14 & 1.1666667 & 0.487179 & 0.512821 & 95 & 643 & -548 \\
 157 & 157^1 & \text{Y} & \text{Y} & -2 & 0 & 1.0000000 & 0.484076 & 0.515924 & 93 & 643 & -550 \\
 158 & 2^1 79^1 & \text{Y} & \text{N} & 5 & 0 & 1.0000000 & 0.487342 & 0.512658 & 98 & 648 & -550 \\
 159 & 3^1 53^1 & \text{Y} & \text{N} & 5 & 0 & 1.0000000 & 0.490566 & 0.509434 & 103 & 653 & -550 \\
 160 & 2^5 5^1 & \text{N} & \text{N} & 13 & 8 & 2.0769231 & 0.493750 & 0.506250 & 116 & 666 & -550 \\
 161 & 7^1 23^1 & \text{Y} & \text{N} & 5 & 0 & 1.0000000 & 0.496894 & 0.503106 & 121 & 671 & -550 \\
 162 & 2^1 3^4 & \text{N} & \text{N} & -11 & 6 & 1.8181818 & 0.493827 & 0.506173 & 110 & 671 & -561 \\
 163 & 163^1 & \text{Y} & \text{Y} & -2 & 0 & 1.0000000 & 0.490798 & 0.509202 & 108 & 671 & -563 \\
 164 & 2^2 41^1 & \text{N} & \text{N} & -7 & 2 & 1.2857143 & 0.487805 & 0.512195 & 101 & 671 & -570 \\
 165 & 3^1 5^1 11^1 & \text{Y} & \text{N} & -16 & 0 & 1.0000000 & 0.484848 & 0.515152 & 85 & 671 & -586 \\
 166 & 2^1 83^1 & \text{Y} & \text{N} & 5 & 0 & 1.0000000 & 0.487952 & 0.512048 & 90 & 676 & -586 \\
 167 & 167^1 & \text{Y} & \text{Y} & -2 & 0 & 1.0000000 & 0.485030 & 0.514970 & 88 & 676 & -588 \\
 168 & 2^3 3^1 7^1 & \text{N} & \text{N} & -48 & 32 & 1.3333333 & 0.482143 & 0.517857 & 40 & 676 & -636 \\
 169 & 13^2 & \text{N} & \text{Y} & 2 & 0 & 1.5000000 & 0.485207 & 0.514793 & 42 & 678 & -636 \\
 170 & 2^1 5^1 17^1 & \text{Y} & \text{N} & -16 & 0 & 1.0000000 & 0.482353 & 0.517647 & 26 & 678 & -652 \\
 171 & 3^2 19^1 & \text{N} & \text{N} & -7 & 2 & 1.2857143 & 0.479532 & 0.520468 & 19 & 678 & -659 \\
 172 & 2^2 43^1 & \text{N} & \text{N} & -7 & 2 & 1.2857143 & 0.476744 & 0.523256 & 12 & 678 & -666 \\
 173 & 173^1 & \text{Y} & \text{Y} & -2 & 0 & 1.0000000 & 0.473988 & 0.526012 & 10 & 678 & -668 \\
 174 & 2^1 3^1 29^1 & \text{Y} & \text{N} & -16 & 0 & 1.0000000 & 0.471264 & 0.528736 & -6 & 678 & -684 \\
 175 & 5^2 7^1 & \text{N} & \text{N} & -7 & 2 & 1.2857143 & 0.468571 & 0.531429 & -13 & 678 & -691 \\
 176 & 2^4 11^1 & \text{N} & \text{N} & -11 & 6 & 1.8181818 & 0.465909 & 0.534091 & -24 & 678 & -702 \\
 177 & 3^1 59^1 & \text{Y} & \text{N} & 5 & 0 & 1.0000000 & 0.468927 & 0.531073 & -19 & 683 & -702 \\
 178 & 2^1 89^1 & \text{Y} & \text{N} & 5 & 0 & 1.0000000 & 0.471910 & 0.528090 & -14 & 688 & -702 \\
 179 & 179^1 & \text{Y} & \text{Y} & -2 & 0 & 1.0000000 & 0.469274 & 0.530726 & -16 & 688 & -704 \\
 180 & 2^2 3^2 5^1 & \text{N} & \text{N} & -74 & 58 & 1.2162162 & 0.466667 & 0.533333 & -90 & 688 & -778 \\
 181 & 181^1 & \text{Y} & \text{Y} & -2 & 0 & 1.0000000 & 0.464088 & 0.535912 & -92 & 688 & -780 \\
 182 & 2^1 7^1 13^1 & \text{Y} & \text{N} & -16 & 0 & 1.0000000 & 0.461538 & 0.538462 & -108 & 688 & -796 \\
 183 & 3^1 61^1 & \text{Y} & \text{N} & 5 & 0 & 1.0000000 & 0.464481 & 0.535519 & -103 & 693 & -796 \\
 184 & 2^3 23^1 & \text{N} & \text{N} & 9 & 4 & 1.5555556 & 0.467391 & 0.532609 & -94 & 702 & -796 \\
 185 & 5^1 37^1 & \text{Y} & \text{N} & 5 & 0 & 1.0000000 & 0.470270 & 0.529730 & -89 & 707 & -796 \\
 186 & 2^1 3^1 31^1 & \text{Y} & \text{N} & -16 & 0 & 1.0000000 & 0.467742 & 0.532258 & -105 & 707 & -812 \\
 187 & 11^1 17^1 & \text{Y} & \text{N} & 5 & 0 & 1.0000000 & 0.470588 & 0.529412 & -100 & 712 & -812 \\
 188 & 2^2 47^1 & \text{N} & \text{N} & -7 & 2 & 1.2857143 & 0.468085 & 0.531915 & -107 & 712 & -819 \\
 189 & 3^3 7^1 & \text{N} & \text{N} & 9 & 4 & 1.5555556 & 0.470899 & 0.529101 & -98 & 721 & -819 \\
 190 & 2^1 5^1 19^1 & \text{Y} & \text{N} & -16 & 0 & 1.0000000 & 0.468421 & 0.531579 & -114 & 721 & -835 \\
 191 & 191^1 & \text{Y} & \text{Y} & -2 & 0 & 1.0000000 & 0.465969 & 0.534031 & -116 & 721 & -837 \\
 192 & 2^6 3^1 & \text{N} & \text{N} & -15 & 10 & 2.3333333 & 0.463542 & 0.536458 & -131 & 721 & -852 \\
 193 & 193^1 & \text{Y} & \text{Y} & -2 & 0 & 1.0000000 & 0.461140 & 0.538860 & -133 & 721 & -854 \\
 194 & 2^1 97^1 & \text{Y} & \text{N} & 5 & 0 & 1.0000000 & 0.463918 & 0.536082 & -128 & 726 & -854 \\
 195 & 3^1 5^1 13^1 & \text{Y} & \text{N} & -16 & 0 & 1.0000000 & 0.461538 & 0.538462 & -144 & 726 & -870 \\
 196 & 2^2 7^2 & \text{N} & \text{N} & 14 & 9 & 1.3571429 & 0.464286 & 0.535714 & -130 & 740 & -870 \\
 197 & 197^1 & \text{Y} & \text{Y} & -2 & 0 & 1.0000000 & 0.461929 & 0.538071 & -132 & 740 & -872 \\
 198 & 2^1 3^2 11^1 & \text{N} & \text{N} & 30 & 14 & 1.1666667 & 0.464646 & 0.535354 & -102 & 770 & -872 \\
 199 & 199^1 & \text{Y} & \text{Y} & -2 & 0 & 1.0000000 & 0.462312 & 0.537688 & -104 & 770 & -874 \\
 200 & 2^3 5^2 & \text{N} & \text{N} & -23 & 18 & 1.4782609 & 0.460000 & 0.540000 & -127 & 770 & -897 \\ 
\end{array}
}
\end{equation*}

\end{table} 

\newpage
\begin{table}[h!]

\centering

\tiny
\begin{equation*}
\boxed{
\begin{array}{cc|cc|ccc|cc|ccc}
 n & \mathbf{Primes} & \mathbf{Sqfree} & \mathbf{PPower} & g^{-1}(n) & 
 \lambda(n) g^{-1}(n) - \widehat{f}_1(n) & 
 \frac{\sum_{d|n} C_{\Omega(d)}(d)}{|g^{-1}(n)|} & 
 \mathcal{L}_{+}(n) & \mathcal{L}_{-}(n) & 
 G^{-1}(n) & G^{-1}_{+}(n) & G^{-1}_{-}(n) \\ \hline 
 201 & 3^1 67^1 & \text{Y} & \text{N} & 5 & 0 & 1.0000000 & 0.462687 & 0.537313 & -122 & 775 & -897 \\
 202 & 2^1 101^1 & \text{Y} & \text{N} & 5 & 0 & 1.0000000 & 0.465347 & 0.534653 & -117 & 780 & -897 \\
 203 & 7^1 29^1 & \text{Y} & \text{N} & 5 & 0 & 1.0000000 & 0.467980 & 0.532020 & -112 & 785 & -897 \\
 204 & 2^2 3^1 17^1 & \text{N} & \text{N} & 30 & 14 & 1.1666667 & 0.470588 & 0.529412 & -82 & 815 & -897 \\
 205 & 5^1 41^1 & \text{Y} & \text{N} & 5 & 0 & 1.0000000 & 0.473171 & 0.526829 & -77 & 820 & -897 \\
 206 & 2^1 103^1 & \text{Y} & \text{N} & 5 & 0 & 1.0000000 & 0.475728 & 0.524272 & -72 & 825 & -897 \\
 207 & 3^2 23^1 & \text{N} & \text{N} & -7 & 2 & 1.2857143 & 0.473430 & 0.526570 & -79 & 825 & -904 \\
 208 & 2^4 13^1 & \text{N} & \text{N} & -11 & 6 & 1.8181818 & 0.471154 & 0.528846 & -90 & 825 & -915 \\
 209 & 11^1 19^1 & \text{Y} & \text{N} & 5 & 0 & 1.0000000 & 0.473684 & 0.526316 & -85 & 830 & -915 \\
 210 & 2^1 3^1 5^1 7^1 & \text{Y} & \text{N} & 65 & 0 & 1.0000000 & 0.476190 & 0.523810 & -20 & 895 & -915 \\
 211 & 211^1 & \text{Y} & \text{Y} & -2 & 0 & 1.0000000 & 0.473934 & 0.526066 & -22 & 895 & -917 \\
 212 & 2^2 53^1 & \text{N} & \text{N} & -7 & 2 & 1.2857143 & 0.471698 & 0.528302 & -29 & 895 & -924 \\
 213 & 3^1 71^1 & \text{Y} & \text{N} & 5 & 0 & 1.0000000 & 0.474178 & 0.525822 & -24 & 900 & -924 \\
 214 & 2^1 107^1 & \text{Y} & \text{N} & 5 & 0 & 1.0000000 & 0.476636 & 0.523364 & -19 & 905 & -924 \\
 215 & 5^1 43^1 & \text{Y} & \text{N} & 5 & 0 & 1.0000000 & 0.479070 & 0.520930 & -14 & 910 & -924 \\
 216 & 2^3 3^3 & \text{N} & \text{N} & 46 & 41 & 1.5000000 & 0.481481 & 0.518519 & 32 & 956 & -924 \\
 217 & 7^1 31^1 & \text{Y} & \text{N} & 5 & 0 & 1.0000000 & 0.483871 & 0.516129 & 37 & 961 & -924 \\
 218 & 2^1 109^1 & \text{Y} & \text{N} & 5 & 0 & 1.0000000 & 0.486239 & 0.513761 & 42 & 966 & -924 \\
 219 & 3^1 73^1 & \text{Y} & \text{N} & 5 & 0 & 1.0000000 & 0.488584 & 0.511416 & 47 & 971 & -924 \\
 220 & 2^2 5^1 11^1 & \text{N} & \text{N} & 30 & 14 & 1.1666667 & 0.490909 & 0.509091 & 77 & 1001 & -924 \\
 221 & 13^1 17^1 & \text{Y} & \text{N} & 5 & 0 & 1.0000000 & 0.493213 & 0.506787 & 82 & 1006 & -924 \\
 222 & 2^1 3^1 37^1 & \text{Y} & \text{N} & -16 & 0 & 1.0000000 & 0.490991 & 0.509009 & 66 & 1006 & -940 \\
 223 & 223^1 & \text{Y} & \text{Y} & -2 & 0 & 1.0000000 & 0.488789 & 0.511211 & 64 & 1006 & -942 \\
 224 & 2^5 7^1 & \text{N} & \text{N} & 13 & 8 & 2.0769231 & 0.491071 & 0.508929 & 77 & 1019 & -942 \\
 225 & 3^2 5^2 & \text{N} & \text{N} & 14 & 9 & 1.3571429 & 0.493333 & 0.506667 & 91 & 1033 & -942 \\
 226 & 2^1 113^1 & \text{Y} & \text{N} & 5 & 0 & 1.0000000 & 0.495575 & 0.504425 & 96 & 1038 & -942 \\
 227 & 227^1 & \text{Y} & \text{Y} & -2 & 0 & 1.0000000 & 0.493392 & 0.506608 & 94 & 1038 & -944 \\
 228 & 2^2 3^1 19^1 & \text{N} & \text{N} & 30 & 14 & 1.1666667 & 0.495614 & 0.504386 & 124 & 1068 & -944 \\
 229 & 229^1 & \text{Y} & \text{Y} & -2 & 0 & 1.0000000 & 0.493450 & 0.506550 & 122 & 1068 & -946 \\
 230 & 2^1 5^1 23^1 & \text{Y} & \text{N} & -16 & 0 & 1.0000000 & 0.491304 & 0.508696 & 106 & 1068 & -962 \\
 231 & 3^1 7^1 11^1 & \text{Y} & \text{N} & -16 & 0 & 1.0000000 & 0.489177 & 0.510823 & 90 & 1068 & -978 \\
 232 & 2^3 29^1 & \text{N} & \text{N} & 9 & 4 & 1.5555556 & 0.491379 & 0.508621 & 99 & 1077 & -978 \\
 233 & 233^1 & \text{Y} & \text{Y} & -2 & 0 & 1.0000000 & 0.489270 & 0.510730 & 97 & 1077 & -980 \\
 234 & 2^1 3^2 13^1 & \text{N} & \text{N} & 30 & 14 & 1.1666667 & 0.491453 & 0.508547 & 127 & 1107 & -980 \\
 235 & 5^1 47^1 & \text{Y} & \text{N} & 5 & 0 & 1.0000000 & 0.493617 & 0.506383 & 132 & 1112 & -980 \\
 236 & 2^2 59^1 & \text{N} & \text{N} & -7 & 2 & 1.2857143 & 0.491525 & 0.508475 & 125 & 1112 & -987 \\
 237 & 3^1 79^1 & \text{Y} & \text{N} & 5 & 0 & 1.0000000 & 0.493671 & 0.506329 & 130 & 1117 & -987 \\
 238 & 2^1 7^1 17^1 & \text{Y} & \text{N} & -16 & 0 & 1.0000000 & 0.491597 & 0.508403 & 114 & 1117 & -1003 \\
 239 & 239^1 & \text{Y} & \text{Y} & -2 & 0 & 1.0000000 & 0.489540 & 0.510460 & 112 & 1117 & -1005 \\
 240 & 2^4 3^1 5^1 & \text{N} & \text{N} & 70 & 54 & 1.5000000 & 0.491667 & 0.508333 & 182 & 1187 & -1005 \\
 241 & 241^1 & \text{Y} & \text{Y} & -2 & 0 & 1.0000000 & 0.489627 & 0.510373 & 180 & 1187 & -1007 \\
 242 & 2^1 11^2 & \text{N} & \text{N} & -7 & 2 & 1.2857143 & 0.487603 & 0.512397 & 173 & 1187 & -1014 \\
 243 & 3^5 & \text{N} & \text{Y} & -2 & 0 & 3.0000000 & 0.485597 & 0.514403 & 171 & 1187 & -1016 \\
 244 & 2^2 61^1 & \text{N} & \text{N} & -7 & 2 & 1.2857143 & 0.483607 & 0.516393 & 164 & 1187 & -1023 \\
 245 & 5^1 7^2 & \text{N} & \text{N} & -7 & 2 & 1.2857143 & 0.481633 & 0.518367 & 157 & 1187 & -1030 \\
 246 & 2^1 3^1 41^1 & \text{Y} & \text{N} & -16 & 0 & 1.0000000 & 0.479675 & 0.520325 & 141 & 1187 & -1046 \\
 247 & 13^1 19^1 & \text{Y} & \text{N} & 5 & 0 & 1.0000000 & 0.481781 & 0.518219 & 146 & 1192 & -1046 \\
 248 & 2^3 31^1 & \text{N} & \text{N} & 9 & 4 & 1.5555556 & 0.483871 & 0.516129 & 155 & 1201 & -1046 \\
 249 & 3^1 83^1 & \text{Y} & \text{N} & 5 & 0 & 1.0000000 & 0.485944 & 0.514056 & 160 & 1206 & -1046 \\
 250 & 2^1 5^3 & \text{N} & \text{N} & 9 & 4 & 1.5555556 & 0.488000 & 0.512000 & 169 & 1215 & -1046 \\
 251 & 251^1 & \text{Y} & \text{Y} & -2 & 0 & 1.0000000 & 0.486056 & 0.513944 & 167 & 1215 & -1048 \\
 252 & 2^2 3^2 7^1 & \text{N} & \text{N} & -74 & 58 & 1.2162162 & 0.484127 & 0.515873 & 93 & 1215 & -1122 \\
 253 & 11^1 23^1 & \text{Y} & \text{N} & 5 & 0 & 1.0000000 & 0.486166 & 0.513834 & 98 & 1220 & -1122 \\
 254 & 2^1 127^1 & \text{Y} & \text{N} & 5 & 0 & 1.0000000 & 0.488189 & 0.511811 & 103 & 1225 & -1122 \\
 255 & 3^1 5^1 17^1 & \text{Y} & \text{N} & -16 & 0 & 1.0000000 & 0.486275 & 0.513725 & 87 & 1225 & -1138 \\
 256 & 2^8 & \text{N} & \text{Y} & 2 & 0 & 4.5000000 & 0.488281 & 0.511719 & 89 & 1227 & -1138 \\
 257 & 257^1 & \text{Y} & \text{Y} & -2 & 0 & 1.0000000 & 0.486381 & 0.513619 & 87 & 1227 & -1140 \\
 258 & 2^1 3^1 43^1 & \text{Y} & \text{N} & -16 & 0 & 1.0000000 & 0.484496 & 0.515504 & 71 & 1227 & -1156 \\
 259 & 7^1 37^1 & \text{Y} & \text{N} & 5 & 0 & 1.0000000 & 0.486486 & 0.513514 & 76 & 1232 & -1156 \\
 260 & 2^2 5^1 13^1 & \text{N} & \text{N} & 30 & 14 & 1.1666667 & 0.488462 & 0.511538 & 106 & 1262 & -1156 \\
 261 & 3^2 29^1 & \text{N} & \text{N} & -7 & 2 & 1.2857143 & 0.486590 & 0.513410 & 99 & 1262 & -1163 \\
 262 & 2^1 131^1 & \text{Y} & \text{N} & 5 & 0 & 1.0000000 & 0.488550 & 0.511450 & 104 & 1267 & -1163 \\
 263 & 263^1 & \text{Y} & \text{Y} & -2 & 0 & 1.0000000 & 0.486692 & 0.513308 & 102 & 1267 & -1165 \\
 264 & 2^3 3^1 11^1 & \text{N} & \text{N} & -48 & 32 & 1.3333333 & 0.484848 & 0.515152 & 54 & 1267 & -1213 \\
 265 & 5^1 53^1 & \text{Y} & \text{N} & 5 & 0 & 1.0000000 & 0.486792 & 0.513208 & 59 & 1272 & -1213 \\
 266 & 2^1 7^1 19^1 & \text{Y} & \text{N} & -16 & 0 & 1.0000000 & 0.484962 & 0.515038 & 43 & 1272 & -1229 \\
 267 & 3^1 89^1 & \text{Y} & \text{N} & 5 & 0 & 1.0000000 & 0.486891 & 0.513109 & 48 & 1277 & -1229 \\
 268 & 2^2 67^1 & \text{N} & \text{N} & -7 & 2 & 1.2857143 & 0.485075 & 0.514925 & 41 & 1277 & -1236 \\
 269 & 269^1 & \text{Y} & \text{Y} & -2 & 0 & 1.0000000 & 0.483271 & 0.516729 & 39 & 1277 & -1238 \\
 270 & 2^1 3^3 5^1 & \text{N} & \text{N} & -48 & 32 & 1.3333333 & 0.481481 & 0.518519 & -9 & 1277 & -1286 \\
 271 & 271^1 & \text{Y} & \text{Y} & -2 & 0 & 1.0000000 & 0.479705 & 0.520295 & -11 & 1277 & -1288 \\
 272 & 2^4 17^1 & \text{N} & \text{N} & -11 & 6 & 1.8181818 & 0.477941 & 0.522059 & -22 & 1277 & -1299 \\
 273 & 3^1 7^1 13^1 & \text{Y} & \text{N} & -16 & 0 & 1.0000000 & 0.476190 & 0.523810 & -38 & 1277 & -1315 \\
 274 & 2^1 137^1 & \text{Y} & \text{N} & 5 & 0 & 1.0000000 & 0.478102 & 0.521898 & -33 & 1282 & -1315 \\
 275 & 5^2 11^1 & \text{N} & \text{N} & -7 & 2 & 1.2857143 & 0.476364 & 0.523636 & -40 & 1282 & -1322 \\
 276 & 2^2 3^1 23^1 & \text{N} & \text{N} & 30 & 14 & 1.1666667 & 0.478261 & 0.521739 & -10 & 1312 & -1322 \\
 277 & 277^1 & \text{Y} & \text{Y} & -2 & 0 & 1.0000000 & 0.476534 & 0.523466 & -12 & 1312 & -1324 \\ 
\end{array}
}
\end{equation*}

\end{table} 

\newpage
\begin{table}[h!]

\centering

\tiny
\begin{equation*}
\boxed{
\begin{array}{cc|cc|ccc|cc|ccc}
 n & \mathbf{Primes} & \mathbf{Sqfree} & \mathbf{PPower} & g^{-1}(n) & 
 \lambda(n) g^{-1}(n) - \widehat{f}_1(n) & 
 \frac{\sum_{d|n} C_{\Omega(d)}(d)}{|g^{-1}(n)|} & 
 \mathcal{L}_{+}(n) & \mathcal{L}_{-}(n) & 
 G^{-1}(n) & G^{-1}_{+}(n) & G^{-1}_{-}(n) \\ \hline 
 278 & 2^1 139^1 & \text{Y} & \text{N} & 5 & 0 & 1.0000000 & 0.478417 & 0.521583 & -7 & 1317 & -1324 \\
 279 & 3^2 31^1 & \text{N} & \text{N} & -7 & 2 & 1.2857143 & 0.476703 & 0.523297 & -14 & 1317 & -1331 \\
 280 & 2^3 5^1 7^1 & \text{N} & \text{N} & -48 & 32 & 1.3333333 & 0.475000 & 0.525000 & -62 & 1317 & -1379 \\
 281 & 281^1 & \text{Y} & \text{Y} & -2 & 0 & 1.0000000 & 0.473310 & 0.526690 & -64 & 1317 & -1381 \\
 282 & 2^1 3^1 47^1 & \text{Y} & \text{N} & -16 & 0 & 1.0000000 & 0.471631 & 0.528369 & -80 & 1317 & -1397 \\
 283 & 283^1 & \text{Y} & \text{Y} & -2 & 0 & 1.0000000 & 0.469965 & 0.530035 & -82 & 1317 & -1399 \\
 284 & 2^2 71^1 & \text{N} & \text{N} & -7 & 2 & 1.2857143 & 0.468310 & 0.531690 & -89 & 1317 & -1406 \\
 285 & 3^1 5^1 19^1 & \text{Y} & \text{N} & -16 & 0 & 1.0000000 & 0.466667 & 0.533333 & -105 & 1317 & -1422 \\
 286 & 2^1 11^1 13^1 & \text{Y} & \text{N} & -16 & 0 & 1.0000000 & 0.465035 & 0.534965 & -121 & 1317 & -1438 \\
 287 & 7^1 41^1 & \text{Y} & \text{N} & 5 & 0 & 1.0000000 & 0.466899 & 0.533101 & -116 & 1322 & -1438 \\
 288 & 2^5 3^2 & \text{N} & \text{N} & -47 & 42 & 1.7659574 & 0.465278 & 0.534722 & -163 & 1322 & -1485 \\
 289 & 17^2 & \text{N} & \text{Y} & 2 & 0 & 1.5000000 & 0.467128 & 0.532872 & -161 & 1324 & -1485 \\
 290 & 2^1 5^1 29^1 & \text{Y} & \text{N} & -16 & 0 & 1.0000000 & 0.465517 & 0.534483 & -177 & 1324 & -1501 \\
 291 & 3^1 97^1 & \text{Y} & \text{N} & 5 & 0 & 1.0000000 & 0.467354 & 0.532646 & -172 & 1329 & -1501 \\
 292 & 2^2 73^1 & \text{N} & \text{N} & -7 & 2 & 1.2857143 & 0.465753 & 0.534247 & -179 & 1329 & -1508 \\
 293 & 293^1 & \text{Y} & \text{Y} & -2 & 0 & 1.0000000 & 0.464164 & 0.535836 & -181 & 1329 & -1510 \\
 294 & 2^1 3^1 7^2 & \text{N} & \text{N} & 30 & 14 & 1.1666667 & 0.465986 & 0.534014 & -151 & 1359 & -1510 \\
 295 & 5^1 59^1 & \text{Y} & \text{N} & 5 & 0 & 1.0000000 & 0.467797 & 0.532203 & -146 & 1364 & -1510 \\
 296 & 2^3 37^1 & \text{N} & \text{N} & 9 & 4 & 1.5555556 & 0.469595 & 0.530405 & -137 & 1373 & -1510 \\
 297 & 3^3 11^1 & \text{N} & \text{N} & 9 & 4 & 1.5555556 & 0.471380 & 0.528620 & -128 & 1382 & -1510 \\
 298 & 2^1 149^1 & \text{Y} & \text{N} & 5 & 0 & 1.0000000 & 0.473154 & 0.526846 & -123 & 1387 & -1510 \\
 299 & 13^1 23^1 & \text{Y} & \text{N} & 5 & 0 & 1.0000000 & 0.474916 & 0.525084 & -118 & 1392 & -1510 \\
 300 & 2^2 3^1 5^2 & \text{N} & \text{N} & -74 & 58 & 1.2162162 & 0.473333 & 0.526667 & -192 & 1392 & -1584 \\
 301 & 7^1 43^1 & \text{Y} & \text{N} & 5 & 0 & 1.0000000 & 0.475083 & 0.524917 & -187 & 1397 & -1584 \\
 302 & 2^1 151^1 & \text{Y} & \text{N} & 5 & 0 & 1.0000000 & 0.476821 & 0.523179 & -182 & 1402 & -1584 \\
 303 & 3^1 101^1 & \text{Y} & \text{N} & 5 & 0 & 1.0000000 & 0.478548 & 0.521452 & -177 & 1407 & -1584 \\
 304 & 2^4 19^1 & \text{N} & \text{N} & -11 & 6 & 1.8181818 & 0.476974 & 0.523026 & -188 & 1407 & -1595 \\
 305 & 5^1 61^1 & \text{Y} & \text{N} & 5 & 0 & 1.0000000 & 0.478689 & 0.521311 & -183 & 1412 & -1595 \\
 306 & 2^1 3^2 17^1 & \text{N} & \text{N} & 30 & 14 & 1.1666667 & 0.480392 & 0.519608 & -153 & 1442 & -1595 \\
 307 & 307^1 & \text{Y} & \text{Y} & -2 & 0 & 1.0000000 & 0.478827 & 0.521173 & -155 & 1442 & -1597 \\
 308 & 2^2 7^1 11^1 & \text{N} & \text{N} & 30 & 14 & 1.1666667 & 0.480519 & 0.519481 & -125 & 1472 & -1597 \\
 309 & 3^1 103^1 & \text{Y} & \text{N} & 5 & 0 & 1.0000000 & 0.482201 & 0.517799 & -120 & 1477 & -1597 \\
 310 & 2^1 5^1 31^1 & \text{Y} & \text{N} & -16 & 0 & 1.0000000 & 0.480645 & 0.519355 & -136 & 1477 & -1613 \\
 311 & 311^1 & \text{Y} & \text{Y} & -2 & 0 & 1.0000000 & 0.479100 & 0.520900 & -138 & 1477 & -1615 \\
 312 & 2^3 3^1 13^1 & \text{N} & \text{N} & -48 & 32 & 1.3333333 & 0.477564 & 0.522436 & -186 & 1477 & -1663 \\
 313 & 313^1 & \text{Y} & \text{Y} & -2 & 0 & 1.0000000 & 0.476038 & 0.523962 & -188 & 1477 & -1665 \\
 314 & 2^1 157^1 & \text{Y} & \text{N} & 5 & 0 & 1.0000000 & 0.477707 & 0.522293 & -183 & 1482 & -1665 \\
 315 & 3^2 5^1 7^1 & \text{N} & \text{N} & 30 & 14 & 1.1666667 & 0.479365 & 0.520635 & -153 & 1512 & -1665 \\
 316 & 2^2 79^1 & \text{N} & \text{N} & -7 & 2 & 1.2857143 & 0.477848 & 0.522152 & -160 & 1512 & -1672 \\
 317 & 317^1 & \text{Y} & \text{Y} & -2 & 0 & 1.0000000 & 0.476341 & 0.523659 & -162 & 1512 & -1674 \\
 318 & 2^1 3^1 53^1 & \text{Y} & \text{N} & -16 & 0 & 1.0000000 & 0.474843 & 0.525157 & -178 & 1512 & -1690 \\
 319 & 11^1 29^1 & \text{Y} & \text{N} & 5 & 0 & 1.0000000 & 0.476489 & 0.523511 & -173 & 1517 & -1690 \\
 320 & 2^6 5^1 & \text{N} & \text{N} & -15 & 10 & 2.3333333 & 0.475000 & 0.525000 & -188 & 1517 & -1705 \\
 321 & 3^1 107^1 & \text{Y} & \text{N} & 5 & 0 & 1.0000000 & 0.476636 & 0.523364 & -183 & 1522 & -1705 \\
 322 & 2^1 7^1 23^1 & \text{Y} & \text{N} & -16 & 0 & 1.0000000 & 0.475155 & 0.524845 & -199 & 1522 & -1721 \\
 323 & 17^1 19^1 & \text{Y} & \text{N} & 5 & 0 & 1.0000000 & 0.476780 & 0.523220 & -194 & 1527 & -1721 \\
 324 & 2^2 3^4 & \text{N} & \text{N} & 34 & 29 & 1.6176471 & 0.478395 & 0.521605 & -160 & 1561 & -1721 \\
 325 & 5^2 13^1 & \text{N} & \text{N} & -7 & 2 & 1.2857143 & 0.476923 & 0.523077 & -167 & 1561 & -1728 \\
 326 & 2^1 163^1 & \text{Y} & \text{N} & 5 & 0 & 1.0000000 & 0.478528 & 0.521472 & -162 & 1566 & -1728 \\
 327 & 3^1 109^1 & \text{Y} & \text{N} & 5 & 0 & 1.0000000 & 0.480122 & 0.519878 & -157 & 1571 & -1728 \\
 328 & 2^3 41^1 & \text{N} & \text{N} & 9 & 4 & 1.5555556 & 0.481707 & 0.518293 & -148 & 1580 & -1728 \\
 329 & 7^1 47^1 & \text{Y} & \text{N} & 5 & 0 & 1.0000000 & 0.483283 & 0.516717 & -143 & 1585 & -1728 \\
 330 & 2^1 3^1 5^1 11^1 & \text{Y} & \text{N} & 65 & 0 & 1.0000000 & 0.484848 & 0.515152 & -78 & 1650 & -1728 \\
 331 & 331^1 & \text{Y} & \text{Y} & -2 & 0 & 1.0000000 & 0.483384 & 0.516616 & -80 & 1650 & -1730 \\
 332 & 2^2 83^1 & \text{N} & \text{N} & -7 & 2 & 1.2857143 & 0.481928 & 0.518072 & -87 & 1650 & -1737 \\
 333 & 3^2 37^1 & \text{N} & \text{N} & -7 & 2 & 1.2857143 & 0.480480 & 0.519520 & -94 & 1650 & -1744 \\
 334 & 2^1 167^1 & \text{Y} & \text{N} & 5 & 0 & 1.0000000 & 0.482036 & 0.517964 & -89 & 1655 & -1744 \\
 335 & 5^1 67^1 & \text{Y} & \text{N} & 5 & 0 & 1.0000000 & 0.483582 & 0.516418 & -84 & 1660 & -1744 \\
 336 & 2^4 3^1 7^1 & \text{N} & \text{N} & 70 & 54 & 1.5000000 & 0.485119 & 0.514881 & -14 & 1730 & -1744 \\
 337 & 337^1 & \text{Y} & \text{Y} & -2 & 0 & 1.0000000 & 0.483680 & 0.516320 & -16 & 1730 & -1746 \\
 338 & 2^1 13^2 & \text{N} & \text{N} & -7 & 2 & 1.2857143 & 0.482249 & 0.517751 & -23 & 1730 & -1753 \\
 339 & 3^1 113^1 & \text{Y} & \text{N} & 5 & 0 & 1.0000000 & 0.483776 & 0.516224 & -18 & 1735 & -1753 \\
 340 & 2^2 5^1 17^1 & \text{N} & \text{N} & 30 & 14 & 1.1666667 & 0.485294 & 0.514706 & 12 & 1765 & -1753 \\
 341 & 11^1 31^1 & \text{Y} & \text{N} & 5 & 0 & 1.0000000 & 0.486804 & 0.513196 & 17 & 1770 & -1753 \\
 342 & 2^1 3^2 19^1 & \text{N} & \text{N} & 30 & 14 & 1.1666667 & 0.488304 & 0.511696 & 47 & 1800 & -1753 \\
 343 & 7^3 & \text{N} & \text{Y} & -2 & 0 & 2.0000000 & 0.486880 & 0.513120 & 45 & 1800 & -1755 \\
 344 & 2^3 43^1 & \text{N} & \text{N} & 9 & 4 & 1.5555556 & 0.488372 & 0.511628 & 54 & 1809 & -1755 \\
 345 & 3^1 5^1 23^1 & \text{Y} & \text{N} & -16 & 0 & 1.0000000 & 0.486957 & 0.513043 & 38 & 1809 & -1771 \\
 346 & 2^1 173^1 & \text{Y} & \text{N} & 5 & 0 & 1.0000000 & 0.488439 & 0.511561 & 43 & 1814 & -1771 \\
 347 & 347^1 & \text{Y} & \text{Y} & -2 & 0 & 1.0000000 & 0.487032 & 0.512968 & 41 & 1814 & -1773 \\
 348 & 2^2 3^1 29^1 & \text{N} & \text{N} & 30 & 14 & 1.1666667 & 0.488506 & 0.511494 & 71 & 1844 & -1773 \\
 349 & 349^1 & \text{Y} & \text{Y} & -2 & 0 & 1.0000000 & 0.487106 & 0.512894 & 69 & 1844 & -1775 \\
 350 & 2^1 5^2 7^1 & \text{N} & \text{N} & 30 & 14 & 1.1666667 & 0.488571 & 0.511429 & 99 & 1874 & -1775 \\ 
\end{array}
}
\end{equation*}

\end{table} 

\newpage
\begin{table}[h!]

\centering
\tiny
\begin{equation*}
\boxed{
\begin{array}{cc|cc|ccc|cc|ccc}
 n & \mathbf{Primes} & \mathbf{Sqfree} & \mathbf{PPower} & g^{-1}(n) & 
 \lambda(n) g^{-1}(n) - \widehat{f}_1(n) & 
 \frac{\sum_{d|n} C_{\Omega(d)}(d)}{|g^{-1}(n)|} & 
 \mathcal{L}_{+}(n) & \mathcal{L}_{-}(n) & 
 G^{-1}(n) & G^{-1}_{+}(n) & G^{-1}_{-}(n) \\ \hline 
 351 & 3^3 13^1 & \text{N} & \text{N} & 9 & 4 & 1.5555556 & 0.490028 & 0.509972 & 108 & 1883 & -1775 \\
 352 & 2^5 11^1 & \text{N} & \text{N} & 13 & 8 & 2.0769231 & 0.491477 & 0.508523 & 121 & 1896 & -1775 \\
 353 & 353^1 & \text{Y} & \text{Y} & -2 & 0 & 1.0000000 & 0.490085 & 0.509915 & 119 & 1896 & -1777 \\
 354 & 2^1 3^1 59^1 & \text{Y} & \text{N} & -16 & 0 & 1.0000000 & 0.488701 & 0.511299 & 103 & 1896 & -1793 \\
 355 & 5^1 71^1 & \text{Y} & \text{N} & 5 & 0 & 1.0000000 & 0.490141 & 0.509859 & 108 & 1901 & -1793 \\
 356 & 2^2 89^1 & \text{N} & \text{N} & -7 & 2 & 1.2857143 & 0.488764 & 0.511236 & 101 & 1901 & -1800 \\
 357 & 3^1 7^1 17^1 & \text{Y} & \text{N} & -16 & 0 & 1.0000000 & 0.487395 & 0.512605 & 85 & 1901 & -1816 \\
 358 & 2^1 179^1 & \text{Y} & \text{N} & 5 & 0 & 1.0000000 & 0.488827 & 0.511173 & 90 & 1906 & -1816 \\
 359 & 359^1 & \text{Y} & \text{Y} & -2 & 0 & 1.0000000 & 0.487465 & 0.512535 & 88 & 1906 & -1818 \\
 360 & 2^3 3^2 5^1 & \text{N} & \text{N} & 145 & 129 & 1.3034483 & 0.488889 & 0.511111 & 233 & 2051 & -1818 \\
 361 & 19^2 & \text{N} & \text{Y} & 2 & 0 & 1.5000000 & 0.490305 & 0.509695 & 235 & 2053 & -1818 \\
 362 & 2^1 181^1 & \text{Y} & \text{N} & 5 & 0 & 1.0000000 & 0.491713 & 0.508287 & 240 & 2058 & -1818 \\
 363 & 3^1 11^2 & \text{N} & \text{N} & -7 & 2 & 1.2857143 & 0.490358 & 0.509642 & 233 & 2058 & -1825 \\
 364 & 2^2 7^1 13^1 & \text{N} & \text{N} & 30 & 14 & 1.1666667 & 0.491758 & 0.508242 & 263 & 2088 & -1825 \\
 365 & 5^1 73^1 & \text{Y} & \text{N} & 5 & 0 & 1.0000000 & 0.493151 & 0.506849 & 268 & 2093 & -1825 \\
 366 & 2^1 3^1 61^1 & \text{Y} & \text{N} & -16 & 0 & 1.0000000 & 0.491803 & 0.508197 & 252 & 2093 & -1841 \\
 367 & 367^1 & \text{Y} & \text{Y} & -2 & 0 & 1.0000000 & 0.490463 & 0.509537 & 250 & 2093 & -1843 \\
 368 & 2^4 23^1 & \text{N} & \text{N} & -11 & 6 & 1.8181818 & 0.489130 & 0.510870 & 239 & 2093 & -1854 \\
 369 & 3^2 41^1 & \text{N} & \text{N} & -7 & 2 & 1.2857143 & 0.487805 & 0.512195 & 232 & 2093 & -1861 \\
 370 & 2^1 5^1 37^1 & \text{Y} & \text{N} & -16 & 0 & 1.0000000 & 0.486486 & 0.513514 & 216 & 2093 & -1877 \\
 371 & 7^1 53^1 & \text{Y} & \text{N} & 5 & 0 & 1.0000000 & 0.487871 & 0.512129 & 221 & 2098 & -1877 \\
 372 & 2^2 3^1 31^1 & \text{N} & \text{N} & 30 & 14 & 1.1666667 & 0.489247 & 0.510753 & 251 & 2128 & -1877 \\
 373 & 373^1 & \text{Y} & \text{Y} & -2 & 0 & 1.0000000 & 0.487936 & 0.512064 & 249 & 2128 & -1879 \\
 374 & 2^1 11^1 17^1 & \text{Y} & \text{N} & -16 & 0 & 1.0000000 & 0.486631 & 0.513369 & 233 & 2128 & -1895 \\
 375 & 3^1 5^3 & \text{N} & \text{N} & 9 & 4 & 1.5555556 & 0.488000 & 0.512000 & 242 & 2137 & -1895 \\
 376 & 2^3 47^1 & \text{N} & \text{N} & 9 & 4 & 1.5555556 & 0.489362 & 0.510638 & 251 & 2146 & -1895 \\
 377 & 13^1 29^1 & \text{Y} & \text{N} & 5 & 0 & 1.0000000 & 0.490716 & 0.509284 & 256 & 2151 & -1895 \\
 378 & 2^1 3^3 7^1 & \text{N} & \text{N} & -48 & 32 & 1.3333333 & 0.489418 & 0.510582 & 208 & 2151 & -1943 \\
 379 & 379^1 & \text{Y} & \text{Y} & -2 & 0 & 1.0000000 & 0.488127 & 0.511873 & 206 & 2151 & -1945 \\
 380 & 2^2 5^1 19^1 & \text{N} & \text{N} & 30 & 14 & 1.1666667 & 0.489474 & 0.510526 & 236 & 2181 & -1945 \\
 381 & 3^1 127^1 & \text{Y} & \text{N} & 5 & 0 & 1.0000000 & 0.490814 & 0.509186 & 241 & 2186 & -1945 \\
 382 & 2^1 191^1 & \text{Y} & \text{N} & 5 & 0 & 1.0000000 & 0.492147 & 0.507853 & 246 & 2191 & -1945 \\
 383 & 383^1 & \text{Y} & \text{Y} & -2 & 0 & 1.0000000 & 0.490862 & 0.509138 & 244 & 2191 & -1947 \\
 384 & 2^7 3^1 & \text{N} & \text{N} & 17 & 12 & 2.5882353 & 0.492188 & 0.507812 & 261 & 2208 & -1947 \\
 385 & 5^1 7^1 11^1 & \text{Y} & \text{N} & -16 & 0 & 1.0000000 & 0.490909 & 0.509091 & 245 & 2208 & -1963 \\
 386 & 2^1 193^1 & \text{Y} & \text{N} & 5 & 0 & 1.0000000 & 0.492228 & 0.507772 & 250 & 2213 & -1963 \\
 387 & 3^2 43^1 & \text{N} & \text{N} & -7 & 2 & 1.2857143 & 0.490956 & 0.509044 & 243 & 2213 & -1970 \\
 388 & 2^2 97^1 & \text{N} & \text{N} & -7 & 2 & 1.2857143 & 0.489691 & 0.510309 & 236 & 2213 & -1977 \\
 389 & 389^1 & \text{Y} & \text{Y} & -2 & 0 & 1.0000000 & 0.488432 & 0.511568 & 234 & 2213 & -1979 \\
 390 & 2^1 3^1 5^1 13^1 & \text{Y} & \text{N} & 65 & 0 & 1.0000000 & 0.489744 & 0.510256 & 299 & 2278 & -1979 \\
 391 & 17^1 23^1 & \text{Y} & \text{N} & 5 & 0 & 1.0000000 & 0.491049 & 0.508951 & 304 & 2283 & -1979 \\
 392 & 2^3 7^2 & \text{N} & \text{N} & -23 & 18 & 1.4782609 & 0.489796 & 0.510204 & 281 & 2283 & -2002 \\
 393 & 3^1 131^1 & \text{Y} & \text{N} & 5 & 0 & 1.0000000 & 0.491094 & 0.508906 & 286 & 2288 & -2002 \\
 394 & 2^1 197^1 & \text{Y} & \text{N} & 5 & 0 & 1.0000000 & 0.492386 & 0.507614 & 291 & 2293 & -2002 \\
 395 & 5^1 79^1 & \text{Y} & \text{N} & 5 & 0 & 1.0000000 & 0.493671 & 0.506329 & 296 & 2298 & -2002 \\
 396 & 2^2 3^2 11^1 & \text{N} & \text{N} & -74 & 58 & 1.2162162 & 0.492424 & 0.507576 & 222 & 2298 & -2076 \\
 397 & 397^1 & \text{Y} & \text{Y} & -2 & 0 & 1.0000000 & 0.491184 & 0.508816 & 220 & 2298 & -2078 \\
 398 & 2^1 199^1 & \text{Y} & \text{N} & 5 & 0 & 1.0000000 & 0.492462 & 0.507538 & 225 & 2303 & -2078 \\
 399 & 3^1 7^1 19^1 & \text{Y} & \text{N} & -16 & 0 & 1.0000000 & 0.491228 & 0.508772 & 209 & 2303 & -2094 \\
 400 & 2^4 5^2 & \text{N} & \text{N} & 34 & 29 & 1.6176471 & 0.492500 & 0.507500 & 243 & 2337 & -2094 \\
 401 & 401^1 & \text{Y} & \text{Y} & -2 & 0 & 1.0000000 & 0.491272 & 0.508728 & 241 & 2337 & -2096 \\
 402 & 2^1 3^1 67^1 & \text{Y} & \text{N} & -16 & 0 & 1.0000000 & 0.490050 & 0.509950 & 225 & 2337 & -2112 \\
 403 & 13^1 31^1 & \text{Y} & \text{N} & 5 & 0 & 1.0000000 & 0.491315 & 0.508685 & 230 & 2342 & -2112 \\
 404 & 2^2 101^1 & \text{N} & \text{N} & -7 & 2 & 1.2857143 & 0.490099 & 0.509901 & 223 & 2342 & -2119 \\
 405 & 3^4 5^1 & \text{N} & \text{N} & -11 & 6 & 1.8181818 & 0.488889 & 0.511111 & 212 & 2342 & -2130 \\
 406 & 2^1 7^1 29^1 & \text{Y} & \text{N} & -16 & 0 & 1.0000000 & 0.487685 & 0.512315 & 196 & 2342 & -2146 \\
 407 & 11^1 37^1 & \text{Y} & \text{N} & 5 & 0 & 1.0000000 & 0.488943 & 0.511057 & 201 & 2347 & -2146 \\
 408 & 2^3 3^1 17^1 & \text{N} & \text{N} & -48 & 32 & 1.3333333 & 0.487745 & 0.512255 & 153 & 2347 & -2194 \\
 409 & 409^1 & \text{Y} & \text{Y} & -2 & 0 & 1.0000000 & 0.486553 & 0.513447 & 151 & 2347 & -2196 \\
 410 & 2^1 5^1 41^1 & \text{Y} & \text{N} & -16 & 0 & 1.0000000 & 0.485366 & 0.514634 & 135 & 2347 & -2212 \\
 411 & 3^1 137^1 & \text{Y} & \text{N} & 5 & 0 & 1.0000000 & 0.486618 & 0.513382 & 140 & 2352 & -2212 \\
 412 & 2^2 103^1 & \text{N} & \text{N} & -7 & 2 & 1.2857143 & 0.485437 & 0.514563 & 133 & 2352 & -2219 \\
 413 & 7^1 59^1 & \text{Y} & \text{N} & 5 & 0 & 1.0000000 & 0.486683 & 0.513317 & 138 & 2357 & -2219 \\
 414 & 2^1 3^2 23^1 & \text{N} & \text{N} & 30 & 14 & 1.1666667 & 0.487923 & 0.512077 & 168 & 2387 & -2219 \\
 415 & 5^1 83^1 & \text{Y} & \text{N} & 5 & 0 & 1.0000000 & 0.489157 & 0.510843 & 173 & 2392 & -2219 \\
 416 & 2^5 13^1 & \text{N} & \text{N} & 13 & 8 & 2.0769231 & 0.490385 & 0.509615 & 186 & 2405 & -2219 \\
 417 & 3^1 139^1 & \text{Y} & \text{N} & 5 & 0 & 1.0000000 & 0.491607 & 0.508393 & 191 & 2410 & -2219 \\
 418 & 2^1 11^1 19^1 & \text{Y} & \text{N} & -16 & 0 & 1.0000000 & 0.490431 & 0.509569 & 175 & 2410 & -2235 \\
 419 & 419^1 & \text{Y} & \text{Y} & -2 & 0 & 1.0000000 & 0.489260 & 0.510740 & 173 & 2410 & -2237 \\
 420 & 2^2 3^1 5^1 7^1 & \text{N} & \text{N} & -155 & 90 & 1.1032258 & 0.488095 & 0.511905 & 18 & 2410 & -2392 \\
 421 & 421^1 & \text{Y} & \text{Y} & -2 & 0 & 1.0000000 & 0.486936 & 0.513064 & 16 & 2410 & -2394 \\
 422 & 2^1 211^1 & \text{Y} & \text{N} & 5 & 0 & 1.0000000 & 0.488152 & 0.511848 & 21 & 2415 & -2394 \\
 423 & 3^2 47^1 & \text{N} & \text{N} & -7 & 2 & 1.2857143 & 0.486998 & 0.513002 & 14 & 2415 & -2401 \\
 424 & 2^3 53^1 & \text{N} & \text{N} & 9 & 4 & 1.5555556 & 0.488208 & 0.511792 & 23 & 2424 & -2401 \\
 425 & 5^2 17^1 & \text{N} & \text{N} & -7 & 2 & 1.2857143 & 0.487059 & 0.512941 & 16 & 2424 & -2408 \\ 
\end{array}
}
\end{equation*}

\end{table} 

\newpage

\begin{table}[h!]
\label{table_conjecture_Mertens_ginvSeq_approx_values_LastPage} 

\centering
\tiny
\begin{equation*}
\boxed{
\begin{array}{cc|cc|ccc|cc|ccc}
 n & \mathbf{Primes} & \mathbf{Sqfree} & \mathbf{PPower} & g^{-1}(n) & 
 \lambda(n) g^{-1}(n) - \widehat{f}_1(n) & 
 \frac{\sum_{d|n} C_{\Omega(d)}(d)}{|g^{-1}(n)|} & 
 \mathcal{L}_{+}(n) & \mathcal{L}_{-}(n) & 
 G^{-1}(n) & G^{-1}_{+}(n) & G^{-1}_{-}(n) \\ \hline 
 426 & 2^1 3^1 71^1 & \text{Y} & \text{N} & -16 & 0 & 1.0000000 & 0.485915 & 0.514085 & 0 & 2424 & -2424 \\
 427 & 7^1 61^1 & \text{Y} & \text{N} & 5 & 0 & 1.0000000 & 0.487119 & 0.512881 & 5 & 2429 & -2424 \\
 428 & 2^2 107^1 & \text{N} & \text{N} & -7 & 2 & 1.2857143 & 0.485981 & 0.514019 & -2 & 2429 & -2431 \\
 429 & 3^1 11^1 13^1 & \text{Y} & \text{N} & -16 & 0 & 1.0000000 & 0.484848 & 0.515152 & -18 & 2429 & -2447 \\
 430 & 2^1 5^1 43^1 & \text{Y} & \text{N} & -16 & 0 & 1.0000000 & 0.483721 & 0.516279 & -34 & 2429 & -2463 \\
 431 & 431^1 & \text{Y} & \text{Y} & -2 & 0 & 1.0000000 & 0.482599 & 0.517401 & -36 & 2429 & -2465 \\
 432 & 2^4 3^3 & \text{N} & \text{N} & -80 & 75 & 1.5625000 & 0.481481 & 0.518519 & -116 & 2429 & -2545 \\
 433 & 433^1 & \text{Y} & \text{Y} & -2 & 0 & 1.0000000 & 0.480370 & 0.519630 & -118 & 2429 & -2547 \\
 434 & 2^1 7^1 31^1 & \text{Y} & \text{N} & -16 & 0 & 1.0000000 & 0.479263 & 0.520737 & -134 & 2429 & -2563 \\
 435 & 3^1 5^1 29^1 & \text{Y} & \text{N} & -16 & 0 & 1.0000000 & 0.478161 & 0.521839 & -150 & 2429 & -2579 \\
 436 & 2^2 109^1 & \text{N} & \text{N} & -7 & 2 & 1.2857143 & 0.477064 & 0.522936 & -157 & 2429 & -2586 \\
 437 & 19^1 23^1 & \text{Y} & \text{N} & 5 & 0 & 1.0000000 & 0.478261 & 0.521739 & -152 & 2434 & -2586 \\
 438 & 2^1 3^1 73^1 & \text{Y} & \text{N} & -16 & 0 & 1.0000000 & 0.477169 & 0.522831 & -168 & 2434 & -2602 \\
 439 & 439^1 & \text{Y} & \text{Y} & -2 & 0 & 1.0000000 & 0.476082 & 0.523918 & -170 & 2434 & -2604 \\
 440 & 2^3 5^1 11^1 & \text{N} & \text{N} & -48 & 32 & 1.3333333 & 0.475000 & 0.525000 & -218 & 2434 & -2652 \\
 441 & 3^2 7^2 & \text{N} & \text{N} & 14 & 9 & 1.3571429 & 0.476190 & 0.523810 & -204 & 2448 & -2652 \\
 442 & 2^1 13^1 17^1 & \text{Y} & \text{N} & -16 & 0 & 1.0000000 & 0.475113 & 0.524887 & -220 & 2448 & -2668 \\
 443 & 443^1 & \text{Y} & \text{Y} & -2 & 0 & 1.0000000 & 0.474041 & 0.525959 & -222 & 2448 & -2670 \\
 444 & 2^2 3^1 37^1 & \text{N} & \text{N} & 30 & 14 & 1.1666667 & 0.475225 & 0.524775 & -192 & 2478 & -2670 \\
 445 & 5^1 89^1 & \text{Y} & \text{N} & 5 & 0 & 1.0000000 & 0.476404 & 0.523596 & -187 & 2483 & -2670 \\
 446 & 2^1 223^1 & \text{Y} & \text{N} & 5 & 0 & 1.0000000 & 0.477578 & 0.522422 & -182 & 2488 & -2670 \\
 447 & 3^1 149^1 & \text{Y} & \text{N} & 5 & 0 & 1.0000000 & 0.478747 & 0.521253 & -177 & 2493 & -2670 \\
 448 & 2^6 7^1 & \text{N} & \text{N} & -15 & 10 & 2.3333333 & 0.477679 & 0.522321 & -192 & 2493 & -2685 \\
 449 & 449^1 & \text{Y} & \text{Y} & -2 & 0 & 1.0000000 & 0.476615 & 0.523385 & -194 & 2493 & -2687 \\
 450 & 2^1 3^2 5^2 & \text{N} & \text{N} & -74 & 58 & 1.2162162 & 0.475556 & 0.524444 & -268 & 2493 & -2761 \\
 451 & 11^1 41^1 & \text{Y} & \text{N} & 5 & 0 & 1.0000000 & 0.476718 & 0.523282 & -263 & 2498 & -2761 \\
 452 & 2^2 113^1 & \text{N} & \text{N} & -7 & 2 & 1.2857143 & 0.475664 & 0.524336 & -270 & 2498 & -2768 \\
 453 & 3^1 151^1 & \text{Y} & \text{N} & 5 & 0 & 1.0000000 & 0.476821 & 0.523179 & -265 & 2503 & -2768 \\
 454 & 2^1 227^1 & \text{Y} & \text{N} & 5 & 0 & 1.0000000 & 0.477974 & 0.522026 & -260 & 2508 & -2768 \\
 455 & 5^1 7^1 13^1 & \text{Y} & \text{N} & -16 & 0 & 1.0000000 & 0.476923 & 0.523077 & -276 & 2508 & -2784 \\
 456 & 2^3 3^1 19^1 & \text{N} & \text{N} & -48 & 32 & 1.3333333 & 0.475877 & 0.524123 & -324 & 2508 & -2832 \\
 457 & 457^1 & \text{Y} & \text{Y} & -2 & 0 & 1.0000000 & 0.474836 & 0.525164 & -326 & 2508 & -2834 \\
 458 & 2^1 229^1 & \text{Y} & \text{N} & 5 & 0 & 1.0000000 & 0.475983 & 0.524017 & -321 & 2513 & -2834 \\
 459 & 3^3 17^1 & \text{N} & \text{N} & 9 & 4 & 1.5555556 & 0.477124 & 0.522876 & -312 & 2522 & -2834 \\
 460 & 2^2 5^1 23^1 & \text{N} & \text{N} & 30 & 14 & 1.1666667 & 0.478261 & 0.521739 & -282 & 2552 & -2834 \\
 461 & 461^1 & \text{Y} & \text{Y} & -2 & 0 & 1.0000000 & 0.477223 & 0.522777 & -284 & 2552 & -2836 \\
 462 & 2^1 3^1 7^1 11^1 & \text{Y} & \text{N} & 65 & 0 & 1.0000000 & 0.478355 & 0.521645 & -219 & 2617 & -2836 \\
 463 & 463^1 & \text{Y} & \text{Y} & -2 & 0 & 1.0000000 & 0.477322 & 0.522678 & -221 & 2617 & -2838 \\
 464 & 2^4 29^1 & \text{N} & \text{N} & -11 & 6 & 1.8181818 & 0.476293 & 0.523707 & -232 & 2617 & -2849 \\
 465 & 3^1 5^1 31^1 & \text{Y} & \text{N} & -16 & 0 & 1.0000000 & 0.475269 & 0.524731 & -248 & 2617 & -2865 \\
 466 & 2^1 233^1 & \text{Y} & \text{N} & 5 & 0 & 1.0000000 & 0.476395 & 0.523605 & -243 & 2622 & -2865 \\
 467 & 467^1 & \text{Y} & \text{Y} & -2 & 0 & 1.0000000 & 0.475375 & 0.524625 & -245 & 2622 & -2867 \\
 468 & 2^2 3^2 13^1 & \text{N} & \text{N} & -74 & 58 & 1.2162162 & 0.474359 & 0.525641 & -319 & 2622 & -2941 \\
 469 & 7^1 67^1 & \text{Y} & \text{N} & 5 & 0 & 1.0000000 & 0.475480 & 0.524520 & -314 & 2627 & -2941 \\
 470 & 2^1 5^1 47^1 & \text{Y} & \text{N} & -16 & 0 & 1.0000000 & 0.474468 & 0.525532 & -330 & 2627 & -2957 \\
 471 & 3^1 157^1 & \text{Y} & \text{N} & 5 & 0 & 1.0000000 & 0.475584 & 0.524416 & -325 & 2632 & -2957 \\
 472 & 2^3 59^1 & \text{N} & \text{N} & 9 & 4 & 1.5555556 & 0.476695 & 0.523305 & -316 & 2641 & -2957 \\
 473 & 11^1 43^1 & \text{Y} & \text{N} & 5 & 0 & 1.0000000 & 0.477801 & 0.522199 & -311 & 2646 & -2957 \\
 474 & 2^1 3^1 79^1 & \text{Y} & \text{N} & -16 & 0 & 1.0000000 & 0.476793 & 0.523207 & -327 & 2646 & -2973 \\
 475 & 5^2 19^1 & \text{N} & \text{N} & -7 & 2 & 1.2857143 & 0.475789 & 0.524211 & -334 & 2646 & -2980 \\
 476 & 2^2 7^1 17^1 & \text{N} & \text{N} & 30 & 14 & 1.1666667 & 0.476891 & 0.523109 & -304 & 2676 & -2980 \\
 477 & 3^2 53^1 & \text{N} & \text{N} & -7 & 2 & 1.2857143 & 0.475891 & 0.524109 & -311 & 2676 & -2987 \\
 478 & 2^1 239^1 & \text{Y} & \text{N} & 5 & 0 & 1.0000000 & 0.476987 & 0.523013 & -306 & 2681 & -2987 \\
 479 & 479^1 & \text{Y} & \text{Y} & -2 & 0 & 1.0000000 & 0.475992 & 0.524008 & -308 & 2681 & -2989 \\
 480 & 2^5 3^1 5^1 & \text{N} & \text{N} & -96 & 80 & 1.6666667 & 0.475000 & 0.525000 & -404 & 2681 & -3085 \\
 481 & 13^1 37^1 & \text{Y} & \text{N} & 5 & 0 & 1.0000000 & 0.476091 & 0.523909 & -399 & 2686 & -3085 \\
 482 & 2^1 241^1 & \text{Y} & \text{N} & 5 & 0 & 1.0000000 & 0.477178 & 0.522822 & -394 & 2691 & -3085 \\
 483 & 3^1 7^1 23^1 & \text{Y} & \text{N} & -16 & 0 & 1.0000000 & 0.476190 & 0.523810 & -410 & 2691 & -3101 \\
 484 & 2^2 11^2 & \text{N} & \text{N} & 14 & 9 & 1.3571429 & 0.477273 & 0.522727 & -396 & 2705 & -3101 \\
 485 & 5^1 97^1 & \text{Y} & \text{N} & 5 & 0 & 1.0000000 & 0.478351 & 0.521649 & -391 & 2710 & -3101 \\
 486 & 2^1 3^5 & \text{N} & \text{N} & 13 & 8 & 2.0769231 & 0.479424 & 0.520576 & -378 & 2723 & -3101 \\
 487 & 487^1 & \text{Y} & \text{Y} & -2 & 0 & 1.0000000 & 0.478439 & 0.521561 & -380 & 2723 & -3103 \\
 488 & 2^3 61^1 & \text{N} & \text{N} & 9 & 4 & 1.5555556 & 0.479508 & 0.520492 & -371 & 2732 & -3103 \\
 489 & 3^1 163^1 & \text{Y} & \text{N} & 5 & 0 & 1.0000000 & 0.480573 & 0.519427 & -366 & 2737 & -3103 \\
 490 & 2^1 5^1 7^2 & \text{N} & \text{N} & 30 & 14 & 1.1666667 & 0.481633 & 0.518367 & -336 & 2767 & -3103 \\
 491 & 491^1 & \text{Y} & \text{Y} & -2 & 0 & 1.0000000 & 0.480652 & 0.519348 & -338 & 2767 & -3105 \\
 492 & 2^2 3^1 41^1 & \text{N} & \text{N} & 30 & 14 & 1.1666667 & 0.481707 & 0.518293 & -308 & 2797 & -3105 \\
 493 & 17^1 29^1 & \text{Y} & \text{N} & 5 & 0 & 1.0000000 & 0.482759 & 0.517241 & -303 & 2802 & -3105 \\
 494 & 2^1 13^1 19^1 & \text{Y} & \text{N} & -16 & 0 & 1.0000000 & 0.481781 & 0.518219 & -319 & 2802 & -3121 \\
 495 & 3^2 5^1 11^1 & \text{N} & \text{N} & 30 & 14 & 1.1666667 & 0.482828 & 0.517172 & -289 & 2832 & -3121 \\
 496 & 2^4 31^1 & \text{N} & \text{N} & -11 & 6 & 1.8181818 & 0.481855 & 0.518145 & -300 & 2832 & -3132 \\
 497 & 7^1 71^1 & \text{Y} & \text{N} & 5 & 0 & 1.0000000 & 0.482897 & 0.517103 & -295 & 2837 & -3132 \\
 498 & 2^1 3^1 83^1 & \text{Y} & \text{N} & -16 & 0 & 1.0000000 & 0.481928 & 0.518072 & -311 & 2837 & -3148 \\
 499 & 499^1 & \text{Y} & \text{Y} & -2 & 0 & 1.0000000 & 0.480962 & 0.519038 & -313 & 2837 & -3150 \\
 500 & 2^2 5^3 & \text{N} & \text{N} & -23 & 18 & 1.4782609 & 0.480000 & 0.520000 & -336 & 2837 & -3173 \\  
\end{array}
}
\end{equation*}

\end{table} 

%\NBRef{A03-2020-04026}
%\NBRef{A04-2020-04026}

\end{document}
