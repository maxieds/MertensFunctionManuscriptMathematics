\documentclass[11pt,reqno,a4letter]{article} 

\usepackage{amsmath,amssymb,amsfonts,amscd}
\usepackage[hidelinks]{hyperref} 
\usepackage{url}
\usepackage[usenames,dvipsnames]{xcolor}
\hypersetup{
    colorlinks,
    linkcolor={black!63!darkgray},
    citecolor={blue!70!white},
    urlcolor={blue!80!white}
}

\usepackage[normalem]{ulem}
\usepackage{graphicx} 
\usepackage{datetime} 
\usepackage{cancel}
\usepackage{subcaption}
\captionsetup{format=hang,labelfont={bf},textfont={small,it}} 
\numberwithin{figure}{section}
\numberwithin{table}{section}

\usepackage{framed} 
\usepackage{ulem}
\usepackage[T1]{fontenc}
\usepackage{pbsi}

\usepackage{enumitem}
\setlist[itemize]{leftmargin=0.65in}

\usepackage{rotating,adjustbox}

\usepackage{diagbox}
\newcommand{\trianglenk}[2]{$\diagbox{#1}{#2}$}
\newcommand{\trianglenkII}[2]{\diagbox{#1}{#2}}

\let\citep\cite

\newcommand{\undersetbrace}[2]{\underset{\displaystyle{#1}}{\underbrace{#2}}}

\usepackage{MnSymbol}
\newcommand{\gkpEII}[2]{\ensuremath{\genfrac{\llangle}{\rrangle}{0pt}{}{#1}{#2}}}

\newcommand{\gkpSI}[2]{\ensuremath{\genfrac{\lbrack}{\rbrack}{0pt}{}{#1}{#2}}} 
\newcommand{\gkpSII}[2]{\ensuremath{\genfrac{\lbrace}{\rbrace}{0pt}{}{#1}{#2}}}
\newcommand{\cf}{\textit{cf.\ }} 
\newcommand{\Iverson}[1]{\ensuremath{\left[#1\right]_{\delta}}} 
\newcommand{\floor}[1]{\left\lfloor #1 \right\rfloor} 
\newcommand{\ceiling}[1]{\left\lceil #1 \right\rceil} 
\newcommand{\e}[1]{e\left(#1\right)} 
\newcommand{\seqnum}[1]{\href{http://oeis.org/#1}{\color{ProcessBlue}{\underline{#1}}}}

\usepackage{upgreek,dsfont,amssymb}
\renewcommand{\chi}{\upchi}
\newcommand{\ChiFunc}[1]{\ensuremath{\chi_{\{#1\}}}}
\newcommand{\OneFunc}[1]{\ensuremath{\mathds{1}_{#1}}}

\usepackage{ifthen}
\newcommand{\Hn}[2]{
     \ifthenelse{\equal{#2}{1}}{H_{#1}}{H_{#1}^{\left(#2\right)}}
}

\newcommand{\Floor}[2]{\ensuremath{\left\lfloor \frac{#1}{#2} \right\rfloor}}
\newcommand{\Ceiling}[2]{\ensuremath{\left\lceil \frac{#1}{#2} \right\rceil}}

\DeclareMathOperator{\DGF}{DGF} 
\DeclareMathOperator{\ds}{ds} 
\DeclareMathOperator{\Id}{Id}
\DeclareMathOperator{\fg}{fg}
\DeclareMathOperator{\Div}{div}
\DeclareMathOperator{\rpp}{rpp}
\DeclareMathOperator{\logll}{\ell\ell}

\title{
       \LARGE{
       New characterizations of the summatory function of the M\"obius function 
       } 
}
\author{{\Large Maxie Dion Schmidt} \\ 
        %{\normalsize \href{mailto:maxieds@gmail.com}{maxieds@gmail.com}} \\[0.1cm] 
        {\normalsize Georgia Institute of Technology} \\[0.025cm] 
        {\normalsize School of Mathematics} 
} 

%\date{\small\underline{Last Revised:} \today \ @\ \hhmmsstime{} \ -- \ Compiled with \LaTeX2e} 

\usepackage{amsthm} 

\theoremstyle{plain} 
\newtheorem{theorem}{Theorem}
\newtheorem{conjecture}[theorem]{Conjecture}
\newtheorem{claim}[theorem]{Claim}
\newtheorem{prop}[theorem]{Proposition}
\newtheorem{lemma}[theorem]{Lemma}
\newtheorem{cor}[theorem]{Corollary}
\numberwithin{theorem}{section}

\theoremstyle{definition} 
\newtheorem{example}[theorem]{Example}
\newtheorem{remark}[theorem]{Remark}
\newtheorem{definition}[theorem]{Definition}
\newtheorem{notation}[theorem]{Notation}
\newtheorem{question}[theorem]{Question}
\newtheorem{discussion}[theorem]{Discussion}
\newtheorem{facts}[theorem]{Facts}
\newtheorem{summary}[theorem]{Summary}
\newtheorem{heuristic}[theorem]{Heuristic}
\newtheorem{ansatz}[theorem]{Ansatz}

\renewcommand{\arraystretch}{1.25} 

\setlength{\textheight}{9.5in}
\setlength{\topmargin}{-0.5in}
\setlength{\textwidth}{7in} 
\setlength{\evensidemargin}{-0.25in} 
\setlength{\oddsidemargin}{-0.25in} 
\setlength{\headsep}{8pt} 

\usepackage{fancyhdr}
\pagestyle{empty}
\pagestyle{fancy}
\fancyhead[RO,RE]{Maxie Dion Schmidt -- \today} 
\fancyhead[LO,LE]{}
\fancyheadoffset{0.005\textwidth} 

\setlength{\parindent}{0in}
\setlength{\parskip}{2cm} 

\renewcommand{\thefootnote}{\textbf{\Alph{footnote}}}

\newcommand{\SuccSim}[0]{\overset{_{\scriptsize{\blacktriangle}}}{\succsim}} 
\newcommand{\PrecSim}[0]{\overset{_{\scriptsize{\blacktriangle}}}{\precsim}} 
\renewcommand{\SuccSim}[0]{\ensuremath{\gg}} 
\renewcommand{\PrecSim}[0]{\ensuremath{\ll}} 

\renewcommand{\Re}{\operatorname{Re}}
\renewcommand{\Im}{\operatorname{Im}}

\usepackage{tikz}
\usetikzlibrary{shapes,arrows}
\usepackage{enumitem} 

%\input{glossaries-bibtex/PreambleGlossaries-Mertens}

\usepackage{longtable}
\usepackage{arydshln} 
\usepackage[symbols,nogroupskip,nomain,automake=true,nonumberlist,toc]{glossaries-extra}
\usepackage{glossary-mcols}

%%%%%%%%%%%%

\providecommand{\glossarytoctitle}{\glossaryname}
\setlength{\glsdescwidth}{0.7\textwidth}

\newglossarystyle{glossstyleSymbol}{%
\renewenvironment{theglossary}%
 {\begin{longtable}{lp{\glsdescwidth}}}%
 {\end{longtable}}%
 \setlength{\parskip}{3.5pt}
 \renewcommand{\glsgroupskip}{}
\renewcommand*\glspostdescription{\dotfill}
\renewcommand*{\glossaryheader}{%
 \bfseries Symbol & \bfseries Definition
 \\\endhead}%
 \renewcommand*{\glsgroupheading}[1]{}%
  \renewcommand{\glossentry}[2]{%
    \glstarget{##1}{\glossentrysymbol{##1}} &
    \glossentrydesc{##1} \tabularnewline
  }%
  \renewcommand*{\glspostdescription}{}
  \renewcommand{\glossarymark}[1]{}
}

\setglossarystyle{glossstyleSymbol}
\glsaddall[types={symbols}]
\makeglossaries

%%%%%%%%%%%%

\newglossaryentry{fCvlg}{
    symbol={\ensuremath{f \ast g}},
    sort={fg},
    description={The Dirichlet convolution of $f$ and $g$, $(f \ast g)(n) := \sum\limits_{d|n} f(d) g(n/d)$, 
    where the sum is taken over the divisors of any $n \geq 1$. 
    },
    type={symbols},
    name={Dirichlet convolution}
    }
\newglossaryentry{coeffExtraction}{
    symbol={\ensuremath{[q^n] F(q)}},
    sort={coeffExtraction},
    description={The coefficient of $q^n$ in the power series expansion of $F(q)$ about zero when 
    $F(q)$ is treated as the ordinary generating function (OGF) of some sequence, $\{f_n\}_{n \geq 0}$. 
    Namely, for integers $n \geq 0$ we define $[q^n] F(q) = f_n$ whenever $F(q) := \sum_{n \geq 0} f_n q^n$. },
    type={symbols},
    name={Series coefficient extraction}
    }
\newglossaryentry{MoebiusMuFunc}{
    symbol={\ensuremath{\mu(n),M(x)}},
    sort={MoebiusMuFunc},
    description={The M\"obius function defined such that $\mu^2(n)$ is the indicator function of the 
                 squarefree integers $n \geq 1$ where 
                 $\mu(n) = (-1)^{\omega(n)}$ whenever $n$ is squarefree. 
                 The Mertens function is the summatory function defined for all integers 
                 $x \geq 1$ by $M(x) := \sum\limits_{n \leq x} \mu(n)$.
                 },
    type={symbols},
    name={M\"obius function}
    }
\newglossaryentry{Iverson}{
    symbol={\ensuremath{\Iverson{n=k}},\ensuremath{\Iverson{\mathtt{cond}}}},
    sort={Iverson},
    description={The symbol $\Iverson{n=k}$ is a synonym for $\delta_{n,k}$ 
                 which is one if and only if $n = k$, and is zero otherwise. 
                 For boolean-valued conditions, \texttt{cond}, the symbol $\Iverson{\mathtt{cond}}$ 
                 evaluates to one precisely when \texttt{cond} is true, and to zero otherwise. 
                 This notation is called \emph{Iverson's convention}.},
    type={symbols},
    name={Iverson's convention}
    }
\newglossaryentry{epsilonN}{
    symbol={\ensuremath{\varepsilon(n)}},
    sort={epsilonN},
    description={The multiplicative identity with respect to Dirichlet convolution, $\varepsilon(n) := \delta_{n,1}$, 
                 defined such that for any arithmetic function $f$ we have that 
                 $f \ast \varepsilon = \varepsilon \ast f = f$ where the operation 
                 $\ast$ denotes Dirichlet convolution 
                 (see definition below).},
    type={symbols},
    name={Dirichlet multiplicative identity}
    }
\newglossaryentry{Zetas}{
    symbol={\ensuremath{\zeta(s)}},
    sort={Zetas},
    description={The Riemann zeta function is defined by $\zeta(s) := \sum_{n \geq 1} n^{-s}$ when $\Re(s) > 1$, 
                 and by analytic continuation on the rest of the complex plane with the exception of a 
                 simple pole at $s = 1$ of residue one.},
    type={symbols},
    name={Riemann zeta function}
    }
\newglossaryentry{fInvn}{
     symbol={\ensuremath{f^{-1}(n)}},
    sort={fInvn},
    description={
     The Dirichlet inverse $f^{-1}$ of any arithmetic function $f$ exists 
     if and only if $f(1) \neq 0$. 
     The Dirichlet inverse of any $f$ such that $f(1) \neq 0$ 
     is defined recursively by 
     $f^{-1}(n) = -\frac{1}{f(1)} \sum\limits_{\substack{d|n \\ d>1}} f(d) f^{-1}(n/d)$ for $n \geq 2$ with $f^{-1}(1) = 1 / f(1)$. 
    When it exists, this inverse function 
    is unique and satisfies the characteristic relations 
    that $f^{-1} \ast f = f \ast f^{-1} = \varepsilon$.},
    type={symbols},
    name={Dirichlet inverse of $f$}
    }
\newglossaryentry{PrimeZetaFunc}{
    symbol={$P(s)$},
    sort={PrimeZetaFunc},
    description={For complex $s$ with $\Re(s) > 1$, we define the prime zeta function to be the 
                 Dirichlet generating function (DGF) 
                 $P(s) = \sum_{n \geq 1} \frac{\chi_{\mathbb{P}}(n)}{n^s} = 
                         \sum_{k \geq 2} \frac{\mu(k)}{k} \log\zeta(ks)$. 
                 },
    type={symbols},
    name={Prime zeta function}
    }
\newglossaryentry{CkngInvAuxFunc}{
    symbol={$C_k(n),C_{\Omega(n)}(n)$},
    sort={CkngInvAuxFunc},
    description={The sequence is defined recursively for integers $n \geq 1$ and $k \geq 0$ as follows: 
                 \[
                 C_k(n) := \begin{cases} 
                      \delta_{n,1}, & \text{ if $k = 0$; } \\ 
                      \sum\limits_{d|n} \omega(d) C_{k-1}\left(\frac{n}{d}\right), & \text{ if $k \geq 1$. } 
                      \end{cases} 
                 \]
                 It represents the multiple ($k$-fold) convolution of the function $\omega(n)$ 
                 with itself. 
                 %For $k := \Omega(n)$ and $n \geq 2$, we have the explicit formula that 
                 %$C_{\Omega(n)}(n) = (\Omega(n))! \times \prod_{p^{\alpha}||n} (\alpha!)^{-1}$. 
                 },
    type={symbols},
    name={Dirichlet inverse component functions}
    }
\newglossaryentry{gInvn}{
    symbol={$g^{-1}(n),G^{-1}(x)$},
    sort={gInvn},
    description={The Dirichlet inverse function, $g^{-1}(n) = (\omega+1)^{-1}(n)$ with corresponding 
                 summatory function $G^{-1}(x) := \sum_{n \leq x} g^{-1}(n)$. },
    type={symbols},
    name={Key Dirichlet inverse functions}
    }
\newglossaryentry{PikPiHatkx}{
    symbol={$\pi_k(x),\widehat{\pi}_k(x)$},
    sort={PikPiHatkx},
    description={For integers $k \geq 1$, the 
                 prime counting function variant $\pi_k(x)$ denotes the number of 
                 $2 \leq n \leq x$ with 
                 exactly $k$ distinct prime factors: $\pi_k(x) := \#\{2 \leq n \leq x: \omega(n) = k\}$. 
                 Similarly, the function 
                 $\widehat{\pi}_k(x) := \#\{2 \leq n \leq x: \Omega(n) = k\}$ for $x \geq 2$ and fixed $k \geq 1$. },
    type={symbols},
    name={Distinct prime counting functions}
    }   
\newglossaryentry{Nupn}{
    symbol={$\nu_p(n)$}, 
    sort={Nupn},
    description={The valuation function that extracts the maximal exponent of $p$ in the prime factorization of $n$, e.g., 
                 $\nu_p(n) = 0$ if $p \nmid n$ and $\nu_p(n) = \alpha$ if $p^{\alpha} || n$ 
                 (that is, when $p^{\alpha}$ exactly divides $n$) for 
                 $p \geq 2$ prime, $\alpha \geq 1$ and $n \geq 2$.},
    type={symbols},
    name={Exponent extraction function}
    }
\newglossaryentry{primeOmegaFunctions}{
    symbol={$\omega(n)$,$\Omega(n)$}, 
    sort={OmegaPrimeOmegaFunctions},
    description={We define the strongly additive function 
                 $\omega(n) := \sum_{p|n} 1$ and the completely additive function 
                 $\Omega(n) := \sum_{p^{\alpha} || n} \alpha$. This means that if the prime 
                 factorization of $n \geq 2$ is 
                 given by $n := p_1^{\alpha_1} \cdots p_r^{\alpha_r}$ with $p_i \neq p_j$ for all $i \neq j$, 
                 then $\omega(n) = r$ and $\Omega(n) = \alpha_1 + \cdots + \alpha_r$. 
                 By convention we set $\omega(1) = \Omega(1) = 0$.},
    type={symbols},
    name={Prime omega functions}
    }
\newglossaryentry{LiouvilleLambdaFunc}{
     symbol={$\lambda(n), L(x)$}, 
    sort={LiouvilleLambdaFunc},
    description={The Liouville lambda function is the completely multiplicative function defined by 
                 $\lambda(n) := (-1)^{\Omega(n)}$. 
                 Its summatory function is defined by $L(x) := \sum_{n \leq x} \lambda(n)$. 
                 },
    type={symbols},
    name={Liouville lambda function}
    }
\newglossaryentry{AverageOrderExpectation}{
    symbol={$\mathbb{E}[f(x)]$},
    sort={AverageOrderExpectation},
    description={We use the expectation notation of $\mathbb{E}[f(x)] = h(x)$ 
                 to denote that $f$ has an \emph{average order} 
                 of $h(x)$. This means that $\frac{1}{x} \times \sum_{n \leq x} f(n) \sim h(x)$. }, 
    type={symbols},
    name={Average order (expectation) notation}
    }
%\newglossaryentry{BConstant}{
%    symbol={$B$},
%    sort={BConstant},
%    description={The absolute constant $B \approx 0.2614972$ from the statement of Mertens theorem. },
%    type={symbols},
%    name={Constant $B$}
%    }
\newglossaryentry{QxSummatoryFunc}{
    symbol={$Q(x)$},
    sort={QxSummatoryFunc},
    description={For $x \geq 1$, we define $Q(x)$ to be the summatory function indicating the number of 
                 squarefree integers $n \leq x$. That is, $Q(x) := \sum_{n \leq x} \mu^2(n)$.}, 
    type={symbols},
    name={Summatory function of the squarefree integers}
    }
\newglossaryentry{ApproxAndSimRelations}{
    symbol={$\approx,\sim$},
    sort={ApproxAndSimRelations},
    description={We write that $f(x) \approx g(x)$ if $|f(x) - g(x)| = O(1)$ 
                 as $x \rightarrow \infty$. 
                 Two arithmetic functions $A(x), B(x)$ satisfy the relation $A \sim B$ if 
                 $\lim_{x \rightarrow \infty} \frac{A(x)}{B(x)} = 1.$ },
    type={symbols},
    name={Asymptotic relation symbol}
    }
\newglossaryentry{GGLLRelations}{
    symbol={$\gg,\ll,\asymp$},
    sort={GGLLRelations},
    description={
                 For functions $A,B$, the notation $A \ll B$ implies that $A = O(B)$. 
                 Similarly, for $B \geq 0$ the notation $A \gg B$ implies that $B = O(A)$. 
                 When we have that $A, B \geq 0$, $A \ll B$ and $B \ll A$, we write $A \asymp B$. },
    type={symbols},
    name={Asymptotic relation symbols}
    }
\newglossaryentry{NormalCDFFunc}{
    symbol={$\Phi(z)$},
    sort={NormalCDFFunc},
    description={For $x \in \mathbb{R}$, we define the CDF of the standard normal distribution to be  
                 $\Phi(z) := \frac{1}{\sqrt{2\pi}} \times \int_{-\infty}^{z} e^{-t^2/2} dt$. },
    type={symbols},
    name={Asymptotic relation symbol}
    }
%\newglossaryentry{dkDensity}{
%    symbol={$d_k$},
%    sort={dkDensity},
%    description={For non-negative integers $k \geq 0$, we define the densities $d_k$ of the 
%                 distinct values of the differences of the prime omega functions by 
%                 $d_k := \lim_{x \rightarrow \infty} \frac{1}{x} \cdot \#\{n \leq x: \Omega(n) - \omega(n) = k\}$. 
%                 },
%    type={symbols},
%    name={Asymptotic densities of prime omega function difference values}
%    }
%\newglossaryentry{lambdaFuncAst}{
%    symbol={$\lambda_{\ast}(n)$},
%    sort={lambdaFuncAst},
%    description={For positive integers $n \geq 2$, we define $\lambda_{\ast}(n) := (-1)^{\omega(n)}$. 
%                 We have the initial condition that $\lambda_{\ast}(1) = 1$. },
%    type={symbols},
%    name={Prime omega function difference lambda function}
%    }
%\newglossaryentry{cHatConstant}{
%    symbol={$\hat{c}$},
%    sort={cHatConstant},
%    description={The absolute constant defined by 
%                 $\hat{c} := \frac{1}{4} \prod\limits_{p > 2} \left(1 - \frac{1}{(p-1)^2}\right)^{-1} \approx 0.378647$. },
%    type={symbols},
%    name={Prime product absolute constant}
%    }
%\newglossaryentry{gammaConstant}{
%    symbol={$\gamma$},
%    sort={gammaConstant},
%    description={The Euler gamma constant defined by 
%                 $\gamma := \lim\limits_{n \rightarrow \infty} \left(\sum\limits_{k=1}^{n} \frac{1}{k} - \log n\right) 
%                  \approx 0.5772157$. },
%    type={symbols},
%    name={Euler gamma constant}
%    }
%\newglossaryentry{muSigmaTheoremConstants}{
%    symbol={$\mu_x(C),\sigma_x(C)$},
%    sort={muSigmaTheoremConstants},
%    description={We define $\mu_x(C) := \log\log x$ and 
%                 $\sigma_x(C) := \sqrt{\log\log x}$. }, 
%    type={symbols},
%    name={Erd{\H{o}}s-Kac theorem mean and variance analogs}
%    }
\newglossaryentry{chiPrimeP}{
    symbol={$\chi_{\mathbb{P}}(n)$},
    sort={chiPrimeP},
    description={The characteristic (or indicator) function of the primes equals one if and only if 
                 $n \in \mathbb{Z}^{+}$ is prime, and is 
                 zero-valued otherwise. },
    type={symbols},
    name={Prime set indicator function}
    }
\newglossaryentry{WLambertWFunction}{
    symbol={$W(x)$},
    sort={WLambertWFunction},
    description={For $x,y \in \mathbb{R}_{\geq 0}$, we write that $x = W(y)$ if and only if $xe^{x} = y$. 
                 This function denotes the principal branch of the multi-valued Lambert $W$ function 
                 on the non-negative reals. },
    type={symbols},
    name={Lambert $W$-Function}
    }
\glsaddall[types={symbols}]

\allowdisplaybreaks 

\begin{document} 

\maketitle

\begin{abstract} 
The Mertens function, $M(x) := \sum_{n \leq x} \mu(n)$, is 
defined as the summatory function of the classical M\"obius function for $x \geq 1$. 
The inverse sequence $\{g^{-1}(n)\}_{n \geq 1}$ 
taken with respect to Dirichlet convolution is defined in terms of the 
strongly additive function $\omega(n)$ that counts the 
number of distinct prime factors of any integer $n \geq 2$ without considering multiplicity. 
For large $x$ and $n \leq x$, we associate a natural combinatorial 
significance to the magnitude of the distinct values of the 
function $g^{-1}(n)$ that depends directly on the exponent patterns in the 
prime factorizations of the integers in $\{2,3,\ldots,x\}$ viewed as multisets. 

We prove an Erd\H{o}s-Kac theorem analog for the distribution of the 
unsigned sequence $|g^{-1}(n)|$ over $n \leq x$ with a central limit theorem 
tendency towards normal as $x \rightarrow \infty$. 
For all $x \geq 1$, discrete convolutions of the summatory function 
$G^{-1}(x) := \sum_{n \leq x} \lambda(n) |g^{-1}(n)|$ with 
the ordinary prime counting function $\pi(x)$ determine 
exact formulas and new characterizations of asymptotic bounds for $M(x)$. 
In this way, we prove another concrete link to the distribution of  
$L(x) := \sum_{n \leq x} \lambda(n)$ with the Mertens function and connect these classical 
summatory functions with an explicit normal tending 
probability distribution at large $x$. 
The proofs of these resulting combinatorially motivated new characterizations of $M(x)$ 
are rigorous and unconditional. 

\bigskip 
\noindent
\textbf{Keywords and Phrases:} {\it M\"obius function; Mertens function; 
                                    Dirichlet inverse; Liouville lambda function; prime omega function; 
                                    prime counting function; Dirichlet generating function; 
                                    Erd\H{o}s-Kac theorem; strongly additive function. } \\ 
% 11-XX			Number theory
%    11A25  	Arithmetic functions; related numbers; inversion formulas
%    11Y70  	Values of arithmetic functions; tables
%    11-04  	Software, source code, etc. for problems pertaining to number theory
% 11Nxx		Multiplicative number theory
%    11N05  	Distribution of primes
%    11N37  	Asymptotic results on arithmetic functions
%    11N56  	Rate of growth of arithmetic functions
%    11N60  	Distribution functions associated with additive and positive multiplicative functions
%    11N64  	Other results on the distribution of values or the characterization of arithmetic functions
\textbf{Math Subject Classifications (MSC 2010):} {\it 11N37; 11A25; 11N60; 11N64; and 11-04. } 
\end{abstract} 

\bigskip\hrule\bigskip

\emph{``It is evident that the primes are randomly distributed but, unfortunately, 
      we do not know what 'random' means.''} -- \textbf{R. C. Vaughan} 

\bigskip\hrule\bigskip

\newpage
\renewcommand{\contentsname}{Article Index}
\tableofcontents

\newpage
\section{Introduction} 
\label{subSection_MertensMxClassical_Intro} 

\subsection{Preliminaries}

\subsubsection{Definitions} 

We define the \emph{M\"obius function} to be the signed indicator function 
of the squarefree integers in the form of \cite[\seqnum{A008683}]{OEIS} 
\[
\mu(n) = \begin{cases} 
     1, & \text{if $n = 1$; } \\ 
     (-1)^{\omega(n)}, & \text{if $\omega(n) = \Omega(n) \wedge n \geq 2$; } \\ 
     0, & \text{otherwise.} 
     \end{cases} 
\]
The \emph{Mertens function}, or summatory function of $\mu(n)$, is defined on the 
positive integers as 
\begin{align*} 
M(x) & = \sum_{n \leq x} \mu(n), x \geq 1. 
\end{align*} 
The sequence of slow growing oscillatory values of this 
summatory function begins as follows \cite[\seqnum{A002321}]{OEIS}: 
\[
\{M(x)\}_{x \geq 1} = \{1, 0, -1, -1, -2, -1, -2, -2, -2, -1, -2, -2, -3, -2, 
     -1, -1, -2, -2, -3, -3, -2, -1, -2, \ldots\}. 
\] 
The Mertens function satisfies that $\sum_{n \leq x} M\left(\Floor{x}{n}\right) = 1$, and is related 
to the summatory function $L(x) := \sum_{n \leq x} \lambda(n)$ via the relation 
\cite{HUMPHRIES-JNT-2013,LEHMAN-1960} 
\[
L(x) = \sum_{d \leq \sqrt{x}} M\left(\Floor{x}{d^2}\right), x \geq 1. 
\]
A positive integer $n \geq 1$ is \emph{squarefree}, or contains no divisors 
which are squares (other than one when $n \geq 2$), if and only if $\mu^2(n) = 1$. 
The summatory function that counts the 
number of \emph{squarefree} integers $n \leq x$ satisfies 
\cite[\S 18.6]{HARDYWRIGHT} \cite[\seqnum{A013928}]{OEIS} 
\[ 
Q(x) = \sum_{n \leq x} \mu^2(n) = \frac{6x}{\pi^2} + O\left(\sqrt{x}\right), 
     \text{\ as $x \rightarrow \infty$}. 
\] 

\subsubsection{Properties} 
\label{subSection_Intro_Mx_properties} 

A conventional approach to evaluating the limiting asymptotic 
behavior of $M(x)$ for large $x \rightarrow \infty$ considers an 
inverse Mellin transformation of the reciprocal of the Riemann zeta function. 
In particular, since 
\[
\frac{1}{\zeta(s)} = \prod_{p} \left(1 - \frac{1}{p^s}\right) = 
     s \times \int_1^{\infty} \frac{M(x)}{x^{s+1}} dx, \Re(s) > 1, 
\]
we obtain that 
\[
M(x) = \lim_{T \rightarrow \infty}\ \frac{1}{2\pi\imath} \times \int_{T-\imath\infty}^{T+\imath\infty} 
     \frac{x^s}{s \zeta(s)} ds. 
\] 
The previous two representations lead us to the 
exact expression of $M(x)$ for any $x > 0$ 
given by the next theorem. 
\nocite{TITCHMARSH} 

\begin{theorem}[Titchmarsh] 
\label{theorem_MxMellinTransformInvFormula} 
Assuming the Riemann Hypothesis (RH), there exists an infinite sequence 
$\{T_k\}_{k \geq 1}$ satisfying $k \leq T_k \leq k+1$ for each integer $k \geq 1$ 
such that for any real $x > 0$ 
\[
M(x) = \lim_{k \rightarrow \infty} 
     \sum_{\substack{\rho: \zeta(\rho) = 0 \\ |\Im(\rho)| < T_k}} 
     \frac{x^{\rho}}{\rho \zeta^{\prime}(\rho)} - 2 + 
     \sum_{n \geq 1} \frac{(-1)^{n-1}}{n (2n)! \zeta(2n+1)} 
     \left(\frac{2\pi}{x}\right)^{2n} + 
     \frac{\mu(x)}{2} \Iverson{x \in \mathbb{Z}^{+}}. 
\] 
\end{theorem} 

A historical unconditional bound on the Mertens function due to Walfisz (circa 1963) 
states that there is an absolute constant $C_1 > 0$ such that 
$$M(x) \ll x \times \exp\left(-C_1 \log^{\frac{3}{5}}(x) 
  (\log\log x)^{-\frac{3}{5}}\right).$$ 
Under the assumption of the RH, Soundararajan improved estimates 
bounding $M(x)$ from above for large $x$ in the following form for any fixed 
$\epsilon > 0$ \cite{SOUND-MERTENS-ANNALS}: 
\begin{align*} 
%M(x) & \ll \sqrt{x} \cdot \exp\left((\log x)^{\frac{1}{2}} (\log\log x)^{14}\right), \\ 
M(x) & = O\left(\sqrt{x} \times \exp\left( 
     (\log x)^{\frac{1}{2}} (\log\log x)^{\frac{5}{2}+\epsilon}\right)\right). 
\end{align*} 

\subsubsection{Conjectures on boundedness and limiting behavior} 

The RH is equivalent to showing that 
$M(x) = O\left(x^{\frac{1}{2}+\epsilon}\right)$ for any 
$0 < \epsilon < \frac{1}{2}$. 
There is a rich history to the original statement of the \emph{Mertens conjecture} which 
asserts that 
\[ 
|M(x)| < C_2 \sqrt{x},\ \text{ for some absolute constant $C_2 > 0$. }
\] 
The conjecture was first verified by Mertens himself for $C_2 = 1$ and all $x < 10000$ 
without the benefit of modern computation. 
Since its beginnings in 1897, the Mertens conjecture has been disproved by computational methods with
non-trivial simple zeta function zeros with comparatively small imaginary parts in the famous paper by 
Odlyzko and t\'{e} Riele \cite{ODLYZ-TRIELE}. 
More recent attempts 
at bounding $M(x)$ naturally consider determining the rates at which the function 
$q(x) := M(x) / \sqrt{x}$ grows with or without bound along infinite 
subsequences, e.g., considering the asymptotics of $q(x)$ 
in the limit supremum and limit infimum senses. 

It is verified by computation 
that \cite[\cf \S 4.1]{PRIMEREC} 
\cite[\cf \seqnum{A051400}; \seqnum{A051401}]{OEIS} 
\[
\limsup_{x\rightarrow\infty} \frac{M(x)}{\sqrt{x}} > 1.060\ \qquad (\text{now } \geq 1.826054), 
\] 
and 
\[ 
\liminf_{x\rightarrow\infty} \frac{M(x)}{\sqrt{x}} < -1.009\ \qquad (\text{now } \leq -1.837625). 
\] 
Based on the work by Odlyzyko and t\'{e} Riele, it seems probable that 
each of these limits should evaluate to $\pm \infty$, respectively 
\cite{ODLYZ-TRIELE,MREVISITED,ORDER-MERTENSFN,HURST-2017}. 
A famous conjecture due to Gonek asserts that in fact 
$M(x)$ satisfies \cite{NG-MERTENS}
$$\limsup_{x \rightarrow \infty} \frac{|M(x)|}{\sqrt{x} \cdot (\log\log\log x)^{\frac{5}{4}}} = O(1).$$ 

\subsection{A concrete new approach to characterizing $M(x)$} 

The main interpretation to take away from the article is 
our rigorous motivation of an 
equivalent characterization of $M(x)$ using two new number theoretic sequences and their 
summatory functions. This characterization is formed by constructing 
combinatorially relevant sequences related to the distribution of the primes 
through convolutions of strongly additive functions. 
These sequences and their summatory functions 
have not yet been studied in the literature surrounding the Mertens function. 
This new perspective offers new exact characterizations of $M(x)$ for all $x \geq 1$ 
through the formulas involving discrete convolutions of 
$G^{-1}(x) := \sum_{n \leq x} g^{-1}(n)$, where $g^{-1}(n) := (\omega+1)^{-1}(n)$, with the 
classical prime counting function $\pi(x)$ we will prove in 
Section \ref{Section_KeyApplications}. 

The sequence $g^{-1}(n)$ and its summatory function 
$G^{-1}(x)$ are crucially tied to 
canonical number theoretic examples of strongly and completely additive functions, 
e.g., to $\omega(n)$ and $\Omega(n)$, respectively. 
The definitions of the primary new auxiliary subsequences we define, and the 
proof methods given in the spirit of Montgomery and Vaughan's work, 
allow us to reconcile the property of strong additivity with signed sums of 
multiplicative functions. 
The proofs of characteristic properties of these new sequences 
imply a scaled normal tending probability distribution for the unsigned 
magnitude of $|g^{-1}(n)|$ that is analogous to the established Erd\H{o}s-Kac type theorems 
characterizing $\omega(n)$ and $\Omega(n)$. 

Since we prove that $\operatorname{sgn}(g^{-1}(n)) = \lambda(n)$ for all $n \geq 1$, it follows 
that we have a new probabilistic perspective from which to express distributional features of 
the summatory functions $G^{-1}(x)$ as $x \rightarrow \infty$ in terms of the properties of 
$|g^{-1}(n)|$ and asymptotics for $L(x) := \sum_{n \leq x} \lambda(n)$. 
The new results in this article then precisely 
connect the distributions of $L(x)$, an explicitly identified 
scaled normally tending probability distribution, and $M(x)$ as $x \rightarrow \infty$. 
Formalizing the properties of the distribution of 
$L(x)$ is typically viewed as a problem that is equally as difficult 
as understanding the distribution of $M(x)$ well at large $x$ or along infinite subsequences. 

\subsubsection{Summatory functions of Dirichlet convolutions of arithmetic functions} 

\begin{theorem}[Summatory functions of Dirichlet convolutions] 
\label{theorem_SummatoryFuncsOfDirCvls} 
Let $f,h: \mathbb{Z}^{+} \rightarrow \mathbb{C}$ be any arithmetic functions such that $f(1) \neq 0$. 
Suppose that $F(x) := \sum_{n \leq x} f(n)$ and $H(x) := \sum_{n \leq x} h(n)$ denote the summatory 
functions of $f$ and $h$, respectively, and that $F^{-1}(x) := \sum_{n \leq x} f^{-1}(n)$ 
denotes the summatory function of the 
Dirichlet inverse of $f$ for any $x \geq 1$. We have the following exact expressions for the 
summatory function of the convolution $f \ast h$ for all integers $x \geq 1$: 
\begin{align*} 
\pi_{f \ast h}(x) & := \sum_{n \leq x} \sum_{d|n} f(d) h\left(\frac{n}{d}\right) \\ 
     & \phantom{:}= \sum_{d \leq x} f(d) H\left(\Floor{x}{d}\right) \\ 
     & \phantom{:}= \sum_{k=1}^{x} H(k) \left[F\left(\Floor{x}{k}\right) - 
     F\left(\Floor{x}{k+1}\right)\right]. 
\end{align*} 
Moreover, for all $x \geq 1$ 
\begin{align*} 
H(x) & = \sum_{j=1}^{x} \pi_{f \ast h}(j) \left[F^{-1}\left(\Floor{x}{j}\right) - 
     F^{-1}\left(\Floor{x}{j+1}\right)\right] \\ 
     & = \sum_{k=1}^{x} f^{-1}(k) \pi_{f \ast h}\left(\Floor{x}{k}\right). 
\end{align*} 
\end{theorem} 

\begin{cor}[Applications of M\"obius inversion] 
\label{cor_CvlGAstMu} 
Suppose that $h$ is an arithmetic function such that 
$h(1) \neq 0$. Define the summatory function of 
the convolution of $h$ with $\mu$ by $\widetilde{H}(x) := \sum_{n \leq x} (h \ast \mu)(n)$. 
Then the Mertens function is expressed by the sum 
\[
M(x) = \sum_{k=1}^{x} \left(\sum_{j=\floor{\frac{x}{k+1}}+1}^{\floor{\frac{x}{k}}} h^{-1}(j)\right) 
     \widetilde{H}(k), \forall x \geq 1. 
\]
\end{cor} 

\begin{cor}[Key Identity] 
\label{cor_Mx_gInvnPixk_formula} 
We have that for all $x \geq 1$ 
\begin{equation} 
\label{eqn_Mx_gInvnPixk_formula} 
M(x) = \sum_{k=1}^{x} (\omega+1)^{-1}(k) \left[\pi\left(\Floor{x}{k}\right) + 1\right]. 
\end{equation} 
\end{cor} 

\subsubsection{An exact expression for $M(x)$ via strongly additive functions} 
\label{example_InvertingARecRelForMx_Intro}

Fix the notation for the Dirichlet invertible function $g(n) := \omega(n) + 1$ and define its 
inverse with respect to Dirichlet convolution by $g^{-1}(n)$ 
\cite[\seqnum{A341444}]{OEIS}. 
We can compute exactly that 
(see Table \ref{table_conjecture_Mertens_ginvSeq_approx_values} on page 
\pageref{table_conjecture_Mertens_ginvSeq_approx_values}) 
\[
\{g^{-1}(n)\}_{n \geq 1} = \{1, -2, -2, 2, -2, 5, -2, -2, 2, 5, -2, -7, -2, 5, 5, 2, -2, -7, -2, 
     -7, 5, 5, -2, 9, \ldots \}. 
\] 
There is not a simple 
direct recursion between the distinct values of $g^{-1}(n)$ that holds for all $n \geq 1$. 
The distribution of distinct sets of prime exponents is still clearly quite regular since 
$\omega(n)$ and $\Omega(n)$ play a crucial role in the repetition of common values of 
$g^{-1}(n)$. 
The following observation is suggestive of the quasi-periodicity of the distribution of 
distinct values of this inverse function over $n \geq 2$: 

\begin{heuristic}[Symmetry in $g^{-1}(n)$ from the prime factorizations of $n \leq x$] 
\label{heuristic_SymmetryIngInvFuncs} 
Suppose that $n_1, n_2 \geq 2$ are such that their factorizations into distinct primes are 
given by $n_1 = p_1^{\alpha_1} \cdots p_r^{\alpha_r}$ and $n_2 = q_1^{\beta_1} \cdots q_r^{\beta_r}$. 
If $\{\alpha_1, \ldots, \alpha_r\} \equiv \{\beta_1, \ldots, \beta_r\}$ as multisets of prime exponents, 
then $g^{-1}(n_1) = g^{-1}(n_2)$. For example, $g^{-1}$ has the same values on the squarefree integers 
with exactly one, two, three (and so on) prime factors 
(\cf Section \ref{subSection_AConnectionToDistOfThePrimes}).  
\end{heuristic} 

%\NBRef{A01-2020-04-26}
\begin{conjecture}[Characteristic properties of the inverse sequence] 
\label{lemma_gInv_MxExample} 
We have the following properties characterizing the 
Dirichlet inverse function $g^{-1}(n)$: 
\begin{itemize} 

\item[\textbf{(A)}] For all $n \geq 1$, $\operatorname{sgn}(g^{-1}(n)) = \lambda(n)$; 
\item[\textbf{(B)}] For all squarefree integers $n \geq 2$, we have that 
     \[
     |g^{-1}(n)| = \sum_{m=0}^{\omega(n)} \binom{\omega(n)}{m} \times m!; 
     \]
\item[\textbf{(C)}] If $n \geq 2$ and $\Omega(n) = k$ for some $k \geq 1$, then 
     \[
     2 \leq |g^{-1}(n)| \leq \sum_{j=0}^{k} \binom{k}{j} \times j!. 
     \]
\end{itemize} 
\end{conjecture} 

The signedness property in (A) is proved precisely in 
Proposition \ref{prop_SignageDirInvsOfPosBddArithmeticFuncs_v1}. 
A proof of (B) follows from 
Lemma \ref{lemma_AnExactFormulaFor_gInvByMobiusInv_v1} 
stated on page \pageref{lemma_AnExactFormulaFor_gInvByMobiusInv_v1}. 

The realization that the beautiful and remarkably simple combinatorial form of property (B) 
in Conjecture \ref{lemma_gInv_MxExample} holds for all squarefree $n \geq 1$ 
motivates our pursuit of simpler formulas for the inverse functions $g^{-1}(n)$ 
through the sums of auxiliary subsequences $C_k(n)$ with $k := \Omega(n)$ 
defined in Section \ref{Section_InvFunc_PreciseExpsAndAsymptotics}. 
That is, we observe a familiar formula for $g^{-1}(n)$ 
on an asymptotically dense infinite subset of integers (with density $\frac{6}{\pi^2}$), 
e.g., that holds for all squarefree $n \geq 2$, and then seek 
to extrapolate by proving there are regular tendencies of the distribution of this sequence viewed 
more generally over any $n \geq 2$. 

An exact expression for $g^{-1}(n)$ is given by 
\[
g^{-1}(n) = \lambda(n) \times \sum_{d|n} \mu^2\left(\frac{n}{d}\right) C_{\Omega(d)}(d), n \geq 1,  
\]
where the sequence $\lambda(n) C_{\Omega(n)}(n)$ has DGF $(P(s)+1)^{-1}$ for $\Re(s) > 1$ 
(see Proposition \ref{prop_SignageDirInvsOfPosBddArithmeticFuncs_v1}). 
The function $C_{\Omega(n)}(n)$ is previously considered in 
\cite{FROBERG-1968} with its exact formula 
given by (\cf \cite{CLT-RANDOM-ORDERED-FACTS-2011}) 
\[
C_{\Omega(n)}(n) = \begin{cases}
     1, & \text{if $n = 1$; } \\ 
     (\Omega(n))! \times \prod\limits_{p^{\alpha}||n} \frac{1}{\alpha!}, & \text{if $n \geq 2$. }
     \end{cases}
\]
In Corollary \ref{cor_ExpectationFormulaAbsgInvn_v2}, we prove that 
for some absolute constant $A_0 > 0$, the average order of the unsigned sequence is 
\[
\mathbb{E}|g^{-1}(n)| = \frac{12A_0}{\pi} \cdot \frac{(\log n)^2}{\sqrt{\log\log n}} (1+o(1)),  
     \mathrm{\ as\ } n \rightarrow \infty. 
\]
In Section \ref{subSection_ErdosKacTheorem_Analogs}, 
we prove a variant of the Erd\H{o}s-Kac theorem 
that characterizes the distribution of the sequence $C_{\Omega(n)}(n)$. 
This leads us to conclude the following statement for any fixed $Y > 0$, with 
$\mu_x(C) := \log\log x - \log\left(4A_0\sqrt{2\pi}\right)$ and 
$\sigma_x(C) := \sqrt{\log\log x}$, 
that holds uniformly for any $-Y \leq y \leq Y$ 
as $x \rightarrow \infty$ (see Corollary \ref{cor_CLT_VII}): 
\begin{align*}
\frac{1}{x} \times & \#\left\{2 \leq n \leq x:|g^{-1}(n)| - 
     \frac{6}{\pi^2} \mathbb{E}|g^{-1}(n)| \leq y\right\} \\ & \qquad = 
     \Phi\left\{\frac{6 \sigma_x(C)}{\pi^2}\left(\frac{\pi^2 y}{6} + \sigma_x(C)\right) - 
     \frac{6}{\pi^2} \log\left(4A_0\sqrt{2\pi}\right)\right\} + 
     O\left(\frac{1}{\sqrt{\log\log x}}\right). 
\end{align*}
The regularity and quasi-periodicity we have alluded 
to in the remarks above are then 
quantifiable in so much as the distribution of $|g^{-1}(n)|$ for $n \leq x$ 
tends to (a predictable multiple of) 
its average order with a normal tendency 
depending on $x$ as $x \rightarrow \infty$. 
That is, if $x > e$ is sufficiently large and 
if we pick any integer $n \in [2, x]$ uniformly at random, then 
the following statement holds: 
\begin{align*} 
%\tag{D}
%\mathbb{P}\left(|g^{-1}(n)| - \frac{6}{\pi^2} \mathbb{E}|g^{-1}(n)| \leq 
%     \frac{6}{\pi^2}(\log\log x)  
%     \right) & = \frac{1}{2} + o(1) \\ 
\tag{D} 
\mathbb{P}\left(|g^{-1}(n)| - \frac{6}{\pi^2} \mathbb{E}|g^{-1}(n)| \leq 
     \frac{6}{\pi^2}\left(\alpha + \log\log x\right)
     \right) & = 
     \Phi\left(\alpha\right) + o(1), \alpha \in \mathbb{R}. 
\end{align*} 
It follows from the last property that as $n \rightarrow \infty$, 
$$|g^{-1}(n)| \leq \frac{6}{\pi^2} \mathbb{E}|g^{-1}(n)|(1+o(1)),$$
on an infinite set of the integers with asymptotic density one over the 
positive integers. 

\subsubsection{Formulas illustrating the new characterizations of $M(x)$} 

Let the summatory function 
$G^{-1}(x) := \sum_{n \leq x} g^{-1}(n)$ for integers $x \geq 1$ 
\cite[\seqnum{A341472}]{OEIS}. 
We prove that (see Proposition \ref{prop_Mx_SBP_IntegralFormula}) 
\begin{align} 
\label{eqn_Mx_gInvnPixk_formula_v2} 
M(x) & = G^{-1}(x) + G^{-1}\left(\Floor{x}{2}\right) + 
     \sum_{k=1}^{\frac{x}{2}-1} G^{-1}(k) \left[ 
     \pi\left(\Floor{x}{k}\right) - \pi\left(\Floor{x}{k+1}\right) 
     \right], x \geq 1. 
\end{align} 
This formula 
implies that we can establish new asymptotic bounds on 
$M(x)$ along infinite subsequences
by sharply bounding the summatory function $G^{-1}(x)$ at those points. 
The take on the regularity of $|g^{-1}(n)|$ is as such imperative to our arguments 
that formally bound the growth 
of $M(x)$ by its new identification with $G^{-1}(x)$. 
An initial combinatorial approach to summing $G^{-1}(x)$ for large $x$ based on the distribution of the primes 
is outlined in our remarks in Section \ref{subSection_AConnectionToDistOfThePrimes}. 

Theorem \ref{cor_ExprForGInvxByLx_v1} proves that for almost every sufficiently large $x$ 
there exists some $1 \leq t_0 \leq x$ such that\footnote{
     By the terminology 
     \emph{almost every} large integer $x$, we mean that the result holds for all large $x$ 
     taken within an infinite subset of $\mathbb{Z}^{+}$ with asymptotic density one. 
} 
\[
G^{-1}(x) = O\left(L(t_0) \times \mathbb{E}|g^{-1}(x)|\right).
\]
If the RH is true, then 
for every $\epsilon > 0$ and all sufficiently large $x$ we have that  
\[
G^{-1}(x) = O\left(\frac{\sqrt{x} (\log x)^2}{\sqrt{\log\log x}} \times \exp\left(
     \sqrt{\log x} (\log\log x)^{\frac{5}{2}+\epsilon}\right) 
     \right). 
\]
In Corollary \ref{cor_IntFormulaGInvx_for_Mx_v1}, 
we also prove that 
\begin{align*}
M(x) & = O\left(G^{-1}(x) + G^{-1}\left(\frac{x}{2}\right) + 
     \frac{x}{\log x} \times \sum_{k \leq \sqrt{x}} \frac{G^{-1}(k)}{k^2} 
     + (\log x)^2 \sqrt{\log\log x}\right). 
\end{align*} 
A complete discussion of the properties of the summatory functions 
$G^{-1}(x)$ motivates more study in future work to 
extend the full range of possibilities for viewing 
the new structure behind $M(x)$ we identify within this article. 
The prime-related combinatorics at hand are again discussed 
in Section \ref{subSection_AConnectionToDistOfThePrimes}. 

%\subsection{Newer analytic methods utilized in the proofs within the article}
%\begin{remark}[Proofs of uniform asymptotics from bivariate counting DGFs] 
%\label{remark_MV_NewDGFApplications} 
%We emphasize the modern method demonstrated by 
%Montgomery and Vaughan in constructing their original proof of 
%Theorem \ref{theorem_HatPi_ExtInTermsOfGz} (stated below). 
%To the best of our knowledge, this textbook reference is 
%one of the first clear cut applications documenting something of a hybrid 
%DGF-and-OGF type approach to enumerating sequences of arithmetic functions 
%and their summatory functions. 
%This interpretation of certain bivariate DGFs 
%offers a window into the best of both generating function type worlds. 
%It combines the additivity 
%implicit to the coefficients indexed by a formal power series variable formed by 
%multiplication of these structures, while coordinating the distinct DGF-best 
%property of the multiplicativity with respect to distinct prime powers invoked 
%by taking powers of a reciprocal Euler type product over the primes. 
%That is, this unique method invokes properties of certain infinite products 
%over the primes that form both a sequence DGF in $s$ and a formal power series 
%in $z$ by which we can also index coefficients in these expansions. 
%We give a proof constructed from this type of bivariate power series 
%DGF in Section \ref{Section_NewFormulasForgInvn}. 
%\end{remark} 

\subsection{Notation and conventions}

The next listing provides a glossary of common notation, conventions and 
abbreviations used throughout the article. 

\renewcommand{\glossarysection}[2][]{}
\printglossary[type={symbols},
               style={glossstyleSymbol},
               nogroupskip=true]

\newpage 
\section{Initial elementary proofs of new results} 
\label{Section_PrelimProofs_Config} 

\subsection{Establishing the summatory function properties and inversion identities} 

We will offer a proof of Theorem \ref{theorem_SummatoryFuncsOfDirCvls} 
suggested by an intuitive construction through matrix based methods in this section. 
Related results on summations of Dirichlet convolutions and their inversion appear in 
\cite[\S 2.14; \S 3.10; \S 3.12; \cf \S 4.9, p.\ 95]{APOSTOLANUMT}. 
It is similarly not difficult to prove the identity that
\[
\sum_{n \leq x} h(n) (f \ast g)(n) = 
     \sum_{n \leq x} f(n) \times \sum_{k \leq \Floor{x}{n}} g(k) h(kn). 
\]

\begin{proof}[Proof of Theorem \ref{theorem_SummatoryFuncsOfDirCvls}] 
\label{proofOf_theorem_SummatoryFuncsOfDirCvls} 
Let $h,g$ be arithmetic functions such that $g(1) \neq 0$. 
Denote the summatory functions of $h$ and $g$, 
respectively, by $H(x) = \sum_{n \leq x} h(n)$ and $G(x) = \sum_{n \leq x} g(n)$. 
We define $\pi_{g \ast h}(x)$ to be the summatory function of the 
Dirichlet convolution of $g$ with $h$. 
We have that the following formulas hold for all $x \geq 1$: 
\begin{align} 
\notag 
\pi_{g \ast h}(x) & := \sum_{n=1}^{x} \sum_{d|n} g(n) h(n/d) = \sum_{d=1}^x g(d) H\left(\floor{\frac{x}{d}}\right) \\ 
\label{eqn_proof_tag_PigAsthx_ExactSummationFormula_exp_v2} 
     & = \sum_{i=1}^x \left[G\left(\floor{\frac{x}{i}}\right) - G\left(\floor{\frac{x}{i+1}}\right)\right] H(i). 
\end{align} 
The first formula above is well known in the references. The second formula is justified directly using 
summation by parts as \cite[\S 2.10(ii)]{NISTHB} 
\begin{align*} 
\pi_{g \ast h}(x) & = \sum_{d=1}^x h(d) G\left(\floor{\frac{x}{d}}\right) \\ 
     & = \sum_{i \leq x} \left(\sum_{j \leq i} h(j)\right) \times 
     \left[G\left(\floor{\frac{x}{i}}\right) - 
     G\left(\floor{\frac{x}{i+1}}\right)\right]. 
\end{align*} 
We form the invertible matrix of coefficients $\widehat{G}$ 
associated with this linear system defining $H(j)$ for all 
$1 \leq j \leq x$ in \eqref{eqn_proof_tag_PigAsthx_ExactSummationFormula_exp_v2} by setting 
\[
g_{x,j} := G\left(\floor{\frac{x}{j}}\right) - G\left(\floor{\frac{x}{j+1}}\right) \equiv G_{x,j} - G_{x,j+1}, 
\] 
where 
\[
G_{x,j} := G\left(\Floor{x}{j}\right), 1 \leq j \leq x. 
\]
Since $g_{x,x} = G(1) = g(1)$ and $g_{x,j} = 0$ for all $j > x$, 
the matrix $\widehat{G}$ we have defined in this problem is lower triangular with a non-zero 
constant on its diagonals, and is hence invertible. 
If we let $\hat{G} := (G_{x,j})$, then this matrix is 
expressed by applying an invertible shift operation as 
\[
(g_{x,j}) = \hat{G} (I - U^{T}). 
\]
The square matrix $U$ of sufficiently large finite dimensions $N \times N$ 
has $(i,j)^{th}$ entries for all $1 \leq i,j \leq N$ that are defined by 
$(U)_{i,j} = \delta_{i+1,j}$ such that 
\[
\left[(I - U^T)^{-1}\right]_{i,j} = \Iverson{j \leq i}. 
\]
Observe that 
\[
\Floor{x}{j} - \Floor{x-1}{j} = \begin{cases} 
     1, & \text{ if $j|x$; } \\ 
     0, & \text{ otherwise. } 
     \end{cases} 
\] 
The previous property implies that 
\begin{equation} 
\label{eqn_proof_tag_FloorFuncDiffsOfSummatoryFuncs_v2} 
G\left(\floor{\frac{x}{j}}\right) - G\left(\floor{\frac{x-1}{j}}\right) = 
     \begin{cases} 
     g\left(\frac{x}{j}\right), & \text{ if $j | x$; } \\ 
     0, & \text{ otherwise. } 
     \end{cases}
\end{equation} 
We use the last property in \eqref{eqn_proof_tag_FloorFuncDiffsOfSummatoryFuncs_v2} 
to shift the matrix $\hat{G}$, and then invert the result to obtain a matrix involving the 
Dirichlet inverse of $g$ in the following forms: 
\begin{align*} 
\left[(I-U^{T}) \hat{G}\right]^{-1} & = \left(g\left(\frac{x}{j}\right) \Iverson{j|x}\right)^{-1} = 
     \left(g^{-1}\left(\frac{x}{j}\right) \Iverson{j|x}\right). 
\end{align*} 
In particular, our target matrix in the inversion problem is defined by 
$$(g_{x,j}) = (I-U^{T}) \left(g\left(\frac{x}{j}\right) \Iverson{j|x}\right) (I-U^{T})^{-1}.$$
We can express its inverse by a similarity transformation conjugated by shift operators as 
\begin{align*} 
(g_{x,j})^{-1} & = (I-U^{T})^{-1} \left(g^{-1}\left(\frac{x}{j}\right) \Iverson{j|x}\right) (I-U^{T}) \\ 
     & = \left(\sum_{k=1}^{\floor{\frac{x}{j}}} g^{-1}(k)\right) (I-U^{T}) \\ 
     & = \left(\sum_{k=1}^{\floor{\frac{x}{j}}} g^{-1}(k) - \sum_{k=1}^{\floor{\frac{x}{j+1}}} g^{-1}(k)\right). 
\end{align*} 
Hence, the summatory function $H(x)$ is given exactly for any integers $x \geq 1$ 
by a vector product with the inverse matrix from the previous equation by 
\begin{align*} 
H(x) & = \sum_{k=1}^x \left(\sum_{j=\floor{\frac{x}{k+1}}+1}^{\floor{\frac{x}{k}}} g^{-1}(j)\right) 
     \times \pi_{g \ast h}(k). 
\end{align*} 
We can prove another inversion formula providing the coefficients of the summatory function 
$G^{-1}(j)$ for $1 \leq j \leq x$ from the last equation by adapting our argument to prove 
\eqref{eqn_proof_tag_PigAsthx_ExactSummationFormula_exp_v2} above. 
This leads to the following equivalent identity expressing $H(x)$: 
\[
H(x) = \sum_{k=1}^{x} g^{-1}(x) \pi_{g \ast h}\left(\Floor{x}{k}\right). 
     \qedhere 
\]
\end{proof} 

\subsection{Proving the characteristic signedness property of $g^{-1}(n)$} 

Let $\chi_{\mathbb{P}}$ denote the characteristic function of the primes, let 
$\varepsilon(n) = \delta_{n,1}$ be the multiplicative identity with respect to Dirichlet convolution, 
and denote by $\omega(n)$ the strongly additive function that counts the number of 
distinct prime factors of $n$ (without multiplicity). We can easily prove using 
elementary methods that 
\begin{equation}
\label{eqn_AntiqueDivisorSumIdent} 
\chi_{\mathbb{P}} + \varepsilon = (\omega + 1) \ast \mu. 
\end{equation} 
In particular, since $\mu \ast 1 = \varepsilon$ and 
\[
\omega(n) = \sum_{p|n} 1 = \sum_{d|n} \chi_{\mathbb{P}}(d), n \geq 1, 
\]
the result in \eqref{eqn_AntiqueDivisorSumIdent} follows by M\"obius inversion. 
When combined with Corollary \ref{cor_CvlGAstMu} 
this convolution identity yields the exact 
formula for $M(x)$ stated in \eqref{eqn_Mx_gInvnPixk_formula} of 
Corollary \ref{cor_Mx_gInvnPixk_formula}. 

\begin{prop}[The signedness of $g^{-1}(n)$]
\label{prop_SignageDirInvsOfPosBddArithmeticFuncs_v1} 
Let the operator 
$\operatorname{sgn}(h(n)) = \frac{h(n)}{|h(n)| + \Iverson{h(n) = 0}} \in \{0, \pm 1\}$ denote the sign 
of the arithmetic function $h$ at integers $n \geq 1$. 
For the Dirichlet invertible function $g(n) := \omega(n) + 1$, 
we have that $\operatorname{sgn}(g^{-1}(n)) = \lambda(n)$ for all $n \geq 1$. 
%\NBRef{A02-2020-04-26}
\end{prop} 
\begin{proof} 
The function $D_f(s) := \sum_{n \geq 1} f(n) n^{-s}$ defines the 
\emph{Dirichlet generating function} (DGF) of any 
arithmetic function $f(n)$ which is convergent for all $s \in \mathbb{C}$ satisfying 
$\Re(s) > \sigma_f$ with $\sigma_f$ the abscissa of convergence of the series. 
Recall that $D_1(s) = \zeta(s)$, $D_{\mu}(s) = \zeta(s)^{-1}$ and $D_{\omega}(s) = P(s) \zeta(s)$ for 
$\Re(s) > 1$. 
Then by \eqref{eqn_AntiqueDivisorSumIdent} and the known property that whenever $f(1) \neq 0$, 
the DGF of $f^{-1}(n)$ is 
the reciprocal of the DGF of the arithmetic function $f$, 
we have for all $\Re(s) > 1$ that 
\begin{align} 
\label{eqn_DGF_of_gInvn} 
D_{(\omega+1)^{-1}}(s) = \frac{1}{(P(s)+1) \zeta(s)}. 
\end{align} 
It follows that $(\omega + 1)^{-1}(n) = (h^{-1} \ast \mu)(n)$ when we take 
$h := \chi_{\mathbb{P}} + \varepsilon$. 
We first show that $\operatorname{sgn}(h^{-1}) = \lambda$. 
This observation then implies 
that $\operatorname{sgn}(h^{-1} \ast \mu) = \lambda$. 

By the recurrence relation that defines the Dirichlet inverse function of any 
arithmetic function $h$ such that $h(1) = 1$, we have that \cite[\S 2.7]{APOSTOLANUMT} 
\begin{equation} 
\label{eqn_proof_tag_hInvn_ExactRecFormula_v1}
h^{-1}(n) = \begin{cases} 
            1, & n = 1; \\ 
            -\sum\limits_{\substack{d|n \\ d>1}} h(d) h^{-1}\left(\frac{n}{d}\right), & n \geq 2. 
            \end{cases} 
\end{equation} 
For $n \geq 2$, the summands in \eqref{eqn_proof_tag_hInvn_ExactRecFormula_v1} 
can be indexed over only the primes $p|n$ given our definition of $h$ from above. 
We can inductively 
unfold these sums into nested divisor sums provided the depth of the 
expanded divisor sums does not exceed the 
capacity to index non-trivial summations over the primes dividing $n$. 
Namely, notice that for $n \geq 2$ 
\begin{align*} 
h^{-1}(n) & = -\sum_{p|n} h^{-1}\left(\frac{n}{p}\right), && \text{\ if\ } \Omega(n) = 1; \\ 
     & = \sum_{p_1|n} \sum_{p_2|\frac{n}{p_1}} h^{-1}\left(\frac{n}{p_1p_2}\right), && \text{\ if\ } \Omega(n) = 2; \\ 
     & = -\sum_{p_1|n} \sum_{p_2|\frac{n}{p_1}} \sum_{p_3|\frac{n}{p_1p_2}} h^{-1}\left(\frac{n}{p_1p_2p_3}\right), 
     && \text{\ if\ } \Omega(n) = 3. 
\end{align*} 
Then by induction with $h^{-1}(1) = h(1) = 1$, we expand these 
nested divisor sums as above to the maximal possible depth as 
\begin{equation} 
\label{eqn_proof_tag_hInvn_ExactNestedSumFormula_v2} 
\lambda(n) \times h^{-1}(n) = \sum_{p_1|n} \sum_{p_2|\frac{n}{p_1}} \times \cdots \times 
     \sum_{p_{\Omega(n)}|\frac{n}{p_1p_2 \cdots p_{\Omega(n)-1}}} 1, n \geq 2. 
\end{equation} 
Moreover, by a combinatorial argument related to multinomial coefficient expansions of the sums in 
\eqref{eqn_proof_tag_hInvn_ExactNestedSumFormula_v2}, we recover exactly that 
(\cf \cite[\S 2]{FROBERG-1968}) 
\begin{equation} 
\label{eqn_proof_tag_hInvn_ExactNestedSumFormula_CombInterpetIdent_v3} 
h^{-1}(n) = \lambda(n) (\Omega(n))! \times \prod_{p^{\alpha} || n} \frac{1}{\alpha!}, n \geq 2. 
\end{equation} 
The last two expansions imply that  
$\operatorname{sgn}(h^{-1}(n)) = \lambda(n)$ for all $n \geq 1$. 
Since $\lambda$ is completely multiplicative we have that 
$\lambda\left(\frac{n}{d}\right) \lambda(d) = \lambda(n)$ for all divisors 
$d|n$ when $n \geq 1$. We also know that $\mu(n) = \lambda(n)$ whenever $n$ is squarefree, 
so that we obtain the following result: 
\[
g^{-1}(n) = (h^{-1} \ast \mu)(n) = \lambda(n) \times \sum_{d|n} \mu^2\left(\frac{n}{d}\right) |h^{-1}(n)|, n \geq 1. 
     \qedhere 
\]
\end{proof} 

The conclusion of the proof of 
Proposition \ref{prop_SignageDirInvsOfPosBddArithmeticFuncs_v1} 
in fact implies the stronger result that 
\[
g^{-1}(n) = \lambda(n) \times \sum_{d|n} \mu^2\left(\frac{n}{d}\right) C_{\Omega(d)}(d).  
\]
We have adopted the notation that for $n \geq 2$, 
$C_{\Omega(n)}(n) = (\Omega(n))! \times \prod_{p^{\alpha} || n} \frac{1}{\alpha!}$, 
where the same function, $C_0(1)$, is taken to be one for $n := 1$ 
(see Section \ref{Section_InvFunc_PreciseExpsAndAsymptotics}). 

\subsection{Results on the distribution of exceptional values of $\omega(n)$ and $\Omega(n)$} 

The next theorems reproduced from \cite[\S 7.4]{MV} characterize the relative 
scarcity of the distributions of $\omega(n)$ and $\Omega(n)$ for $n \leq x$ such that 
$\omega(n),\Omega(n) > \log\log x$. 
Since $\mathbb{E}[\omega(n)] = \log\log n + B_1$ and 
$\mathbb{E}[\Omega(n)] = \log\log n + B_2$ for $B_1,B_2 \in (0, 1)$ 
absolute constants in each case, 
these results imply a regular tendency 
of these additive arithmetic functions towards their respective average orders. 

\begin{theorem}[Upper bounds on exceptional values of $\Omega(n)$ for large $n$] 
\label{theorem_MV_Thm7.20-init_stmt} 
For $x \geq 2$ and $r > 0$, let 
\begin{align*} 
A(x, r) & := \#\left\{n \leq x: \Omega(n) \leq r \log\log x\right\}, \\ 
B(x, r) & := \#\left\{n \leq x: \Omega(n) \geq r \log\log x\right\}. 
\end{align*} 
If $0 < r \leq 1$ and $x \geq 2$, then 
\[
A(x, r) \ll x (\log x)^{r-1 - r\log r}, \text{ \ as\ } x \rightarrow \infty. 
\]
If $1 \leq r \leq R < 2$ and $x \geq 2$, then 
\[
B(x, r) \ll_R x (\log x)^{r-1-r \log r}, \text{ \ as\ } x \rightarrow \infty. 
\]
\end{theorem} 

Theorem \ref{theorem_MV_Thm7.21-init_stmt} is a special case analog to the 
celebrated Erd\H{o}s-Kac theorem typically stated for the 
normally distributed values of the function $\omega(n)$ over $n \leq x$ as 
$x \rightarrow \infty$ \cite[\cf Thm.\ 7.21]{MV} \cite[\cf \S 1.7]{IWANIEC-KOWALSKI}. 

\begin{theorem}
\label{theorem_MV_Thm7.21-init_stmt} 
We have that as $x \rightarrow \infty$ 
\[
\#\left\{3 \leq n \leq x: \Omega(n) \leq \log\log n \right\} = 
     \frac{x}{2} + O\left(\frac{x}{\sqrt{\log\log x}}\right). 
\]
\end{theorem} 

\begin{theorem}[Montgomery and Vaughan]
\label{theorem_HatPi_ExtInTermsOfGz} 
Recall that for integers $k \geq 1$ and $x \geq 2$ we have defined 
$$\widehat{\pi}_k(x) := \#\{2 \leq n \leq x: \Omega(n)=k\}.$$ 
For $0 < R < 2$ we have uniformly for all $1 \leq k \leq R \log\log x$ that 
\[
\widehat{\pi}_k(x) = \mathcal{G}\left(\frac{k-1}{\log\log x}\right) \frac{x}{\log x} \cdot 
     \frac{(\log\log x)^{k-1}}{(k-1)!} \left[1 + O_R\left(\frac{k}{(\log\log x)^2}\right)\right], 
\]
where we define 
\[
\mathcal{G}(z) := \frac{1}{\Gamma(z+1)} \times 
     \prod_p \left(1-\frac{z}{p}\right)^{-1} \left(1-\frac{1}{p}\right)^z, 0 \leq |z| < R. 
\]
\end{theorem} 

\begin{remark} 
\label{remark_MV_Pikx_FuncResultsAnnotated_v1} 
We can extend the work in \cite{MV} on the distribution of $\Omega(n)$ to find 
analogous results bounding the distribution of $\omega(n)$. In particular, we have 
for $0 < R < 2$ that as $x \rightarrow \infty$ 
\begin{equation}
\label{eqn_Pikx_UniformAsymptoticsStmt_from_MV_v2} 
\pi_k(x) = \widetilde{\mathcal{G}}\left(\frac{k-1}{\log\log x}\right) 
     \frac{x}{\log x} \cdot \frac{(\log\log x)^{k-1}}{(k-1)!} \left[ 
     1 + O_R\left(\frac{k}{(\log\log x)^2}\right) 
     \right], 
     \mathrm{\ uniformly\ for\ } 1 \leq k \leq R\log\log x. 
\end{equation}
The analogous function to express these bounds for $\omega(n)$ is 
defined by $\widetilde{\mathcal{G}}(z) := \widetilde{F}(1, z) / \Gamma(1 + z)$ where 
we take 
\[
\widetilde{F}(s, z) := \prod_p \left(1 + \frac{z}{p^s-1}\right)^{-1} \left(1 - \frac{1}{p^s}\right)^{z}, 
     \Re(s) > \frac{1}{2}; |z| \leq R < 2. 
\]
Let the functions 
\begin{align*} 
C(x, r) & := \#\{n \leq x: \omega(n) \leq r \log\log x\} \\ 
D(x, r) & := \#\{n \leq x: \omega(n) \geq r \log\log x\}. 
\end{align*} 
Then we have upper bounds on these functions given by 
\begin{align*} 
C(x, r) & \ll x (\log x)^{r - 1 - r \log r}, \mathrm{\ uniformly\ for\ } 0 < r \leq 1, \\ 
D(x, r) & \ll_R x (\log x)^{r - 1 - r \log r}, \mathrm{\ uniformly\ for\ } 1 \leq r \leq R < 2.
\end{align*} 
\end{remark} 

\newpage
\section{Auxiliary sequences related to the Dirichlet inverse function $g^{-1}(n)$} 
\label{Section_InvFunc_PreciseExpsAndAsymptotics} 

The computational data given as Table \ref{table_conjecture_Mertens_ginvSeq_approx_values} 
in the appendix section (refer to 
page \pageref{table_conjecture_Mertens_ginvSeq_approx_values}) is intended to 
provide clear insight into why we eventually arrived at the stated formulas for  
$g^{-1}(n)$ proved in this section. The table provides illustrative 
numerical data by examining the approximate behavior 
at hand for the cases of $1 \leq n \leq 500$ with \emph{Mathematica} 
\cite{SCHMIDT-MERTENS-COMPUTATIONS}. 
In Section \ref{Section_NewFormulasForgInvn}, 
we will use these relations between $g^{-1}(n)$ and 
$C_{\Omega(n)}(n)$ to prove an exact Erd\H{o}s-Kac like analog that characterizes 
the distribution of the unsigned function $|g^{-1}(n)|$. 

\subsection{Definitions and properties of triangular component function sequences} 

We define the following sequence for integers $n \geq 1$ and $k \geq 0$: 
\begin{align} 
\label{eqn_CknFuncDef_v2} 
C_k(n) := \begin{cases} 
     \varepsilon(n), & \text{ if $k = 0$; } \\ 
     \sum\limits_{d|n} \omega(d) C_{k-1}(n/d), & \text{ if $k \geq 1$. } 
     \end{cases} 
\end{align} 

By recursively expanding the definition of $C_k(n)$ 
at any fixed $n \geq 2$, we see that 
we can form a chain of at most $\Omega(n)$ iterated (or nested) divisor sums by 
unfolding the definition of \eqref{eqn_CknFuncDef_v2} inductively. 
By the same argument, we see that at fixed $n$, the function 
$C_k(n)$ is seen to be non-zero only for positive integers 
$k \leq \Omega(n)$ whenever $n \geq 2$. 
A sequence of relevant signed semi-diagonals of the functions $C_k(n)$ begins as follows 
\cite[\seqnum{A008480}]{OEIS}: 
\[
\{\lambda(n) C_{\Omega(n)}(n) \}_{n \geq 1} = \{
     1, -1, -1, 1, -1, 2, -1, -1, 1, 2, -1, -3, -1, 2, 2, 1, -1, -3, -1, 
     -3, 2, 2, -1, 4, 1, 2, \ldots \}. 
\]
We can see that $C_{\Omega(n)}(n) \leq (\Omega(n))!$ for all $n \geq 1$ with equality precisely at the 
squarefree integers. In fact, 
$h^{-1}(n) \equiv \lambda(n) C_{\Omega(n)}(n)$ is the same function given by 
the formula in \eqref{eqn_proof_tag_hInvn_ExactNestedSumFormula_CombInterpetIdent_v3} from 
Proposition \ref{prop_SignageDirInvsOfPosBddArithmeticFuncs_v1}. 

\subsection{Relating the function $C_{\Omega(n)}(n)$ to exact formulas for $g^{-1}(n)$} 
\label{subSection_Relating_CknFuncs_to_gInvn} 

\begin{lemma}[An exact formula for $g^{-1}(n)$] 
\label{lemma_AnExactFormulaFor_gInvByMobiusInv_v1} 
For all $n \geq 1$, we have that 
\[
g^{-1}(n) = \sum_{d|n} \mu\left(\frac{n}{d}\right) \lambda(d) C_{\Omega(d)}(d). 
\]
\end{lemma}
\begin{proof} 
We first write out the standard recurrence relation for the Dirichlet inverse as 
\begin{align} 
\label{eqn_proof_tag_gInvCvlOne_EQ_omegaCvlgInvCvl_v1} 
g^{-1}(n) & = - \sum_{\substack{d|n \\ d>1}} (\omega(d) + 1) g^{-1}(n/d) 
     \quad\implies\quad 
     (g^{-1} \ast 1)(n) = -(\omega \ast g^{-1})(n). 
\end{align} 
We argue that for $1 \leq m \leq \Omega(n)$, we can inductively expand the 
implication on the right-hand-side of \eqref{eqn_proof_tag_gInvCvlOne_EQ_omegaCvlgInvCvl_v1} 
in the form of $(g^{-1} \ast 1)(n) = F_m(n)$ where 
$F_m(n) := (-1)^{m} (C_m(-) \ast g^{-1})(n)$, so that 
\[
F_m(n) = - 
     \begin{cases} 
     \sum\limits_{\substack{d|n \\ d > 1}} F_{m-1}(d) \times \sum\limits_{\substack{r|\frac{n}{d} \\ r > 1}} 
     \omega(r) g^{-1}\left(\frac{n}{dr}\right), & 2 \leq m \leq \Omega(n), \\ 
     (\omega \ast g^{-1})(n), & m = 1. 
     \end{cases} 
\]
By repeatedly expanding the right-hand-side of the previous equation, 
we find that when $m := \Omega(n)$ (i.e., with the expansions 
in the previous equation taken to a maximal depth) we get the relation 
\begin{equation} 
\label{eqn_proof_tag_gInvCvlOne_EQ_omegaCvlgInvCvl_v2} 
(g^{-1} \ast 1)(n) = (-1)^{\Omega(n)} C_{\Omega(n)}(n) = \lambda(n) C_{\Omega(n)}(n). 
\end{equation} 
The formula then follows from \eqref{eqn_proof_tag_gInvCvlOne_EQ_omegaCvlgInvCvl_v2} 
by M\"obius inversion applied to each side of the last equation. 
\end{proof} 

\begin{cor} 
\label{lemma_AbsValueOf_gInvn_FornSquareFree_v1} 
For all positive integers $n \geq 1$, we have that 
\begin{equation} 
\label{eqn_AbsValueOf_gInvn_FornSquareFree_v1} 
|g^{-1}(n)| = \sum_{d|n} \mu^2\left(\frac{n}{d}\right) C_{\Omega(d)}(d). 
\end{equation} 
\end{cor} 
\begin{proof} 
By applying 
Lemma \ref{lemma_AnExactFormulaFor_gInvByMobiusInv_v1}, 
Proposition \ref{prop_SignageDirInvsOfPosBddArithmeticFuncs_v1} and the 
complete multiplicativity of $\lambda(n)$, 
we easily obtain the stated result. 
In particular, since $\mu(n)$ is non-zero only at squarefree integers and since 
at any squarefree $d \geq 1$ we have $\mu(d) = (-1)^{\omega(d)} = \lambda(d)$, 
Lemma \ref{lemma_AnExactFormulaFor_gInvByMobiusInv_v1} implies 
\begin{align*} 
|g^{-1}(n)| & = \lambda(n) \times \sum_{d|n} \mu\left(\frac{n}{d}\right) \lambda(d) C_{\Omega(d)}(d) \\ 
     & = \sum_{d|n} \mu^2\left(\frac{n}{d}\right) \lambda\left(\frac{n}{d}\right) 
     \lambda(nd) C_{\Omega(d)}(d) \\ 
     & = \lambda(n^2) \times \sum_{d|n} \mu^2\left(\frac{n}{d}\right) C_{\Omega(d)}(d). 
\end{align*} 
We see that 
that $\lambda(n^2) = +1$ for all $n \geq 1$ since the number of distinct 
prime factors (counting multiplicity) of any square integer is even. 
\end{proof} 

Since $C_{\Omega(n)}(n) = |h^{-1}(n)|$ using the notation defined in the the proof of 
Proposition \ref{prop_SignageDirInvsOfPosBddArithmeticFuncs_v1}, we can see that 
$C_{\Omega(n)}(n) = (\omega(n))!$ for all squarefree $n \geq 1$. 
A proof of part (B) of Conjecture \ref{lemma_gInv_MxExample} 
follows as an immediate consequence. 
%Moreover, since $\Omega(n) \geq \omega(n)$ for all $n \geq 1$ 
%\cite[\cf \S 2.4]{MV}, the claim in (C) of the same 
%conjecture follows as well. 

\begin{remark} 
Combined with the signedness property of $g^{-1}(n)$ guaranteed by 
Proposition \ref{prop_SignageDirInvsOfPosBddArithmeticFuncs_v1}, 
Corollary \ref{lemma_AbsValueOf_gInvn_FornSquareFree_v1} shows that the summatory 
function of this sequence satisfies 
\[
G^{-1}(x) = \sum_{d \leq x} \lambda(d) C_{\Omega(d)}(d) M\left(\Floor{x}{d}\right). 
\]
Additionally, equation \eqref{eqn_AntiqueDivisorSumIdent} implies that 
$$\lambda(d) C_{\Omega(d)}(d) = (g^{-1} \ast 1)(d) = (\chi_{\mathbb{P}} + \varepsilon)^{-1}(d).$$ 
We clearly recover by inversion that 
\[
M(x) = G^{-1}(x) + \sum_{p \leq x} G^{-1}\left(\Floor{x}{p}\right), x \geq 1. 
\]
\end{remark} 

\subsection{Another combinatorial connection to the distribution of the primes} 
\label{subSection_AConnectionToDistOfThePrimes} 

The combinatorial properties of $g^{-1}(n)$ are deeply tied to the distribution of the primes 
$p \leq n$ as $n \rightarrow \infty$. 
The magnitudes and dispersion of the primes $p \leq n$ certainly restricts the 
repeating of these distinct sequence values. 
Nonetheless, we can see that the following 
is still clear about the relation of the weight functions $|g^{-1}(n)|$ to the 
distribution of the primes: 
The value of $|g^{-1}(n)|$ is entirely dependent \emph{only on the pattern of the exponents} 
(viewed as multisets) of the distinct prime factors of $n \geq 2$, rather than on the 
prime factor weights themselves 
(\cf Heuristic \ref{heuristic_SymmetryIngInvFuncs}). 
This observation implies that $|g^{-1}(n)|$ has an inherently additive, rather than 
multiplicative, structure behind the distribution of its distinct values over $n \leq x$. 
The results we obtain at the conclusion of 
Section \ref{Section_NewFormulasForgInvn} are then 
\'{a} priori more intuitive as strongly additive functions are known to have explicit 
probability measures and central limit type theorems 
that characterize their distributions over $n \leq x$ as $x \rightarrow \infty$ 
\cite[\cf Thm.~1.5; \S 1.7]{IWANIEC-KOWALSKI}. 

\begin{example} 
We have a natural extremal behavior with respect to distinct values of $\Omega(n)$ 
corresponding to squarefree integers and prime powers. If for integers 
$k \geq 1$ we define the 
infinite sets $M_k$ and $m_k$ to correspond to the maximal (minimal) sets of 
positive integers such that 
\begin{align*} 
M_k & := \left\{n \geq 2: |g^{-1}(n)| = \underset{{\substack{j \geq 2 \\ \Omega(j) = k}}}{\operatorname{sup}} 
     |g^{-1}(j)|\right\} \subseteq \mathbb{Z}^{+}, \\  
m_k & := \left\{n \geq 2: |g^{-1}(n)| = \underset{{\substack{j \geq 2 \\ \Omega(j) = k}}}{\operatorname{inf}} 
     |g^{-1}(j)|\right\} \subseteq \mathbb{Z}^{+}, 
\end{align*} 
then any element of $M_k$ is squarefree and any element of $m_k$ is a prime power. 
Moreover, for any fixed $k \geq 1$ 
we have that for any $N_k \in M_k$ and $n_k \in m_k$
\[
(-1)^{k} g^{-1}(N_k) = \sum_{j=0}^{k} \binom{k}{j} j!, 
     \quad \mathrm{\ and\ } \quad 
     (-1)^{k} g^{-1}(n_k) = 2. 
\]
\end{example}

\begin{remark}[Combinatorial properties]
The formula for the function $h^{-1}(n) = (g^{-1} \ast 1)(n)$ defined in the proof of 
Proposition \ref{prop_SignageDirInvsOfPosBddArithmeticFuncs_v1} implies that we can express 
an exact formula for $g^{-1}(n)$ in terms of symmetric polynomials in the 
exponents of the prime factorization of $n$. 
For $n \geq 2$ and $0 \leq k \leq \omega(n)$ let 
\[
\widehat{e}_k(n) := [z^k] \prod_{p|n} (1 + z \nu_p(n)) = [z^k] \prod_{p^{\alpha} || n} (1 + \alpha z). 
\]
Then we can prove using 
\eqref{eqn_proof_tag_hInvn_ExactNestedSumFormula_CombInterpetIdent_v3} and 
\eqref{eqn_AbsValueOf_gInvn_FornSquareFree_v1} that we can expand exact formulas for 
the signed inverse sequence in the following form: 
\[
g^{-1}(n) = h^{-1}(n) \times \sum_{k=0}^{\omega(n)} \binom{\Omega(n)}{k}^{-1} 
     \frac{\widehat{e}_k(n)}{k!}, n \geq 2. 
\]
The combinatorial formula for 
$h^{-1}(n) = \lambda(n) (\Omega(n))! \times \prod_{p^{\alpha} || n} (\alpha !)^{-1}$ 
we discovered in the proof of the proposition from 
Section \ref{Section_PrelimProofs_Config}\footnote{ 
     This sequence is also considered using a different motivation based on the DGFs 
     $(1\pm P(s))^{-1}$ in \cite[\S 2]{FROBERG-1968}. 
}  
suggests additional patterns and more regularity in the contributions of the distinct sign weighted 
terms in the summands of $G^{-1}(x)$. 
A preliminary analysis suggests that bounds of this type 
will improve on those we are able to prove within this article for $G^{-1}(x)$ in 
Section \ref{Section_ProofOfValidityOfAverageOrderLowerBounds}. 
\end{remark}

%In particular, we give the next ansatz which relies upon an unproven assertion 
%about the asymptotic main term of certain sums which we concretely identify below. 
%This ansatz is intended as an example that can be later refined to bound the 
%summatory function $G^{-1}(x)$ in meaningful new ways by exploiting the combinatorial 
%properties of the prime exponent patterns that lead to duplicate values of 
%the sequence $g^{-1}(n)$ for $n \leq x$. 
%\begin{ansatz}[Towards a counting argument based bound for $G^{-1}(x)$]
%For $m,r \geq 1$, let $n_{m,r}$ denote any integer (not uniquely) 
%such that $\Omega(n) = m$ and $\omega(n) = r$. 
%We estimate using a conditional probability formula that 
%\begin{align*}
%\#\{2 \leq n \leq x: \Omega(n)=m \wedge \omega(n)-r\} & = x \cdot 
%     \mathbb{P}(\Omega(n)=m|\omega(n)=r) \mathbb{P}(\omega(n)=r) 
%     \approx \frac{1}{x} \widehat{\pi}_m(x) \pi_r(x) \\ 
%     & \sim \frac{x}{(\log x)^2} \cdot \frac{(\log\log x)^{m+r-2}}{(m-1)! (r-1)!}, 
%     \text{ for any } 1 \leq m, r \leq \log\log x. 
%\end{align*}
%So to sum $G^{-1}(x)$, we surmise that its main term satisfies 
%\begin{align*}
%G^{-1}(x) & \asymp \sum_{1 \leq m \leq \log\log x} \sum_{1 \leq r \leq m} 
%     (-1)^m |g^{-1}(n_{m,r})| \cdot \frac{x}{(\log x)^2} \cdot \frac{(\log\log x)^{m+r-3}}{ 
%     (m-1)! (r-1)!} \\ 
%     & \approx \frac{2x}{(\log x)^2} \times \sum_{m=1}^{\log\log x} 
%          \frac{(-1)^m (\log\log x)^{2m-3}}{(m-1)!^2} \\ 
%     & \sim -\frac{2x}{(\log x)^2 (\log\log x)} \times I_0(-2\log\log x) \left[ 
%     1 + O\left(\frac{1}{\log\log x}\right)\right] \\ 
%     & \sim -\frac{x}{\sqrt{\pi} (\log\log x)^{3/2}} \left(1 + o(1)\right). 
%\end{align*}
%The second line above relies on an unproven assertion that we expect the dominant 
%term in the sums over $1 \leq r \leq m$ to correspond to the case where the power of 
%$(\log\log x)^{r}$ is the highest, e.g., when $r = m$. 
%In computing the final estimate on the last line of the previous equation, 
%we have also asserted the known asymptotic estimate for the \emph{incomplete Bessel function}, 
%$I_0(z)$, which states that \cite[\S 10.40]{NISTHB}
%\[
%I_0(z) \sim \frac{e^z}{\sqrt{2\pi z}} (1+o(1)), \text{ as } |z| \rightarrow \infty. 
%\]
%\end{ansatz}

\newpage
\section{The distributions of $C_{\Omega(n)}(n)$ and $|g^{-1}(n)|$} 
\label{Section_NewFormulasForgInvn} 

We suggested in the introduction that the relation of the component 
functions, $g^{-1}(n)$ and $C_{\Omega(n)}(n)$, to the canonical additive functions 
$\omega(n)$ and $\Omega(n)$ leads to the regular properties of these functions 
cited in the numerical data from 
Table \ref{table_conjecture_Mertens_ginvSeq_approx_values}. 
Each of $\omega(n)$ and $\Omega(n)$ satisfies 
an Erd\H{o}s-Kac theorem that provides a central limit type theorem for the 
distributions of these functions over $n \leq x$ as $x \rightarrow \infty$ 
\cite{ERDOS-KAC-REF,BILLINGSLY-CLT-PRIMEDIVFUNC,RENYI-TURAN} 
(\cf \cite{CLT-RANDOM-ORDERED-FACTS-2011}). 
In the remainder of this section we establish more analytical proofs of 
related properties of these key sequences used to express the unweighted summands that define 
$G^{-1}(x)$. 

\subsection{Analytic proofs and adaptations of bivariate DGF methods for additive functions} 

\begin{theorem} 
\label{prop_HatAzx_ModSummatoryFuncExps_RelatedToCkn} 
Let the bivariate DGF $\widehat{F}(s, z)$ be defined in terms of the prime zeta function, $P(s)$,  
for $\Re(s) \geq 2$ and $|z| < |P(s)|^{-1}$ by 
\[
\widehat{F}(s, z) := \frac{1}{1+P(s) z} 
     \times \prod_p \left(1 - \frac{1}{p^s}\right)^{z}. 
\]
For $|z| < P(2)^{-1}$, the 
summatory function of the coefficients of 
$\widehat{F}(s, z) \zeta(s)^{z}$ corresponds to 
\[
\widehat{A}_z(x) := \sum_{n \leq x} (-1)^{\omega(n)} 
     C_{\Omega(n)}(n) z^{\Omega(n)}. 
\]
We have for all sufficiently large $x \geq 2$ and any $|z|< P(2)^{-1}$ that
\[
\widehat{A}_z(x) = \frac{x \widehat{F}(2, z)}{\Gamma(z)} \times (\log x)^{z-1} + 
     O_{z}\left(x (\log x)^{\Re(z) - 2}\right). 
\]
\end{theorem} 
\begin{proof} 
We see from the proof of 
Proposition \ref{prop_SignageDirInvsOfPosBddArithmeticFuncs_v1} 
that 
\[
C_{\Omega(n)}(n) = \begin{cases} 
     1, & n = 1; \\ 
     (\Omega(n))! \times \prod\limits_{p^{\alpha}||n} \frac{1}{\alpha!}, & n \geq 2. 
     \end{cases} 
\]
We can then generate exponentially scaled forms of these terms through a 
product identity of the following form: 
\begin{align*} 
\sum_{n \geq 1} \frac{C_{\Omega(n)}(n)}{(\Omega(n))!} \cdot 
     \frac{(-1)^{\omega(n)} z^{\Omega(n)}}{n^s} & = \prod_p \left(1 + \sum_{r \geq 1} 
     \frac{z^{\Omega(p^r)}}{r! p^{rs}}\right)^{-1} 
     = \exp\left(-z P(s)\right), \Re(s) \geq 2 \wedge \Re(P(s)z) > -1. 
\end{align*} 
This Euler-type product based expansion is similar in construction to the parameterized bivariate 
DGF defined by Montgomery and Vaughan in \cite[\S 7.4]{MV}.
By computing a termwise Laplace transform on the right-hand-side of the 
above equation, we obtain that 
\begin{align*} 
\sum_{n \geq 1} \frac{C_{\Omega(n)}(n) (-1)^{\omega(n)} z^{\Omega(n)}}{n^s} & = 
     \int_0^{\infty} e^{-t} \exp\left(-tz P(s)\right) dt = \frac{1}{1 + P(s) z}, 
     \Re(s) > 1 \wedge \Re(P(s)z) > -1. 
\end{align*} 
It follows that 
\[
\sum_{n \geq 1} \frac{(-1)^{\omega(n)} C_{\Omega(n)}(n) z^{\Omega(n)}}{n^s} = 
     \zeta(s)^z \times \widehat{F}(s, z), \Re(s) > 1 \wedge |z| < |P(s)|^{-1}. 
\]
Since $\widehat{F}(s, z)$ is an analytic function of $s$ for all $\Re(s) \geq 2$ 
whenever the parameter $|z| < |P(s)|^{-1}$, 
if the sequence $\{b_z(n)\}_{n \geq 1}$ indexes the coefficients in 
the DGF expansion of $\widehat{F}(s, z) \zeta(s)^{z}$, then the series 
\[
\left\lvert \sum_{n \geq 1} \frac{b_z(n) (\log n)^{2R+1}}{n^s} \right\rvert < +\infty, 
     \Re(s) \geq 2, 
\]
is uniformly bounded for $|z| \leq R < |P(s)|^{-1} < +\infty$. This fact follows by repeated 
termwise differentiation of the DGF series $\ceiling{2R+1}$ times with respect to $s$. 

For fixed $0 < |z| < 2$, let the sequence $d_z(n)$ be generated as the coefficients of the DGF 
$$\zeta(s)^{z} = \sum_{n \geq 1} \frac{d_z(n)}{n^s}, \Re(s) > 1,$$ with corresponding 
summatory function defined by $D_z(x) := \sum_{n \leq x} d_z(n)$. 
The theorem proved in the reference 
\cite[Thm.\ 7.17; \S 7.4]{MV} shows that for any $0 < |z| < 2$ 
and all integers $x \geq 2$ 
\[
D_z(x) = \frac{x (\log x)^{z-1}}{\Gamma(z)} + O_z\left(x (\log x)^{\Re(z)-2}\right). 
\]
We set 
$b_z(n) := (-1)^{\omega(n)} C_{\Omega(n)}(n) z^{\Omega(n)}$, define the convolution 
$a_z(n) := \sum_{d|n} b_z(d) d_z(n/d)$, and take its summatory function to be 
$A_z(x) := \sum_{n \leq x} a_z(n)$. 
Then we have that 
\begin{align} 
\notag 
A_z(x) & = \sum_{m \leq x/2} b_z(m) D_z(x/m) + \sum_{x/2 < m \leq x} b_z(m) \\ 
\label{eqn_proof_tag_Azx_FullTermsFormulaSum_v1} 
     & = \frac{x}{\Gamma(z)} \times \sum_{m \leq x/2} 
     \frac{b_z(m)}{m^2} \times m \log\left(\frac{x}{m}\right)^{z-1} + 
     O\left(\sum_{m \leq x} \frac{x |b_z(m)|}{m^2} \times m 
     \log\left(\frac{2x}{m}\right)^{\Re(z) - 2}\right). 
\end{align} 
We can sum the coefficients $b_z(m) / m$ 
for integers $m \leq u$ with $u \geq 2$ taken sufficiently large as follows: 
\begin{align*} 
\sum_{m \leq u} \frac{b_z(m)}{m} & = \left(\widehat{F}(2, z) + O(u^{-2})\right) u - \int_1^{u} 
     \left(\widehat{F}(2, z) + O(t^{-2})\right) dt 
     = \widehat{F}(2, z) + O(u^{-1}). 
\end{align*} 
Suppose that $0 < |z| \leq R < P(2)^{-1} \approx 2.21118$. 
For large $x$, the error term in \eqref{eqn_proof_tag_Azx_FullTermsFormulaSum_v1} satisfies 
\begin{align*} 
\sum_{m \leq x} \frac{x \cdot |b_z(m)|}{m^2} \times m 
     \log\left(\frac{2x}{m}\right)^{\Re(z) - 2} & \ll 
     x (\log x)^{\Re(z) - 2} \times \sum_{m \leq \sqrt{x}} \frac{|b_z(m)|}{m} \\ 
     & \phantom{\ll x\ } + 
     x (\log x)^{-(R+2)} \times \sum_{m > \sqrt{x}} \frac{|b_z(m)|}{m} (\log m)^{2R} \\ 
     & = O_z\left(x (\log x)^{\Re(z) - 2}\right), 0 < |z| \leq R. 
\end{align*} 
When $m \leq \sqrt{x}$ we have 
\[
\log\left(\frac{x}{m}\right)^{z-1} = (\log x)^{z-1} + 
     O\left((\log m) (\log x)^{\Re(z) - 2}\right). 
\]
The total sum over the interval $m \leq x/2$ corresponds to bounding the 
sum components when we take $0 < |z| \leq R$ as follows: 
\begin{align*} 
\sum_{m \leq x/2} b_z(m) D_z(x/m) & = \frac{x}{\Gamma(z)} (\log x)^{z-1} \times 
     \sum_{m \leq x/2} \frac{b_z(m)}{m} \\ 
     & \phantom{=\quad\ } + 
     O_R\left(x (\log x)^{\Re(z)-2} \times \sum_{m \leq \sqrt{x}} \frac{|b_z(m)| \log m}{m} + 
     x (\log x)^{R-1} \times \sum_{m > \sqrt{x}} \frac{|b_z(m)|}{m}\right) \\ 
     & = \frac{x \widehat{F}(2, z)}{\Gamma(z)} (\log x)^{z-1} + O_R\left( 
     x (\log x)^{\Re(z)-2} \times \sum_{m \geq 1} \frac{b_z(m) (\log m)^{2R+1}}{m^2} 
     \right) \\ 
     & = \frac{x \widehat{F}(2, z)}{\Gamma(z)} (\log x)^{z-1} + O_{R}\left( 
     x (\log x)^{\Re(z)-2}\right). 
     \qedhere  
\end{align*} 
\end{proof} 

\begin{theorem} 
\label{theorem_CnkSpCasesScaledSummatoryFuncs} 
For large $x > e$ and integers $k \geq 1$, let 
\[
\widehat{C}_{k,\ast}(x) := \sum_{\substack{n \leq x \\ \Omega(n) = k}} 
     (-1)^{\omega(n)} C_k(n) 
\]
Let the function $\widehat{G}(z) := \widehat{F}(2, z) / \Gamma(z+1)$ for 
$|z| < P(2)^{-1}$ where the function $\widehat{F}(s, z)$ is defined as 
in Theorem \ref{prop_HatAzx_ModSummatoryFuncExps_RelatedToCkn} for $\Re(s) > 1$. 
As $x \rightarrow +\infty$, we have uniformly for any $1 \leq k \leq 2\log\log x$ that 
\[
\widehat{C}_{k,\ast}(x) = -\widehat{G}\left(-\frac{k-1}{\log\log x}\right) \frac{x}{\log x} \cdot 
     \frac{(\log\log x)^{k-1}}{(k-1)!} \left[ 
     1 + O\left(\frac{k}{(\log\log x)^2}\right)\right]. 
\]
\end{theorem} 
\begin{proof} 
When $k = 1$, we have that $\Omega(n) = \omega(n)$ for all $n \leq x$ such that $\Omega(n) = k$. 
The $n \leq x$ that satisfy this requirement are precisely the primes $p \leq x$. 
Thus we get that the bound is satisfied as 
\[
\sum_{p \leq x} (-1)^{\omega(p)} C_1(p) = -\sum_{p \leq x} 1 = 
     - \frac{x}{\log x} \left[1 + O\left(\frac{1}{\log x}\right)\right]. 
\]
Since $O((\log x)^{-1}) = O((\log\log x)^{-2})$ as $x \rightarrow \infty$, 
we obtain the required error term bound when $k := 1$. 

For $2 \leq k \leq 2\log\log x$, we will apply the error estimate from 
Theorem \ref{prop_HatAzx_ModSummatoryFuncExps_RelatedToCkn} at 
$r := \frac{k-1}{\log\log x}$. 
At large $x$, the error from this bound contributes terms that are bounded from above by 
\begin{align*} 
x (\log x)^{-(r+2)} r^{-(k+1)} & \ll \frac{x}{(\log x)^2} \cdot 
     \frac{(\log\log x)^{k+1}}{(k-1)^{k+1}} \cdot \frac{1}{e^{k-1}} 
     \ll \frac{x}{(\log x)^2} \cdot \frac{(\log\log x)^{k+1}}{(k-1)^{3/2}} \cdot 
     \frac{1}{e^{2k} (k-1)!} \\ 
     & \ll \frac{x}{(\log x)^2} \cdot \frac{(\log\log x)^{k-1}}{(k-1)!} \ll 
     \frac{x}{\log x} \cdot \frac{k (\log\log x)^{k-5}}{(k-1)!}. 
\end{align*} 
We next find an asymptotically accurate main term approximation to the coefficients 
of the following contour integral for $r \in [0, z_{\max}] \subseteq \left[0, P(2)^{-1}\right)$ 
to satisfy Theorem \ref{prop_HatAzx_ModSummatoryFuncExps_RelatedToCkn}: 
\begin{align} 
\label{eqn_WideTildeArx_CountourIntDef_v1} 
\widetilde{A}_r(x) := 
     \frac{(-1)^k x}{\log x} 
     \times \int_{|v|=r} \frac{(\log x)^{-v} \zeta(2)^{v}}{\Gamma(1+v) 
     v^{k} (1 - P(2) v)} dv. 
\end{align} 
The main term for the sums $\widehat{C}_{k,\ast}(x)$ 
is given by $-\frac{x}{\log x} \times I_k(r, x)$, where we take 
\begin{align*}
I_k(r, x) & = \frac{(-1)^{k-1}}{2\pi\imath} \times \int_{|v|=r} 
     \frac{\widehat{G}(-v) (\log x)^{-v}}{v^k} dv \\ 
     & =: I_{1,k}(r, x) + I_{2,k}(r, x). 
\end{align*}
Taking $r = \frac{k-1}{\log\log x}$, the 
first of the component integrals in the last equation is defined to be 
\begin{align*}
I_{1,k}(r, x) & := \frac{(-1)^{k-1} \widehat{G}(-r)}{2\pi\imath} \times \int_{|v|=r} 
     \frac{(\log x)^{-v}}{v^k} dv = \widehat{G}(-r) \times \frac{(\log\log x)^{k-1}}{(k-1)!}. 
\end{align*}
The second component integral, $I_{2,k}(r, x)$, corresponds to error terms in our approximation 
that we must bound. This function is defined by 
\[
I_{2,k}(r, x) := \frac{(-1)^{k-1}}{2\pi\imath} \times \int_{|v|=r} (\widehat{G}(-r) - \widehat{G}(-v)) 
     \times \frac{(\log x)^{-v}}{v^k} dv. 
\]
After integrating by parts \cite[\cf Thm.\ 7.19; \S 7.4]{MV}, we write that 
\[
I_{2,k}(r, x) := \frac{(-1)^{k-1}}{2\pi\imath} \times \int_{|v|=r} (\widehat{G}(-r) - \widehat{G}(-v) - 
     \widehat{G}^{\prime}(-r)(v+r)) 
     (\log x)^{-v} v^{-k} dv. 
\]
Notice that 
\[
\left\lvert \widehat{G}(-r) - \widehat{G}(-v) - \widehat{G}^{\prime}(-r)(v+r) \right\rvert = 
     \left\lvert \int_{-r}^{-v} 
     (v+w) \widehat{G}^{\prime\prime}(-w) dw \right\rvert 
     \ll |v+r|^2. 
\]
With the parameterization $v = -re^{2\pi\imath\theta}$ for real $\theta \in [-1/2,1/2]$ and 
with $r := \frac{k-1}{\log\log x}$ (as before), 
we get that 
\[
|I_{2,k}(r, x)| \ll r^{3-k} \int_{-1/2}^{1/2} (\sin \pi\theta)^2 e^{(k-1) \cos(2\pi\theta)} d\theta. 
\]
Since $|\sin x| \leq |x|$ for all $|x| < 1$ and $\cos(2\pi\theta) \leq 1 - 8\theta^2$ whenever 
$-1/2 \leq \theta \leq 1/2$, we obtain bounds of the next forms by again 
setting $r := \frac{k-1}{\log\log x}$ for any $1 \leq k \leq \log\log x$. 
\begin{align*}
|I_{2,k}(r, x)| & \ll r^{3-k} e^{k-1} \times \int_0^{\infty} \theta^2 e^{-8(k-1) \theta^2} d\theta \\ 
     & \ll \frac{r^{3-k} e^{k-1}}{(k-1)^{3/2}} \ll 
     \frac{(\log\log x)^{k-3} e^{k-1}}{(k-1)^{3/2} (k-1)^{k-3}} 
     \ll 
     \frac{k (\log\log x)^{k-3}}{(k-1)!}. 
\end{align*}
Finally, we see that whenever $1 \leq k \leq 2\log\log x$, we have 
\[
\widehat{G}\left(\frac{1-k}{\log\log x}\right) = \frac{1}{\Gamma\left(2-\frac{k-1}{\log\log x}\right)} \cdot 
     \frac{\zeta(2)^{(k-1)/\log\log x}}{\left(1-\frac{(k-1)}{\log\log x}\right)^{-1}} \gg 1. 
\]
In fact, we can show that the the function on the left-hand-side of the last equation is 
asymptotic to $e^{o(1)}$ as $x \rightarrow \infty$. 
This implies the result of our theorem. 
\end{proof} 

\begin{cor} 
\label{cor_SummatoryFuncsOfUnsignedSeqs_v2} 
We have for large $x > e$ uniformly for $1 \leq k \leq 2\log\log x$ that  
\begin{align*} 
\widehat{C}_k(x) := 
     \sum_{\substack{n \leq x \\ \Omega(n) = k}} C_{\Omega(n)}(n) & \sim 
     \frac{4A_0\sqrt{2\pi} \cdot x}{(2k-1)} \cdot \frac{(\log\log x)^{k-1/2}}{(k-1)!}, 
\end{align*} 
for some absolute constant $A_0 \in (0, +\infty)$. 
\end{cor} 
\begin{proof} 
Suppose that $h(t)$ and $\sum_{n \leq t} \lambda_{\ast}(n)$ are any sufficiently 
piecewise smooth and differentiable functions on $\mathbb{R}^{+}$. 
We have integral formulas that result by 
applying Abel summation and integration by parts 
in the form of the next equations. 
\begin{subequations}
\begin{align} 
\label{eqn_AbelSummationIBPReverseFormula_stmt_v1} 
     \sum_{n \leq x} \lambda_{\ast}(n) h(n) & = \left(\sum_{n \leq x} \lambda_{\ast}(n)\right) h(x) - 
     \int_{1}^{x} \left(\sum_{n \leq t} \lambda_{\ast}(n)\right) h^{\prime}(t) dt \\ 
\label{eqn_AbelSummationIBPReverseFormula_stmt_v2}
     & \sim 
     \int_1^{x} \frac{d}{dt}\left[\sum_{n \leq t} \lambda_{\ast}(n)\right] h(t) dt
\end{align} 
\end{subequations}
Let the signed left-hand-side summatory function of our function $\lambda_{\ast}(n)$ 
in \eqref{eqn_AbelSummationIBPReverseFormula_stmt_v1} when $h(n) := C_{\Omega(n)}(n)$ be defined precisely 
for large $x > e$ and any integers $1 \leq k \leq \log\log x$ by 
\begin{align*} 
\widehat{C}_{k,\ast}(x) & := \sum_{\substack{n \leq x \\ \Omega(n)=k}} 
     (-1)^{\omega(n)} C_{\Omega(n)}(n) \\ 
     & \phantom{:} \sim 
     \frac{x}{\log x} \cdot \frac{(\log\log x)^{k-1}}{(k-1)!} \left[ 
     1 + O\left(\frac{1}{\log\log x}\right)\right].
\end{align*} 
The second equation above follows from the proof of 
Theorem \ref{theorem_CnkSpCasesScaledSummatoryFuncs} where 
we note that $\widehat{G}((1-k)/\log\log x) \sim e^{o(1)}$ as 
$x \rightarrow \infty$. 

Set $L_{\ast}(x) := \left\lvert \sum_{n \leq 2\log\log x} (-1)^k \pi_k(x) \right\rvert$ for $x \geq 1$. 
We can then transform our previous results for the partial sums over the signed sequences 
$(-1)^{\omega(n)} C_{\Omega(n)}(n)$ such that $\Omega(n) = k$ to approximate 
the same sum over only the unsigned summands $C_{\Omega(n)}(n)$. 
In particular, since $1 \leq k \leq \log\log x$ 
\[
\widehat{C}_{k,\ast}(x) = 
     \sum_{\substack{n \leq x \\ \Omega(n)=k}} (-1)^{\omega(n)} C_{\Omega(n)}(n) = 
     \sum_{n \leq x} (-1)^{\omega(n)} \Iverson{\omega(n) \leq 2\log\log x} \times 
     C_{\Omega(n)}(n) \Iverson{\Omega(n) = k}. 
\]
The next argument is based on first approximating $L_{\ast}(t)$ for large $t$ 
using the following uniform asymptotics for $\pi_k(x)$ that hold when 
$1 \leq k \leq \log\log x$\footnote{
     We can in fact show that for any $1 \leq k \leq x$, 
     the function $\widetilde{\mathcal{G}}(z)$ defined in 
     Remark \ref{remark_MV_Pikx_FuncResultsAnnotated_v1} satisfies 
     \[ 
     \widetilde{\mathcal{G}}\left(\frac{k-1}{\log\log x}\right) = e^{o(1)} \xrightarrow{x \rightarrow \infty} 1. 
     \]
}:
\[
\pi_k(x) = \frac{x}{\log x} \cdot \frac{(\log\log x)^{k-1}}{(k-1)!} \left[1 + 
     O\left(\frac{k}{(\log\log x)^2}\right)\right], 
     \mathrm{\ as\ } x \rightarrow \infty. 
\]
We have by Lemma \ref{cor_AsymptoticsForSignedSumsOfomegan_v1} that as $t \rightarrow \infty$ 
(\cf equation \eqref{eqn_ConvenientIncGammaFuncTypePartialSumAsymptotics_va3} in the next section) 
\begin{align} 
\label{eqn_ProofTag_LAsttSummatoryFuncAsymptotics_v1}
L_{\ast}(t) & := \left\lvert \sum_{\substack{n \leq t \\ \omega(n) \leq 2\log\log t}} (-1)^{\omega(n)} \right\rvert \sim 
     \left\lvert \sum_{k=1}^{2\log\log t} (-1)^{k} \pi_k(t) \right\rvert \asymp 
     \frac{t}{\sqrt{\log\log t}} + O\left(\frac{t}{(\log\log t)^{3/2}} + E_{\omega}(t)\right). 
\end{align} 
The error term in \eqref{eqn_ProofTag_LAsttSummatoryFuncAsymptotics_v1} 
corresponds to the asymptotics of the 
following sum as $t \rightarrow \infty$ over the error term for $\pi_k(t)$ above. 
The error estimate is obtained from Stirling's formula, \eqref{eqn_IncompleteGamma_PropA} and 
\eqref{eqn_IncompleteGamma_PropC} from the appendix section, respectively, with 
$\widetilde{\mathcal{G}}\left(\frac{k-1}{\log\log t}\right) \gg 1$ for all 
$1 \leq k \leq \log\log t$ as
\begin{align*} 
E_{\omega}(t) \ll \frac{t}{\log t} \times \sum_{1 \leq k \leq \log\log t} \frac{(\log\log t)^{k-2}}{(k-1)!} & = 
     \frac{t \Gamma(\log\log t, \log\log t)}{\Gamma(\log\log t + 1)} 
     \sim \frac{t}{\sqrt{2\pi} (\log\log t)^{3/2}}. 
\end{align*}
The main term for the reciprocal of the derivative of the main term approximation of this 
summatory function is then given by computation as follows for some absolute constant $A_0 > 0$: 
\[
\frac{1}{L_{\ast}^{\prime}(t)} \sim 2A_0 \sqrt{2\pi \log\log t}. 
\]
We apply the formula from \eqref{eqn_AbelSummationIBPReverseFormula_stmt_v2},  
to deduce that the unsigned summatory function variant satisfies the following 
relations as $x \rightarrow \infty$: 
\begin{align*} 
     \widehat{C}_{k,\ast}(x) & = \int_1^{x} L_{\ast}^{\prime}(t) C_{\Omega(t)}(t) \Iverson{\Omega(t) = k} 
     dt \qquad \implies \\ 
     & 
     C_{\Omega(x)}(x) \Iverson{\Omega(x) = k} 
     \sim \frac{\widehat{C}_{k,\ast}^{\prime}(x)}{L_{\ast}^{\prime}(x)} \qquad\quad \implies \\ 
     C_{\Omega(x)}(x) \Iverson{\Omega(x) = k} & \sim 
     2A_0 \sqrt{2\pi \log\log x} \times \widehat{C}_{k,\ast}^{\prime}(x) (1+o(1)) 
     =: \widehat{C}_{k,\ast\ast}(x). 
\end{align*} 
We have that 
\begin{align*}
     \widehat{C}_{k,\ast\ast}(x) & \sim 
     -2A_0 \sqrt{2\pi \log\log x}\left[ 
     \frac{(\log\log x)^{k-1}}{(\log x) (k-1)!} \left( 
     1 - \frac{1}{\log x}\right) + 
     \frac{(\log\log x)^{k-2}}{(\log x)^2 (k-2)!}\right]. 
\end{align*} 
Hence, integration by parts and Proposition \ref{prop_IncGammaLambdaTypeBounds_v1} 
(from the appendix) yield the next main term. 
\begin{align}
\label{eqn_proofTag_CkAstAstHatx_DerivativeFormula_NeedsIBP_v1} 
\sum_{\substack{n \leq x \\ \Omega(n)=k}} C_{\Omega(n)}(n) & \sim
     \left\lvert \int \widehat{C}_{k,\ast\ast}(x)dx \right\rvert \\ 
\notag
     & \sim 
     \frac{4A_0\sqrt{2\pi} \cdot x (\log\log x)^{k-1/2}}{(2k-1) (k-1)!} + 
     \frac{2A_0\sqrt{2\pi} \cdot x \Gamma\left(k - \frac{1}{2}, \log\log x\right)}{(k-1)!} - 
     \frac{2A_0\sqrt{2\pi} \cdot x \Gamma\left(k - \frac{3}{2}, \log\log x\right)}{(k-1)!} \\ 
     & \sim \frac{4A_0\sqrt{2\pi} \cdot x (\log\log x)^{k-1/2}}{(2k-1) (k-1)!} 
     \qedhere 
\end{align}
\end{proof}

\subsection{Average orders of the unsigned sequences}
\label{subSection_AvgOrdersOfTheUnsignedSequences} 

\begin{lemma} 
\label{cor_AsymptoticsForSignedSumsOfomegan_v1}
As $x \rightarrow \infty$, we have that 
\[
\left\lvert \sum_{n \leq x} (-1)^{\omega(n)} \right\rvert \ll 
     \frac{x}{\sqrt{\log\log x}}. 
\]
\end{lemma}
\begin{proof}
An adaptation of the proof of Lemma \ref{lemma_ConvenientIncGammaFuncTypePartialSumAsymptotics_v2} 
from the appendix shows that for any $a \in (1, 2]$ we have that 
\begin{equation}
\label{eqn_ConvenientIncGammaFuncTypePartialSumAsymptotics_va3} 
\frac{x}{\log x} \times \left\lvert \sum_{k=1}^{a\log\log x} \frac{(-1)^{k} (\log\log x)^{k-1}}{(k-1)!} 
     \right\rvert = \frac{\sqrt{a} \cdot x}{2\sqrt{2\pi}} \cdot \frac{(\log x)^{a-1-a\log a}}{\sqrt{\log\log x}} 
     \left(1 + O\left(\frac{1}{\log\log x}\right)\right). 
\end{equation}
Suppose that we take $a := 3/2$ so that 
$a-1-a\log a = \frac{1}{2}\left(1-3\log\left(\frac{3}{2}\right)\right) \approx -0.108198$. 
We can write the summatory function 
\begin{align*}
L_{\ast\ast}(x) := & \left\lvert \sum_{n \leq x} (-1)^{\omega(n)} \right\rvert = 
     \left\lvert \sum_{k \leq \log x} (-1)^{k} \pi_k(x) \right\rvert. 
\end{align*} 
By the uniform asymptotics for $\pi_k(x)$ as $x \rightarrow \infty$ when $1 \leq k \leq R \log\log x$ for 
$1 \leq R < 2$ guaranteed by the results from 
Remark \ref{remark_MV_Pikx_FuncResultsAnnotated_v1}, 
we have by Lemma \ref{lemma_ConvenientIncGammaFuncTypePartialSumAsymptotics_v2} 
(from the appendix) and 
\eqref{eqn_ConvenientIncGammaFuncTypePartialSumAsymptotics_va3} above that for large $x$ 
\begin{align*}
L_{\ast\ast}(x) & \ll \frac{x}{2\sqrt{2\pi \log\log x}} + 
     \frac{\sqrt{3} \cdot x}{4\sqrt{\pi} (\log x)^{0.108198} \sqrt{\log\log x}} + 
     \#\left\{n \leq x: \omega(x) \geq \frac{3}{2}\log\log x\right\} + O\left(
     \frac{x}{(\log\log x)^{3/2}}\right). 
\end{align*} 
Similarly, by applying the second set of results stated in 
Remark \ref{remark_MV_Pikx_FuncResultsAnnotated_v1}, we see that 
\[
\#\left\{n \leq x: \omega(x) \geq \frac{3}{2}\log\log x\right\} \ll 
     \frac{x}{(\log x)^{0.108198}}. 
\]
The result follows by removing constant factors from the 
main term in the second to last inequality above. 
\end{proof}

\begin{prop} 
\label{lemma_HatCAstxSum_ExactFormulaWithError_v1} 
We have that as $n \rightarrow \infty$ 
\[
\mathbb{E}\left[C_{\Omega(n)}(n)\right] = 
     \frac{2A_0 \sqrt{2\pi} (\log n)}{\sqrt{\log\log n}}. 
\] 
\end{prop} 
\begin{proof} 
We first compute the following 
summatory function by applying 
Corollary \ref{cor_SummatoryFuncsOfUnsignedSeqs_v2} and 
Lemma \ref{lemma_ConvenientIncGammaFuncTypePartialSumAsymptotics_v3} from the appendix:
\begin{align} 
\label{eqn_proof_tag_PartialSumsOver_HatCkx_v3} 
\sum_{k=1}^{2\log\log x} \sum_{\substack{n \leq x \\ \Omega(n) = k}} C_{\Omega(n)}(n) & = 
     \frac{2A_0 \sqrt{2\pi} \cdot x \log x}{\sqrt{\log\log x}} + O\left(
     \frac{x \log x}{(\log\log x)^{3/2}}\right). 
\end{align} 
We claim that 
\begin{align} 
\notag 
\frac{1}{x} \times \sum_{n \leq x} C_{\Omega(n)}(n) & = \frac{1}{x} \times 
     \sum_{k \geq 1} \sum_{\substack{n \leq x \\ \Omega(n) = k}} C_{\Omega(n)}(n) \\ 
\label{eqn_proof_tag_PartialSumsOver_HatCkx_v1} 
     & = 
     \frac{1}{x} \times \sum_{k=1}^{2\log\log x} \sum_{\substack{n \leq x \\ \Omega(n) = k}} 
     C_{\Omega(n)}(n) (1+o(1)), 
     \text{ as } x \rightarrow \infty. 
\end{align} 
To prove \eqref{eqn_proof_tag_PartialSumsOver_HatCkx_v1}, by 
\eqref{eqn_proof_tag_PartialSumsOver_HatCkx_v3} it suffices to show that 
\begin{equation} 
\label{eqn_proof_tag_PartialSumsOver_HatCkx_EquivCond_v2} 
\frac{1}{x} \times 
     \sum\limits_{k > 2\log\log x} \sum\limits_{\substack{n \leq x \\ \Omega(n) = k}} C_{\Omega(n)}(n)
     = O\left((\log x)^{0.613706} \times (\log\log x)\right), 
     \mathrm{\ as\ } x \rightarrow \infty. 
\end{equation} 
We proved in Theorem \ref{prop_HatAzx_ModSummatoryFuncExps_RelatedToCkn} 
that for all sufficiently large $x$ 
\[
\sum_{n \leq x} (-1)^{\omega(n)} C_{\Omega(n)}(n) z^{\Omega(n)} = 
     \frac{x \widehat{F}(2, z)}{\Gamma(z)} (\log x)^{z-1} + O\left( 
     x (\log x)^{\Re(z)-2}\right). 
\]
By Lemma \ref{cor_AsymptoticsForSignedSumsOfomegan_v1} 
we have that the summatory function 
\[
\left\lvert \sum_{n \leq x} (-1)^{\omega(n)} \right\rvert \ll 
     \frac{x}{\sqrt{\log\log x}}, 
\]
where $\frac{d}{dx}\left[\frac{x}{\sqrt{\log\log x}}\right] = \frac{1}{\sqrt{\log\log x}} + o(1)$. 
We can argue as in the proof of Corollary \ref{cor_SummatoryFuncsOfUnsignedSeqs_v2} 
using integration by parts with the Abel summation formula that whenever $1 < |z| < P(2)^{-1}$ 
and $x > e$ is sufficiently large we have 
\begin{align}
\notag
\sum_{n \leq x} C_{\Omega(n)}(n) z^{\Omega(n)} & \ll \frac{x \widehat{F}(2, z)}{\Gamma(z)} \times 
     \int_e^{x} \frac{\sqrt{\log\log t}}{t} \cdot 
     \frac{\partial}{\partial t}\left[ t (\log t)^{z-1}\right] dt \\ 
\notag 
     & \ll 
     \frac{x \widehat{F}(2, z)}{\Gamma(z)} \Biggl[
     \frac{(\log x)^{z-1}(z+\log x)}{z} \sqrt{\log\log x} - 
     \frac{\sqrt{\pi}}{2\sqrt{z-1}} \operatorname{erfi}\left(\sqrt{(z-1)\log\log x}\right) \\ 
\notag
     & \phantom{\ll\Biggl[\qquad\qquad\ } - 
     \frac{\sqrt{\pi}}{2 z^{3/2}} \operatorname{erfi}\left(\sqrt{z \log\log x}\right) 
     \Biggr] \\ 
\label{eqn_COmegannzPowOmeganLLRelation_v1} 
     & \ll \frac{x \widehat{F}(2, z)}{\Gamma(1+z)} (\log x)^{z} \sqrt{\log\log x}. 
\end{align}
The omitted error term in the last formula follows from the asymptotic series for 
$\operatorname{erfi}(z)$ in \eqref{eqn_Erfix_KnownAsymptoticSeries_v1}. 
Namely, as $|z| \rightarrow \infty$, the \emph{imaginary error function}, 
$\operatorname{erfi}(z)$, has the following asymptotic expansion 
\cite[\S 7.12]{NISTHB}: 
\begin{equation}
\label{eqn_Erfix_KnownAsymptoticSeries_v1}
\operatorname{erfi}(z) := \frac{2}{\sqrt{\pi} \cdot \imath} \times \int_0^{\imath z} e^{t^2} dt = 
     \frac{e^{z^2}}{\sqrt{\pi}} \left(z^{-1} + \frac{z^{-3}}{2} + 
     \frac{3z^{-5}}{4} + \frac{15 z^{-7}}{8} + O\left(z^{-9}\right)\right). 
\end{equation}
For all large enough $x > e$ and $r > 0$, we define 
\[
\widehat{B}(x, r) := \sum_{\substack{n \leq x \\ \Omega(n) \geq r\log\log x}} 
     C_{\Omega(n)}(n). 
\]
We argue as in the proof from the reference \cite[\cf Thm.\ 7.20; \S 7.4]{MV} 
applying \eqref{eqn_COmegannzPowOmeganLLRelation_v1} when $1 \leq r < P(2)^{-1}$. 
Since $\widehat{F}(2, r) = \frac{\zeta(2)^{-r}}{1+P(2)r} \ll 1$ for $r \in [1, P(2)^{-1})$, and 
similarly since we have that $\frac{1}{\Gamma(1+r)} \gg 1$ for $r$ taken within this same range, 
we obtain that 
\[
x \sqrt{\log\log x} (\log x)^{r} \gg \sum_{\substack{n \leq x \\ \Omega(n) \geq r\log\log x}} 
     C_{\Omega(n)}(n) r^{\Omega(n)} \gg 
     \sum_{\substack{n \leq x \\ \Omega(n) \geq r\log\log x}} 
     C_{\Omega(n)}(n) r^{r \log\log x}, \text{\ for\ } 1 \leq r < 2. 
\]
This implies that for $r := 2$ we have 
\begin{equation}
\label{eqn_BHatxrUpperBound_v1}
\widehat{B}(x, r) \ll x (\log x)^{r-r\log r} \sqrt{\log\log x} = 
     O\left(x (\log x)^{0.613706} \times \sqrt{\log\log x}\right)
\end{equation}
We wish to evaluate the limiting asymptotics of the sum 
\begin{align*}
S_2(x) & := \frac{1}{x \sqrt{\log\log x}} \times 
     \sum_{k \geq 2\log\log x} \sum_{\substack{n \leq x \\ \Omega(n)=k}} 
     C_{\Omega(n)}(n) \ll \widehat{B}(x, 2). 
\end{align*} 
We have proved that 
$S_2(x) \sqrt{\log\log x} = O\left((\log x)^{0.61306} (\log\log x)\right)$ 
as $x \rightarrow \infty$, as required to show that 
\eqref{eqn_proof_tag_PartialSumsOver_HatCkx_EquivCond_v2} holds. 
\end{proof} 

\begin{cor}
\label{cor_ExpectationFormulaAbsgInvn_v2} 
We have that as $n \rightarrow \infty$ 
\begin{align*} 
\mathbb{E}|g^{-1}(n)| & = \frac{12A_0}{\pi} \cdot 
     \frac{(\log n)^2}{\sqrt{\log\log n}}. 
\end{align*} 
\end{cor} 
\begin{proof} 
We use the formula from Proposition \ref{lemma_HatCAstxSum_ExactFormulaWithError_v1} 
to sum $\mathbb{E}[C_{\Omega(n)}(n)]$ as $n \rightarrow \infty$.
This result and \eqref{eqn_Erfix_KnownAsymptoticSeries_v1} 
imply that for all sufficiently large $t \rightarrow +\infty$ 
\begin{align*} 
\frac{1}{2 \sqrt{2\pi}} \times \int \frac{\mathbb{E}[C_{\Omega(t)}(t)]}{t} dt & = 
     \pi \operatorname{erfi}\left(\sqrt{2\log\log t}\right) = 
     \sqrt{\frac{\pi}{2}} \cdot \frac{(\log t)^2}{\sqrt{\log\log t}} (1+o(1)). 
\end{align*} 
Recall that for large $x$ 
\[
Q(x) := \sum_{n \leq x} \mu^2(n) = \frac{6x}{\pi^2} + O(\sqrt{x}). 
\]
Therefore summing over the formula from 
\eqref{eqn_AbsValueOf_gInvn_FornSquareFree_v1} we find that  
\begin{align} 
\notag 
\mathbb{E}|g^{-1}(n)| & = \frac{1}{n} \times \sum_{d \leq n} 
     C_{\Omega(d)}(d) Q\left(\Floor{n}{d}\right) \\ 
\notag 
     & \sim \sum_{d \leq n} C_{\Omega(d)}(d) \left[\frac{6}{d \cdot \pi^2} + O\left(\frac{1}{\sqrt{dn}}\right) 
     \right] \\ 
\notag 
     & = \frac{6}{\pi^2} \left[\mathbb{E}[C_{\Omega(n)}(n)] + \sum_{d<n} 
     \frac{\mathbb{E}[C_{\Omega(d)}(d)]}{d}\right] + O(1). 
     \qedhere 
\end{align} 
\end{proof} 

\subsection{Erd\H{o}s-Kac theorem analogs for the distributions of the unsigned sequences} 
\label{subSection_ErdosKacTheorem_Analogs} 

\begin{theorem}[Central limit theorem for the distribution of $C_{\Omega(n)}(n)$] 
\label{theorem_CLT_VI} 
For large $x > e$, let the mean and variance parameter analogs be defined by 
\[
     \mu_x(C) := \log\log x - \log\left(4A_0\sqrt{2\pi}\right), 
     \qquad \mathrm{\ and\ } \qquad 
     \sigma_x(C) := \sqrt{\log\log x}. 
\]
Let $Y > 0$ be fixed. 
We have uniformly for all $-Y \leq z \leq Y$ that 
\[
\frac{1}{x} \times \#\left\{2 \leq n \leq x: 
     \frac{C_{\Omega(n)}(n) - \mu_x(C)}{\sigma_x(C)} \leq z\right\} = 
     \Phi\left(z\right) + O\left(\frac{1}{\sqrt{\log\log x}}\right), 
     \mathrm{\ as\ } x \rightarrow \infty. 
\] 
\end{theorem} 
\begin{proof} 
Fix any $Y > 0$ and set $z \in [-Y, Y]$. 
For large $x$ and $2 \leq n \leq x$, define the following auxiliary variables: 
\[
\alpha_n := \frac{C_{\Omega(n)}(n) - \mu_n(C)}{\sigma_n(C)}, \quad\mathrm{and}\quad 
     \beta_{n,x} := \frac{C_{\Omega(n)}(n) - \mu_x(C)}{\sigma_x(C)}. 
\] 
Let the corresponding densities be defined by the functions 
\[
\Phi_1(x, z) := \frac{1}{x} \times \#\{n \leq x: \alpha_n \leq z\}, 
\]
and 
\[
\Phi_2(x, z) := \frac{1}{x} \times \#\{n \leq x: \beta_{n,x} \leq z\}. 
\] 
We assert that it suffices to show that $\Phi_2(x, z) = \Phi(x) + O\left(\frac{1}{\sqrt{\log\log x}}\right)$ as 
$x \rightarrow \infty$ in place of considering the distribution of 
$\Phi_1(x, z)$ to obtain the conclusion. 
The normalizing terms $\mu_n(C)$ and $\sigma_n(C)$ hardly change over 
$\sqrt{x} \leq n \leq x$. Namely, for $n \in [\sqrt{x}, x]$ 
as $x \rightarrow \infty$ we see that 
\[
|\mu_n(C) - \mu_x(C)| \leq 
     \frac{\log 2}{\log x} + o(1), 
\]
and 
\[
|\sigma_n(C) - \sigma_x(C)| 
     \leq \frac{\log 2}{(\log x) \sqrt{\log\log x}} + o(1). 
\]
In particular, for 
$\sqrt{x} \leq n \leq x$ and $C_{\Omega(n)}(n) \leq 2 \mu_x(C)$ we can show 
using \eqref{eqn_BHatxrUpperBound_v1} 
that the following is true: 
\[
|\alpha_n - \beta_{n,x}| \xrightarrow{x \rightarrow \infty} 0. 
\]
Thus we can replace $\alpha_n$ by $\beta_{n,x}$ and estimate the limiting 
densities corresponding to these alternate terms as $x \rightarrow \infty$. 
The rest of our argument closely parallels the method from the proof of the related theorem in 
\cite[Thm.\ 7.21; \S 7.4]{MV} from Montgomery and Vaughan. 
After a change of variable in our proof, we obtain the limiting 
CLT statement in analog to their analytic proof of the Erd\H{o}s-Kac theorem 
for the distribution of $\Omega(n)$. 

We use the formula proved in Corollary \ref{cor_SummatoryFuncsOfUnsignedSeqs_v2} 
to estimate the densities claimed within the ranges bounded by 
$z$ as $x \rightarrow \infty$. 
Let $k \geq 1$ be a natural number such that $k := t_x + \mu_x(C)$ where 
$t_x := \frac{t \sqrt{\log\log x}}{(\log x)}$. 
For fixed large $x$, we define the small parameter $\delta_{t,x} := \frac{t_x}{\mu_x(C)}$ 
for some target PDF parameter $t \in \mathbb{R}$. 
When $|t| \leq \frac{1}{2} \mu_x(C)$, we have by Stirling's formula that 
\begin{align*} 
\frac{1}{x} \times \sum_{\substack{n \leq x \\ \Omega(n)=k}} C_{\Omega(n)}(n) & 
     \sim 
     \frac{4A_0\sqrt{2\pi} (\log\log x)^{k-\frac{1}{2}}}{(2k-1) (k-1)!} \\ 
     & \sim 
     \frac{(\log x)}{\sqrt{2\pi \log\log x} \cdot \sigma_x(C)  
     \left(1-\frac{1}{2k}\right)} \times e^{t_x} (1+o(1))^{k-\frac{1}{2}} \times 
     \left(1+\delta_{t,x}\right)^{-\mu_x(C)(1+\delta_{t,x})-\frac{1}{2}}. 
\end{align*} 
Notice that 
\begin{align*}
\frac{1}{1-\frac{1}{2k}} & \sim \sum_{m \geq 0} \frac{1}{(2 \mu_x(C))^m (1+\delta_{t,x})^m} 
     \sim 1 + \frac{1}{2 \mu_x(C)} \left(1+\delta_{t,x}+O(\delta_{t,x}^2)\right) \\ 
     & = 1 + o_{\delta_{t,x}}(1), 
     \text{\ for\ } \delta_{t,x} \approx 0 \text{\ as\ } x \rightarrow \infty. 
\end{align*}
We have the uniform estimate that 
$\log(1+\delta_{t,x}) = \delta_{t,x} - \frac{\delta_{t,x}^2}{2} + O(|\delta_{t,x}|^3)$ whenever 
$|\delta_{t,x}| \leq \frac{1}{2}$. Then we can expand the factor involving $\delta_{t,x}$ 
from the previous equation as follows: 
\begin{align*} 
(1+\delta_{t,x})^{-\mu_x(C) (1+\delta_{t,x}) - \frac{1}{2}} & = 
     \exp\left(\left(\frac{1}{2}+\mu_x(C) (1+\delta_{t,x})\right) \times 
     \left(-\delta_{t,x} + \frac{\delta_{t,x}^2}{2} + O(|\delta_{t,x}|^3)\right)\right) \\ 
     & = \exp\left(-t_x - \frac{t_x+t_x^2}{2\mu_x(C)} + \frac{t_x^2}{4\mu_x(C)^2} + 
     O\left(\frac{|t_x|^3}{\mu_x(C)^2}\right)\right). 
\end{align*} 
For both $|t| \leq \mu_x(C)^{1/2}$ and 
$\mu_x(C)^{1/2} < |t| \leq \mu_x(C)^{2/3}$, 
we can see that 
\[
\frac{t}{\mu_x(C)} \ll \frac{1}{\sqrt{\mu_x(C)}} + \frac{|t|^3}{\mu_x(C)^2}. 
\]
Similarly, for both  $|t| \leq 1$ and $|t| > 1$, we have that 
\[
\frac{t^2}{\mu_x(C)^2} \ll \frac{1}{\sqrt{\mu_x(C)}} + 
     \frac{|t|^3}{\mu_x(C)^2}. 
\] 
Let the corresponding error terms in $x$ and $t$ be denoted by 
\[
\widetilde{E}(x, t) := O\left(\frac{1}{\sigma_x(C)} + \frac{|t|^3}{\mu_x(C)^2}\right). 
\]
Combining these estimates with the previous computations, we deduce 
uniformly for $|t| \leq \mu_x(C)^{2/3}$ that 
\begin{align*} 
\frac{4A_0\sqrt{2\pi} (\log\log x)^{k-\frac{1}{2}}}{(2k-1)(k-1)!} & \sim 
     \frac{\log x}{\sqrt{2\pi \log\log x} \cdot \sigma_x(C)} 
     \times \exp\left(-\frac{t_x^2}{2\sigma_x(C)^2}\right) \times 
     \left[1 + \widetilde{E}(x, t_x)\right]. 
\end{align*} 
It follows that uniformly for $1 \leq k \leq \log\log x$ 
\begin{align*}
f(k, x) & := 
     \frac{1}{x} \times \sum_{\substack{n \leq x \\ \Omega(n)=k}} C_{\Omega(n)}(n) \\ 
     & \sim 
     \frac{(\log x)}{\sqrt{2\pi \log\log x} \cdot \sigma_x(C)} 
     \times \exp\left(-\frac{(k-\mu_x(C))^2 \sqrt{\log\log x}}{2 (\log x) \sigma_x(C)^2}\right) \times 
     \left[1 + \widetilde{E}\left(x, \frac{|k - \mu_x(C)| \sqrt{\log\log x}}{(\log x)}\right)\right]. 
\end{align*}
Since our target probability density function approximating the PDF (in $t$) of the 
normal distribution is given here by 
$$\frac{f(k, x) \sqrt{\log\log x}}{(\log x)} 
  \rightarrow \frac{1}{\sqrt{2\pi} \cdot \sigma_x(C)} \times \exp\left(-\frac{t^2}{2
  \sigma_x(C)^2}\right),$$ 
we perform the change of variable $t \mapsto \frac{t \sqrt{\log\log x}}{(\log x)}$ to obtain the 
form of our theorem stated above. 

By the same argument utilized in the proof of 
Proposition \ref{lemma_HatCAstxSum_ExactFormulaWithError_v1}, we see that 
the contributions of these summatory functions for 
$k \leq \mu_x(C) - \mu_x(C)^{2/3}$ is negligible. 
We also require that $k \leq 2\log\log x$ for all large $x$ as we required by 
Theorem \ref{theorem_CnkSpCasesScaledSummatoryFuncs}. We then sum over a 
corresponding range of 
\[
\mu_x(C) -\mu_x(C)^{2/3} \leq k \leq \mu_x(C) + z \sigma_x(C), 
\] 
to approximate the stated normalized densities. 
As $x \rightarrow \infty$ the 
three terms that result (one main term and two error terms, respectively) 
can be considered to each correspond to a Riemann sum for an associated integral whose 
limiting formula corresponds to a main term given by the standard normal CDF, $\Phi(z)$. 
\end{proof} 

\begin{cor} 
\label{cor_CLT_VII} 
Let $Y > 0$. Suppose that $\mu_x(C)$ and $\sigma_x(C)$ are defined as in 
Theorem \ref{theorem_CLT_VI} for large $x > e$. 
For $Y > 0$ and we have uniformly for all $-Y \leq y \leq Y$ 
that as $x \rightarrow \infty$ 
\begin{align*} 
\frac{1}{x} \cdot \#\left\{2 \leq n \leq x:|g^{-1}(n)| - 
     \frac{6}{\pi^2} \mathbb{E}|g^{-1}(n)| \leq y\right\} & = 
     \Phi\left\{\frac{6 \sigma_x(C)}{\pi^2}\left(\frac{\pi^2 y}{6} + \sigma_x(C)\right) - 
     \frac{6}{\pi^2} \log\left(4A_0\sqrt{2\pi}\right)\right\} + o(1). 
\end{align*} 
\end{cor} 
\begin{proof} 
We claim that 
\begin{align*} 
|g^{-1}(n)| - \frac{6}{\pi^2} \mathbb{E}|g^{-1}(n)| & \sim \frac{6}{\pi^2} C_{\Omega(n)}(n), 
     \text{\ as\ } n \rightarrow \infty. 
\end{align*} 
As in the proof of Corollary \ref{cor_ExpectationFormulaAbsgInvn_v2}, 
we obtain that 
\begin{align*} 
\frac{1}{x} \times \sum_{n \leq x} |g^{-1}(n)| & = 
     \frac{6}{\pi^2} \left[\mathbb{E}[C_{\Omega(x)}(x)] + \sum_{d<x} 
     \frac{\mathbb{E}[C_{\Omega(d)}(d)]}{d}\right] + O(1). 
\end{align*} 
Let the \emph{backwards difference operator} with respect to $x$ 
be defined for $x \geq 2$ and any arithmetic function $f$ as 
$\Delta_x(f(x)) := f(x) - f(x-1)$. 
We see that for large $n$ 
(\cf last lines in the proof of Corollary \ref{cor_ExpectationFormulaAbsgInvn_v2}) 
\begin{align*} 
|g^{-1}(n)| & = \Delta_n(n \cdot \mathbb{E}|g^{-1}(n)|) 
     \sim \Delta_n\left(\sum_{d \leq n} \frac{6}{\pi^2} \cdot C_{\Omega(d)}(d) \cdot \frac{n}{d}\right) \\ 
     & = \frac{6}{\pi^2}\left[C_{\Omega(n)}(n) + \sum_{d < n} C_{\Omega(d)}(d) \frac{n}{d} - 
     \sum_{d<n} C_{\Omega(d)}(d) \frac{(n-1)}{d}\right] \\ 
     & \sim \frac{6}{\pi^2} C_{\Omega(n)}(n) + \frac{6}{\pi^2} \mathbb{E}|g^{-1}(n-1)|, 
     \mathrm{\ as\ } n \rightarrow \infty. 
\end{align*} 
Since $\mathbb{E}|g^{-1}(n-1)| \sim \mathbb{E}|g^{-1}(n)|$ for all sufficiently large $n$, 
the result finally follows by a normalization of Theorem \ref{theorem_CLT_VI}. 
\end{proof} 

\subsection{Probabilistic interpretations} 
\label{subSection_ProbInterprets_Of_ErdosKacAnalogs} 

\begin{lemma} 
\label{lemma_ProbsOfAbsgInvnDist_v2} 
For all $x$ sufficiently large, if we pick any integer $n \in [2, x]$ uniformly at random, then 
each of the following statements holds: 
\begin{align*} 
\tag{A}
     \mathbb{P}\left(|g^{-1}(n)| - \frac{6}{\pi^2} \mathbb{E}|g^{-1}(n)| \leq 
     \frac{6}{\pi^2} (\log\log x) 
     \right) & = \frac{1}{2} + o(1) \\ 
\tag{B} 
\mathbb{P}\left(|g^{-1}(n)| - \frac{6}{\pi^2} \mathbb{E}|g^{-1}(n)| \leq 
     \frac{6}{\pi^2}\left(\alpha + \log\log x\right)
     \right) & = 
     \Phi\left(\alpha\right) + o(1), \alpha \in \mathbb{R}. 
\end{align*} 
\end{lemma} 
\begin{proof} 
Each of these results is a consequence of Corollary \ref{cor_CLT_VII}. 
The result in (A) follows since $\Phi(0) = \frac{1}{2}$ by taking 
$$z = \frac{\left(\alpha + \frac{6}{\pi^2} \log\left(4A_0\sqrt{2\pi}\right)\right)}{\sigma_x(C)} - 
  \frac{6}{\pi^2} \sigma_x(C),$$ 
in Corollary \ref{cor_CLT_VII} 
for $\alpha = 0$ for $\sigma_x(C) := \log\log x$. Note that as $\alpha \rightarrow +\infty$, 
we get that the right-hand-side of (B) tends to one for large $x \rightarrow \infty$. 
\end{proof} 

\label{remark_ProbsOfAbsgInvnDist_v3} 
It follows from Lemma \ref{lemma_ProbsOfAbsgInvnDist_v2} and 
Corollary \ref{cor_ExpectationFormulaAbsgInvn_v2} that 
\[
\lim_{x \rightarrow \infty}\ \frac{1}{x} \times \#\left\{n \leq x: 
     |g^{-1}(n)| \leq \frac{6}{\pi^2} \mathbb{E}|g^{-1}(n)| (1+o(1))\right\} = 1. 
\]
That is, for almost every sufficiently large integer $n$ we recover that 
$$|g^{-1}(n)| \leq \frac{6}{\pi^2} \mathbb{E}|g^{-1}(n)| (1+o(1)).$$

\newpage 
\section{New formulas and limiting relations characterizing $M(x)$} 
\label{Section_KeyApplications} 

\subsection{Establishing initial asymptotic bounds on the summatory function $G^{-1}(x)$} 
\label{Section_ProofOfValidityOfAverageOrderLowerBounds} 

Let $L(x) := \sum_{n \leq x} \lambda(n)$ for $x \geq 1$. 
A recent upper bound on $L(x)$ (assuming the RH) is proved by 
Humphries based on Soundararajan's result bounding $M(x)$. It is 
stated in the following form \cite{HUMPHRIES-JNT-2013}: 
\begin{equation} 
\label{eqn_LxBigOhAsymptotics_Humphries_v1}
L(x) = O\left(\sqrt{x} \times \exp\left( (\log x)^{\frac{1}{2}} 
     (\log\log x)^{\frac{5}{2} + \epsilon}\right)\right), 
     \mathrm{\ for\ any\ } \epsilon > 0, 
     \mathrm{\ as\ } x \rightarrow \infty.
\end{equation}

\begin{theorem} 
\label{cor_ExprForGInvxByLx_v1} 
We have that for almost every sufficiently large $x$, there exists $1 \leq t_0 \leq x$ such that 
$$G^{-1}(x) = O\left(L(t_0) \times \mathbb{E}|g^{-1}(x)|\right).$$ 
If the RH is true, then 
for any $\epsilon > 0$ and all large integers $x > e$ we have that 
\[
G^{-1}(x) = O\left(\frac{\sqrt{x} (\log x)^2}{\sqrt{\log\log x}} \times \exp\left( 
     \sqrt{\log x} (\log\log x)^{\frac{5}{2} + \epsilon}\right)\right). 
\]
\end{theorem} 
\begin{proof} 
We write the next formulas for $G^{-1}(x)$ at almost every large $x > e$ 
by Abel summation and applying the mean value theorem: 
\begin{align} 
\notag 
G^{-1}(x) & = \sum_{n \leq x} \lambda(n) |g^{-1}(n)| \\ 
\notag
     & = L(x) |g^{-1}(x)| - \int_1^x L(x) \frac{d}{dx}\left\lvert g^{-1}(x) \right\rvert dx \\ 
\label{proof_tag_GInvx_LxAbelSummationIntFormula_v1} 
     & = O\left(L(t_0) \times \mathbb{E}|g^{-1}(x)|\right), \mathrm{\ for\ some\ } t_0 \in [1, x]. 
\end{align} 
The proof of this result appeals to the 
last few results we used to establish the 
probabilistic interpretations of the distribution of $|g^{-1}(n)|$ as 
$n \rightarrow \infty$ in 
Section \ref{subSection_ProbInterprets_Of_ErdosKacAnalogs}. 

We need to bound the sums of the maximal extreme values of $|g^{-1}(n)|$ over $n \leq x$
as $x \rightarrow \infty$ to prove the second claim. 
We know by a result of Robin that \cite{ROBIN-PRIMEOMEGAFUNC-BOUNDS}
\[
\omega(n) \ll \frac{\log n}{\log\log n}, \mathrm{\ as\ } n \rightarrow \infty. 
\]
Recall that the values of $|g^{-1}(n)|$ are locally maximized when $n$ is squarefree with 
\begin{align*}
|g^{-1}(n)| & \leq \sum_{j=0}^{\omega(n)} \binom{\omega(n)}{j} \times j! 
     \ll \Gamma(\omega(n)+1) 
     \ll \left(\frac{\log n}{\log\log n}\right)^{\frac{\log n}{\log\log n} + \frac{1}{2}}. 
\end{align*}
Since we deduced that the set of $n \leq x$ on which $|g^{-1}(n)|$ is substantially larger 
than its average order is asymptotically thin at the end of the last section, 
we find the largest possible bounds asserting that 
\begin{align*}
     \left\lvert \int_{x-o(1)}^{x} L^{\prime}(t) |g^{-1}(t)| dt\right\vert & \ll 
\int_{x-o(1)}^{x} \left(\frac{\log t}{\log\log t}\right)^{\frac{\log t}{\log\log t} + 
     \frac{1}{2}} dt 
     = o\left(\left(\frac{\log x}{\log\log x}\right)^{\frac{\log x}{\log\log x} + \frac{1}{2}}\right) \\ 
     & \ll o\left(\frac{x}{(\log x)^{m-1/2} (\log\log x)^r}\right), \text{\ for\ any\ } 
     m, r = o\left(\frac{\log\log\log x}{\log\log x}\right), 
     \mathrm{\ as\ } x \rightarrow \infty. 
\end{align*}
Indeed, we can see that the limit 
\begin{align*}
     \lim_{x \rightarrow \infty}\ & \frac{1}{x} \left(\frac{\log x}{\log\log x} 
     \right)^{\frac{\log x}{\log\log x} + \frac{1}{2}} (\log x)^{m-1/2} (\log\log x)^r \ll 
\lim_{x \rightarrow \infty} x^{-\frac{\log\log\log x}{\log\log x}} 
     (\log x)^{m+r} \\ 
     & = \lim_{x \rightarrow \infty} \exp\left((m+r) \log x - (\log x) 
     \frac{\log\log\log x}{\log\log x}\right) 
     = \lim_{t \rightarrow \infty} e^{-t} = 0. 
\end{align*} 
For large $x$, let $\mathcal{R}_x := \{t \leq x: |g^{-1}(t)| \ggg \mathbb{E}|g^{-1}(t)|\}$ such that 
$|\mathcal{R}_x| = o(1)$ (as we argued before). 
The formula from \eqref{eqn_AbelSummationIBPReverseFormula_stmt_v1} then implies 
that for large $x$ and any $m, r = o\left(\frac{\log\log\log x}{\log\log x}\right)$ 
\begin{align*}
G^{-1}(x) & = O\left(\int_1^x L^{\prime}(x) |g^{-1}(x)| dx\right) 
     = O\left(\mathbb{E}|g^{-1}(x)| \times \int_1^x L^{\prime}(x) dx + 
     \int_{x-|\mathcal{R}_x|}^x |L^{\prime}(t)| \times |g^{-1}(t)| dt\right) \\ 
     & = O\left(\mathbb{E}|g^{-1}(x)| \times |L(x)| + 
     o\left(\frac{x}{(\log x)^{m-1/2} (\log\log x)^r}\right)\right). 
\end{align*} 
If the RH is true, by applying Humphries' result in 
\eqref{eqn_LxBigOhAsymptotics_Humphries_v1} and 
Corollary \ref{cor_ExpectationFormulaAbsgInvn_v2}, 
then for any $\epsilon > 0$, 
$m,r = o\left(\frac{\log\log\log x}{\log\log x}\right)$ and 
large integers $x \geq 1$ we have that 
\begin{align*}
G^{-1}(x) & = O\left( 
     \frac{\sqrt{x} (\log x)^2}{\sqrt{\log\log x}} \times 
     \exp\left(\sqrt{\log x} \cdot (\log\log x)^{\frac{5}{2} + \epsilon}\right) + 
     o\left(\frac{x}{(\log x)^{m-1/2} (\log\log x)^r}\right) 
     \right). 
\end{align*}
To obtain the conclusion in the second result, we take limits as $x \rightarrow \infty$ 
to see that the dominant term is given by the leftmost term in the last bound. 
\end{proof} 

\subsection{Bounding $M(x)$ by asymptotics for $G^{-1}(x)$} 

\begin{prop} 
\label{prop_Mx_SBP_IntegralFormula} 
For all sufficiently large $x$, we have that the Mertens function satisfies 
\begin{align} 
\label{eqn_pf_tag_v2-restated_v2} 
M(x) & = G^{-1}(x) + 
     \sum_{k=1}^{\frac{x}{2}} G^{-1}(k) \left[ 
     \pi\left(\Floor{x}{k}\right) - \pi\left(\Floor{x}{k+1}\right) 
     \right]. 
\end{align} 
\end{prop} 
\begin{proof} 
We know by applying Corollary \ref{cor_Mx_gInvnPixk_formula} that 
\begin{align} 
\notag
M(x) & = \sum_{k=1}^{x} g^{-1}(k) \left[\pi\left(\Floor{x}{k}\right)+1\right] \\ 
\notag 
     & = G^{-1}(x) + \sum_{k=1}^{\frac{x}{2}} g^{-1}(k) \pi\left(\Floor{x}{k}\right) \\ 
\notag 
     & = G^{-1}(x) + G^{-1}\left(\Floor{x}{2}\right) + 
     \sum_{k=1}^{\frac{x}{2}-1} G^{-1}(k) \left[ 
     \pi\left(\Floor{x}{k}\right) - \pi\left(\Floor{x}{k+1}\right) 
     \right].
\end{align} 
The upper bound on the sum is truncated to $k \in \left[1, \frac{x}{2}\right]$ in the second equation 
above due to the fact that $\pi(1) = 0$. 
The third formula above follows directly by (ordinary) summation by parts. 
\end{proof} 

\begin{lemma}
\label{lemma_PrimePix_ErrorBoundDiffs_SimplifyingConditions_v1} 
For sufficiently large $x$, integers $k \in \left[1, \sqrt{x}\right]$ and 
$m \geq 0$, we have that 
\begin{equation} 
\tag{A} 
\frac{x}{k \cdot \log^m\left(\frac{x}{k}\right)} - 
     \frac{x}{(k+1) \cdot \log^m\left(\frac{x}{k+1}\right)}
     \asymp \frac{x}{(\log x)^m \cdot k(k+1)}, 
\end{equation} 
and 
\begin{equation} 
\tag{B} 
\sum_{k=\sqrt{x}}^{\frac{x}{2}} \frac{x}{k(k+1)} = 
     \sum_{k=\sqrt{x}}^{\frac{x}{2}} \frac{x}{k^2} + O(1). 
\end{equation} 
\end{lemma} 
\begin{proof} 
To prove (A), we first notice that for any $k \in \left[1, \sqrt{x}\right]$ 
\begin{align*}
\frac{\log\left(\frac{x}{k}\right)}{\log\left(\frac{x}{k+1}\right)} & = 
     \frac{1 - \frac{\log k}{\log x}}{1 - \frac{\log k}{\log x} + O\left(\frac{1}{k\log x}\right)} 
     = 1 + O\left(\frac{1}{k\log x\left(1 - \frac{\log k}{\log x}\right)}\right) 
     = 1 + o(1), \text{\ as\ } x \rightarrow \infty. 
\end{align*}
Then for any $m \geq 0$ and $k$ within these bounds, we see that 
\begin{align*}
\frac{x}{k \cdot \log^m\left(\frac{x}{k}\right)} - 
     \frac{x}{(k+1) \cdot \log^m\left(\frac{x}{k+1}\right)} & = \frac{x}{\log^m\left(\frac{x}{k+1}\right)} \left[ 
     \frac{(1+o(1))^m}{k} - \frac{1}{k+1}\right] \\ 
     & \asymp \frac{x}{(\log x)^m} \left[\frac{1}{k(k+1)} + o\left(\frac{1}{k}\right)\right], 
\end{align*}
where for any $k \in \left[1, \sqrt{x}\right]$ we have that 
$o\left(k^{-1}\right) = o(1)$ for all large $x \rightarrow \infty$. 

To prove (B), notice that 
\[
\frac{x}{k(k+1)} - \frac{x}{k^2} = -\frac{x}{k^2(k+1)}. 
\]
Then we see that 
\[
\left\lvert \int_{\sqrt{x}}^{\frac{x}{2}} \frac{x}{t^2(t+1)} dt \right\rvert \leq 
     \left\lvert \int_{\sqrt{x}}^{\frac{x}{2}} \frac{x}{t^3} dt \right\rvert = O(1). 
     \qedhere 
\]
\end{proof} 

\begin{cor} 
\label{cor_IntFormulaGInvx_for_Mx_v1} 
We have that as $x \rightarrow \infty$ 
\begin{align*}
     M(x) & = O\left(G^{-1}(x) + G^{-1}\left(\frac{x}{2}\right) + 
     \frac{x}{\log x} \times \sum_{k \leq \sqrt{x}} \frac{G^{-1}(k)}{k^2}
     + (\log x)^2 \sqrt{\log\log x}\right). 
\end{align*} 
\end{cor}
\begin{proof}
We need to first bound the prime counting function differences in the formula given by 
Proposition \ref{prop_Mx_SBP_IntegralFormula}. 
We will require the following known bounds on the prime counting 
function due to Rosser and Schoenfeld for all sufficiently large $x > 59$ 
\cite[Thm.\ 1]{ROSSER-SCHOENFELD-1962}: 
\begin{equation} 
\label{eqn_RosserSchoenfeld_PrimePixBounds_v2} 
\frac{x}{\log x}\left(1 + \frac{1}{2\log x}\right) \leq \pi(x) \leq 
     \frac{x}{\log x}\left(1 + \frac{3}{2 \log x}\right). 
\end{equation} 
The bounds in \eqref{eqn_RosserSchoenfeld_PrimePixBounds_v2} together with 
Lemma \ref{lemma_PrimePix_ErrorBoundDiffs_SimplifyingConditions_v1} implies that 
for $\sqrt{x} \leq k \leq \frac{x}{2}$ 
\begin{equation}
\label{eqn_PrimePiDifferenceOfTermsBound_v1}
\pi\left(\Floor{x}{k}\right) - \pi\left(\Floor{x}{k+1}\right) = 
     O\left(\frac{x}{k^2 \log\left(\frac{x}{k}\right)}\right). 
\end{equation}
We will rewrite the intermediate formula from the proof of 
Proposition \ref{prop_Mx_SBP_IntegralFormula} 
as a sum of two components with summands taken over disjoint intervals. 
For large $x > e$, let 
\begin{align*}
S_1(x) & := \sum_{1 \leq k \leq \sqrt{x}} g^{-1}(k) \pi\left(\frac{x}{k}\right) \\ 
S_2(x) & := \sum_{\sqrt{x} < k \leq \frac{x}{2}} g^{-1}(k) \pi\left(\frac{x}{k}\right).
\end{align*}
We then assert by the asymptotic formulas for the 
prime counting function that 
\[
S_1(x) = O\left(\frac{x}{\log x} \times \sum_{k \leq \sqrt{x}} 
     \frac{G^{-1}(k)}{k^2}\right). 
\]
To bound the second sum, we perform summation by parts as in the proof of 
the proposition and apply \eqref{eqn_PrimePiDifferenceOfTermsBound_v1} above for the difference of the 
summand functions to obtain that 
\begin{align*} 
S_2(x) & = O\left(G^{-1}\left(\frac{x}{2}\right) + \int_{\sqrt{x}}^{\frac{x}{2}} 
     \frac{G^{-1}(t)}{t^2 \log\left(\frac{x}{t}\right)} dt\right) \\ 
     & = O\left(G^{-1}\left(\frac{x}{2}\right) + 
     \max_{\sqrt{x} < k < \frac{x}{2}} \frac{|G^{-1}(k)|}{k} \times 
     \int_{\sqrt{x}}^{\frac{x}{2}} \frac{dt}{t \log\left(\frac{x}{t}\right)}\right) \\ 
     & = O\left(G^{-1}\left(\frac{x}{2}\right) + (\log\log x) \times 
     \max_{\sqrt{x} < k < \frac{x}{2}} \frac{|G^{-1}(k)|}{k}\right). 
\end{align*} 
The rightmost maximum term in the previous equation 
satisfies $\frac{|G^{-1}(k)|}{k} \ll \mathbb{E}|g^{-1}(k)|$ as 
$k \rightarrow \infty$. 
The conclusion follows since the average order of $|g^{-1}(n)|$ is 
increasing for sufficiently large $n$ by 
Corollary \ref{cor_ExpectationFormulaAbsgInvn_v2}. 
\end{proof} 

\newpage
\section{Conclusions}

We have identified a new sequence, 
$\{g^{-1}(n)\}_{n \geq 1}$, that is the Dirichlet inverse of the 
shifted strongly additive function, $g := \omega + 1$. 
As we discussed in the remarks in 
Section \ref{subSection_AConnectionToDistOfThePrimes}, 
it happens that there is a natural combinatorial interpretation to the 
distribution of distinct values 
of $|g^{-1}(n)|$ for $n \leq x$ involving the distribution of the 
primes $p \leq x$ at large $x$. 
In particular, the magnitude of $|g^{-1}(n)|$ depends only on the pattern of 
the exponents of the prime factorization of $n$. 
The signedness of $g^{-1}(n)$ is given by $\lambda(n)$ for all $n \geq 1$. 
This leads to a familiar dependence of the 
summatory functions $G^{-1}(x)$ on the distribution of the summatory function $L(x)$. 
Section \ref{Section_KeyApplications} 
provides equivalent characterizations of the limiting properties of 
$M(x)$ by exact formulas and asymptotic relations involving 
$G^{-1}(x)$ and $L(x)$. 
We emphasize that our new work on the Mertens function proved within this article 
is significant in providing a new window through which we can view bounding $M(x)$. 
The computational data generated in 
Table \ref{table_conjecture_Mertens_ginvSeq_approx_values} suggests numerically 
that the distribution of $G^{-1}(x)$ may be easier to work with 
than those of $M(x)$ or $L(x)$. 
The remarks given in Section \ref{subSection_AConnectionToDistOfThePrimes} 
about the direct combinatorial 
relation of the distinct (and repetition of) values of $|g^{-1}(n)|$ 
for $n \leq x$ are suggestive towards bounding main terms for $G^{-1}(x)$ along 
infinite subsequences in future work. 

One topic that we do not touch on in the article is to consider what 
correlation (if any) exists between $\lambda(n)$ and the unsigned sequence of $|g^{-1}(n)|$ 
with the limiting distribution proved in 
Corollary \ref{cor_CLT_VII}. 
Much in the same way that variants of the famous Erd\H{o}s-Kac theorem are historically established 
by defining random variables related to $\omega(n)$, 
we also suggest an analysis of the summatory function $G^{-1}(x)$ by scaling 
the explicitly distributed $|g^{-1}(n)|$ for $n \leq x$ as $x \rightarrow \infty$ by 
its signed weight of $\lambda(n)$ using an initial heuristic along these lines 
for future work. 

Another experiment illustrated in the online computational reference 
\cite{SCHMIDT-MERTENS-COMPUTATIONS} suggests that 
for many sufficiently large $x$ we may consider replacing the 
summatory function with other summands weighted by $\lambda(n)$. 
These alternate sums can be seen to average these sequences differently while still 
preserving the original asymptotic order of $|G^{-1}(x)|$ heuristically. 
For example, each of the following three summatory functions offer a unique 
interpretation of an average of sorts that non-routinely ``mixes'' the values of 
$\lambda(n)$ with the unsigned sequence $|g^{-1}(n)|$ over $1 \leq n \leq x$: 
\begin{align*}
G_{\ast}^{-1}(x) & := \sum_{n \leq x} \frac{1}{2\gamma-1+\log n} \times 
     \sum_{d|n} \lambda\left(\frac{n}{d}\right) |g^{-1}(d)| \\ 
G_{\ast\ast}^{-1}(x) & := \sum_{n \leq x} \frac{1}{2\gamma-1+\log n} \times 
     \sum_{d|n} \lambda\left(\frac{n}{d}\right) g^{-1}(d) \\ 
G_{\ast\ast\ast}^{-1}(x) & := \sum_{n \leq x} \frac{1}{2\gamma-1+\log n} \times 
     \sum_{d|n} g^{-1}(d). 
\end{align*}
Then based on preliminary numerical results, a large proportion of the $y \leq x$ for 
large $x$ satisfy 
\[
     \left\lvert \frac{G_{\ast}^{-1}(y)}{G^{-1}(y)} \right\rvert^{-1}, 
     \left\lvert \frac{G_{\ast\ast}^{-1}(y)}{G^{-1}(y)} \right\rvert, 
     \left\lvert \frac{G_{\ast\ast\ast}^{-1}(y)}{G^{-1}(y)} \right\rvert \in 
     (0, 3]. 
\]
Variants of this type of summatory function identity exchange are similarly 
suggested for future work to extend the topics and new results proved in this article.

\section*{Acknowledgments}

We thank the following professors that offered 
discussion, feedback and correspondence while the article was written: 
Paul Pollack, Steven J.~Miller, Gerg\H{o} Nemes and Bruce Reznick. 
The work on the article was supported in part by 
funding made available within the School of Mathematics at the 
Georgia Institute of Technology in 2020 and 2021. 
Without this combined support, the article would not have been possible.

\newpage 
\renewcommand{\refname}{References} 
%\bibliography{glossaries-bibtex/thesis-references}{}
\bibliographystyle{plain}

\begin{thebibliography}{10}

\bibitem{APOSTOLANUMT}
T.~M. Apostol.
\newblock {\em Introduction to Analytic Number Theory}.
\newblock Springer--Verlag, 1976.

\bibitem{BILLINGSLY-CLT-PRIMEDIVFUNC}
P.~Billingsley.
\newblock On the central limit theorem for the prime divisor function.
\newblock {\em Amer. Math. Monthly}, 76(2):132--139, 1969.

\bibitem{ERDOS-KAC-REF}
P.~Erd{\H{o}}s and M.~Kac.
\newblock The guassian errors in the theory of additive arithmetic functions.
\newblock {\em American Journal of Mathematics}, 62(1):738--742, 1940.

\bibitem{FROBERG-1968}
C.~E. Fr{\"{o}}berg.
\newblock On the prime zeta function.
\newblock {\em BIT Numerical Mathematics}, 8:87--202, 1968.

\bibitem{HARDYWRIGHT}
G.~H. Hardy and E.~M. Wright, editors.
\newblock {\em An Introduction to the Theory of Numbers}.
\newblock Oxford University Press, 2008 (Sixth Edition).

\bibitem{HUMPHRIES-JNT-2013}
P.~Humphries.
\newblock The distribution of weighted sums of the {L}iouville function and
  {P}\'{o}lya's conjecture.
\newblock {\em J. Number Theory}, 133:545--582, 2013.

\bibitem{HURST-2017}
G.~Hurst.
\newblock Computations of the {M}ertens function and improved bounds on the
  {M}ertens conjecture.
\newblock {\em Math. Comp.}, 87:1013--1028, 2018.

\bibitem{CLT-RANDOM-ORDERED-FACTS-2011}
H.~Hwang and S.~Janson.
\newblock A central limit theorem for random ordered factorizations of
  integers.
\newblock {\em Electron. J. Probab.}, 16(12):347--361, 2011.

\bibitem{IWANIEC-KOWALSKI}
H.~Iwaniec and E.~Kowalski.
\newblock {\em Analytic Number Theory}, volume~53.
\newblock AMS Colloquium Publications, 2004.

\bibitem{MREVISITED}
T.~Kotnik and H.~t{\'{e}}~Riele.
\newblock The {M}ertens conjecture revisited.
\newblock {\em Algorithmic Number Theory}, $7^{th}$ International Symposium,
  2006.

\bibitem{ORDER-MERTENSFN}
T.~Kotnik and J.~van~de Lune.
\newblock On the order of the {M}ertens function.
\newblock {\em Exp. Math.}, 2004.

\bibitem{LEHMAN-1960}
R.~S. Lehman.
\newblock On {L}iouville's function.
\newblock {\em Math. Comput.}, 14:311--320, 1960.

\bibitem{MV}
H.~L. Montgomery and R.~C. Vaughan.
\newblock {\em Multiplicative Number Theory: I. Classical Theory}.
\newblock Cambridge, 2006.

\bibitem{NEMES2015C}
G.~Nemes.
\newblock The resurgence properties of the incomplete gamma function ii.
\newblock {\em Stud. Appl. Math.}, 135(1):86--116, 2015.

\bibitem{NEMES2016}
G.~Nemes.
\newblock The resurgence properties of the incomplete gamma function i.
\newblock {\em Anal. Appl. (Singap.)}, 14(5):631--677, 2016.

\bibitem{NEMES2019}
G.~Nemes and A.~B.~Olde Daalhuis.
\newblock Asymptotic expansions for the incomplete gamma function in the
  transition regions.
\newblock {\em Math. Comp.}, 88(318):1805--1827, 2019.

\bibitem{NG-MERTENS}
N.~Ng.
\newblock The distribution of the summatory function of the {M}{\'{o}}bius
  function.
\newblock {\em Proc. London Math. Soc.}, 89(3):361--389, 2004.

\bibitem{ODLYZ-TRIELE}
A.~M. Odlyzko and H.~J.~J. t{\'{e}}~Riele.
\newblock Disproof of the {M}ertens conjecture.
\newblock {\em J. Reine Angew. Math}, 1985.

\bibitem{NISTHB}
Frank W.~J. Olver, Daniel~W. Lozier, Ronald~F. Boisvert, and Charles~W. Clark,
  editors.
\newblock {\em {NIST} Handbook of Mathematical Functions}.
\newblock Cambridge University Press, 2010.

\bibitem{RENYI-TURAN}
A.~Renyi and P.~Turan.
\newblock On a theorem of {E}rd{\H{o}}s-{K}ac.
\newblock {\em Acta Arithmetica}, 4(1):71--84, 1958.

\bibitem{PRIMEREC}
P.~Ribenboim.
\newblock {\em The new book of prime number records}.
\newblock Springer, 1996.

\bibitem{ROBIN-PRIMEOMEGAFUNC-BOUNDS}
G.~Robin.
\newblock Estimate of the {C}hebyshev function $\theta$ on the $k^{th}$ prime
  number and large values of the number of prime divisors function $\omega(n)$
  of $n$.
\newblock {\em Acta Arith.}, 42(4):367--389, 1983.

\bibitem{ROSSER-SCHOENFELD-1962}
J.~B. Rosser and L.~Schoenfeld.
\newblock Approximate formulas for some functions of prime numbers.
\newblock {\em Illinois J. Math.}, 6:64--94, 1962.

\bibitem{SCHMIDT-MERTENS-COMPUTATIONS}
M.~D. Schmidt.
\newblock {S}age{M}ath and {M}athematica software for computations with the
  {M}ertens function, 2021.
\newblock \url{https://github.com/maxieds/MertensFunctionComputations}.

\bibitem{OEIS}
N.~J.~A. Sloane.
\newblock The {O}nline {E}ncyclopedia of {I}nteger {S}equences, 2021.
\newblock \url{http://oeis.org}.

\bibitem{SOUND-MERTENS-ANNALS}
K.~Soundararajan.
\newblock Partial sums of the {M}{\"{o}}bius function.
\newblock {\em Annals of Mathematics}, 2009.

\bibitem{TITCHMARSH}
E.~C. Titchmarsh.
\newblock {\em The theory of the {R}iemann zeta function}.
\newblock Clarendon Press, 1951.

\end{thebibliography}

\setcounter{section}{0} 
\renewcommand{\thesection}{\Alph{section}} 

\newpage
\section{Appendix: Asymptotic formulas} 
\label{subSection_OtherFactsAndResults} 

We thank Gerg\H{o} Nemes from the Alfr\'{e}d R\'{e}nyi Institute of Mathematics for his 
careful notes on the limiting asymptotics for the sums identified in this section. 
We have adapted the communication of his proofs to establish the next few lemmas. 

\begin{facts}[The incomplete gamma function] 
\label{facts_ExpIntIncGammaFuncs} 
\begin{subequations}
The (upper) \emph{incomplete gamma function} is defined by \cite[\S 8.4]{NISTHB} 
\[
\Gamma(a, z) = \int_{z}^{\infty} t^{a-1} e^{-t} dt, a \in \mathbb{R}, |\arg z| < \pi.  
\]
The function $\Gamma(a, z)$ can be continued to an analytic function of $z$ on the 
universal covering of $\mathbb{C} \mathbin{\big\backslash} \{0\}$. 
For $a \in \mathbb{Z}^{+}$, the function $\Gamma(a, z)$ is an entire function of $z$. 
The following properties of $\Gamma(a, z)$ hold \cite[\S 8.4; \S 8.11(i)]{NISTHB}: 
\begin{align} 
\label{eqn_IncompleteGamma_PropA} 
\Gamma(a, z) & = (a-1)! \cdot e^{-z} \times \sum_{k=0}^{a-1} \frac{z^k}{k!}, \mathrm{\ for\ } 
     a \in \mathbb{Z}^{+}, z \in \mathbb{C}; \\ 
\label{eqn_IncompleteGamma_PropB} 
\Gamma(a, z) & \sim z^{a-1} e^{-z}, \mathrm{\ for\ fixed\ } a \in \mathbb{C}, \mathrm{\ as\ } z \rightarrow +\infty. 
\end{align}
Moreover, for real $z > 0$, as $z \rightarrow +\infty$ we have that \cite{NEMES2015C} 
\begin{equation} 
\label{eqn_IncompleteGamma_PropC}
\Gamma(z, z) = \sqrt{\frac{\pi}{2}} z^{z-\frac{1}{2}} e^{-z} + 
     O\left(z^{z-1} e^{-z}\right), 
\end{equation} 
If $z,a \rightarrow \infty$ with $z = \lambda a$ for some $\lambda > 1$ such that 
$(\lambda - 1)^{-1} = o(|a|^{1/2})$, then \cite{NEMES2015C}
\begin{equation}
\label{eqn_IncompleteGamma_PropD}
\Gamma(a, z) = z^a e^{-z} \times \sum_{n \geq 0} \frac{(-a)^n b_n(\lambda)}{(z-a)^{2n+1}}, 
\end{equation} 
where the sequence $b_n(\lambda)$ satisfies the characteristic relation that 
$b_0(\lambda) = 1$ and\footnote{
     An exact formula for $b_n(\lambda)$ is given in terms of the \emph{second-order Eulerian number triangle} 
     \cite[\seqnum{A008517}]{OEIS} as follows: 
     \[
          b_n(\lambda) = \sum_{k=0}^{n} \gkpEII{n}{k} \lambda^{k+1}. 
     \]
}
\[
b_n(\lambda) = \lambda(1-\lambda) b_{n-1}^{\prime}(\lambda) + \lambda(2n-1) b_{n-1}(\lambda), n \geq 1. 
\]
\end{subequations}
\end{facts} 

\begin{prop}
\label{prop_IncGammaLambdaTypeBounds_v1}
Let $a,z,\lambda$ be positive real parameters such that $z=\lambda a$. 
If $\lambda \in (0, 1)$, then as $z \rightarrow +\infty$ 
\[
\Gamma(a, z) = \Gamma(a) + O_{\lambda}\left(z^{a-1} e^{-z}\right). 
\]
If $\lambda > 1$, then as 
$z \rightarrow +\infty$ 
\[
\Gamma(a, z) = \frac{z^{a-1} e^{-z}}{1-\lambda^{-1}} + O_{\lambda}\left(z^{a-2} e^{-z}\right). 
\]
If $\lambda > 0.567142 > W(1)$ where $W(x)$ denotes the principal branch of the 
Lambert $W$-function for $x \geq 0$, 
then as $z \rightarrow +\infty$ 
\[
\Gamma(a, z e^{\pm\pi\imath}) = -e^{\pm \pi\imath a} \frac{z^{a-1} e^{z}}{1 + \lambda^{-1}} + 
     O_{\lambda}\left(z^{a-2} e^{z}\right). 
\]
\end{prop}
The first two asymptotic estmates are only useful when $\lambda$ is bounded away from the 
transition point at $1$. 
Note that we cannot write the last expansion above 
as $\Gamma(a, -z)$ directly unless $a \in \mathbb{Z}^{+}$ 
as the incomplete gamma function 
has a branch point at the origin with respect to its second variable. 
This function becomes a single-valued 
analytic function of its second input by continuation 
on the universal covering of $\mathbb{C} \mathbin{\big\backslash} \{0\}$. 
\begin{proof}
The first asymptotic estimate follows directly from the following 
asymptotic series expansion that holds as $z \rightarrow +\infty$ 
\cite[Eq.\ (2.1)]{NEMES2019}: 
\[
\Gamma(a, z) \sim \Gamma(a) + z^a e^{-z} \times \sum_{k \geq 0} 
     \frac{(-a)^k b_k(\lambda)}{(z-a)^{2k+1}}. 
\]
Using the notation from \eqref{eqn_IncompleteGamma_PropD} and \cite{NEMES2016}, 
we have that 
\[
\Gamma(a, z) = \frac{z^{a-1} e^{-z}}{1-\lambda^{-1}} + z^{a} e^{-z} R_1(a, \lambda). 
\]
From the bounds in \cite[\S 3.1]{NEMES2016}, we get that 
\[
\left\lvert z^{a} e^{-z} R_1(a, \lambda) \right\rvert \leq 
     z^a e^{-z} \times \frac{a \cdot b_1(\lambda)}{(z-a)^{3}} = 
     \frac{z^{a-2} e^{-z}}{(1-\lambda^{-1})^{3}}
\]
Note that the main and error terms in the previous equation can also be 
seen by applying the asymptotic series in 
\eqref{eqn_IncompleteGamma_PropD} directly. 

The proof of the third equation above follows from the following asymptotics 
\cite[Eq.\ (1.1)]{NEMES2015C}
\[
\Gamma(-a, z) \sim z^{-a} e^{-z} \times \sum_{n \geq 0} \frac{a^n b_n(-\lambda)}{(z+a)^{2n+1}}, 
\]
by setting $a \mapsto a e^{\pm \pi\imath}$ and $z \mapsto z e^{\pm \pi\imath}$ with 
$\lambda = z/a > 0.567142 > W(1)$. 
The restriction on the range of $\lambda$ over which the third formula holds is made to ensure that 
the last formula from the reference is valid at negative real $a$. 
\end{proof}

\begin{lemma}
\label{lemma_ConvenientIncGammaFuncTypePartialSumAsymptotics_v2}
For $x \rightarrow +\infty$, we have that 
\begin{align*}
S_1(x) & := \frac{x}{\log x} \times \left\lvert \sum_{1 \leq k \leq \floor{\log\log x}} 
     \frac{(-1)^k (\log\log x)^{k-1}}{(k-1)!} \right\rvert 
     = \frac{x}{2\sqrt{2\pi \log\log x}} + O\left(\frac{x}{(\log\log x)^{3/2}}\right). 
\end{align*}
\end{lemma}
\begin{proof}
We have for $n \geq 1$ and any $t > 0$ by 
\eqref{eqn_IncompleteGamma_PropA} that 
\[
\sum_{1 \leq k \leq n} \frac{(-1)^k t^{k-1}}{(k-1)!} = -e^{-t} \times 
     \frac{\Gamma(n, -t)}{(n-1)!}. 
\]
Suppose that $t = n + \xi$ with $\xi = O(1)$ (e.g., so we can take the floor of the input $n$ to 
truncate the last sum). By the third formula 
in Proposition \ref{prop_IncGammaLambdaTypeBounds_v1} 
with the parameters $(a, z, \lambda) \mapsto \left(n, t, 1 + \frac{\xi}{n}\right)$, 
we deduce that as $n,t \rightarrow +\infty$. 
\begin{equation}
\label{eqn_ProofTag_lemma_ConvenientIncGammaFuncTypePartialSumAsymptotics_v2}
\Gamma(n, -t) = (-1)^{n+1} \times \frac{t^n e^{t}}{t+n} + 
     O\left(\frac{n t^n e^{t}}{(t+n)^3}\right) = 
     (-1)^{n+1} \frac{t^n e^t}{2n} + O\left(\frac{t^{n-1} e^t}{n}\right). 
\end{equation}
Accordingly, we see that 
\[
\sum_{1 \leq k \leq n} \frac{(-1)^k t^{k-1}}{(k-1)!} = 
     (-1)^{n} \frac{t^n}{2n!} + O\left(\frac{t^{n-1}}{n!}\right). 
\]
By the variant of Stirling's formula in \cite[\cf Eq.\ (5.11.8)]{NISTHB}, we have 
\[
n! = \Gamma(1 + t - \xi) = \sqrt{2\pi} \cdot t^{t-\xi+1/2} e^{-t} \left(1 + O(t^{-1})\right) = 
     \sqrt{2\pi} \cdot t^{n+1/2} e^{-t} \left(1 + O(t^{-1})\right). 
\]
Hence, as $n \rightarrow +\infty$ with $t := n + \xi$ and $\xi = O(1)$, we obtain that 
\[
\sum_{k=1}^{n} \frac{(-1)^k t^{k-1}}{(k-1)!} = (-1)^n \frac{e^t}{2 \sqrt{2\pi t}} + 
     O\left(\frac{e^t}{t^{3/2}}\right). 
\]
The conclusion follows by taking $n := \floor{\log\log x}$, 
$t := \log\log x$ and applying the triangle inequality 
to obtain the result. 
\end{proof}

\begin{lemma}
\label{lemma_ConvenientIncGammaFuncTypePartialSumAsymptotics_v3}
For $x \rightarrow +\infty$, we have that 
\begin{align*}
S_3(x) := \sum_{1 \leq k \leq \floor{2\log\log x}} \frac{(\log\log x)^{k-1/2}}{(2k-1) (k-1)!} = 
     \frac{\log x}{2\sqrt{\log\log x}} + O\left(\frac{\log x}{(\log\log x)^{3/2}}\right).       
\end{align*}
\end{lemma}
\begin{proof}
For $n \geq 1$ and any $t > 0$, let 
\[
\widetilde{S}_n(t) := \sum_{1 \leq k \leq n} \frac{t^{k-1}}{(2k-1) (k-1)!}. 
\]
By the formula in \eqref{eqn_IncompleteGamma_PropA} and a change of variable, we get that 
\begin{align*}
\widetilde{S}_n(t) & = \int_0^1 \left(\sum_{k=1}^{n} \frac{(s^2t)^{k-1}}{(k-1)!}\right) ds \\ 
     & = \frac{1}{(n-1)!} \times \int_0^1 e^{s^2t} \Gamma(n, s^2t) ds \\ 
     & = \frac{1}{2t^{1/2}(n-1)!} \times \int_0^{t} \frac{e^{u}}{\sqrt{u}} \times \Gamma(n, u) du. 
\end{align*}
Integration by parts performed one time with 
\[
     \left\{\begin{array}{ll} 
     u_x = \Gamma(n, x) & v_x^{\prime} = \frac{e^{x}}{\sqrt{x}} dx \\ 
     u_x^{\prime} = -x^{n-1} e^{-x} dx & v_x = 
     \sqrt{\pi} \operatorname{erfi}\left(\sqrt{x}\right)
     \end{array} 
     \right\}, 
\]
implies that 
\begin{align}
\label{eqn_ProofTag_AppendixLastSumLemma_IBP_resulting_exp_terms_v1}
\widetilde{S}_n(t) & = 
     \frac{1}{2(n-1)!} \sqrt{\frac{\pi}{t}} \times \Gamma(n, t) \operatorname{erfi}\left(
     \sqrt{t}\right) + 
     \frac{1}{2(n-1)!} \sqrt{\frac{\pi}{t}} \times \int_0^{t} u^{n-1} e^{-u} 
     \operatorname{erfi}\left(\sqrt{u}\right) du. 
\end{align}
For the remainder of the proof, we assume that $t = \frac{n}{2} + \xi$ where $\xi = O(1)$. 
By \cite[Eq.\ (7.12.1)]{NISTHB} and \eqref{eqn_Erfix_KnownAsymptoticSeries_v1}, 
we find that as $t \rightarrow +\infty$ 
\[
e^{-t} \operatorname{erfi}\left(\sqrt{t}\right) = \frac{1}{\sqrt{\pi t}} + O\left(\frac{1}{t^{3/2}}\right) = 
     O\left(\frac{1}{t^{1/2}}\right). 
\]
Consequently, we see that as $t \rightarrow +\infty$ 
\[
\frac{1}{2(n-1)!} \sqrt{\frac{\pi}{t}} \times \int_0^{t} u^{n-1} e^{-u} 
     \operatorname{erfi}\left(\sqrt{u}\right) du = 
     O\left(\frac{t^{n-2}}{(n-1)!}\right). 
\]
Applying the first estimate in 
Proposition \ref{prop_IncGammaLambdaTypeBounds_v1} with the parameters 
$(a, z, \lambda) \mapsto \left(n, t, \frac{1}{2} + \frac{\xi}{n}\right)$, 
we find that 
\[
\Gamma(n, t) = \Gamma(n) + O\left(t^{n-1} e^{-t}\right), \text{\ as\ } t \rightarrow + \infty.
\]
Thus, as $t \rightarrow +\infty$ we have that 
\[
\widetilde{S}_n(t) = \frac{e^t}{2t} + O\left(\frac{e^t}{t^2} + \frac{t^{n-2}}{(n-1)!}\right). 
\]
By applying \cite[Eq.\ (5.11.8)]{NISTHB} we have 
\[
(n-1)! = \Gamma(2t-2\xi) = O\left((2t)^{2t-2\xi-1/2} e^{-2t}\right) = 
     O\left(e^{-t} t^{n-1/2} \left(\frac{4}{e}\right)^t\right). 
\]
Whence, with $t := \frac{n}{2} + O(1)$ as $n \rightarrow +\infty$, we find 
\[
\widetilde{S}_n(t) = \frac{e^t}{2t} + O\left(\frac{e^t}{t^2}\right). 
\]
The conclusion follows taking $n = \floor{2\log\log x}$, $t = \log\log x$ and 
mulitplying $\widetilde{S}_n(t)$ by $(\log\log x)^{1/2}$. 
\end{proof}

\newpage
\section{Table: The Dirichlet inverse function $g^{-1}(n)$} 
\label{table_conjecture_Mertens_ginvSeq_approx_values}

\begin{table}[ht!]

\centering

\tiny
\begin{equation*}
\boxed{
\begin{array}{cc|cc|ccc|cc|ccc}
 n & \mathbf{Primes} & \mathbf{Sqfree} & \mathbf{PPower} & g^{-1}(n) & 
 \lambda(n) g^{-1}(n) - \widehat{f}_1(n) & 
 \frac{\sum_{d|n} C_{\Omega(d)}(d)}{|g^{-1}(n)|} & 
 \mathcal{L}_{+}(n) & \mathcal{L}_{-}(n) & 
 G^{-1}(n) & G^{-1}_{+}(n) & G^{-1}_{-}(n) \\ \hline 
1 & 1^1 & \text{Y} & \text{N} & 1 & 0 & 1.0000000 & 1.000000 & 0.000000 & 1 & 1 & 0 \\
 2 & 2^1 & \text{Y} & \text{Y} & -2 & 0 & 1.0000000 & 0.500000 & 0.500000 & -1 & 1 & -2 \\
 3 & 3^1 & \text{Y} & \text{Y} & -2 & 0 & 1.0000000 & 0.333333 & 0.666667 & -3 & 1 & -4 \\
 4 & 2^2 & \text{N} & \text{Y} & 2 & 0 & 1.5000000 & 0.500000 & 0.500000 & -1 & 3 & -4 \\
 5 & 5^1 & \text{Y} & \text{Y} & -2 & 0 & 1.0000000 & 0.400000 & 0.600000 & -3 & 3 & -6 \\
 6 & 2^1 3^1 & \text{Y} & \text{N} & 5 & 0 & 1.0000000 & 0.500000 & 0.500000 & 2 & 8 & -6 \\
 7 & 7^1 & \text{Y} & \text{Y} & -2 & 0 & 1.0000000 & 0.428571 & 0.571429 & 0 & 8 & -8 \\
 8 & 2^3 & \text{N} & \text{Y} & -2 & 0 & 2.0000000 & 0.375000 & 0.625000 & -2 & 8 & -10 \\
 9 & 3^2 & \text{N} & \text{Y} & 2 & 0 & 1.5000000 & 0.444444 & 0.555556 & 0 & 10 & -10 \\
 10 & 2^1 5^1 & \text{Y} & \text{N} & 5 & 0 & 1.0000000 & 0.500000 & 0.500000 & 5 & 15 & -10 \\
 11 & 11^1 & \text{Y} & \text{Y} & -2 & 0 & 1.0000000 & 0.454545 & 0.545455 & 3 & 15 & -12 \\
 12 & 2^2 3^1 & \text{N} & \text{N} & -7 & 2 & 1.2857143 & 0.416667 & 0.583333 & -4 & 15 & -19 \\
 13 & 13^1 & \text{Y} & \text{Y} & -2 & 0 & 1.0000000 & 0.384615 & 0.615385 & -6 & 15 & -21 \\
 14 & 2^1 7^1 & \text{Y} & \text{N} & 5 & 0 & 1.0000000 & 0.428571 & 0.571429 & -1 & 20 & -21 \\
 15 & 3^1 5^1 & \text{Y} & \text{N} & 5 & 0 & 1.0000000 & 0.466667 & 0.533333 & 4 & 25 & -21 \\
 16 & 2^4 & \text{N} & \text{Y} & 2 & 0 & 2.5000000 & 0.500000 & 0.500000 & 6 & 27 & -21 \\
 17 & 17^1 & \text{Y} & \text{Y} & -2 & 0 & 1.0000000 & 0.470588 & 0.529412 & 4 & 27 & -23 \\
 18 & 2^1 3^2 & \text{N} & \text{N} & -7 & 2 & 1.2857143 & 0.444444 & 0.555556 & -3 & 27 & -30 \\
 19 & 19^1 & \text{Y} & \text{Y} & -2 & 0 & 1.0000000 & 0.421053 & 0.578947 & -5 & 27 & -32 \\
 20 & 2^2 5^1 & \text{N} & \text{N} & -7 & 2 & 1.2857143 & 0.400000 & 0.600000 & -12 & 27 & -39 \\
 21 & 3^1 7^1 & \text{Y} & \text{N} & 5 & 0 & 1.0000000 & 0.428571 & 0.571429 & -7 & 32 & -39 \\
 22 & 2^1 11^1 & \text{Y} & \text{N} & 5 & 0 & 1.0000000 & 0.454545 & 0.545455 & -2 & 37 & -39 \\
 23 & 23^1 & \text{Y} & \text{Y} & -2 & 0 & 1.0000000 & 0.434783 & 0.565217 & -4 & 37 & -41 \\
 24 & 2^3 3^1 & \text{N} & \text{N} & 9 & 4 & 1.5555556 & 0.458333 & 0.541667 & 5 & 46 & -41 \\
 25 & 5^2 & \text{N} & \text{Y} & 2 & 0 & 1.5000000 & 0.480000 & 0.520000 & 7 & 48 & -41 \\
 26 & 2^1 13^1 & \text{Y} & \text{N} & 5 & 0 & 1.0000000 & 0.500000 & 0.500000 & 12 & 53 & -41 \\
 27 & 3^3 & \text{N} & \text{Y} & -2 & 0 & 2.0000000 & 0.481481 & 0.518519 & 10 & 53 & -43 \\
 28 & 2^2 7^1 & \text{N} & \text{N} & -7 & 2 & 1.2857143 & 0.464286 & 0.535714 & 3 & 53 & -50 \\
 29 & 29^1 & \text{Y} & \text{Y} & -2 & 0 & 1.0000000 & 0.448276 & 0.551724 & 1 & 53 & -52 \\
 30 & 2^1 3^1 5^1 & \text{Y} & \text{N} & -16 & 0 & 1.0000000 & 0.433333 & 0.566667 & -15 & 53 & -68 \\
 31 & 31^1 & \text{Y} & \text{Y} & -2 & 0 & 1.0000000 & 0.419355 & 0.580645 & -17 & 53 & -70 \\
 32 & 2^5 & \text{N} & \text{Y} & -2 & 0 & 3.0000000 & 0.406250 & 0.593750 & -19 & 53 & -72 \\
 33 & 3^1 11^1 & \text{Y} & \text{N} & 5 & 0 & 1.0000000 & 0.424242 & 0.575758 & -14 & 58 & -72 \\
 34 & 2^1 17^1 & \text{Y} & \text{N} & 5 & 0 & 1.0000000 & 0.441176 & 0.558824 & -9 & 63 & -72 \\
 35 & 5^1 7^1 & \text{Y} & \text{N} & 5 & 0 & 1.0000000 & 0.457143 & 0.542857 & -4 & 68 & -72 \\
 36 & 2^2 3^2 & \text{N} & \text{N} & 14 & 9 & 1.3571429 & 0.472222 & 0.527778 & 10 & 82 & -72 \\
 37 & 37^1 & \text{Y} & \text{Y} & -2 & 0 & 1.0000000 & 0.459459 & 0.540541 & 8 & 82 & -74 \\
 38 & 2^1 19^1 & \text{Y} & \text{N} & 5 & 0 & 1.0000000 & 0.473684 & 0.526316 & 13 & 87 & -74 \\
 39 & 3^1 13^1 & \text{Y} & \text{N} & 5 & 0 & 1.0000000 & 0.487179 & 0.512821 & 18 & 92 & -74 \\
 40 & 2^3 5^1 & \text{N} & \text{N} & 9 & 4 & 1.5555556 & 0.500000 & 0.500000 & 27 & 101 & -74 \\
 41 & 41^1 & \text{Y} & \text{Y} & -2 & 0 & 1.0000000 & 0.487805 & 0.512195 & 25 & 101 & -76 \\
 42 & 2^1 3^1 7^1 & \text{Y} & \text{N} & -16 & 0 & 1.0000000 & 0.476190 & 0.523810 & 9 & 101 & -92 \\
 43 & 43^1 & \text{Y} & \text{Y} & -2 & 0 & 1.0000000 & 0.465116 & 0.534884 & 7 & 101 & -94 \\
 44 & 2^2 11^1 & \text{N} & \text{N} & -7 & 2 & 1.2857143 & 0.454545 & 0.545455 & 0 & 101 & -101 \\
 45 & 3^2 5^1 & \text{N} & \text{N} & -7 & 2 & 1.2857143 & 0.444444 & 0.555556 & -7 & 101 & -108 \\
 46 & 2^1 23^1 & \text{Y} & \text{N} & 5 & 0 & 1.0000000 & 0.456522 & 0.543478 & -2 & 106 & -108 \\
 47 & 47^1 & \text{Y} & \text{Y} & -2 & 0 & 1.0000000 & 0.446809 & 0.553191 & -4 & 106 & -110 \\
 48 & 2^4 3^1 & \text{N} & \text{N} & -11 & 6 & 1.8181818 & 0.437500 & 0.562500 & -15 & 106 & -121 \\ 
\end{array}
}
\end{equation*}

\bigskip\hrule\smallskip 

\captionsetup{singlelinecheck=off} 
\caption*{\textbf{\rm \bf Table \thesection:} 
          \textbf{Computations with $\mathbf{g^{-1}(n) \equiv (\omega+1)^{-1}(n)}$ 
          for $\mathbf{1 \leq n \leq 500}$.} 
          \begin{itemize}[noitemsep,topsep=0pt,leftmargin=0.23in] 
          \item[$\blacktriangleright$] 
          The column labeled \texttt{Primes} provides the prime factorization of each $n$ so that the values of 
          $\omega(n)$ and $\Omega(n)$ are easily extracted. 
          The columns labeled \texttt{Sqfree} and \texttt{PPower}, respectively, 
          list inclusion of $n$ in the sets of squarefree integers and the prime powers. 
          \item[$\blacktriangleright$] 
          The next three columns provide the 
          explicit values of the inverse function $g^{-1}(n)$ and compare its explicit value with other estimates. 
          We define the function $\widehat{f}_1(n) := \sum_{k=0}^{\omega(n)} \binom{\omega(n)}{k} \cdot k!$. 
          \item[$\blacktriangleright$] 
          The last columns indicate properties of the summatory function of $g^{-1}(n)$. 
          The notation for the densities of the sign weight of $g^{-1}(n)$ is defined as 
          $\mathcal{L}_{\pm}(x) := \frac{1}{n} \cdot \#\left\{n \leq x: \lambda(n) = \pm 1\right\}$. 
          The last three 
          columns then show the explicit components to the signed summatory function, 
          $G^{-1}(x) := \sum_{n \leq x} g^{-1}(n)$, decomposed into its 
          respective positive and negative magnitude sum contributions: $G^{-1}(x) = G^{-1}_{+}(x) + G^{-1}_{-}(x)$ where 
          $G^{-1}_{+}(x) > 0$ and $G^{-1}_{-}(x) < 0$ for all $x \geq 1$. 
          \end{itemize} 
          } 
\clearpage 

\end{table}

\newpage
\begin{table}[ht]

\centering

\tiny
\begin{equation*}
\boxed{
\begin{array}{cc|cc|ccc|cc|ccc}
 n & \mathbf{Primes} & \mathbf{Sqfree} & \mathbf{PPower} & g^{-1}(n) & 
 \lambda(n) g^{-1}(n) - \widehat{f}_1(n) & 
 \frac{\sum_{d|n} C_{\Omega(d)}(d)}{|g^{-1}(n)|} & 
 \mathcal{L}_{+}(n) & \mathcal{L}_{-}(n) & 
 G^{-1}(n) & G^{-1}_{+}(n) & G^{-1}_{-}(n) \\ \hline 
 49 & 7^2 & \text{N} & \text{Y} & 2 & 0 & 1.5000000 & 0.448980 & 0.551020 & -13 & 108 & -121 \\
 50 & 2^1 5^2 & \text{N} & \text{N} & -7 & 2 & 1.2857143 & 0.440000 & 0.560000 & -20 & 108 & -128 \\
 51 & 3^1 17^1 & \text{Y} & \text{N} & 5 & 0 & 1.0000000 & 0.450980 & 0.549020 & -15 & 113 & -128 \\
 52 & 2^2 13^1 & \text{N} & \text{N} & -7 & 2 & 1.2857143 & 0.442308 & 0.557692 & -22 & 113 & -135 \\
 53 & 53^1 & \text{Y} & \text{Y} & -2 & 0 & 1.0000000 & 0.433962 & 0.566038 & -24 & 113 & -137 \\
 54 & 2^1 3^3 & \text{N} & \text{N} & 9 & 4 & 1.5555556 & 0.444444 & 0.555556 & -15 & 122 & -137 \\
 55 & 5^1 11^1 & \text{Y} & \text{N} & 5 & 0 & 1.0000000 & 0.454545 & 0.545455 & -10 & 127 & -137 \\
 56 & 2^3 7^1 & \text{N} & \text{N} & 9 & 4 & 1.5555556 & 0.464286 & 0.535714 & -1 & 136 & -137 \\
 57 & 3^1 19^1 & \text{Y} & \text{N} & 5 & 0 & 1.0000000 & 0.473684 & 0.526316 & 4 & 141 & -137 \\
 58 & 2^1 29^1 & \text{Y} & \text{N} & 5 & 0 & 1.0000000 & 0.482759 & 0.517241 & 9 & 146 & -137 \\
 59 & 59^1 & \text{Y} & \text{Y} & -2 & 0 & 1.0000000 & 0.474576 & 0.525424 & 7 & 146 & -139 \\
 60 & 2^2 3^1 5^1 & \text{N} & \text{N} & 30 & 14 & 1.1666667 & 0.483333 & 0.516667 & 37 & 176 & -139 \\
 61 & 61^1 & \text{Y} & \text{Y} & -2 & 0 & 1.0000000 & 0.475410 & 0.524590 & 35 & 176 & -141 \\
 62 & 2^1 31^1 & \text{Y} & \text{N} & 5 & 0 & 1.0000000 & 0.483871 & 0.516129 & 40 & 181 & -141 \\
 63 & 3^2 7^1 & \text{N} & \text{N} & -7 & 2 & 1.2857143 & 0.476190 & 0.523810 & 33 & 181 & -148 \\
 64 & 2^6 & \text{N} & \text{Y} & 2 & 0 & 3.5000000 & 0.484375 & 0.515625 & 35 & 183 & -148 \\
 65 & 5^1 13^1 & \text{Y} & \text{N} & 5 & 0 & 1.0000000 & 0.492308 & 0.507692 & 40 & 188 & -148 \\
 66 & 2^1 3^1 11^1 & \text{Y} & \text{N} & -16 & 0 & 1.0000000 & 0.484848 & 0.515152 & 24 & 188 & -164 \\
 67 & 67^1 & \text{Y} & \text{Y} & -2 & 0 & 1.0000000 & 0.477612 & 0.522388 & 22 & 188 & -166 \\
 68 & 2^2 17^1 & \text{N} & \text{N} & -7 & 2 & 1.2857143 & 0.470588 & 0.529412 & 15 & 188 & -173 \\
 69 & 3^1 23^1 & \text{Y} & \text{N} & 5 & 0 & 1.0000000 & 0.478261 & 0.521739 & 20 & 193 & -173 \\
 70 & 2^1 5^1 7^1 & \text{Y} & \text{N} & -16 & 0 & 1.0000000 & 0.471429 & 0.528571 & 4 & 193 & -189 \\
 71 & 71^1 & \text{Y} & \text{Y} & -2 & 0 & 1.0000000 & 0.464789 & 0.535211 & 2 & 193 & -191 \\
 72 & 2^3 3^2 & \text{N} & \text{N} & -23 & 18 & 1.4782609 & 0.458333 & 0.541667 & -21 & 193 & -214 \\
 73 & 73^1 & \text{Y} & \text{Y} & -2 & 0 & 1.0000000 & 0.452055 & 0.547945 & -23 & 193 & -216 \\
 74 & 2^1 37^1 & \text{Y} & \text{N} & 5 & 0 & 1.0000000 & 0.459459 & 0.540541 & -18 & 198 & -216 \\
 75 & 3^1 5^2 & \text{N} & \text{N} & -7 & 2 & 1.2857143 & 0.453333 & 0.546667 & -25 & 198 & -223 \\
 76 & 2^2 19^1 & \text{N} & \text{N} & -7 & 2 & 1.2857143 & 0.447368 & 0.552632 & -32 & 198 & -230 \\
 77 & 7^1 11^1 & \text{Y} & \text{N} & 5 & 0 & 1.0000000 & 0.454545 & 0.545455 & -27 & 203 & -230 \\
 78 & 2^1 3^1 13^1 & \text{Y} & \text{N} & -16 & 0 & 1.0000000 & 0.448718 & 0.551282 & -43 & 203 & -246 \\
 79 & 79^1 & \text{Y} & \text{Y} & -2 & 0 & 1.0000000 & 0.443038 & 0.556962 & -45 & 203 & -248 \\
 80 & 2^4 5^1 & \text{N} & \text{N} & -11 & 6 & 1.8181818 & 0.437500 & 0.562500 & -56 & 203 & -259 \\
 81 & 3^4 & \text{N} & \text{Y} & 2 & 0 & 2.5000000 & 0.444444 & 0.555556 & -54 & 205 & -259 \\
 82 & 2^1 41^1 & \text{Y} & \text{N} & 5 & 0 & 1.0000000 & 0.451220 & 0.548780 & -49 & 210 & -259 \\
 83 & 83^1 & \text{Y} & \text{Y} & -2 & 0 & 1.0000000 & 0.445783 & 0.554217 & -51 & 210 & -261 \\
 84 & 2^2 3^1 7^1 & \text{N} & \text{N} & 30 & 14 & 1.1666667 & 0.452381 & 0.547619 & -21 & 240 & -261 \\
 85 & 5^1 17^1 & \text{Y} & \text{N} & 5 & 0 & 1.0000000 & 0.458824 & 0.541176 & -16 & 245 & -261 \\
 86 & 2^1 43^1 & \text{Y} & \text{N} & 5 & 0 & 1.0000000 & 0.465116 & 0.534884 & -11 & 250 & -261 \\
 87 & 3^1 29^1 & \text{Y} & \text{N} & 5 & 0 & 1.0000000 & 0.471264 & 0.528736 & -6 & 255 & -261 \\
 88 & 2^3 11^1 & \text{N} & \text{N} & 9 & 4 & 1.5555556 & 0.477273 & 0.522727 & 3 & 264 & -261 \\
 89 & 89^1 & \text{Y} & \text{Y} & -2 & 0 & 1.0000000 & 0.471910 & 0.528090 & 1 & 264 & -263 \\
 90 & 2^1 3^2 5^1 & \text{N} & \text{N} & 30 & 14 & 1.1666667 & 0.477778 & 0.522222 & 31 & 294 & -263 \\
 91 & 7^1 13^1 & \text{Y} & \text{N} & 5 & 0 & 1.0000000 & 0.483516 & 0.516484 & 36 & 299 & -263 \\
 92 & 2^2 23^1 & \text{N} & \text{N} & -7 & 2 & 1.2857143 & 0.478261 & 0.521739 & 29 & 299 & -270 \\
 93 & 3^1 31^1 & \text{Y} & \text{N} & 5 & 0 & 1.0000000 & 0.483871 & 0.516129 & 34 & 304 & -270 \\
 94 & 2^1 47^1 & \text{Y} & \text{N} & 5 & 0 & 1.0000000 & 0.489362 & 0.510638 & 39 & 309 & -270 \\
 95 & 5^1 19^1 & \text{Y} & \text{N} & 5 & 0 & 1.0000000 & 0.494737 & 0.505263 & 44 & 314 & -270 \\
 96 & 2^5 3^1 & \text{N} & \text{N} & 13 & 8 & 2.0769231 & 0.500000 & 0.500000 & 57 & 327 & -270 \\
 97 & 97^1 & \text{Y} & \text{Y} & -2 & 0 & 1.0000000 & 0.494845 & 0.505155 & 55 & 327 & -272 \\
 98 & 2^1 7^2 & \text{N} & \text{N} & -7 & 2 & 1.2857143 & 0.489796 & 0.510204 & 48 & 327 & -279 \\
 99 & 3^2 11^1 & \text{N} & \text{N} & -7 & 2 & 1.2857143 & 0.484848 & 0.515152 & 41 & 327 & -286 \\
 100 & 2^2 5^2 & \text{N} & \text{N} & 14 & 9 & 1.3571429 & 0.490000 & 0.510000 & 55 & 341 & -286 \\
 101 & 101^1 & \text{Y} & \text{Y} & -2 & 0 & 1.0000000 & 0.485149 & 0.514851 & 53 & 341 & -288 \\
 102 & 2^1 3^1 17^1 & \text{Y} & \text{N} & -16 & 0 & 1.0000000 & 0.480392 & 0.519608 & 37 & 341 & -304 \\
 103 & 103^1 & \text{Y} & \text{Y} & -2 & 0 & 1.0000000 & 0.475728 & 0.524272 & 35 & 341 & -306 \\
 104 & 2^3 13^1 & \text{N} & \text{N} & 9 & 4 & 1.5555556 & 0.480769 & 0.519231 & 44 & 350 & -306 \\
 105 & 3^1 5^1 7^1 & \text{Y} & \text{N} & -16 & 0 & 1.0000000 & 0.476190 & 0.523810 & 28 & 350 & -322 \\
 106 & 2^1 53^1 & \text{Y} & \text{N} & 5 & 0 & 1.0000000 & 0.481132 & 0.518868 & 33 & 355 & -322 \\
 107 & 107^1 & \text{Y} & \text{Y} & -2 & 0 & 1.0000000 & 0.476636 & 0.523364 & 31 & 355 & -324 \\
 108 & 2^2 3^3 & \text{N} & \text{N} & -23 & 18 & 1.4782609 & 0.472222 & 0.527778 & 8 & 355 & -347 \\
 109 & 109^1 & \text{Y} & \text{Y} & -2 & 0 & 1.0000000 & 0.467890 & 0.532110 & 6 & 355 & -349 \\
 110 & 2^1 5^1 11^1 & \text{Y} & \text{N} & -16 & 0 & 1.0000000 & 0.463636 & 0.536364 & -10 & 355 & -365 \\
 111 & 3^1 37^1 & \text{Y} & \text{N} & 5 & 0 & 1.0000000 & 0.468468 & 0.531532 & -5 & 360 & -365 \\
 112 & 2^4 7^1 & \text{N} & \text{N} & -11 & 6 & 1.8181818 & 0.464286 & 0.535714 & -16 & 360 & -376 \\
 113 & 113^1 & \text{Y} & \text{Y} & -2 & 0 & 1.0000000 & 0.460177 & 0.539823 & -18 & 360 & -378 \\
 114 & 2^1 3^1 19^1 & \text{Y} & \text{N} & -16 & 0 & 1.0000000 & 0.456140 & 0.543860 & -34 & 360 & -394 \\
 115 & 5^1 23^1 & \text{Y} & \text{N} & 5 & 0 & 1.0000000 & 0.460870 & 0.539130 & -29 & 365 & -394 \\
 116 & 2^2 29^1 & \text{N} & \text{N} & -7 & 2 & 1.2857143 & 0.456897 & 0.543103 & -36 & 365 & -401 \\
 117 & 3^2 13^1 & \text{N} & \text{N} & -7 & 2 & 1.2857143 & 0.452991 & 0.547009 & -43 & 365 & -408 \\
 118 & 2^1 59^1 & \text{Y} & \text{N} & 5 & 0 & 1.0000000 & 0.457627 & 0.542373 & -38 & 370 & -408 \\
 119 & 7^1 17^1 & \text{Y} & \text{N} & 5 & 0 & 1.0000000 & 0.462185 & 0.537815 & -33 & 375 & -408 \\
 120 & 2^3 3^1 5^1 & \text{N} & \text{N} & -48 & 32 & 1.3333333 & 0.458333 & 0.541667 & -81 & 375 & -456 \\
 121 & 11^2 & \text{N} & \text{Y} & 2 & 0 & 1.5000000 & 0.462810 & 0.537190 & -79 & 377 & -456 \\
 122 & 2^1 61^1 & \text{Y} & \text{N} & 5 & 0 & 1.0000000 & 0.467213 & 0.532787 & -74 & 382 & -456 \\
 123 & 3^1 41^1 & \text{Y} & \text{N} & 5 & 0 & 1.0000000 & 0.471545 & 0.528455 & -69 & 387 & -456 \\
 124 & 2^2 31^1 & \text{N} & \text{N} & -7 & 2 & 1.2857143 & 0.467742 & 0.532258 & -76 & 387 & -463 \\ 
\end{array}
}
\end{equation*}
\clearpage 

\end{table} 


\newpage
\begin{table}[ht]

\centering

\tiny
\begin{equation*}
\boxed{
\begin{array}{cc|cc|ccc|cc|ccc}
 n & \mathbf{Primes} & \mathbf{Sqfree} & \mathbf{PPower} & g^{-1}(n) & 
 \lambda(n) g^{-1}(n) - \widehat{f}_1(n) & 
 \frac{\sum_{d|n} C_{\Omega(d)}(d)}{|g^{-1}(n)|} & 
 \mathcal{L}_{+}(n) & \mathcal{L}_{-}(n) & 
 G^{-1}(n) & G^{-1}_{+}(n) & G^{-1}_{-}(n) \\ \hline 
 125 & 5^3 & \text{N} & \text{Y} & -2 & 0 & 2.0000000 & 0.464000 & 0.536000 & -78 & 387 & -465 \\
 126 & 2^1 3^2 7^1 & \text{N} & \text{N} & 30 & 14 & 1.1666667 & 0.468254 & 0.531746 & -48 & 417 & -465 \\
 127 & 127^1 & \text{Y} & \text{Y} & -2 & 0 & 1.0000000 & 0.464567 & 0.535433 & -50 & 417 & -467 \\
 128 & 2^7 & \text{N} & \text{Y} & -2 & 0 & 4.0000000 & 0.460938 & 0.539062 & -52 & 417 & -469 \\
 129 & 3^1 43^1 & \text{Y} & \text{N} & 5 & 0 & 1.0000000 & 0.465116 & 0.534884 & -47 & 422 & -469 \\
 130 & 2^1 5^1 13^1 & \text{Y} & \text{N} & -16 & 0 & 1.0000000 & 0.461538 & 0.538462 & -63 & 422 & -485 \\
 131 & 131^1 & \text{Y} & \text{Y} & -2 & 0 & 1.0000000 & 0.458015 & 0.541985 & -65 & 422 & -487 \\
 132 & 2^2 3^1 11^1 & \text{N} & \text{N} & 30 & 14 & 1.1666667 & 0.462121 & 0.537879 & -35 & 452 & -487 \\
 133 & 7^1 19^1 & \text{Y} & \text{N} & 5 & 0 & 1.0000000 & 0.466165 & 0.533835 & -30 & 457 & -487 \\
 134 & 2^1 67^1 & \text{Y} & \text{N} & 5 & 0 & 1.0000000 & 0.470149 & 0.529851 & -25 & 462 & -487 \\
 135 & 3^3 5^1 & \text{N} & \text{N} & 9 & 4 & 1.5555556 & 0.474074 & 0.525926 & -16 & 471 & -487 \\
 136 & 2^3 17^1 & \text{N} & \text{N} & 9 & 4 & 1.5555556 & 0.477941 & 0.522059 & -7 & 480 & -487 \\
 137 & 137^1 & \text{Y} & \text{Y} & -2 & 0 & 1.0000000 & 0.474453 & 0.525547 & -9 & 480 & -489 \\
 138 & 2^1 3^1 23^1 & \text{Y} & \text{N} & -16 & 0 & 1.0000000 & 0.471014 & 0.528986 & -25 & 480 & -505 \\
 139 & 139^1 & \text{Y} & \text{Y} & -2 & 0 & 1.0000000 & 0.467626 & 0.532374 & -27 & 480 & -507 \\
 140 & 2^2 5^1 7^1 & \text{N} & \text{N} & 30 & 14 & 1.1666667 & 0.471429 & 0.528571 & 3 & 510 & -507 \\
 141 & 3^1 47^1 & \text{Y} & \text{N} & 5 & 0 & 1.0000000 & 0.475177 & 0.524823 & 8 & 515 & -507 \\
 142 & 2^1 71^1 & \text{Y} & \text{N} & 5 & 0 & 1.0000000 & 0.478873 & 0.521127 & 13 & 520 & -507 \\
 143 & 11^1 13^1 & \text{Y} & \text{N} & 5 & 0 & 1.0000000 & 0.482517 & 0.517483 & 18 & 525 & -507 \\
 144 & 2^4 3^2 & \text{N} & \text{N} & 34 & 29 & 1.6176471 & 0.486111 & 0.513889 & 52 & 559 & -507 \\
 145 & 5^1 29^1 & \text{Y} & \text{N} & 5 & 0 & 1.0000000 & 0.489655 & 0.510345 & 57 & 564 & -507 \\
 146 & 2^1 73^1 & \text{Y} & \text{N} & 5 & 0 & 1.0000000 & 0.493151 & 0.506849 & 62 & 569 & -507 \\
 147 & 3^1 7^2 & \text{N} & \text{N} & -7 & 2 & 1.2857143 & 0.489796 & 0.510204 & 55 & 569 & -514 \\
 148 & 2^2 37^1 & \text{N} & \text{N} & -7 & 2 & 1.2857143 & 0.486486 & 0.513514 & 48 & 569 & -521 \\
 149 & 149^1 & \text{Y} & \text{Y} & -2 & 0 & 1.0000000 & 0.483221 & 0.516779 & 46 & 569 & -523 \\
 150 & 2^1 3^1 5^2 & \text{N} & \text{N} & 30 & 14 & 1.1666667 & 0.486667 & 0.513333 & 76 & 599 & -523 \\
 151 & 151^1 & \text{Y} & \text{Y} & -2 & 0 & 1.0000000 & 0.483444 & 0.516556 & 74 & 599 & -525 \\
 152 & 2^3 19^1 & \text{N} & \text{N} & 9 & 4 & 1.5555556 & 0.486842 & 0.513158 & 83 & 608 & -525 \\
 153 & 3^2 17^1 & \text{N} & \text{N} & -7 & 2 & 1.2857143 & 0.483660 & 0.516340 & 76 & 608 & -532 \\
 154 & 2^1 7^1 11^1 & \text{Y} & \text{N} & -16 & 0 & 1.0000000 & 0.480519 & 0.519481 & 60 & 608 & -548 \\
 155 & 5^1 31^1 & \text{Y} & \text{N} & 5 & 0 & 1.0000000 & 0.483871 & 0.516129 & 65 & 613 & -548 \\
 156 & 2^2 3^1 13^1 & \text{N} & \text{N} & 30 & 14 & 1.1666667 & 0.487179 & 0.512821 & 95 & 643 & -548 \\
 157 & 157^1 & \text{Y} & \text{Y} & -2 & 0 & 1.0000000 & 0.484076 & 0.515924 & 93 & 643 & -550 \\
 158 & 2^1 79^1 & \text{Y} & \text{N} & 5 & 0 & 1.0000000 & 0.487342 & 0.512658 & 98 & 648 & -550 \\
 159 & 3^1 53^1 & \text{Y} & \text{N} & 5 & 0 & 1.0000000 & 0.490566 & 0.509434 & 103 & 653 & -550 \\
 160 & 2^5 5^1 & \text{N} & \text{N} & 13 & 8 & 2.0769231 & 0.493750 & 0.506250 & 116 & 666 & -550 \\
 161 & 7^1 23^1 & \text{Y} & \text{N} & 5 & 0 & 1.0000000 & 0.496894 & 0.503106 & 121 & 671 & -550 \\
 162 & 2^1 3^4 & \text{N} & \text{N} & -11 & 6 & 1.8181818 & 0.493827 & 0.506173 & 110 & 671 & -561 \\
 163 & 163^1 & \text{Y} & \text{Y} & -2 & 0 & 1.0000000 & 0.490798 & 0.509202 & 108 & 671 & -563 \\
 164 & 2^2 41^1 & \text{N} & \text{N} & -7 & 2 & 1.2857143 & 0.487805 & 0.512195 & 101 & 671 & -570 \\
 165 & 3^1 5^1 11^1 & \text{Y} & \text{N} & -16 & 0 & 1.0000000 & 0.484848 & 0.515152 & 85 & 671 & -586 \\
 166 & 2^1 83^1 & \text{Y} & \text{N} & 5 & 0 & 1.0000000 & 0.487952 & 0.512048 & 90 & 676 & -586 \\
 167 & 167^1 & \text{Y} & \text{Y} & -2 & 0 & 1.0000000 & 0.485030 & 0.514970 & 88 & 676 & -588 \\
 168 & 2^3 3^1 7^1 & \text{N} & \text{N} & -48 & 32 & 1.3333333 & 0.482143 & 0.517857 & 40 & 676 & -636 \\
 169 & 13^2 & \text{N} & \text{Y} & 2 & 0 & 1.5000000 & 0.485207 & 0.514793 & 42 & 678 & -636 \\
 170 & 2^1 5^1 17^1 & \text{Y} & \text{N} & -16 & 0 & 1.0000000 & 0.482353 & 0.517647 & 26 & 678 & -652 \\
 171 & 3^2 19^1 & \text{N} & \text{N} & -7 & 2 & 1.2857143 & 0.479532 & 0.520468 & 19 & 678 & -659 \\
 172 & 2^2 43^1 & \text{N} & \text{N} & -7 & 2 & 1.2857143 & 0.476744 & 0.523256 & 12 & 678 & -666 \\
 173 & 173^1 & \text{Y} & \text{Y} & -2 & 0 & 1.0000000 & 0.473988 & 0.526012 & 10 & 678 & -668 \\
 174 & 2^1 3^1 29^1 & \text{Y} & \text{N} & -16 & 0 & 1.0000000 & 0.471264 & 0.528736 & -6 & 678 & -684 \\
 175 & 5^2 7^1 & \text{N} & \text{N} & -7 & 2 & 1.2857143 & 0.468571 & 0.531429 & -13 & 678 & -691 \\
 176 & 2^4 11^1 & \text{N} & \text{N} & -11 & 6 & 1.8181818 & 0.465909 & 0.534091 & -24 & 678 & -702 \\
 177 & 3^1 59^1 & \text{Y} & \text{N} & 5 & 0 & 1.0000000 & 0.468927 & 0.531073 & -19 & 683 & -702 \\
 178 & 2^1 89^1 & \text{Y} & \text{N} & 5 & 0 & 1.0000000 & 0.471910 & 0.528090 & -14 & 688 & -702 \\
 179 & 179^1 & \text{Y} & \text{Y} & -2 & 0 & 1.0000000 & 0.469274 & 0.530726 & -16 & 688 & -704 \\
 180 & 2^2 3^2 5^1 & \text{N} & \text{N} & -74 & 58 & 1.2162162 & 0.466667 & 0.533333 & -90 & 688 & -778 \\
 181 & 181^1 & \text{Y} & \text{Y} & -2 & 0 & 1.0000000 & 0.464088 & 0.535912 & -92 & 688 & -780 \\
 182 & 2^1 7^1 13^1 & \text{Y} & \text{N} & -16 & 0 & 1.0000000 & 0.461538 & 0.538462 & -108 & 688 & -796 \\
 183 & 3^1 61^1 & \text{Y} & \text{N} & 5 & 0 & 1.0000000 & 0.464481 & 0.535519 & -103 & 693 & -796 \\
 184 & 2^3 23^1 & \text{N} & \text{N} & 9 & 4 & 1.5555556 & 0.467391 & 0.532609 & -94 & 702 & -796 \\
 185 & 5^1 37^1 & \text{Y} & \text{N} & 5 & 0 & 1.0000000 & 0.470270 & 0.529730 & -89 & 707 & -796 \\
 186 & 2^1 3^1 31^1 & \text{Y} & \text{N} & -16 & 0 & 1.0000000 & 0.467742 & 0.532258 & -105 & 707 & -812 \\
 187 & 11^1 17^1 & \text{Y} & \text{N} & 5 & 0 & 1.0000000 & 0.470588 & 0.529412 & -100 & 712 & -812 \\
 188 & 2^2 47^1 & \text{N} & \text{N} & -7 & 2 & 1.2857143 & 0.468085 & 0.531915 & -107 & 712 & -819 \\
 189 & 3^3 7^1 & \text{N} & \text{N} & 9 & 4 & 1.5555556 & 0.470899 & 0.529101 & -98 & 721 & -819 \\
 190 & 2^1 5^1 19^1 & \text{Y} & \text{N} & -16 & 0 & 1.0000000 & 0.468421 & 0.531579 & -114 & 721 & -835 \\
 191 & 191^1 & \text{Y} & \text{Y} & -2 & 0 & 1.0000000 & 0.465969 & 0.534031 & -116 & 721 & -837 \\
 192 & 2^6 3^1 & \text{N} & \text{N} & -15 & 10 & 2.3333333 & 0.463542 & 0.536458 & -131 & 721 & -852 \\
 193 & 193^1 & \text{Y} & \text{Y} & -2 & 0 & 1.0000000 & 0.461140 & 0.538860 & -133 & 721 & -854 \\
 194 & 2^1 97^1 & \text{Y} & \text{N} & 5 & 0 & 1.0000000 & 0.463918 & 0.536082 & -128 & 726 & -854 \\
 195 & 3^1 5^1 13^1 & \text{Y} & \text{N} & -16 & 0 & 1.0000000 & 0.461538 & 0.538462 & -144 & 726 & -870 \\
 196 & 2^2 7^2 & \text{N} & \text{N} & 14 & 9 & 1.3571429 & 0.464286 & 0.535714 & -130 & 740 & -870 \\
 197 & 197^1 & \text{Y} & \text{Y} & -2 & 0 & 1.0000000 & 0.461929 & 0.538071 & -132 & 740 & -872 \\
 198 & 2^1 3^2 11^1 & \text{N} & \text{N} & 30 & 14 & 1.1666667 & 0.464646 & 0.535354 & -102 & 770 & -872 \\
 199 & 199^1 & \text{Y} & \text{Y} & -2 & 0 & 1.0000000 & 0.462312 & 0.537688 & -104 & 770 & -874 \\
 200 & 2^3 5^2 & \text{N} & \text{N} & -23 & 18 & 1.4782609 & 0.460000 & 0.540000 & -127 & 770 & -897 \\ 
\end{array}
}
\end{equation*}
\clearpage 

\end{table} 

\newpage
\begin{table}[ht]

\centering

\tiny
\begin{equation*}
\boxed{
\begin{array}{cc|cc|ccc|cc|ccc}
 n & \mathbf{Primes} & \mathbf{Sqfree} & \mathbf{PPower} & g^{-1}(n) & 
 \lambda(n) g^{-1}(n) - \widehat{f}_1(n) & 
 \frac{\sum_{d|n} C_{\Omega(d)}(d)}{|g^{-1}(n)|} & 
 \mathcal{L}_{+}(n) & \mathcal{L}_{-}(n) & 
 G^{-1}(n) & G^{-1}_{+}(n) & G^{-1}_{-}(n) \\ \hline 
 201 & 3^1 67^1 & \text{Y} & \text{N} & 5 & 0 & 1.0000000 & 0.462687 & 0.537313 & -122 & 775 & -897 \\
 202 & 2^1 101^1 & \text{Y} & \text{N} & 5 & 0 & 1.0000000 & 0.465347 & 0.534653 & -117 & 780 & -897 \\
 203 & 7^1 29^1 & \text{Y} & \text{N} & 5 & 0 & 1.0000000 & 0.467980 & 0.532020 & -112 & 785 & -897 \\
 204 & 2^2 3^1 17^1 & \text{N} & \text{N} & 30 & 14 & 1.1666667 & 0.470588 & 0.529412 & -82 & 815 & -897 \\
 205 & 5^1 41^1 & \text{Y} & \text{N} & 5 & 0 & 1.0000000 & 0.473171 & 0.526829 & -77 & 820 & -897 \\
 206 & 2^1 103^1 & \text{Y} & \text{N} & 5 & 0 & 1.0000000 & 0.475728 & 0.524272 & -72 & 825 & -897 \\
 207 & 3^2 23^1 & \text{N} & \text{N} & -7 & 2 & 1.2857143 & 0.473430 & 0.526570 & -79 & 825 & -904 \\
 208 & 2^4 13^1 & \text{N} & \text{N} & -11 & 6 & 1.8181818 & 0.471154 & 0.528846 & -90 & 825 & -915 \\
 209 & 11^1 19^1 & \text{Y} & \text{N} & 5 & 0 & 1.0000000 & 0.473684 & 0.526316 & -85 & 830 & -915 \\
 210 & 2^1 3^1 5^1 7^1 & \text{Y} & \text{N} & 65 & 0 & 1.0000000 & 0.476190 & 0.523810 & -20 & 895 & -915 \\
 211 & 211^1 & \text{Y} & \text{Y} & -2 & 0 & 1.0000000 & 0.473934 & 0.526066 & -22 & 895 & -917 \\
 212 & 2^2 53^1 & \text{N} & \text{N} & -7 & 2 & 1.2857143 & 0.471698 & 0.528302 & -29 & 895 & -924 \\
 213 & 3^1 71^1 & \text{Y} & \text{N} & 5 & 0 & 1.0000000 & 0.474178 & 0.525822 & -24 & 900 & -924 \\
 214 & 2^1 107^1 & \text{Y} & \text{N} & 5 & 0 & 1.0000000 & 0.476636 & 0.523364 & -19 & 905 & -924 \\
 215 & 5^1 43^1 & \text{Y} & \text{N} & 5 & 0 & 1.0000000 & 0.479070 & 0.520930 & -14 & 910 & -924 \\
 216 & 2^3 3^3 & \text{N} & \text{N} & 46 & 41 & 1.5000000 & 0.481481 & 0.518519 & 32 & 956 & -924 \\
 217 & 7^1 31^1 & \text{Y} & \text{N} & 5 & 0 & 1.0000000 & 0.483871 & 0.516129 & 37 & 961 & -924 \\
 218 & 2^1 109^1 & \text{Y} & \text{N} & 5 & 0 & 1.0000000 & 0.486239 & 0.513761 & 42 & 966 & -924 \\
 219 & 3^1 73^1 & \text{Y} & \text{N} & 5 & 0 & 1.0000000 & 0.488584 & 0.511416 & 47 & 971 & -924 \\
 220 & 2^2 5^1 11^1 & \text{N} & \text{N} & 30 & 14 & 1.1666667 & 0.490909 & 0.509091 & 77 & 1001 & -924 \\
 221 & 13^1 17^1 & \text{Y} & \text{N} & 5 & 0 & 1.0000000 & 0.493213 & 0.506787 & 82 & 1006 & -924 \\
 222 & 2^1 3^1 37^1 & \text{Y} & \text{N} & -16 & 0 & 1.0000000 & 0.490991 & 0.509009 & 66 & 1006 & -940 \\
 223 & 223^1 & \text{Y} & \text{Y} & -2 & 0 & 1.0000000 & 0.488789 & 0.511211 & 64 & 1006 & -942 \\
 224 & 2^5 7^1 & \text{N} & \text{N} & 13 & 8 & 2.0769231 & 0.491071 & 0.508929 & 77 & 1019 & -942 \\
 225 & 3^2 5^2 & \text{N} & \text{N} & 14 & 9 & 1.3571429 & 0.493333 & 0.506667 & 91 & 1033 & -942 \\
 226 & 2^1 113^1 & \text{Y} & \text{N} & 5 & 0 & 1.0000000 & 0.495575 & 0.504425 & 96 & 1038 & -942 \\
 227 & 227^1 & \text{Y} & \text{Y} & -2 & 0 & 1.0000000 & 0.493392 & 0.506608 & 94 & 1038 & -944 \\
 228 & 2^2 3^1 19^1 & \text{N} & \text{N} & 30 & 14 & 1.1666667 & 0.495614 & 0.504386 & 124 & 1068 & -944 \\
 229 & 229^1 & \text{Y} & \text{Y} & -2 & 0 & 1.0000000 & 0.493450 & 0.506550 & 122 & 1068 & -946 \\
 230 & 2^1 5^1 23^1 & \text{Y} & \text{N} & -16 & 0 & 1.0000000 & 0.491304 & 0.508696 & 106 & 1068 & -962 \\
 231 & 3^1 7^1 11^1 & \text{Y} & \text{N} & -16 & 0 & 1.0000000 & 0.489177 & 0.510823 & 90 & 1068 & -978 \\
 232 & 2^3 29^1 & \text{N} & \text{N} & 9 & 4 & 1.5555556 & 0.491379 & 0.508621 & 99 & 1077 & -978 \\
 233 & 233^1 & \text{Y} & \text{Y} & -2 & 0 & 1.0000000 & 0.489270 & 0.510730 & 97 & 1077 & -980 \\
 234 & 2^1 3^2 13^1 & \text{N} & \text{N} & 30 & 14 & 1.1666667 & 0.491453 & 0.508547 & 127 & 1107 & -980 \\
 235 & 5^1 47^1 & \text{Y} & \text{N} & 5 & 0 & 1.0000000 & 0.493617 & 0.506383 & 132 & 1112 & -980 \\
 236 & 2^2 59^1 & \text{N} & \text{N} & -7 & 2 & 1.2857143 & 0.491525 & 0.508475 & 125 & 1112 & -987 \\
 237 & 3^1 79^1 & \text{Y} & \text{N} & 5 & 0 & 1.0000000 & 0.493671 & 0.506329 & 130 & 1117 & -987 \\
 238 & 2^1 7^1 17^1 & \text{Y} & \text{N} & -16 & 0 & 1.0000000 & 0.491597 & 0.508403 & 114 & 1117 & -1003 \\
 239 & 239^1 & \text{Y} & \text{Y} & -2 & 0 & 1.0000000 & 0.489540 & 0.510460 & 112 & 1117 & -1005 \\
 240 & 2^4 3^1 5^1 & \text{N} & \text{N} & 70 & 54 & 1.5000000 & 0.491667 & 0.508333 & 182 & 1187 & -1005 \\
 241 & 241^1 & \text{Y} & \text{Y} & -2 & 0 & 1.0000000 & 0.489627 & 0.510373 & 180 & 1187 & -1007 \\
 242 & 2^1 11^2 & \text{N} & \text{N} & -7 & 2 & 1.2857143 & 0.487603 & 0.512397 & 173 & 1187 & -1014 \\
 243 & 3^5 & \text{N} & \text{Y} & -2 & 0 & 3.0000000 & 0.485597 & 0.514403 & 171 & 1187 & -1016 \\
 244 & 2^2 61^1 & \text{N} & \text{N} & -7 & 2 & 1.2857143 & 0.483607 & 0.516393 & 164 & 1187 & -1023 \\
 245 & 5^1 7^2 & \text{N} & \text{N} & -7 & 2 & 1.2857143 & 0.481633 & 0.518367 & 157 & 1187 & -1030 \\
 246 & 2^1 3^1 41^1 & \text{Y} & \text{N} & -16 & 0 & 1.0000000 & 0.479675 & 0.520325 & 141 & 1187 & -1046 \\
 247 & 13^1 19^1 & \text{Y} & \text{N} & 5 & 0 & 1.0000000 & 0.481781 & 0.518219 & 146 & 1192 & -1046 \\
 248 & 2^3 31^1 & \text{N} & \text{N} & 9 & 4 & 1.5555556 & 0.483871 & 0.516129 & 155 & 1201 & -1046 \\
 249 & 3^1 83^1 & \text{Y} & \text{N} & 5 & 0 & 1.0000000 & 0.485944 & 0.514056 & 160 & 1206 & -1046 \\
 250 & 2^1 5^3 & \text{N} & \text{N} & 9 & 4 & 1.5555556 & 0.488000 & 0.512000 & 169 & 1215 & -1046 \\
 251 & 251^1 & \text{Y} & \text{Y} & -2 & 0 & 1.0000000 & 0.486056 & 0.513944 & 167 & 1215 & -1048 \\
 252 & 2^2 3^2 7^1 & \text{N} & \text{N} & -74 & 58 & 1.2162162 & 0.484127 & 0.515873 & 93 & 1215 & -1122 \\
 253 & 11^1 23^1 & \text{Y} & \text{N} & 5 & 0 & 1.0000000 & 0.486166 & 0.513834 & 98 & 1220 & -1122 \\
 254 & 2^1 127^1 & \text{Y} & \text{N} & 5 & 0 & 1.0000000 & 0.488189 & 0.511811 & 103 & 1225 & -1122 \\
 255 & 3^1 5^1 17^1 & \text{Y} & \text{N} & -16 & 0 & 1.0000000 & 0.486275 & 0.513725 & 87 & 1225 & -1138 \\
 256 & 2^8 & \text{N} & \text{Y} & 2 & 0 & 4.5000000 & 0.488281 & 0.511719 & 89 & 1227 & -1138 \\
 257 & 257^1 & \text{Y} & \text{Y} & -2 & 0 & 1.0000000 & 0.486381 & 0.513619 & 87 & 1227 & -1140 \\
 258 & 2^1 3^1 43^1 & \text{Y} & \text{N} & -16 & 0 & 1.0000000 & 0.484496 & 0.515504 & 71 & 1227 & -1156 \\
 259 & 7^1 37^1 & \text{Y} & \text{N} & 5 & 0 & 1.0000000 & 0.486486 & 0.513514 & 76 & 1232 & -1156 \\
 260 & 2^2 5^1 13^1 & \text{N} & \text{N} & 30 & 14 & 1.1666667 & 0.488462 & 0.511538 & 106 & 1262 & -1156 \\
 261 & 3^2 29^1 & \text{N} & \text{N} & -7 & 2 & 1.2857143 & 0.486590 & 0.513410 & 99 & 1262 & -1163 \\
 262 & 2^1 131^1 & \text{Y} & \text{N} & 5 & 0 & 1.0000000 & 0.488550 & 0.511450 & 104 & 1267 & -1163 \\
 263 & 263^1 & \text{Y} & \text{Y} & -2 & 0 & 1.0000000 & 0.486692 & 0.513308 & 102 & 1267 & -1165 \\
 264 & 2^3 3^1 11^1 & \text{N} & \text{N} & -48 & 32 & 1.3333333 & 0.484848 & 0.515152 & 54 & 1267 & -1213 \\
 265 & 5^1 53^1 & \text{Y} & \text{N} & 5 & 0 & 1.0000000 & 0.486792 & 0.513208 & 59 & 1272 & -1213 \\
 266 & 2^1 7^1 19^1 & \text{Y} & \text{N} & -16 & 0 & 1.0000000 & 0.484962 & 0.515038 & 43 & 1272 & -1229 \\
 267 & 3^1 89^1 & \text{Y} & \text{N} & 5 & 0 & 1.0000000 & 0.486891 & 0.513109 & 48 & 1277 & -1229 \\
 268 & 2^2 67^1 & \text{N} & \text{N} & -7 & 2 & 1.2857143 & 0.485075 & 0.514925 & 41 & 1277 & -1236 \\
 269 & 269^1 & \text{Y} & \text{Y} & -2 & 0 & 1.0000000 & 0.483271 & 0.516729 & 39 & 1277 & -1238 \\
 270 & 2^1 3^3 5^1 & \text{N} & \text{N} & -48 & 32 & 1.3333333 & 0.481481 & 0.518519 & -9 & 1277 & -1286 \\
 271 & 271^1 & \text{Y} & \text{Y} & -2 & 0 & 1.0000000 & 0.479705 & 0.520295 & -11 & 1277 & -1288 \\
 272 & 2^4 17^1 & \text{N} & \text{N} & -11 & 6 & 1.8181818 & 0.477941 & 0.522059 & -22 & 1277 & -1299 \\
 273 & 3^1 7^1 13^1 & \text{Y} & \text{N} & -16 & 0 & 1.0000000 & 0.476190 & 0.523810 & -38 & 1277 & -1315 \\
 274 & 2^1 137^1 & \text{Y} & \text{N} & 5 & 0 & 1.0000000 & 0.478102 & 0.521898 & -33 & 1282 & -1315 \\
 275 & 5^2 11^1 & \text{N} & \text{N} & -7 & 2 & 1.2857143 & 0.476364 & 0.523636 & -40 & 1282 & -1322 \\
 276 & 2^2 3^1 23^1 & \text{N} & \text{N} & 30 & 14 & 1.1666667 & 0.478261 & 0.521739 & -10 & 1312 & -1322 \\
 277 & 277^1 & \text{Y} & \text{Y} & -2 & 0 & 1.0000000 & 0.476534 & 0.523466 & -12 & 1312 & -1324 \\ 
\end{array}
}
\end{equation*}
\clearpage 

\end{table} 

\newpage
\begin{table}[ht]

\centering

\tiny
\begin{equation*}
\boxed{
\begin{array}{cc|cc|ccc|cc|ccc}
 n & \mathbf{Primes} & \mathbf{Sqfree} & \mathbf{PPower} & g^{-1}(n) & 
 \lambda(n) g^{-1}(n) - \widehat{f}_1(n) & 
 \frac{\sum_{d|n} C_{\Omega(d)}(d)}{|g^{-1}(n)|} & 
 \mathcal{L}_{+}(n) & \mathcal{L}_{-}(n) & 
 G^{-1}(n) & G^{-1}_{+}(n) & G^{-1}_{-}(n) \\ \hline 
 278 & 2^1 139^1 & \text{Y} & \text{N} & 5 & 0 & 1.0000000 & 0.478417 & 0.521583 & -7 & 1317 & -1324 \\
 279 & 3^2 31^1 & \text{N} & \text{N} & -7 & 2 & 1.2857143 & 0.476703 & 0.523297 & -14 & 1317 & -1331 \\
 280 & 2^3 5^1 7^1 & \text{N} & \text{N} & -48 & 32 & 1.3333333 & 0.475000 & 0.525000 & -62 & 1317 & -1379 \\
 281 & 281^1 & \text{Y} & \text{Y} & -2 & 0 & 1.0000000 & 0.473310 & 0.526690 & -64 & 1317 & -1381 \\
 282 & 2^1 3^1 47^1 & \text{Y} & \text{N} & -16 & 0 & 1.0000000 & 0.471631 & 0.528369 & -80 & 1317 & -1397 \\
 283 & 283^1 & \text{Y} & \text{Y} & -2 & 0 & 1.0000000 & 0.469965 & 0.530035 & -82 & 1317 & -1399 \\
 284 & 2^2 71^1 & \text{N} & \text{N} & -7 & 2 & 1.2857143 & 0.468310 & 0.531690 & -89 & 1317 & -1406 \\
 285 & 3^1 5^1 19^1 & \text{Y} & \text{N} & -16 & 0 & 1.0000000 & 0.466667 & 0.533333 & -105 & 1317 & -1422 \\
 286 & 2^1 11^1 13^1 & \text{Y} & \text{N} & -16 & 0 & 1.0000000 & 0.465035 & 0.534965 & -121 & 1317 & -1438 \\
 287 & 7^1 41^1 & \text{Y} & \text{N} & 5 & 0 & 1.0000000 & 0.466899 & 0.533101 & -116 & 1322 & -1438 \\
 288 & 2^5 3^2 & \text{N} & \text{N} & -47 & 42 & 1.7659574 & 0.465278 & 0.534722 & -163 & 1322 & -1485 \\
 289 & 17^2 & \text{N} & \text{Y} & 2 & 0 & 1.5000000 & 0.467128 & 0.532872 & -161 & 1324 & -1485 \\
 290 & 2^1 5^1 29^1 & \text{Y} & \text{N} & -16 & 0 & 1.0000000 & 0.465517 & 0.534483 & -177 & 1324 & -1501 \\
 291 & 3^1 97^1 & \text{Y} & \text{N} & 5 & 0 & 1.0000000 & 0.467354 & 0.532646 & -172 & 1329 & -1501 \\
 292 & 2^2 73^1 & \text{N} & \text{N} & -7 & 2 & 1.2857143 & 0.465753 & 0.534247 & -179 & 1329 & -1508 \\
 293 & 293^1 & \text{Y} & \text{Y} & -2 & 0 & 1.0000000 & 0.464164 & 0.535836 & -181 & 1329 & -1510 \\
 294 & 2^1 3^1 7^2 & \text{N} & \text{N} & 30 & 14 & 1.1666667 & 0.465986 & 0.534014 & -151 & 1359 & -1510 \\
 295 & 5^1 59^1 & \text{Y} & \text{N} & 5 & 0 & 1.0000000 & 0.467797 & 0.532203 & -146 & 1364 & -1510 \\
 296 & 2^3 37^1 & \text{N} & \text{N} & 9 & 4 & 1.5555556 & 0.469595 & 0.530405 & -137 & 1373 & -1510 \\
 297 & 3^3 11^1 & \text{N} & \text{N} & 9 & 4 & 1.5555556 & 0.471380 & 0.528620 & -128 & 1382 & -1510 \\
 298 & 2^1 149^1 & \text{Y} & \text{N} & 5 & 0 & 1.0000000 & 0.473154 & 0.526846 & -123 & 1387 & -1510 \\
 299 & 13^1 23^1 & \text{Y} & \text{N} & 5 & 0 & 1.0000000 & 0.474916 & 0.525084 & -118 & 1392 & -1510 \\
 300 & 2^2 3^1 5^2 & \text{N} & \text{N} & -74 & 58 & 1.2162162 & 0.473333 & 0.526667 & -192 & 1392 & -1584 \\
 301 & 7^1 43^1 & \text{Y} & \text{N} & 5 & 0 & 1.0000000 & 0.475083 & 0.524917 & -187 & 1397 & -1584 \\
 302 & 2^1 151^1 & \text{Y} & \text{N} & 5 & 0 & 1.0000000 & 0.476821 & 0.523179 & -182 & 1402 & -1584 \\
 303 & 3^1 101^1 & \text{Y} & \text{N} & 5 & 0 & 1.0000000 & 0.478548 & 0.521452 & -177 & 1407 & -1584 \\
 304 & 2^4 19^1 & \text{N} & \text{N} & -11 & 6 & 1.8181818 & 0.476974 & 0.523026 & -188 & 1407 & -1595 \\
 305 & 5^1 61^1 & \text{Y} & \text{N} & 5 & 0 & 1.0000000 & 0.478689 & 0.521311 & -183 & 1412 & -1595 \\
 306 & 2^1 3^2 17^1 & \text{N} & \text{N} & 30 & 14 & 1.1666667 & 0.480392 & 0.519608 & -153 & 1442 & -1595 \\
 307 & 307^1 & \text{Y} & \text{Y} & -2 & 0 & 1.0000000 & 0.478827 & 0.521173 & -155 & 1442 & -1597 \\
 308 & 2^2 7^1 11^1 & \text{N} & \text{N} & 30 & 14 & 1.1666667 & 0.480519 & 0.519481 & -125 & 1472 & -1597 \\
 309 & 3^1 103^1 & \text{Y} & \text{N} & 5 & 0 & 1.0000000 & 0.482201 & 0.517799 & -120 & 1477 & -1597 \\
 310 & 2^1 5^1 31^1 & \text{Y} & \text{N} & -16 & 0 & 1.0000000 & 0.480645 & 0.519355 & -136 & 1477 & -1613 \\
 311 & 311^1 & \text{Y} & \text{Y} & -2 & 0 & 1.0000000 & 0.479100 & 0.520900 & -138 & 1477 & -1615 \\
 312 & 2^3 3^1 13^1 & \text{N} & \text{N} & -48 & 32 & 1.3333333 & 0.477564 & 0.522436 & -186 & 1477 & -1663 \\
 313 & 313^1 & \text{Y} & \text{Y} & -2 & 0 & 1.0000000 & 0.476038 & 0.523962 & -188 & 1477 & -1665 \\
 314 & 2^1 157^1 & \text{Y} & \text{N} & 5 & 0 & 1.0000000 & 0.477707 & 0.522293 & -183 & 1482 & -1665 \\
 315 & 3^2 5^1 7^1 & \text{N} & \text{N} & 30 & 14 & 1.1666667 & 0.479365 & 0.520635 & -153 & 1512 & -1665 \\
 316 & 2^2 79^1 & \text{N} & \text{N} & -7 & 2 & 1.2857143 & 0.477848 & 0.522152 & -160 & 1512 & -1672 \\
 317 & 317^1 & \text{Y} & \text{Y} & -2 & 0 & 1.0000000 & 0.476341 & 0.523659 & -162 & 1512 & -1674 \\
 318 & 2^1 3^1 53^1 & \text{Y} & \text{N} & -16 & 0 & 1.0000000 & 0.474843 & 0.525157 & -178 & 1512 & -1690 \\
 319 & 11^1 29^1 & \text{Y} & \text{N} & 5 & 0 & 1.0000000 & 0.476489 & 0.523511 & -173 & 1517 & -1690 \\
 320 & 2^6 5^1 & \text{N} & \text{N} & -15 & 10 & 2.3333333 & 0.475000 & 0.525000 & -188 & 1517 & -1705 \\
 321 & 3^1 107^1 & \text{Y} & \text{N} & 5 & 0 & 1.0000000 & 0.476636 & 0.523364 & -183 & 1522 & -1705 \\
 322 & 2^1 7^1 23^1 & \text{Y} & \text{N} & -16 & 0 & 1.0000000 & 0.475155 & 0.524845 & -199 & 1522 & -1721 \\
 323 & 17^1 19^1 & \text{Y} & \text{N} & 5 & 0 & 1.0000000 & 0.476780 & 0.523220 & -194 & 1527 & -1721 \\
 324 & 2^2 3^4 & \text{N} & \text{N} & 34 & 29 & 1.6176471 & 0.478395 & 0.521605 & -160 & 1561 & -1721 \\
 325 & 5^2 13^1 & \text{N} & \text{N} & -7 & 2 & 1.2857143 & 0.476923 & 0.523077 & -167 & 1561 & -1728 \\
 326 & 2^1 163^1 & \text{Y} & \text{N} & 5 & 0 & 1.0000000 & 0.478528 & 0.521472 & -162 & 1566 & -1728 \\
 327 & 3^1 109^1 & \text{Y} & \text{N} & 5 & 0 & 1.0000000 & 0.480122 & 0.519878 & -157 & 1571 & -1728 \\
 328 & 2^3 41^1 & \text{N} & \text{N} & 9 & 4 & 1.5555556 & 0.481707 & 0.518293 & -148 & 1580 & -1728 \\
 329 & 7^1 47^1 & \text{Y} & \text{N} & 5 & 0 & 1.0000000 & 0.483283 & 0.516717 & -143 & 1585 & -1728 \\
 330 & 2^1 3^1 5^1 11^1 & \text{Y} & \text{N} & 65 & 0 & 1.0000000 & 0.484848 & 0.515152 & -78 & 1650 & -1728 \\
 331 & 331^1 & \text{Y} & \text{Y} & -2 & 0 & 1.0000000 & 0.483384 & 0.516616 & -80 & 1650 & -1730 \\
 332 & 2^2 83^1 & \text{N} & \text{N} & -7 & 2 & 1.2857143 & 0.481928 & 0.518072 & -87 & 1650 & -1737 \\
 333 & 3^2 37^1 & \text{N} & \text{N} & -7 & 2 & 1.2857143 & 0.480480 & 0.519520 & -94 & 1650 & -1744 \\
 334 & 2^1 167^1 & \text{Y} & \text{N} & 5 & 0 & 1.0000000 & 0.482036 & 0.517964 & -89 & 1655 & -1744 \\
 335 & 5^1 67^1 & \text{Y} & \text{N} & 5 & 0 & 1.0000000 & 0.483582 & 0.516418 & -84 & 1660 & -1744 \\
 336 & 2^4 3^1 7^1 & \text{N} & \text{N} & 70 & 54 & 1.5000000 & 0.485119 & 0.514881 & -14 & 1730 & -1744 \\
 337 & 337^1 & \text{Y} & \text{Y} & -2 & 0 & 1.0000000 & 0.483680 & 0.516320 & -16 & 1730 & -1746 \\
 338 & 2^1 13^2 & \text{N} & \text{N} & -7 & 2 & 1.2857143 & 0.482249 & 0.517751 & -23 & 1730 & -1753 \\
 339 & 3^1 113^1 & \text{Y} & \text{N} & 5 & 0 & 1.0000000 & 0.483776 & 0.516224 & -18 & 1735 & -1753 \\
 340 & 2^2 5^1 17^1 & \text{N} & \text{N} & 30 & 14 & 1.1666667 & 0.485294 & 0.514706 & 12 & 1765 & -1753 \\
 341 & 11^1 31^1 & \text{Y} & \text{N} & 5 & 0 & 1.0000000 & 0.486804 & 0.513196 & 17 & 1770 & -1753 \\
 342 & 2^1 3^2 19^1 & \text{N} & \text{N} & 30 & 14 & 1.1666667 & 0.488304 & 0.511696 & 47 & 1800 & -1753 \\
 343 & 7^3 & \text{N} & \text{Y} & -2 & 0 & 2.0000000 & 0.486880 & 0.513120 & 45 & 1800 & -1755 \\
 344 & 2^3 43^1 & \text{N} & \text{N} & 9 & 4 & 1.5555556 & 0.488372 & 0.511628 & 54 & 1809 & -1755 \\
 345 & 3^1 5^1 23^1 & \text{Y} & \text{N} & -16 & 0 & 1.0000000 & 0.486957 & 0.513043 & 38 & 1809 & -1771 \\
 346 & 2^1 173^1 & \text{Y} & \text{N} & 5 & 0 & 1.0000000 & 0.488439 & 0.511561 & 43 & 1814 & -1771 \\
 347 & 347^1 & \text{Y} & \text{Y} & -2 & 0 & 1.0000000 & 0.487032 & 0.512968 & 41 & 1814 & -1773 \\
 348 & 2^2 3^1 29^1 & \text{N} & \text{N} & 30 & 14 & 1.1666667 & 0.488506 & 0.511494 & 71 & 1844 & -1773 \\
 349 & 349^1 & \text{Y} & \text{Y} & -2 & 0 & 1.0000000 & 0.487106 & 0.512894 & 69 & 1844 & -1775 \\
 350 & 2^1 5^2 7^1 & \text{N} & \text{N} & 30 & 14 & 1.1666667 & 0.488571 & 0.511429 & 99 & 1874 & -1775 \\ 
\end{array}
}
\end{equation*}
\clearpage 

\end{table} 

\newpage
\begin{table}[ht]

\centering
\tiny
\begin{equation*}
\boxed{
\begin{array}{cc|cc|ccc|cc|ccc}
 n & \mathbf{Primes} & \mathbf{Sqfree} & \mathbf{PPower} & g^{-1}(n) & 
 \lambda(n) g^{-1}(n) - \widehat{f}_1(n) & 
 \frac{\sum_{d|n} C_{\Omega(d)}(d)}{|g^{-1}(n)|} & 
 \mathcal{L}_{+}(n) & \mathcal{L}_{-}(n) & 
 G^{-1}(n) & G^{-1}_{+}(n) & G^{-1}_{-}(n) \\ \hline 
 351 & 3^3 13^1 & \text{N} & \text{N} & 9 & 4 & 1.5555556 & 0.490028 & 0.509972 & 108 & 1883 & -1775 \\
 352 & 2^5 11^1 & \text{N} & \text{N} & 13 & 8 & 2.0769231 & 0.491477 & 0.508523 & 121 & 1896 & -1775 \\
 353 & 353^1 & \text{Y} & \text{Y} & -2 & 0 & 1.0000000 & 0.490085 & 0.509915 & 119 & 1896 & -1777 \\
 354 & 2^1 3^1 59^1 & \text{Y} & \text{N} & -16 & 0 & 1.0000000 & 0.488701 & 0.511299 & 103 & 1896 & -1793 \\
 355 & 5^1 71^1 & \text{Y} & \text{N} & 5 & 0 & 1.0000000 & 0.490141 & 0.509859 & 108 & 1901 & -1793 \\
 356 & 2^2 89^1 & \text{N} & \text{N} & -7 & 2 & 1.2857143 & 0.488764 & 0.511236 & 101 & 1901 & -1800 \\
 357 & 3^1 7^1 17^1 & \text{Y} & \text{N} & -16 & 0 & 1.0000000 & 0.487395 & 0.512605 & 85 & 1901 & -1816 \\
 358 & 2^1 179^1 & \text{Y} & \text{N} & 5 & 0 & 1.0000000 & 0.488827 & 0.511173 & 90 & 1906 & -1816 \\
 359 & 359^1 & \text{Y} & \text{Y} & -2 & 0 & 1.0000000 & 0.487465 & 0.512535 & 88 & 1906 & -1818 \\
 360 & 2^3 3^2 5^1 & \text{N} & \text{N} & 145 & 129 & 1.3034483 & 0.488889 & 0.511111 & 233 & 2051 & -1818 \\
 361 & 19^2 & \text{N} & \text{Y} & 2 & 0 & 1.5000000 & 0.490305 & 0.509695 & 235 & 2053 & -1818 \\
 362 & 2^1 181^1 & \text{Y} & \text{N} & 5 & 0 & 1.0000000 & 0.491713 & 0.508287 & 240 & 2058 & -1818 \\
 363 & 3^1 11^2 & \text{N} & \text{N} & -7 & 2 & 1.2857143 & 0.490358 & 0.509642 & 233 & 2058 & -1825 \\
 364 & 2^2 7^1 13^1 & \text{N} & \text{N} & 30 & 14 & 1.1666667 & 0.491758 & 0.508242 & 263 & 2088 & -1825 \\
 365 & 5^1 73^1 & \text{Y} & \text{N} & 5 & 0 & 1.0000000 & 0.493151 & 0.506849 & 268 & 2093 & -1825 \\
 366 & 2^1 3^1 61^1 & \text{Y} & \text{N} & -16 & 0 & 1.0000000 & 0.491803 & 0.508197 & 252 & 2093 & -1841 \\
 367 & 367^1 & \text{Y} & \text{Y} & -2 & 0 & 1.0000000 & 0.490463 & 0.509537 & 250 & 2093 & -1843 \\
 368 & 2^4 23^1 & \text{N} & \text{N} & -11 & 6 & 1.8181818 & 0.489130 & 0.510870 & 239 & 2093 & -1854 \\
 369 & 3^2 41^1 & \text{N} & \text{N} & -7 & 2 & 1.2857143 & 0.487805 & 0.512195 & 232 & 2093 & -1861 \\
 370 & 2^1 5^1 37^1 & \text{Y} & \text{N} & -16 & 0 & 1.0000000 & 0.486486 & 0.513514 & 216 & 2093 & -1877 \\
 371 & 7^1 53^1 & \text{Y} & \text{N} & 5 & 0 & 1.0000000 & 0.487871 & 0.512129 & 221 & 2098 & -1877 \\
 372 & 2^2 3^1 31^1 & \text{N} & \text{N} & 30 & 14 & 1.1666667 & 0.489247 & 0.510753 & 251 & 2128 & -1877 \\
 373 & 373^1 & \text{Y} & \text{Y} & -2 & 0 & 1.0000000 & 0.487936 & 0.512064 & 249 & 2128 & -1879 \\
 374 & 2^1 11^1 17^1 & \text{Y} & \text{N} & -16 & 0 & 1.0000000 & 0.486631 & 0.513369 & 233 & 2128 & -1895 \\
 375 & 3^1 5^3 & \text{N} & \text{N} & 9 & 4 & 1.5555556 & 0.488000 & 0.512000 & 242 & 2137 & -1895 \\
 376 & 2^3 47^1 & \text{N} & \text{N} & 9 & 4 & 1.5555556 & 0.489362 & 0.510638 & 251 & 2146 & -1895 \\
 377 & 13^1 29^1 & \text{Y} & \text{N} & 5 & 0 & 1.0000000 & 0.490716 & 0.509284 & 256 & 2151 & -1895 \\
 378 & 2^1 3^3 7^1 & \text{N} & \text{N} & -48 & 32 & 1.3333333 & 0.489418 & 0.510582 & 208 & 2151 & -1943 \\
 379 & 379^1 & \text{Y} & \text{Y} & -2 & 0 & 1.0000000 & 0.488127 & 0.511873 & 206 & 2151 & -1945 \\
 380 & 2^2 5^1 19^1 & \text{N} & \text{N} & 30 & 14 & 1.1666667 & 0.489474 & 0.510526 & 236 & 2181 & -1945 \\
 381 & 3^1 127^1 & \text{Y} & \text{N} & 5 & 0 & 1.0000000 & 0.490814 & 0.509186 & 241 & 2186 & -1945 \\
 382 & 2^1 191^1 & \text{Y} & \text{N} & 5 & 0 & 1.0000000 & 0.492147 & 0.507853 & 246 & 2191 & -1945 \\
 383 & 383^1 & \text{Y} & \text{Y} & -2 & 0 & 1.0000000 & 0.490862 & 0.509138 & 244 & 2191 & -1947 \\
 384 & 2^7 3^1 & \text{N} & \text{N} & 17 & 12 & 2.5882353 & 0.492188 & 0.507812 & 261 & 2208 & -1947 \\
 385 & 5^1 7^1 11^1 & \text{Y} & \text{N} & -16 & 0 & 1.0000000 & 0.490909 & 0.509091 & 245 & 2208 & -1963 \\
 386 & 2^1 193^1 & \text{Y} & \text{N} & 5 & 0 & 1.0000000 & 0.492228 & 0.507772 & 250 & 2213 & -1963 \\
 387 & 3^2 43^1 & \text{N} & \text{N} & -7 & 2 & 1.2857143 & 0.490956 & 0.509044 & 243 & 2213 & -1970 \\
 388 & 2^2 97^1 & \text{N} & \text{N} & -7 & 2 & 1.2857143 & 0.489691 & 0.510309 & 236 & 2213 & -1977 \\
 389 & 389^1 & \text{Y} & \text{Y} & -2 & 0 & 1.0000000 & 0.488432 & 0.511568 & 234 & 2213 & -1979 \\
 390 & 2^1 3^1 5^1 13^1 & \text{Y} & \text{N} & 65 & 0 & 1.0000000 & 0.489744 & 0.510256 & 299 & 2278 & -1979 \\
 391 & 17^1 23^1 & \text{Y} & \text{N} & 5 & 0 & 1.0000000 & 0.491049 & 0.508951 & 304 & 2283 & -1979 \\
 392 & 2^3 7^2 & \text{N} & \text{N} & -23 & 18 & 1.4782609 & 0.489796 & 0.510204 & 281 & 2283 & -2002 \\
 393 & 3^1 131^1 & \text{Y} & \text{N} & 5 & 0 & 1.0000000 & 0.491094 & 0.508906 & 286 & 2288 & -2002 \\
 394 & 2^1 197^1 & \text{Y} & \text{N} & 5 & 0 & 1.0000000 & 0.492386 & 0.507614 & 291 & 2293 & -2002 \\
 395 & 5^1 79^1 & \text{Y} & \text{N} & 5 & 0 & 1.0000000 & 0.493671 & 0.506329 & 296 & 2298 & -2002 \\
 396 & 2^2 3^2 11^1 & \text{N} & \text{N} & -74 & 58 & 1.2162162 & 0.492424 & 0.507576 & 222 & 2298 & -2076 \\
 397 & 397^1 & \text{Y} & \text{Y} & -2 & 0 & 1.0000000 & 0.491184 & 0.508816 & 220 & 2298 & -2078 \\
 398 & 2^1 199^1 & \text{Y} & \text{N} & 5 & 0 & 1.0000000 & 0.492462 & 0.507538 & 225 & 2303 & -2078 \\
 399 & 3^1 7^1 19^1 & \text{Y} & \text{N} & -16 & 0 & 1.0000000 & 0.491228 & 0.508772 & 209 & 2303 & -2094 \\
 400 & 2^4 5^2 & \text{N} & \text{N} & 34 & 29 & 1.6176471 & 0.492500 & 0.507500 & 243 & 2337 & -2094 \\
 401 & 401^1 & \text{Y} & \text{Y} & -2 & 0 & 1.0000000 & 0.491272 & 0.508728 & 241 & 2337 & -2096 \\
 402 & 2^1 3^1 67^1 & \text{Y} & \text{N} & -16 & 0 & 1.0000000 & 0.490050 & 0.509950 & 225 & 2337 & -2112 \\
 403 & 13^1 31^1 & \text{Y} & \text{N} & 5 & 0 & 1.0000000 & 0.491315 & 0.508685 & 230 & 2342 & -2112 \\
 404 & 2^2 101^1 & \text{N} & \text{N} & -7 & 2 & 1.2857143 & 0.490099 & 0.509901 & 223 & 2342 & -2119 \\
 405 & 3^4 5^1 & \text{N} & \text{N} & -11 & 6 & 1.8181818 & 0.488889 & 0.511111 & 212 & 2342 & -2130 \\
 406 & 2^1 7^1 29^1 & \text{Y} & \text{N} & -16 & 0 & 1.0000000 & 0.487685 & 0.512315 & 196 & 2342 & -2146 \\
 407 & 11^1 37^1 & \text{Y} & \text{N} & 5 & 0 & 1.0000000 & 0.488943 & 0.511057 & 201 & 2347 & -2146 \\
 408 & 2^3 3^1 17^1 & \text{N} & \text{N} & -48 & 32 & 1.3333333 & 0.487745 & 0.512255 & 153 & 2347 & -2194 \\
 409 & 409^1 & \text{Y} & \text{Y} & -2 & 0 & 1.0000000 & 0.486553 & 0.513447 & 151 & 2347 & -2196 \\
 410 & 2^1 5^1 41^1 & \text{Y} & \text{N} & -16 & 0 & 1.0000000 & 0.485366 & 0.514634 & 135 & 2347 & -2212 \\
 411 & 3^1 137^1 & \text{Y} & \text{N} & 5 & 0 & 1.0000000 & 0.486618 & 0.513382 & 140 & 2352 & -2212 \\
 412 & 2^2 103^1 & \text{N} & \text{N} & -7 & 2 & 1.2857143 & 0.485437 & 0.514563 & 133 & 2352 & -2219 \\
 413 & 7^1 59^1 & \text{Y} & \text{N} & 5 & 0 & 1.0000000 & 0.486683 & 0.513317 & 138 & 2357 & -2219 \\
 414 & 2^1 3^2 23^1 & \text{N} & \text{N} & 30 & 14 & 1.1666667 & 0.487923 & 0.512077 & 168 & 2387 & -2219 \\
 415 & 5^1 83^1 & \text{Y} & \text{N} & 5 & 0 & 1.0000000 & 0.489157 & 0.510843 & 173 & 2392 & -2219 \\
 416 & 2^5 13^1 & \text{N} & \text{N} & 13 & 8 & 2.0769231 & 0.490385 & 0.509615 & 186 & 2405 & -2219 \\
 417 & 3^1 139^1 & \text{Y} & \text{N} & 5 & 0 & 1.0000000 & 0.491607 & 0.508393 & 191 & 2410 & -2219 \\
 418 & 2^1 11^1 19^1 & \text{Y} & \text{N} & -16 & 0 & 1.0000000 & 0.490431 & 0.509569 & 175 & 2410 & -2235 \\
 419 & 419^1 & \text{Y} & \text{Y} & -2 & 0 & 1.0000000 & 0.489260 & 0.510740 & 173 & 2410 & -2237 \\
 420 & 2^2 3^1 5^1 7^1 & \text{N} & \text{N} & -155 & 90 & 1.1032258 & 0.488095 & 0.511905 & 18 & 2410 & -2392 \\
 421 & 421^1 & \text{Y} & \text{Y} & -2 & 0 & 1.0000000 & 0.486936 & 0.513064 & 16 & 2410 & -2394 \\
 422 & 2^1 211^1 & \text{Y} & \text{N} & 5 & 0 & 1.0000000 & 0.488152 & 0.511848 & 21 & 2415 & -2394 \\
 423 & 3^2 47^1 & \text{N} & \text{N} & -7 & 2 & 1.2857143 & 0.486998 & 0.513002 & 14 & 2415 & -2401 \\
 424 & 2^3 53^1 & \text{N} & \text{N} & 9 & 4 & 1.5555556 & 0.488208 & 0.511792 & 23 & 2424 & -2401 \\
 425 & 5^2 17^1 & \text{N} & \text{N} & -7 & 2 & 1.2857143 & 0.487059 & 0.512941 & 16 & 2424 & -2408 \\ 
\end{array}
}
\end{equation*}
\clearpage 

\end{table} 

\newpage

\begin{table}[ht]
\label{table_conjecture_Mertens_ginvSeq_approx_values_LastPage} 

\centering
\tiny
\begin{equation*}
\boxed{
\begin{array}{cc|cc|ccc|cc|ccc}
 n & \mathbf{Primes} & \mathbf{Sqfree} & \mathbf{PPower} & g^{-1}(n) & 
 \lambda(n) g^{-1}(n) - \widehat{f}_1(n) & 
 \frac{\sum_{d|n} C_{\Omega(d)}(d)}{|g^{-1}(n)|} & 
 \mathcal{L}_{+}(n) & \mathcal{L}_{-}(n) & 
 G^{-1}(n) & G^{-1}_{+}(n) & G^{-1}_{-}(n) \\ \hline 
 426 & 2^1 3^1 71^1 & \text{Y} & \text{N} & -16 & 0 & 1.0000000 & 0.485915 & 0.514085 & 0 & 2424 & -2424 \\
 427 & 7^1 61^1 & \text{Y} & \text{N} & 5 & 0 & 1.0000000 & 0.487119 & 0.512881 & 5 & 2429 & -2424 \\
 428 & 2^2 107^1 & \text{N} & \text{N} & -7 & 2 & 1.2857143 & 0.485981 & 0.514019 & -2 & 2429 & -2431 \\
 429 & 3^1 11^1 13^1 & \text{Y} & \text{N} & -16 & 0 & 1.0000000 & 0.484848 & 0.515152 & -18 & 2429 & -2447 \\
 430 & 2^1 5^1 43^1 & \text{Y} & \text{N} & -16 & 0 & 1.0000000 & 0.483721 & 0.516279 & -34 & 2429 & -2463 \\
 431 & 431^1 & \text{Y} & \text{Y} & -2 & 0 & 1.0000000 & 0.482599 & 0.517401 & -36 & 2429 & -2465 \\
 432 & 2^4 3^3 & \text{N} & \text{N} & -80 & 75 & 1.5625000 & 0.481481 & 0.518519 & -116 & 2429 & -2545 \\
 433 & 433^1 & \text{Y} & \text{Y} & -2 & 0 & 1.0000000 & 0.480370 & 0.519630 & -118 & 2429 & -2547 \\
 434 & 2^1 7^1 31^1 & \text{Y} & \text{N} & -16 & 0 & 1.0000000 & 0.479263 & 0.520737 & -134 & 2429 & -2563 \\
 435 & 3^1 5^1 29^1 & \text{Y} & \text{N} & -16 & 0 & 1.0000000 & 0.478161 & 0.521839 & -150 & 2429 & -2579 \\
 436 & 2^2 109^1 & \text{N} & \text{N} & -7 & 2 & 1.2857143 & 0.477064 & 0.522936 & -157 & 2429 & -2586 \\
 437 & 19^1 23^1 & \text{Y} & \text{N} & 5 & 0 & 1.0000000 & 0.478261 & 0.521739 & -152 & 2434 & -2586 \\
 438 & 2^1 3^1 73^1 & \text{Y} & \text{N} & -16 & 0 & 1.0000000 & 0.477169 & 0.522831 & -168 & 2434 & -2602 \\
 439 & 439^1 & \text{Y} & \text{Y} & -2 & 0 & 1.0000000 & 0.476082 & 0.523918 & -170 & 2434 & -2604 \\
 440 & 2^3 5^1 11^1 & \text{N} & \text{N} & -48 & 32 & 1.3333333 & 0.475000 & 0.525000 & -218 & 2434 & -2652 \\
 441 & 3^2 7^2 & \text{N} & \text{N} & 14 & 9 & 1.3571429 & 0.476190 & 0.523810 & -204 & 2448 & -2652 \\
 442 & 2^1 13^1 17^1 & \text{Y} & \text{N} & -16 & 0 & 1.0000000 & 0.475113 & 0.524887 & -220 & 2448 & -2668 \\
 443 & 443^1 & \text{Y} & \text{Y} & -2 & 0 & 1.0000000 & 0.474041 & 0.525959 & -222 & 2448 & -2670 \\
 444 & 2^2 3^1 37^1 & \text{N} & \text{N} & 30 & 14 & 1.1666667 & 0.475225 & 0.524775 & -192 & 2478 & -2670 \\
 445 & 5^1 89^1 & \text{Y} & \text{N} & 5 & 0 & 1.0000000 & 0.476404 & 0.523596 & -187 & 2483 & -2670 \\
 446 & 2^1 223^1 & \text{Y} & \text{N} & 5 & 0 & 1.0000000 & 0.477578 & 0.522422 & -182 & 2488 & -2670 \\
 447 & 3^1 149^1 & \text{Y} & \text{N} & 5 & 0 & 1.0000000 & 0.478747 & 0.521253 & -177 & 2493 & -2670 \\
 448 & 2^6 7^1 & \text{N} & \text{N} & -15 & 10 & 2.3333333 & 0.477679 & 0.522321 & -192 & 2493 & -2685 \\
 449 & 449^1 & \text{Y} & \text{Y} & -2 & 0 & 1.0000000 & 0.476615 & 0.523385 & -194 & 2493 & -2687 \\
 450 & 2^1 3^2 5^2 & \text{N} & \text{N} & -74 & 58 & 1.2162162 & 0.475556 & 0.524444 & -268 & 2493 & -2761 \\
 451 & 11^1 41^1 & \text{Y} & \text{N} & 5 & 0 & 1.0000000 & 0.476718 & 0.523282 & -263 & 2498 & -2761 \\
 452 & 2^2 113^1 & \text{N} & \text{N} & -7 & 2 & 1.2857143 & 0.475664 & 0.524336 & -270 & 2498 & -2768 \\
 453 & 3^1 151^1 & \text{Y} & \text{N} & 5 & 0 & 1.0000000 & 0.476821 & 0.523179 & -265 & 2503 & -2768 \\
 454 & 2^1 227^1 & \text{Y} & \text{N} & 5 & 0 & 1.0000000 & 0.477974 & 0.522026 & -260 & 2508 & -2768 \\
 455 & 5^1 7^1 13^1 & \text{Y} & \text{N} & -16 & 0 & 1.0000000 & 0.476923 & 0.523077 & -276 & 2508 & -2784 \\
 456 & 2^3 3^1 19^1 & \text{N} & \text{N} & -48 & 32 & 1.3333333 & 0.475877 & 0.524123 & -324 & 2508 & -2832 \\
 457 & 457^1 & \text{Y} & \text{Y} & -2 & 0 & 1.0000000 & 0.474836 & 0.525164 & -326 & 2508 & -2834 \\
 458 & 2^1 229^1 & \text{Y} & \text{N} & 5 & 0 & 1.0000000 & 0.475983 & 0.524017 & -321 & 2513 & -2834 \\
 459 & 3^3 17^1 & \text{N} & \text{N} & 9 & 4 & 1.5555556 & 0.477124 & 0.522876 & -312 & 2522 & -2834 \\
 460 & 2^2 5^1 23^1 & \text{N} & \text{N} & 30 & 14 & 1.1666667 & 0.478261 & 0.521739 & -282 & 2552 & -2834 \\
 461 & 461^1 & \text{Y} & \text{Y} & -2 & 0 & 1.0000000 & 0.477223 & 0.522777 & -284 & 2552 & -2836 \\
 462 & 2^1 3^1 7^1 11^1 & \text{Y} & \text{N} & 65 & 0 & 1.0000000 & 0.478355 & 0.521645 & -219 & 2617 & -2836 \\
 463 & 463^1 & \text{Y} & \text{Y} & -2 & 0 & 1.0000000 & 0.477322 & 0.522678 & -221 & 2617 & -2838 \\
 464 & 2^4 29^1 & \text{N} & \text{N} & -11 & 6 & 1.8181818 & 0.476293 & 0.523707 & -232 & 2617 & -2849 \\
 465 & 3^1 5^1 31^1 & \text{Y} & \text{N} & -16 & 0 & 1.0000000 & 0.475269 & 0.524731 & -248 & 2617 & -2865 \\
 466 & 2^1 233^1 & \text{Y} & \text{N} & 5 & 0 & 1.0000000 & 0.476395 & 0.523605 & -243 & 2622 & -2865 \\
 467 & 467^1 & \text{Y} & \text{Y} & -2 & 0 & 1.0000000 & 0.475375 & 0.524625 & -245 & 2622 & -2867 \\
 468 & 2^2 3^2 13^1 & \text{N} & \text{N} & -74 & 58 & 1.2162162 & 0.474359 & 0.525641 & -319 & 2622 & -2941 \\
 469 & 7^1 67^1 & \text{Y} & \text{N} & 5 & 0 & 1.0000000 & 0.475480 & 0.524520 & -314 & 2627 & -2941 \\
 470 & 2^1 5^1 47^1 & \text{Y} & \text{N} & -16 & 0 & 1.0000000 & 0.474468 & 0.525532 & -330 & 2627 & -2957 \\
 471 & 3^1 157^1 & \text{Y} & \text{N} & 5 & 0 & 1.0000000 & 0.475584 & 0.524416 & -325 & 2632 & -2957 \\
 472 & 2^3 59^1 & \text{N} & \text{N} & 9 & 4 & 1.5555556 & 0.476695 & 0.523305 & -316 & 2641 & -2957 \\
 473 & 11^1 43^1 & \text{Y} & \text{N} & 5 & 0 & 1.0000000 & 0.477801 & 0.522199 & -311 & 2646 & -2957 \\
 474 & 2^1 3^1 79^1 & \text{Y} & \text{N} & -16 & 0 & 1.0000000 & 0.476793 & 0.523207 & -327 & 2646 & -2973 \\
 475 & 5^2 19^1 & \text{N} & \text{N} & -7 & 2 & 1.2857143 & 0.475789 & 0.524211 & -334 & 2646 & -2980 \\
 476 & 2^2 7^1 17^1 & \text{N} & \text{N} & 30 & 14 & 1.1666667 & 0.476891 & 0.523109 & -304 & 2676 & -2980 \\
 477 & 3^2 53^1 & \text{N} & \text{N} & -7 & 2 & 1.2857143 & 0.475891 & 0.524109 & -311 & 2676 & -2987 \\
 478 & 2^1 239^1 & \text{Y} & \text{N} & 5 & 0 & 1.0000000 & 0.476987 & 0.523013 & -306 & 2681 & -2987 \\
 479 & 479^1 & \text{Y} & \text{Y} & -2 & 0 & 1.0000000 & 0.475992 & 0.524008 & -308 & 2681 & -2989 \\
 480 & 2^5 3^1 5^1 & \text{N} & \text{N} & -96 & 80 & 1.6666667 & 0.475000 & 0.525000 & -404 & 2681 & -3085 \\
 481 & 13^1 37^1 & \text{Y} & \text{N} & 5 & 0 & 1.0000000 & 0.476091 & 0.523909 & -399 & 2686 & -3085 \\
 482 & 2^1 241^1 & \text{Y} & \text{N} & 5 & 0 & 1.0000000 & 0.477178 & 0.522822 & -394 & 2691 & -3085 \\
 483 & 3^1 7^1 23^1 & \text{Y} & \text{N} & -16 & 0 & 1.0000000 & 0.476190 & 0.523810 & -410 & 2691 & -3101 \\
 484 & 2^2 11^2 & \text{N} & \text{N} & 14 & 9 & 1.3571429 & 0.477273 & 0.522727 & -396 & 2705 & -3101 \\
 485 & 5^1 97^1 & \text{Y} & \text{N} & 5 & 0 & 1.0000000 & 0.478351 & 0.521649 & -391 & 2710 & -3101 \\
 486 & 2^1 3^5 & \text{N} & \text{N} & 13 & 8 & 2.0769231 & 0.479424 & 0.520576 & -378 & 2723 & -3101 \\
 487 & 487^1 & \text{Y} & \text{Y} & -2 & 0 & 1.0000000 & 0.478439 & 0.521561 & -380 & 2723 & -3103 \\
 488 & 2^3 61^1 & \text{N} & \text{N} & 9 & 4 & 1.5555556 & 0.479508 & 0.520492 & -371 & 2732 & -3103 \\
 489 & 3^1 163^1 & \text{Y} & \text{N} & 5 & 0 & 1.0000000 & 0.480573 & 0.519427 & -366 & 2737 & -3103 \\
 490 & 2^1 5^1 7^2 & \text{N} & \text{N} & 30 & 14 & 1.1666667 & 0.481633 & 0.518367 & -336 & 2767 & -3103 \\
 491 & 491^1 & \text{Y} & \text{Y} & -2 & 0 & 1.0000000 & 0.480652 & 0.519348 & -338 & 2767 & -3105 \\
 492 & 2^2 3^1 41^1 & \text{N} & \text{N} & 30 & 14 & 1.1666667 & 0.481707 & 0.518293 & -308 & 2797 & -3105 \\
 493 & 17^1 29^1 & \text{Y} & \text{N} & 5 & 0 & 1.0000000 & 0.482759 & 0.517241 & -303 & 2802 & -3105 \\
 494 & 2^1 13^1 19^1 & \text{Y} & \text{N} & -16 & 0 & 1.0000000 & 0.481781 & 0.518219 & -319 & 2802 & -3121 \\
 495 & 3^2 5^1 11^1 & \text{N} & \text{N} & 30 & 14 & 1.1666667 & 0.482828 & 0.517172 & -289 & 2832 & -3121 \\
 496 & 2^4 31^1 & \text{N} & \text{N} & -11 & 6 & 1.8181818 & 0.481855 & 0.518145 & -300 & 2832 & -3132 \\
 497 & 7^1 71^1 & \text{Y} & \text{N} & 5 & 0 & 1.0000000 & 0.482897 & 0.517103 & -295 & 2837 & -3132 \\
 498 & 2^1 3^1 83^1 & \text{Y} & \text{N} & -16 & 0 & 1.0000000 & 0.481928 & 0.518072 & -311 & 2837 & -3148 \\
 499 & 499^1 & \text{Y} & \text{Y} & -2 & 0 & 1.0000000 & 0.480962 & 0.519038 & -313 & 2837 & -3150 \\
 500 & 2^2 5^3 & \text{N} & \text{N} & -23 & 18 & 1.4782609 & 0.480000 & 0.520000 & -336 & 2837 & -3173 \\  
\end{array}
}
\end{equation*}

\end{table} 

\clearpage 

%% working-topics-2020.09.13-v1.tex


\end{document}
