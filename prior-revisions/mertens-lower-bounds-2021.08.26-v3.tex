\documentclass[11pt,reqno,a4letter]{article} 

\usepackage{amsmath,amssymb,amsfonts,amscd}
\usepackage[hidelinks]{hyperref} 
\usepackage{url}
\usepackage[usenames,dvipsnames]{xcolor}
\hypersetup{
    colorlinks,
    linkcolor={black!63!darkgray},
    citecolor={blue!70!white},
    urlcolor={blue!80!white}
}

%\usepackage[normalem]{ulem}
\usepackage{graphicx} 
\usepackage{datetime} 
\usepackage{cancel}
\usepackage{subcaption}
\captionsetup{format=hang,labelfont={bf},textfont={small,it}} 
\numberwithin{figure}{section}
\numberwithin{table}{section}

\usepackage{framed} 
%\usepackage{ulem}
%\usepackage[T1]{fontenc}
%\usepackage{pbsi}

\usepackage{enumitem}
\setlist[itemize]{leftmargin=0.65in}

\usepackage{changepage}
\usepackage{rotating,adjustbox}

\usepackage{diagbox}
\newcommand{\trianglenk}[2]{$\diagbox{#1}{#2}$}
\newcommand{\trianglenkII}[2]{\diagbox{#1}{#2}}

\let\citep\cite

\newcommand{\undersetbrace}[2]{\underset{\displaystyle{#1}}{\underbrace{#2}}}

\newcommand{\gkpSI}[2]{\ensuremath{\genfrac{\lbrack}{\rbrack}{0pt}{}{#1}{#2}}} 
\newcommand{\gkpSII}[2]{\ensuremath{\genfrac{\lbrace}{\rbrace}{0pt}{}{#1}{#2}}}
\newcommand{\cf}{\textit{cf.\ }} 
\newcommand{\Iverson}[1]{\ensuremath{\left[#1\right]_{\delta}}} 
\newcommand{\floor}[1]{\left\lfloor #1 \right\rfloor} 
\newcommand{\ceiling}[1]{\left\lceil #1 \right\rceil} 
\newcommand{\e}[1]{e\left(#1\right)} 
\newcommand{\seqnum}[1]{\href{http://oeis.org/#1}{\color{ProcessBlue}{\underline{#1}}}}

\usepackage{upgreek,dsfont,amssymb}
\renewcommand{\chi}{\upchi}
\newcommand{\ChiFunc}[1]{\ensuremath{\chi_{\{#1\}}}}
\newcommand{\OneFunc}[1]{\ensuremath{\mathds{1}_{#1}}}

\usepackage{MnSymbol}
\newcommand{\gkpEII}[2]{\ensuremath{\genfrac{\llangle}{\rrangle}{0pt}{}{#1}{#2}}}

\usepackage{ifthen}
\newcommand{\Hn}[2]{
     \ifthenelse{\equal{#2}{1}}{H_{#1}}{H_{#1}^{\left(#2\right)}}
}

\newcommand{\Floor}[2]{\ensuremath{\left\lfloor \frac{#1}{#2} \right\rfloor}}
\newcommand{\Ceiling}[2]{\ensuremath{\left\lceil \frac{#1}{#2} \right\rceil}}

\DeclareMathOperator{\DGF}{DGF} 
\DeclareMathOperator{\ds}{ds} 
\DeclareMathOperator{\Id}{Id}
\DeclareMathOperator{\fg}{fg}
\DeclareMathOperator{\Div}{div}
\DeclareMathOperator{\rpp}{rpp}
\DeclareMathOperator{\logll}{\ell\ell}

\title{
       \LARGE{
       New characterizations of partial sums of the M\"obius function 
       } 
}
\author{{\Large Maxie Dion Schmidt} \\ 
        %{\normalsize \href{mailto:maxieds@gmail.com}{maxieds@gmail.com}} \\[0.1cm] 
        {\normalsize Georgia Institute of Technology} \\[0.025cm] 
        {\normalsize School of Mathematics} 
} 

%\date{\small\underline{Last Revised:} \today \ @\ \hhmmsstime{} \ -- \ Compiled with \LaTeX2e} 

\usepackage{amsthm} 

\theoremstyle{plain} 
\newtheorem{theorem}{Theorem}
\newtheorem{conjecture}[theorem]{Conjecture}
\newtheorem{claim}[theorem]{Claim}
\newtheorem{prop}[theorem]{Proposition}
\newtheorem{lemma}[theorem]{Lemma}
\newtheorem{cor}[theorem]{Corollary}
\numberwithin{theorem}{section}

\theoremstyle{definition} 
\newtheorem{example}[theorem]{Example}
\newtheorem{remark}[theorem]{Remark}
\newtheorem{definition}[theorem]{Definition}
\newtheorem{notation}[theorem]{Notation}
\newtheorem{question}[theorem]{Question}
\newtheorem{discussion}[theorem]{Discussion}
\newtheorem{facts}[theorem]{Facts}
\newtheorem{summary}[theorem]{Summary}
\newtheorem{heuristic}[theorem]{Heuristic}
\newtheorem{observation}[theorem]{Observation}
\newtheorem{ansatz}[theorem]{Ansatz}

\renewcommand{\arraystretch}{1.25} 
\usepackage[total={7in, 9.5in},tmargin=0.75in,headsep=8pt,footskip=30pt,footnotesep=0.5in]{geometry}

\usepackage{fancyhdr}
\pagestyle{empty}
\pagestyle{fancy}
\fancyhead[RO,RE]{Maxie Dion Schmidt -- \today} 
\fancyhead[LO,LE]{}
\fancyheadoffset{0.005\textwidth} 

\setlength{\parindent}{0em}
\setlength{\parskip}{2.5em} 

\renewcommand{\thefootnote}{\textbf{\arabic{footnote}}}

\renewcommand{\Re}{\operatorname{Re}}
\renewcommand{\Im}{\operatorname{Im}}

\usepackage{tikz}
\usetikzlibrary{shapes,arrows}

\usepackage{longtable}
\usepackage{arydshln} 
\usepackage[symbols,nogroupskip,nomain,automake=true,nonumberlist,section=section,toc]{glossaries-extra}
\usepackage{glossary-mcols}

\defglsdisplayfirst[main]{#1#4\protect\footnote{#2}}

%%%%%%%%%%%%

\providecommand{\glossarytoctitle}{\glossaryname}
\setlength{\glsdescwidth}{0.7\textwidth}

\newglossarystyle{glossstyleSymbol}{%
\renewenvironment{theglossary}%
 {\begin{longtable}{lp{\glsdescwidth}}}%
 {\end{longtable}}%
 \setlength{\parskip}{3.5pt}
 \renewcommand{\glsgroupskip}{}
\renewcommand*\glspostdescription{\dotfill}
\renewcommand*{\glossaryheader}{%
 \bfseries Symbols & \bfseries Definition
 \\\endhead}%
 \renewcommand*{\glsgroupheading}[1]{}%
  \renewcommand{\glossentry}[2]{%
    \glstarget{##1}{\glossentrysymbol{##1}} &
    \glossentrydesc{##1} \tabularnewline
  }%
  \renewcommand*{\glspostdescription}{}
  \renewcommand{\glossarymark}[1]{}
}

\setglossarystyle{glossstyleSymbol}
\makeglossaries

%%%%%%%%%%%%

\newglossaryentry{fCvlg}{
    symbol={\ensuremath{f \ast g}},
    sort={fg},
    description={The Dirichlet convolution of any two arithmetic functions 
    $f$ and $g$ is denoted by  
    the divisor sum $(f \ast g)(n) := \sum\limits_{d|n} f(d) g\left(\frac{n}{d}\right)$ 
    for $n \geq 1$. 
    },
    type={symbols},
    name={Dirichlet convolution}
    }
\newglossaryentry{coeffExtraction}{
    symbol={\ensuremath{[q^n] F(q)}},
    sort={coeffExtraction},
    description={The coefficient of $q^n$ in the power series expansion of $F(q)$ about zero when 
    $F(q)$ is treated as the ordinary generating function (OGF) of some sequence, $\{f_n\}_{n \geq 0}$. 
    Namely, for integers $n \geq 0$ we define $[q^n] F(q) = f_n$ whenever 
    $F(q) := \sum\limits_{n \geq 0} f_n q^n$. },
    type={symbols},
    name={Series coefficient extraction}
    }
\newglossaryentry{MoebiusMuFunc}{
    symbol={\ensuremath{\mu(n),M(x)}},
    sort={MoebiusMuFunc},
    description={The M\"obius function defined such that $\mu^2(n)$ is the indicator function of the 
                 squarefree integers $n \geq 1$ where 
                 $\mu(n) = (-1)^{\omega(n)}$ whenever $n$ is squarefree. 
                 The Mertens function is the summatory function defined for all integers 
                 $x \geq 1$ by $M(x) := \sum\limits_{n \leq x} \mu(n)$.
                 },
    type={symbols},
    name={M\"obius function}
    }
\newglossaryentry{Iverson}{
    symbol={\ensuremath{\Iverson{n=k}},\ensuremath{\Iverson{\mathtt{cond}}}},
    sort={Iverson},
    description={The symbol $\Iverson{n=k}$ is a synonym for $\delta_{n,k}$ 
                 which is one if and only if $n = k$, and is zero otherwise. 
                 For boolean-valued conditions, \texttt{cond}, the symbol $\Iverson{\mathtt{cond}}$ 
                 evaluates to one precisely when \texttt{cond} is true, and to zero otherwise.},
    type={symbols},
    name={Iverson's convention}
    }
\newglossaryentry{epsilonN}{
    symbol={\ensuremath{\varepsilon(n)}},
    sort={epsilonN},
    description={The multiplicative identity with respect to Dirichlet convolution, $\varepsilon(n) := \delta_{n,1}$, 
                 defined such that for any arithmetic function $f$ we have that 
                 $f \ast \varepsilon = \varepsilon \ast f = f$ where the operation 
                 $\ast$ denotes Dirichlet convolution 
                 (see definition below).},
    type={symbols},
    name={Dirichlet multiplicative identity}
    }
\newglossaryentry{Zetas}{
    symbol={\ensuremath{\zeta(s)}},
    sort={Zetas},
    description={The Riemann zeta function is defined by 
                 $\zeta(s) := \sum\limits_{n \geq 1} \frac{1}{n^{s}}$ when $\Re(s) > 1$, 
                 and by analytic continuation to any $s \in \mathbb{C}$ with the exception of a 
                 simple pole at $s = 1$ of residue one.},
    type={symbols},
    name={Riemann zeta function}
    }
\newglossaryentry{fInvn}{
     symbol={\ensuremath{f^{-1}(n)}},
    sort={fInvn},
    description={
     The Dirichlet inverse $f^{-1}$ of an arithmetic function $f$ exists 
     if and only if $f(1) \neq 0$. 
     The Dirichlet inverse of any $f$ such that $f(1) \neq 0$ 
     is defined recursively by 
     $f^{-1}(n) = -\frac{1}{f(1)} \times \sum\limits_{\substack{d|n \\ d>1}} f(d) f^{-1}\left(\frac{n}{d}\right)$ 
     for $n \geq 2$ with $f^{-1}(1) = f(1)^{-1}$. 
     When it exists, this inverse function 
     is unique and satisfies  $f^{-1} \ast f = f \ast f^{-1} = \varepsilon$.},
    type={symbols},
    name={Dirichlet inverse of $f$}
    }
\newglossaryentry{CkngInvAuxFunc}{
    symbol={$C_k(n),C_{\Omega(n)}(n)$},
    sort={CkngInvAuxFunc},
    description={The sequence is defined recursively for integers $n \geq 1$ and $k \geq 0$ as follows: 
                 \[
                 C_k(n) := \begin{cases} 
                      \delta_{n,1}, & \text{ if $k = 0$; } \\ 
                      \sum\limits_{d|n} \omega(d) C_{k-1}\left(\frac{n}{d}\right), & \text{ if $k \geq 1$. } 
                      \end{cases} 
                 \]
                 It represents the multiple ($k$-fold) convolution of the function $\omega(n)$ 
                 with itself. 
                 The function $C_{\Omega(n)}(n)$ has the DGF $(1-P(s))^{-1}$ for $\Re(s) > 1$. 
                 },
    type={symbols},
    name={Dirichlet inverse component functions}
    }
\newglossaryentry{gInvn}{
    symbol={$g^{-1}(n),G^{-1}(x)$},
    sort={gInvn},
    description={The Dirichlet inverse function, $g^{-1}(n) = (\omega+1)^{-1}(n)$ with corresponding 
                 summatory function $G^{-1}(x) := \sum\limits_{n \leq x} g^{-1}(n)$ for $x \geq 1$. },
    type={symbols},
    name={Key Dirichlet inverse functions}
    }
\newglossaryentry{PikPiHatkx}{
    symbol={$\pi_k(x),\widehat{\pi}_k(x)$},
    sort={PikPiHatkx},
    description={For integers $k \geq 1$, the 
                 function $\pi_k(x)$ denotes the number of 
                 $2 \leq n \leq x$ with 
                 exactly $k$ distinct prime factors: $\pi_k(x) := \#\{2 \leq n \leq x: \omega(n) = k\}$. 
                 Similarly, the function 
                 $\widehat{\pi}_k(x) := \#\{2 \leq n \leq x: \Omega(n) = k\}$ for $x \geq 2$ and fixed $k \geq 1$. },
    type={symbols},
    name={Distinct prime counting functions}
    }   
\newglossaryentry{Nupn}{
    symbol={$\nu_p(n)$}, 
    sort={Nupn},
    description={The valuation function that extracts the maximal exponent of $p$ in the prime factorization of $n$, e.g., 
                 $\nu_p(n) = 0$ if $p \nmid n$ and $\nu_p(n) = \alpha$ if $p^{\alpha} || n$ 
                 for $p \geq 2$ prime, $\alpha \geq 1$ and $n \geq 2$.},
    type={symbols},
    name={Exponent extraction function}
    }
\newglossaryentry{primeOmegaFunctions}{
    symbol={$\omega(n)$,$\Omega(n)$}, 
    sort={OmegaPrimeOmegaFunctions},
    description={We define the strongly additive function 
                 $\omega(n) := \sum\limits_{p|n} 1$ and the completely additive function 
                 $\Omega(n) := \sum\limits_{p^{\alpha} || n} \alpha$. This means that if the prime 
                 factorization of $n \geq 2$ is 
                 given by $n := p_1^{\alpha_1} \times \cdots \times p_r^{\alpha_r}$ 
                 with $p_i \neq p_j$ for all $i \neq j$, 
                 then $\omega(n) = r$ and $\Omega(n) = \alpha_1 + \cdots + \alpha_r$. 
                 We set $\omega(1) = \Omega(1) = 0$ by convention.},
    type={symbols},
    name={Prime omega functions}
    }
\newglossaryentry{LiouvilleLambdaFunc}{
     symbol={$\lambda(n), L(x)$}, 
    sort={LiouvilleLambdaFunc},
    description={The Liouville lambda function is the completely multiplicative function defined by 
                 $\lambda(n) := (-1)^{\Omega(n)}$. 
                 Its summatory function is defined by the partial sums 
                 $L(x) := \sum\limits_{n \leq x} \lambda(n)$ for $x \geq 1$. 
                 },
    type={symbols},
    name={Liouville lambda function}
    }
\newglossaryentry{QxSummatoryFunc}{
    symbol={$Q(x)$},
    sort={QxSummatoryFunc},
    description={For $x \geq 1$, we define $Q(x)$ to be the summatory function indicating the number of 
                 squarefree integers $n \leq x$. That is, $Q(x) := \sum\limits_{n \leq x} \mu^2(n)$ where 
                 $Q(x) = \frac{6x}{\pi^2} + O(\sqrt{x})$.}, 
    type={symbols},
    name={Summatory function of the squarefree integers}
    }
\newglossaryentry{ApproxAndSimRelations}{
    symbol={$\approx,\sim$},
    sort={ApproxAndSimRelations},
    description={We write that $f(x) \approx g(x)$ if $|f(x) - g(x)| \ll 1$ 
                 as $x \rightarrow \infty$. 
                 Two arithmetic functions $A(x), B(x)$ satisfy the relation $A \sim B$ if 
                 $\lim_{x \rightarrow \infty} \frac{A(x)}{B(x)} = 1.$ },
    type={symbols},
    name={Asymptotic relation symbol}
    }
\newglossaryentry{AApproxSimGGLLRelations}{
    symbol={$\gg,\ll,\asymp$},
    sort={AApproxSimGGLLRelations},
    description={
                 For functions $A,B$, the notation $A \ll B$ implies that $A = O(B)$. 
                 Similarly, for $B \geq 0$ the notation $A \gg B$ implies that $B = O(A)$. 
                 When we have that $A, B \geq 0$, $A \ll B$ and $B \ll A$, we write $A \asymp B$. },
    type={symbols},
    name={Asymptotic relation symbols}
    }
\newglossaryentry{NormalCDFFunc}{
    symbol={$\Phi(z)$},
    sort={NormalCDFFunc},
    description={For $z \in \mathbb{R}$, we take the cumulative density function 
                 of the standard normal distribution to be denoted by 
                 $\Phi(z) := \frac{1}{\sqrt{2\pi}} \times \int\limits_{-\infty}^{z} e^{-\frac{t^2}{2}} dt$. },
    type={symbols},
    name={Asymptotic relation symbol}
    }

\newglossaryentry{chiPrimeP}{
    symbol={$\chi_{\mathbb{P}}(n), P(s)$},
    sort={chiPrimeP},
    description={The indicator function of the primes equals one if and only if 
                 $n \in \mathbb{Z}^{+}$ is prime, and is 
                 zero-valued otherwise. 
                 For any $s \in \mathbb{C}$ such that $\Re(s) > 1$, 
                 we define the prime zeta function to be the 
                 Dirichlet generating function (DGF) defined by 
                 $P(s) = \sum\limits_{n \geq 1} \frac{\chi_{\mathbb{P}}(n)}{n^s}$. 
                 The function $P(s)$ has an analytic continuation to the half-plane 
                 $\Re(s) > 0$ through the formula 
                 $P(s) = \sum\limits_{k \geq 1} \frac{\mu(k)}{k} \log\zeta(ks)$ with 
                 poles at the reciprocal of each positive integer and a natural boundary 
                 at the line $\Re(s) = 0$. },
    type={symbols},
    name={Prime set indicator function}
    }
\newglossaryentry{WLambertWFunction}{
    symbol={$W(x)$},
    sort={WLambertWFunction},
    description={For $x,y \in \mathbb{R}_{\geq 0}$, we write that $x = W(y)$ if and only if $xe^{x} = y$. 
                 This function denotes the principal branch of the multi-valued Lambert $W$ function 
                 defined on the non-negative reals. },
    type={symbols},
    name={Lambert $W$-Function}
    }
\glsaddall[types={symbols}]

\allowdisplaybreaks 

\begin{document} 

\maketitle

\begin{abstract} 
The Mertens function, $M(x) := \sum_{n \leq x} \mu(n)$, is 
defined as the summatory function of the classical M\"obius function for $x \geq 1$. 
The inverse function $g^{-1}(n) := (\omega+1)^{-1}(n)$
taken with respect to Dirichlet convolution is defined in terms of the 
strongly additive function $\omega(n)$ that counts the 
number of distinct prime factors of the integers $n \geq 2$ without multiplicity. 
For large $x$ and $n \leq x$, we associate a natural combinatorial 
significance to the magnitude of the distinct values of 
$g^{-1}(n)$ that depends directly on the exponent patterns in the 
prime factorizations of the integers in $\{2,3,\ldots,x\}$ viewed as multisets. 
We have an Erd\H{o}s-Kac theorem analog for the distribution of the 
unsigned sequence $|g^{-1}(n)|$ over $n \leq x$ as $x \rightarrow \infty$. 
The key connection of the partial sums of the auxiliary function 
$C_{\Omega(n)}(n) := (\Omega(n))! \times \prod_{p^{\alpha}||n} (\alpha!)^{-1}$ 
to $|g^{-1}(n)|$ is proved using assumptions on the independence of the completely 
additive function $\Omega(n)$ and the distribution of the exponents of the 
distinct prime factors of $2 \leq n \leq x$ when $x$ is large. 
Discrete convolutions of the summatory function 
$G^{-1}(x) := \sum_{n \leq x} \lambda(n) |g^{-1}(n)|$ with 
the prime counting function $\pi(x)$ determine 
exact formulas and new characterizations of asymptotic approaches to $M(x)$. 
In this way, we prove another characteristic 
link of the Mertens function to the distribution of the partial sums  
$L(x) := \sum_{n \leq x} \lambda(n)$ and connect these two classical 
summatory functions with an explicit probability distribution at large $x$. 

\bigskip 
\noindent
\textbf{Keywords and Phrases:} {\it M\"obius function; Mertens function; 
                                    Dirichlet inverse; Liouville lambda function; prime omega function; 
                                    prime counting function; Dirichlet generating function; 
                                    Erd\H{o}s-Kac theorem; strongly additive function. } \\ 
% 11-XX			Number theory
%    11A25  	Arithmetic functions; related numbers; inversion formulas
%    11Y70  	Values of arithmetic functions; tables
%    11-04  	Software, source code, etc. for problems pertaining to number theory
% 11Nxx		Multiplicative number theory
%    11N05  	Distribution of primes
%    11N37  	Asymptotic results on arithmetic functions
%    11N56  	Rate of growth of arithmetic functions
%    11N60  	Distribution functions associated with additive and positive multiplicative functions
%    11N64  	Other results on the distribution of values or the characterization of arithmetic functions
\textbf{Math Subject Classifications (MSC 2010):} {\it 11N37; 11A25; 11N60; 11N64; and 11-04. } 
\end{abstract} 

%\bigskip\hrule\medskip
%\begin{center}
%\begin{adjustwidth}{3.5cm}{}
%\emph{It is evident that the primes are randomly distributed but, \\ 
%        unfortunately, we do not know what 'random' means.} \\ 
%      \textbf{R. C. Vaughan} 
%\end{adjustwidth}
%\end{center}
%\medskip\hrule\bigskip

\newpage
\renewcommand{\contentsname}{Article Index}
\setcounter{tocdepth}{2}
\addcontentsline{toc}{section}{\nameref{Section_NotationAndConventions}}
\tableofcontents

\newpage
\section*{Notation and conventions}
\label{Section_NotationAndConventions}

The next listing provides a glossary of common notation, conventions and 
abbreviations employed throughout the article. 

\renewcommand*{\glsclearpage}{}
\renewcommand{\glossarysection}[2][]{}
\printglossary[type={symbols},
               style={glossstyleSymbol},
               nogroupskip=true]

\newpage
\section{Introduction} 
\label{subSection_MertensMxClassical_Intro} 

The \emph{Mertens function}, or the summatory function of $\mu(n)$, is defined on the 
positive integers by the partial sums  
\begin{align*} 
M(x) & = \sum_{n \leq x} \mu(n), x \geq 1. 
\end{align*} 
The first several values of this 
summatory function are calculated as follows 
\cite[\seqnum{A008683}; \seqnum{A002321}]{OEIS}: 
\[
\{M(x)\}_{x \geq 1} = \{1, 0, -1, -1, -2, -1, -2, -2, -2, -1, -2, -2, -3, -2, 
     -1, -1, -2, -2, -3, -3, -2, -1, -2, \ldots\}. 
\] 
The Mertens function is related 
to the partial sums of the Liouville lambda function, 
denoted by $L(x) := \sum\limits_{n \leq x} \lambda(n)$, 
via the relation \cite{HUMPHRIES-JNT-2013,LEHMAN-1960} 
\cite[\seqnum{A008836}; \seqnum{A002819}]{OEIS}
\[
L(x) = \sum_{d \leq \sqrt{x}} M\left(\Floor{x}{d^2}\right), x \geq 1. 
\] 
The main interpretation to take away from the article is 
the new characterization of $M(x)$ through two primary 
auxiliary unsigned sequences and their 
summatory functions, namely, the functions $C_{\Omega(n)}(n)$, $g^{-1}(n)$ and their 
partial sums. This characterization is formed by constructing the 
combinatorially motivated sequences related to the distribution of the primes 
by convolutions of the strongly additive function $\omega(n)$. 
The methods in this article initially stem from a 
curiosity about an elementary identity from 
the list of exercises in \cite[\S 2; \cf \S 11]{APOSTOLANUMT}. 
In particular, the indicator function of the primes is given by M\"obius inversion as the 
Dirichlet convolution $\chi_{\mathbb{P}} + \varepsilon = (\omega + 1) \ast \mu$. 
We form partial sums of $(\omega + 1) \ast \mu(n)$ over $n \leq x$ for 
any $x \geq 1$ and then apply classical inversion theorems to relate  
$M(x)$ to the partial sums of $g^{-1}(n) := (\omega+1)^{-1}(n)$ 
(\cf Theorem \ref{theorem_SummatoryFuncsOfDirCvls}; 
Corollary \ref{cor_CvlGAstMu}; and  
Corollary \ref{cor_Mx_gInvnPixk_formula}). 

\subsection{Motivation} 

There is a natural relationship of $g^{-1}(n)$ with the auxiliary function 
$C_{\Omega(n)}(n)$, or the $\Omega(n)$-fold Dirichlet convolution of $\omega(n)$ 
with itself at $n$, which we prove by elementary methods in 
Section \ref{Section_InvFunc_PreciseExpsAndAsymptotics}. 
These identities inspire the deep connection between the 
unsigned inverse function and additive prime counting combinatorics we find in 
Section \ref{subSection_AConnectionToDistOfThePrimes}. 
In this sense, the new results stated within this article diverge from the proofs 
typified by previous anayltic and combinatorial 
methods to bound $M(x)$ cited in the references. 
The function $C_{\Omega(n)}(n)$ was considered under alternate notation 
by Fr\"oberg (circa 1968) in his work on the series expansions of the 
\emph{prime zeta function}, $P(s)$, 
e.g., the prime sums defined as the 
Dirichlet generating function (DGF) of $\chi_{\mathbb{P}}(n)$. 
The clear interpretation of the function $C_{\Omega(n)}(n)$ in connection with $M(x)$ 
is unique to our work to establish the 
properties of this auxiliary sequence. 
References to uniform asymptotics for restricted partial sums of 
$C_{\Omega(n)}(n)$ and the features of the limiting distribution 
of this function are missing in surrounding literature 
(\cf Corollary \ref{cor_SummatoryFuncsOfUnsignedSeqs_v2}; 
Proposition \ref{lemma_HatCAstxSum_ExactFormulaWithError_v1}; and 
Theorem \ref{theorem_CLT_VI}). 

We utilize the results in \cite[\S 7.4; \S 2.4]{MV} 
that apply traditional analytic methods to formulate limiting asymptotics and to 
prove an Erd\H{o}s-Kac theorem analog characterizing key properties of the 
distribution of the completely additive 
function $\Omega(n)$. Adaptations of the key ideas 
from the exposition in the reference 
provide a foundation for analytic proofs of several limiting 
properties of, asymptotic formulae for restricted partial sums involving, and in part the 
Erd\H{o}s-Kac type theorem for both 
$C_{\Omega(n)}(n)$ and $|g^{-1}(n)|$. 
Our Erd\H{o}s-Kac type theorem variants characterizing the distributions of both 
$C_{\Omega(n)}(n)$ and $|g^{-1}(n)|$ are established under reasonable limiting assumptions 
on the random variables $X_{n,k} := \frac{C_{\Omega(n)}(n)}{(\log n)(\log\log n)}$ when 
$\Omega(n) = k$ for $k \geq 1$ and $n \leq x$ as $x \rightarrow \infty$. 
The sequence $g^{-1}(n)$ and its partial sums defined by 
$G^{-1}(x) := \sum_{n \leq x} g^{-1}(n)$ are linked to 
canonical examples of strongly and completely additive functions, 
e.g., in relation to $\omega(n)$ and $\Omega(n)$, respectively. 
The definitions of the sequences we define, and the 
proof methods given in the spirit of Montgomery and Vaughan's work, 
allow us to reconcile the property of strong additivity with the signed 
partial sums of a multiplicative function. 
We leverage the connection of $C_{\Omega(n)}(n)$ and $|g^{-1}(n)|$ with the 
canonical number theoretic additive functions to obtain the results proved primarily 
in Section \ref{Section_NewFormulasForgInvn}. 

We also formulate a probabilistic perspective from which to express 
our intuition about features of the distribution of $G^{-1}(x)$ 
via the properties of its summands. 
Since we prove that $\operatorname{sgn}(g^{-1}(n)) = \lambda(n)$ for all $n \geq 1$ in 
Proposition \ref{prop_SignageDirInvsOfPosBddArithmeticFuncs_v1}, 
the partial sums defined by $G^{-1}(x)$ are precisely related to the properties of 
$|g^{-1}(n)|$ and asymptotics for $L(x)$. 
Our new results then relate the 
distribution of $L(x)$, an explicitly identified 
probability distribution, and $M(x)$ as $x \rightarrow \infty$. 
Formalizing the properties of the distribution of 
$L(x)$ is still viewed as a problem that is equally as difficult 
as understanding the properties of $M(x)$ well at large $x$ or along infinite subsequences. 

Our characterizations of $M(x)$ by the summatory function of the signed 
inverse sequence, $G^{-1}(x)$, 
is suggestive of new approaches to bounding the Mertens function. 
These results motivate future work to state upper (and possibly lower) bounds 
on $M(x)$ in terms of the additive combinatorial properties of the repeated distinct 
values of the sign weighted summands of $G^{-1}(x)$. 
We also expect that an outline of the method behind the collective proofs we 
provide with respect to the Mertens function case can be generalized to identify 
associated additive functions with the same role of $\omega(n)$ in this paper to 
express asymptotics for partial sums of other signed multiplicative functions. 

\subsection{Preliminaries on the Mertens function}
\label{subSection_Intro_Mx_properties} 

An approach to evaluating the limiting asymptotic 
behavior of $M(x)$ for large $x \rightarrow \infty$ considers an 
inverse Mellin transform of the reciprocal of the Riemann zeta function given by 
\[
\frac{1}{\zeta(s)} = \prod_{p} \left(1 - \frac{1}{p^s}\right) = 
     s \times \int_1^{\infty} \frac{M(x)}{x^{s+1}} dx, \Re(s) > 1. 
\]
In particular, we obtain that 
\[
M(x) = \lim_{T \rightarrow \infty}\ \frac{1}{2\pi\imath} \times \int_{T-\imath\infty}^{T+\imath\infty} 
     \frac{x^s}{s \zeta(s)} ds. 
\] 
The previous formulas lead to the 
exact expression of $M(x)$ for any $x > 0$ 
given by the next theorem. 
\nocite{TITCHMARSH} 

\begin{theorem}[Titchmarsh] 
\label{theorem_MxMellinTransformInvFormula} 
Assuming the Riemann Hypothesis (RH), there exists an infinite sequence 
$\{T_k\}_{k \geq 1}$ satisfying $k \leq T_k \leq k+1$ for each integer $k \geq 1$ 
such that for any real $x > 0$ 
\[
M(x) = \lim_{k \rightarrow \infty} 
     \sum_{\substack{\rho: \zeta(\rho) = 0 \\ 0 < |\Im(\rho)| < T_k}} 
     \frac{x^{\rho}}{\rho \zeta^{\prime}(\rho)} - 2 + 
     \sum_{n \geq 1} \frac{(-1)^{n-1}}{n (2n)! \zeta(2n+1)} 
     \left(\frac{2\pi}{x}\right)^{2n} + 
     \frac{\mu(x)}{2} \Iverson{x \in \mathbb{Z}^{+}}. 
\] 
\end{theorem} 

An unconditional bound on the Mertens function due to Walfisz (circa 1963) 
states that there is an absolute constant $C_1 > 0$ such that 
$$M(x) \ll x \times \exp\left(-C_1 \log^{\frac{3}{5}}(x) 
  (\log\log x)^{-\frac{1}{5}}\right).$$ 
Under the assumption of the RH, Soundararajan and Humphries, respectively, 
improved estimates bounding $M(x)$ from above for large $x$ in the 
following forms 
\cite{SOUND-MERTENS-ANNALS,HUMPHRIES-JNT-2013}: 
\begin{align*} 
M(x) & \ll \sqrt{x} \times \exp\left(\sqrt{\log x} (\log\log x)^{14}\right), \\ 
M(x) & \ll \sqrt{x} \times \exp\left( 
     \sqrt{\log x} (\log\log x)^{\frac{5}{2}+\epsilon}\right), 
     \text{ for all } \epsilon > 0. 
\end{align*} 
The RH is equivalent to showing that 
$M(x) = O\left(x^{\frac{1}{2}+\epsilon}\right)$ for any 
$0 < \epsilon < \frac{1}{2}$. 
There is a rich history to the original statement of the \emph{Mertens conjecture} which 
asserts that $|M(x)| < C_2 \sqrt{x}$ for some absolute constant $C_2 > 0$. 
The conjecture was first verified by F.~Mertens himself for $C_2 = 1$ and all $x < 10000$ 
without the benefit of modern computation. 
Since its beginnings in 1897, the Mertens conjecture was disproved by computational methods involving 
non-trivial simple zeta function zeros with comparatively small imaginary parts in the 
famous paper from the mid 1980's by 
Odlyzko and te Riele \cite{ODLYZ-TRIELE}. 

More recent attempts 
at bounding $M(x)$ naturally consider determining the rates at which the function 
$M(x) x^{-\frac{1}{2}}$ grows with or without bound along infinite 
subsequences, i.e., considering the asymptotics of the function 
in the limit supremum and limit infimum senses. 

It is verified by computation 
that \cite[\cf \S 4.1]{PRIMEREC} 
\cite[\cf \seqnum{A051400}; \seqnum{A051401}]{OEIS} 
\[
\limsup_{x\rightarrow\infty} \frac{M(x)}{\sqrt{x}} > 1.060\ \qquad (\text{more recently } \geq 1.826054), 
\] 
and 
\[ 
\liminf_{x\rightarrow\infty} \frac{M(x)}{\sqrt{x}} < -1.009\ \qquad (\text{more recently } \leq -1.837625). 
\] 
Based on the work by Odlyzko and te Riele, it is likely that 
each of these limiting bounds evaluates to $\pm \infty$, respectively 
\cite{ODLYZ-TRIELE,MREVISITED,ORDER-MERTENSFN,HURST-2017}. 
A conjecture due to Gonek asserts that in fact 
$M(x)$ satisfies \cite{NG-MERTENS}
$$\limsup_{x \rightarrow \infty} \frac{|M(x)|}{\sqrt{x} (\log\log\log x)^{\frac{5}{4}}} = C_3,$$ 
for $C_3$ an absolute constant. 

\subsection{A concrete new approach to characterizing $M(x)$} 

\subsubsection{Summatory functions of Dirichlet convolutions of arithmetic functions} 

We prove the formulas in the next inversion theorem by matrix methods in 
Section \ref{subSection_PrelimProofs_Config_InversionTheorem}. 

\begin{theorem}[Partial sums of Dirichlet convolutions and their inversions] 
\label{theorem_SummatoryFuncsOfDirCvls} 
Let $r,h: \mathbb{Z}^{+} \rightarrow \mathbb{C}$ be any arithmetic functions such that $r(1) \neq 0$. 
Suppose that $R(x) := \sum_{n \leq x} r(n)$ and $H(x) := \sum_{n \leq x} h(n)$ denote the summatory 
functions of $r$ and $h$, respectively, and that $R^{-1}(x) := \sum_{n \leq x} r^{-1}(n)$ 
denotes the summatory function of the 
Dirichlet inverse of $r$ for any $x \geq 1$. We have the following exact expressions that hold 
for all integers $x \geq 1$: 
\begin{align*} 
\pi_{r \ast h}(x) & := \sum_{n \leq x} \sum_{d|n} r(d) h\left(\frac{n}{d}\right) \\ 
     & \phantom{:}= \sum_{d \leq x} r(d) H\left(\Floor{x}{d}\right) \\ 
     & \phantom{:}= \sum_{k=1}^{x} H(k) \left(R\left(\Floor{x}{k}\right) - 
     R\left(\Floor{x}{k+1}\right)\right). 
\end{align*} 
Moreover, for any $x \geq 1$ we have 
\begin{align*} 
H(x) & = \sum_{j=1}^{x} \pi_{r \ast h}(j) \left(R^{-1}\left(\Floor{x}{j}\right) - 
     R^{-1}\left(\Floor{x}{j+1}\right)\right) \\ 
     & = \sum_{k=1}^{x} r^{-1}(k) \pi_{r \ast h}\left(\Floor{x}{k}\right). 
\end{align*} 
\end{theorem} 

Key consequences of Theorem \ref{theorem_SummatoryFuncsOfDirCvls} 
as it applies to $M(x)$ in the special case of $h(n) := \mu(n)$ for all $n \geq 1$ 
are stated as the next two corollaries. 

\begin{cor}[Applications of M\"obius inversion] 
\label{cor_CvlGAstMu} 
Suppose that $r$ is an arithmetic function such that 
$r(1) \neq 0$. Define the summatory function of 
the convolution of $r$ with $\mu$ by $\widetilde{R}(x) := \sum_{n \leq x} (r \ast \mu)(n)$. 
Then the Mertens function is expressed by the partial sums 
\[
M(x) = \sum_{k=1}^{x} \left(\sum_{j=\floor{\frac{x}{k+1}}+1}^{\floor{\frac{x}{k}}} r^{-1}(j)\right) 
     \widetilde{R}(k), \forall x \geq 1. 
\]
\end{cor} 

\begin{cor}[Key Identity] 
\label{cor_Mx_gInvnPixk_formula} 
We have that for all $x \geq 1$ 
\begin{equation} 
\label{eqn_Mx_gInvnPixk_formula} 
M(x) = \sum_{k=1}^{x} (\omega+1)^{-1}(k) \left(\pi\left(\Floor{x}{k}\right) + 1\right). 
\end{equation} 
\end{cor} 

\subsubsection{An exact expression for $M(x)$ via strongly additive functions} 
\label{example_InvertingARecRelForMx_Intro}

We fix the notation for the Dirichlet invertible function $g(n) := \omega(n) + 1$ and define its 
inverse with respect to Dirichlet convolution by $g^{-1}(n)$ 
\cite[\seqnum{A341444}]{OEIS}. 
We compute the first several values of this sequence as follows: 
\[
\{g^{-1}(n)\}_{n \geq 1} = \{1, -2, -2, 2, -2, 5, -2, -2, 2, 5, -2, -7, -2, 5, 5, 2, -2, -7, -2, 
     -7, 5, 5, -2, 9, \ldots \}. 
\] 
There is not a simple 
direct recursion between the distinct values of $g^{-1}(n)$ that holds for all $n \geq 1$. 
However, the next observation is suggestive of the quasi-periodicity of the distribution of 
distinct values of this inverse function over $n \geq 2$. 

\begin{observation}[Additive symmetry in $g^{-1}(n)$ from the prime factorizations of $n \leq x$] 
\label{heuristic_SymmetryIngInvFuncs} 
Suppose that $n_1, n_2 \geq 2$ are such that their factorizations into distinct primes are 
given by $n_1 = p_1^{\alpha_1} \times \cdots \times p_r^{\alpha_r}$ and 
$n_2 = q_1^{\beta_1} \times \cdots \times q_s^{\beta_s}$. 
If $r = s$ and $\{\alpha_1, \ldots, \alpha_r\} \equiv \{\beta_1, \ldots, \beta_r\}$ 
as multisets of the prime exponents, 
then $g^{-1}(n_1) = g^{-1}(n_2)$. For example, $g^{-1}$ has the same values on the squarefree integers 
with exactly one, two, three (and so on) prime factors. 
Hence, there is an essentially additive structure underneath the sequence 
$\{g^{-1}(n)\}_{n \geq 2}$. 
\end{observation} 

\begin{prop} 
\label{lemma_gInv_MxExample} 
We have the following properties of the 
Dirichlet inverse function $g^{-1}(n)$: 
\begin{itemize} 

\item[(A)] For all $n \geq 1$, $\operatorname{sgn}(g^{-1}(n)) = \lambda(n)$; 
\item[(B)] For all squarefree integers $n \geq 1$, we have that 
     \[
     |g^{-1}(n)| = \sum_{m=0}^{\omega(n)} \binom{\omega(n)}{m} \times m!; 
     \]
\item[(C)] If $n \geq 2$ and $\Omega(n) = k$ for some $k \geq 1$, then 
     \[
     2 \leq |g^{-1}(n)| \leq \sum_{j=0}^{k} \binom{k}{j} \times j!. 
     \]
\end{itemize} 
\end{prop} 

The signedness property in (A) is proved precisely in 
Proposition \ref{prop_SignageDirInvsOfPosBddArithmeticFuncs_v1}. 
A proof of (B) follows from 
Lemma \ref{lemma_AnExactFormulaFor_gInvByMobiusInv_v1}. 
The realization that the beautiful and remarkably simple combinatorial form of property (B) 
in Proposition \ref{lemma_gInv_MxExample} holds for all squarefree integers 
motivates our pursuit of simpler formulas for the inverse function $g^{-1}(n)$ 
through the sums of auxiliary subsequences $C_k(n)$ when $k := \Omega(n)$ 
defined in Section \ref{Section_InvFunc_PreciseExpsAndAsymptotics}. 
That is, we observe a familiar formula for $g^{-1}(n)$ 
on an asymptotically dense infinite subset of integers (with density $\frac{6}{\pi^2}$), 
e.g., that holds for all squarefree $n \geq 2$, and then seek 
to extrapolate by proving there are in fact 
regular properties of the distribution of this sequence when viewed 
more generally over the positive integers. 

An exact expression for $g^{-1}(n)$ is given by 
\[
g^{-1}(n) = \lambda(n) \times \sum_{d|n} \mu^2\left(\frac{n}{d}\right) C_{\Omega(d)}(d), n \geq 1,  
\]
where the sequence $\lambda(n) C_{\Omega(n)}(n)$ has the DGF $(1 + P(s))^{-1}$ and 
$C_{\Omega(n)}(n)$ has DGF $(1-P(s))^{-1}$ for $\Re(s) > 1$ 
(see Proposition \ref{prop_SignageDirInvsOfPosBddArithmeticFuncs_v1}). 
The function $C_{\Omega(n)}(n)$ was considered in 
\cite{FROBERG-1968} with its exact formula 
given by \cite[\cf \S 3]{CLT-RANDOM-ORDERED-FACTS-2011} 
\[
C_{\Omega(n)}(n) = \begin{cases}
     1, & \text{if $n = 1$; } \\ 
     (\Omega(n))! \times \prod\limits_{p^{\alpha}||n} \frac{1}{\alpha!}, & \text{if $n \geq 2$. }
     \end{cases}
\]
In Corollary \ref{cor_SummatoryFuncsOfUnsignedSeqs_v2}, 
we use the result proved in 
Theorem \ref{theorem_CnkSpCasesScaledSummatoryFuncs} 
to show that uniformly for $1 \leq k \leq 2\log\log x$ there is an absolute 
constant $A_0 > 0$ such that 
\[
\sum_{\substack{n \leq x \\ \Omega(n)=k}} C_{\Omega(n)}(n) = 
     \frac{A_0 \sqrt{2\pi} x}{\log x} \times 
     \widehat{G}\left(\frac{k-1}{\log\log x}\right) 
     \frac{(\log\log x)^{k-\frac{1}{2}}}{(k-1)!} \left( 
     1 + O\left(\frac{1}{\log\log x}\right)\right), 
     \mathrm{\ as\ } x \rightarrow \infty, 
\]
where $\widehat{G}(z) := \frac{\zeta(2)^{-z}}{\Gamma(1+z) (1+P(2)z)}$ for 
$0 \leq |z| < P(2)^{-1}$. 

In Proposition \ref{lemma_HatCAstxSum_ExactFormulaWithError_v1}, 
we use an adaptation of the asymptotic formulas for the summations 
proved in the appendix section of this article
combined with the form of \emph{Rankin's method} from \cite[Thm.~7.20]{MV} to show that 
there is another absolute constant $B_0 > 0$ such that 
\[
\frac{1}{n} \times \sum_{k \leq n} C_{\Omega(k)}(k) = 
     B_0 (\log n) \sqrt{\log\log n} 
     \left(1 + O\left(\frac{1}{\log\log n}\right)\right), 
     \mathrm{\ as\ } n \rightarrow \infty. 
\]
In Corollary \ref{cor_ExpectationFormulaAbsgInvn_v2}, we prove that 
the average order of $|g^{-1}(n)|$ is 
\[
\frac{1}{n} \times \sum_{k \leq n} |g^{-1}(k)| = 
     \frac{6B_0 (\log n)^2 \sqrt{\log\log n}}{\pi^2} 
     \left(1 + O\left(\frac{1}{\log\log n}\right)\right), 
     \mathrm{\ as\ } n \rightarrow \infty. 
\]
In Section \ref{subSection_ErdosKacTheorem_Analogs}, 
we prove a variant of the Erd\H{o}s-Kac theorem 
that characterizes the distribution of $C_{\Omega(n)}(n)$ 
which holds under reasonable assumptions on independence
(see Theorem \ref{theorem_CLT_VI}; 
\cf Ansatz \ref{ansatz_KeyIndependenceAssumptions}). 
The theorem leads the conclusion of the following statement for any fixed $Y > 0$, with 
$\mu_x(C) := \log\log x - 
 \log\left(\frac{\sqrt{2\pi}A_0}{\zeta(2)(1+P(2))}\right)$ and 
$\sigma_x(C) := \sqrt{\log\log x}$, 
and holds uniformly for any $-Y \leq y \leq Y$ 
(see Corollary \ref{cor_CLT_VII}): 
\begin{align*}
\frac{1}{x} & \times \#\left\{3 \leq n \leq x: 
     \frac{|g^{-1}(n)|}{(\log n) \sqrt{\log\log n}} - 
     \frac{6}{\pi^2 n (\log n) \sqrt{\log\log n}} 
     \times \sum_{k \leq n} |g^{-1}(k)| \leq y\right\} \\ 
     & = 
     \Phi\left(\frac{\frac{\pi^2 y}{6}-\mu_x(C)}{\sigma_x(C)}\right) + 
     o(1), \text{ as } x \rightarrow \infty. 
\end{align*}
The regularity and quasi-periodicity we alluded 
to in the previous few remarks are then 
quantifiable insomuch as $|g^{-1}(n)|$ 
tends to a scaled multiple of its average order 
with a non-centrally normal tendency. 
If $x$ is sufficiently large and 
if we pick any integer $n \in [2, x]$ uniformly at random, then 
the following statement also holds as $x \rightarrow \infty$: 
\begin{align*} 
\mathbb{P}\left(|g^{-1}(n)| - \frac{6}{\pi^2 n} \times \sum_{k \leq n} |g^{-1}(k)| \leq 
     \frac{6}{\pi^2} (\log n) \sqrt{\log\log n} 
     \left(\alpha \sigma_x(C) + \mu_x(C)\right)
     \right) & = 
     \Phi\left(\alpha\right) + o(1), \alpha \in \mathbb{R}. 
\end{align*} 

\subsubsection{Formulas illustrating the new characterizations of $M(x)$} 

Let the partial sums 
$G^{-1}(x) := \sum_{n \leq x} g^{-1}(n)$ for integers $x \geq 1$ 
\cite[\seqnum{A341472}]{OEIS}. 
We prove that (see Proposition \ref{prop_Mx_SBP_IntegralFormula}) 
\begin{align} 
\label{eqn_Mx_gInvnPixk_formula_v2} 
M(x) & = G^{-1}(x) + 
     \sum_{k=1}^{\frac{x}{2}} G^{-1}(k) \left( 
     \pi\left(\Floor{x}{k}\right) - \pi\left(\Floor{x}{k+1}\right) 
     \right), x \geq 1, 
\end{align} 
and that (\cf Section \ref{subSection_Relating_CknFuncs_to_gInvn}) 
\[
M(x) = G^{-1}(x) + \sum_{p \leq x} G^{-1}\left(\Floor{x}{p}\right), x \geq 1. 
\]
These formulas 
imply that we can establish asymptotic bounds on 
$M(x)$ along infinite subsequences
by sharply bounding the summatory function $G^{-1}(x)$ along those points. 
We also have an identification of $G^{-1}(x)$ with $L(x)$ given by 
\[
G^{-1}(x) 
     = L(x)|g^{-1}(x)| - \sum_{n < x} 
     L(n) \left(\left\lvert g^{-1}(n+1)\right\rvert - \left\lvert g^{-1}(n)\right\rvert\right), 
\]
where the distribution of $|g^{-1}(n)|$ is characterized by 
Corollary \ref{cor_CLT_VII}. 
In Section \ref{subSection_AsymptoticsOfGinvx}, 
we use the anaytic methods due to H.~Davenport and H.~Heilbronn 
suggested by R.~C.~Vaughan to prove that for 
$\sigma_1 \approx 1.39943$ the unique solution to $P(\sigma) = 1$ on 
$(1, \infty)$ we have 
\[
\limsup_{x \rightarrow \infty} \frac{\log\left\lvert G^{-1}(x) \right\rvert}{\log x} \geq \sigma_1. 
\]
Hence, for any $\epsilon > 0$, 
Corollary \ref{cor_Vaughan_LimSupLowerBounds_On_GInvx_AtLarge_x_v2} 
proves that there are arbitrarily large $x$ such that 
\[
|G^{-1}(x)| > x^{\sigma_1-\epsilon}. 
\]
Nonetheless, we still expect substantial local cancellation in the terms involving 
$G^{-1}(x)$ in our new formulas for $M(x)$ at almost every large $x$ 
(see Section \ref{subSection_LocalCancellationOfGInvx}). 

\newpage 
\section{Initial elementary proofs of new results} 
\label{Section_PrelimProofs_Config} 

\subsection{Establishing the summatory function properties and inversion identities} 
\label{subSection_PrelimProofs_Config_InversionTheorem}

We give a proof of the inversion type results in 
Theorem \ref{theorem_SummatoryFuncsOfDirCvls} 
by matrix methods in this section. 
Related results on summations of Dirichlet convolutions and their inversion appear in 
\cite[\S 2.14; \S 3.10; \S 3.12; \cf \S 4.9, p.\ 95]{APOSTOLANUMT}. 
It is similarly not difficult to establish the identity 
\[
\sum_{n \leq x} h(n) (q \ast r)(n) = 
     \sum_{n \leq x} q(n) \times \sum_{k \leq \Floor{x}{n}} r(k) h(kn). 
\]

\begin{proof}[Proof of Theorem \ref{theorem_SummatoryFuncsOfDirCvls}] 
\label{proofOf_theorem_SummatoryFuncsOfDirCvls} 
Let $h,r$ be arithmetic functions such that $r(1) \neq 0$. 
Denote the summatory functions of $h$, $r$ and $r^{-1}$, 
respectively, by $H(x) = \sum_{n \leq x} h(n)$, $R(x) = \sum_{n \leq x} r(n)$, 
and $R^{-1}(x) = \sum_{n \leq x} r^{-1}(n)$. 
We define $\pi_{r \ast h}(x)$ to be the summatory function of the 
Dirichlet convolution of $r$ with $h$. 
We have that the following formulas hold for all $x \geq 1$: 
\begin{align} 
\notag 
\pi_{r \ast h}(x) & := \sum_{n=1}^{x} \sum_{d|n} r(n) h\left(\frac{n}{d}\right) = 
     \sum_{d=1}^x r(d) H\left(\floor{\frac{x}{d}}\right) \\ 
\label{eqn_proof_tag_PigAsthx_ExactSummationFormula_exp_v2} 
     & = \sum_{i=1}^x \left(R\left(\floor{\frac{x}{i}}\right) - R\left(\floor{\frac{x}{i+1}}\right)\right) H(i). 
\end{align} 
The first formula above is well known from the references cited above. 
The second formula is justified directly using 
summation by parts as \cite[\S 2.10(ii)]{NISTHB} 
\begin{align*} 
\pi_{r \ast h}(x) & = \sum_{d=1}^x h(d) R\left(\floor{\frac{x}{d}}\right) \\ 
     & = \sum_{i \leq x} \left(\sum_{j \leq i} h(j)\right) \times 
     \left(R\left(\floor{\frac{x}{i}}\right) - 
     R\left(\floor{\frac{x}{i+1}}\right)\right). 
\end{align*} 
We form the invertible matrix of coefficients $\hat{R}$ 
associated with the linear system defining $H(j)$ for all 
$1 \leq j \leq x$ in \eqref{eqn_proof_tag_PigAsthx_ExactSummationFormula_exp_v2} by setting 
\[
r_{x,j} := R\left(\floor{\frac{x}{j}}\right) - R\left(\floor{\frac{x}{j+1}}\right) 
     \equiv R_{x,j} - R_{x,j+1}, 
\] 
where 
\[
R_{x,j} := R\left(\Floor{x}{j}\right), \text{ for } 1 \leq j \leq x. 
\]
Since $r_{x,x} = R(1) = r(1) \neq 0$ for all $x \geq 1$ and $r_{x,j} = 0$ for all $j > x$, 
the matrix we have defined in this problem is lower triangular with a non-zero 
constant on its diagonals, and is hence invertible. 
If we let $\hat{R} := (R_{x,j})$, then the next matrix is 
expressed by applying an invertible shift operation as 
\[
(r_{x,j}) = \hat{R} \left(I - U^{T}\right). 
\]
Note that the square matrix $U$ of sufficiently large finite dimensions $N \times N$ 
has $(i,j)^{th}$ entries for all $1 \leq i,j \leq N$ that are defined by 
$(U)_{i,j} = \delta_{i+1,j}$ so that 
\[
\left[(I - U^T)^{-1}\right]_{i,j} = \Iverson{j \leq i}. 
\]
We also observe that 
\[
\Floor{x}{j} - \Floor{x-1}{j} = \begin{cases} 
     1, & \text{ if $j|x$; } \\ 
     0, & \text{ otherwise. } 
     \end{cases} 
\] 
The previous equation implies that 
\begin{equation} 
\label{eqn_proof_tag_FloorFuncDiffsOfSummatoryFuncs_v2} 
R\left(\floor{\frac{x}{j}}\right) - R\left(\floor{\frac{x-1}{j}}\right) = 
     \begin{cases} 
     r\left(\frac{x}{j}\right), & \text{ if $j | x$; } \\ 
     0, & \text{ otherwise. } 
     \end{cases}
\end{equation} 
We use the property in \eqref{eqn_proof_tag_FloorFuncDiffsOfSummatoryFuncs_v2} 
to shift the matrix $\hat{R}$, and then invert the result to obtain a matrix involving the 
Dirichlet inverse of $r$ as 
\begin{align*} 
\left(\left(I-U^{T}\right) \hat{R}\right)^{-1} & = 
     \left(r\left(\frac{x}{j}\right) \Iverson{j|x}\right)^{-1} = 
     \left(r^{-1}\left(\frac{x}{j}\right) \Iverson{j|x}\right). 
\end{align*} 
In particular, our target matrix in the inversion problem is defined by 
$$(r_{x,j}) = \left(I-U^{T}\right) \left(r\left(\frac{x}{j}\right) \Iverson{j|x}\right) \left(I-U^{T}\right)^{-1}.$$
We can express its inverse by a similarity transformation conjugated by shift operators in the form of 
\begin{align*} 
(r_{x,j})^{-1} & = \left(I-U^{T}\right)^{-1} \left(r^{-1}\left(\frac{x}{j}\right) 
     \Iverson{j|x}\right) \left(I-U^{T}\right) \\ 
     & = \left(\sum_{k=1}^{\floor{\frac{x}{j}}} r^{-1}(k)\right) (I-U^{T}) \\ 
     & = \left(\sum_{k=1}^{\floor{\frac{x}{j}}} r^{-1}(k) - \sum_{k=1}^{\floor{\frac{x}{j+1}}} r^{-1}(k)\right). 
\end{align*} 
Hence, the summatory function $H(x)$ is given exactly for any integers $x \geq 1$ 
by a vector product with the inverse matrix from the previous equation by 
\begin{align*} 
H(x) & = \sum_{k=1}^x \left(\sum_{j=\floor{\frac{x}{k+1}}+1}^{\floor{\frac{x}{k}}} r^{-1}(j)\right) 
     \times \pi_{r \ast h}(k). 
\end{align*} 
We can prove a second inversion formula providing the coefficients of the summatory function 
$R^{-1}(j)$ for $1 \leq j \leq x$ from the last equation by adapting our argument to prove 
\eqref{eqn_proof_tag_PigAsthx_ExactSummationFormula_exp_v2} above. 
This leads to the following alternate identity expressing $H(x)$: 
\[
H(x) = \sum_{k=1}^{x} r^{-1}(k) \times \pi_{r \ast h}\left(\Floor{x}{k}\right). 
     \qedhere 
\]
\end{proof} 

\subsection{Proving the characteristic signedness property of $g^{-1}(n)$} 

Let $\chi_{\mathbb{P}}(n)$ denote the characteristic function of the primes, let 
$\varepsilon(n) = \delta_{n,1}$ be the multiplicative identity with respect to Dirichlet convolution, 
and denote by $\omega(n)$ the strongly additive function that counts the number of 
distinct prime factors of $n$ (without multiplicity). We can see using 
elementary methods that 
\begin{equation}
\label{eqn_AntiqueDivisorSumIdent} 
\chi_{\mathbb{P}} + \varepsilon = (\omega + 1) \ast \mu. 
\end{equation} 
Namely, since $\mu \ast 1 = \varepsilon$ and 
\[
\omega(n) = \sum_{p|n} 1 = \sum_{d|n} \chi_{\mathbb{P}}(d), \text{ for } n \geq 1, 
\]
the result in \eqref{eqn_AntiqueDivisorSumIdent} follows by M\"obius inversion. 
When combined with Corollary \ref{cor_CvlGAstMu}, 
this convolution identity yields the key exact 
formula for $M(x)$ stated in \eqref{eqn_Mx_gInvnPixk_formula} of 
Corollary \ref{cor_Mx_gInvnPixk_formula}. 

\begin{prop}[The signedness of $g^{-1}(n)$]
\label{prop_SignageDirInvsOfPosBddArithmeticFuncs_v1} 
Let the operator 
$\operatorname{sgn}(h(n)) = \frac{h(n)}{|h(n)| + \Iverson{h(n) = 0}} \in \{0, \pm 1\}$ 
denote the signedness of the arithmetic function $h$ at any $n \geq 1$. 
For the Dirichlet invertible function $g(n) := \omega(n) + 1$, 
we have that $\operatorname{sgn}(g^{-1}(n)) = \lambda(n)$ for all $n \geq 1$. 
\end{prop} 
\begin{proof} 
The function $D_f(s) := \sum_{n \geq 1} f(n) n^{-s}$ defines the 
Dirichlet generating function (DGF) of any 
arithmetic function $f$ which is convergent for all $s \in \mathbb{C}$ satisfying 
$\Re(s) > \sigma_f$ where $\sigma_f$ is the abscissa of convergence of the series. 
Recall that $D_1(s) = \zeta(s)$, $D_{\mu}(s) = \zeta(s)^{-1}$ and 
$D_{\omega}(s) = P(s) \zeta(s)$ for $\Re(s) > 1$. 
Then by \eqref{eqn_AntiqueDivisorSumIdent} and the fact that whenever $f(1) \neq 0$, 
the DGF of $f^{-1}(n)$ is $D_f(s)^{-1}$, we have that 
\begin{align} 
\label{eqn_DGF_of_gInvn} 
D_{(\omega+1)^{-1}}(s) = \frac{1}{\zeta(s) (1+P(s))}, \Re(s) > 1. 
\end{align} 
It follows that $(\omega + 1)^{-1}(n) = (h^{-1} \ast \mu)(n)$ when we take 
$h := \chi_{\mathbb{P}} + \varepsilon$. 
We first show that $\operatorname{sgn}(h^{-1}) = \lambda$. 
We see that this observation implies 
$\operatorname{sgn}(h^{-1} \ast \mu) = \lambda$. 

First, by a combinatorial argument related to multinomial coefficient expansions of these sums, 
we recover exactly that \cite[\cf \S 2]{FROBERG-1968} 
\begin{equation} 
\label{eqn_proof_tag_hInvn_ExactNestedSumFormula_CombInterpetIdent_v3} 
h^{-1}(n) = \begin{cases} 
     1, & n = 1; \\ 
     \lambda(n) (\Omega(n))! \times \prod\limits_{p^{\alpha} || n} \frac{1}{\alpha!}, & n \geq 2. 
     \end{cases}
\end{equation} 
In particular, by expanding the DGF of $h^{-1}$ in powers of $P(s)$
we count that 
\begin{align*}
\frac{1}{1+P(s)} & = \sum_{n \geq 1} \frac{h^{-1}(n)}{n^s} = \sum_{k \geq 0} (-1)^k P(s)^k \\ 
     & = 
     \sum_{\substack{n \geq 1 \\ n =p_1^{\alpha_1}p_2^{\alpha_2} \times \cdots \times p_k^{\alpha_k}}} 
     \frac{(-1)^{\alpha_1+\alpha_2+\cdots+\alpha_k}}{n^s} \times 
     \binom{\alpha_1+\alpha_2+\cdots+\alpha_k}{\alpha_1,\alpha_2,\ldots,\alpha_k} = 
     \sum_{\substack{n \geq 1 \\ n =p_1^{\alpha_1}p_2^{\alpha_2} \times \cdots \times p_k^{\alpha_k}}} 
     \frac{\lambda(n)}{n^s} \times \binom{\Omega(n)}{\alpha_1,\alpha_2,\ldots,\alpha_k}. 
\end{align*}
Since $\lambda$ is completely multiplicative we have that 
$\lambda\left(\frac{n}{d}\right) \lambda(d) = \lambda(n)$ for all divisors 
$d|n$ when $n \geq 1$. We also know that $\mu(n) = \lambda(n)$ whenever $n$ is squarefree, 
so that we obtain the following results: 
\[
g^{-1}(n) = (h^{-1} \ast \mu)(n) = \lambda(n) \times \sum_{d|n} \mu^2\left(\frac{n}{d}\right) |h^{-1}(n)|, n \geq 1. 
     \qedhere 
\]
\end{proof} 

The conclusion of the proof of 
Proposition \ref{prop_SignageDirInvsOfPosBddArithmeticFuncs_v1} 
implies the stronger result that 
\[
g^{-1}(n) = \lambda(n) \times \sum_{d|n} \mu^2\left(\frac{n}{d}\right) C_{\Omega(d)}(d).  
\]
We have adopted the notation that for $n \geq 2$, 
$C_{\Omega(n)}(n) = (\Omega(n))! \times \prod_{p^{\alpha} || n} (\alpha!)^{-1}$, 
where the same function, $C_0(n)$, is taken to be one for $n := 1$ 
(see Section \ref{Section_InvFunc_PreciseExpsAndAsymptotics}). 
We see that the scaled functions $f_1(n) := \frac{C_{\Omega(n)}(n)}{(\Omega(n))!}$ and 
$f_2(n) := \frac{\lambda(n) C_{\Omega(n)}(n)}{(\Omega(n))!}$ are multiplicative. 

\subsection{The distributions of $\omega(n)$ and $\Omega(n)$} 

The next theorems reproduced from \cite[\S 7.4]{MV} characterize the relative 
scarcity of the distributions of $\omega(n)$ and $\Omega(n)$ for $n \leq x$ such that 
$\omega(n),\Omega(n) > \log\log x$. 
Since $\frac{1}{n} \times \sum_{k \leq n} \omega(k) = \log\log n + B_1$ and 
$\frac{1}{n} \times \sum_{k \leq n} \Omega(k) = \log\log n + B_2$ for 
$B_1 \approx 0.261497$ and $B_2 \approx 1.03465$ 
absolute constants in each case, 
these results imply a distinctively regular tendency 
of these additive arithmetic functions towards their respective average orders. 

\begin{theorem}[Upper bounds on exceptional values of $\Omega(n)$ for large $n$] 
\label{theorem_MV_Thm7.20-init_stmt} 
For $x \geq 2$ and $r > 0$, let 
\begin{align*} 
A(x, r) & := \#\left\{n \leq x: \Omega(n) \leq r \log\log x\right\}, \\ 
B(x, r) & := \#\left\{n \leq x: \Omega(n) \geq r \log\log x\right\}. 
\end{align*} 
If $0 < r \leq 1$ and $x \geq 2$, then 
\[
A(x, r) \ll\phantom{_R} x (\log x)^{r-1 - r\log r}, \text{ as } x \rightarrow \infty. 
\]
If $1 \leq r \leq R < 2$ and $x \geq 2$, then 
\[
B(x, r) \ll_R x (\log x)^{r-1-r \log r}, \text{ as } x \rightarrow \infty. 
\]
\end{theorem} 

Theorem \ref{theorem_MV_Thm7.21-init_stmt} is a special case analog to the 
Erd\H{o}s-Kac theorem stated for the 
normally distributed values of 
$\frac{\omega(n) - \log\log n}{\sqrt{\log\log n}}$ over $n \leq x$ as 
$x \rightarrow \infty$ \cite[\cf Thm.\ 7.21]{MV} \cite[\cf \S 1.7]{IWANIEC-KOWALSKI}. 

\begin{theorem}
\label{theorem_MV_Thm7.21-init_stmt} 
We have that as $x \rightarrow \infty$ 
\[
\#\left\{3 \leq n \leq x: \Omega(n) \leq \log\log n \right\} = 
     \frac{x}{2} + O\left(\frac{x}{\sqrt{\log\log x}}\right). 
\]
\end{theorem} 

\begin{theorem}[Montgomery and Vaughan]
\label{theorem_HatPi_ExtInTermsOfGz} 
Recall that for integers $k \geq 1$ and $x \geq 2$ we have defined 
$$\widehat{\pi}_k(x) := \#\{2 \leq n \leq x: \Omega(n)=k\}.$$ 
For $0 < R < 2$ we have uniformly for all $1 \leq k \leq R \log\log x$ that 
\[
\widehat{\pi}_k(x) = \frac{x}{\log x} \times \mathcal{G}\left(\frac{k-1}{\log\log x}\right) 
     \frac{(\log\log x)^{k-1}}{(k-1)!} \left(1 + O_R\left(\frac{k}{(\log\log x)^2}\right)\right), 
\]
where we define 
\[
\mathcal{G}(z) := \frac{1}{\Gamma(1+z)} \times 
     \prod_p \left(1-\frac{z}{p}\right)^{-1} \left(1-\frac{1}{p}\right)^z, 0 \leq |z| < R. 
\]
\end{theorem} 

\begin{remark} 
\label{remark_MV_Pikx_FuncResultsAnnotated_v1} 
We can extend the work in \cite{MV} on the distribution of $\Omega(n)$ to obtain 
corresponding analogs for the distribution of $\omega(n)$. 
For $0 < R < 2$ we have that as $x \rightarrow \infty$ 
\begin{equation}
\label{eqn_Pikx_UniformAsymptoticsStmt_from_MV_v2} 
\pi_k(x) = \frac{x}{\log x} \times 
     \widetilde{\mathcal{G}}\left(\frac{k-1}{\log\log x}\right) 
     \frac{(\log\log x)^{k-1}}{(k-1)!} \left( 
     1 + O_R\left(\frac{k}{(\log\log x)^2}\right) 
     \right), 
\end{equation}
uniformly for any $1 \leq k \leq R\log\log x$. 
The analogous function to express these bounds for $\omega(n)$ is 
defined by $\widetilde{\mathcal{G}}(z) := \widetilde{F}(1, z) \times \Gamma(1+z)^{-1}$ where 
we define 
\[
\widetilde{F}(s, z) := \prod_p \left(1 + \frac{z}{p^s-1}\right) \left(1 - \frac{1}{p^s}\right)^{z}, 
     \Re(s) > \frac{1}{2}; |z| \leq R < 2. 
\]
Let the functions 
\begin{align*} 
C(x, r) & := \#\{n \leq x: \omega(n) \leq r \log\log x\}, \\ 
D(x, r) & := \#\{n \leq x: \omega(n) \geq r \log\log x\}. 
\end{align*} 
Then we have upper bounds given by the following asymptotics as $x \rightarrow \infty$: 
\begin{align*} 
C(x, r) & \ll\phantom{_R} x (\log x)^{r - 1 - r \log r}, \text{ uniformly for } 0 < r \leq 1, \\ 
D(x, r) & \ll_R x (\log x)^{r - 1 - r \log r}, \text{ uniformly for } 1 \leq r \leq R < 2.
\end{align*} 
\end{remark} 

\newpage
\section{Auxiliary sequences related to the inverse function $g^{-1}(n)$} 
\label{Section_InvFunc_PreciseExpsAndAsymptotics} 

The computational data given as Table \ref{table_conjecture_Mertens_ginvSeq_approx_values} 
in the appendix section is intended to 
provide clear insight into the significance of the few characteristic formulas for  
$g^{-1}(n)$ proved in this section. The table provides illustrative 
numerical data by examining the first cases of $1 \leq n \leq 500$ with 
\emph{Mathematica} and \emph{Sage} 
\cite{SCHMIDT-MERTENS-COMPUTATIONS}. 

\subsection{Definitions and properties of triangular component function sequences} 

We define the following sequence for integers $n \geq 1$ and $k \geq 0$: 
\begin{align} 
\label{eqn_CknFuncDef_v2} 
C_k(n) := \begin{cases} 
     \varepsilon(n), & \text{ if $k = 0$; } \\ 
     \sum\limits_{d|n} \omega(d) C_{k-1}\left(\frac{n}{d}\right), & \text{ if $k \geq 1$. } 
     \end{cases} 
\end{align} 
The Dirichlet 
inverse $f^{-1}(n)$ of any arithmetic function $f$ such that $f(1) \neq 0$ is 
computed exactly by an $\Omega(n)$-fold convolution of $f$ with itself. 
The motivation for considering the auxiliary sequence representing the $k$-fold 
Dirichlet convolution of $\omega(n)$ with itself follows from our definition of 
$g^{-1}(n) := (\omega+1)^{-1}(n)$. 
We prove a few precise relations of the function $C_{\Omega(n)}(n)$ to the inverse 
sequence $g^{-1}(n)$ that result in the next subsections. 
Indeed, 
$h^{-1}(n) \equiv \lambda(n) C_{\Omega(n)}(n)$ is the same function given by 
\eqref{eqn_proof_tag_hInvn_ExactNestedSumFormula_CombInterpetIdent_v3} from 
Proposition \ref{prop_SignageDirInvsOfPosBddArithmeticFuncs_v1}. 

By recursively expanding the definition of $C_k(n)$ 
at any fixed $n \geq 2$, we see that 
we can form a chain of at most $\Omega(n)$ iterated (or nested) divisor sums by 
unfolding the definition of \eqref{eqn_CknFuncDef_v2} inductively. 
By the same argument, we see that at fixed $n$, the function 
$C_k(n)$ is non-zero only possibly when 
$1 \leq k \leq \Omega(n)$ whenever $n \geq 2$. 
A sequence of signed semi-diagonals of the functions $C_k(n)$ begins as follows 
\cite[\seqnum{A008480}]{OEIS}: 
\[
\{\lambda(n) C_{\Omega(n)}(n) \}_{n \geq 1} = \{
     1, -1, -1, 1, -1, 2, -1, -1, 1, 2, -1, -3, -1, 2, 2, 1, -1, -3, -1, 
     -3, 2, 2, -1, 4, 1, 2, \ldots \}. 
\]
We see by 
\eqref{eqn_proof_tag_hInvn_ExactNestedSumFormula_CombInterpetIdent_v3} 
that $C_{\Omega(n)}(n) \leq (\Omega(n))!$ for all $n \geq 1$ with 
equality precisely at the squarefree integers so that 
$(\Omega(n))! = (\omega(n))!$. 

\subsection{Formulas relating $C_{\Omega(n)}(n)$ and $g^{-1}(n)$} 
\label{subSection_Relating_CknFuncs_to_gInvn} 

\begin{lemma} 
\label{lemma_AnExactFormulaFor_gInvByMobiusInv_v1} 
For all $n \geq 1$, we have that 
\[
g^{-1}(n) = \sum_{d|n} \mu\left(\frac{n}{d}\right) \lambda(d) C_{\Omega(d)}(d). 
\]
\end{lemma}
\begin{proof} 
We first expand the recurrence relation for the Dirichlet inverse 
when $g^{-1}(1) = g(1)^{-1} = 1$ as 
\begin{align} 
\label{eqn_proof_tag_gInvCvlOne_EQ_omegaCvlgInvCvl_v1} 
g^{-1}(n) & = - \sum_{\substack{d|n \\ d>1}} (\omega(d) + 1) g^{-1}\left(\frac{n}{d}\right) 
     \quad\implies\quad 
     (g^{-1} \ast 1)(n) = -(\omega \ast g^{-1})(n). 
\end{align} 
We argue that for $1 \leq m \leq \Omega(n)$, we can inductively expand the 
implication on the right-hand-side of \eqref{eqn_proof_tag_gInvCvlOne_EQ_omegaCvlgInvCvl_v1} 
in the form of $(g^{-1} \ast 1)(n) = F_m(n)$ where 
$F_m(n) := (-1)^{m} (C_m(-) \ast g^{-1})(n)$, so that 
\[
F_m(n) = - 
     \begin{cases} 
     (\omega \ast g^{-1})(n), & m = 1; \\ 
     \sum\limits_{\substack{d|n \\ d > 1}} F_{m-1}(d) \times \sum\limits_{\substack{r|\frac{n}{d} \\ r > 1}} 
     \omega(r) g^{-1}\left(\frac{n}{dr}\right), & 2 \leq m \leq \Omega(n); \\ 
     0, & \text{otherwise.} 
     \end{cases} 
\]
When $m := \Omega(n)$, e.g., with the expansions 
in the previous equation taken to a maximal depth, we obtain the relation 
\begin{equation} 
\label{eqn_proof_tag_gInvCvlOne_EQ_omegaCvlgInvCvl_v2} 
(g^{-1} \ast 1)(n) = (-1)^{\Omega(n)} C_{\Omega(n)}(n) = \lambda(n) C_{\Omega(n)}(n). 
\end{equation} 
The formula for $g^{-1}(n)$ then follows from 
\eqref{eqn_proof_tag_gInvCvlOne_EQ_omegaCvlgInvCvl_v2} 
by M\"obius inversion. 
\end{proof} 

\begin{cor} 
\label{lemma_AbsValueOf_gInvn_FornSquareFree_v1} 
For all positive integers $n \geq 1$, we have that 
\begin{equation} 
\label{eqn_AbsValueOf_gInvn_FornSquareFree_v1} 
|g^{-1}(n)| = \sum_{d|n} \mu^2\left(\frac{n}{d}\right) C_{\Omega(d)}(d). 
\end{equation} 
\end{cor} 
\begin{proof} 
By applying 
Lemma \ref{lemma_AnExactFormulaFor_gInvByMobiusInv_v1}, 
Proposition \ref{prop_SignageDirInvsOfPosBddArithmeticFuncs_v1} and the 
complete multiplicativity of $\lambda(n)$, 
we easily obtain the stated result. 
In particular, since $\mu(n)$ is non-zero only at squarefree integers and since 
at any squarefree $d \geq 1$ we have $\mu(d) = (-1)^{\omega(d)} = \lambda(d)$, 
Lemma \ref{lemma_AnExactFormulaFor_gInvByMobiusInv_v1} and 
Proposition \ref{prop_SignageDirInvsOfPosBddArithmeticFuncs_v1} imply that  
\begin{align*} 
|g^{-1}(n)| & = \lambda(n) \times \sum_{d|n} \mu\left(\frac{n}{d}\right) \lambda(d) C_{\Omega(d)}(d) \\ 
     %& = \sum_{d|n} \mu^2\left(\frac{n}{d}\right) \lambda\left(\frac{n}{d}\right) 
     %\lambda(nd) C_{\Omega(d)}(d) \\ 
     & = \lambda(n^2) \times \sum_{d|n} \mu^2\left(\frac{n}{d}\right) C_{\Omega(d)}(d). 
\end{align*} 
We see that 
that $\lambda(n^2) = +1$ for all $n \geq 1$ since the number of distinct 
prime factors (counting multiplicity) of any square integer is even. 
\end{proof} 

\begin{remark}
Since $C_{\Omega(n)}(n) = |h^{-1}(n)|$ in the notation from the proof of 
Proposition \ref{prop_SignageDirInvsOfPosBddArithmeticFuncs_v1}, we can see that 
$C_{\Omega(n)}(n) = (\omega(n))!$ for all squarefree $n \geq 1$. 
We also have that whenever $n \geq 1$ is squarefree 
\[
|g^{-1}(n)| = \sum_{d|n} C_{\Omega(d)}(d). 
\]
Since all divisors of a squarefree integer are squarefree, 
a proof of part (B) of Proposition \ref{lemma_gInv_MxExample} 
follows by an elementary counting argument as an immediate consequence 
of the previous equation. 
\end{remark}

\begin{remark} 
Lemma \ref{lemma_AnExactFormulaFor_gInvByMobiusInv_v1} 
shows that the summatory 
function of this sequence satisfies 
\[
G^{-1}(x) = \sum_{d \leq x} \lambda(d) C_{\Omega(d)}(d) M\left(\Floor{x}{d}\right). 
\]
Equation \eqref{eqn_AntiqueDivisorSumIdent} implies that 
$$\lambda(d) C_{\Omega(d)}(d) = (g^{-1} \ast 1)(d) = (\chi_{\mathbb{P}} + \varepsilon)^{-1}(d).$$ 
We recover by inversion that 
\begin{equation}
\label{eqn_RmkInitialConnectionOfMxToGInvx_ProvedByInversion_v1} 
M(x) = G^{-1}(x) + \sum_{p \leq x} G^{-1}\left(\Floor{x}{p}\right), x \geq 1. 
\end{equation}
The proof of Corollary \ref{cor_ExpectationFormulaAbsgInvn_v2} 
shows that 
\[
\sum_{n \leq x} |g^{-1}(n)| = \sum_{d \leq x} C_{\Omega(d)}(d) Q\left(\Floor{x}{d}\right), x \geq 1, 
\]
where $Q(x) := \sum_{n \leq x} \mu^2(n)$ counts the number of squarefree $n \leq x$. 
\end{remark} 

\subsection{Combinatorial connections to the distribution of the primes} 
\label{subSection_AConnectionToDistOfThePrimes} 

The combinatorial properties of $g^{-1}(n)$ are deeply tied to the distribution of the primes 
$p \leq n$ as $n \rightarrow \infty$. 
The magnitudes of and spacings between the primes $p \leq n$ certainly restricts the 
repeating of these distinct sequence values. 
We can see that the following 
is still clear about the relation of the weight functions $|g^{-1}(n)|$ to the 
distribution of the primes: 
The value of $|g^{-1}(n)|$ is entirely dependent only on the pattern of the exponents 
(viewed as multisets) of the distinct prime factors of $n \geq 2$, rather than on the 
prime factor weights themselves 
(\cf Observation \ref{heuristic_SymmetryIngInvFuncs}). 
This property implies that $|g^{-1}(n)|$ has an inherently additive, rather than 
multiplicative, structure underneath the distribution of its distinct values over $n \leq x$. 

\begin{example} 
There is a natural extremal behavior of $|g^{-1}(n)|$ 
with respect to the distinct values of $\Omega(n)$ 
at squarefree integers and prime powers. For integers 
$k \geq 1$ we define the 
infinite sets $\overline{M}_k$ and $\underline{m}_k$ to correspond to the maximal (minimal) sets of 
positive integers such that 
\begin{align*} 
\overline{M}_k & := \left\{n \geq 2: |g^{-1}(n)| = \underset{{\substack{j \geq 2 \\ \Omega(j) = k}}}{\operatorname{sup}} 
     |g^{-1}(j)|\right\} \subseteq \mathbb{Z}^{+}, \\  
\underline{m}_k & := \left\{n \geq 2: |g^{-1}(n)| = \underset{{\substack{j \geq 2 \\ \Omega(j) = k}}}{\operatorname{inf}} 
     |g^{-1}(j)|\right\} \subseteq \mathbb{Z}^{+}. 
\end{align*} 
Any element of $\overline{M}_k$ is squarefree and any element of 
$\underline{m}_k$ is a prime power. 
Moreover, for any fixed $k \geq 1$ 
we have that for any $N_k \in \overline{M}_k$ and $n_k \in \underline{m}_k$
\[
(-1)^{k} g^{-1}(N_k) = \sum_{j=0}^{k} \binom{k}{j} \times j!, 
     \quad \mathrm{\ and\ } \quad 
     (-1)^{k} g^{-1}(n_k) = 2., 
\]
where $\lambda(N_k) = \lambda(n_k) = (-1)^{k}$. 
\end{example}

\begin{remark} 
The formula for the function $h^{-1}(n) = (g^{-1} \ast 1)(n)$ defined in the proof of 
Proposition \ref{prop_SignageDirInvsOfPosBddArithmeticFuncs_v1} shows that we can express 
$g^{-1}(n)$ in terms of symmetric polynomials in the 
exponents of the prime factorization of $n$. 
For $n \geq 2$ and $0 \leq k \leq \omega(n)$ let 
\[
\widehat{e}_k(n) := [z^k] \prod_{p|n} (1 + z \nu_p(n)) = [z^k] \prod_{p^{\alpha} || n} (1 + \alpha z). 
\]
Then we can prove using 
\eqref{eqn_proof_tag_hInvn_ExactNestedSumFormula_CombInterpetIdent_v3} and 
\eqref{eqn_AbsValueOf_gInvn_FornSquareFree_v1} that the following formula holds: 
\[
g^{-1}(n) = h^{-1}(n) \times \sum_{k=0}^{\omega(n)} \binom{\Omega(n)}{k}^{-1} 
     \frac{\widehat{e}_k(n)}{k!}, n \geq 2. 
\]
The combinatorial formula for 
$h^{-1}(n) = \lambda(n) (\Omega(n))! \times \prod_{p^{\alpha} || n} (\alpha !)^{-1}$ 
suggests additional patterns and regularity in the contributions of the distinct sign weighted 
terms in the summands of $G^{-1}(x)$\footnote{ 
     This sequence is also considered using a different motivation based on the DGFs 
     $(1\pm P(s))^{-1}$ in \cite[\S 2]{FROBERG-1968}. 
}. 
Sections \ref{subSection_AsymptoticsOfGinvx} and 
\ref{subSection_LocalCancellationOfGInvx} 
discuss limiting asymptotic properties and local cancellation in the formula for $M(x)$ 
from \eqref{eqn_RmkInitialConnectionOfMxToGInvx_ProvedByInversion_v1} 
that is expanded exactly through the auxiliary sums $G^{-1}(x)$. 
\end{remark}

\newpage
\section{The distributions of $C_{\Omega(n)}(n)$ and $|g^{-1}(n)|$ and their partial sums} 
\label{Section_NewFormulasForgInvn} 

We observed an intuition in the introduction that the relation of the unsigned auxiliary 
functions, $g^{-1}(n)$ and $C_{\Omega(n)}(n)$, to the canonically additive functions 
$\omega(n)$ and $\Omega(n)$ leads to the regular properties 
illustrated in Table \ref{table_conjecture_Mertens_ginvSeq_approx_values}. 
Each of $\omega(n)$ and $\Omega(n)$ satisfies 
an Erd\H{o}s-Kac theorem that provides a central limiting 
distribution for each of these functions over 
$n \leq x$ as $x \rightarrow \infty$ 
\cite{ERDOS-KAC-REF,BILLINGSLY-CLT-PRIMEDIVFUNC,RENYI-TURAN} 
(\cf \cite{CLT-RANDOM-ORDERED-FACTS-2011}). 
In the remainder of this section, we use analytic methods in the spirit of 
\cite[\S 7.4]{MV} to prove new properties that characterize the distributions of the 
auxiliary functions in analogous ways. 
The probabilistic ansatz given at the start of 
Section \ref{subSection_ErdosKacTheorem_Analogs} 
is reminiscent of preliminaries behind the first proofs of the Erd\H{os}-Kac theorem. 
It is thus suggestive of deeper connections of $C_{\Omega(n)}(n)$, $|g^{-1}(n)|$, 
and classes of functions constructed (and enumerated) through 
similar procedures to strong additivity. 

\subsection{Analytic proofs extending bivariate DGF methods for additive functions} 

\begin{theorem} 
\label{prop_HatAzx_ModSummatoryFuncExps_RelatedToCkn} 
Let the bivariate DGF $\widehat{F}(s, z)$ be defined in terms of the prime zeta function, $P(s)$,  
for $\Re(s) > 1$ and $|z| < |P(s)|^{-1}$ by 
\[
\widehat{F}(s, z) := \frac{1}{1+P(s) z} 
     \times \prod_p \left(1 - \frac{1}{p^s}\right)^{z}. 
\]
The partial sums of the coefficients of 
$\widehat{F}(s, z) \zeta(s)^{z}$ are given by 
\[
\widehat{A}_z(x) := \sum_{n \leq x} (-1)^{\omega(n)} 
     C_{\Omega(n)}(n) z^{\Omega(n)}. 
\]
We have for all sufficiently large $x$ and any $|z|< P(2)^{-1} \approx 2.21118$ that
\[
\widehat{A}_z(x) = \frac{x \widehat{F}(2, z)}{\Gamma(z)} (\log x)^{z-1} + 
     O_{z}\left(x (\log x)^{\Re(z) - 2}\right). 
\]
\end{theorem} 
\begin{proof} 
It follows from \eqref{eqn_proof_tag_hInvn_ExactNestedSumFormula_CombInterpetIdent_v3} that 
we can generate exponentially scaled forms of the function $C_{\Omega(n)}(n)$ by 
product identity of the following form: 
\begin{align*} 
\sum_{n \geq 1} \frac{C_{\Omega(n)}(n)}{(\Omega(n))!} \cdot 
     \frac{(-1)^{\omega(n)} z^{\Omega(n)}}{n^s} & = \prod_p \left(1 + \sum_{r \geq 1} 
     \frac{z^{\Omega(p^r)}}{r! p^{rs}}\right)^{-1} 
     = \exp\left(-z P(s)\right), \text{ for } \Re(s) > 1 \wedge \Re(P(s)z) > -1. 
\end{align*} 
This Euler type product expansion is similar in construction to the parameterized bivariate 
DGFs in \cite[\S 7.4]{MV}.
By computing a termwise Laplace transform applied to the right-hand-side of the 
above equation, we obtain that 
\begin{align*} 
\sum_{n \geq 1} \frac{C_{\Omega(n)}(n) (-1)^{\omega(n)} z^{\Omega(n)}}{n^s} & = 
     \int_0^{\infty} e^{-t} \exp\left(-tz P(s)\right) dt = \frac{1}{1 + P(s) z}, 
     \text{ for } \Re(s) > 1 \wedge \Re(P(s)z) > -1. 
\end{align*} 
It follows from the Euler product representation of $\zeta(s)$ which holds for any 
$\Re(s) > 1$ that 
\[
\sum_{n \geq 1} \frac{(-1)^{\omega(n)} C_{\Omega(n)}(n) z^{\Omega(n)}}{n^s} = 
     \widehat{F}(s, z) \zeta(s)^{z}, \text{ for } \Re(s) > 1 \wedge |z| < |P(s)|^{-1}. 
\]
The bivariate DGF $\widehat{F}(s, z)$ is an analytic function of $s$ for all $\Re(s) > 1$ 
whenever the parameter $|z| < |P(s)|^{-1}$. 
If the sequence $\{b_z(n)\}_{n \geq 1}$ indexes the coefficients in 
the DGF expansion of $\widehat{F}(s, z) \zeta(s)^{z}$, then the series 
\[
\left\lvert \sum_{n \geq 1} \frac{b_z(n) (\log n)^{2R+1}}{n^s} \right\rvert < +\infty. 
\]
Moreover, the series in the last equation is uniformly bounded for all $\Re(s) \geq 2$ and 
$|z| \leq R < |P(s)|^{-1}$. This fact follows by repeated 
termwise differentiation of the series for the original function 
$\ceiling{2R+1}$ times with respect to $s$. 

For fixed $0 < |z| < 2$, let the sequence $\{d_z(n)\}_{n \geq 1}$ be generated as the coefficients of the DGF 
$$\zeta(s)^{z} = \sum_{n \geq 1} \frac{d_z(n)}{n^s}, \text{ for } \Re(s) > 1.$$ The corresponding 
summatory function of $d_z(n)$ is defined by $D_z(x) := \sum\limits_{n \leq x} d_z(n)$. 
The theorem proved in 
\cite[Thm.\ 7.17; \S 7.4]{MV} shows that for any $0 < |z| < 2$ 
and all integers $x \geq 2$ we have 
\[
D_z(x) = \frac{x (\log x)^{z-1}}{\Gamma(z)} + O_z\left(x (\log x)^{\Re(z)-2}\right). 
\]
Set 
$b_z(n) := (-1)^{\omega(n)} C_{\Omega(n)}(n) z^{\Omega(n)}$, define the convolution 
$a_z(n) := \sum\limits_{d|n} b_z(d) d_z\left(\frac{n}{d}\right)$, 
and take its partial sums to be  
$A_z(x) := \sum\limits_{n \leq x} a_z(n)$. 
Then we have that 
\begin{align} 
\notag 
A_z(x) & = \sum_{m \leq \frac{x}{2}} b_z(m) D_z\left(\frac{x}{m}\right) + 
     \sum_{\frac{x}{2} < m \leq x} b_z(m) \\ 
\label{eqn_proof_tag_Azx_FullTermsFormulaSum_v1} 
     & = \frac{x}{\Gamma(z)} \times \sum_{m \leq \frac{x}{2}} 
     \frac{b_z(m)}{m} \log\left(\frac{x}{m}\right)^{z-1} + 
     O\left(\sum_{m \leq x} \frac{x |b_z(m)|}{m} \times
     \log\left(\frac{2x}{m}\right)^{\Re(z) - 2}\right). 
\end{align} 
We can sum the coefficients $\frac{b_z(m)}{m}$ 
for integers $m \leq u$ when $u$ is taken sufficiently large as 
\begin{align*} 
\sum_{m \leq u} \frac{b_z(m)}{m^2} \times m & = \left(\widehat{F}(2, z) + 
     O_z\left(u^{-2}\right)\right) u - \int_1^{u} 
     \left(\widehat{F}(2, z) + O_z\left(t^{-2}\right)\right) dt 
     = \widehat{F}(2, z) + O_z\left(u^{-1}\right). 
\end{align*} 
Suppose that $0 < |z| \leq R < P(2)^{-1}$. 
For large $x$, the error term in \eqref{eqn_proof_tag_Azx_FullTermsFormulaSum_v1} satisfies 
\begin{align*} 
\sum_{m \leq x} \frac{x |b_z(m)|}{m} 
     \log\left(\frac{2x}{m}\right)^{\Re(z) - 2} & \ll 
     x (\log x)^{\Re(z) - 2} \times \sum_{m \leq \sqrt{x}} \frac{|b_z(m)|}{m} \\ 
     & \phantom{\ll x\ } + 
     x (\log x)^{-(R+2)} \times \sum_{m > \sqrt{x}} \frac{|b_z(m)|}{m} (\log m)^{2R} \\ 
     & = O_z\left(x (\log x)^{\Re(z) - 2}\right), 
\end{align*} 
whenever $0 < |z| \leq R$. 
When $m \leq \sqrt{x}$ we have 
\[
\log\left(\frac{x}{m}\right)^{z-1} = (\log x)^{z-1} + 
     O\left((\log m) (\log x)^{\Re(z) - 2}\right). 
\]
A related upper bound is obtained for the left-hand-side of the previous equation when 
$\sqrt{x} < m < x$ and $0 < |z| < R$. 
The combined sum over the interval $m \leq \frac{x}{2}$ corresponds to bounding the 
sum components when we take $0 < |z| \leq R$ by 
\begin{align*} 
\sum_{m \leq \frac{x}{2}} b_z(m) D_z\left(\frac{x}{m}\right) & = \frac{x}{\Gamma(z)} (\log x)^{z-1} \times 
     \sum_{m \leq \frac{x}{2}} \frac{b_z(m)}{m} \\ 
     & \phantom{=\quad\ } + 
     O_R\left(x (\log x)^{\Re(z)-2} \times \sum_{m \leq \sqrt{x}} \frac{|b_z(m)| \log m}{m} + 
     x (\log x)^{R-1} \times \sum_{m > \sqrt{x}} \frac{|b_z(m)|}{m}\right) \\ 
     & = \frac{x \widehat{F}(2, z)}{\Gamma(z)} (\log x)^{z-1} + O_R\left( 
     x (\log x)^{\Re(z)-2} \times \sum_{m \geq 1} \frac{b_z(m) (\log m)^{2R+1}}{m^2} 
     \right) \\ 
     & = \frac{x \widehat{F}(2, z)}{\Gamma(z)} (\log x)^{z-1} + O_{R}\left( 
     x (\log x)^{\Re(z)-2}\right). 
     \qedhere  
\end{align*} 
\end{proof} 

\begin{theorem} 
\label{theorem_CnkSpCasesScaledSummatoryFuncs} 
For all large $x \geq 3$ and integers $k \geq 1$, let 
\[
\widehat{C}_{k,\ast}(x) := \sum_{\substack{n \leq x \\ \Omega(n) = k}} 
     (-1)^{\omega(n)} C_k(n) 
\]
Let $\widehat{G}(z) := \widehat{F}(2, z) \times \Gamma(1+z)^{-1}$ when 
$0 \leq |z| < P(2)^{-1}$ where $\widehat{F}(s, z)$ is defined as 
in Theorem \ref{prop_HatAzx_ModSummatoryFuncExps_RelatedToCkn}. 
As $x \rightarrow \infty$, we have uniformly for any $1 \leq k \leq 2\log\log x$ that 
\[
\widehat{C}_{k,\ast}(x) = -\widehat{G}\left(\frac{k-1}{\log\log x}\right) \frac{x}{\log x} \cdot 
     \frac{(\log\log x)^{k-1}}{(k-1)!} \left( 
     1 + O\left(\frac{k}{(\log\log x)^2}\right)\right). 
\]
\end{theorem} 
\begin{proof} 
When $k = 1$, we have that $\Omega(n) = \omega(n)$ for all $n \leq x$ such that $\Omega(n) = k$. 
The positive integers $n$ that satisfy this requirement are precisely the primes $p \leq x$. 
Hence, the formula is satisfied as 
\[
\sum_{p \leq x} (-1)^{\omega(p)} C_1(p) = -\sum_{p \leq x} 1 = 
     - \frac{x}{\log x} \left(1 + O\left(\frac{1}{\log x}\right)\right). 
\]
Since $O\left((\log x)^{-1}\right) = O\left((\log\log x)^{-2}\right)$ as 
$x \rightarrow \infty$, we obtain the required error term bound at $k = 1$. 

For $2 \leq k \leq 2\log\log x$, we will apply the error estimate from 
Theorem \ref{prop_HatAzx_ModSummatoryFuncExps_RelatedToCkn} with 
$r := \frac{k-1}{\log\log x}$ in the formula 
\[
\widehat{C}_{k,\ast}(x) = \frac{(-1)^{k+1}}{2\pi\imath} \times \int_{|v|=r} 
     \frac{\widehat{A}_{-v}(x)}{v^{k+1}} dv. 
\]
Since $(\log x)^{\frac{1}{\log\log x}} = e$, the error in the formula 
contributes terms that are bounded by 
\begin{align*} 
\left\lvert x (\log x)^{-(\Re(v)+2)} v^{-(k+1)} \right\rvert & \ll 
     \left\lvert x (\log x)^{-(r+2)} r^{-(k+1)} \right\rvert 
     \ll \frac{x}{(\log x)^{2-\frac{k-1}{\log\log x}}} \cdot 
     \frac{(\log\log x)^{k}}{(k-1)^{k}} \\ 
     & \ll \frac{x}{(\log x)^2} \cdot \frac{(\log\log x)^{k}}{(k-1)^{\frac{1}{2}} (k-1)!} 
     \ll \frac{x}{\log x} \cdot \frac{k (\log\log x)^{k-5}}{(k-1)!}, 
     \text{ as } x \rightarrow \infty. 
\end{align*} 
We next find the main term for the coefficients 
of the following contour integral when 
$r \in [0, z_{\max}] \subseteq \left[0, P(2)^{-1}\right)$: 
\begin{align} 
\label{eqn_WideTildeArx_CountourIntDef_v1} 
\widehat{C}_{k,\ast}(x) \sim  
     \frac{(-1)^{k} x}{\log x} 
     \times \int_{|v|=r} \frac{(\log x)^{-v} \zeta(2)^{v}}{\Gamma(1 - v) 
     v^{k} (1 - P(2) v)} dv. 
\end{align} 
The main term of $\widehat{C}_{k,\ast}(x)$ 
is given by $-\frac{x}{\log x} \times I_k(r, x)$, where we define 
\begin{align*}
I_k(r, x) & = \frac{1}{2\pi\imath} \times \int_{|v|=r} 
     \frac{\widehat{G}(v) (\log x)^{v}}{v^k} dv \\ 
     & =: I_{1,k}(r, x) + I_{2,k}(r, x). 
\end{align*}
Taking $r = \frac{k-1}{\log\log x}$, the 
first of the component integrals is defined to be 
\begin{align*}
I_{1,k}(r, x) & := \frac{\widehat{G}(r)}{2\pi\imath} \times \int_{|v|=r} 
     \frac{(\log x)^{v}}{v^k} dv = \widehat{G}(r) \times \frac{(\log\log x)^{k-1}}{(k-1)!}. 
\end{align*}
The second integral, $I_{2,k}(r, x)$, corresponds to an error term in our approximation. 
This component function is defined by 
\[
I_{2,k}(r, x) := \frac{1}{2\pi\imath} \times \int_{|v|=r} 
     \left(\widehat{G}(v) - \widehat{G}(r)\right) 
     \frac{(\log x)^{v}}{v^k} dv. 
\]
Integrating by parts shows that \cite[\cf Thm.\ 7.19; \S 7.4]{MV} 
\[
\frac{(r-v)}{2\pi\imath} \times \int_{|v|=r} (\log x)^v v^{-k} dv = 0, 
\]
so that integrating by parts once again we have 
\[
I_{2,k}(r, x) := \frac{1}{2\pi\imath} \times \int_{|v|=r} 
     \left(\widehat{G}(v) - \widehat{G}(r) - 
     \widehat{G}^{\prime}(r)(v-r)\right) 
     (\log x)^{v} v^{-k} dv. 
\]
We find that 
\[
\widehat{G}(v) - \widehat{G}(r) - \widehat{G}^{\prime}(r)(v-r) = 
     \int_{r}^{v} (v-w) \widehat{G}^{\prime\prime}(w) dw 
     \ll |v-r|^2. 
\]
With the parameterization $v = re^{2\pi\imath\theta}$ for 
$\theta \in \left[-\frac{1}{2}, \frac{1}{2}\right]$ and 
selecting $r := \frac{k-1}{\log\log x}$, we obtain 
\[
|I_{2,k}(r, x)| \ll r^{3-k} \times 
     \int_{-\frac{1}{2}}^{\frac{1}{2}} (\sin \pi\theta)^2 e^{(k-1) \cos(2\pi\theta)} d\theta. 
\]
Since $|\sin x| \leq |x|$ for all $|x| < 1$ and $\cos(2\pi\theta) \leq 1 - 8\theta^2$ if 
$-\frac{1}{2} \leq \theta \leq \frac{1}{2}$, we arrive at the next bounds by again taking 
setting $r = \frac{k-1}{\log\log x}$ at any $1 \leq k \leq 2\log\log x$. 
\begin{align*}
|I_{2,k}(r, x)| & \ll r^{3-k} e^{k-1} \times \int_0^{\infty} \theta^2 e^{-8(k-1) \theta^2} d\theta \\ 
     & \ll \frac{r^{3-k} e^{k-1}}{(k-1)^{\frac{3}{2}}} = 
     \frac{(\log\log x)^{k-3} e^{k-1}}{(k-1)^{k-\frac{3}{2}}} 
     \ll 
     \frac{k (\log\log x)^{k-3}}{(k-1)!}. 
\end{align*}
Finally, whenever $1 \leq k \leq 2\log\log x$ we have 
\[
1 = \widehat{G}(0) \geq \widehat{G}\left(\frac{k-1}{\log\log x}\right) = 
     \frac{1}{\Gamma\left(1+\frac{k-1}{\log\log x}\right)} \times 
     \frac{\zeta(2)^{\frac{1-k}{\log\log x}}}{\left(1+\frac{P(2)(k-1)}{\log\log x}\right)} 
     \geq \widehat{G}(2) \approx 0.097027. 
\]
In particular, the function 
$\widehat{G}\left(\frac{k-1}{\log\log x}\right) \gg 1$ for 
all $1 \leq k \leq 2\log\log x$. 
This implies the result of the theorem. 
\end{proof} 

\begin{lemma} 
\label{cor_AsymptoticsForSignedSumsOfomegan_v1}
As $x \rightarrow \infty$, there is an absolute constant $A_0 > 0$ such that 
\[
\sum_{n \leq x} (-1)^{\omega(n)} = 
     \frac{(-1)^{\floor{\log\log x}} x}{A_0 \sqrt{2\pi \log\log x}} + 
     O\left(\frac{x}{\log\log x}\right). 
\]
\end{lemma}
\begin{proof}
An adaptation of the proof of Lemma \ref{lemma_ConvenientIncGammaFuncTypePartialSumAsymptotics_v2} 
from the appendix provides that for any $a \in (1, 1.76322)$ 
\begin{align}
\notag 
S_a(x) := 
     \frac{x}{\log x} & \times \left\lvert \sum_{k=1}^{\floor{a\log\log x}} \frac{(-1)^{k} (\log\log x)^{k-1}}{(k-1)!} 
     \right\rvert \\ 
\label{eqn_ConvenientIncGammaFuncTypePartialSumAsymptotics_va3} 
     & = \frac{\sqrt{a} x}{\sqrt{2\pi}(a+1) a^{\{a\log\log x\}}} 
     \times \frac{(\log x)^{a-1-a\log a}}{\sqrt{\log\log x}} 
     \left(1 + O\left(\frac{1}{\log\log x}\right)\right). 
\end{align}
Here, we define $\{x\} = x - \floor{x} \in [0, 1)$ to be the \emph{fractional part} of $x$. 
Suppose that we take $a := \frac{3}{2}$ so that $a-1-a\log a \approx -0.108198$. 
We define and expand the next partial sums as 
\begin{align*}
L_{\ast\ast}(x) := & \sum_{n \leq x} (-1)^{\omega(n)} = 
     \sum_{k \leq \log\log x} 2 (-1)^{k} \pi_k(x) + S_{\frac{3}{2}}(x) + 
     O\left( 
     \#\left\{n \leq x: \omega(n) \geq \frac{3}{2}\log\log x\right\}\right). 
\end{align*} 
We can show that for any $1 < k \leq \log\log x$, 
the function $\widetilde{\mathcal{G}}\left(\frac{k-1}{\log\log x}\right)$ from 
Remark \ref{remark_MV_Pikx_FuncResultsAnnotated_v1} is decreasing in $k$ with 
$\widetilde{\mathcal{G}}(0) = 1$ and satisfies 
\[ 
\widetilde{\mathcal{G}}\left(\frac{k-1}{\log\log x}\right) \geq 
     \widetilde{\mathcal{G}}\left(1-\frac{1}{\log\log x}\right) \geq 
     \widetilde{\mathcal{G}}(1) = 1. 
\]
We apply the uniform asymptotics for $\pi_k(x)$ that hold as $x \rightarrow \infty$ when 
$1 \leq k \leq R \log\log x$ for $1 \leq R < 2$. We then see 
by Lemma \ref{lemma_ConvenientIncGammaFuncTypePartialSumAsymptotics_v2} 
and \eqref{eqn_ConvenientIncGammaFuncTypePartialSumAsymptotics_va3} that at sufficiently large $x$ 
there is some absolute constant $A_0 > 0$ such that 
\begin{align*}
L_{\ast\ast}(x) & = \frac{(-1)^{\floor{\log\log x}} x}{A_0 \sqrt{2\pi \log\log x}} + 
     O\left(E_{\omega}(x) + 
     \frac{x}{(\log x)^{0.108198} \sqrt{\log\log x}} + 
     \#\left\{n \leq x: \omega(x) \geq \frac{3}{2}\log\log x\right\}\right). 
\end{align*} 
The error term in the previous equation 
is bounded by the next sum as $x \rightarrow \infty$. 
In particular, the following estimate is obtained from Stirling's formula, 
\eqref{eqn_IncompleteGamma_PropA} and 
\eqref{eqn_IncompleteGamma_PropC} from the appendix: 
\begin{align*} 
E_{\omega}(x) & \ll \frac{x}{\log x} \times 
     \sum_{1 \leq k \leq \log\log x} \frac{(\log\log x)^{k-2}}{(k-1)!} \\ 
     & = 
     \frac{x \Gamma(\log\log x, \log\log x)}{\Gamma(\log\log x + 1)} 
     \sim \frac{x}{2\log\log x} \left(1 + O\left(\frac{1}{\sqrt{\log\log x}}\right)\right). 
\end{align*}
By an application of the second set of results in 
Remark \ref{remark_MV_Pikx_FuncResultsAnnotated_v1}, we see that 
\[
\#\left\{n \leq x: \omega(x) \geq \frac{3}{2}\log\log x\right\} \ll 
     \frac{x}{(\log x)^{0.108198}}. 
     \qedhere 
\]
\end{proof}

\begin{cor} 
\label{cor_SummatoryFuncsOfUnsignedSeqs_v2} 
We have at all sufficiently large $x$ uniformly for 
$1 \leq k \leq \frac{3}{2} \log\log x$ that 
\begin{align*} 
\widehat{C}_k(x) := 
     \sum_{\substack{n \leq x \\ \Omega(n) = k}} C_{\Omega(n)}(n) & 
     = A_0 \sqrt{2\pi} x 
     \widehat{G}\left(\frac{k-1}{\log\log x}\right) 
     \frac{(\log\log x)^{k-\frac{1}{2}}}{(k-1)!} \left( 
     1 + O\left(\frac{1}{\log\log x}\right)\right). 
\end{align*} 
\end{cor} 
\begin{proof} 
Suppose that $h(t)$ and $\sum_{n \leq t} \lambda_{\ast}(n)$ are 
piecewise smooth and differentiable functions on $\mathbb{R}^{+}$. 
The next integral formulas result by 
Abel summation and integration by parts. 
\begin{subequations}
\begin{align} 
\label{eqn_AbelSummationIBPReverseFormula_stmt_v1} 
     \sum_{n \leq x} \lambda_{\ast}(n) h(n) & = \left(\sum_{n \leq x} \lambda_{\ast}(n)\right) h(x) - 
     \int_{1}^{x} \left(\sum_{n \leq t} \lambda_{\ast}(n)\right) h^{\prime}(t) dt \\ 
\label{eqn_AbelSummationIBPReverseFormula_stmt_v2}
     & \sim 
     \int_1^{x} \frac{d}{dt}\left[\sum_{n \leq t} \lambda_{\ast}(n)\right] h(t) dt
\end{align} 
\end{subequations}
We transform our previous results for the partial sums of 
$(-1)^{\omega(n)} C_{\Omega(n)}(n)$ such that $\Omega(n) = k$ to approximate 
the corresponding partial sums of only $C_{\Omega(n)}(n)$. 
In particular, since $1 \leq k \leq \frac{3}{2} \log\log x$, we have that 
\[
\widehat{C}_{k,\ast}(x) = 
     \sum_{\substack{n \leq x \\ \Omega(n)=k}} (-1)^{\omega(n)} C_{\Omega(n)}(n) = 
     \sum_{n \leq x} (-1)^{\omega(n)} \Iverson{\omega(n) \leq \frac{3}{2} \log\log x} \times 
     C_{\Omega(n)}(n) \Iverson{\Omega(n) = k}. 
\]
We have by the proof of Lemma \ref{cor_AsymptoticsForSignedSumsOfomegan_v1} 
that as $t \rightarrow \infty$ 
\begin{align} 
\label{eqn_ProofTag_LAsttSummatoryFuncAsymptotics_v1}
L_{\ast}(t) & := \sum_{\substack{n \leq t \\ \omega(n) \leq \frac{3}{2} \log\log t}} 
     (-1)^{\omega(n)} 
     = \frac{(-1)^{\floor{\log\log t}} t}{A_0 \sqrt{2\pi \log\log t}}\left(1 + 
     O\left(\frac{1}{\sqrt{\log\log t}}\right)\right). 
\end{align} 
Except for $t$ within a subset of $(0, \infty)$ of measure zero on which 
$L_{\ast}(t)$ changes sign, the main term of the derivative of this summatory function 
is given almost everywhere by 
\[
L_{\ast}^{\prime}(t) \sim \frac{(-1)^{\floor{\log\log t}}}{A_0 \sqrt{2\pi \log\log t}}. 
\]
We apply the formula from \eqref{eqn_AbelSummationIBPReverseFormula_stmt_v2},  
to deduce that as $x \rightarrow \infty$ with $1 \leq k \leq \frac{3}{2} \log\log x$ 
\begin{align*} 
     \widehat{C}_{k,\ast}(x) & \sim 
     \sum_{j=1}^{\log\log x-1} \frac{2 \cdot (-1)^{j+1}}{A_0\sqrt{2\pi}} \times \int_{e^{e^j}}^{e^{e^{j+1}}} 
     \frac{C_{\Omega(t)}(t) \Iverson{\Omega(t) = k}}{\sqrt{\log\log t}} dt \\ 
     & \sim -\int_1^{\frac{\log\log x}{2}} \int_{e^{e^{2s-1}}}^{e^{e^{2s}}} 
     \frac{2 C_{\Omega(t)}(t) \Iverson{\Omega(t) = k}}{A_0 \sqrt{2\pi \log\log t}} dt ds +
     \frac{1}{A_0 \sqrt{2\pi}} \times \int_{e^e}^x 
     \frac{C_{\Omega(t)}(t) \Iverson{\Omega(t) = k}}{\sqrt{\log\log t}} dt. 
\end{align*} 
For large $x$, $(\log\log t)^{-\frac{1}{2}}$ is continuous and monotone decreasing on 
$\left[x^{e^{-1}}, x\right]$ with 
\[
\frac{1}{\sqrt{\log\log x}} - \frac{1}{\sqrt{\log\log\left(x^{e^{-1}}\right)}} = 
     O\left(\frac{1}{(\log x) \sqrt{\log\log x}}\right), 
\]
Hence, we have that 
\begin{equation} 
\label{eqn_ProofTag_HatCkx_Asymptotics_v1_v0}
     -A_0 \sqrt{2\pi} x (\log x) \sqrt{\log\log x} \widehat{C}_{k,\ast}^{\prime}(x) = 
     \left(\widehat{C}_k(x) - \widehat{C}_k\left(x^{e^{-1}}\right)\right)(1+o(1)) - 
     x (\log x) \widehat{C}_k^{\prime}(x). 
\end{equation} 
For $1 \leq k < \frac{3}{2} \log\log x$, we expect contributions from the squarefree integers $n \leq x$ 
such that $\omega(n) = \Omega(n) = k$ to be on the order of 
\[
\widehat{C}_k^{\prime}(x) \asymp \frac{6}{\pi^2} \times k! \times \widehat{\pi}_k(x) \sim 
     \frac{6xk}{\pi^2} \times \frac{(\log\log x)^{k-1}}{\log x}. 
\]
We conclude that 
$\widehat{C}_k\left(x^{e^{-1}}\right) = o\left(\widehat{C}_k(x)\right)$. 
Then equation \eqref{eqn_ProofTag_HatCkx_Asymptotics_v1_v0} 
becomes an ordinary differential equation for $\widehat{C}_k(x)$ under this 
observation. Its solution has the form 
\[
\widehat{C}_k(x) = A_0\sqrt{2\pi}(\log x) \times \int_3^x 
     \frac{\sqrt{\log\log t}}{\log t} \widehat{C}_{k,\ast}^{\prime}(t) dt + 
     O(\log x). 
\]
When we integrate by parts and apply the result from 
Theorem \ref{theorem_CnkSpCasesScaledSummatoryFuncs}, we find that 
\begin{align*}
\widehat{C}_k(x) & = \frac{\sqrt{\log\log x}}{\log x} \widehat{C}_{k,\ast}(x) + 
     O\left(x \times \int_3^x \frac{\sqrt{\log\log t} \widehat{C}_{k,\ast}(t)}{t^2 (\log t)^2} dt\right) \\ 
     & = \frac{\sqrt{\log\log x}}{\log x} \widehat{C}_{k,\ast}(x) + 
     O\left(\frac{x}{2^k} \times \Gamma\left(k+\frac{1}{2}, 2\log\log x\right)\right). 
\end{align*} 
Finally, whenever we assume that $1 \leq k \leq \frac{3}{2} \log\log x$ so that $\lambda > 1$ in 
Proposition \ref{prop_IncGammaLambdaTypeBounds_v1} 
(\cf Facts \ref{facts_ExpIntIncGammaFuncs} for $k$ of substantially lesser order than this upper bound), 
Theorem \ref{theorem_CnkSpCasesScaledSummatoryFuncs} 
implies the conclusion of our corollary. 
\end{proof}

\subsection{Average orders of the unsigned sequences}
\label{subSection_AvgOrdersOfTheUnsignedSequences} 

\begin{prop} 
\label{lemma_HatCAstxSum_ExactFormulaWithError_v1} 
There is an absolute constant $B_0 > 0$ such that 
as $n \rightarrow \infty$ 
\[
\frac{1}{n} \times \sum_{k \leq n} C_{\Omega(k)}(k) = 
B_0 (\log n) \sqrt{\log\log n}\left(1 + O\left(\frac{1}{\log\log n}\right)\right). 
\] 
\end{prop} 
\begin{proof} 
By Corollary \ref{cor_SummatoryFuncsOfUnsignedSeqs_v2} and 
Proposition \ref{prop_IncGammaLambdaTypeBounds_v1} 
with $\lambda = \frac{1}{2}$, we have that 
\begin{align} 
\notag 
\sum_{k=1}^{\frac{3}{2} \log\log x} \sum_{\substack{n \leq x \\ \Omega(n) = k}} C_{\Omega(n)}(n) & \asymp 
     \sum_{k=1}^{\frac{3}{2} \log\log x} \frac{x (\log\log x)^{k-\frac{1}{2}}}{(k-1)!} 
     \left(1 + O\left(\frac{1}{\log\log x}\right)\right) \\ 
\notag 
     & = \frac{x (\log x) \sqrt{\log\log x} \Gamma\left(\frac{3}{2} \log\log x, \log\log x\right)}{ 
     \Gamma\left(\frac{3}{2} \log\log x\right)} 
     \left(1 + O\left(\frac{1}{\log\log x}\right)\right) \\ 
\notag 
     & = 
     \frac{4x (\log x)}{\sqrt{2\pi \log\log x}} \left(1 + O\left(\frac{1}{\log\log x}\right)\right). 
\end{align}
For real $0 \leq z \leq 2$, the function $\widehat{G}(z)$ is piecewise monotone in 
$z$ with $\widehat{G}(0) = 1$ and $\widehat{G}(2) \approx 0.303964$. 
Then we see that there is an absolute constant $B_0 > 0$ such that 
\begin{align} 
\notag 
\frac{1}{x} \times \sum_{k=1}^{\frac{3}{2} \log\log x} 
     \sum_{\substack{n \leq x \\ \Omega(n) = k}} C_{\Omega(n)}(n) & = 
     B_0 (\log x) \sqrt{\log\log x} \left(1 + O\left(\frac{1}{\log\log x}\right)\right). 
\end{align} 
We claim that 
\begin{align} 
\notag 
\frac{1}{x} \times \sum_{n \leq x} C_{\Omega(n)}(n) & = \frac{1}{x} \times 
     \sum_{k \geq 1} \sum_{\substack{n \leq x \\ \Omega(n) = k}} C_{\Omega(n)}(n) \\ 
\notag 
     & = 
     \frac{1}{x} \times \sum_{k=1}^{\frac{3}{2} \log\log x} 
     \sum_{\substack{n \leq x \\ \Omega(n) = k}} 
     C_{\Omega(n)}(n) (1+o(1)), 
     \text{ as } x \rightarrow \infty. 
\end{align} 
To prove the claim it suffices to show that 
\begin{equation} 
\label{eqn_proof_tag_PartialSumsOver_HatCkx_EquivCond_v2} 
\frac{1}{x} \times 
     \sum\limits_{\substack{n \leq x \\ \Omega(n) \geq \frac{3}{2} \log\log x}} C_{\Omega(n)}(n)
     = o\left(\frac{\log x}{\log\log x}\right). 
\end{equation} 
We proved in Theorem \ref{prop_HatAzx_ModSummatoryFuncExps_RelatedToCkn} 
that for all sufficiently large $x$ and $|z| < P(2)^{-1}$ 
\[
\sum_{n \leq x} (-1)^{\omega(n)} C_{\Omega(n)}(n) z^{\Omega(n)} = 
     \frac{x \widehat{F}(2, z)}{\Gamma(z)} (\log x)^{z-1} + O\left( 
     x (\log x)^{\Re(z)-2}\right). 
\]
By Lemma \ref{cor_AsymptoticsForSignedSumsOfomegan_v1}, 
we have that the summatory function 
\[
\sum_{n \leq x} (-1)^{\omega(n)} = 
     \frac{(-1)^{\floor{\log\log x}} x}{A_0 \sqrt{2\pi \log\log x}} 
     \left(1+O\left(\frac{1}{\sqrt{\log\log x}}\right)\right), 
\]
where $\frac{d}{dx}\left[\frac{x}{\sqrt{\log\log x}}\right] = \frac{1}{\sqrt{\log\log x}} + o(1)$. 
We can argue as in the proof of Corollary \ref{cor_SummatoryFuncsOfUnsignedSeqs_v2} 
that whenever $0 < |z| < P(2)^{-1}$ and $x$ is sufficiently large we have 
\begin{align}
\notag
\sum_{n \leq x} C_{\Omega(n)}(n) z^{\Omega(n)} & \ll 
     \frac{\widehat{F}(2, z) x (\log x) \sqrt{\log\log x}}{\Gamma(z)} \times 
     \frac{\partial}{\partial x}\left[x (\log x)^{z-1}\right] \\ 
\label{eqn_COmegannzPowOmeganLLRelation_v1} 
     & \ll 
     \frac{\widehat{F}(2, z) x \sqrt{\log\log x}}{\Gamma(z)} (\log x)^{z}. 
\end{align}
For large $x$ and any fixed $0 < r < P(2)^{-1}$, we define 
\[
\widehat{B}(x, r) := \sum_{\substack{n \leq x \\ \Omega(n) \geq r\log\log x}} 
     C_{\Omega(n)}(n). 
\]
We adapt the proof from the reference \cite[\cf Thm.\ 7.20; \S 7.4]{MV} by 
applying \eqref{eqn_COmegannzPowOmeganLLRelation_v1} when $1 \leq r < P(2)^{-1}$. 
Since $r \widehat{F}(2, r) = \frac{r \zeta(2)^{-r}}{1+P(2)r} \ll 1$ 
for $r \in [1, P(2)^{-1})$, and similarly since we have that 
$\frac{1}{\Gamma(1+r)} \gg 1$ for $r$ within the same range, 
we find that 
\[
x \sqrt{\log\log x} (\log x)^{r} \gg \sum_{\substack{n \leq x \\ \Omega(n) \geq r\log\log x}} 
     C_{\Omega(n)}(n) r^{\Omega(n)} \gg 
     \sum_{\substack{n \leq x \\ \Omega(n) \geq r\log\log x}} 
     C_{\Omega(n)}(n) r^{r \log\log x}. 
\]
This implies that for $r := \frac{3}{2}$ we have 
\begin{equation}
\label{eqn_BHatxrUpperBound_v1}
\widehat{B}(x, r) \ll x (\log x)^{r-r\log r} \sqrt{\log\log x} = 
     O\left(x (\log x)^{0.891802} \sqrt{\log\log x}\right)
\end{equation}
We evaluate the limiting asymptotics of the sum 
\begin{align*}
S_2(x) & := \frac{1}{x} \times 
     \sum_{k \geq \frac{3}{2} \log\log x} \sum_{\substack{n \leq x \\ \Omega(n)=k}} 
     C_{\Omega(n)}(n) \ll \frac{1}{x} \times \widehat{B}(x, 2) = 
     O\left((\log x)^{0.891802} \sqrt{\log\log x}\right), 
     \text{ as } x \rightarrow \infty. 
\end{align*} 
This implies that 
\eqref{eqn_proof_tag_PartialSumsOver_HatCkx_EquivCond_v2} holds. 
\end{proof} 

\begin{cor}
\label{cor_ExpectationFormulaAbsgInvn_v2} 
We have that as $n \rightarrow \infty$ 
\begin{align*} 
\frac{1}{n} \times \sum_{k \leq n} |g^{-1}(k)| & = 
     \frac{6B_0 (\log n)^2 \sqrt{\log\log n}}{\pi^2} 
     \left(1 + O\left(\frac{1}{\log\log n}\right)\right). 
\end{align*} 
\end{cor} 
\begin{proof} 
As $|z| \rightarrow \infty$, the \emph{imaginary error function}, 
$\operatorname{erfi}(z)$, has the following asymptotic expansion 
\cite[\S 7.12]{NISTHB}: 
\begin{equation}
\label{eqn_Erfix_KnownAsymptoticSeries_v1}
\operatorname{erfi}(z) := \frac{2}{\sqrt{\pi} \imath} \times \int_0^{\imath z} e^{t^2} dt = 
     \frac{e^{z^2}}{\sqrt{\pi}} \left(\frac{1}{z} + \frac{1}{2z^3} + 
     \frac{3}{4z^5} + \frac{15}{8z^7} + O\left(\frac{1}{z^{9}}\right)\right). 
\end{equation}
We use the formula from Proposition \ref{lemma_HatCAstxSum_ExactFormulaWithError_v1} 
to sum the average order of $C_{\Omega(n)}(n)$.
The proposition and error terms obtained from \eqref{eqn_Erfix_KnownAsymptoticSeries_v1} 
imply that for all sufficiently large $t \rightarrow \infty$ 
\begin{align*} 
\int \frac{\sum_{n \leq t} C_{\Omega(n)}(n)}{t^2} dt & = 
     B_0 (\log t)^2 \sqrt{\log\log t} - \frac{1}{4} \sqrt{\frac{\pi}{2}} 
     \operatorname{erfi}\left(\sqrt{2\log\log t}\right) \\ 
     & = 
     B_0 (\log t)^2 \sqrt{\log\log t} \left(1 + O\left(\frac{1}{\log\log t}\right)\right). 
\end{align*} 
The summatory function that counts the 
number of \emph{squarefree} integers $n \leq x$ satisfies 
\cite[\S 18.6]{HARDYWRIGHT} \cite[\seqnum{A013928}]{OEIS} 
\[ 
Q(x) = \sum_{n \leq x} \mu^2(n) = \frac{6x}{\pi^2} + O\left(\sqrt{x}\right), 
     \text{\ as $x \rightarrow \infty$}. 
\]
Therefore, summing over the formula from 
\eqref{eqn_AbsValueOf_gInvn_FornSquareFree_v1} in 
Section \ref{subSection_Relating_CknFuncs_to_gInvn}, we find that  
\begin{align} 
\notag 
\frac{1}{n} \times \sum_{k \leq n} |g^{-1}(k)| & = \frac{1}{n} \times \sum_{d \leq n} 
     C_{\Omega(d)}(d) Q\left(\Floor{n}{d}\right) \\ 
\notag 
     & \sim \sum_{d \leq n} C_{\Omega(d)}(d) \left[\frac{6}{d \cdot \pi^2} + O\left(\frac{1}{\sqrt{dn}}\right) 
     \right] \\ 
\notag 
     & = \frac{6}{\pi^2} \left[\frac{1}{n} \times \sum_{k \leq n} C_{\Omega(k)}(k) + \sum_{d<n} 
     \sum_{k \leq d} \frac{C_{\Omega(k)}(k)}{d^2}\right] + O(1). 
     \qedhere 
\end{align} 
\end{proof} 

\subsection{Erd\H{o}s-Kac theorem analogs for the distributions of the unsigned functions} 
\label{subSection_ErdosKacTheorem_Analogs} 

We show in the proof of 
Theorem \ref{theorem_CLT_VI} that for 
$1 \leq k \leq \frac{3}{2} \log\log x$ 
\begin{equation}
\label{eqn_EKThmAnalogsPreface_formula_intuition_v1}
     \frac{1}{x} \times \sum_{\substack{n \leq x \\ \Omega(n)=k}} 
     \frac{C_{\Omega(n)}(n)}{(\log n) \sqrt{\log\log n}} \asymp 
     \frac{(\log\log x)^{k-1}}{(\log x) (k-1)!} \left(1 + 
     O\left(\frac{1}{\log\log x}\right)\right). 
\end{equation} 
The non-centrally normal tending densities resulting from the right-hand-side of the 
previous equation summed over $k$ follow from the analytic arguments given in 
\cite[Thm.~7.21; \S 7.4]{MV}. 
Nonetheless, showing that the distribution of 
$\frac{C_{\Omega(n)}(n)}{(\log n) \sqrt{\log\log n}}$ over $n \leq x$ as $x \rightarrow \infty$ 
has the same non-centrally normal CDF does not follow from 
\eqref{eqn_EKThmAnalogsPreface_formula_intuition_v1} in the 
same manner as the corresponding distribution obtained in the reference. 
We need a deeper assumption to prove this result. 
Namely, we need the next probabilistic ansatz to prove 
Theorem \ref{theorem_CLT_VI}. 

\begin{ansatz}
\label{ansatz_KeyIndependenceAssumptions} 
We require the assumption that the functions 
$$X_{n,k} := \frac{C_{\Omega(n)}(n)}{(\log n)\sqrt{\log\log n}},$$ defined for 
distinct $n \leq x$ such that $\Omega(n) = k$ when $1 \leq k \leq \frac{3}{2} \log\log x$ 
can be viewed as independent random variables 
(\cf \cite{BILLINGSLY-CLT-PRIMEDIVFUNC}). 
The reasoning for this assumption on the independence 
of $X_{n_1,k},X_{n_2,k}$ whenever $\Omega(n_1) = \Omega(n_2) = k$ is 
explained using the notation for the asymptotic densities defined in 
\cite[\S 2.4]{MV} as 
\[
N_m(x) := \#\{n \leq x: \Omega(n) - \omega(n) = m\} = d_m x + O\left(
     \left(\frac{3}{4}\right)^k \sqrt{x} (\log x)^{\frac{4}{3}}\right), m \geq 0, 
\]
where 
\[
\sum_{k \geq 0} d_k z^k = \prod_p \left(1 - \frac{1}{p}\right)\left(1+\frac{1}{p-z}\right), 
\]
and such that the possible $d_m$ sum to 
$\sum_{m \geq 0} d_m \sim 1$ as $x \rightarrow \infty$. 
For $1 \leq k \leq \frac{3}{2} \log\log x$ the total sum of 
$n \leq x$ such that $\Omega(n) = k$ over all possible exponent patterns 
that contribute to the distinct values of $C_{\Omega(n)}(n)$ has main term 
$(d_0+d_1+\cdots+d_{k-1}) \times \widehat{\pi}_k(x)$ for large $x$. 
Since $\frac{\left(p\widehat{\pi}_k(x)-1\right)}{\widehat{\pi}_k(x)} = p + o(1)$, 
when there is overlap in the values $r_1,r_2$ assumed by 
$X_{n_1,k},X_{n_2,k}$ for fixed $k$, we still 
see that these variables are (approximately) independent as 
$x \rightarrow \infty$ by computing 
\[
\mathbb{P}(X_{n_1,k} = r_1 \mid X_{n_2,k} = r_2) = \begin{cases} 
     \mathbb{P}(X_{n_1,k} = r_1), & r_1 \neq r_2; \\ 
     \mathbb{P}(X_{n_1,k} = r_1) + O\left(\frac{(\log x) \sqrt{\log\log x}}{x}\right), & r_1 = r_2, 
     \end{cases} 
\]
and vice versa. 
\end{ansatz}

\begin{theorem} 
\label{theorem_CLT_VI} 
For sufficiently large $x$, let the mean and variance parameters be defined by 
\[
\mu_x(C) := \log\log x - 
     \log\left(\sqrt{2\pi}A_0 \widehat{G}(1)\right), 
     \quad \mathrm{\ and\ } \quad 
     \sigma_x(C) := \sqrt{\log\log x}. 
\]
We have that 
\[
\frac{1}{x} \times \#\left\{2 \leq n \leq x: 
     \frac{\frac{C_{\Omega(n)}(n)}{(\log n)\sqrt{\log\log n}} - 
     \mu_x(C)}{\sigma_x(C)} \leq z\right\} = 
     \Phi\left(z\right) + o(1), 
     \mathrm{\ as\ } x \rightarrow \infty. 
\]
\end{theorem}
\begin{proof}
We will provide a rigorous outline to prove the theorem under the assumption 
of the ansatz. The complete remaining details behind the rest of the 
proof are left to the reader to verify. 
For $1 \leq k \leq \frac{3}{2} \log\log x$, let 
\[
\widehat{\mu}_k(x) := \frac{1}{x} \times \sum_{\substack{n \leq x \\ \Omega(n)=k}} 
          \frac{C_{\Omega(n)}(n)}{(\log n) \sqrt{\log\log n}}. 
\]
Using integration by parts applied to 
Corollary \ref{cor_SummatoryFuncsOfUnsignedSeqs_v2}, we have 
uniformly for any $1 \leq k \leq \frac{3}{2} \log\log x$ that 
\begin{align} 
\notag 
x \cdot \widehat{\mu}_k(x) & = 
     \frac{\widehat{C}_k(x)}{(\log x) \sqrt{\log\log x}} + O\left( 
     \int_3^x \frac{dt}{(\log t) (\log\log t)}\right) \\ 
\notag 
     & = 
     \frac{\widehat{C}_k(x)}{(\log x) \sqrt{\log\log x}} + O\left( 
     \frac{x}{(\log x)^2 \sqrt{\log\log x}}\right) \\ 
\label{eqn_ProofTag_AnsatzRVDiffFromMeanBound_v0} 
     & = \frac{A_0 \sqrt{2\pi} x}{\log x} \times \widehat{G}\left(\frac{k-1}{\log\log x}\right) 
     \frac{(\log\log x)^{k-1}}{(k-1)!}\left(1 + O\left(\frac{1}{\log\log x}\right)\right), 
     \text{ as } x \rightarrow \infty. 
\end{align} 
For $1 \leq k \leq \frac{3}{2} \log\log x$, let 
\[
\sigma_k^2(x) := \frac{1}{x} \times \sum_{\substack{n \leq x \\ \Omega(n)=k}} 
     \frac{C_{\Omega(n)}(n)^2}{(\log n)^2 (\log\log n)}. 
\]
We then define the following variance parameters for large $x$: 
\[
s_x^2 := \sum_{n \leq x} \sigma_{\Omega(n)}^2(n). 
\]
We can show that the sequence of random variables $\{X_{n,\Omega(n)}\}_{n \geq 1}$ 
satisfies \emph{Lindeberg's condition}, i.e., for all fixed $\epsilon > 0$
\[
\lim_{x \rightarrow \infty} \frac{1}{s_x^2} \times \sum_{n \leq x} \left(\frac{1}{n} \times 
     \sum_{m \leq n} 
     (X_{m,\Omega(m)}-\widehat{\mu}_{\Omega(m)}(m))^2 
     \mathds{1}_{\{|X_{m,\Omega(m)}-\widehat{\mu}_{\Omega(m)}(m)| > \epsilon s_n\}}(m) 
     \right) = 0. 
\]
Then we have convergence in distribution to standard normal in the form of  
\[
\frac{1}{x \cdot s_x} \times \sum_{1 \leq k \leq 2\log\log x} \left( 
     \sum_{\substack{n \leq x \\ \Omega(n)=k}} \frac{C_{\Omega(n)}(n)}{(\log n) \sqrt{\log\log n}} - 
     x \cdot \widehat{\mu}_k(x)\right) \overset{d}{\implies} \mathcal{N}(0, 1), 
     \text{ as } x \rightarrow \infty. 
\]
We find that $s_x^2 = o(1)$ so that both 
\[
\frac{1}{x} \times \sum_{1 \leq k \leq 2\log\log x} 
     \sum_{\substack{n \leq x \\ \Omega(n)=k}} \frac{C_{\Omega(n)}(n)}{(\log n) \sqrt{\log\log n}}, 
     \quad \text{ and } \quad 
     \sum_{1 \leq k \leq 2\log\log x} \widehat{\mu}_k(x), 
\]
have identical distributions as $x \rightarrow \infty$. 
A straightforward extension of the arguments given in 
\cite[Thm.~7.21; \S 7.4]{MV} shows for any $Y > 0$ uniformly for 
$-Y \leq z \leq Y$ that 
\[
\frac{1}{x} \times \#\left\{2 \leq n \leq x: 
     \frac{\widehat{\mu}_{\Omega(n)}(n) - \mu_x(C)}{\sigma_x(C)} \leq z\right\} 
     = \Phi(z) + O\left(\frac{1}{\sqrt{\log\log x}}\right). 
\]
In fact we see that as $x \rightarrow \infty$ 
\[
\sum_{k} \widehat{\mu}_k(x) \overset{d}{\implies} \mathcal{N}\left(\mu_x(C), \sigma_x^2(C)\right). 
\]
Hence, we also have that 
\[
\frac{1}{x} \times \sum_{n \leq x} \frac{C_{\Omega(n)}(n)}{(\log n) \sqrt{\log\log n}}
     \overset{d}{\implies} \mathcal{N}\left(\mu_x(C), \sigma_x^2(C)\right), 
\]
with maximally the same error term. 
\end{proof}

%\begin{remark}
%Fix any $Y > 0$ and set $z \in [-Y, Y]$. 
%For large $x$ and $2 \leq n \leq x$, define the following sequences: 
%\[
%\alpha_n := \frac{\frac{C_{\Omega(n)}(n)}{(\log n) \sqrt{\log\log n}} - \mu_n(C)}{\sigma_n(C)}, 
%     \quad\mathrm{and}\quad 
%\beta_{n,x} := \frac{\frac{C_{\Omega(n)}(n)}{(\log n) \sqrt{\log\log n}} - \mu_x(C)}{\sigma_x(C)}. 
%\] 
%Let the corresponding densities be defined by 
%\[
%\Phi_1(x, z) := \frac{1}{x} \times \#\{n \leq x: \alpha_n \leq z\}, 
%\]
%and 
%\[
%\Phi_2(x, z) := \frac{1}{x} \times \#\{n \leq x: \beta_{n,x} \leq z\}. 
%\] 
%We assert that it suffices to show that $\Phi_1(x, z) = \Phi(x) + O\left(\frac{1}{\sqrt{\log\log x}}\right)$ as 
%$x \rightarrow \infty$ in place of considering the distribution of 
%$\Phi_2(x, z)$ to obtain the conclusion of Theorem \ref{theorem_CLT_VI}. 
%
%The differences of the normalizing terms $\sigma_n(C)$ and $\sigma_x(C)$ and their 
%reciprocals are small for $\sqrt{x} \leq n \leq x$. For $n \in [\sqrt{x}, x]$ 
%as $x \rightarrow \infty$ we see that 
%\[
%\left\lvert \frac{\mu_n(C)}{\sigma_n(C)} - \frac{\mu_x(C)}{\sigma_x(C)} \right\rvert \leq 
%     \frac{\log 2}{\sqrt{\log\log x}}(1 + o(1)), 
%\]
%and 
%\[
%\left\lvert \frac{1}{\sigma_n(C)} - \frac{1}{\sigma_x(C)} \right\rvert 
%     \leq \frac{2\log 2}{(\log x) \sqrt{\log\log x}}(1 + o(1)). 
%\]
%By applying \eqref{eqn_BHatxrUpperBound_v1}, when $\Omega(n) \geq 2\mu_x(C)$ 
%\[
%\lim_{x \rightarrow \infty} \frac{1}{x} \times 
%     \left\{\sqrt{x} \leq n \leq x: 
%     \frac{2}{(\log x)^{0.386294}} \geq 
%     \frac{C_{\Omega(n)}(n)}{(\log n) \sqrt{\log\log n}} 
%     \right\} = 0.  
%\]
%Similarly, using 
%Proposition \ref{lemma_HatCAstxSum_ExactFormulaWithError_v1} when 
%$\Omega(n) \leq 2 \mu_x(C)$ 
%\[
%\lim_{x \rightarrow \infty} \frac{1}{x} \times 
%     \left\{\sqrt{x} \leq n \leq x: 
%     \frac{4\log(2)}{(\log x)\sqrt{\log\log x}} \geq  
%     \frac{C_{\Omega(n)}(n)}{(\log n) \sqrt{\log\log n}} 
%     \right\} = 0. 
%\]
%It follows that when $\sqrt{x} \leq n \leq x$
%we have $\left\lvert \alpha_n - \beta_{n,x} \right\rvert = o(1)$. 
%Since $\Phi^{\prime}(z) \ll 1$, 
%it suffices to replace $\alpha_n$ by $\beta_{n,x}$ and estimate the limiting 
%densities corresponding to these alternate terms as $x \rightarrow \infty$ to 
%obtain the stated result in Theorem \ref{theorem_CLT_VI}. 
%\end{remark} 

\begin{cor} 
\label{cor_CLT_VII} 
Suppose that $\mu_x(C)$ and $\sigma_x(C)$ are defined as in 
Theorem \ref{theorem_CLT_VI} for large $x$. 
Let $Y > 0$. 
We have uniformly for all $-Y \leq y \leq Y$ 
that as $x \rightarrow \infty$ 
\begin{align*} 
\frac{1}{x} \cdot \#\left\{2 \leq n \leq x: \frac{|g^{-1}(n)|}{(\log n) \sqrt{\log\log n}} - 
     \frac{6}{\pi^2 n (\log n) \sqrt{\log\log n}} \times \sum_{k \leq n} |g^{-1}(k)| \leq y\right\} & = 
     \Phi\left(\frac{\frac{\pi^2 y}{6}-\mu_x(C)}{\sigma_x(C)}\right) + o(1). 
\end{align*} 
\end{cor} 
\begin{proof} 
We claim that 
\begin{align*} 
|g^{-1}(n)| - \frac{6}{\pi^2 n} \times \sum_{k \leq n} |g^{-1}(k)| & \sim \frac{6}{\pi^2} C_{\Omega(n)}(n), 
     \text{\ as\ } n \rightarrow \infty. 
\end{align*} 
As in the proof of Corollary \ref{cor_ExpectationFormulaAbsgInvn_v2}, 
we obtain that 
\begin{align*} 
\frac{1}{x} \times \sum_{n \leq x} |g^{-1}(n)| & = 
     \frac{6}{\pi^2} \left(\frac{1}{x} \times \sum_{n \leq x} C_{\Omega(n)}(n) + \sum_{d<x} 
     \sum_{k \leq d} \frac{C_{\Omega(k)}(k)}{d^2}\right) + O(1). 
\end{align*} 
Let the \emph{backwards difference operator} with respect to $x$ 
be defined for $x \geq 2$ and any arithmetic function $f$ as 
$\Delta_x(f(x)) := f(x) - f(x-1)$. 
We see that for large $n$ 
\begin{align*} 
     |g^{-1}(n)| & = \Delta_n\left(\sum_{k \leq n} g^{-1}(k)\right)  
     \sim \frac{6}{\pi^2} \times 
     \Delta_n\left(\sum_{d \leq n} C_{\Omega(d)}(d) \cdot \frac{n}{d}\right) \\ 
     & = \frac{6}{\pi^2}\left(C_{\Omega(n)}(n) + \sum_{d < n} C_{\Omega(d)}(d) \frac{n}{d} - 
     \sum_{d<n} C_{\Omega(d)}(d) \frac{(n-1)}{d}\right) \\ 
     & \sim \frac{6}{\pi^2} \left(C_{\Omega(n)}(n) + \frac{1}{n-1} \times \sum_{k < n} |g^{-1}(k)|\right), 
     \mathrm{\ as\ } n \rightarrow \infty. 
\end{align*} 
Since $\frac{1}{n-1} \times \sum_{k < n} |g^{-1}(k)| \sim \frac{1}{n} \times \sum_{k \leq n} |g^{-1}(k)|$ 
for all sufficiently large $n$, 
the result follows by a re-normalization of Theorem \ref{theorem_CLT_VI}. 
\end{proof} 

\label{subSection_ProbInterprets_Of_ErdosKacAnalogs} 

\begin{lemma} 
\label{lemma_ProbsOfAbsgInvnDist_v2} 
Let $\mu_x(C)$ and $\sigma_x(C)$ be defined as in 
Theorem \ref{theorem_CLT_VI}. 
For all sufficiently large $x$, 
if we pick any integer $n \in [2, x]$ uniformly at random, then 
each of the following statements holds as $x \rightarrow \infty$: 
\begin{align*} 
\tag{A}
\mathbb{P}\left(\frac{|g^{-1}(n)|}{(\log n)\sqrt{\log\log n}} - 
     \frac{6}{\pi^2 n (\log n) \sqrt{\log\log n}} \times \sum_{k \leq n} |g^{-1}(k)| \leq 
     \frac{6}{\pi^2} \mu_x(C) 
     \right) & = \frac{1}{2} + o(1) \\ 
\tag{B} 
\mathbb{P}\left(\frac{|g^{-1}(n)|}{(\log n) \sqrt{\log\log n}} - 
     \frac{6}{\pi^2 n (\log n) \sqrt{\log\log n}} \times \sum_{k \leq n} |g^{-1}(k)| \leq 
     \frac{6}{\pi^2}\left(\alpha \sigma_x(C) + \mu_x(C)\right)
     \right) & = 
     \Phi\left(\alpha\right) + o(1), \alpha \in \mathbb{R}. 
\end{align*} 
\end{lemma} 
\begin{proof} 
Each of these results is a consequence of Corollary \ref{cor_CLT_VII}. 
The result in (A) follows since $\Phi(0) = \frac{1}{2}$ by taking 
$$y = \frac{6}{\pi^2}\left(\alpha \sigma_x(C) + \mu_x(C)\right),$$ 
in Corollary \ref{cor_CLT_VII} 
for $\alpha = 0$. 
\end{proof} 

\newpage 
\section{New formulas and limiting relations characterizing $M(x)$} 
\label{Section_KeyApplications} 

\subsection{Formulas relating $M(x)$ to the summatory function $G^{-1}(x)$} 

\begin{prop} 
\label{prop_Mx_SBP_IntegralFormula} 
For all sufficiently large $x$, we have that 
\begin{align} 
\label{eqn_pf_tag_v2-restated_v2} 
M(x) & = G^{-1}(x) + 
     \sum_{k=1}^{\frac{x}{2}} G^{-1}(k) \left(
     \pi\left(\Floor{x}{k}\right) - \pi\left(\Floor{x}{k+1}\right) 
     \right). 
\end{align} 
\end{prop} 
\begin{proof} 
We know by applying Corollary \ref{cor_Mx_gInvnPixk_formula} that 
\begin{align} 
\notag
M(x) & = \sum_{k=1}^{x} g^{-1}(k) \left(\pi\left(\Floor{x}{k}\right)+1\right) \\ 
\notag 
     & = G^{-1}(x) + \sum_{k=1}^{\frac{x}{2}} g^{-1}(k) \pi\left(\Floor{x}{k}\right) \\ 
\notag 
     & = G^{-1}(x) + G^{-1}\left(\Floor{x}{2}\right) + 
     \sum_{k=1}^{\frac{x}{2}-1} G^{-1}(k) \left( 
     \pi\left(\Floor{x}{k}\right) - \pi\left(\Floor{x}{k+1}\right) 
     \right).
\end{align} 
The upper bound on the sum is truncated to $k \in \left[1, \frac{x}{2}\right]$ in the second equation 
above due to the fact that $\pi(1) = 0$. 
The third formula above follows directly by (ordinary) summation by parts. 
\end{proof} 

By the result from \eqref{eqn_RmkInitialConnectionOfMxToGInvx_ProvedByInversion_v1} 
proved in Section \ref{subSection_Relating_CknFuncs_to_gInvn}, 
we recall that 
\[
M(x) = G^{-1}(x) + \sum_{p \leq x} G^{-1}\left(\Floor{x}{p}\right), 
     \text{ for } x \geq 1. 
\]
Summation by parts implies that we can also express $G^{-1}(x)$ in terms 
of the summatory function $L(x)$ and differences of the unsigned sequence 
whose distribution is given by 
Corollary \ref{cor_CLT_VII}. That is, we have  
\[
G^{-1}(x) 
     = \sum_{n \leq x} \lambda(n) |g^{-1}(n)| 
     = L(x)|g^{-1}(x)| - \sum_{n < x} 
     L(n) \left(|g^{-1}(n+1)| - |g^{-1}(n)|\right), 
     \text{ for } x \geq 1. 
\]

\subsection{Asymptotics of $G^{-1}(x)$} 
\label{subSection_AsymptoticsOfGinvx} 

The following proofs are credited to Professor R.~C.~Vaughan and his suggestions 
about approaches to upper bounds on $|G^{-1}(x)|$ that are attained 
along infinite subsequences as $x \rightarrow \infty$. 
The ideas at the crux of the proof of the next theorem 
are found in the references by Davenport and Heilbronn \cite{DAVHEIL-1936A,DAVHEIL-1936B} 
and are known to date back to the work of 
Hans Bohr \cite[\cf \S 11]{TITCHMARSH}. 

\begin{theorem}
\label{theorem_VaughanGrowthOfGInvxAndZerosOfPrimeZetaFunc_v1}
Let $\sigma_1$ denote the unique solution to the equation 
$P(\sigma) = 1$ for $\sigma > 1$. 
There are complex $s$ with $\Re(s)$ arbitrarily close to $\sigma_1$
such that $1+P(s) = 0$. 
\end{theorem}
\begin{proof}
The function $P(\sigma)$ is decreasing on $(1, \infty)$, 
tends to $+\infty$ as $\sigma \rightarrow 1^{+}$, and tends to zero as 
$\sigma \rightarrow \infty$. Thus we find that the equation $P(\sigma) = 1$ 
has a unique solution for $\sigma > 1$, which we denote by 
$\sigma = \sigma_1 \approx 1.39943$. 
Let $\delta > 0$ be chosen small enough that $|1-P(z)| > 0$ for all 
$z$ such that $|z-\sigma_1| = \delta$. Set 
\[
\eta = \min_{\substack{z \in \mathbb{C} \\ |z-\sigma_1|=\delta}} |1-P(z)|. 
\]
Since $P(z)$ is continuous whenever $\Re(z) > 1$, we have that 
$\eta > 0$. 
Let $X \geq 2$ be a sufficiently large integer so that 
\[
\sum_{p > X} p^{\delta-\sigma_1} < \frac{\eta}{4}. 
\]
Kronecker's theorem provides a fixed $t$ such that the 
following inequality holds \cite[\S XXIII]{HARDYWRIGHT}: 
\[
\max_{2 < p \leq X} \min_{n \in \mathbb{Z}} \left\lvert 
     \frac{t \log p}{2\pi} - n - \frac{1}{2} \right\rvert < \delta\eta. 
\]
Thus we have that 
\[
\sum_{p > 2} p^{\delta-\sigma_1} \left\lvert p^{\imath t} + 1 \right\rvert < 
     \frac{\eta}{2}. 
\]
Hence, for all $z$ such that $|z-\sigma_1|=\delta$, we have 
\[
\left\lvert P(z+\imath t) + P(z) \right\rvert < \frac{\eta}{2}. 
\]
We apply Rouch\'{e}'s theorem to see that the functions 
$1-P(z)$ and $1-P(z) + P(z+\imath t) + P(z)$ have the same number of zeros in 
the disk $\mathcal{D}_{\delta} = \{z \in \mathbb{C}: |z-\sigma_1| < \delta\}$. 
Since $1-P(z)$ has at least one zero within $\mathcal{D}_{\delta}$, we must have that 
$1+P(w)$ has at least one zero with $|w-\sigma_1-\imath t| < \delta$. 
Since we can take $\delta$ as small as necessary, 
there are zeros of the function $1+P(s)$ that are arbitrarily close to the 
line $s = \sigma_1$. 
\end{proof}

\begin{cor}
\label{cor_Vaughan_LimSupLowerBounds_On_GInvx_AtLarge_x_v2} 
Let $\sigma_1 > 1$ be defined as in 
Theorem \ref{theorem_VaughanGrowthOfGInvxAndZerosOfPrimeZetaFunc_v1}. 
For any $\epsilon > 0$, there are arbitrarily large $x$ such that 
\[
|G^{-1}(x)| > x^{\sigma_1-\epsilon}. 
\]
\end{cor}
\begin{proof}
We have by \eqref{eqn_DGF_of_gInvn} that 
\[
D_{g^{-1}}(s) := \sum_{n \geq 1} \frac{g^{-1}(n)}{n^s} = \frac{1}{\zeta(s)(1+P(s))}, 
     \text{ for } \Re(s) > 1. 
\]
Theorem \ref{theorem_VaughanGrowthOfGInvxAndZerosOfPrimeZetaFunc_v1} implies that 
$D_{g^{-1}}(s)$ has singularities $s \in \mathbb{C}$ such that 
the $\Re(s)$ are arbitrarily close to $\sigma_1$. 
By applying \cite[Cor.~1.2; \S 1.2]{MV}, we have that any Dirichlet series 
is locally uniformly convergent in its half-plane of convergence, 
e.g., for $\Re(s) > \sigma_c$, and is hence analytic in this half-plane. 
It follows that the abscissa of convergence of 
$D_{g^{-1}}(s)$ is given by $\sigma_c \geq \sigma_1 > 1$. In particular, 
the abscissa of convergence of this DGF cannot be smaller than $\sigma_1$. 
The result proved in \cite[Thm.~1.3; \S 1.2]{MV} then shows that 
\[
\limsup_{x \rightarrow \infty} \frac{\log |G^{-1}(x)|}{\log x} = \sigma_c \geq \sigma_1. 
     \qedhere 
\]
\end{proof}

\subsection{Local cancellation of $G^{-1}(x)$ in the new formulas for $M(x)$} 
\label{subSection_LocalCancellationOfGInvx} 

\begin{lemma}
\label{theorem_PrimorialSeqGInvCalcs_v1} 
Suppose that $p_n$ denotes the $n^{th}$ prime for $n \geq 1$ 
\cite[\seqnum{A000040}]{OEIS}. 
Let $\mathcal{P}_{\#}$ denote the set of positive primorial integers as 
\cite[\seqnum{A002110}]{OEIS} 
\[
\mathcal{P}_{\#} = \left\{n\#\right\}_{n \geq 1} = \left\{\prod_{k=1}^{n} p_k : n \geq 1\right\} = 
     \{2, 6, 30, 210, 2310, 30030, \ldots\}. 
\]
As $m \rightarrow \infty$ we have 
\begin{align*}
     -G^{-1}((4m+1)\#) & = (4m+1)! \left(1 + O\left(\frac{1}{m^2}\right)\right), \\ 
G^{-1}\left(\frac{(4m+1)\#}{p_k}\right) & = (4m)! \left(1 + O\left(\frac{1}{m^2}\right)\right), 
     \mathrm{\ for\ all\ } 1 \leq k \leq 4m+1. 
\end{align*} 
\end{lemma}
\begin{proof} 
We have by part (B) of Proposition \ref{lemma_gInv_MxExample} 
that for all squarefree integers $n \geq 1$ 
\begin{align*} 
|g^{-1}(n)| & = \sum_{j=0}^{\omega(n)} \binom{\omega(n)}{j} \times j! 
     = (\omega(n))! \times \sum_{j=0}^{\omega(n)} \frac{1}{j!} \\ 
     & = (\omega(n))! \times \left(e + O\left(\frac{1}{(\omega(n)+1)!}\right)\right). 
\end{align*} 
Let $m$ be a large positive integer. 
We obtain main terms of the form 
\begin{align*} 
G_{U}^{-1}((4m+1)\#) & := \sum_{\substack{n \leq (4m+1)\# \\ \omega(n)=\Omega(n)}} \lambda(n) |g^{-1}(n)| \\ 
     & = \sum_{0 \leq k \leq 4m+1} \binom{4m+1}{k} (-1)^{k} k! 
     \left(e + O\left(\frac{1}{(k+1)!}\right)\right) \\ 
     & = -(4m+1)! + O(1). 
\end{align*} 
We argue that the analogous sums over the non-squarefree $n \leq (4m+1)\#$ 
contribute strictly less than the order of $G_{U}^{-1}((4m+1)\#)$ to the 
main term of $G^{-1}((4m+1)\#)$. 
Suppose that $2 \leq n \leq (4m+1)\#$ is not squarefree. We have the next largest 
order of growth of the sequence along those $n$ with 
$|g^{-1}(n)| \leq |g^{-1}(p_s^2 t)|$ for some $1 \leq s \leq 4m+1$ and where $t$ is squarefree. 
If $s = 1$ so that $p_s = 2$, we have that the largest possible squarefree part $t$ 
satisfies $t \leq p_3p_4 \cdots p_{4m+1}$. 
A corresponding $t$ with $\omega(t) = 4m-1$ that attains the 
same bound on $|g^{-1}(n)|$ corresponds to taking any 
(unordered) rearrangement 
of the distinct prime factors bounding $t$ from above by the previous product. 
By Corollary \ref{lemma_AbsValueOf_gInvn_FornSquareFree_v1}, we have that 
\[
\left\lvert g^{-1}(p_1^kt) \right\rvert = \sum_{\substack{d=p_1^kd_0,p_1^{k-1}d_0 \\ d_0|t}} 
     C_{\Omega(d)}(d) = 
     \sum_{d_0|t} \left(\binom{k+\omega(d_0)}{k} + \binom{k-1+\omega(d_0)}{k-1}\right) (\omega(d_0))!. 
\]
Then we see that 
\begin{align*} 
\left\lvert \sum_{k=2}^{\log_2\left((4m+1)\#\right)} 
     \sum_{\substack{1 \leq t \leq \frac{(4m+1)\#}{p_1^k} \\ \omega(t)=\Omega(t)=4m-1}} 
     g^{-1}(p_1^k t) \right\rvert 
     & \leq \sum_{k \geq 2} \sum_{i=0}^{4m-1} \binom{4m-1}{i} (-1)^{k+i} i! \left( 
     \binom{k+i}{k} + \binom{k-1+i}{k-1}\right) \\ 
     & = \frac{(4m-1)! (4m+1)}{4 e m} + O(1). 
\end{align*} 
We consider the contributions from subsequent leading powers of the other 
$p_s \leq (4m+1)\#$ when $2 \leq s \leq 4m+1$. When we have that 
$|g^{-1}(n)| \leq |g^{-1}(p_s^2 t)|$ for $p_s \geq 3$ and 
$t \leq p_{r+1} p_{r+2} \cdots p_{4m+1}$ squarefree, we obtain 
\begin{align*}
\left\lvert \sum_{k=2}^{\log_{p_s}\left((4m+1)\#\right)} 
     \sum_{\substack{1 \leq t \leq \frac{(4m+1)\#}{p_1^k} \\ \omega(t)=\Omega(t)=4m+1-r}} 
     g^{-1}(p_s^k t) \right\rvert & \leq 
     \frac{(4m-r)! (4m+1-r)}{e} + O(1) \\ 
     & \ll \frac{(4m-1)! (4m+1-r)}{r!}. 
\end{align*}
For any fixed $p_s$ with $2 \leq s \leq 4m+1$, we bound the lower 
index $r$ according to $p_s^2 (1+o(1)) \leq r \log r$ using the prime number theorem. 
The inequality requires that 
\[
r \geq e^{W_0(p_s^2(1+o(1)))} = e^{2\log p_s-\log\log(p_s^2) + o(1)} \sim p_s^2 - 2 \log p_s. 
\]
The lower order term sums $G_L^{-1}((4m+1)\#)$ are then bounded from above by 
\begin{align*} 
G_L^{-1}((4m+1)\#) & := \left\lvert 
     \sum_{\substack{n \leq (4m+1)\# \\ \omega(n)<\Omega(n)}} g^{-1}(n) 
     \right\rvert \\ 
     & \leq \sum_{r=2}^{4m} \frac{(4m-1)! (4m+1-r)}{e r!} \\ 
     & \asymp -(4m)! \left(1 + O\left(\frac{1}{m}\right)\right), 
     \text{ as } m \rightarrow \infty. 
\end{align*} 
Hence, we find that $-G^{-1}((4m+1)\#) \sim (4m+1)!$. 
We can similarly derive for any $1 \leq k \leq 4m+1$ that 
\begin{align*}
G^{-1}\left(\frac{(4m+1)\#}{p_k}\right) & \sim \sum_{0 \leq k \leq 4m} \binom{4m}{k} (-1)^{k} k! 
     \left(e + O\left(\frac{1}{(k+1)!}\right)\right) \sim (4m)!. 
     \qedhere 
\end{align*}
\end{proof}

\begin{remark}
The analysis of the maximal limiting bounds on $G^{-1}(x)$ from below as 
$x \rightarrow \infty$ guaranteed by 
Corollary \ref{cor_Vaughan_LimSupLowerBounds_On_GInvx_AtLarge_x_v2} 
complicate the interpretation of 
Proposition \ref{prop_Mx_SBP_IntegralFormula} 
to form new asymptotics for $M(x)$. 
Even though we get comparitively large order growth of 
$G^{-1}(x)$ infinitely often, 
we expect that there is usually (nearly almost always) 
a large cancellation between the successive 
values of this summatory function in 
\eqref{eqn_RmkInitialConnectionOfMxToGInvx_ProvedByInversion_v1}. 
Lemma \ref{theorem_PrimorialSeqGInvCalcs_v1} 
demonstrates the phenomenon well along the asymptotically large infinite 
subsequence of $x$ taken along the primorials, or 
the integers $x = (4m+1)\#$ 
that each precisely the product of the first $4m+1$ primes. 

Since we have for sufficiently large $n$ that 
\cite{DUSART-1999,DUSART-2010} 
\[
n\# \sim e^{\vartheta(p_n)} \asymp n^n (\log n)^n e^{-n(1+o(1))}, 
     \text{ as } n \rightarrow \infty, 
\]
the RH requires that the leading constants with opposing signs 
on the asymptotics for the functions from the last lemma match. 
This observation follows from the fact that if we obtain a contrary result, 
equation \eqref{eqn_RmkInitialConnectionOfMxToGInvx_ProvedByInversion_v1} would imply that 
\[
\frac{M((4m+1)\#}{\sqrt{(4m+1)\#}} \gg \left[(4m+1)\#\right]^{\delta_0}, 
     \text{ as } m \rightarrow \infty, 
\]
for some fixed $\delta_0 > 0$. 
The formula in \eqref{eqn_RmkInitialConnectionOfMxToGInvx_ProvedByInversion_v1} 
implies that under the RH we witness the expected substantial cancellation from the 
summatory function terms involving $G^{-1}(x)$ in the formula for $M(x)$ 
along this notable subsequence. 
In fact, for sufficiently large $m$, we have that the following properties holds: 
\begin{itemize} 
\item[(i)] $\operatorname{sgn}\left(G^{-1}((4m+1)\#)\right) = - \operatorname{sgn}\left( 
     \sum\limits_{p \leq (4m+1)\#} G^{-1}\left(\frac{(4m+1)\#}{p}\right)\right)$; 
\item[(ii)] $\lim\limits_{m \rightarrow \infty} \frac{G^{-1}((4m+1)\#)}{ 
     \sum\limits_{p \leq (4m+1)\#} G^{-1}\left(\frac{(4m+1)\#}{p}\right)} = -1$; 
\item[(iii)] $M((4m+1)\#) \gg \sum\limits_{\substack{n \leq (4m+1)\# \\ \omega(n)=\Omega(n)}} 
     g^{-1}(n) \left(1+\pi\left(\frac{(4m+1)\#}{n}\right)\right)$. 
\end{itemize} 
That is, along this primorial subsequence, the contributions of the 
local maxima for the absolute values of 
$|g^{-1}(n)|$ at the squarefree integers cancel considerably and do not 
contribute the main term for the limiting asymptotic expansion of $M(x)$ along 
$x = (4m+1)\#$ as $m \rightarrow \infty$. 
\end{remark}

\newpage
\section{Conclusions}

We have identified a new sequence, 
$\{g^{-1}(n)\}_{n \geq 1}$, that is the Dirichlet inverse of the 
shifted strongly additive function $\omega(n)$. 
Section \ref{subSection_AConnectionToDistOfThePrimes}, 
shows that there is a natural combinatorial interpretation to the 
distribution of distinct values 
of $|g^{-1}(n)|$ for $n \leq x$ involving the distribution of the 
primes $p \leq x$ at large $x$. 
In particular, the magnitude of $g^{-1}(n)$ depends only on the pattern of 
the exponents of the prime factorization of $n$. 
The signedness of $g^{-1}(n)$ is given by $\lambda(n)$ for all $n \geq 1$. 
This leads to a new relations of the 
summatory function $G^{-1}(x)$ that characterize the distribution of $M(x)$ 
to the distribution of the summatory function $L(x)$. 

We emphasize that our new work on the Mertens function proved within this article 
is significant in providing a new window through which we can view bounding $M(x)$ 
through asymptotics of auxiliary sequences and partial sums. 
The computational data generated in 
Table \ref{table_conjecture_Mertens_ginvSeq_approx_values} of the appendix section 
suggests numerically that the distribution of $G^{-1}(x)$ may be easier to work with 
than that of $M(x)$ or $L(x)$. 
The additively combinatorial 
relation of the distinct (and repetition of) values of $|g^{-1}(n)|$ 
for $n \leq x$ are suggestive towards bounding main terms for $G^{-1}(x)$ along 
infinite subsequences in future work. 

\section*{Acknowledgments}
\addcontentsline{toc}{section}{Acknowledgments}

We thank the following professors that offered 
discussion, feedback and correspondence while the article was written: 
Bob Vaughan, Gerg\H{o} Nemes, Steven J.~Miller, Paul Pollack and Bruce Reznick. 
The work on the article was supported in part by 
funding made available within the School of Mathematics at the 
Georgia Institute of Technology in 2020 and 2021. 
Without this combined support, the article would not have been possible.

\newpage 
\renewcommand{\refname}{References} 
\addcontentsline{toc}{section}{References}
%\bibliography{glossaries-bibtex/thesis-references}{}
\bibliographystyle{plain}

\begin{thebibliography}{10}

\bibitem{APOSTOLANUMT}
T.~M. Apostol.
\newblock {\em Introduction to Analytic Number Theory}.
\newblock Springer--Verlag, 1976.

\bibitem{BILLINGSLY-CLT-PRIMEDIVFUNC}
P.~Billingsley.
\newblock On the central limit theorem for the prime divisor function.
\newblock {\em Amer. Math. Monthly}, 76(2):132--139, 1969.

\bibitem{DAVHEIL-1936A}
H.~Davenport and H.~Heilbronn.
\newblock On the zeros of certain {D}irichlet series {I}.
\newblock {\em J. London Math. Soc.}, 11:181--185, 1936.

\bibitem{DAVHEIL-1936B}
H.~Davenport and H.~Heilbronn.
\newblock On the zeros of certain {D}irichlet series {II}.
\newblock {\em J. London Math. Soc.}, 11:307--312, 1936.

\bibitem{DUSART-1999}
P.~Dusart.
\newblock The $k^{th}$ prime is greater than $k(\log k +\log\log k-1)$ for $k
  \geq 2$.
\newblock {\em Math. Comp.}, 68(225):411--415, 1999.

\bibitem{DUSART-2010}
P.~Dusart.
\newblock Estimates of some functions over primes without {R}.{H}, 2010.

\bibitem{ERDOS-KAC-REF}
P.~Erd{\H{o}}s and M.~Kac.
\newblock The guassian errors in the theory of additive arithmetic functions.
\newblock {\em American Journal of Mathematics}, 62(1):738--742, 1940.

\bibitem{FROBERG-1968}
C.~E. Fr{\"{o}}berg.
\newblock On the prime zeta function.
\newblock {\em BIT Numerical Mathematics}, 8:87--202, 1968.

\bibitem{HARDYWRIGHT}
G.~H. Hardy and E.~M. Wright, editors.
\newblock {\em An Introduction to the Theory of Numbers}.
\newblock Oxford University Press, 2008 (Sixth Edition).

\bibitem{HUMPHRIES-JNT-2013}
P.~Humphries.
\newblock The distribution of weighted sums of the {L}iouville function and
  {P}\'{o}lya's conjecture.
\newblock {\em J. Number Theory}, 133:545--582, 2013.

\bibitem{HURST-2017}
G.~Hurst.
\newblock Computations of the {M}ertens function and improved bounds on the
  {M}ertens conjecture.
\newblock {\em Math. Comp.}, 87:1013--1028, 2018.

\bibitem{CLT-RANDOM-ORDERED-FACTS-2011}
H.~Hwang and S.~Janson.
\newblock A central limit theorem for random ordered factorizations of
  integers.
\newblock {\em Electron. J. Probab.}, 16(12):347--361, 2011.

\bibitem{IWANIEC-KOWALSKI}
H.~Iwaniec and E.~Kowalski.
\newblock {\em Analytic Number Theory}, volume~53.
\newblock AMS Colloquium Publications, 2004.

\bibitem{MREVISITED}
T.~Kotnik and H.~te~Riele.
\newblock The {M}ertens conjecture revisited.
\newblock {\em Algorithmic Number Theory}, $7^{th}$ International Symposium,
  2006.

\bibitem{ORDER-MERTENSFN}
T.~Kotnik and J.~van~de Lune.
\newblock On the order of the {M}ertens function.
\newblock {\em Exp. Math.}, 2004.

\bibitem{LEHMAN-1960}
R.~S. Lehman.
\newblock On {L}iouville's function.
\newblock {\em Math. Comput.}, 14:311--320, 1960.

\bibitem{MV}
H.~L. Montgomery and R.~C. Vaughan.
\newblock {\em Multiplicative Number Theory: I. Classical Theory}.
\newblock Cambridge, 2006.

\bibitem{NEMES2015C}
G.~Nemes.
\newblock The resurgence properties of the incomplete gamma function {II}.
\newblock {\em Stud. Appl. Math.}, 135(1):86--116, 2015.

\bibitem{NEMES2016}
G.~Nemes.
\newblock The resurgence properties of the incomplete gamma function {I}.
\newblock {\em Anal. Appl. (Singap.)}, 14(5):631--677, 2016.

\bibitem{NEMES2019}
G.~Nemes and A.~B.~Olde Daalhuis.
\newblock Asymptotic expansions for the incomplete gamma function in the
  transition regions.
\newblock {\em Math. Comp.}, 88(318):1805--1827, 2019.

\bibitem{NG-MERTENS}
N.~Ng.
\newblock The distribution of the summatory function of the {M}{\'{o}}bius
  function.
\newblock {\em Proc. London Math. Soc.}, 89(3):361--389, 2004.

\bibitem{ODLYZ-TRIELE}
A.~M. Odlyzko and H.~J.~J. te~Riele.
\newblock Disproof of the {M}ertens conjecture.
\newblock {\em J. Reine Angew. Math.}, 1985.

\bibitem{NISTHB}
Frank W.~J. Olver, Daniel~W. Lozier, Ronald~F. Boisvert, and Charles~W. Clark,
  editors.
\newblock {\em {NIST} Handbook of Mathematical Functions}.
\newblock Cambridge University Press, 2010.

\bibitem{RENYI-TURAN}
A.~Renyi and P.~Turan.
\newblock On a theorem of {E}rd{\H{o}}s-{K}ac.
\newblock {\em Acta Arithmetica}, 4(1):71--84, 1958.

\bibitem{PRIMEREC}
P.~Ribenboim.
\newblock {\em The new book of prime number records}.
\newblock Springer, 1996.

\bibitem{SCHMIDT-MERTENS-COMPUTATIONS}
M.~D. Schmidt.
\newblock {S}age{M}ath and {M}athematica software for computations with the
  {M}ertens function, 2021.
\newblock \url{https://github.com/maxieds/MertensFunctionComputations}.

\bibitem{OEIS}
N.~J.~A. Sloane.
\newblock The {O}nline {E}ncyclopedia of {I}nteger {S}equences, 2021.
\newblock \url{http://oeis.org}.

\bibitem{SOUND-MERTENS-ANNALS}
K.~Soundararajan.
\newblock Partial sums of the {M}{\"{o}}bius function.
\newblock {\em J. Reine Angew. Math.}, 2009(631):141--152, 2009.

\bibitem{TITCHMARSH}
E.~C. Titchmarsh.
\newblock {\em The theory of the {R}iemann zeta function}.
\newblock Oxford University Press, second edition, 1986.

\end{thebibliography}

\setcounter{section}{0} 
\renewcommand{\thesection}{\Alph{section}} 

\newpage
\section{Appendix: Asymptotic formulas for partial sums} 
\label{subSection_OtherFactsAndResults} 

We appreciate the kind online correspondence with Gerg\H{o} Nemes 
from the Alfr\'{e}d R\'{e}nyi Institute of Mathematics and his 
careful notes on the limiting asymptotics for the sums identified in this section. 
We have adapted the communication of his proofs to establish the next few lemmas based on his 
recent work in the references \cite{NEMES2015C,NEMES2016,NEMES2019}. 

\begin{facts}[The incomplete gamma function] 
\label{facts_ExpIntIncGammaFuncs} 
\begin{subequations}
The (upper) \emph{incomplete gamma function} is defined by \cite[\S 8.4]{NISTHB} 
\[
\Gamma(a, z) = \int_{z}^{\infty} t^{a-1} e^{-t} dt, a \in \mathbb{R}, |\arg z| < \pi.  
\]
The function $\Gamma(a, z)$ can be continued to an analytic function of $z$ on the 
universal covering of $\mathbb{C} \mathbin{\backslash} \{0\}$. 
For $a \in \mathbb{Z}^{+}$, the function $\Gamma(a, z)$ is an entire function of $z$. 
The following properties of $\Gamma(a, z)$ hold \cite[\S 8.4; \S 8.11(i)]{NISTHB}: 
\begin{align} 
\label{eqn_IncompleteGamma_PropA} 
\Gamma(a, z) & = (a-1)! e^{-z} \times \sum_{k=0}^{a-1} \frac{z^k}{k!}, \mathrm{\ for\ } 
     a \in \mathbb{Z}^{+}, z \in \mathbb{C}, \\ 
\label{eqn_IncompleteGamma_PropB} 
\Gamma(a, z) & \sim z^{a-1} e^{-z}, \mathrm{\ for\ fixed\ } a \in \mathbb{C}, 
     \mathrm{\ as\ } z \rightarrow +\infty. 
\end{align}
Moreover, for real $z > 0$, as $z \rightarrow +\infty$ we have that \cite{NEMES2015C} 
\begin{equation} 
\label{eqn_IncompleteGamma_PropC}
\Gamma(z, z) = \sqrt{\frac{\pi}{2}} z^{z-\frac{1}{2}} e^{-z} + 
     O\left(z^{z-1} e^{-z}\right), 
\end{equation} 
If $z,a \rightarrow \infty$ with $z = \lambda a$ for some $\lambda > 1$ such that 
$(\lambda - 1)^{-1} = o\left(\sqrt{|a|}\right)$, then \cite{NEMES2015C}
\begin{equation}
\label{eqn_IncompleteGamma_PropD}
\Gamma(a, z) = z^a e^{-z} \times \sum_{n \geq 0} \frac{(-a)^n b_n(\lambda)}{(z-a)^{2n+1}}. 
\end{equation} 
The sequence $b_n(\lambda)$ satisfies the characteristic recurrence relation that 
$b_0(\lambda) = 1$ and\footnote{
     An exact formula for $b_n(\lambda)$ is given in terms of the 
     \emph{second-order Eulerian number triangle} 
     \cite[\seqnum{A008517}]{OEIS} as follows: 
     \[
          b_n(\lambda) = \sum_{k=0}^{n} \gkpEII{n}{k} \lambda^{k+1}. 
     \]
}
\[
b_n(\lambda) = \lambda(1-\lambda) b_{n-1}^{\prime}(\lambda) + \lambda(2n-1) b_{n-1}(\lambda), n \geq 1. 
\]
\end{subequations}
\end{facts} 

\begin{prop}
\label{prop_IncGammaLambdaTypeBounds_v1}
Let $a,z,\lambda$ be positive real parameters such that $z=\lambda a$. 
If $\lambda \in (0, 1)$, then as $z \rightarrow \infty$ 
\[
\Gamma(a, z) = \Gamma(a) + O_{\lambda}\left(z^{a-1} e^{-z}\right). 
\]
If $\lambda > 1$, then as 
$z \rightarrow \infty$ 
\[
\Gamma(a, z) = \frac{z^{a-1} e^{-z}}{1-\lambda^{-1}} + O_{\lambda}\left(z^{a-2} e^{-z}\right). 
\]
If $\lambda > 0.567142 > W(1)$ where $W(x)$ denotes the principal branch of the 
Lambert $W$-function for $x \geq 0$, 
then as $z \rightarrow \infty$ 
\[
\Gamma(a, z e^{\pm\pi\imath}) = -e^{\pm \pi\imath a} \frac{z^{a-1} e^{z}}{1 + \lambda^{-1}} + 
     O_{\lambda}\left(z^{a-2} e^{z}\right). 
\]
\end{prop}
Note that the first two asymptotic estmates are only useful when $\lambda$ is bounded away from the 
transition point at $1$. 
We cannot write the last expansion above 
as $\Gamma(a, -z)$ directly unless $a \in \mathbb{Z}^{+}$ 
as the incomplete gamma function 
has a branch point at the origin with respect to its second variable. 
This function becomes a single-valued 
analytic function of its second input by continuation 
on the universal covering of $\mathbb{C} \mathbin{\backslash} \{0\}$. 
\begin{proof}
The first asymptotic estimate follows directly from the following 
asymptotic series expansion that holds as $z \rightarrow +\infty$ 
\cite[Eq.\ (2.1)]{NEMES2019}: 
\[
\Gamma(a, z) \sim \Gamma(a) + z^a e^{-z} \times \sum_{k \geq 0} 
     \frac{(-a)^k b_k(\lambda)}{(z-a)^{2k+1}}. 
\]
Using the notation from \eqref{eqn_IncompleteGamma_PropD} and \cite{NEMES2016}, 
we have that 
\[
\Gamma(a, z) = \frac{z^{a-1} e^{-z}}{1-\lambda^{-1}} + z^{a} e^{-z} R_1(a, \lambda). 
\]
From the bounds in \cite[\S 3.1]{NEMES2016}, we have that 
\[
\left\lvert z^{a} e^{-z} R_1(a, \lambda) \right\rvert \leq 
     z^a e^{-z} \times \frac{a \cdot b_1(\lambda)}{(z-a)^{3}} = 
     \frac{z^{a-2} e^{-z}}{(1-\lambda^{-1})^{3}}
\]
The main and error terms in the previous equation can also be 
seen by applying the asymptotic series in 
\eqref{eqn_IncompleteGamma_PropD} directly. 

The proof of the third equation above follows from the following asymptotics 
\cite[Eq.\ (1.1)]{NEMES2015C}
\[
\Gamma(-a, z) \sim z^{-a} e^{-z} \times \sum_{n \geq 0} \frac{a^n b_n(-\lambda)}{(z+a)^{2n+1}}, 
\]
by setting $(a, z) \mapsto \left(a e^{\pm \pi\imath}, z e^{\pm \pi\imath}\right)$ so that 
$\lambda = \frac{z}{a} > 0.567142 > W(1)$. 
The restriction on the range of $\lambda$ over which the third formula holds is made to ensure that 
the last formula from the reference is valid at negative real $a$. 
\end{proof}

\begin{lemma}
\label{lemma_ConvenientIncGammaFuncTypePartialSumAsymptotics_v2}
For $x \rightarrow +\infty$, we have that 
\begin{align*}
S_1(x) & := \frac{x}{\log x} \times \left\lvert \sum_{1 \leq k \leq \floor{\log\log x}} 
     \frac{(-1)^k (\log\log x)^{k-1}}{(k-1)!} \right\rvert 
     = \frac{x}{2\sqrt{2\pi \log\log x}} + O\left(\frac{x}{(\log\log x)^{\frac{3}{2}}}\right). 
\end{align*}
\end{lemma}
\begin{proof}
We have for $n \geq 1$ and any $t > 0$ by 
\eqref{eqn_IncompleteGamma_PropA} that 
\[
\sum_{1 \leq k \leq n} \frac{(-1)^k t^{k-1}}{(k-1)!} = -e^{-t} \times 
     \frac{\Gamma(n, -t)}{(n-1)!}. 
\]
Suppose that $t = n + \xi$ with $\xi = O(1)$, e.g., so we can 
formally take the floor of the input $n$ to truncate the last sum. 
By the third formula 
in Proposition \ref{prop_IncGammaLambdaTypeBounds_v1} 
with the parameters $(a, z, \lambda) \mapsto \left(n, t, 1 + \frac{\xi}{n}\right)$, 
we deduce that as $n,t \rightarrow +\infty$. 
\begin{equation}
\label{eqn_ProofTag_lemma_ConvenientIncGammaFuncTypePartialSumAsymptotics_v2}
\Gamma(n, -t) = (-1)^{n+1} \times \frac{t^n e^{t}}{t+n} + 
     O\left(\frac{n t^n e^{t}}{(t+n)^3}\right) = 
     (-1)^{n+1} \frac{t^n e^t}{2n} + O\left(\frac{t^{n-1} e^t}{n}\right). 
\end{equation}
Accordingly, we see that 
\[
\sum_{1 \leq k \leq n} \frac{(-1)^k t^{k-1}}{(k-1)!} = 
     (-1)^{n} \frac{t^n}{2n!} + O\left(\frac{t^{n-1}}{n!}\right). 
\]
By the variant of Stirling's formula in \cite[\cf Eq.\ (5.11.8)]{NISTHB}, we have 
\[
n! = \Gamma(1 + t - \xi) = \sqrt{2\pi} t^{t-\xi+\frac{1}{2}} e^{-t} \left(1 + O\left(t^{-1}\right)\right) = 
     \sqrt{2\pi} t^{n+\frac{1}{2}} e^{-t} \left(1 + O\left(t^{-1}\right)\right). 
\]
Hence, as $n \rightarrow +\infty$ with $t := n + \xi$ and $\xi = O(1)$, we obtain that 
\[
\sum_{k=1}^{n} \frac{(-1)^k t^{k-1}}{(k-1)!} = (-1)^n \frac{e^t}{2 \sqrt{2\pi t}} + 
     O\left(e^t t^{-\frac{3}{2}}\right). 
\]
The conclusion follows by taking $n := \floor{\log\log x}$, 
$t := \log\log x$ and applying the triangle inequality 
to obtain the result. 
\end{proof}

\newpage
\section{Table: Compuations involving $g^{-1}(n)$ and $G^{-1}(n)$ for $1 \leq n \leq 500$} 
\label{table_conjecture_Mertens_ginvSeq_approx_values}

\begin{table}[ht!]

\centering

\tiny
\begin{equation*}
\boxed{
\begin{array}{cc|cc|ccc|cc|ccc}
 n & \mathbf{Primes} & \mathbf{Sqfree} & \mathbf{PPower} & g^{-1}(n) & 
 \lambda(n) g^{-1}(n) - \widehat{f}_1(n) & 
 \frac{\sum_{d|n} C_{\Omega(d)}(d)}{|g^{-1}(n)|} & 
 \mathcal{L}_{+}(n) & \mathcal{L}_{-}(n) & 
 G^{-1}(n) & G^{-1}_{+}(n) & G^{-1}_{-}(n) \\ \hline 
1 & 1^1 & \text{Y} & \text{N} & 1 & 0 & 1.0000000 & 1.000000 & 0.000000 & 1 & 1 & 0 \\
 2 & 2^1 & \text{Y} & \text{Y} & -2 & 0 & 1.0000000 & 0.500000 & 0.500000 & -1 & 1 & -2 \\
 3 & 3^1 & \text{Y} & \text{Y} & -2 & 0 & 1.0000000 & 0.333333 & 0.666667 & -3 & 1 & -4 \\
 4 & 2^2 & \text{N} & \text{Y} & 2 & 0 & 1.5000000 & 0.500000 & 0.500000 & -1 & 3 & -4 \\
 5 & 5^1 & \text{Y} & \text{Y} & -2 & 0 & 1.0000000 & 0.400000 & 0.600000 & -3 & 3 & -6 \\
 6 & 2^1 3^1 & \text{Y} & \text{N} & 5 & 0 & 1.0000000 & 0.500000 & 0.500000 & 2 & 8 & -6 \\
 7 & 7^1 & \text{Y} & \text{Y} & -2 & 0 & 1.0000000 & 0.428571 & 0.571429 & 0 & 8 & -8 \\
 8 & 2^3 & \text{N} & \text{Y} & -2 & 0 & 2.0000000 & 0.375000 & 0.625000 & -2 & 8 & -10 \\
 9 & 3^2 & \text{N} & \text{Y} & 2 & 0 & 1.5000000 & 0.444444 & 0.555556 & 0 & 10 & -10 \\
 10 & 2^1 5^1 & \text{Y} & \text{N} & 5 & 0 & 1.0000000 & 0.500000 & 0.500000 & 5 & 15 & -10 \\
 11 & 11^1 & \text{Y} & \text{Y} & -2 & 0 & 1.0000000 & 0.454545 & 0.545455 & 3 & 15 & -12 \\
 12 & 2^2 3^1 & \text{N} & \text{N} & -7 & 2 & 1.2857143 & 0.416667 & 0.583333 & -4 & 15 & -19 \\
 13 & 13^1 & \text{Y} & \text{Y} & -2 & 0 & 1.0000000 & 0.384615 & 0.615385 & -6 & 15 & -21 \\
 14 & 2^1 7^1 & \text{Y} & \text{N} & 5 & 0 & 1.0000000 & 0.428571 & 0.571429 & -1 & 20 & -21 \\
 15 & 3^1 5^1 & \text{Y} & \text{N} & 5 & 0 & 1.0000000 & 0.466667 & 0.533333 & 4 & 25 & -21 \\
 16 & 2^4 & \text{N} & \text{Y} & 2 & 0 & 2.5000000 & 0.500000 & 0.500000 & 6 & 27 & -21 \\
 17 & 17^1 & \text{Y} & \text{Y} & -2 & 0 & 1.0000000 & 0.470588 & 0.529412 & 4 & 27 & -23 \\
 18 & 2^1 3^2 & \text{N} & \text{N} & -7 & 2 & 1.2857143 & 0.444444 & 0.555556 & -3 & 27 & -30 \\
 19 & 19^1 & \text{Y} & \text{Y} & -2 & 0 & 1.0000000 & 0.421053 & 0.578947 & -5 & 27 & -32 \\
 20 & 2^2 5^1 & \text{N} & \text{N} & -7 & 2 & 1.2857143 & 0.400000 & 0.600000 & -12 & 27 & -39 \\
 21 & 3^1 7^1 & \text{Y} & \text{N} & 5 & 0 & 1.0000000 & 0.428571 & 0.571429 & -7 & 32 & -39 \\
 22 & 2^1 11^1 & \text{Y} & \text{N} & 5 & 0 & 1.0000000 & 0.454545 & 0.545455 & -2 & 37 & -39 \\
 23 & 23^1 & \text{Y} & \text{Y} & -2 & 0 & 1.0000000 & 0.434783 & 0.565217 & -4 & 37 & -41 \\
 24 & 2^3 3^1 & \text{N} & \text{N} & 9 & 4 & 1.5555556 & 0.458333 & 0.541667 & 5 & 46 & -41 \\
 25 & 5^2 & \text{N} & \text{Y} & 2 & 0 & 1.5000000 & 0.480000 & 0.520000 & 7 & 48 & -41 \\
 26 & 2^1 13^1 & \text{Y} & \text{N} & 5 & 0 & 1.0000000 & 0.500000 & 0.500000 & 12 & 53 & -41 \\
 27 & 3^3 & \text{N} & \text{Y} & -2 & 0 & 2.0000000 & 0.481481 & 0.518519 & 10 & 53 & -43 \\
 28 & 2^2 7^1 & \text{N} & \text{N} & -7 & 2 & 1.2857143 & 0.464286 & 0.535714 & 3 & 53 & -50 \\
 29 & 29^1 & \text{Y} & \text{Y} & -2 & 0 & 1.0000000 & 0.448276 & 0.551724 & 1 & 53 & -52 \\
 30 & 2^1 3^1 5^1 & \text{Y} & \text{N} & -16 & 0 & 1.0000000 & 0.433333 & 0.566667 & -15 & 53 & -68 \\
 31 & 31^1 & \text{Y} & \text{Y} & -2 & 0 & 1.0000000 & 0.419355 & 0.580645 & -17 & 53 & -70 \\
 32 & 2^5 & \text{N} & \text{Y} & -2 & 0 & 3.0000000 & 0.406250 & 0.593750 & -19 & 53 & -72 \\
 33 & 3^1 11^1 & \text{Y} & \text{N} & 5 & 0 & 1.0000000 & 0.424242 & 0.575758 & -14 & 58 & -72 \\
 34 & 2^1 17^1 & \text{Y} & \text{N} & 5 & 0 & 1.0000000 & 0.441176 & 0.558824 & -9 & 63 & -72 \\
 35 & 5^1 7^1 & \text{Y} & \text{N} & 5 & 0 & 1.0000000 & 0.457143 & 0.542857 & -4 & 68 & -72 \\
 36 & 2^2 3^2 & \text{N} & \text{N} & 14 & 9 & 1.3571429 & 0.472222 & 0.527778 & 10 & 82 & -72 \\
 37 & 37^1 & \text{Y} & \text{Y} & -2 & 0 & 1.0000000 & 0.459459 & 0.540541 & 8 & 82 & -74 \\
 38 & 2^1 19^1 & \text{Y} & \text{N} & 5 & 0 & 1.0000000 & 0.473684 & 0.526316 & 13 & 87 & -74 \\
 39 & 3^1 13^1 & \text{Y} & \text{N} & 5 & 0 & 1.0000000 & 0.487179 & 0.512821 & 18 & 92 & -74 \\
 40 & 2^3 5^1 & \text{N} & \text{N} & 9 & 4 & 1.5555556 & 0.500000 & 0.500000 & 27 & 101 & -74 \\
 41 & 41^1 & \text{Y} & \text{Y} & -2 & 0 & 1.0000000 & 0.487805 & 0.512195 & 25 & 101 & -76 \\
 42 & 2^1 3^1 7^1 & \text{Y} & \text{N} & -16 & 0 & 1.0000000 & 0.476190 & 0.523810 & 9 & 101 & -92 \\
 43 & 43^1 & \text{Y} & \text{Y} & -2 & 0 & 1.0000000 & 0.465116 & 0.534884 & 7 & 101 & -94 \\
 44 & 2^2 11^1 & \text{N} & \text{N} & -7 & 2 & 1.2857143 & 0.454545 & 0.545455 & 0 & 101 & -101 \\
 45 & 3^2 5^1 & \text{N} & \text{N} & -7 & 2 & 1.2857143 & 0.444444 & 0.555556 & -7 & 101 & -108 \\
 46 & 2^1 23^1 & \text{Y} & \text{N} & 5 & 0 & 1.0000000 & 0.456522 & 0.543478 & -2 & 106 & -108 \\
 47 & 47^1 & \text{Y} & \text{Y} & -2 & 0 & 1.0000000 & 0.446809 & 0.553191 & -4 & 106 & -110 \\
 48 & 2^4 3^1 & \text{N} & \text{N} & -11 & 6 & 1.8181818 & 0.437500 & 0.562500 & -15 & 106 & -121 \\ 
\end{array}
}
\end{equation*}

\bigskip\hrule\smallskip 

\captionsetup{singlelinecheck=off} 
\caption*{{\large{\rm \textbf{\rm \bf Table \thesection:} 
          Computations involving $g^{-1}(n) \equiv (\omega+1)^{-1}(n)$ 
          and $G^{-1}(x)$ for $1 \leq n \leq 500$.}} 
          \begin{itemize}[noitemsep,topsep=0pt,leftmargin=0.23in] 
          \item[$\blacktriangleright$] 
          The column labeled \texttt{Primes} provides the prime factorization of each $n$ so that the values of 
          $\omega(n)$ and $\Omega(n)$ are easily extracted. 
          The columns labeled \texttt{Sqfree} and \texttt{PPower}, respectively, 
          list inclusion of $n$ in the sets of squarefree integers and the prime powers. 
          \item[$\blacktriangleright$] 
          The next three columns provide the 
          explicit values of the inverse function $g^{-1}(n)$ and compare its explicit value with other estimates. 
          We define the function $\widehat{f}_1(n) := \sum_{k=0}^{\omega(n)} \binom{\omega(n)}{k} \times k!$. 
          \item[$\blacktriangleright$] 
          The last columns indicate properties of the summatory function of $g^{-1}(n)$. 
          The notation for the densities of the sign weight of $g^{-1}(n)$ is defined as 
          $\mathcal{L}_{\pm}(x) := \frac{1}{n} \times \#\left\{n \leq x: \lambda(n) = \pm 1\right\}$. 
          The last three 
          columns then show the explicit components to the signed summatory function, 
          $G^{-1}(x) := \sum_{n \leq x} g^{-1}(n)$, decomposed into its 
          respective positive and negative magnitude sum contributions: $G^{-1}(x) = G^{-1}_{+}(x) + G^{-1}_{-}(x)$ where 
          $G^{-1}_{+}(x) > 0$ and $G^{-1}_{-}(x) < 0$ for all $x \geq 1$. 
          That is, the component functions $G^{-1}_{\pm}(x)$ displayed in the last two columns 
          of the table correspond to the summatory function $G^{-1}(x)$ with summands that are 
          positive and negative, respectively. 
          \end{itemize} 
          } 
\clearpage 

\end{table}

\newpage
\begin{table}[ht]

\centering

\tiny
\begin{equation*}
\boxed{
\begin{array}{cc|cc|ccc|cc|ccc}
 n & \mathbf{Primes} & \mathbf{Sqfree} & \mathbf{PPower} & g^{-1}(n) & 
 \lambda(n) g^{-1}(n) - \widehat{f}_1(n) & 
 \frac{\sum_{d|n} C_{\Omega(d)}(d)}{|g^{-1}(n)|} & 
 \mathcal{L}_{+}(n) & \mathcal{L}_{-}(n) & 
 G^{-1}(n) & G^{-1}_{+}(n) & G^{-1}_{-}(n) \\ \hline 
 49 & 7^2 & \text{N} & \text{Y} & 2 & 0 & 1.5000000 & 0.448980 & 0.551020 & -13 & 108 & -121 \\
 50 & 2^1 5^2 & \text{N} & \text{N} & -7 & 2 & 1.2857143 & 0.440000 & 0.560000 & -20 & 108 & -128 \\
 51 & 3^1 17^1 & \text{Y} & \text{N} & 5 & 0 & 1.0000000 & 0.450980 & 0.549020 & -15 & 113 & -128 \\
 52 & 2^2 13^1 & \text{N} & \text{N} & -7 & 2 & 1.2857143 & 0.442308 & 0.557692 & -22 & 113 & -135 \\
 53 & 53^1 & \text{Y} & \text{Y} & -2 & 0 & 1.0000000 & 0.433962 & 0.566038 & -24 & 113 & -137 \\
 54 & 2^1 3^3 & \text{N} & \text{N} & 9 & 4 & 1.5555556 & 0.444444 & 0.555556 & -15 & 122 & -137 \\
 55 & 5^1 11^1 & \text{Y} & \text{N} & 5 & 0 & 1.0000000 & 0.454545 & 0.545455 & -10 & 127 & -137 \\
 56 & 2^3 7^1 & \text{N} & \text{N} & 9 & 4 & 1.5555556 & 0.464286 & 0.535714 & -1 & 136 & -137 \\
 57 & 3^1 19^1 & \text{Y} & \text{N} & 5 & 0 & 1.0000000 & 0.473684 & 0.526316 & 4 & 141 & -137 \\
 58 & 2^1 29^1 & \text{Y} & \text{N} & 5 & 0 & 1.0000000 & 0.482759 & 0.517241 & 9 & 146 & -137 \\
 59 & 59^1 & \text{Y} & \text{Y} & -2 & 0 & 1.0000000 & 0.474576 & 0.525424 & 7 & 146 & -139 \\
 60 & 2^2 3^1 5^1 & \text{N} & \text{N} & 30 & 14 & 1.1666667 & 0.483333 & 0.516667 & 37 & 176 & -139 \\
 61 & 61^1 & \text{Y} & \text{Y} & -2 & 0 & 1.0000000 & 0.475410 & 0.524590 & 35 & 176 & -141 \\
 62 & 2^1 31^1 & \text{Y} & \text{N} & 5 & 0 & 1.0000000 & 0.483871 & 0.516129 & 40 & 181 & -141 \\
 63 & 3^2 7^1 & \text{N} & \text{N} & -7 & 2 & 1.2857143 & 0.476190 & 0.523810 & 33 & 181 & -148 \\
 64 & 2^6 & \text{N} & \text{Y} & 2 & 0 & 3.5000000 & 0.484375 & 0.515625 & 35 & 183 & -148 \\
 65 & 5^1 13^1 & \text{Y} & \text{N} & 5 & 0 & 1.0000000 & 0.492308 & 0.507692 & 40 & 188 & -148 \\
 66 & 2^1 3^1 11^1 & \text{Y} & \text{N} & -16 & 0 & 1.0000000 & 0.484848 & 0.515152 & 24 & 188 & -164 \\
 67 & 67^1 & \text{Y} & \text{Y} & -2 & 0 & 1.0000000 & 0.477612 & 0.522388 & 22 & 188 & -166 \\
 68 & 2^2 17^1 & \text{N} & \text{N} & -7 & 2 & 1.2857143 & 0.470588 & 0.529412 & 15 & 188 & -173 \\
 69 & 3^1 23^1 & \text{Y} & \text{N} & 5 & 0 & 1.0000000 & 0.478261 & 0.521739 & 20 & 193 & -173 \\
 70 & 2^1 5^1 7^1 & \text{Y} & \text{N} & -16 & 0 & 1.0000000 & 0.471429 & 0.528571 & 4 & 193 & -189 \\
 71 & 71^1 & \text{Y} & \text{Y} & -2 & 0 & 1.0000000 & 0.464789 & 0.535211 & 2 & 193 & -191 \\
 72 & 2^3 3^2 & \text{N} & \text{N} & -23 & 18 & 1.4782609 & 0.458333 & 0.541667 & -21 & 193 & -214 \\
 73 & 73^1 & \text{Y} & \text{Y} & -2 & 0 & 1.0000000 & 0.452055 & 0.547945 & -23 & 193 & -216 \\
 74 & 2^1 37^1 & \text{Y} & \text{N} & 5 & 0 & 1.0000000 & 0.459459 & 0.540541 & -18 & 198 & -216 \\
 75 & 3^1 5^2 & \text{N} & \text{N} & -7 & 2 & 1.2857143 & 0.453333 & 0.546667 & -25 & 198 & -223 \\
 76 & 2^2 19^1 & \text{N} & \text{N} & -7 & 2 & 1.2857143 & 0.447368 & 0.552632 & -32 & 198 & -230 \\
 77 & 7^1 11^1 & \text{Y} & \text{N} & 5 & 0 & 1.0000000 & 0.454545 & 0.545455 & -27 & 203 & -230 \\
 78 & 2^1 3^1 13^1 & \text{Y} & \text{N} & -16 & 0 & 1.0000000 & 0.448718 & 0.551282 & -43 & 203 & -246 \\
 79 & 79^1 & \text{Y} & \text{Y} & -2 & 0 & 1.0000000 & 0.443038 & 0.556962 & -45 & 203 & -248 \\
 80 & 2^4 5^1 & \text{N} & \text{N} & -11 & 6 & 1.8181818 & 0.437500 & 0.562500 & -56 & 203 & -259 \\
 81 & 3^4 & \text{N} & \text{Y} & 2 & 0 & 2.5000000 & 0.444444 & 0.555556 & -54 & 205 & -259 \\
 82 & 2^1 41^1 & \text{Y} & \text{N} & 5 & 0 & 1.0000000 & 0.451220 & 0.548780 & -49 & 210 & -259 \\
 83 & 83^1 & \text{Y} & \text{Y} & -2 & 0 & 1.0000000 & 0.445783 & 0.554217 & -51 & 210 & -261 \\
 84 & 2^2 3^1 7^1 & \text{N} & \text{N} & 30 & 14 & 1.1666667 & 0.452381 & 0.547619 & -21 & 240 & -261 \\
 85 & 5^1 17^1 & \text{Y} & \text{N} & 5 & 0 & 1.0000000 & 0.458824 & 0.541176 & -16 & 245 & -261 \\
 86 & 2^1 43^1 & \text{Y} & \text{N} & 5 & 0 & 1.0000000 & 0.465116 & 0.534884 & -11 & 250 & -261 \\
 87 & 3^1 29^1 & \text{Y} & \text{N} & 5 & 0 & 1.0000000 & 0.471264 & 0.528736 & -6 & 255 & -261 \\
 88 & 2^3 11^1 & \text{N} & \text{N} & 9 & 4 & 1.5555556 & 0.477273 & 0.522727 & 3 & 264 & -261 \\
 89 & 89^1 & \text{Y} & \text{Y} & -2 & 0 & 1.0000000 & 0.471910 & 0.528090 & 1 & 264 & -263 \\
 90 & 2^1 3^2 5^1 & \text{N} & \text{N} & 30 & 14 & 1.1666667 & 0.477778 & 0.522222 & 31 & 294 & -263 \\
 91 & 7^1 13^1 & \text{Y} & \text{N} & 5 & 0 & 1.0000000 & 0.483516 & 0.516484 & 36 & 299 & -263 \\
 92 & 2^2 23^1 & \text{N} & \text{N} & -7 & 2 & 1.2857143 & 0.478261 & 0.521739 & 29 & 299 & -270 \\
 93 & 3^1 31^1 & \text{Y} & \text{N} & 5 & 0 & 1.0000000 & 0.483871 & 0.516129 & 34 & 304 & -270 \\
 94 & 2^1 47^1 & \text{Y} & \text{N} & 5 & 0 & 1.0000000 & 0.489362 & 0.510638 & 39 & 309 & -270 \\
 95 & 5^1 19^1 & \text{Y} & \text{N} & 5 & 0 & 1.0000000 & 0.494737 & 0.505263 & 44 & 314 & -270 \\
 96 & 2^5 3^1 & \text{N} & \text{N} & 13 & 8 & 2.0769231 & 0.500000 & 0.500000 & 57 & 327 & -270 \\
 97 & 97^1 & \text{Y} & \text{Y} & -2 & 0 & 1.0000000 & 0.494845 & 0.505155 & 55 & 327 & -272 \\
 98 & 2^1 7^2 & \text{N} & \text{N} & -7 & 2 & 1.2857143 & 0.489796 & 0.510204 & 48 & 327 & -279 \\
 99 & 3^2 11^1 & \text{N} & \text{N} & -7 & 2 & 1.2857143 & 0.484848 & 0.515152 & 41 & 327 & -286 \\
 100 & 2^2 5^2 & \text{N} & \text{N} & 14 & 9 & 1.3571429 & 0.490000 & 0.510000 & 55 & 341 & -286 \\
 101 & 101^1 & \text{Y} & \text{Y} & -2 & 0 & 1.0000000 & 0.485149 & 0.514851 & 53 & 341 & -288 \\
 102 & 2^1 3^1 17^1 & \text{Y} & \text{N} & -16 & 0 & 1.0000000 & 0.480392 & 0.519608 & 37 & 341 & -304 \\
 103 & 103^1 & \text{Y} & \text{Y} & -2 & 0 & 1.0000000 & 0.475728 & 0.524272 & 35 & 341 & -306 \\
 104 & 2^3 13^1 & \text{N} & \text{N} & 9 & 4 & 1.5555556 & 0.480769 & 0.519231 & 44 & 350 & -306 \\
 105 & 3^1 5^1 7^1 & \text{Y} & \text{N} & -16 & 0 & 1.0000000 & 0.476190 & 0.523810 & 28 & 350 & -322 \\
 106 & 2^1 53^1 & \text{Y} & \text{N} & 5 & 0 & 1.0000000 & 0.481132 & 0.518868 & 33 & 355 & -322 \\
 107 & 107^1 & \text{Y} & \text{Y} & -2 & 0 & 1.0000000 & 0.476636 & 0.523364 & 31 & 355 & -324 \\
 108 & 2^2 3^3 & \text{N} & \text{N} & -23 & 18 & 1.4782609 & 0.472222 & 0.527778 & 8 & 355 & -347 \\
 109 & 109^1 & \text{Y} & \text{Y} & -2 & 0 & 1.0000000 & 0.467890 & 0.532110 & 6 & 355 & -349 \\
 110 & 2^1 5^1 11^1 & \text{Y} & \text{N} & -16 & 0 & 1.0000000 & 0.463636 & 0.536364 & -10 & 355 & -365 \\
 111 & 3^1 37^1 & \text{Y} & \text{N} & 5 & 0 & 1.0000000 & 0.468468 & 0.531532 & -5 & 360 & -365 \\
 112 & 2^4 7^1 & \text{N} & \text{N} & -11 & 6 & 1.8181818 & 0.464286 & 0.535714 & -16 & 360 & -376 \\
 113 & 113^1 & \text{Y} & \text{Y} & -2 & 0 & 1.0000000 & 0.460177 & 0.539823 & -18 & 360 & -378 \\
 114 & 2^1 3^1 19^1 & \text{Y} & \text{N} & -16 & 0 & 1.0000000 & 0.456140 & 0.543860 & -34 & 360 & -394 \\
 115 & 5^1 23^1 & \text{Y} & \text{N} & 5 & 0 & 1.0000000 & 0.460870 & 0.539130 & -29 & 365 & -394 \\
 116 & 2^2 29^1 & \text{N} & \text{N} & -7 & 2 & 1.2857143 & 0.456897 & 0.543103 & -36 & 365 & -401 \\
 117 & 3^2 13^1 & \text{N} & \text{N} & -7 & 2 & 1.2857143 & 0.452991 & 0.547009 & -43 & 365 & -408 \\
 118 & 2^1 59^1 & \text{Y} & \text{N} & 5 & 0 & 1.0000000 & 0.457627 & 0.542373 & -38 & 370 & -408 \\
 119 & 7^1 17^1 & \text{Y} & \text{N} & 5 & 0 & 1.0000000 & 0.462185 & 0.537815 & -33 & 375 & -408 \\
 120 & 2^3 3^1 5^1 & \text{N} & \text{N} & -48 & 32 & 1.3333333 & 0.458333 & 0.541667 & -81 & 375 & -456 \\
 121 & 11^2 & \text{N} & \text{Y} & 2 & 0 & 1.5000000 & 0.462810 & 0.537190 & -79 & 377 & -456 \\
 122 & 2^1 61^1 & \text{Y} & \text{N} & 5 & 0 & 1.0000000 & 0.467213 & 0.532787 & -74 & 382 & -456 \\
 123 & 3^1 41^1 & \text{Y} & \text{N} & 5 & 0 & 1.0000000 & 0.471545 & 0.528455 & -69 & 387 & -456 \\
 124 & 2^2 31^1 & \text{N} & \text{N} & -7 & 2 & 1.2857143 & 0.467742 & 0.532258 & -76 & 387 & -463 \\ 
\end{array}
}
\end{equation*}
\clearpage 

\end{table} 


\newpage
\begin{table}[ht]

\centering

\tiny
\begin{equation*}
\boxed{
\begin{array}{cc|cc|ccc|cc|ccc}
 n & \mathbf{Primes} & \mathbf{Sqfree} & \mathbf{PPower} & g^{-1}(n) & 
 \lambda(n) g^{-1}(n) - \widehat{f}_1(n) & 
 \frac{\sum_{d|n} C_{\Omega(d)}(d)}{|g^{-1}(n)|} & 
 \mathcal{L}_{+}(n) & \mathcal{L}_{-}(n) & 
 G^{-1}(n) & G^{-1}_{+}(n) & G^{-1}_{-}(n) \\ \hline 
 125 & 5^3 & \text{N} & \text{Y} & -2 & 0 & 2.0000000 & 0.464000 & 0.536000 & -78 & 387 & -465 \\
 126 & 2^1 3^2 7^1 & \text{N} & \text{N} & 30 & 14 & 1.1666667 & 0.468254 & 0.531746 & -48 & 417 & -465 \\
 127 & 127^1 & \text{Y} & \text{Y} & -2 & 0 & 1.0000000 & 0.464567 & 0.535433 & -50 & 417 & -467 \\
 128 & 2^7 & \text{N} & \text{Y} & -2 & 0 & 4.0000000 & 0.460938 & 0.539062 & -52 & 417 & -469 \\
 129 & 3^1 43^1 & \text{Y} & \text{N} & 5 & 0 & 1.0000000 & 0.465116 & 0.534884 & -47 & 422 & -469 \\
 130 & 2^1 5^1 13^1 & \text{Y} & \text{N} & -16 & 0 & 1.0000000 & 0.461538 & 0.538462 & -63 & 422 & -485 \\
 131 & 131^1 & \text{Y} & \text{Y} & -2 & 0 & 1.0000000 & 0.458015 & 0.541985 & -65 & 422 & -487 \\
 132 & 2^2 3^1 11^1 & \text{N} & \text{N} & 30 & 14 & 1.1666667 & 0.462121 & 0.537879 & -35 & 452 & -487 \\
 133 & 7^1 19^1 & \text{Y} & \text{N} & 5 & 0 & 1.0000000 & 0.466165 & 0.533835 & -30 & 457 & -487 \\
 134 & 2^1 67^1 & \text{Y} & \text{N} & 5 & 0 & 1.0000000 & 0.470149 & 0.529851 & -25 & 462 & -487 \\
 135 & 3^3 5^1 & \text{N} & \text{N} & 9 & 4 & 1.5555556 & 0.474074 & 0.525926 & -16 & 471 & -487 \\
 136 & 2^3 17^1 & \text{N} & \text{N} & 9 & 4 & 1.5555556 & 0.477941 & 0.522059 & -7 & 480 & -487 \\
 137 & 137^1 & \text{Y} & \text{Y} & -2 & 0 & 1.0000000 & 0.474453 & 0.525547 & -9 & 480 & -489 \\
 138 & 2^1 3^1 23^1 & \text{Y} & \text{N} & -16 & 0 & 1.0000000 & 0.471014 & 0.528986 & -25 & 480 & -505 \\
 139 & 139^1 & \text{Y} & \text{Y} & -2 & 0 & 1.0000000 & 0.467626 & 0.532374 & -27 & 480 & -507 \\
 140 & 2^2 5^1 7^1 & \text{N} & \text{N} & 30 & 14 & 1.1666667 & 0.471429 & 0.528571 & 3 & 510 & -507 \\
 141 & 3^1 47^1 & \text{Y} & \text{N} & 5 & 0 & 1.0000000 & 0.475177 & 0.524823 & 8 & 515 & -507 \\
 142 & 2^1 71^1 & \text{Y} & \text{N} & 5 & 0 & 1.0000000 & 0.478873 & 0.521127 & 13 & 520 & -507 \\
 143 & 11^1 13^1 & \text{Y} & \text{N} & 5 & 0 & 1.0000000 & 0.482517 & 0.517483 & 18 & 525 & -507 \\
 144 & 2^4 3^2 & \text{N} & \text{N} & 34 & 29 & 1.6176471 & 0.486111 & 0.513889 & 52 & 559 & -507 \\
 145 & 5^1 29^1 & \text{Y} & \text{N} & 5 & 0 & 1.0000000 & 0.489655 & 0.510345 & 57 & 564 & -507 \\
 146 & 2^1 73^1 & \text{Y} & \text{N} & 5 & 0 & 1.0000000 & 0.493151 & 0.506849 & 62 & 569 & -507 \\
 147 & 3^1 7^2 & \text{N} & \text{N} & -7 & 2 & 1.2857143 & 0.489796 & 0.510204 & 55 & 569 & -514 \\
 148 & 2^2 37^1 & \text{N} & \text{N} & -7 & 2 & 1.2857143 & 0.486486 & 0.513514 & 48 & 569 & -521 \\
 149 & 149^1 & \text{Y} & \text{Y} & -2 & 0 & 1.0000000 & 0.483221 & 0.516779 & 46 & 569 & -523 \\
 150 & 2^1 3^1 5^2 & \text{N} & \text{N} & 30 & 14 & 1.1666667 & 0.486667 & 0.513333 & 76 & 599 & -523 \\
 151 & 151^1 & \text{Y} & \text{Y} & -2 & 0 & 1.0000000 & 0.483444 & 0.516556 & 74 & 599 & -525 \\
 152 & 2^3 19^1 & \text{N} & \text{N} & 9 & 4 & 1.5555556 & 0.486842 & 0.513158 & 83 & 608 & -525 \\
 153 & 3^2 17^1 & \text{N} & \text{N} & -7 & 2 & 1.2857143 & 0.483660 & 0.516340 & 76 & 608 & -532 \\
 154 & 2^1 7^1 11^1 & \text{Y} & \text{N} & -16 & 0 & 1.0000000 & 0.480519 & 0.519481 & 60 & 608 & -548 \\
 155 & 5^1 31^1 & \text{Y} & \text{N} & 5 & 0 & 1.0000000 & 0.483871 & 0.516129 & 65 & 613 & -548 \\
 156 & 2^2 3^1 13^1 & \text{N} & \text{N} & 30 & 14 & 1.1666667 & 0.487179 & 0.512821 & 95 & 643 & -548 \\
 157 & 157^1 & \text{Y} & \text{Y} & -2 & 0 & 1.0000000 & 0.484076 & 0.515924 & 93 & 643 & -550 \\
 158 & 2^1 79^1 & \text{Y} & \text{N} & 5 & 0 & 1.0000000 & 0.487342 & 0.512658 & 98 & 648 & -550 \\
 159 & 3^1 53^1 & \text{Y} & \text{N} & 5 & 0 & 1.0000000 & 0.490566 & 0.509434 & 103 & 653 & -550 \\
 160 & 2^5 5^1 & \text{N} & \text{N} & 13 & 8 & 2.0769231 & 0.493750 & 0.506250 & 116 & 666 & -550 \\
 161 & 7^1 23^1 & \text{Y} & \text{N} & 5 & 0 & 1.0000000 & 0.496894 & 0.503106 & 121 & 671 & -550 \\
 162 & 2^1 3^4 & \text{N} & \text{N} & -11 & 6 & 1.8181818 & 0.493827 & 0.506173 & 110 & 671 & -561 \\
 163 & 163^1 & \text{Y} & \text{Y} & -2 & 0 & 1.0000000 & 0.490798 & 0.509202 & 108 & 671 & -563 \\
 164 & 2^2 41^1 & \text{N} & \text{N} & -7 & 2 & 1.2857143 & 0.487805 & 0.512195 & 101 & 671 & -570 \\
 165 & 3^1 5^1 11^1 & \text{Y} & \text{N} & -16 & 0 & 1.0000000 & 0.484848 & 0.515152 & 85 & 671 & -586 \\
 166 & 2^1 83^1 & \text{Y} & \text{N} & 5 & 0 & 1.0000000 & 0.487952 & 0.512048 & 90 & 676 & -586 \\
 167 & 167^1 & \text{Y} & \text{Y} & -2 & 0 & 1.0000000 & 0.485030 & 0.514970 & 88 & 676 & -588 \\
 168 & 2^3 3^1 7^1 & \text{N} & \text{N} & -48 & 32 & 1.3333333 & 0.482143 & 0.517857 & 40 & 676 & -636 \\
 169 & 13^2 & \text{N} & \text{Y} & 2 & 0 & 1.5000000 & 0.485207 & 0.514793 & 42 & 678 & -636 \\
 170 & 2^1 5^1 17^1 & \text{Y} & \text{N} & -16 & 0 & 1.0000000 & 0.482353 & 0.517647 & 26 & 678 & -652 \\
 171 & 3^2 19^1 & \text{N} & \text{N} & -7 & 2 & 1.2857143 & 0.479532 & 0.520468 & 19 & 678 & -659 \\
 172 & 2^2 43^1 & \text{N} & \text{N} & -7 & 2 & 1.2857143 & 0.476744 & 0.523256 & 12 & 678 & -666 \\
 173 & 173^1 & \text{Y} & \text{Y} & -2 & 0 & 1.0000000 & 0.473988 & 0.526012 & 10 & 678 & -668 \\
 174 & 2^1 3^1 29^1 & \text{Y} & \text{N} & -16 & 0 & 1.0000000 & 0.471264 & 0.528736 & -6 & 678 & -684 \\
 175 & 5^2 7^1 & \text{N} & \text{N} & -7 & 2 & 1.2857143 & 0.468571 & 0.531429 & -13 & 678 & -691 \\
 176 & 2^4 11^1 & \text{N} & \text{N} & -11 & 6 & 1.8181818 & 0.465909 & 0.534091 & -24 & 678 & -702 \\
 177 & 3^1 59^1 & \text{Y} & \text{N} & 5 & 0 & 1.0000000 & 0.468927 & 0.531073 & -19 & 683 & -702 \\
 178 & 2^1 89^1 & \text{Y} & \text{N} & 5 & 0 & 1.0000000 & 0.471910 & 0.528090 & -14 & 688 & -702 \\
 179 & 179^1 & \text{Y} & \text{Y} & -2 & 0 & 1.0000000 & 0.469274 & 0.530726 & -16 & 688 & -704 \\
 180 & 2^2 3^2 5^1 & \text{N} & \text{N} & -74 & 58 & 1.2162162 & 0.466667 & 0.533333 & -90 & 688 & -778 \\
 181 & 181^1 & \text{Y} & \text{Y} & -2 & 0 & 1.0000000 & 0.464088 & 0.535912 & -92 & 688 & -780 \\
 182 & 2^1 7^1 13^1 & \text{Y} & \text{N} & -16 & 0 & 1.0000000 & 0.461538 & 0.538462 & -108 & 688 & -796 \\
 183 & 3^1 61^1 & \text{Y} & \text{N} & 5 & 0 & 1.0000000 & 0.464481 & 0.535519 & -103 & 693 & -796 \\
 184 & 2^3 23^1 & \text{N} & \text{N} & 9 & 4 & 1.5555556 & 0.467391 & 0.532609 & -94 & 702 & -796 \\
 185 & 5^1 37^1 & \text{Y} & \text{N} & 5 & 0 & 1.0000000 & 0.470270 & 0.529730 & -89 & 707 & -796 \\
 186 & 2^1 3^1 31^1 & \text{Y} & \text{N} & -16 & 0 & 1.0000000 & 0.467742 & 0.532258 & -105 & 707 & -812 \\
 187 & 11^1 17^1 & \text{Y} & \text{N} & 5 & 0 & 1.0000000 & 0.470588 & 0.529412 & -100 & 712 & -812 \\
 188 & 2^2 47^1 & \text{N} & \text{N} & -7 & 2 & 1.2857143 & 0.468085 & 0.531915 & -107 & 712 & -819 \\
 189 & 3^3 7^1 & \text{N} & \text{N} & 9 & 4 & 1.5555556 & 0.470899 & 0.529101 & -98 & 721 & -819 \\
 190 & 2^1 5^1 19^1 & \text{Y} & \text{N} & -16 & 0 & 1.0000000 & 0.468421 & 0.531579 & -114 & 721 & -835 \\
 191 & 191^1 & \text{Y} & \text{Y} & -2 & 0 & 1.0000000 & 0.465969 & 0.534031 & -116 & 721 & -837 \\
 192 & 2^6 3^1 & \text{N} & \text{N} & -15 & 10 & 2.3333333 & 0.463542 & 0.536458 & -131 & 721 & -852 \\
 193 & 193^1 & \text{Y} & \text{Y} & -2 & 0 & 1.0000000 & 0.461140 & 0.538860 & -133 & 721 & -854 \\
 194 & 2^1 97^1 & \text{Y} & \text{N} & 5 & 0 & 1.0000000 & 0.463918 & 0.536082 & -128 & 726 & -854 \\
 195 & 3^1 5^1 13^1 & \text{Y} & \text{N} & -16 & 0 & 1.0000000 & 0.461538 & 0.538462 & -144 & 726 & -870 \\
 196 & 2^2 7^2 & \text{N} & \text{N} & 14 & 9 & 1.3571429 & 0.464286 & 0.535714 & -130 & 740 & -870 \\
 197 & 197^1 & \text{Y} & \text{Y} & -2 & 0 & 1.0000000 & 0.461929 & 0.538071 & -132 & 740 & -872 \\
 198 & 2^1 3^2 11^1 & \text{N} & \text{N} & 30 & 14 & 1.1666667 & 0.464646 & 0.535354 & -102 & 770 & -872 \\
 199 & 199^1 & \text{Y} & \text{Y} & -2 & 0 & 1.0000000 & 0.462312 & 0.537688 & -104 & 770 & -874 \\
 200 & 2^3 5^2 & \text{N} & \text{N} & -23 & 18 & 1.4782609 & 0.460000 & 0.540000 & -127 & 770 & -897 \\ 
\end{array}
}
\end{equation*}
\clearpage 

\end{table} 

\newpage
\begin{table}[ht]

\centering

\tiny
\begin{equation*}
\boxed{
\begin{array}{cc|cc|ccc|cc|ccc}
 n & \mathbf{Primes} & \mathbf{Sqfree} & \mathbf{PPower} & g^{-1}(n) & 
 \lambda(n) g^{-1}(n) - \widehat{f}_1(n) & 
 \frac{\sum_{d|n} C_{\Omega(d)}(d)}{|g^{-1}(n)|} & 
 \mathcal{L}_{+}(n) & \mathcal{L}_{-}(n) & 
 G^{-1}(n) & G^{-1}_{+}(n) & G^{-1}_{-}(n) \\ \hline 
 201 & 3^1 67^1 & \text{Y} & \text{N} & 5 & 0 & 1.0000000 & 0.462687 & 0.537313 & -122 & 775 & -897 \\
 202 & 2^1 101^1 & \text{Y} & \text{N} & 5 & 0 & 1.0000000 & 0.465347 & 0.534653 & -117 & 780 & -897 \\
 203 & 7^1 29^1 & \text{Y} & \text{N} & 5 & 0 & 1.0000000 & 0.467980 & 0.532020 & -112 & 785 & -897 \\
 204 & 2^2 3^1 17^1 & \text{N} & \text{N} & 30 & 14 & 1.1666667 & 0.470588 & 0.529412 & -82 & 815 & -897 \\
 205 & 5^1 41^1 & \text{Y} & \text{N} & 5 & 0 & 1.0000000 & 0.473171 & 0.526829 & -77 & 820 & -897 \\
 206 & 2^1 103^1 & \text{Y} & \text{N} & 5 & 0 & 1.0000000 & 0.475728 & 0.524272 & -72 & 825 & -897 \\
 207 & 3^2 23^1 & \text{N} & \text{N} & -7 & 2 & 1.2857143 & 0.473430 & 0.526570 & -79 & 825 & -904 \\
 208 & 2^4 13^1 & \text{N} & \text{N} & -11 & 6 & 1.8181818 & 0.471154 & 0.528846 & -90 & 825 & -915 \\
 209 & 11^1 19^1 & \text{Y} & \text{N} & 5 & 0 & 1.0000000 & 0.473684 & 0.526316 & -85 & 830 & -915 \\
 210 & 2^1 3^1 5^1 7^1 & \text{Y} & \text{N} & 65 & 0 & 1.0000000 & 0.476190 & 0.523810 & -20 & 895 & -915 \\
 211 & 211^1 & \text{Y} & \text{Y} & -2 & 0 & 1.0000000 & 0.473934 & 0.526066 & -22 & 895 & -917 \\
 212 & 2^2 53^1 & \text{N} & \text{N} & -7 & 2 & 1.2857143 & 0.471698 & 0.528302 & -29 & 895 & -924 \\
 213 & 3^1 71^1 & \text{Y} & \text{N} & 5 & 0 & 1.0000000 & 0.474178 & 0.525822 & -24 & 900 & -924 \\
 214 & 2^1 107^1 & \text{Y} & \text{N} & 5 & 0 & 1.0000000 & 0.476636 & 0.523364 & -19 & 905 & -924 \\
 215 & 5^1 43^1 & \text{Y} & \text{N} & 5 & 0 & 1.0000000 & 0.479070 & 0.520930 & -14 & 910 & -924 \\
 216 & 2^3 3^3 & \text{N} & \text{N} & 46 & 41 & 1.5000000 & 0.481481 & 0.518519 & 32 & 956 & -924 \\
 217 & 7^1 31^1 & \text{Y} & \text{N} & 5 & 0 & 1.0000000 & 0.483871 & 0.516129 & 37 & 961 & -924 \\
 218 & 2^1 109^1 & \text{Y} & \text{N} & 5 & 0 & 1.0000000 & 0.486239 & 0.513761 & 42 & 966 & -924 \\
 219 & 3^1 73^1 & \text{Y} & \text{N} & 5 & 0 & 1.0000000 & 0.488584 & 0.511416 & 47 & 971 & -924 \\
 220 & 2^2 5^1 11^1 & \text{N} & \text{N} & 30 & 14 & 1.1666667 & 0.490909 & 0.509091 & 77 & 1001 & -924 \\
 221 & 13^1 17^1 & \text{Y} & \text{N} & 5 & 0 & 1.0000000 & 0.493213 & 0.506787 & 82 & 1006 & -924 \\
 222 & 2^1 3^1 37^1 & \text{Y} & \text{N} & -16 & 0 & 1.0000000 & 0.490991 & 0.509009 & 66 & 1006 & -940 \\
 223 & 223^1 & \text{Y} & \text{Y} & -2 & 0 & 1.0000000 & 0.488789 & 0.511211 & 64 & 1006 & -942 \\
 224 & 2^5 7^1 & \text{N} & \text{N} & 13 & 8 & 2.0769231 & 0.491071 & 0.508929 & 77 & 1019 & -942 \\
 225 & 3^2 5^2 & \text{N} & \text{N} & 14 & 9 & 1.3571429 & 0.493333 & 0.506667 & 91 & 1033 & -942 \\
 226 & 2^1 113^1 & \text{Y} & \text{N} & 5 & 0 & 1.0000000 & 0.495575 & 0.504425 & 96 & 1038 & -942 \\
 227 & 227^1 & \text{Y} & \text{Y} & -2 & 0 & 1.0000000 & 0.493392 & 0.506608 & 94 & 1038 & -944 \\
 228 & 2^2 3^1 19^1 & \text{N} & \text{N} & 30 & 14 & 1.1666667 & 0.495614 & 0.504386 & 124 & 1068 & -944 \\
 229 & 229^1 & \text{Y} & \text{Y} & -2 & 0 & 1.0000000 & 0.493450 & 0.506550 & 122 & 1068 & -946 \\
 230 & 2^1 5^1 23^1 & \text{Y} & \text{N} & -16 & 0 & 1.0000000 & 0.491304 & 0.508696 & 106 & 1068 & -962 \\
 231 & 3^1 7^1 11^1 & \text{Y} & \text{N} & -16 & 0 & 1.0000000 & 0.489177 & 0.510823 & 90 & 1068 & -978 \\
 232 & 2^3 29^1 & \text{N} & \text{N} & 9 & 4 & 1.5555556 & 0.491379 & 0.508621 & 99 & 1077 & -978 \\
 233 & 233^1 & \text{Y} & \text{Y} & -2 & 0 & 1.0000000 & 0.489270 & 0.510730 & 97 & 1077 & -980 \\
 234 & 2^1 3^2 13^1 & \text{N} & \text{N} & 30 & 14 & 1.1666667 & 0.491453 & 0.508547 & 127 & 1107 & -980 \\
 235 & 5^1 47^1 & \text{Y} & \text{N} & 5 & 0 & 1.0000000 & 0.493617 & 0.506383 & 132 & 1112 & -980 \\
 236 & 2^2 59^1 & \text{N} & \text{N} & -7 & 2 & 1.2857143 & 0.491525 & 0.508475 & 125 & 1112 & -987 \\
 237 & 3^1 79^1 & \text{Y} & \text{N} & 5 & 0 & 1.0000000 & 0.493671 & 0.506329 & 130 & 1117 & -987 \\
 238 & 2^1 7^1 17^1 & \text{Y} & \text{N} & -16 & 0 & 1.0000000 & 0.491597 & 0.508403 & 114 & 1117 & -1003 \\
 239 & 239^1 & \text{Y} & \text{Y} & -2 & 0 & 1.0000000 & 0.489540 & 0.510460 & 112 & 1117 & -1005 \\
 240 & 2^4 3^1 5^1 & \text{N} & \text{N} & 70 & 54 & 1.5000000 & 0.491667 & 0.508333 & 182 & 1187 & -1005 \\
 241 & 241^1 & \text{Y} & \text{Y} & -2 & 0 & 1.0000000 & 0.489627 & 0.510373 & 180 & 1187 & -1007 \\
 242 & 2^1 11^2 & \text{N} & \text{N} & -7 & 2 & 1.2857143 & 0.487603 & 0.512397 & 173 & 1187 & -1014 \\
 243 & 3^5 & \text{N} & \text{Y} & -2 & 0 & 3.0000000 & 0.485597 & 0.514403 & 171 & 1187 & -1016 \\
 244 & 2^2 61^1 & \text{N} & \text{N} & -7 & 2 & 1.2857143 & 0.483607 & 0.516393 & 164 & 1187 & -1023 \\
 245 & 5^1 7^2 & \text{N} & \text{N} & -7 & 2 & 1.2857143 & 0.481633 & 0.518367 & 157 & 1187 & -1030 \\
 246 & 2^1 3^1 41^1 & \text{Y} & \text{N} & -16 & 0 & 1.0000000 & 0.479675 & 0.520325 & 141 & 1187 & -1046 \\
 247 & 13^1 19^1 & \text{Y} & \text{N} & 5 & 0 & 1.0000000 & 0.481781 & 0.518219 & 146 & 1192 & -1046 \\
 248 & 2^3 31^1 & \text{N} & \text{N} & 9 & 4 & 1.5555556 & 0.483871 & 0.516129 & 155 & 1201 & -1046 \\
 249 & 3^1 83^1 & \text{Y} & \text{N} & 5 & 0 & 1.0000000 & 0.485944 & 0.514056 & 160 & 1206 & -1046 \\
 250 & 2^1 5^3 & \text{N} & \text{N} & 9 & 4 & 1.5555556 & 0.488000 & 0.512000 & 169 & 1215 & -1046 \\
 251 & 251^1 & \text{Y} & \text{Y} & -2 & 0 & 1.0000000 & 0.486056 & 0.513944 & 167 & 1215 & -1048 \\
 252 & 2^2 3^2 7^1 & \text{N} & \text{N} & -74 & 58 & 1.2162162 & 0.484127 & 0.515873 & 93 & 1215 & -1122 \\
 253 & 11^1 23^1 & \text{Y} & \text{N} & 5 & 0 & 1.0000000 & 0.486166 & 0.513834 & 98 & 1220 & -1122 \\
 254 & 2^1 127^1 & \text{Y} & \text{N} & 5 & 0 & 1.0000000 & 0.488189 & 0.511811 & 103 & 1225 & -1122 \\
 255 & 3^1 5^1 17^1 & \text{Y} & \text{N} & -16 & 0 & 1.0000000 & 0.486275 & 0.513725 & 87 & 1225 & -1138 \\
 256 & 2^8 & \text{N} & \text{Y} & 2 & 0 & 4.5000000 & 0.488281 & 0.511719 & 89 & 1227 & -1138 \\
 257 & 257^1 & \text{Y} & \text{Y} & -2 & 0 & 1.0000000 & 0.486381 & 0.513619 & 87 & 1227 & -1140 \\
 258 & 2^1 3^1 43^1 & \text{Y} & \text{N} & -16 & 0 & 1.0000000 & 0.484496 & 0.515504 & 71 & 1227 & -1156 \\
 259 & 7^1 37^1 & \text{Y} & \text{N} & 5 & 0 & 1.0000000 & 0.486486 & 0.513514 & 76 & 1232 & -1156 \\
 260 & 2^2 5^1 13^1 & \text{N} & \text{N} & 30 & 14 & 1.1666667 & 0.488462 & 0.511538 & 106 & 1262 & -1156 \\
 261 & 3^2 29^1 & \text{N} & \text{N} & -7 & 2 & 1.2857143 & 0.486590 & 0.513410 & 99 & 1262 & -1163 \\
 262 & 2^1 131^1 & \text{Y} & \text{N} & 5 & 0 & 1.0000000 & 0.488550 & 0.511450 & 104 & 1267 & -1163 \\
 263 & 263^1 & \text{Y} & \text{Y} & -2 & 0 & 1.0000000 & 0.486692 & 0.513308 & 102 & 1267 & -1165 \\
 264 & 2^3 3^1 11^1 & \text{N} & \text{N} & -48 & 32 & 1.3333333 & 0.484848 & 0.515152 & 54 & 1267 & -1213 \\
 265 & 5^1 53^1 & \text{Y} & \text{N} & 5 & 0 & 1.0000000 & 0.486792 & 0.513208 & 59 & 1272 & -1213 \\
 266 & 2^1 7^1 19^1 & \text{Y} & \text{N} & -16 & 0 & 1.0000000 & 0.484962 & 0.515038 & 43 & 1272 & -1229 \\
 267 & 3^1 89^1 & \text{Y} & \text{N} & 5 & 0 & 1.0000000 & 0.486891 & 0.513109 & 48 & 1277 & -1229 \\
 268 & 2^2 67^1 & \text{N} & \text{N} & -7 & 2 & 1.2857143 & 0.485075 & 0.514925 & 41 & 1277 & -1236 \\
 269 & 269^1 & \text{Y} & \text{Y} & -2 & 0 & 1.0000000 & 0.483271 & 0.516729 & 39 & 1277 & -1238 \\
 270 & 2^1 3^3 5^1 & \text{N} & \text{N} & -48 & 32 & 1.3333333 & 0.481481 & 0.518519 & -9 & 1277 & -1286 \\
 271 & 271^1 & \text{Y} & \text{Y} & -2 & 0 & 1.0000000 & 0.479705 & 0.520295 & -11 & 1277 & -1288 \\
 272 & 2^4 17^1 & \text{N} & \text{N} & -11 & 6 & 1.8181818 & 0.477941 & 0.522059 & -22 & 1277 & -1299 \\
 273 & 3^1 7^1 13^1 & \text{Y} & \text{N} & -16 & 0 & 1.0000000 & 0.476190 & 0.523810 & -38 & 1277 & -1315 \\
 274 & 2^1 137^1 & \text{Y} & \text{N} & 5 & 0 & 1.0000000 & 0.478102 & 0.521898 & -33 & 1282 & -1315 \\
 275 & 5^2 11^1 & \text{N} & \text{N} & -7 & 2 & 1.2857143 & 0.476364 & 0.523636 & -40 & 1282 & -1322 \\
 276 & 2^2 3^1 23^1 & \text{N} & \text{N} & 30 & 14 & 1.1666667 & 0.478261 & 0.521739 & -10 & 1312 & -1322 \\
 277 & 277^1 & \text{Y} & \text{Y} & -2 & 0 & 1.0000000 & 0.476534 & 0.523466 & -12 & 1312 & -1324 \\ 
\end{array}
}
\end{equation*}
\clearpage 

\end{table} 

\newpage
\begin{table}[ht]

\centering

\tiny
\begin{equation*}
\boxed{
\begin{array}{cc|cc|ccc|cc|ccc}
 n & \mathbf{Primes} & \mathbf{Sqfree} & \mathbf{PPower} & g^{-1}(n) & 
 \lambda(n) g^{-1}(n) - \widehat{f}_1(n) & 
 \frac{\sum_{d|n} C_{\Omega(d)}(d)}{|g^{-1}(n)|} & 
 \mathcal{L}_{+}(n) & \mathcal{L}_{-}(n) & 
 G^{-1}(n) & G^{-1}_{+}(n) & G^{-1}_{-}(n) \\ \hline 
 278 & 2^1 139^1 & \text{Y} & \text{N} & 5 & 0 & 1.0000000 & 0.478417 & 0.521583 & -7 & 1317 & -1324 \\
 279 & 3^2 31^1 & \text{N} & \text{N} & -7 & 2 & 1.2857143 & 0.476703 & 0.523297 & -14 & 1317 & -1331 \\
 280 & 2^3 5^1 7^1 & \text{N} & \text{N} & -48 & 32 & 1.3333333 & 0.475000 & 0.525000 & -62 & 1317 & -1379 \\
 281 & 281^1 & \text{Y} & \text{Y} & -2 & 0 & 1.0000000 & 0.473310 & 0.526690 & -64 & 1317 & -1381 \\
 282 & 2^1 3^1 47^1 & \text{Y} & \text{N} & -16 & 0 & 1.0000000 & 0.471631 & 0.528369 & -80 & 1317 & -1397 \\
 283 & 283^1 & \text{Y} & \text{Y} & -2 & 0 & 1.0000000 & 0.469965 & 0.530035 & -82 & 1317 & -1399 \\
 284 & 2^2 71^1 & \text{N} & \text{N} & -7 & 2 & 1.2857143 & 0.468310 & 0.531690 & -89 & 1317 & -1406 \\
 285 & 3^1 5^1 19^1 & \text{Y} & \text{N} & -16 & 0 & 1.0000000 & 0.466667 & 0.533333 & -105 & 1317 & -1422 \\
 286 & 2^1 11^1 13^1 & \text{Y} & \text{N} & -16 & 0 & 1.0000000 & 0.465035 & 0.534965 & -121 & 1317 & -1438 \\
 287 & 7^1 41^1 & \text{Y} & \text{N} & 5 & 0 & 1.0000000 & 0.466899 & 0.533101 & -116 & 1322 & -1438 \\
 288 & 2^5 3^2 & \text{N} & \text{N} & -47 & 42 & 1.7659574 & 0.465278 & 0.534722 & -163 & 1322 & -1485 \\
 289 & 17^2 & \text{N} & \text{Y} & 2 & 0 & 1.5000000 & 0.467128 & 0.532872 & -161 & 1324 & -1485 \\
 290 & 2^1 5^1 29^1 & \text{Y} & \text{N} & -16 & 0 & 1.0000000 & 0.465517 & 0.534483 & -177 & 1324 & -1501 \\
 291 & 3^1 97^1 & \text{Y} & \text{N} & 5 & 0 & 1.0000000 & 0.467354 & 0.532646 & -172 & 1329 & -1501 \\
 292 & 2^2 73^1 & \text{N} & \text{N} & -7 & 2 & 1.2857143 & 0.465753 & 0.534247 & -179 & 1329 & -1508 \\
 293 & 293^1 & \text{Y} & \text{Y} & -2 & 0 & 1.0000000 & 0.464164 & 0.535836 & -181 & 1329 & -1510 \\
 294 & 2^1 3^1 7^2 & \text{N} & \text{N} & 30 & 14 & 1.1666667 & 0.465986 & 0.534014 & -151 & 1359 & -1510 \\
 295 & 5^1 59^1 & \text{Y} & \text{N} & 5 & 0 & 1.0000000 & 0.467797 & 0.532203 & -146 & 1364 & -1510 \\
 296 & 2^3 37^1 & \text{N} & \text{N} & 9 & 4 & 1.5555556 & 0.469595 & 0.530405 & -137 & 1373 & -1510 \\
 297 & 3^3 11^1 & \text{N} & \text{N} & 9 & 4 & 1.5555556 & 0.471380 & 0.528620 & -128 & 1382 & -1510 \\
 298 & 2^1 149^1 & \text{Y} & \text{N} & 5 & 0 & 1.0000000 & 0.473154 & 0.526846 & -123 & 1387 & -1510 \\
 299 & 13^1 23^1 & \text{Y} & \text{N} & 5 & 0 & 1.0000000 & 0.474916 & 0.525084 & -118 & 1392 & -1510 \\
 300 & 2^2 3^1 5^2 & \text{N} & \text{N} & -74 & 58 & 1.2162162 & 0.473333 & 0.526667 & -192 & 1392 & -1584 \\
 301 & 7^1 43^1 & \text{Y} & \text{N} & 5 & 0 & 1.0000000 & 0.475083 & 0.524917 & -187 & 1397 & -1584 \\
 302 & 2^1 151^1 & \text{Y} & \text{N} & 5 & 0 & 1.0000000 & 0.476821 & 0.523179 & -182 & 1402 & -1584 \\
 303 & 3^1 101^1 & \text{Y} & \text{N} & 5 & 0 & 1.0000000 & 0.478548 & 0.521452 & -177 & 1407 & -1584 \\
 304 & 2^4 19^1 & \text{N} & \text{N} & -11 & 6 & 1.8181818 & 0.476974 & 0.523026 & -188 & 1407 & -1595 \\
 305 & 5^1 61^1 & \text{Y} & \text{N} & 5 & 0 & 1.0000000 & 0.478689 & 0.521311 & -183 & 1412 & -1595 \\
 306 & 2^1 3^2 17^1 & \text{N} & \text{N} & 30 & 14 & 1.1666667 & 0.480392 & 0.519608 & -153 & 1442 & -1595 \\
 307 & 307^1 & \text{Y} & \text{Y} & -2 & 0 & 1.0000000 & 0.478827 & 0.521173 & -155 & 1442 & -1597 \\
 308 & 2^2 7^1 11^1 & \text{N} & \text{N} & 30 & 14 & 1.1666667 & 0.480519 & 0.519481 & -125 & 1472 & -1597 \\
 309 & 3^1 103^1 & \text{Y} & \text{N} & 5 & 0 & 1.0000000 & 0.482201 & 0.517799 & -120 & 1477 & -1597 \\
 310 & 2^1 5^1 31^1 & \text{Y} & \text{N} & -16 & 0 & 1.0000000 & 0.480645 & 0.519355 & -136 & 1477 & -1613 \\
 311 & 311^1 & \text{Y} & \text{Y} & -2 & 0 & 1.0000000 & 0.479100 & 0.520900 & -138 & 1477 & -1615 \\
 312 & 2^3 3^1 13^1 & \text{N} & \text{N} & -48 & 32 & 1.3333333 & 0.477564 & 0.522436 & -186 & 1477 & -1663 \\
 313 & 313^1 & \text{Y} & \text{Y} & -2 & 0 & 1.0000000 & 0.476038 & 0.523962 & -188 & 1477 & -1665 \\
 314 & 2^1 157^1 & \text{Y} & \text{N} & 5 & 0 & 1.0000000 & 0.477707 & 0.522293 & -183 & 1482 & -1665 \\
 315 & 3^2 5^1 7^1 & \text{N} & \text{N} & 30 & 14 & 1.1666667 & 0.479365 & 0.520635 & -153 & 1512 & -1665 \\
 316 & 2^2 79^1 & \text{N} & \text{N} & -7 & 2 & 1.2857143 & 0.477848 & 0.522152 & -160 & 1512 & -1672 \\
 317 & 317^1 & \text{Y} & \text{Y} & -2 & 0 & 1.0000000 & 0.476341 & 0.523659 & -162 & 1512 & -1674 \\
 318 & 2^1 3^1 53^1 & \text{Y} & \text{N} & -16 & 0 & 1.0000000 & 0.474843 & 0.525157 & -178 & 1512 & -1690 \\
 319 & 11^1 29^1 & \text{Y} & \text{N} & 5 & 0 & 1.0000000 & 0.476489 & 0.523511 & -173 & 1517 & -1690 \\
 320 & 2^6 5^1 & \text{N} & \text{N} & -15 & 10 & 2.3333333 & 0.475000 & 0.525000 & -188 & 1517 & -1705 \\
 321 & 3^1 107^1 & \text{Y} & \text{N} & 5 & 0 & 1.0000000 & 0.476636 & 0.523364 & -183 & 1522 & -1705 \\
 322 & 2^1 7^1 23^1 & \text{Y} & \text{N} & -16 & 0 & 1.0000000 & 0.475155 & 0.524845 & -199 & 1522 & -1721 \\
 323 & 17^1 19^1 & \text{Y} & \text{N} & 5 & 0 & 1.0000000 & 0.476780 & 0.523220 & -194 & 1527 & -1721 \\
 324 & 2^2 3^4 & \text{N} & \text{N} & 34 & 29 & 1.6176471 & 0.478395 & 0.521605 & -160 & 1561 & -1721 \\
 325 & 5^2 13^1 & \text{N} & \text{N} & -7 & 2 & 1.2857143 & 0.476923 & 0.523077 & -167 & 1561 & -1728 \\
 326 & 2^1 163^1 & \text{Y} & \text{N} & 5 & 0 & 1.0000000 & 0.478528 & 0.521472 & -162 & 1566 & -1728 \\
 327 & 3^1 109^1 & \text{Y} & \text{N} & 5 & 0 & 1.0000000 & 0.480122 & 0.519878 & -157 & 1571 & -1728 \\
 328 & 2^3 41^1 & \text{N} & \text{N} & 9 & 4 & 1.5555556 & 0.481707 & 0.518293 & -148 & 1580 & -1728 \\
 329 & 7^1 47^1 & \text{Y} & \text{N} & 5 & 0 & 1.0000000 & 0.483283 & 0.516717 & -143 & 1585 & -1728 \\
 330 & 2^1 3^1 5^1 11^1 & \text{Y} & \text{N} & 65 & 0 & 1.0000000 & 0.484848 & 0.515152 & -78 & 1650 & -1728 \\
 331 & 331^1 & \text{Y} & \text{Y} & -2 & 0 & 1.0000000 & 0.483384 & 0.516616 & -80 & 1650 & -1730 \\
 332 & 2^2 83^1 & \text{N} & \text{N} & -7 & 2 & 1.2857143 & 0.481928 & 0.518072 & -87 & 1650 & -1737 \\
 333 & 3^2 37^1 & \text{N} & \text{N} & -7 & 2 & 1.2857143 & 0.480480 & 0.519520 & -94 & 1650 & -1744 \\
 334 & 2^1 167^1 & \text{Y} & \text{N} & 5 & 0 & 1.0000000 & 0.482036 & 0.517964 & -89 & 1655 & -1744 \\
 335 & 5^1 67^1 & \text{Y} & \text{N} & 5 & 0 & 1.0000000 & 0.483582 & 0.516418 & -84 & 1660 & -1744 \\
 336 & 2^4 3^1 7^1 & \text{N} & \text{N} & 70 & 54 & 1.5000000 & 0.485119 & 0.514881 & -14 & 1730 & -1744 \\
 337 & 337^1 & \text{Y} & \text{Y} & -2 & 0 & 1.0000000 & 0.483680 & 0.516320 & -16 & 1730 & -1746 \\
 338 & 2^1 13^2 & \text{N} & \text{N} & -7 & 2 & 1.2857143 & 0.482249 & 0.517751 & -23 & 1730 & -1753 \\
 339 & 3^1 113^1 & \text{Y} & \text{N} & 5 & 0 & 1.0000000 & 0.483776 & 0.516224 & -18 & 1735 & -1753 \\
 340 & 2^2 5^1 17^1 & \text{N} & \text{N} & 30 & 14 & 1.1666667 & 0.485294 & 0.514706 & 12 & 1765 & -1753 \\
 341 & 11^1 31^1 & \text{Y} & \text{N} & 5 & 0 & 1.0000000 & 0.486804 & 0.513196 & 17 & 1770 & -1753 \\
 342 & 2^1 3^2 19^1 & \text{N} & \text{N} & 30 & 14 & 1.1666667 & 0.488304 & 0.511696 & 47 & 1800 & -1753 \\
 343 & 7^3 & \text{N} & \text{Y} & -2 & 0 & 2.0000000 & 0.486880 & 0.513120 & 45 & 1800 & -1755 \\
 344 & 2^3 43^1 & \text{N} & \text{N} & 9 & 4 & 1.5555556 & 0.488372 & 0.511628 & 54 & 1809 & -1755 \\
 345 & 3^1 5^1 23^1 & \text{Y} & \text{N} & -16 & 0 & 1.0000000 & 0.486957 & 0.513043 & 38 & 1809 & -1771 \\
 346 & 2^1 173^1 & \text{Y} & \text{N} & 5 & 0 & 1.0000000 & 0.488439 & 0.511561 & 43 & 1814 & -1771 \\
 347 & 347^1 & \text{Y} & \text{Y} & -2 & 0 & 1.0000000 & 0.487032 & 0.512968 & 41 & 1814 & -1773 \\
 348 & 2^2 3^1 29^1 & \text{N} & \text{N} & 30 & 14 & 1.1666667 & 0.488506 & 0.511494 & 71 & 1844 & -1773 \\
 349 & 349^1 & \text{Y} & \text{Y} & -2 & 0 & 1.0000000 & 0.487106 & 0.512894 & 69 & 1844 & -1775 \\
 350 & 2^1 5^2 7^1 & \text{N} & \text{N} & 30 & 14 & 1.1666667 & 0.488571 & 0.511429 & 99 & 1874 & -1775 \\ 
\end{array}
}
\end{equation*}
\clearpage 

\end{table} 

\newpage
\begin{table}[ht]

\centering
\tiny
\begin{equation*}
\boxed{
\begin{array}{cc|cc|ccc|cc|ccc}
 n & \mathbf{Primes} & \mathbf{Sqfree} & \mathbf{PPower} & g^{-1}(n) & 
 \lambda(n) g^{-1}(n) - \widehat{f}_1(n) & 
 \frac{\sum_{d|n} C_{\Omega(d)}(d)}{|g^{-1}(n)|} & 
 \mathcal{L}_{+}(n) & \mathcal{L}_{-}(n) & 
 G^{-1}(n) & G^{-1}_{+}(n) & G^{-1}_{-}(n) \\ \hline 
 351 & 3^3 13^1 & \text{N} & \text{N} & 9 & 4 & 1.5555556 & 0.490028 & 0.509972 & 108 & 1883 & -1775 \\
 352 & 2^5 11^1 & \text{N} & \text{N} & 13 & 8 & 2.0769231 & 0.491477 & 0.508523 & 121 & 1896 & -1775 \\
 353 & 353^1 & \text{Y} & \text{Y} & -2 & 0 & 1.0000000 & 0.490085 & 0.509915 & 119 & 1896 & -1777 \\
 354 & 2^1 3^1 59^1 & \text{Y} & \text{N} & -16 & 0 & 1.0000000 & 0.488701 & 0.511299 & 103 & 1896 & -1793 \\
 355 & 5^1 71^1 & \text{Y} & \text{N} & 5 & 0 & 1.0000000 & 0.490141 & 0.509859 & 108 & 1901 & -1793 \\
 356 & 2^2 89^1 & \text{N} & \text{N} & -7 & 2 & 1.2857143 & 0.488764 & 0.511236 & 101 & 1901 & -1800 \\
 357 & 3^1 7^1 17^1 & \text{Y} & \text{N} & -16 & 0 & 1.0000000 & 0.487395 & 0.512605 & 85 & 1901 & -1816 \\
 358 & 2^1 179^1 & \text{Y} & \text{N} & 5 & 0 & 1.0000000 & 0.488827 & 0.511173 & 90 & 1906 & -1816 \\
 359 & 359^1 & \text{Y} & \text{Y} & -2 & 0 & 1.0000000 & 0.487465 & 0.512535 & 88 & 1906 & -1818 \\
 360 & 2^3 3^2 5^1 & \text{N} & \text{N} & 145 & 129 & 1.3034483 & 0.488889 & 0.511111 & 233 & 2051 & -1818 \\
 361 & 19^2 & \text{N} & \text{Y} & 2 & 0 & 1.5000000 & 0.490305 & 0.509695 & 235 & 2053 & -1818 \\
 362 & 2^1 181^1 & \text{Y} & \text{N} & 5 & 0 & 1.0000000 & 0.491713 & 0.508287 & 240 & 2058 & -1818 \\
 363 & 3^1 11^2 & \text{N} & \text{N} & -7 & 2 & 1.2857143 & 0.490358 & 0.509642 & 233 & 2058 & -1825 \\
 364 & 2^2 7^1 13^1 & \text{N} & \text{N} & 30 & 14 & 1.1666667 & 0.491758 & 0.508242 & 263 & 2088 & -1825 \\
 365 & 5^1 73^1 & \text{Y} & \text{N} & 5 & 0 & 1.0000000 & 0.493151 & 0.506849 & 268 & 2093 & -1825 \\
 366 & 2^1 3^1 61^1 & \text{Y} & \text{N} & -16 & 0 & 1.0000000 & 0.491803 & 0.508197 & 252 & 2093 & -1841 \\
 367 & 367^1 & \text{Y} & \text{Y} & -2 & 0 & 1.0000000 & 0.490463 & 0.509537 & 250 & 2093 & -1843 \\
 368 & 2^4 23^1 & \text{N} & \text{N} & -11 & 6 & 1.8181818 & 0.489130 & 0.510870 & 239 & 2093 & -1854 \\
 369 & 3^2 41^1 & \text{N} & \text{N} & -7 & 2 & 1.2857143 & 0.487805 & 0.512195 & 232 & 2093 & -1861 \\
 370 & 2^1 5^1 37^1 & \text{Y} & \text{N} & -16 & 0 & 1.0000000 & 0.486486 & 0.513514 & 216 & 2093 & -1877 \\
 371 & 7^1 53^1 & \text{Y} & \text{N} & 5 & 0 & 1.0000000 & 0.487871 & 0.512129 & 221 & 2098 & -1877 \\
 372 & 2^2 3^1 31^1 & \text{N} & \text{N} & 30 & 14 & 1.1666667 & 0.489247 & 0.510753 & 251 & 2128 & -1877 \\
 373 & 373^1 & \text{Y} & \text{Y} & -2 & 0 & 1.0000000 & 0.487936 & 0.512064 & 249 & 2128 & -1879 \\
 374 & 2^1 11^1 17^1 & \text{Y} & \text{N} & -16 & 0 & 1.0000000 & 0.486631 & 0.513369 & 233 & 2128 & -1895 \\
 375 & 3^1 5^3 & \text{N} & \text{N} & 9 & 4 & 1.5555556 & 0.488000 & 0.512000 & 242 & 2137 & -1895 \\
 376 & 2^3 47^1 & \text{N} & \text{N} & 9 & 4 & 1.5555556 & 0.489362 & 0.510638 & 251 & 2146 & -1895 \\
 377 & 13^1 29^1 & \text{Y} & \text{N} & 5 & 0 & 1.0000000 & 0.490716 & 0.509284 & 256 & 2151 & -1895 \\
 378 & 2^1 3^3 7^1 & \text{N} & \text{N} & -48 & 32 & 1.3333333 & 0.489418 & 0.510582 & 208 & 2151 & -1943 \\
 379 & 379^1 & \text{Y} & \text{Y} & -2 & 0 & 1.0000000 & 0.488127 & 0.511873 & 206 & 2151 & -1945 \\
 380 & 2^2 5^1 19^1 & \text{N} & \text{N} & 30 & 14 & 1.1666667 & 0.489474 & 0.510526 & 236 & 2181 & -1945 \\
 381 & 3^1 127^1 & \text{Y} & \text{N} & 5 & 0 & 1.0000000 & 0.490814 & 0.509186 & 241 & 2186 & -1945 \\
 382 & 2^1 191^1 & \text{Y} & \text{N} & 5 & 0 & 1.0000000 & 0.492147 & 0.507853 & 246 & 2191 & -1945 \\
 383 & 383^1 & \text{Y} & \text{Y} & -2 & 0 & 1.0000000 & 0.490862 & 0.509138 & 244 & 2191 & -1947 \\
 384 & 2^7 3^1 & \text{N} & \text{N} & 17 & 12 & 2.5882353 & 0.492188 & 0.507812 & 261 & 2208 & -1947 \\
 385 & 5^1 7^1 11^1 & \text{Y} & \text{N} & -16 & 0 & 1.0000000 & 0.490909 & 0.509091 & 245 & 2208 & -1963 \\
 386 & 2^1 193^1 & \text{Y} & \text{N} & 5 & 0 & 1.0000000 & 0.492228 & 0.507772 & 250 & 2213 & -1963 \\
 387 & 3^2 43^1 & \text{N} & \text{N} & -7 & 2 & 1.2857143 & 0.490956 & 0.509044 & 243 & 2213 & -1970 \\
 388 & 2^2 97^1 & \text{N} & \text{N} & -7 & 2 & 1.2857143 & 0.489691 & 0.510309 & 236 & 2213 & -1977 \\
 389 & 389^1 & \text{Y} & \text{Y} & -2 & 0 & 1.0000000 & 0.488432 & 0.511568 & 234 & 2213 & -1979 \\
 390 & 2^1 3^1 5^1 13^1 & \text{Y} & \text{N} & 65 & 0 & 1.0000000 & 0.489744 & 0.510256 & 299 & 2278 & -1979 \\
 391 & 17^1 23^1 & \text{Y} & \text{N} & 5 & 0 & 1.0000000 & 0.491049 & 0.508951 & 304 & 2283 & -1979 \\
 392 & 2^3 7^2 & \text{N} & \text{N} & -23 & 18 & 1.4782609 & 0.489796 & 0.510204 & 281 & 2283 & -2002 \\
 393 & 3^1 131^1 & \text{Y} & \text{N} & 5 & 0 & 1.0000000 & 0.491094 & 0.508906 & 286 & 2288 & -2002 \\
 394 & 2^1 197^1 & \text{Y} & \text{N} & 5 & 0 & 1.0000000 & 0.492386 & 0.507614 & 291 & 2293 & -2002 \\
 395 & 5^1 79^1 & \text{Y} & \text{N} & 5 & 0 & 1.0000000 & 0.493671 & 0.506329 & 296 & 2298 & -2002 \\
 396 & 2^2 3^2 11^1 & \text{N} & \text{N} & -74 & 58 & 1.2162162 & 0.492424 & 0.507576 & 222 & 2298 & -2076 \\
 397 & 397^1 & \text{Y} & \text{Y} & -2 & 0 & 1.0000000 & 0.491184 & 0.508816 & 220 & 2298 & -2078 \\
 398 & 2^1 199^1 & \text{Y} & \text{N} & 5 & 0 & 1.0000000 & 0.492462 & 0.507538 & 225 & 2303 & -2078 \\
 399 & 3^1 7^1 19^1 & \text{Y} & \text{N} & -16 & 0 & 1.0000000 & 0.491228 & 0.508772 & 209 & 2303 & -2094 \\
 400 & 2^4 5^2 & \text{N} & \text{N} & 34 & 29 & 1.6176471 & 0.492500 & 0.507500 & 243 & 2337 & -2094 \\
 401 & 401^1 & \text{Y} & \text{Y} & -2 & 0 & 1.0000000 & 0.491272 & 0.508728 & 241 & 2337 & -2096 \\
 402 & 2^1 3^1 67^1 & \text{Y} & \text{N} & -16 & 0 & 1.0000000 & 0.490050 & 0.509950 & 225 & 2337 & -2112 \\
 403 & 13^1 31^1 & \text{Y} & \text{N} & 5 & 0 & 1.0000000 & 0.491315 & 0.508685 & 230 & 2342 & -2112 \\
 404 & 2^2 101^1 & \text{N} & \text{N} & -7 & 2 & 1.2857143 & 0.490099 & 0.509901 & 223 & 2342 & -2119 \\
 405 & 3^4 5^1 & \text{N} & \text{N} & -11 & 6 & 1.8181818 & 0.488889 & 0.511111 & 212 & 2342 & -2130 \\
 406 & 2^1 7^1 29^1 & \text{Y} & \text{N} & -16 & 0 & 1.0000000 & 0.487685 & 0.512315 & 196 & 2342 & -2146 \\
 407 & 11^1 37^1 & \text{Y} & \text{N} & 5 & 0 & 1.0000000 & 0.488943 & 0.511057 & 201 & 2347 & -2146 \\
 408 & 2^3 3^1 17^1 & \text{N} & \text{N} & -48 & 32 & 1.3333333 & 0.487745 & 0.512255 & 153 & 2347 & -2194 \\
 409 & 409^1 & \text{Y} & \text{Y} & -2 & 0 & 1.0000000 & 0.486553 & 0.513447 & 151 & 2347 & -2196 \\
 410 & 2^1 5^1 41^1 & \text{Y} & \text{N} & -16 & 0 & 1.0000000 & 0.485366 & 0.514634 & 135 & 2347 & -2212 \\
 411 & 3^1 137^1 & \text{Y} & \text{N} & 5 & 0 & 1.0000000 & 0.486618 & 0.513382 & 140 & 2352 & -2212 \\
 412 & 2^2 103^1 & \text{N} & \text{N} & -7 & 2 & 1.2857143 & 0.485437 & 0.514563 & 133 & 2352 & -2219 \\
 413 & 7^1 59^1 & \text{Y} & \text{N} & 5 & 0 & 1.0000000 & 0.486683 & 0.513317 & 138 & 2357 & -2219 \\
 414 & 2^1 3^2 23^1 & \text{N} & \text{N} & 30 & 14 & 1.1666667 & 0.487923 & 0.512077 & 168 & 2387 & -2219 \\
 415 & 5^1 83^1 & \text{Y} & \text{N} & 5 & 0 & 1.0000000 & 0.489157 & 0.510843 & 173 & 2392 & -2219 \\
 416 & 2^5 13^1 & \text{N} & \text{N} & 13 & 8 & 2.0769231 & 0.490385 & 0.509615 & 186 & 2405 & -2219 \\
 417 & 3^1 139^1 & \text{Y} & \text{N} & 5 & 0 & 1.0000000 & 0.491607 & 0.508393 & 191 & 2410 & -2219 \\
 418 & 2^1 11^1 19^1 & \text{Y} & \text{N} & -16 & 0 & 1.0000000 & 0.490431 & 0.509569 & 175 & 2410 & -2235 \\
 419 & 419^1 & \text{Y} & \text{Y} & -2 & 0 & 1.0000000 & 0.489260 & 0.510740 & 173 & 2410 & -2237 \\
 420 & 2^2 3^1 5^1 7^1 & \text{N} & \text{N} & -155 & 90 & 1.1032258 & 0.488095 & 0.511905 & 18 & 2410 & -2392 \\
 421 & 421^1 & \text{Y} & \text{Y} & -2 & 0 & 1.0000000 & 0.486936 & 0.513064 & 16 & 2410 & -2394 \\
 422 & 2^1 211^1 & \text{Y} & \text{N} & 5 & 0 & 1.0000000 & 0.488152 & 0.511848 & 21 & 2415 & -2394 \\
 423 & 3^2 47^1 & \text{N} & \text{N} & -7 & 2 & 1.2857143 & 0.486998 & 0.513002 & 14 & 2415 & -2401 \\
 424 & 2^3 53^1 & \text{N} & \text{N} & 9 & 4 & 1.5555556 & 0.488208 & 0.511792 & 23 & 2424 & -2401 \\
 425 & 5^2 17^1 & \text{N} & \text{N} & -7 & 2 & 1.2857143 & 0.487059 & 0.512941 & 16 & 2424 & -2408 \\ 
\end{array}
}
\end{equation*}
\clearpage 

\end{table} 

\newpage

\begin{table}[ht]
\label{table_conjecture_Mertens_ginvSeq_approx_values_LastPage} 

\centering
\tiny
\begin{equation*}
\boxed{
\begin{array}{cc|cc|ccc|cc|ccc}
 n & \mathbf{Primes} & \mathbf{Sqfree} & \mathbf{PPower} & g^{-1}(n) & 
 \lambda(n) g^{-1}(n) - \widehat{f}_1(n) & 
 \frac{\sum_{d|n} C_{\Omega(d)}(d)}{|g^{-1}(n)|} & 
 \mathcal{L}_{+}(n) & \mathcal{L}_{-}(n) & 
 G^{-1}(n) & G^{-1}_{+}(n) & G^{-1}_{-}(n) \\ \hline 
 426 & 2^1 3^1 71^1 & \text{Y} & \text{N} & -16 & 0 & 1.0000000 & 0.485915 & 0.514085 & 0 & 2424 & -2424 \\
 427 & 7^1 61^1 & \text{Y} & \text{N} & 5 & 0 & 1.0000000 & 0.487119 & 0.512881 & 5 & 2429 & -2424 \\
 428 & 2^2 107^1 & \text{N} & \text{N} & -7 & 2 & 1.2857143 & 0.485981 & 0.514019 & -2 & 2429 & -2431 \\
 429 & 3^1 11^1 13^1 & \text{Y} & \text{N} & -16 & 0 & 1.0000000 & 0.484848 & 0.515152 & -18 & 2429 & -2447 \\
 430 & 2^1 5^1 43^1 & \text{Y} & \text{N} & -16 & 0 & 1.0000000 & 0.483721 & 0.516279 & -34 & 2429 & -2463 \\
 431 & 431^1 & \text{Y} & \text{Y} & -2 & 0 & 1.0000000 & 0.482599 & 0.517401 & -36 & 2429 & -2465 \\
 432 & 2^4 3^3 & \text{N} & \text{N} & -80 & 75 & 1.5625000 & 0.481481 & 0.518519 & -116 & 2429 & -2545 \\
 433 & 433^1 & \text{Y} & \text{Y} & -2 & 0 & 1.0000000 & 0.480370 & 0.519630 & -118 & 2429 & -2547 \\
 434 & 2^1 7^1 31^1 & \text{Y} & \text{N} & -16 & 0 & 1.0000000 & 0.479263 & 0.520737 & -134 & 2429 & -2563 \\
 435 & 3^1 5^1 29^1 & \text{Y} & \text{N} & -16 & 0 & 1.0000000 & 0.478161 & 0.521839 & -150 & 2429 & -2579 \\
 436 & 2^2 109^1 & \text{N} & \text{N} & -7 & 2 & 1.2857143 & 0.477064 & 0.522936 & -157 & 2429 & -2586 \\
 437 & 19^1 23^1 & \text{Y} & \text{N} & 5 & 0 & 1.0000000 & 0.478261 & 0.521739 & -152 & 2434 & -2586 \\
 438 & 2^1 3^1 73^1 & \text{Y} & \text{N} & -16 & 0 & 1.0000000 & 0.477169 & 0.522831 & -168 & 2434 & -2602 \\
 439 & 439^1 & \text{Y} & \text{Y} & -2 & 0 & 1.0000000 & 0.476082 & 0.523918 & -170 & 2434 & -2604 \\
 440 & 2^3 5^1 11^1 & \text{N} & \text{N} & -48 & 32 & 1.3333333 & 0.475000 & 0.525000 & -218 & 2434 & -2652 \\
 441 & 3^2 7^2 & \text{N} & \text{N} & 14 & 9 & 1.3571429 & 0.476190 & 0.523810 & -204 & 2448 & -2652 \\
 442 & 2^1 13^1 17^1 & \text{Y} & \text{N} & -16 & 0 & 1.0000000 & 0.475113 & 0.524887 & -220 & 2448 & -2668 \\
 443 & 443^1 & \text{Y} & \text{Y} & -2 & 0 & 1.0000000 & 0.474041 & 0.525959 & -222 & 2448 & -2670 \\
 444 & 2^2 3^1 37^1 & \text{N} & \text{N} & 30 & 14 & 1.1666667 & 0.475225 & 0.524775 & -192 & 2478 & -2670 \\
 445 & 5^1 89^1 & \text{Y} & \text{N} & 5 & 0 & 1.0000000 & 0.476404 & 0.523596 & -187 & 2483 & -2670 \\
 446 & 2^1 223^1 & \text{Y} & \text{N} & 5 & 0 & 1.0000000 & 0.477578 & 0.522422 & -182 & 2488 & -2670 \\
 447 & 3^1 149^1 & \text{Y} & \text{N} & 5 & 0 & 1.0000000 & 0.478747 & 0.521253 & -177 & 2493 & -2670 \\
 448 & 2^6 7^1 & \text{N} & \text{N} & -15 & 10 & 2.3333333 & 0.477679 & 0.522321 & -192 & 2493 & -2685 \\
 449 & 449^1 & \text{Y} & \text{Y} & -2 & 0 & 1.0000000 & 0.476615 & 0.523385 & -194 & 2493 & -2687 \\
 450 & 2^1 3^2 5^2 & \text{N} & \text{N} & -74 & 58 & 1.2162162 & 0.475556 & 0.524444 & -268 & 2493 & -2761 \\
 451 & 11^1 41^1 & \text{Y} & \text{N} & 5 & 0 & 1.0000000 & 0.476718 & 0.523282 & -263 & 2498 & -2761 \\
 452 & 2^2 113^1 & \text{N} & \text{N} & -7 & 2 & 1.2857143 & 0.475664 & 0.524336 & -270 & 2498 & -2768 \\
 453 & 3^1 151^1 & \text{Y} & \text{N} & 5 & 0 & 1.0000000 & 0.476821 & 0.523179 & -265 & 2503 & -2768 \\
 454 & 2^1 227^1 & \text{Y} & \text{N} & 5 & 0 & 1.0000000 & 0.477974 & 0.522026 & -260 & 2508 & -2768 \\
 455 & 5^1 7^1 13^1 & \text{Y} & \text{N} & -16 & 0 & 1.0000000 & 0.476923 & 0.523077 & -276 & 2508 & -2784 \\
 456 & 2^3 3^1 19^1 & \text{N} & \text{N} & -48 & 32 & 1.3333333 & 0.475877 & 0.524123 & -324 & 2508 & -2832 \\
 457 & 457^1 & \text{Y} & \text{Y} & -2 & 0 & 1.0000000 & 0.474836 & 0.525164 & -326 & 2508 & -2834 \\
 458 & 2^1 229^1 & \text{Y} & \text{N} & 5 & 0 & 1.0000000 & 0.475983 & 0.524017 & -321 & 2513 & -2834 \\
 459 & 3^3 17^1 & \text{N} & \text{N} & 9 & 4 & 1.5555556 & 0.477124 & 0.522876 & -312 & 2522 & -2834 \\
 460 & 2^2 5^1 23^1 & \text{N} & \text{N} & 30 & 14 & 1.1666667 & 0.478261 & 0.521739 & -282 & 2552 & -2834 \\
 461 & 461^1 & \text{Y} & \text{Y} & -2 & 0 & 1.0000000 & 0.477223 & 0.522777 & -284 & 2552 & -2836 \\
 462 & 2^1 3^1 7^1 11^1 & \text{Y} & \text{N} & 65 & 0 & 1.0000000 & 0.478355 & 0.521645 & -219 & 2617 & -2836 \\
 463 & 463^1 & \text{Y} & \text{Y} & -2 & 0 & 1.0000000 & 0.477322 & 0.522678 & -221 & 2617 & -2838 \\
 464 & 2^4 29^1 & \text{N} & \text{N} & -11 & 6 & 1.8181818 & 0.476293 & 0.523707 & -232 & 2617 & -2849 \\
 465 & 3^1 5^1 31^1 & \text{Y} & \text{N} & -16 & 0 & 1.0000000 & 0.475269 & 0.524731 & -248 & 2617 & -2865 \\
 466 & 2^1 233^1 & \text{Y} & \text{N} & 5 & 0 & 1.0000000 & 0.476395 & 0.523605 & -243 & 2622 & -2865 \\
 467 & 467^1 & \text{Y} & \text{Y} & -2 & 0 & 1.0000000 & 0.475375 & 0.524625 & -245 & 2622 & -2867 \\
 468 & 2^2 3^2 13^1 & \text{N} & \text{N} & -74 & 58 & 1.2162162 & 0.474359 & 0.525641 & -319 & 2622 & -2941 \\
 469 & 7^1 67^1 & \text{Y} & \text{N} & 5 & 0 & 1.0000000 & 0.475480 & 0.524520 & -314 & 2627 & -2941 \\
 470 & 2^1 5^1 47^1 & \text{Y} & \text{N} & -16 & 0 & 1.0000000 & 0.474468 & 0.525532 & -330 & 2627 & -2957 \\
 471 & 3^1 157^1 & \text{Y} & \text{N} & 5 & 0 & 1.0000000 & 0.475584 & 0.524416 & -325 & 2632 & -2957 \\
 472 & 2^3 59^1 & \text{N} & \text{N} & 9 & 4 & 1.5555556 & 0.476695 & 0.523305 & -316 & 2641 & -2957 \\
 473 & 11^1 43^1 & \text{Y} & \text{N} & 5 & 0 & 1.0000000 & 0.477801 & 0.522199 & -311 & 2646 & -2957 \\
 474 & 2^1 3^1 79^1 & \text{Y} & \text{N} & -16 & 0 & 1.0000000 & 0.476793 & 0.523207 & -327 & 2646 & -2973 \\
 475 & 5^2 19^1 & \text{N} & \text{N} & -7 & 2 & 1.2857143 & 0.475789 & 0.524211 & -334 & 2646 & -2980 \\
 476 & 2^2 7^1 17^1 & \text{N} & \text{N} & 30 & 14 & 1.1666667 & 0.476891 & 0.523109 & -304 & 2676 & -2980 \\
 477 & 3^2 53^1 & \text{N} & \text{N} & -7 & 2 & 1.2857143 & 0.475891 & 0.524109 & -311 & 2676 & -2987 \\
 478 & 2^1 239^1 & \text{Y} & \text{N} & 5 & 0 & 1.0000000 & 0.476987 & 0.523013 & -306 & 2681 & -2987 \\
 479 & 479^1 & \text{Y} & \text{Y} & -2 & 0 & 1.0000000 & 0.475992 & 0.524008 & -308 & 2681 & -2989 \\
 480 & 2^5 3^1 5^1 & \text{N} & \text{N} & -96 & 80 & 1.6666667 & 0.475000 & 0.525000 & -404 & 2681 & -3085 \\
 481 & 13^1 37^1 & \text{Y} & \text{N} & 5 & 0 & 1.0000000 & 0.476091 & 0.523909 & -399 & 2686 & -3085 \\
 482 & 2^1 241^1 & \text{Y} & \text{N} & 5 & 0 & 1.0000000 & 0.477178 & 0.522822 & -394 & 2691 & -3085 \\
 483 & 3^1 7^1 23^1 & \text{Y} & \text{N} & -16 & 0 & 1.0000000 & 0.476190 & 0.523810 & -410 & 2691 & -3101 \\
 484 & 2^2 11^2 & \text{N} & \text{N} & 14 & 9 & 1.3571429 & 0.477273 & 0.522727 & -396 & 2705 & -3101 \\
 485 & 5^1 97^1 & \text{Y} & \text{N} & 5 & 0 & 1.0000000 & 0.478351 & 0.521649 & -391 & 2710 & -3101 \\
 486 & 2^1 3^5 & \text{N} & \text{N} & 13 & 8 & 2.0769231 & 0.479424 & 0.520576 & -378 & 2723 & -3101 \\
 487 & 487^1 & \text{Y} & \text{Y} & -2 & 0 & 1.0000000 & 0.478439 & 0.521561 & -380 & 2723 & -3103 \\
 488 & 2^3 61^1 & \text{N} & \text{N} & 9 & 4 & 1.5555556 & 0.479508 & 0.520492 & -371 & 2732 & -3103 \\
 489 & 3^1 163^1 & \text{Y} & \text{N} & 5 & 0 & 1.0000000 & 0.480573 & 0.519427 & -366 & 2737 & -3103 \\
 490 & 2^1 5^1 7^2 & \text{N} & \text{N} & 30 & 14 & 1.1666667 & 0.481633 & 0.518367 & -336 & 2767 & -3103 \\
 491 & 491^1 & \text{Y} & \text{Y} & -2 & 0 & 1.0000000 & 0.480652 & 0.519348 & -338 & 2767 & -3105 \\
 492 & 2^2 3^1 41^1 & \text{N} & \text{N} & 30 & 14 & 1.1666667 & 0.481707 & 0.518293 & -308 & 2797 & -3105 \\
 493 & 17^1 29^1 & \text{Y} & \text{N} & 5 & 0 & 1.0000000 & 0.482759 & 0.517241 & -303 & 2802 & -3105 \\
 494 & 2^1 13^1 19^1 & \text{Y} & \text{N} & -16 & 0 & 1.0000000 & 0.481781 & 0.518219 & -319 & 2802 & -3121 \\
 495 & 3^2 5^1 11^1 & \text{N} & \text{N} & 30 & 14 & 1.1666667 & 0.482828 & 0.517172 & -289 & 2832 & -3121 \\
 496 & 2^4 31^1 & \text{N} & \text{N} & -11 & 6 & 1.8181818 & 0.481855 & 0.518145 & -300 & 2832 & -3132 \\
 497 & 7^1 71^1 & \text{Y} & \text{N} & 5 & 0 & 1.0000000 & 0.482897 & 0.517103 & -295 & 2837 & -3132 \\
 498 & 2^1 3^1 83^1 & \text{Y} & \text{N} & -16 & 0 & 1.0000000 & 0.481928 & 0.518072 & -311 & 2837 & -3148 \\
 499 & 499^1 & \text{Y} & \text{Y} & -2 & 0 & 1.0000000 & 0.480962 & 0.519038 & -313 & 2837 & -3150 \\
 500 & 2^2 5^3 & \text{N} & \text{N} & -23 & 18 & 1.4782609 & 0.480000 & 0.520000 & -336 & 2837 & -3173 \\  
\end{array}
}
\end{equation*}

\end{table} 

\clearpage 

%% working-topics-2020.09.13-v1.tex


\end{document}
