\documentclass[11pt,reqno,a4letter]{article} 

\usepackage{amsthm,amsfonts,amscd,amsmath}
\usepackage[hidelinks]{hyperref} 
\usepackage{url}
\usepackage[usenames,dvipsnames]{xcolor}
\hypersetup{
    colorlinks,
    linkcolor={green!63!darkgray},
    citecolor={blue!70!white},
    urlcolor={blue!80!white}
}

\usepackage[normalem]{ulem}
\usepackage{graphicx} 
\usepackage{datetime} 
\usepackage{cancel}
\usepackage{subcaption}
\captionsetup{format=hang,labelfont={bf},textfont={small,it}} 
\numberwithin{figure}{section}
\numberwithin{table}{section}

\usepackage{stmaryrd,tikzsymbols} 
\usepackage{framed} 
\usepackage{ulem}
\usepackage[T1]{fontenc}
\usepackage{pbsi}


\usepackage{enumitem}
\setlist[itemize]{leftmargin=0.65in}

\usepackage{rotating,adjustbox}

\usepackage{diagbox}
\newcommand{\trianglenk}[2]{$\diagbox{#1}{#2}$}
\newcommand{\trianglenkII}[2]{\diagbox{#1}{#2}}

\let\citep\cite

\newcommand{\undersetbrace}[2]{\underset{\displaystyle{#1}}{\underbrace{#2}}}

\newcommand{\gkpSI}[2]{\ensuremath{\genfrac{\lbrack}{\rbrack}{0pt}{}{#1}{#2}}} 
\newcommand{\gkpSII}[2]{\ensuremath{\genfrac{\lbrace}{\rbrace}{0pt}{}{#1}{#2}}}
\newcommand{\cf}{\textit{cf.\ }} 
\newcommand{\Iverson}[1]{\ensuremath{\left[#1\right]_{\delta}}} 
\newcommand{\floor}[1]{\left\lfloor #1 \right\rfloor} 
\newcommand{\ceiling}[1]{\left\lceil #1 \right\rceil} 
\newcommand{\e}[1]{e\left(#1\right)} 
\newcommand{\seqnum}[1]{\href{http://oeis.org/#1}{\color{ProcessBlue}{\underline{#1}}}}

\usepackage{upgreek,dsfont,amssymb}
\renewcommand{\chi}{\upchi}
\newcommand{\ChiFunc}[1]{\ensuremath{\chi_{\{#1\}}}}
\newcommand{\OneFunc}[1]{\ensuremath{\mathds{1}_{#1}}}

\usepackage{ifthen}
\newcommand{\Hn}[2]{
     \ifthenelse{\equal{#2}{1}}{H_{#1}}{H_{#1}^{\left(#2\right)}}
}

\newcommand{\Floor}[2]{\ensuremath{\left\lfloor \frac{#1}{#2} \right\rfloor}}
\newcommand{\Ceiling}[2]{\ensuremath{\left\lceil \frac{#1}{#2} \right\rceil}}

\DeclareMathOperator{\DGF}{DGF} 
\DeclareMathOperator{\ds}{ds} 
\DeclareMathOperator{\Id}{Id}
\DeclareMathOperator{\fg}{fg}
\DeclareMathOperator{\Div}{div}
\DeclareMathOperator{\rpp}{rpp}
\DeclareMathOperator{\logll}{\ell\ell}

\title{
       \LARGE{
       Lower bounds on the Mertens function $M(x)$ along infinite subsequences for large 
       $x \gg 2.3315 \times 10^{1656520}$ 
       } 
       %\\ 
       %\large{\it New unique lower bounds on $M(x) / \sqrt{x}$ along an asymptotically 
       %   huge infinite subsequence of reals} 
}
\author{{\large Maxie Dion Schmidt} \\ 
        %{\normalsize \href{mailto:maxieds@gmail.com}{maxieds@gmail.com}} \\[0.1cm] 
        {\small Georgia Institute of Technology} \\ 
        {\small School of Mathematics} 
} 

\date{\small\underline{Last Revised:} \today\ \ -- \ \ Compiled with \LaTeX2e} 

\theoremstyle{plain} 
\newtheorem{theorem}{Theorem}
\newtheorem{conjecture}[theorem]{Conjecture}
\newtheorem{claim}[theorem]{Claim}
\newtheorem{prop}[theorem]{Proposition}
\newtheorem{lemma}[theorem]{Lemma}
\newtheorem{cor}[theorem]{Corollary}
\numberwithin{theorem}{section}

\theoremstyle{definition} 
\newtheorem{example}[theorem]{Example}
\newtheorem{remark}[theorem]{Remark}
\newtheorem{definition}[theorem]{Definition}
\newtheorem{notation}[theorem]{Notation}
\newtheorem{question}[theorem]{Question}
\newtheorem{discussion}[theorem]{Discussion}
\newtheorem{facts}[theorem]{Facts}
\newtheorem{summary}[theorem]{Summary}
\newtheorem{heuristic}[theorem]{Heuristic}

\renewcommand{\arraystretch}{1.25} 

\setlength{\textheight}{9in}
\setlength{\topmargin}{-.1in}
\setlength{\textwidth}{7.5in} 
\setlength{\evensidemargin}{-0.25in} 
\setlength{\oddsidemargin}{-0.25in} 

\usepackage{geometry}
%\newgeometry{top=0.65in, bottom=18mm, left=15mm, right=15mm, outer=2in, heightrounded, marginparwidth=1.5in, marginparsep=0.15in}
\newgeometry{top=0.65in, bottom=18mm, left=15mm, right=15mm, heightrounded, marginparwidth=0in, marginparsep=0.15in}

\usepackage{fancyhdr}
\pagestyle{empty}
\pagestyle{fancy}
\fancyhead[RO,RE]{M. D. Schmidt -- Prepared for Myself to Enjoy Eventually -- on \today} 
\fancyhead[LO,LE]{}
\fancyheadoffset{0.005\textwidth} 

\setlength{\parindent}{0in}
\setlength{\parskip}{2cm} 

\renewcommand{\thefootnote}{\fnsymbol{footnote}}
\makeatletter
\@addtoreset{footnote}{section}
\makeatother

%\usepackage{marginnote,todonotes}
%\colorlet{NBRefColor}{RoyalBlue!73} 
%\newcommand{\NBRef}[1]{
%     \todo[linecolor=green!85!white,backgroundcolor=orange!50!white,bordercolor=blue!30!black,textcolor=cyan!15!black,shadow,size=\small,fancyline]{
%     \color{NBRefColor}{\textbf{#1}
%     }
%     }
%}
\newcommand{\NBRef}[1]{}  

\newcommand{\SuccSim}[0]{\overset{_{\scriptsize{\blacktriangle}}}{\succsim}} 
\newcommand{\PrecSim}[0]{\overset{_{\scriptsize{\blacktriangle}}}{\precsim}} 

\input{glossaries-bibtex/PreambleGlossaries-Mertens}

\usepackage{tikz}
\usetikzlibrary{shapes,arrows}

\usepackage{enumitem} 

\allowdisplaybreaks 

\begin{document} 

\maketitle

\begin{abstract} 
The Mertens function, $M(x) = \sum_{n \leq x} \mu(n)$, is classically 
defined to be the summatory function of the M\"obius function 
$\mu(n)$. The 
Mertens conjecture stating that $|M(x)| < C \cdot \sqrt{x}$ with $C > 0$ for all 
$x \geq 1$ has a well-known disproof due to Odlyzko and t\'{e} Riele given in the early 1980's by computation of 
non-trivial zeta function zeros in conjunction with integral formulas for expressions of $M(x)$. 
It is conjectured and widely believed that $M(x) / \sqrt{x}$ changes sign infinitely often and grows 
unbounded in the direction of both $\pm \infty$ along subsequences of integers $x \geq 1$. 
Our proof of a result close to this property of $M(x)/\sqrt{x}$, e.g., showing that 
$$\limsup_{x \rightarrow \infty} \frac{|M(x)| \log x}{\sqrt{x}} = +\infty,$$ is not based on 
standard estimates of $M(x)$ by Mellin inversion, which are intimately tied to the 
intricate distribution of the non-trivial zeros of the Riemann zeta function. 
There is a distinct stylistic 
flavor and new element of combinatorial analysis 
peppered in with the standard methods from analytic and elementary number theory. 
This stylistic tendency distinguishes 
our methods from other proofs of established upper, rather than lower, bounds on $M(x)$. 
%The proof we eventually give of the key nearly classical tendency of the still oscillatory 
%$M(x) / \sqrt{x}$ towards 
%unboundedness as $x \rightarrow \infty$ relies on new uses of the following core constructions: 
%properties and known bounds on strongly (and completely) additive functions, 
%formulae for related signed summatory functions of special 
%Dirichlet inverse function sequences, and an intimate connection to the distribution of the 
%prime counting function, $\pi(x)$. 

\bigskip 
\noindent
\textbf{Keywords and Phrases:} {\it M\"obius function sums; Mertens function; summatory function; 
                                    arithmetic functions; 
                                    Dirichlet inverse; Liouville lambda function; prime omega functions; 
                                    prime counting functions; Dirichlet series and DGFs; 
                                    asymptotic lower bounds; Mertens conjecture. } \\ 
% 11-XX			Number theory
%    11A25  	Arithmetic functions; related numbers; inversion formulas
%    11Y70  	Values of arithmetic functions; tables
%    11-04  	Software, source code, etc. for problems pertaining to number theory
% 11Nxx		Multiplicative number theory
%    11N05  	Distribution of primes
%    11N37  	Asymptotic results on arithmetic functions
%    11N56  	Rate of growth of arithmetic functions
%    11N60  	Distribution functions associated with additive and positive multiplicative functions
%    11N64  	Other results on the distribution of values or the characterization of arithmetic functions
\textbf{Primary Math Subject Classifications (2010):} {\it 11N37; 11A25; 11N60; 11N64; and 11-04 (TBD). } 
\end{abstract} 

\bigskip\hrule\bigskip

\newpage
%\section{Reference on abbreviations, special notation and other conventions} 
\label{Appendix_Glossary_NotationConvs}
     \vskip 0in
     \printglossary[type={symbols},
                    title={Reference on special notation and other conventions},
                    style={glossstyleSymbol},
                    nogroupskip=true]


%\newpage
%\setcounter{tocdepth}{2}
%\renewcommand{\contentsname}{Listing of major sections and topics} 
%\tableofcontents 

\newpage
\section{Preface: Explanations of unconventional notions and preconceptions of asymptotics and 
         notation for asymptotic relations} 

We exphasize that the next itermized careful explanation of the subtle distinctions to our usage of 
what we consider to be traditional notation for asymptotic relations are key to 
understanding our choices of upper and lower bound expressions given throughout the article. 
Thus, to avoid any confusion that may linger as we begin to state our new results and bounds on the 
functions we work with in this article, we preface the article starting with this section detailing 
our precise definitions, meanings and assumptions on the uses of certain symbols, operators, and 
relations. The interpretation of this notation forms the core of how we choose 
to convey the growth rates of arithmetic functions on their domain of $x$ within this article 
when $x$ is taken to be very large, and typically tending to infinity 
\cite[\cf \S 2]{NISTHB} \cite{ACOMB-BOOK}. 

\subsection{Average order, similarity and approximation of asymptotic growth rates of quantities} 

\subsubsection{Similarity and average order (expectation)} 

First, we say that two functions $A(x), B(x)$ satisfy the relation $A \sim B$ if 
\[
\lim_{x \rightarrow \infty} \frac{A(x)}{B(x)} = 1. 
\] 
It is sometimes standard to express the \emph{average order} of an arithmetic function f as 
$f \sim h$, even when the values of $f(n)$ may actually non-monotonically 
oscillate, or say have value of one infinitely often. What the notation $f \sim h$ means in expressing 
the average order of $f$ is that 
$\frac{1}{x} \cdot \sum_{n \leq x} f(n) \sim h(x)$. 
For example, in the acceptably classic language of \cite{HARDYWRIGHT} we would normally write that 
$\Omega(n) \sim \log\log n$, even though technically, 
$1 \leq \Omega(n) \leq \frac{\log n}{\log 2}$. 
To be absolutely clear about notation, we intentionally do not re-use the $\sim$ relation by 
instead writing $\mathbb{E}[f(x)] = h(x)$ (as in expectation of $f$) 
to denote that $f$ has a limiting average order growing at the 
rate of $h$. 

A related conception of $f$ having a so-called \emph{normal order} of $g$ holds whenever 
$$f(n) = (1+o(1)) g(n), \mathrm{a.e.}$$
            
\subsubsection{Approximation} 
     
We choose to adpot the convention to write that $f(x) \approx g(x)$ if $|f(x) - g(x)| = O(1)$. 
That is, we write $f(x) \approx g(x)$ to denote that $f$ is approximately equal to $g$ at $x$ modulo at most a
small constant difference between the functions. 

The formula we prefer for the Abel summation variant of summation by parts 
to express finite sums of a product of two functions is stated as follows 
\cite[\cf \S 4.3]{APOSTOLANUMT} \footnote{
     Compare to the exact formula for \emph{summation by parts} of any arithmetic functions, $u_n,v_n$, 
     stated as in \cite[\S 2.10(ii)]{NISTHB} for $U_j := u_1+u_2+\cdots+u_j$ when $j \geq 1$: 
     \[
     \sum_{j=1}^{n-1} u_j \cdot v_j = U_{n-1} v_n + \sum_{j=1}^{n-1} U_j \left(v_j - v_{j+1}\right), n \geq 2. 
     \]
}: 
 
\begin{prop}[Abel Summation Integral Formula] 
\label{prop_AbelSummationFormula} 
Suppose that $t > 0$ is real-valued, and that $A(t) \sim \sum_{n \leq t} a(n)$ for some weighting 
arithmetic function $a(n)$ with $A(t)$ continuously differentiable on $(0, \infty)$. Furthermore, suppose that 
$b(n) \sim f(n)$ with $f$ a differentiable function of $n \geq 0$ -- that is, $f^{\prime}(t)$ exists and is smooth for all 
$t \in (0, \infty)$. 
Then for $0 \leq y < x$, where we typcially assume that the bounds of summation satisfy 
$x, y \in \mathbb{Z}^{+}$, we have that 
\[
\sum_{y < n \leq x} a(n) b(n) \sim A(x)b(x) - A(y)b(y) - \int_y^{x} A(t) f^{\prime}(t) dt. 
\] 
\end{prop}

\begin{remark}
The classical proof of the Abel summation formula given in Apostol's book has an alternate proof method 
noted in Section 4.3 of this reference. In particular, since $A(x)$ is a step function with jump of $a(n)$ 
at each integer-valued $n \geq 1$, the integral formula stated in 
Proposition \ref{prop_AbelSummationFormula} can be expressed in the following Riemann-Stieltjes integral 
notation: 
\[
\sum_{y < n \leq x} a(n) b(n) = \int_y^x f(t) dA(t). 
\]
A notable special case yields an integral approximation to summations we stated above where $[t]$ is the 
\emph{greatest integer (ceiling) function}: 
\[
\sum_{y < n \leq x} f(n) = f(x) [x] - f(y) [y] - \int_{y}^x [t] f^{\prime}(t) dt. 
\]
\end{remark}

\subsubsection{Vinogradov's notation for asymptotics} 

We use the conventional relations $f(x) \gg g(x)$ and $h(x) \ll r(x)$ to symbolically express that we should expect 
$f$ to be ``substantially`` larger than $g$, and $h$ to be ``significantly'' smaller, in asymptotic order 
(e.g., rate of growth when $x$ is large). In practice, we adopt a somewhat looser definition of these symbols which 
allows $f \gg g$ and $h \ll r$ provided that there are constants $C, D > 0$ such that whenever $x$ is sufficiently 
large we have that $f(x) \geq C \cdot g(x)$ and $h(x) \leq D \cdot r(x)$. This notation is sometimes called 
\emph{Vinogradov's asymptotic notation}. 

Another way of expressing our precise meaning of these relations is by writing 
$$f \gg g \iff g = O(f),$$ and $$h \ll r \iff r = \Omega(h),$$ using Knuth's well-trodden 
style of big-$O$ (and Landau notation) and big-$\Omega$ (Hardy-Littlewood notation)
symbols from the language of theoretical computer science and in the analysis of algorithms. 

\subsection{An unconventional pair of asymptotic relations employed to drop 
            lower-order terms in upper and lower bounds on arithmetic functions} 
         
We define two new defintions of relations for expressing limiting asymptotic bounds on functions by 
adapting notation for existing operators for clarity of the way we use them here.       
Namely, we say that $h(x) \SuccSim r(x)$ if $h \gg r$ as $x \rightarrow \infty$, and define 
the relation $\PrecSim$ similarly as 
$h(x) \PrecSim r(x)$ if $h \ll r$ as $x \rightarrow \infty$. 
This usage of the notation of $\SuccSim,\PrecSim$ intentionally breaks with the usual conventions for the use of 
these more standard relations of $\succsim,\precsim$. 
Our distinct, intentional usage of these relations in our different context is intended to 
simplify the ways we express otherwise tricky and complicated expressions for upper and lower bounds that hold only 
exactly in limiting cases where $x$ is large as $x \rightarrow \infty$. 

The use of the new (modified) notation for $\SuccSim$ is intended to capture both that we are conveying a lower bound for the 
function, and crucially that this lower bound is valid only when $x$ is very large, i.e., in some sense that the lower bound 
holds in the same sense as the relation $\sim$: for example, entending a notion similar to 
$|f(x)| \geq g(x)$ with $g(x) \sim h(x)$. 
This is a subtle distinction that comes into play when we later use it to state lower bounds in our new results. 

An demonstratively constructed mock example motivating this usage of these 
relations that clarifies the point of making this subtle distinction in notation appears below. 

\begin{example}
Suppose that exactly for all $x \geq 1$ we have 
\[
f(x) \geq -(\log\log\log x)^2 + 3 \times 10^{1000000} \cdot (\log\log\log x)^{1.999999999} + E(x), 
\]
where $E(x) = o\left((\log\log\log x)^2\right)$ and the unusually complicated expression for $E(x)$ requires 
more than $100000$ ascii characters to typeset accurately, e.g., is far too exceedingly complicated to write down and 
include as a component of our expression for the terms in the primary bound. 
Then naturally, we prefer to work with only the expression
for the asymptotically dominant main term in the lower bounds on $f(x)$ stated above. 

Note that in this case the main term contribution 
does not dominate the bound until $x$ is very large, so that replacing the right-hand-side expression with just this 
term yields an invalid inequality except for in limiting cases. In this instance, we prefer to write 
\[
f(x) \SuccSim -(\log\log\log x)^2, \mathrm{\ as \ } x \rightarrow \infty, 
\]
or more conventionally applying this notation only to unsigned functions that 
\[
|f(x)| \SuccSim (\log\log\log x)^2, \mathrm{\ as \ } x \rightarrow \infty, 
\]
which indicates that this substantially simplifed form of the lower bound on $f$ holds as $x \rightarrow \infty$. 
In particular, it is problematic to only write that 
\[
f(x) \geq -(\log\log\log x)^2, 
\]
since there is a substantial (however, asymptotically negligible) initial range of $x \geq 1$ where this lower bound is 
invalid as stated in the previous equation. 
\end{example} 

\begin{remark}[Emphasizing the rationale of the use of the new notation]
Hence, we emphasize that our new uses of the traditional symbols are as asymptotic 
relations defined to simplify our results by dropping expressions involving more precise, exact terms 
that are nonetheless asymptotically insignificant, to obtain accurate statements 
in limiting cases of large $x$ that hold as $x \rightarrow \infty$. In principle, this convention allows us to 
write out simplified bounds that still capture the most simple 
essence of the upper or lower bound as we choose to view it in this article. 

This take on the new meanings denoted by $\SuccSim,\PrecSim$ is particularly 
powerful and is utilized in this article when we express many lower bound estimates for functions that would 
otherwise require literally pages of typeset symbols to state exactly, but which have simple enough 
formulae when considered as bounds that hold in this type of limiting asymptotic context. 
\end{remark} 

\subsection{Asymptotic expansions and uniformity} 

Because a subset of the results we cite that are proved in the references 
(e.g., borrowed from Chapter 7 of \cite{MV}) provide statements of 
asymptotic bounds that hold \emph{uniformly} for $x$ large, though in a bounded range depending on parameters, 
we need to briefly make precise what our preconceptions are of this terminology. 
We introduce the notation for asymptotic expansions of a function $f: \mathbb{R} \rightarrow \mathbb{R}$ from 
\cite[\S 2.1(iii)]{NISTHB}. 

\subsubsection{Ordinary asymptotic expansions of a function} 

Let $\sum_{n} a_n x^{-n}$ denote a formal power series expansion in $x$ where we 
ignore any necessary conditions to guarantee convergence of the series. For each integer $n \geq 1$, suppose that 
\[
f(x) = \sum_{s=0}^{n-1} a_s x^{-s} + O(x^{-n}), 
\]
as $|x| \rightarrow \infty$ where this limiting bound holds for $x \in \mathbb{X}$ in some unbounded set 
$\mathbb{X} \subseteq \mathbb{R}, \mathbb{C}$. 
When such a bound holds, we say that $\sum_s a_s x^{-s}$ is a \emph{Poincar\'{e} asymptotic expansion}, 
or just \emph{asymptotic series expansion}, of $f(x)$ as $x \rightarrow \infty$ along the fixed set $\mathbb{X}$. 
The condition in the previous equation is equivalent to writing 
\[
f(x) \sim a_0 + a_1 x^{-1} + a_2 x^{-2} + \cdots; x \in \mathbb{X}, \mathrm{\ for \ } |x| \rightarrow \infty. 
\]
The prior two characterizations of an asymptotic expansion for $f$ are also equivalent to the 
statement that 
\[
x^n \left(f(x) - \sum_{s=0}^{n-1} a_s x^{-s}\right) \xrightarrow{x \rightarrow \infty} a_n. 
\] 

\subsubsection{Uniform asymptotic expansions of a function} 

Let the set $\mathbb{X}$ from the definition in the last subsection correspond to a 
closed sector of the form 
$$\mathbb{X} := \{x \in \mathbb{C}: \alpha \leq \operatorname{arg}(x) \leq \beta\}.$$ 
Then we say that the asymptotic property 
\[
f(x) = \sum_{s=0}^{n-1} a_s x^{-s} + O(x^{-n}), 
\]
from before holds \emph{uniformly} with respect to $\operatorname{arg}(x) \in [\alpha, \beta]$ as 
$|x| \rightarrow \infty$. 

Another useful, important notion of uniform asymptotic bounds is taken with respect to some parameter $u$ 
(or set of parameters, respectively) that ranges over the point set (point sets, respectively) 
$u \in \mathbb{U}$. In this case, if we have that the $u$-parameterized expressions 
\[
\left\lvert x^n\left(f(u, x) - \sum_{s=0}^{n-1} a_s(u) x^{-s}\right) \right\rvert, 
\]
are bounded for all integers $n \geq 1$ for $x \in \mathbb{X}$ as $|x| \rightarrow \infty$, then we say that 
the asymptotic expansion of $f$ holds \emph{uniformly} for $u \in \mathbb{U}$. 
Now the function $f \equiv f(u, x)$ and the 
asymptotic series coefficients $a_s(u)$ may have an implicit dependence on the parameter $u$. 
If the previous boundedness condition holds for all positive integers $n$, we write that 
\[
f(u, x) \sim \sum_{s=0}^{\infty} a_s(u) x^{-s}; x \in \mathbb{X}, \mathrm{\ as \ } |x| \rightarrow \infty, 
\]
and say that this asymptotic expansion, or bound, holds \emph{uniformly with respect to $u \in \mathbb{U}$}. 
For $u$ taken outside of $\mathbb{U}$, the stated bound may fail to be valid even for $x \in \mathbb{X}$ as 
$|x| \rightarrow \infty$. 

\newpage
\section{An introduction to the Mertens function -- definition, properties, known results and conjectures} 
\label{subSection_MertensMxClassical_Intro} 

Suppose that $n \geq 1$ is a natural number with factorization into 
distinct primes given by 
$n = p_1^{\alpha_1} p_2^{\alpha_2} \cdots p_k^{\alpha_k}$. 
We define the \emph{M\"oebius function} to be the signed indicator function 
of the squarefree integers: 
\[
\mu(n) = \begin{cases} 
     1, & \text{if $n = 1$; } \\ 
     (-1)^k, & \text{if $\alpha_i = 1$, $\forall 1 \leq i \leq k$; } \\ 
     0, & \text{otherwise.} 
     \end{cases} 
\]
There are many known variants and special properties of the M\"oebius function 
and its generalizations \cite[\cf \S 2]{HANDBOOKNT-2004}, however, for our 
purposes we seek to explore the properties and asymptotics of weighted 
summatory functions over $\mu(n)$. 
The Mertens summatory function, or \emph{Mertens function}, is defined as 
\cite[\seqnum{A002321}]{OEIS} 
\begin{align*} 
M(x) & = \sum_{n \leq x} \mu(n),\ x \geq 1, \\ 
     & \longmapsto \{1, 0, -1, -1, -2, -1, -2, -2, -2, -1, -2, -2, -3, -2, 
     -1, -1, -2, -2, -3, -3, -2, -1, -2, -2, \ldots\}
\end{align*} 
A related function which counts the 
number of \emph{squarefree} integers than $x$ sums the average order of the M\"obius function as 
\cite[\seqnum{A013928}]{OEIS} 
\[ 
Q(n) = \sum_{n \leq x} |\mu(n)| \sim \frac{6x}{\pi^2} + O\left(\sqrt{x}\right). 
\] 
It is known that the asymptotic density of the positively versus negatively 
weighted sets of squarefree numbers are in fact equal as $x \rightarrow \infty$: 
\[
\mu_{+}(x) = \frac{\#\{1 \leq n \leq x: \mu(n) = +1\}}{Q(x)} = 
     \mu_{-}(x) = \frac{\#\{1 \leq n \leq x: \mu(n) = -1\}}{Q(x)} 
     \xrightarrow[x \rightarrow \infty]{} \frac{3}{\pi^2}. 
\]
While this limiting law suggests an even bias for the Mertens function, 
in practice $M(x)$ has an apparent unproven negative bias in its values, and the actual 
local oscillations between the approximate densities of the sets 
$\mu_{\pm}(x)$ lend an unpredictable nature to the function and the function's 
characteristic oscillatory sawtooth shaped plot viewed over the positive integers. 

\subsection{Properties} 

The conventional approach to evaluating the behavior of $M(x)$ for large 
$x \rightarrow \infty$ results from a formulation of this summatory 
function as a predictable exact sum involving $x$ and the non-trivial 
zeros of the Riemann zeta function for all real $x > 0$. 
This formula is easily expressed via an inverse Mellin transformation 
over the reciprocal zeta function. In particular, 
we notice that since 
\[
\frac{1}{\zeta(s)} = \int_1^{\infty} \frac{s \cdot M(x)}{x^{s+1}} dx, 
\]
we then obtain that 
\[
M(x) = \lim_{T \rightarrow \infty}\ \frac{1}{2\pi\imath} \int_{T-\imath\infty}^{T+\imath\infty} 
     \frac{x^s}{s \cdot \zeta(s)} ds. 
\] 
This representation along with the standard Euler product 
representation for the reciprocal zeta function leads us to the 
exact expression for $M(x)$ when $x > 0$ given by the next theorem. 

\begin{theorem}[Analytic Formula for $M(x)$] 
\label{theorem_MxMellinTransformInvFormula} 
Assuming the Riemann Hypothesis (RH), we can show that there exists an infinite sequence 
$\{T_k\}_{k \geq 1}$ satisfying $k \leq T_k \leq k+1$ for each $k$ 
such that for any $x \in \mathbb{R}_{>0}$ 
\[
M(x) = \lim_{k \rightarrow \infty} 
     \sum_{\substack{\rho: \zeta(\rho) = 0 \\ |\Im(\rho)| < T_k}} 
     \frac{x^{\rho}}{\rho \cdot \zeta^{\prime}(\rho)} - 2 + 
     \sum_{n \geq 1} \frac{(-1)^{n-1}}{n \cdot (2n)! \zeta(2n+1)} 
     \left(\frac{2\pi}{x}\right)^{2n} + 
     \frac{\mu(x)}{2} \Iverson{x \in \mathbb{Z}^{+}}. 
\] 
\end{theorem} 

A historical unconditional bound on the Mertens function due to Walfisz (1963) 
states that there is an absolute constant $C > 0$ such that 
$$M(x) \ll x \cdot \exp\left(-C \cdot \log^{3/5}(x) 
  (\log\log x)^{-3/5}\right).$$ 
Under the assumption of the RH, Soundararajan proved in 2009 new updated estimates 
bounding $M(x)$ for large $x$ of the following forms \cite{SOUND-MERTENS-ANNALS}: 
\begin{align*} 
M(x) & \ll \sqrt{x} \cdot \exp\left(\log^{1/2}(x) (\log\log x)^{14}\right), \\ 
M(x) & = O\left(\sqrt{x} \cdot \exp\left( 
     \log^{1/2}(x) (\log\log x)^{5/2+\epsilon}\right)\right),\ 
     \forall \epsilon > 0. 
\end{align*} 
To date, due to the oscillatory nature of $M(x)$ via the signedness of $\mu(n)$, 
considerably less has been conjectured about explicit lower bounds on $|M(x)|$ along 
subsequences. 

\subsection{Conjectures} 

The RH is equivalent to showing that 
$M(x) = O\left(x^{1/2+\varepsilon}\right)$ for any 
$0 < \varepsilon < \frac{1}{2}$. 
It is still unresolved whether 
\[ 
\limsup_{x\rightarrow\infty} |M(x)| / \sqrt{x} = \infty, 
\] 
although computational evidence suggests that this is a likely conjecture 
\cite{ORDER-MERTENSFN,HURST-2017}. 
There is a rich history to the original statement of the \emph{Mertens conjecture} which 
states that 
\[ 
|M(x)| < c \cdot x^{1/2},\ \text{ some absolute constant $c > 0$. }
\] 
Mertens conjecture was first verified by Mertens for $c = 1$ and $x < 10000$, 
although since its beginnings in 1897, the conjecture has since been disproved by computation 
of low-lying zeta function zeros in a famous paper by 
Odlyzko and t\'{e} Riele from the early 1980's. 

There are a number of other interesting unsolved and at 
least somewhat accessible open problems 
related to the asymptotic behavior of $M(x)$ at large $x$. 
It is believed that the sign of $M(x)$ changes infinitely often. 
That is to say that it is widely believed that $M(x)$ is 
ocsillatory and exhibits a negative bias insomuch as 
$M(x) < 0$ more frequently than $M(x) > 0$ over all 
$x \in \mathbb{N}$. 

One of the most famous still unanswered questions about the Mertens 
function concerns whether $|M(x)| / \sqrt{x}$ is in actuality unbounded on the 
natural numbers. A precise statement of this 
problem is to produce an affirmative answer whether 
$\limsup_{x \rightarrow \infty} M(x) / \sqrt{x} = +\infty$ and 
$\liminf_{x \rightarrow \infty} M(x) / \sqrt{x} = -\infty$, or 
equivalently whether there is an infinite sequence of natural numbers 
$\{x_1, x_2, x_3, \ldots\}$ such that the magnitude of 
$M(x_i) x_i^{-1/2}$ grows without bound along the subsequence. 
Currently, an exact rigorous 
proof that $M(x) / \sqrt{x}$ is unbounded still remains elusive, though there is suggestive probabilistic 
evidence of this property established by Ng in 2008. 
We cite that prior to this point it is known that \cite[\cf \S 4.1]{PRIMEREC} 
\[
\limsup_{x\rightarrow\infty} \frac{M(x)}{\sqrt{x}} > 1.060\ \qquad (\text{now } 1.826054), 
\] 
and 
\[ 
\liminf_{x\rightarrow\infty} \frac{M(x)}{\sqrt{x}} < -1.009\ \qquad (\text{now } -1.837625), 
\] 
although based on work by Odlyzyko and t\'{e} Riele it seems probable that 
each of these limits should be $\pm \infty$, respectively 
\cite{ODLYZ-TRIELE,MREVISITED,ORDER-MERTENSFN,HURST-2017}. 

Extensive computational evidence has produced 
a conjecture due to Gonek (among attempts on bounds by others) that in fact the limiting behavior of 
$M(x)$ satisfies 
that $$\limsup_{x \rightarrow \infty} \frac{|M(x)|}{\sqrt{x} 
(\log\log x)^{5/4}} = O(1).$$ 
While it seems to be widely believed that $|M(x)| / \sqrt{x}$ tends to $+\infty$ at a logarithmic rate 
along subsequences, infinitely tending factors such as the $(\log\log x)^{\frac{5}{4}}$ in Gonek's conjecture 
do not appear to readily fall out of work on bounds for $M(x)$ by existing methods. 

\newpage
\section{A summary outline: Listing the core logical steps and critical components to the proof in order of exposition} 

\subsection{Step-by-step overview} 

We offer another brief step-by-step summary overview of the critical components 
to our proof outlined in the introduction above, 
and then which are proved piece-by-piece in the next sections of the article below. 
This outline is provided to help 
the reader see our logic and proof methodology as easily and quickly as possible. 
\begin{itemize} 

\item[\textbf{(1)}] We prove a matrix inversion formula relating the summatory 
           functions of an arithmetic function $f$ and its Dirichlet inverse $f^{-1}$ (for $f(1) \neq 0$). 
           See 
           Theorem \ref{theorem_SummatoryFuncsOfDirCvls} in 
           Section \ref{Section_PrelimProofs_Config}.  
\item[\textbf{(2)}] This crucial step provides us with an exact formula for $M(x)$ in terms of $\pi(x)$, the seemingly 
           unconnected prime counting function, and the 
           Dirichlet inverse of the shifted additive function $g(n) := \omega(n)+1$. This 
           formula is already stated in \eqref{eqn_Mx_gInvnPixk_formula} expanded above.  
\item[\textbf{(3)}] We tighten a result from \cite[\S 7]{MV} providing summatory functions that indicate the parity of 
           $\lambda(n)$ using elementary arguments and more combinatorially flavored expansions of Dirichlet series in 
           our proof of Theorem \ref{theorem_GFs_SymmFuncs_SumsOfRecipOfPowsOfPrimes}. 
           We use this result to sum $\sum_{n \leq x} \lambda(n) f(n)$ for particular non-negative arithmetic 
           functions $f$ by Abel summation when $x$ is large. 
\item[\textbf{(4)}] We then turn to the asymptotics if the quasi-periodic $g^{-1}(n)$, estimating this inverse function's 
           limiting asymptotics for large $n$ (or $n \leq x$ when $x$ is very large) in 
           Section \ref{Section_InvFunc_PreciseExpsAndAsymptotics}. 
           We eventually use these estimates to prove a substantially unique new lower bound formula 
           for the summatory function $G^{-1}(x) := \sum_{n \leq x} g^{-1}(n)$ along prescribed asymptotically large 
           infinite subsequences that tend to $+\infty$ (see Theorem \ref{theorem_gInv_GeneralAsymptoticsForms}). 
\item[\textbf{(5)}] When we return to step \textbf{(2)} 
           with our new lower bounds at hand, and bootstrap, we find ``magic'' in the form of 
           showing the unboundedness of $\frac{|M(x)| \log x}{\sqrt{x}}$ 
           along a very large increasing infinite subsequence 
           of positive natural numbers. What we recover is a quick, and rigorous, proof of 
           Corollary \ref{cor_ThePipeDreamResult_v1} given in 
           Section \ref{subSection_TheCoreResultProof}. 
           
\end{itemize} 

\begin{remark}
Note that much of the effort and length taken in proving the complete set of results in this article is 
attributed to showing that we can in fact discard asymptotically negligible terms from exact expansions  
of formulas we require to bound. This machinery is necessary to rigorously justify that the actually most 
interesting new bounds we obtain here are valid as $x \rightarrow \infty$. The preceeding steps have sketched 
the core components that are central to building our more combinatorially motivated approach to approximating 
$M(x)$ that we develop in this article. The full listing of components to the proof including 
``routine'', or so-called cut-and-dry, results we need as glue 
to make our full improvements precise are included in the key to the schematic of our proof logic given in the next 
subsection.
\end{remark} 

\subsection{Diagramatic flowchart of the proof logic with references to results} 

\subsubsection*{Flowchart schematic diagram: } 

The next flowchart diagramed below shows how the seemingly disparate components of the proof are organized. 
It also indicates how the top two separate ``lands'' of material and corresponding sets of requisite results 
forming the connected components to steps $\mathcal{A}$ and $\mathcal{B}$ (as viewed below) 
combine to form the next core stage of the proof. 

\tikzstyle{CoreComponent} = [diamond, draw, fill=blue!35, text width=4.5em, text badly centered, 
                             node distance=3cm, inner sep=0.1cm]
\tikzstyle{SubComponent} = [rectangle, draw, fill=blue!25, text width=4.5em, text centered, 
                            rounded corners, minimum height=4em, node distance=3cm]
\tikzstyle{ComponentConnectionLine} = [draw, -latex]

\begin{center}
\fbox{
\begin{tikzpicture}[node distance = 2cm, auto]
%% : == Nodes: 
\node[CoreComponent] (A)  {Step $\mathcal{A}$}; 
\node[SubComponent, right of=A]  (A2) {A.2}; 
\node[right of=A2] (CenterDiagram) {             };
\node[CoreComponent, right of=CenterDiagram] (B)  {Step $\mathcal{B}$}; 
\node[SubComponent, right of=B]  (B2) {B.2}; 
\node[SubComponent, below of=B2]  (B3) {B.3}; 
\node[CoreComponent, below of=CenterDiagram] (C)  {Step $\mathcal{C}$}; 
\node[CoreComponent, below of=C] (D)  {Step $\mathcal{D}$}; 
%% : == Arrows:
\path[ComponentConnectionLine, dashed, style={<->}] (A) -- (A2);
\path[ComponentConnectionLine, dashed, style={<->}] (A2) -- (C);
\path[ComponentConnectionLine, dashed, style={<->}] (B) -- (B2);
\path[ComponentConnectionLine, dashed, style={<->}] (B2) -- (B3);
\path[ComponentConnectionLine, dashed, style={<->}] (B3) -- (B);
\path[ComponentConnectionLine] (A) -- (C);
\path[ComponentConnectionLine] (B) -- (C);
\path[ComponentConnectionLine] (C) -- (D);
\end{tikzpicture} 
}
\end{center}

\subsubsection*{Key to the diagram stages: } 
\begin{itemize}[noitemsep,topsep=0pt]

\item[\textbf{Step A:}] \textit{Citations and re-statements of existing theorems proved elsewhere}: 
     E.g., statements of non-trivial theorems and key results we need that are proved in the references. 
     \begin{itemize}[noitemsep,topsep=0pt] 
     \item[\textbf{A.A}] Key results and constructions: 
          \begin{itemize}[noitemsep,topsep=0pt]
          \item[--] \small{Theorem \ref{theorem_HatPi_ExtInTermsOfGz}} 
          \item[--] \small{Theorem \ref{theorem_MV_Thm7.20-init_stmt}} 
          \item[--] \small{Corollary \ref{theorem_MV_Thm7.20}} 
          \item[--] \small{The results, lemmas, and facts cited in Section \ref{subSection_OtherFactsAndResults}}
          \end{itemize} 
     \item[\textbf{A.2}] Lower bounds on the Abel summation based formula for $G^{-1}(x)$: 
          \begin{itemize}[noitemsep,topsep=0pt]
          \item[--] \small{Theorem \ref{theorem_GFs_SymmFuncs_SumsOfRecipOfPowsOfPrimes} 
                    (on page \pageref{proofOf_theorem_GFs_SymmFuncs_SumsOfRecipOfPowsOfPrimes})} 
          \item[--] \small{Corollary \ref{cor_PartialSumsOfReciprocalsOfPrimePowers}} 
          \item[--] \small{Theorem \ref{theorem_gInv_GeneralAsymptoticsForms}} 
          \item[--] \small{Lemma \ref{lemma_CLT_and_AbelSummation}} 
          \item[--] \small{Lemma \ref{lemma_lowerBoundsOnLambdaFuncParitySummFuncs}} 
          \end{itemize} 
     \end{itemize} 
\item[\textbf{Step B:}] \textit{Constructions of an exact formula for $M(x)$}: The exact formula we prove 
     uses special arithmetic functions with particularly ``nice'' properties and bounds. This choice of 
     the expression from Theorem \ref{theorem_SummatoryFuncsOfDirCvls} 
     is key to how far we have traveled along the new approaches in this article. 
     In particular, the additivity of $\omega(n)$ and the easily integrable logarithmically weighted bound on 
     $\pi(x)$ for large $x$ are indispensible components to why this proof works well. 
     \begin{itemize}[noitemsep,topsep=0pt] 
     \item[\textbf{B.B}] Key results and constructions: 
          \begin{itemize}[noitemsep,topsep=0pt]
          \item[--] \small{Theorem \ref{theorem_SummatoryFuncsOfDirCvls} 
                    (on page \pageref{proofOf_theorem_SummatoryFuncsOfDirCvls})} 
          \item[--] \small{Corollary \ref{cor_CvlGAstMu}} 
          \item[--] \small{Corollary \ref{cor_Mx_gInvnPixk_formula}} 
          \item[--] \small{Conjecture \ref{lemma_gInv_MxExample} (to a lesser expository only extent)} 
          \item[--] \small{Proposition \ref{prop_AntiqueDivisorSumIdent}} 
          \item[--] \small{Proposition \ref{prop_SignageDirInvsOfPosBddArithmeticFuncs_v1}} 
          \end{itemize} 
     \item[\textbf{B.2}] Asymptotics for the component functions $g^{-1}(n)$ and $G^{-1}(x)$: 
          \begin{itemize}[noitemsep,topsep=0pt]
          \item[--] \small{Theorem \ref{theorem_Ckn_GeneralAsymptoticsForms} 
                    (on page \pageref{proofOf_theorem_Ckn_GeneralAsymptoticsForms})} 
          \item[--] \small{Lemma \ref{lemma_AnExactFormulaFor_gInvByMobiusInv_v1}} 
          \end{itemize} 
     \item[\textbf{B.3}] Simplifying the requisite formulas for $g^{-1}(n)$ and $G^{-1}(x)$: 
          \begin{itemize}[noitemsep,topsep=0pt]
          \item[--] \small{Corollary \ref{cor_ComputingInvFuncs_InPractice_DivSumgInvAst1_v1}} 
          \item[--] \small{Corollary \ref{cor_ASemiForm_ForGInvx_v1}} 
          \end{itemize} 
     \end{itemize} 
\item[\textbf{Step C:}] \textit{Re-writing the exact formula for $M(x)$}: Key interpretations used in 
     formulating the lower bounds based on the re-phrased integral formula. 
     \begin{itemize}[noitemsep,topsep=0pt]
     \item[--] \small{Proposition \ref{prop_Mx_SBP_IntegralFormula}} 
     \end{itemize} 
\item[\textbf{Step D:}] \textit{The Holy Grail}: A big leap twoards proving that 
     $\frac{|M(x)| \log x}{\sqrt{x}}$ is 
     unbounded in the limit supremum sense. 
     \begin{itemize}[noitemsep,topsep=0pt]
     \item[--] \small{Corollary \ref{cor_ThePipeDreamResult_v1} (on page \pageref{proofOf_cor_ThePipeDreamResult_v1})} 
     \end{itemize} 

\end{itemize} 

\newpage 
\section{An introduction to our new methodology: A concrete approach to bounding $M(x)$ from below} 

\subsection{Summing series over Dirichlet convolutions} 

\begin{theorem}[Summatory functions of Dirichlet convolutions] 
\label{theorem_SummatoryFuncsOfDirCvls} 
Let $f,g: \mathbb{Z}^{+} \rightarrow \mathbb{C}$ be any arithmetic functions such that $f(1) \neq 0$. 
Suppose that $F(x) := \sum_{n \leq x} f(n)$ and $H(x) := \sum_{n \leq x} h(n)$ denote the summatory 
functions of $f,g$, respectively, and that $F^{-1}(x)$ denotes the summatory function of the 
Dirichlet inverse $f^{-1}(n)$ of $f$. Then, letting the counting function $\pi_{f \ast h}(x)$ be defined 
as in the first equation below, we have the following equivalent expressions for the 
summatory function of $f \ast h$ for integers $x \geq 1$: 
\begin{align*} 
\pi_{f \ast h}(x) & = \sum_{n \leq x} \sum_{d|n} f(d) h(n/d) \\ 
     & = \sum_{d \leq x} f(d) H\left(\Floor{x}{d}\right) \\ 
     & = \sum_{k=1}^{x} H(k) \left[F\left(\Floor{x}{k}\right) - 
     F\left(\Floor{x}{k+1}\right)\right]. 
\end{align*} 
Moreover, we can invert the linear system determining the coefficients of $H(k)$ for $1 \leq k \leq x$ 
naturally to express $H(x)$ as a linear combination of the original left-hand-side 
summatory function as follows:
\begin{align*} 
H(x) & = \sum_{j=1}^{x} \pi_{f \ast h}(j) \left[F^{-1}\left(\Floor{x}{j}\right) - 
     F^{-1}\left(\Floor{x}{j+1}\right)\right] \\ 
     & = \sum_{n=1}^{x} f^{-1}(n) \pi_{f \ast h}\left(\Floor{x}{n}\right). 
\end{align*} 
\end{theorem} 

\begin{cor}[Convolutions Arising From M\"obius Inversion] 
\label{cor_CvlGAstMu} 
Suppose that $g$ is an arithmetic function with $g(1) \neq 0$. Define the summatory function of 
the convolution of $g$ with $\mu$ by $\widetilde{G}(x) := \sum_{n \leq x} (g \ast \mu)(n)$. 
Then the Mertens function equals 
\[
M(x) = \sum_{k=1}^{x} \left(\sum_{j=\floor{\frac{x}{k+1}}+1}^{\floor{\frac{x}{k}}} g^{-1}(j)\right) 
     \widetilde{G}(k), \forall x \geq 1. 
\]
\end{cor} 

\begin{cor}[A motivating special case] 
\label{cor_Mx_gInvnPixk_formula} 
We have exactly that for all $x \geq 1$ 
\begin{equation} 
\label{eqn_Mx_gInvnPixk_formula} 
M(x) = \sum_{k=1}^{x} (\omega+1)^{-1}(k) \left[\pi\left(\Floor{x}{k}\right) + 1\right]. 
\end{equation} 
\end{cor} 

\subsection{Elaborating on construction behind the motivating special case} 
\label{example_InvertingARecRelForMx_Intro}

We can compute the first few terms for the
Dirichlet inverse sequence of the arithmetic function 
$g(n) := \omega(n) + 1$ from 
Corollary \ref{cor_Mx_gInvnPixk_formula} 
numerically for the first few sequence values as 
\[
\{g^{-1}(n)\}_{n \geq 1} = \{1, -2, -2, 2, -2, 5, -2, -2, 2, 5, -2, -7, -2, 5, 5, 2, -2, -7, -2, 
     -7, 5, 5, -2, 9, \ldots \}. 
\] 
The sign of these terms is given by $\operatorname{sgn}(g^{-1}(n)) = \frac{g^{-1}(n)}{|g^{-1}(n)|} = \lambda(n)$ 
(see Proposition \ref{prop_SignageDirInvsOfPosBddArithmeticFuncs_v1}). 
This useful property is inherited from the distinctly 
additive nature of the component function $\omega(n)$. 
We will still require substantially simpler asymptotic formulae for $g^{-1}(n)$ than what 
complications are suggested by inspection of the initial 
numerical calculations of this sequence. 
It does happen that we can find fruitful and combinatorially meaningful 
ways to express asymptotics for this special inverse sequence 
(see Theorem \ref{theorem_Ckn_GeneralAsymptoticsForms}). 

Consider first the following motivating conjecture: 
\NBRef{A01-2020-04-26}

\begin{conjecture}
\label{lemma_gInv_MxExample} 
Suppose that $n \geq 1$ is a squarefree integer. We have the following properties characterizing the 
Dirichlet inverse function $g^{-1}(n) = (\omega+1)^{-1}(n)$ over these integers: 
\begin{itemize} 

\item[(A)] $g^{-1}(1) = 1$; 
\item[(B)] $\operatorname{sgn}(g^{-1}(n)) = \mu(n) \equiv \lambda(n)$; 
\item[(C)] We can write the inverse function at squarefree $n$ as 
     \[
     g^{-1}(n) = \sum_{m=0}^{\omega(n)} \binom{\omega(n)}{m} \cdot m!. 
     \]
\end{itemize} 
We illustrate parts (B)--(C) of this conjecture clearly using 
Table \ref{table_conjecture_Mertens_ginvSeq_approx_values} given on 
page \pageref{table_conjecture_Mertens_ginvSeq_approx_values} of the appendix section. 
\end{conjecture} 

The realization that the beautiful and remarkably simple form of property (C) 
in Conjecture \ref{lemma_gInv_MxExample} holds for all squarefree $n \geq 1$ 
motivates our pursuit of formulas for the inverse functions $g^{-1}(n)$ based on the configuration of the 
exponents in the prime factorization of any $n \geq 2$. 
The summation methods we employ in Section \ref{Section_InvFunc_PreciseExpsAndAsymptotics} 
to weight sums of our arithmetic functions according to the sign of 
$\lambda(n)$ (or parity of $\Omega(n)$) is also 
reminiscent of the combinatorially motivated sieve methods in 
\cite[\S 17]{OPERADECRIBERO}. 

\begin{remark}[Comparison to formative methods for bounding $M(x)$]
Note that since the DGF of $\omega(n)$ is given by 
$D_{\omega}(s) = P(s) \zeta(s)$ where $P(s)$ is the \emph{prime zeta function}, we do have a 
Dirichlet series for the inverse functions to invert coefficient-wise using more classical 
contour integral methods\footnote{
E.g., using contour integration or the following integral formula for Dirichlet series 
inversion \cite[\S 11]{APOSTOLANUMT}: 
\[
f(n) = \lim_{T \rightarrow \infty} \frac{1}{2T} \int_{-T}^{T} 
     \frac{n^{\sigma+\imath t}}{\zeta(\sigma+\imath t)(P(\sigma+\imath t) + 1)}, \sigma > 1. 
\]
Fr\"oberg has also previously done some preliminary investigation as to the properties of the 
inversion to find the coefficients of $(1+P(s))^{-1}$ in \cite{FROBERG-1968}. 
}. 
However, the uniqueness to our new methods is that our new approach does not rely on typical constructions for 
bounding $M(x)$ based on estimates of the non-trivial zeros of the Riemann zeta function that have so far 
been employed to bound the Mertens function from above. 
That is, we will instead take a more combinatorial tack to investigating bounds on this inverse function 
sequence in the coming sections. By Corollary \ref{cor_Mx_gInvnPixk_formula}, 
once we have established bounds on this $g^{-1}(n)$ and its summatory function, we will be able to 
formulate new lower bounds (in the limit supremum sense) on $M(x)$. 
\end{remark} 

\subsection{Fixing an exact expression for $M(x)$ through special sequences of arithmetic functions} 

From this point on, we fix the notation for the Dirichlet invertible function $g(n) := \omega(n) + 1$ and denote its 
inverse with respect to Dirichlet convolution by $g^{-1}(n) = (\omega+1)^{-1}(n)$. 
For natural numbers $n \geq 1, k \geq 0$, let 
\begin{align*} 
C_k(n) := \begin{cases} 
     \varepsilon(n) = \delta_{n,1}, & \text{ if $k = 0$; } \\ 
     \sum\limits_{d|n} \omega(d) C_{k-1}(n/d), & \text{ if $k \geq 1$. } 
     \end{cases} 
\end{align*} 
We have limiting asymptotics on these functions given by the following theorem: 
\begin{theorem}[Asymptotics for the functions $C_k(n)$] 
\label{theorem_Ckn_GeneralAsymptoticsForms} 
For $k := 0$, we have by definition that $C_0(n) = \delta_{n,1}$. 
For all $k \geq 1$, we obtain that the dominant asymptotic term for 
$C_k(n)$ is given by 
\[
\mathbb{E}[C_k(n)] = (\log\log n)^{2k-1}, \mathrm{\ as\ }n \rightarrow \infty. 
\]
\end{theorem} 

Since we have that 
\begin{equation} 
\label{eqn_AnExactFormulaFor_gInvByMobiusInv_v1} 
(g^{-1} \ast 1)(n) = \lambda(n) \cdot C_{\Omega(n)}(n), \forall n \geq 1, 
\end{equation} 
M\"{o}bius inversion provides us with an exact divisor sum based expression for $g^{-1}(n)$ 
(see Lemma \ref{lemma_AnExactFormulaFor_gInvByMobiusInv_v1}). 
Then we can prove (see Corollary \ref{cor_ASemiForm_ForGInvx_v1}) that we can obtain lower bounds on 
the magnitude of $g^{-1}(n)$ by approximating it by the simpler divisor sums 
\[
\lambda(n) \times \sum_{d|n} C_{\Omega(d)}(d). 
\]
Notice that this formula is substantially easier to evaluate than the corresponding sums in 
\eqref{eqn_AnExactFormulaFor_gInvByMobiusInv_v1} given directly through M\"obius inversion. Hence, 
we prefer to work with bounds on it that we prove as new results instead rather than with 
results relying on the more complicated exact formula from the cited equation above. 

Specifically, the last result in turn implies that 
\begin{equation} 
\label{eqn_GInvx_prelim_sum_formulas_intro_v1} 
|G^{-1}(x)| \SuccSim \left\lvert 
     \sum_{n \leq x} \lambda(n) \cdot C_{\Omega(n)}(n) \times 
     \sum_{d=1}^{\Floor{x}{n}} \lambda(d) \right\rvert. 
\end{equation} 
In light of the fact that 
(by an integral-based interpretation of integer convolution using summation by parts; see 
Proposition \ref{prop_Mx_SBP_IntegralFormula}) 
\[
M(x) \sim G^{-1}(x) - \sum_{k=1}^{x/2} G^{-1}(k) \cdot \frac{x}{k^2 \log(x/k)}, 
\]
the formula in \eqref{eqn_GInvx_prelim_sum_formulas_intro_v1} implies that we can establish 
new \emph{lower bounds} on $M(x)$ by appropriate estimates of the summatory function 
$G^{-1}(x)$ where trivially we have the bounded inner sums 
$L_0(x) := \sum_{n \leq x} \lambda(n) \in [-x, x]$ for all finite $x \geq 2$. 
As explicit lower bounds for $M(x)$ along particular subsequences are not obvious, and are 
historically ellusive non-trivial features of the function to obtain as 
we expect sign changes of this function infinitely often, we find this approach to be an effective one. 

\subsection{Enumerative (or counting based) DGFs from Mongomery and Vaughan} 

Now, having motivated why we must carefully estimate the $G^{-1}(x)$ bounds using our new 
methods, we will require the bounds suggested in the next section to work at bounding the 
summatory functions, $G^{-1}(x)$, for large $x$ as $x \rightarrow \infty$. 
The precise formulations of the inverse function asymptotics 
proved in Section \ref{Section_InvFunc_PreciseExpsAndAsymptotics} depend on being able to form 
significant lower bounds on the summatory functions of an always positive arithmetic function 
weighted by $\lambda(n)$. 

Our inspiration for the new bounds found in the last sections of this article allows us to sum 
non-negative arithmetic functions weighted by the Liouville lambda function, 
$\lambda(n) = (-1)^{\Omega(n)}$. In particular, it uses a hybrid generating function and 
enumerative DGF method 
under which we are able to recover ``good enough'' asymptotics about the summatory functions that 
encapsulate the parity of $\lambda(n)$ through the summatory count functions $\widehat{\pi}_k(x)$. 
The precise statement of the theorem that we transform to state these new bounds is re-stated as 
Theorem \ref{theorem_HatPi_ExtInTermsOfGz} below. 

\begin{theorem}[Montgomery and Vaughan, \S 7.4]
\label{theorem_HatPi_ExtInTermsOfGz} 
Recall that we have defined 
$$\widehat{\pi}_k(x) := \#\{n \leq x: \Omega(n)=k\}.$$ 
For $R < 2$ we have that 
\[
\widehat{\pi}_k(x) = \mathcal{G}\left(\frac{k-1}{\log\log x}\right) \frac{x}{\log x} 
     \frac{(\log\log x)^{k-1}}{(k-1)!} \left(1 + O_R\left(\frac{k}{(\log\log x)^2}\right)\right),  
\]
uniformly for $1 \leq k \leq R \log\log x$ where 
\[
\mathcal{G}(z) := \frac{F(1, z)}{\Gamma(z+1)} = \frac{1}{\Gamma(z+1)} \times 
     \prod_p \left(1-\frac{z}{p}\right)^{-1} \left(1-\frac{1}{p}\right)^z, z \geq 0. 
\]
\end{theorem} 

The next theorem, proved carefully in Section \ref{Section_MVCh7_GzBounds}, 
is the primary starting point for our new asymptotic lower bounds. 

\begin{theorem}[Generating functions of symmetric functions] 
\label{theorem_GFs_SymmFuncs_SumsOfRecipOfPowsOfPrimes} 
\label{cor_BoundsOnGz_FromMVBook_initial_stmt_v1} 
We obtain lower bounds of the following form on the function 
$\mathcal{G}(z)$ from Theorem \ref{theorem_HatPi_ExtInTermsOfGz} 
for $A_0 > 0$ an absolute constant, for 
$C_0(z)$ a strictly linear function only in $z$, and 
where we must take $0 \leq z \leq 1$, or equivalently $1 \leq k \leq \log\log x$ for $x$ large: 
\begin{align*} 
\mathcal{G}(z) \geq A_0 \cdot (1-z)^{3} \cdot C_0(z)^{z}. 
\end{align*} 
It suffices to take the components to the bound in the previous equation as 
\begin{align*}
A_0 & = \frac{2^{9/16} \exp\left(-\frac{55}{4} \log^2(2)\right)}{ 
     (3e\log 2)^3 \cdot \Gamma\left(\frac{5}{2}\right)} \approx 3.81296 \times 10^{-6} \\ 
C_0(z) & = \frac{4(1-z)}{3e \log 2}. 
\end{align*} 
In particular, with $0 \leq z \leq 1$ and 
$z \equiv z(k, x) = \frac{k-1}{\log\log x}$, by Theorem \ref{theorem_HatPi_ExtInTermsOfGz}, 
we have that 
\[
\widehat{\pi}_k(x) \SuccSim \frac{A_0 \cdot x}{\log x \cdot (\log\log x)^4 \cdot (k-1)!} \cdot 
     \left(\frac{4}{3e\log 2}\right)^{k}.
\]
\end{theorem} 

\subsection{Nearly cracking the classical unboundedness barrier} 

In Section \ref{Section_KeyApplications}, 
we provide the culmination of what we build up to in the proofs established in 
prior sections of the article. 
Namely, we prove the form of an explicit limiting lower bound for the 
summatory function, $G^{-1}(x) := \sum_{n \leq x} g^{-1}(n)$, along a specific subsequence. 
What we obtain is the following important summary corollary that comes close 
(very close, by a factor of $\log x$) to resolving the classical question of the 
unboundedness of the scaled function Mertens function 
$|M(x)| / \sqrt{x}$ in the limit supremum sense: 

\begin{cor}[Lower Bounds for the Mertens function] 
\label{cor_ThePipeDreamResult_v1} 
Let $u_0 := e^{e^{e^{e^{e^{e}}}}}$ and define the infinite increasing subsequence, 
$\{x_{0,n}\}_{n \geq 1}$, by $x_{0,n} := e^{e^{e^{e^{4n \cdot \ceiling{e^{e^{4n}}}}}}}$. 
We have that along the increasing subsequence $x_y$, for some 
$x_y \in \left(x_{0,y-1}, x_{0,y+1}\right)$, for large all sufficiently large 
$y \geq \max\left(\ceiling{x_{0,1}}+1, u_0+2\right)$ the following bound holds: 
\begin{align*} 
\frac{|M(x_y)| \log \sqrt{x_y}}{\sqrt{x_y}} & \SuccSim 
     2C_{\ell,1} \cdot (\log\log \sqrt{x_y}) 
     \frac{(\log\log\log \sqrt{x_y})^{2\log 2+\frac{1}{3\log 2} - 2}}{ 
     (\log\log\log\log \sqrt{x_y})^{\frac{5}{2}}} \cdot 
     \frac{\log_{\ast}^5(\sqrt{x_y})^{2\log 2 + \frac{1}{3\log 2}}}{ 
     \log_{\ast}^{6}(\sqrt{x_y})^{\frac{5}{2}}},  
\end{align*} 
as $y \rightarrow \infty$. In the previous equation, we adopt the notation for the 
absolute constant $C_{\ell,1} > 0$ defined more precisely by 
\[
C_{\ell,1} := \frac{B_{\ell} \cdot 128 \cdot 2^{1/8}}{6561 \cdot e^6 \pi \log^6(2)} 
     \exp\left(-\frac{55}{2} \log^2(2)\right) 
     \approx B_{\ell} \cdot 2.76631 \times 10^{-10}, 
\]
for some constant $B_{\ell} > 0$. 
\end{cor} 

This is all to say that in establishing the rigorous proof of 
Corollary \ref{cor_ThePipeDreamResult_v1} 
based on our new methods, we not only show that 
\[
\limsup_{x \rightarrow \infty} \frac{|M(x)| \log x}{\sqrt{x}} = +\infty, 
\]
but also set a minimal rate (along a large infinite subsequence) at which this form of the 
scaled Mertens function grows without bound. 

\newpage 
\section{Preliminary proofs of lemmas and new results} 
\label{Section_PrelimProofs_Config} 

The purpose of this section in the exposition of the article is to provide proofs and statements 
of fairly elementary and known facts and results. In particular, the proof of 
Theorem \ref{theorem_SummatoryFuncsOfDirCvls} allows us to easily justify the formula in 
\eqref{eqn_Mx_gInvnPixk_formula}. 
This formula is the crucial formulation that constiutes an exact expression for $M(x)$. 
The indispensible properties inherent to the arithmetic functions, $\omega(n)$ and $\pi(x)$ and $g^{-1}(n)$, 
that are used to state the formula are strong additivity, which leads to the sign of the inverse function 
$g^{-1}(n)$ being given by $\lambda(n)$ and hence intimately tied to the exact limiting 
distribution of the average values of $\Omega(n)$, and the fact that $\pi(x) \sim \frac{x}{\log x}$ 
is a monotonically non-decreasing function with asymptotic approximation represented by a smooth, integrable 
function of $x$. 
The proof of Proposition \ref{prop_SignageDirInvsOfPosBddArithmeticFuncs_v1} 
is not deep in and of itself, though does nonetheless require careful attention since we rely so 
heavily on bounds that are constructed using this property in later sections, e.g., by bounding 
signed summatory functions of $\widehat{\pi}_k(x)$ on the average. 

\subsection{Establishing the summatory function inversion identities} 

There are a vast number of Dirichlet convolution 
identities for special number theoretic functions over which 
we can form summatory functions and perform inversion via 
Theorem \ref{theorem_SummatoryFuncsOfDirCvls} \cite{CATALOG-INTDIRSERIES,CATALOG-LAMBERTSERIES}. 
This compendia of identities and standard applications suggests that our new methods may be 
broadly applicable to obtaining new bounds on the summatory functions of other classically special 
arithmetic functions. 
We will prove this useful theorem, a crucial component to our new more combinatorial 
formulations used to bound $M(x)$ in later sections, using matrix methods before moving on. 
Related results on summations of Dirichlet convolutions appear in 
\cite[\S 2.14; \S 3.10; \S 3.12; \cf \S 4.9, p.\ 95]{APOSTOLANUMT}. 

\begin{proof}[Proof of Theorem \ref{theorem_SummatoryFuncsOfDirCvls}] 
\label{proofOf_theorem_SummatoryFuncsOfDirCvls} 
Let $h,g$ be arithmetic functions where $g(1) \neq 0$ 
necessarily has a Dirichlet inverse. Denote the summatory functions of $h$ and $g$, 
respectively, by $H(x) = \sum_{n \leq x} h(n)$ and $G(x) = \sum_{n \leq x} g(n)$. 
We define $\pi_{g \ast h}(x)$ to be the summatory function of the 
Dirichlet convolution of $g$ with $h$: $g \ast h$. 
Then we can easily see that the following expansions hold: 
\begin{align*} 
\pi_{g \ast h}(x) & := \sum_{n=1}^{x} \sum_{d|n} g(n) h(n/d) = \sum_{d=1}^x g(d) H\left(\floor{\frac{x}{d}}\right) \\ 
     & = \sum_{i=1}^x \left[G\left(\floor{\frac{x}{i}}\right) - G\left(\floor{\frac{x}{i+1}}\right)\right] H(i). 
\end{align*} 
We form the matrix of coefficients associated with this system for $H(x)$, and proceed to invert it to express an 
exact solution for this function over all $x \geq 1$. Let the ordinary (initial, non-inverse) matrix entries be denoted by 
\[
g_{x,j} := G\left(\floor{\frac{x}{j}}\right) - G\left(\floor{\frac{x}{j+1}}\right) \equiv G_{x,j} - G_{x,j+1}. 
\]
The matrix we must invert in this problem is lower triangular, with ones on its diagonals -- and hence is invertible. 
Moreover, if we let $\hat{G} := (G_{x,j})$, then this matrix is 
expressable by an invertible shift operation as 
\[
(g_{x,j}) = \hat{G} (I - U^{T}); \qquad U = (\delta_{i,j+1}), (I - U^T)^{-1} = (\Iverson{j \leq i}). 
\]
Here, $U$ is the $N \times N$ matrix whose $(i,j)^{th}$ entries are defined by 
$(U)_{i,j} = \delta_{i+1,j}$. 

It is a useful fact that if we take successive differences of floor functions, we get non-zero behavior at divisors: 
\[
G\left(\floor{\frac{x}{j}}\right) - G\left(\floor{\frac{x-1}{j}}\right) = 
     \begin{cases} 
     g\left(\frac{x}{j}\right), & \text{ if $j | x$; } \\ 
     0, & \text{ otherwise. } 
     \end{cases}
\]
We use this property to shift the matrix $\hat{G}$, and then invert the result, to obtain a matrix involving the 
Dirichlet inverse of $g$: 
\begin{align*} 
\left[(I-U^{T}) \hat{G}\right]^{-1} & = \left(g\left(\frac{x}{j}\right) \Iverson{j|x}\right)^{-1} = 
     \left(g^{-1}\left(\frac{x}{j}\right) \Iverson{j|x}\right). 
\end{align*} 
Now we can express the inverse of the target matrix $(g_{x,j})$ in terms of these Dirichlet inverse functions 
as follows: 
\begin{align*} 
(g_{x,j}) & = (I-U^{T})^{-1} \left(g\left(\frac{x}{j}\right) \Iverson{j|x}\right) (I-U^{T}) \\ 
(g_{x,j})^{-1} & = (I-U^{T})^{-1} \left(g^{-1}\left(\frac{x}{j}\right) \Iverson{j|x}\right) (I-U^{T}) \\ 
     & = \left(\sum_{k=1}^{\floor{\frac{x}{j}}} g^{-1}(k)\right) (I-U^{T}) \\ 
     & = \left(\sum_{k=1}^{\floor{\frac{x}{j}}} g^{-1}(k) - \sum_{k=1}^{\floor{\frac{x}{j+1}}} g^{-1}(k)\right). 
\end{align*} 
Thus the summatory function $H$ is exactly expressed by the inverse vector product of the form 
\begin{align*} 
H(x) & = \sum_{k=1}^x g_{x,k}^{-1} \cdot \pi_{g \ast h}(k) \\ 
     & = \sum_{k=1}^x \left(\sum_{j=\floor{\frac{x}{k+1}}+1}^{\floor{\frac{x}{k}}} g^{-1}(j)\right) \cdot \pi_{g \ast h}(k). 
     \qedhere
\end{align*} 
\end{proof} 

\subsection{Proving the crucial signedness property from the conjecture} 

\begin{prop}[The characteristic function of the primes] 
\label{prop_AntiqueDivisorSumIdent} 
Let $\chi_{\mathbb{P}}$ denote the characteristic function of the primes, 
$\varepsilon(n) = \delta_{n,1}$ be the multiplicative identity with respect to Dirichlet convolution, 
and denote by $\omega(n)$ the incompletely additive function that counts the number of 
distinct prime factors of $n$. 
Then we have the convolution identity given by 
$$\chi_{\mathbb{P}} + \varepsilon = (\omega + 1) \ast \mu.$$ 
The summatory function of the left-hand-side of the previous equation is 
clearly $\widetilde{G}(x) = \pi(x)+1$ in the notation of 
Corollary \ref{cor_CvlGAstMu} for all $x \geq 1$. 
\end{prop}
\begin{proof} 
The core is to prove that for all $n \geq 1$, 
$\chi_{\mathbb{P}}(n) = (\mu \ast \omega)(n)$, an equivalent form of our essential claim. 
We notice that the Mellin transform of $\pi(x)$, the summatory function of 
$\chi_{\mathbb{P}}(n)$, evaluated at the parameter $-s$ is given by 
\begin{align*} 
s \cdot \int_1^{\infty} \frac{\pi(x)}{x^{s+1}} dx & = \sum_{n \geq 1} \frac{\nabla[\pi](n-1)}{n^s} \\ 
     & = \sum_{n \geq 1} \frac{\chi_{\mathbb{P}}(n)}{n^s} = P(s), 
\end{align*} 
where $\nabla[f](n) := f(n+1) - f(n)$ denotes the standard 
\emph{forward difference operator} used to express a discrete derivative type operation on 
arithmetic functions. 
This is a typical construction that is used as a tool to more generally relate the Mellin transform 
$s \cdot \mathcal{M}[S_f](-s)$ to the 
DGF of the sequence $f(n)$, cited, for example, as in \cite[\S 11]{APOSTOLANUMT}. 
In essence, what the previous equation says is that the DGF of $\chi_{\mathbb{P}}$ is $P(s)$. 

Now to show the equivalence of the prime indicator function with the Dirichlet convolution based 
expression, $\omega \ast \mu$, we consider the 
DGF of the right-hand-side function, $f(n) := (\mu \ast \omega)(n)$, as 
\[
D_f(s) = \frac{1}{\zeta(s)} \times \sum_{n \geq 1} \frac{\omega(n)}{n^s} = P(s),  
\]
where it is not difficult to prove that the DGF of $\omega(n)$ is $P(s) \cdot \zeta(s)$. 

Thus for any $\Re(s) > 1$, the DGFs of each side of the 
claimed equation coincide. So by uniqueness of Dirichlet series, we see that in fact the claim 
holds. To obtain the full result, we add to each side of this equation a term of 
$\varepsilon(n) \equiv (\mu \ast 1)(n)$, and then factor the resulting convolution identity. 
\end{proof} 

When combined with Corollary \ref{cor_CvlGAstMu}, the 
proof of Proposition \ref{prop_AntiqueDivisorSumIdent} yields the crucial 
starting point providing an exact 
formula for $M(x)$ stated in \eqref{eqn_Mx_gInvnPixk_formula} of 
Corollary \ref{cor_Mx_gInvnPixk_formula}. Thus, while the formula in \eqref{eqn_Mx_gInvnPixk_formula} 
is a key component utilized in our proof moving forward, we do not need to explicitly show that it holds 
for all $x \geq 1$ from this point. 

\begin{prop}[The key signedness property of $g^{-1}(n)$]
\label{prop_SignageDirInvsOfPosBddArithmeticFuncs_v1} 
For the Dirichlet invertible function, $g(n) := \omega(n) + 1$ defined such that $g(1) = 1$, at any 
$n \geq 1$, we have that $\operatorname{sgn}(g^{-1}(n)) = \lambda(n)$. 
The notation for the operation given by 
$\operatorname{sgn}(h(n)) = \frac{h(n)}{|h(n)| + \Iverson{h(n) = 0}} \in \{0, \pm 1\}$ denotes the sign 
of the arithmetic function $h$ at $n$. 
\NBRef{A02-2020-04-26}
\end{prop} 
\begin{proof} 
Let $D_f(s) := \sum_{n \geq 1} f(n) n^{-s}$ denote the Dirichlet generating function (DGF) of any 
arithmetic function $f(n)$ convergent for $\Re(s) > \sigma_f$. 
Using Proposition \ref{prop_AntiqueDivisorSumIdent} and the known property that the DGF of $f^{-1}(n)$ is 
the reciprocal of the DGF of the original arithmetic function $f$, we can express the DGF of our particular 
$g^{-1}(n)$ explicitly as an analytic function of $s$ for $\Re(s) > 1$. 
For all $\Re(s) > 1$, 
expanding the DGF for the function $g^{-1}(n)$ yields 
\begin{align} 
\label{eqn_DGF_of_gInvn} 
D_{(\omega+1)^{-1}}(s) = \frac{1}{(P(s)+1) \zeta(s)}. 
\end{align} 
Let $h^{-1}(n) := (\omega \ast \mu + \varepsilon)^{-1}(n) = [n^{-s}](P(s) + 1)^{-1}$. 
By the standard recurrence relation we used to define the Dirichlet inverse function of any 
arithmetic function $f$ such that $f(1) \neq 0$ in the initial listing of notation for the article, 
we have that 
\begin{equation} 
\label{eqn_proof_tag_hInvn_ExactRecFormula_v1}
h^{-1}(n) = \begin{cases} 
            1, & n = 1; \\ 
            -\sum\limits_{\substack{d|n \\ d>1}} h(d) h^{-1}(n/d), & n \geq 2. 
            \end{cases} 
\end{equation} 
We also can see by the definition of $h^{-1}(n)$ we derived from the partial DGF term above, that 
with $h = \omega \ast \mu + \varepsilon$, we obtain 
\[
h(n) = \begin{cases} 
       1, & n = 1; \\ 
       \chi_{\mathbb{P}}(n), & n \geq 2. 
       \end{cases} 
\]
So for $n \geq 2$, the summands in \eqref{eqn_proof_tag_hInvn_ExactRecFormula_v1} 
can be simply indexed over the primes $p|n$. This observation yields that we can inductively 
expand these sums into nested divisor sums provided the depth of the sums does not exceed the 
capacity to index summations over the primes dividing $n$. In particular, notice that for $n \geq 2$ 
\begin{align*} 
h^{-1}(n) & = -\sum_{p|n} h^{-1}(n/p) = -\sum_{p|n} h^{-1}(p), && \text{\ if\ } \Omega(n) \geq 1 \\ 
     & = \sum_{p_1|n} \sum_{p_2|\frac{n}{p_1}} h^{-1}\left(\frac{n}{p_1p_2}\right) = 
     \sum_{p_1|n} \sum_{p_2|\frac{n}{p_1}} h^{-1}(p_2), && \text{\ if\ } \Omega(n) \geq 2 \\ 
     & = -\sum_{p_1|n} \sum_{p_2|\frac{n}{p_1}} \sum_{p_3|\frac{n}{p_1p_2}} h^{-1}\left(\frac{n}{p_1p_2p_3}\right) = 
     -\sum_{p_1|n} \sum_{p_2|\frac{n}{p_1}} \sum_{p_3|\frac{n}{p_1p_2}} h^{-1}(p_3), && \text{\ if\ } \Omega(n) \geq 3. 
\end{align*} 
Then by induction, again with $h^{-1}(1) = 1$, we obtain by expanding the 
nested divisor sums as above to their maximal depth that 
\[
h^{-1}(n) = \lambda(n) \times \sum_{p_1|n} \sum_{p_2|\frac{n}{p_2}} \times \cdots \times 
     \sum_{p_{\Omega(n)}|\frac{n}{p_1p_2 \cdots p_{\Omega(n)-1}}} 1, n \geq 2. 
\]
Notice that since the nested sums in the previous equation are only taken over individual primes 
(as opposed to prime powers, or maximal prime powers) that divide the relevant index, 
we obtain repition of the values of these sums with respect to the exponents of the distinct primes dividing $n$. 

Namely, if for $n \geq 2$ we write the prime factorization of $n$ as 
$n = p_1^{\alpha_1} p_2^{\alpha_2} \cdots p_{\omega(n)}^{\alpha_{\omega(n)}}$ where the exponents $\alpha_i \geq 1$ are all 
non-zero for $1 \leq i \leq \omega(n)$, we can see that 
depending on the order in which the primes in the nested sums is invoked we have somewhat trivially 
(maximally, minimally) stated bounds on $h^{-1}$ providing that 
\begin{align*} 
h^{-1}(n) & \geq \lambda(n) \times 1 \cdot 2 \cdot 3 \cdots \omega(n) = \lambda(n) \times (\omega(n))!, && n \geq 2 \\ 
h^{-1}(n) & \leq \lambda(n) \times (\omega(n))!^{\max(\alpha_1, \alpha_2, \ldots, \alpha_{\omega(n)})}, && n \geq 2. 
\end{align*} 
In other words, what these bounds show is that for all $n \geq 1$ (with $\lambda(1) = 1$), 
\begin{equation} 
\label{eqn_proof_tag_SignedTimesPosConstantFormOf_hInvn_v2}
\exists 1 \leq C_{0,n} < +\infty: h^{-1}(n) = \lambda(n) \times C_{0,n}.
\end{equation}
Now since the DGF of $\mu(n)$ is $1 / \zeta(s)$ for $\Re(s) > 1$, and we are aiming to compute the sign terms on 
the coefficients on the DGF of $g^{-1}(n)$ given in \eqref{eqn_DGF_of_gInvn}, 
we actually need the $\pm 1$ signage on the convolution functions $(h^{-1} \ast \mu)(n)$ for each $n \geq 1$ 
(since the DGF of a convolution is the product of the component function DGFs, and conversely). 

By \eqref{eqn_proof_tag_SignedTimesPosConstantFormOf_hInvn_v2}, we immediately have bounding constants 
$1 \leq C_{1,n}, C_{2,n} < +\infty$ for all $n \geq 1$ so that 
\begin{equation} 
\label{eqn_proof_tag_hInvMunCvl_UpperLowerBounds_v3} 
C_{1,n} \cdot (\lambda \ast \mu)(n) \leq (h^{-1} \ast \mu)(n) \leq C_{2,n} \cdot (\lambda \ast \mu)(n). 
\end{equation} 
Thus, if we can show that the sign of $\pm 1$ on each of the upper and lower bounds in the previous equation 
match (are the same), then whatever that sign term corresponds to at each $n \geq 1$ provides an arithmetic 
function indicating the sign of $g^{-1}(n)$ over all $n \geq 1$. The remainder of the proof uses properties of multiplicative 
functions applied to the common factors of the 
bounding convolution function terms, $\lambda \ast \mu$, in 
\eqref{eqn_proof_tag_hInvMunCvl_UpperLowerBounds_v3}.  

Since both $\lambda,\mu$ are multiplicative, $\lambda \ast \mu$ is multiplicative, where we know that the values of 
any multiplicative function are uniquely determined by its action at prime powers. 
So we can compute that for any prime $p$ and non-negative integer exponents $\alpha \geq 1$ that 
\begin{align*} 
(\lambda \ast \mu)(p^{\alpha}) & = \sum_{i=0}^{\alpha} \lambda(p^{\alpha-i}) \mu(p^{i}) \\ 
     & = \lambda(p^{\alpha}) - \lambda(p^{\alpha-1}) \\ 
     & = 
     (-1)^{\Omega(p^{\alpha})} - (-1)^{\Omega(p^{\alpha-1})} = 
     (-1)^{\alpha} - (-1)^{\alpha-1} = 
     2 \lambda(p^{\alpha}). 
\end{align*} 
Then by the multiplicativity of $\lambda(n)$, the previous inequalities derived in 
\eqref{eqn_proof_tag_hInvMunCvl_UpperLowerBounds_v3} are re-stated in the form of 
\[
2 C_{1,n} \cdot \lambda(n) \leq h^{-1}(n) \leq 2 C_{2,n} \cdot \lambda(n). 
\] 
Since the absolute constants $C_{1,n}, C_{2,n}$ are positive for each $n \geq 1$ (as we argued above), 
we clearly recover the signedness of $g^{-1}(n)$ as $\lambda(n)$ from the characterization on the bounds 
in the previous equation. That is, $\operatorname{sgn}(g^{-1}(n)) = \lambda(n)$, 
as we have claimed above in the statement of this proposition. 
\end{proof} 

\subsection{Other facts and listings of results we will need in our proofs} 
\label{subSection_OtherFactsAndResults} 

\begin{theorem}[Mertens theorem]
\label{theorem_Mertens_theorem}  
\[
P_1(x) := \sum_{p \leq x} \frac{1}{p} = \log\log x + B + o(1), 
\]
where $B \approx 0.2614972128476427837554$ is an explicitly defined absolute constant.
\end{theorem} 

\begin{cor}
\label{lemma_Gz_productTermV2} 
We have that for sufficiently large $x \gg 1$ 
\[
\prod_{p \leq x} \left(1 - \frac{1}{p}\right) = \frac{e^{-B}}{\log x}\left( 
     1 + o(1)\right). 
\]
Hence, for $1 < |z| < R < 2$ we obtain that 
\[
\prod_{p \leq x} \left(1 - \frac{1}{p}\right)^{z} = \frac{e^{-Bz}}{(\log x)^{z}} \left(1+o(1)\right)^{z}. 
\]
\end{cor} 

\begin{facts}[Exponential Integrals and Incomplete Gamma Functions] 
\label{facts_ExpIntIncGammaFuncs} 
\begin{subequations}
The following two variants of the \emph{exponential integral function} are defined by 
\cite[\S 8.19]{NISTHB} 
\begin{align*} 
\operatorname{Ei}(x) & := \int_{-x}^{\infty} \frac{e^{-t}}{t} dt, \\ 
E_1(z) & := \int_1^{\infty} \frac{e^{-tz}}{t} dt, \Re(z) \geq 0, 
\end{align*} 
where $\operatorname{Ei}(-kz) = -E_1(kz)$ for real $k > 0$. 
We have the following inequalities providing 
quasi-polynomial upper and lower bounds on $E_1(z)$: 
\begin{equation}
1-\frac{3}{4} z \leq E_1(z) - \gamma - \log z \leq 1-\frac{3}{4} z + \frac{11}{36} z^2. 
\end{equation}
A related function is the (upper) \emph{incomplete gamma function} defined by \cite[\S 8.4]{NISTHB} 
\[
\Gamma(s, x) = \int_{x}^{\infty} t^{s-1} e^{-t} dt, \Re(s) > 0. 
\]
We have the following properties of $\Gamma(s, x)$: 
\begin{align} 
\Gamma(s, x) & = (s-1)! \cdot e^{-x} \times \sum_{k=0}^{s-1} \frac{x^k}{k!}, s \in \mathbb{Z}^{+}, \\ 
\Gamma(s+1, 1) & = e^{-1} \Floor{s!}{e}, s \in \mathbb{Z}^{+}, \\ 
\Gamma(s, x) & \sim x^{s-1} \cdot e^{-x}, |x| \rightarrow +\infty. 
\end{align}
\end{subequations}
\end{facts} 

\newpage 
\section{Summing arithmetic functions weighted by $\lambda(n)$} 
\label{Section_MVCh7_GzBounds} 

The prupose of this section in the exposition of the article is two fold. 
First, we re-state a couple of key results proved in \cite[\S 7.4]{MV} that we rely on 
to state and prove Corollary \ref{theorem_MV_Thm7.20} below. This corollary is important as it shows 
that (signed) summatory functions over $\widehat{\pi}(x)$ (bounded from below in the subsequent section) 
capture the dominant asymptotics of the full summatory function formed by taking $1 \leq k \leq \log_2(x)$ when 
we truncate and instead sum only up to the uniform bound of $1 \leq k \leq \log\log x$ guaranteed by 
Theorem \ref{theorem_HatPi_ExtInTermsOfGz}. We also prove 
Theorem \ref{theorem_GFs_SymmFuncs_SumsOfRecipOfPowsOfPrimes} in this section. 
This key theorem allows us to establish a global minimum we can attain on the function $\mathcal{G}(z)$ from 
Theorem \ref{theorem_HatPi_ExtInTermsOfGz} by truncating the formerly stated infinite 
range of the primes $p$ over which we take a component product in the definition of this function. 
This in turn implies the uniform lower bounds on $\widehat{\pi}_k(x)$ guaranteed by that theorem by 
a straightforward manipulation of inequalities. Since we use signed summatory functions over 
$\widehat{\pi}_k(x)$ to derive average order lower bounds on $G^{-1}(x)$ in later sections of the article, 
this sub-step is also a crucial prerequisite to eventually rigirously establishing the statement of 
Corollary \ref{cor_ThePipeDreamResult_v1} -- 
our central new result offered within this article. 

\subsection{Discussion: The enumerative DGF result in Theorem \ref{theorem_HatPi_ExtInTermsOfGz} from 
            Montgomery and Vaughan} 

What the enumeratively-flavored result of Montgomery and Vaughan 
in Theorem \ref{theorem_HatPi_ExtInTermsOfGz} allows us to do is get a 
``good enough'' lower bound on sums of positive and asymptotically bounded arithmetic functions 
weighted by the Liouville lambda function, $\lambda(n) = (-1)^{\Omega(n)}$. 
For comparison, we already have known, more classical bounds due to Erd\"os (and earlier) that 
we can tightly bound \cite{ERDOS-PRIMEK-FUNC,MV} 
\[
\pi_k(x) = (1 + o(1)) \cdot \frac{x}{\log x} \frac{(\log\log x)^{k-1}}{(k-1)!}. 
\] 
We seek to approximate the right-hand-side of $\mathcal{G}(z)$ by only taking the products of the primes 
$p \leq u$, e.g., indexing the component products only over those primes 
$p \in \left\{2,3,5,\ldots,u\right\}$ for some minimal upper bound $u$ (depending on $x$) 
as $x \rightarrow \infty$. 
The results proved in Section \ref{subSection_PartialPrimeProducts_Proofs} identify a minimal parameter $u$. 

We also state the following theorem sreproduced from \cite[Thm.\ 7.20]{MV} that handle the relative 
scarcity of the distribution of the $\Omega(n)$ for $n \leq x$ such that 
$\Omega(n) > \log\log x$. 

\begin{theorem}[Bounds on exceptional values of $\Omega(n)$ for large $n$, MV 7.20] 
\label{theorem_MV_Thm7.20-init_stmt} 
Let 
\begin{align*} 
A(x, r) & := \#\left\{n \leq x: \Omega(n) \leq r \cdot \log\log x\right\}, \\ 
B(x, r) & := \#\left\{n \leq x: \Omega(n) \geq r \cdot \log\log x\right\}. 
\end{align*} 
If $0 < r \leq 1$ and $x \geq 2$, then 
\[
A(x, r) \ll x (\log x)^{r-1 - r\log r}, \text{ \ as\ } x \rightarrow \infty. 
\]
If $1 \leq r \leq R < 2$ and $x \geq 2$, then 
\[
B(x, r) \ll_R x \cdot (\log x)^{r-1-r \log r}, \text{ \ as\ } x \rightarrow \infty. 
\]
\end{theorem} 

\begin{theorem}[Bounds on exceptional values of $\Omega(n)$ for large $n$, MV 7.21] 
\label{theorem_MV_Thm7.21-init_stmt} 
We have that uniformly 
\[
\#\left\{3 \leq n \leq x: \frac{\Omega(n) - \log\log n}{\sqrt{\log\log n}} \leq 0\right\} = 
     \frac{x}{2} + O\left(\frac{x}{\sqrt{\log\log x}}\right). 
\]
\end{theorem} 

\begin{remark} 
The proofs of Theorem \ref{theorem_MV_Thm7.20-init_stmt} and 
Theorem \ref{theorem_MV_Thm7.21-init_stmt} 
are found in the cited reference as Chapter 7 of Montgomery and Vaughan where the 
CDF of the normal distribution at zero is given by $\Phi(0) = \frac{1}{2}$. 
The key interpretation we need is the result stated in the next corollary. 
In the previous theorem, the dependence on $R$, and the necessity of using the 
conditional relation $\ll_R$, serves to denote this $R$ as a 
bounding (maximally limiting) parameter on the 
input $r \in (1, R)$ to the functions $B(x, r)$. 
The precise way in which the bound 
stated in this cited theorem depends on this bounded, 
indeterminate paramater $R$ can be reviewed for reference in the proof 
algebra and relations cited in the reference \cite[\S 7]{MV}. 

The role of the parameter $R$ involved in stating the previous theorem 
is notably important as a scalar factor the upper bound on $k \leq R\log\log x$ in 
Theorem \ref{theorem_HatPi_ExtInTermsOfGz} up to which 
we obtain the valid uniform bounds in $x$ on the asymptotics for 
$\widehat{\pi}_k(x)$. 
Namely, we have a discrepancy to work out in so much as we 
can only form summatory functions over the $\widehat{\pi}_k(x)$ for 
$1 \leq k \leq R\log\log x$ using the desirable, or ``nice'', asymptotic formulas
guaranteed by Theorem \ref{theorem_HatPi_ExtInTermsOfGz}, even though we can actually 
have contributions from values distributed throughout the range $1 \leq \Omega(n) \leq \log_2(n)$ 
to handle over the full range of $n \leq x$ (i.e., when $k$ is required as an index 
into the full interval of $1 \leq k \leq \log_2(x) \leq x$). 

It is then crucial that we can show that the dominant growth of the asymptotic formulas we obtain 
for these summatory functions is captured by summing only over $k$ in the truncated range 
where the uniform formulas hold. In particular, we will require a proof 
that we can discard the terms in the summatory function 
asymptotic formulas as negligible (up to at most a constant) 
for large $x$ when they happen to fall in the 
limiting exceptional range of $\Omega(n) > R\log\log x$ for $n \leq x$. 
\end{remark} 

\begin{cor} 
\label{theorem_MV_Thm7.20} 
Using the notation for $A(x, r)$ and $B(x, r)$ from 
Theorem \ref{theorem_MV_Thm7.20-init_stmt}, 
we have that for $\delta > 0$, 
\[
0 \leq \left\lvert \frac{B(x, 1+\delta)}{A(x, 1)} \right\rvert \ll 2, 
     \mathrm{\ as\ } \delta \rightarrow 0^{+}, x \rightarrow \infty. 
\]
\end{cor} 
\begin{proof} 
The lower bound stated above should be clear. To show that the asymptotic 
upper bound is correct, we compute using Theorem \ref{theorem_MV_Thm7.20-init_stmt} and 
Theorem \ref{theorem_MV_Thm7.21-init_stmt} that 
\begin{align*} 
\left\lvert \frac{B(x, 1+\delta)}{A(x, 1)} \right\rvert & \ll 
     \left\lvert \frac{x \cdot (\log x)^{\delta - \delta\log(1+\delta)}}{ 
     \widehat{\pi}_1(x) + \widehat{\pi}_2(x) + \frac{x}{2} + 
     O\left(\frac{x}{\sqrt{\log\log x}}\right)} \right\rvert \\ 
     & \sim 
     \left\lvert \frac{x \cdot (\log x)^{\delta - \delta\log(1+\delta)}}{ 
     \frac{x}{\log x} + \frac{x \cdot (\log\log x)}{\log x} + \frac{x}{2} + 
     O\left(\frac{x}{\sqrt{\log\log x}}\right)} \right\rvert \\ 
     & = 
     \left\lvert \frac{(\log x)^{1 + \delta - \delta\log(1+\delta)}}{ 
     1 + \log\log x + \frac{\log x}{2} + o(1)}\right\rvert \\ 
     & \xrightarrow{\delta \rightarrow 0^{+}} 
     \left\lvert \frac{(\log x)}{ 
     1 + \log\log x + \frac{\log x}{2} + o(1)} \right\rvert \\ 
     & \sim 2, 
\end{align*} 
as $x \rightarrow \infty$. 
\end{proof} 

We again emphasize that 
Corollary \ref{theorem_MV_Thm7.20} implies that for sums involving $\widehat{\pi}_k(x)$ indexed by $k$, 
we can capture the dominant asymptotic behavior of these sums by taking $k$ in the truncated range 
$1 \leq k \leq \log\log x$, e.g., $0 \leq z \leq 1$ in Theorem \ref{theorem_HatPi_ExtInTermsOfGz}. 
This fact will be important when we prove 
Theorem \ref{theorem_gInv_GeneralAsymptoticsForms} in 
Section \ref{Section_KeyApplications} using a sign-weighted 
summatory function in Abel summation that depends on these functions 
(see Lemma \ref{lemma_CLT_and_AbelSummation}). 


\subsection{The key new results utilizing Theorem \ref{theorem_HatPi_ExtInTermsOfGz}} 
\label{subSection_PartialPrimeProducts_Proofs} 

We will require a handle on partial sums of integer powers of the reciprocal primes as 
functions of the integral exponent and the upper summation index $x$. 
The next corollary is not a triviality as it comes in handy when we take to the task of 
proving Theorem \ref{theorem_GFs_SymmFuncs_SumsOfRecipOfPowsOfPrimes} below. 
The next statement of Corollary \ref{cor_PartialSumsOfReciprocalsOfPrimePowers} 
effectively generalizes Mertens theorem stated previously as Theorem \ref{theorem_Mertens_theorem} 
by providing a coarse rate in $x$ below which the reciprocal prime sums tend to 
absolute constants given by the prime zeta function, $P(s)$. 

\begin{cor} 
\label{cor_PartialSumsOfReciprocalsOfPrimePowers} 
For real $s \geq 1$, let 
\[
P_s(x) := \sum_{p \leq x} p^{-s}, x \gg 2. 
\]
When $s := 1$, we have the known bound in Mertens theorem 
(see Theorem \ref{theorem_Mertens_theorem}). For $s > 1$, we obtain that 
\[
P_s(x) \approx E_1((s-1) \log 2) - E_1((s-1) \log x) + o(1). 
\]
It follows that for $s \geq 2$ we have that 
\[
P_s(x) \leq \gamma_1(s, x) + o(1). 
\]
It suffices to take the bounding function in the previous equation as 
\begin{align*}
%\gamma_0(z, x) & = -s\log\left(\frac{\log x}{\log 2}\right) + \frac{3}{4}s(s-1) \log(x/2) - 
%     \frac{11}{36} s(s-1)^2 \log^2(x) \\ 
\gamma_1(s, x) & = -s\log\left(\frac{\log x}{\log 2}\right) + \frac{3}{4}s(s-1) \log(x/2) + 
     \frac{11}{36} s(s-1)^2 \log^2(2). 
\end{align*}
\end{cor} 
\NBRef{A05-2020-04-26} 
\begin{proof} 
Let $s > 1$ be real-valued. 
By Abel summation with the summatory function $A(x) = \pi(x) \sim \frac{x}{\log x}$ and where 
our target function $f(t) = t^{-s}$ with $f^{\prime}(t) = -s \cdot t^{-(s+1)}$, we obtain that 
\begin{align*} 
P_s(x) & = \frac{1}{x^s \cdot \log x} + s \cdot \int_2^{x} \frac{dt}{t^s \log t} \\ 
     & = E_1((s-1) \log 2) - E_1((s-1) \log x) + o(1), |x| \rightarrow \infty. 
\end{align*} 
Now using the inequalities in Facts \ref{facts_ExpIntIncGammaFuncs}, we obtain that the 
difference of the exponential integral functions is bounded above and below by 
\begin{align*} 
\frac{P_s(x)}{s} & \geq -\log\left(\frac{\log x}{\log 2}\right) + \frac{3}{4}(s-1) \log(x/2) - 
     \frac{11}{36} (s-1)^2 \log^2(x) \\ 
\frac{P_s(x)}{s} & \leq -\log\left(\frac{\log x}{\log 2}\right) + \frac{3}{4}(s-1) \log(x/2) + 
     \frac{11}{36} (s-1)^2 \log^2(2). 
\end{align*} 
This completes the proof of the bounds cited above in the statement of this lemma. 
\end{proof} 

\NBRef{A06-2020-04-26} 
\begin{proof}[Proof of Theorem \ref{theorem_GFs_SymmFuncs_SumsOfRecipOfPowsOfPrimes}] 
\label{proofOf_theorem_GFs_SymmFuncs_SumsOfRecipOfPowsOfPrimes} 
We have that for all integers $0 \leq k \leq m$
\begin{equation} 
\label{eqn_pf_tag_hSymmPolysGF} 
[z^k] \prod_{1 \leq i \leq m} (1-f(i) z)^{-1} = [z^k] \exp\left(\sum_{j \geq 1} 
     \left(\sum_{i=1}^m f(i)^j\right) \frac{z^j}{j}\right). 
\end{equation} 
In our case we have that $f(i)$ denotes the $i^{th}$ prime. 
Hence, summing over all $p \leq ux$ 
in place of $0 \leq k \leq m$ in the previous formula, and in tandem with 
Corollary \ref{cor_PartialSumsOfReciprocalsOfPrimePowers}, we obtain that the logarithm of the 
generating function series we are after when we sum over all $p \leq ux$ for some parameter 
$u$ that we must next determine corresponds to 
\begin{align*} 
\log\left[\prod_{p \leq ux} \left(1-\frac{z}{p}\right)^{-1}\right] & \geq (B + \log\log (ux)) z + 
     \sum_{j \geq 2} \left[a(ux) + b(ux)(j-1) + c(ux) (j-1)^2\right] z^j \\ 
     & = (B + \log\log (ux)) z - a(ux) \left(1 + \frac{1}{z-1} + z\right) \\ 
     & \phantom{= (B + \log\log (ux)) z\ } + 
     b(ux) \left( 
     1 + \frac{2}{z-1} + \frac{1}{(z-1)^2}\right) \\ 
     & \phantom{= (B + \log\log (ux)) z\ } - 
     c(ux) \left( 
     1 + \frac{4}{z-1} + \frac{5}{(z-1)^2} + \frac{2}{(z-1)^3}\right) \\ 
     & =: \widehat{\mathcal{B}}(u, x; z). 
\end{align*} 
In the previous equations, the lower bounds formed by the functions $(a,b,c)$ 
evaluated at $ux$ are 
given by the corresponding upper bounds from 
Corollary \ref{cor_PartialSumsOfReciprocalsOfPrimePowers} 
due to the leading sign on the previous expansions as 
\begin{align*} 
(a_{\ell}, b_{\ell}, c_{\ell}) & := \left(-\log\left(\frac{\log (ux)}{\log 2}\right), 
     \frac{3}{4} \log\left(\frac{ux}{2}\right), \frac{11}{36} \log^2 2\right). 
\end{align*} 
Now we make a practical decision to set the uniform bound parameter to a middle ground value of 
$1 < R < 2$ at $R := \frac{3}{2}$ 
(practically, to be truncated and taken as though $R \equiv 1$ in sums) so that 
$$z \equiv z(k, x) = \frac{k}{\log\log x} \in (0, R),$$ for $x \gg 1$ very large. 
Thus $(z-1)^{-m} \in [(-1)^m, 2^m]$ for integers $m \geq 1$, and so we can obtain the 
lower bound stated below. Namely, these bounds on the signed reciprocals of $z-1$ 
lead to an effective bound of the following form: 
\begin{align*} 
\widehat{\mathcal{B}}(u, x; z) & \geq (B + \log\log (ux)) z - a(ux) \left(1 + \frac{1}{z-1} + z\right) \\ 
     & \phantom{= (B + \log\log (ux)) z\ } + 
     b(ux) \left( 
     1 + \frac{2}{z-1} + \frac{1}{(z-1)^2}\right) - 
     45 \cdot c(ux). 
\end{align*} 
Since the function $c(ux)$ is constant, we then also obtain a refined bound of the next form. 
\begin{align} 
\notag 
\frac{e^{-Bz}}{(\log (ux))^{z}} \times \exp\left(\widehat{\mathcal{B}}(u, x; z)\right) & \geq 
    \exp\left(-\frac{55}{4} \log^2(2)\right) \times \left(\frac{\log(ux)}{\log 2}\right)^{1 + \frac{1}{z-1} + z} \\ 
\notag 
    & \phantom{\geq \times} \times \left(\frac{ux}{2}\right)^{\frac{3}{4}\left(1 + \frac{2}{z-1} + \frac{1}{(z-1)^2}\right)} \\ 
\label{eqn_proof_simpl_v1} 
     & =: \widehat{\mathcal{C}}(u, x; z). 
\end{align} 
Now we need to determine which values of $u$ minimize the expression for the function defined 
in \eqref{eqn_proof_simpl_v1}. 
For this we will use a somewhat weak elementary method from 
introdutory calculus in the form of the second derivative test with respect to $u$ that 
immediately discards most of the 
dependence of \eqref{eqn_proof_simpl_v1} on $x$ as we apply it. 
In particular, we can symbolically invoke the equation solver functionality in \emph{Mathematica} 
to see that 
\[
\frac{\partial}{\partial u}\left[\widehat{\mathcal{C}}(u, x; z)\right] \Biggr\rvert_{u \mapsto u_0} = 0 \implies 
     u_0 \in \left\{\frac{1}{x}, \frac{1}{x} e^{-\frac{4}{3}(z-1)}\right\}. 
\]
When we substitute this outstanding parameter value of $u_0 =: \hat{u}_0 \mapsto \frac{1}{x} e^{-\frac{4}{3}(z-1)}$ 
into the next expression for the second derivative of the same function 
$\widehat{\mathcal{C}}(u, x; z)$ we obtain 
\begin{align*} 
\frac{\partial^2}{{\partial u}^2}\left[\widehat{\mathcal{C}}(u, x; z)\right] \Biggr\rvert_{u = \hat{u}_0} & = 
     \exp\left(-\frac{55}{4} \log^2(2)\right) x^2 2^{\frac{8 z^3-27 z^2+32 z-16}{4 (z-1)^2}} 
     3^{-z+\frac{1}{1-z}+1} e^{\frac{5 z^2-16 z+8}{3 (z-1)}} \times \\ 
     & \phantom{=\times} \times (1-z)^{z+\frac{1}{z-1}-2} z^2
     \log^{\frac{z^2}{1-z}}(2) > 0, 
\end{align*} 
provided that $z < 1$, e.g., so that $k \leq \log\log x$ in Theorem \ref{theorem_HatPi_ExtInTermsOfGz}. 
This restriction on $k$ to note 
leads to a minimum value on the partial product, or lower bound, at this $u = \hat{u}_0$ 
since the second derivative is positive at the zero of the first derivative whenever $z < 1$. 

After a substitution of $u = \frac{1}{x} e^{-\frac{4}{3}(z-1)}$ into the expression for 
$\widehat{\mathcal{C}}(u, x; z)$ defined above, we have that 
\[
\widehat{\mathcal{C}}(u, x; z) \geq \exp\left(-\frac{55}{4} \log^2(2)\right) \cdot 2^{\frac{9}{16}} 
     \left(\frac{1-z}{3e\log 2}\right)^3 \times \left(\frac{4(1-z)}{3e\log 2}\right)^z. 
\]
Finally, since $z \equiv z(k, x) = \frac{k}{\log\log x}$ and $k \in [0, R\log\log x)$, we obtain that 
for small $k$ and $x \gg 1$ large $\Gamma(z+1) \approx 1$, and for $k$ towards the upper range of 
its interval that $\Gamma(z+1) \approx \Gamma(5/2) = \frac{3}{4} \sqrt{\pi}$. 
In total, what we get out of these formulas is stated as in the theorem bounds. 
\end{proof} 

\newpage
\section{Precisely bounding the Dirichlet inverse functions, $g^{-1}(n)$} 
\label{Section_InvFunc_PreciseExpsAndAsymptotics} 

This section is central to the exposition of the article because we prove key results that allow us to 
bound (on average) the oscillatory 
Dirichlet inverse functions $g^{-1}(n)$ from the exact formula for $M(x)$ given in 
\eqref{eqn_Mx_gInvnPixk_formula}. 
Using summation by parts, we eventually show that this formula can be approximated with a clear 
dependency on the summatory functions $G^{-1}(x)$ of $g^{-1}(n)$ by the integral formula we later 
state and prove as Proposition \ref{prop_Mx_SBP_IntegralFormula}. 
The pages of tabular data given as Table \ref{table_conjecture_Mertens_ginvSeq_approx_values} 
given starting on page \pageref{table_conjecture_Mertens_ginvSeq_approx_values} (as an appendix) 
provide insight into why we arrived at these convenient approximations to $g^{-1}(n)$ based on 
clear numerical insights formed by examining the approximate behavior 
at work here for the asymptotically 
small order cases of $1 \leq n \leq 350$ with \emph{Mathematica}. 

It happens that Conjecture \ref{lemma_gInv_MxExample} is not the most accurate 
simple way to express the limiting behavior of the 
Dirichlet inverse functions $g^{-1}(n)$ we can formulate, 
though it does capture an important characteristic. Namely, that these 
functions can be expressed via more simple formulas than inspection of the initial 
repetitive, quasi-periodic sequence properties might otherwise suggest. 

With all of this in mind, we define the following sequence for integers $n \geq 1, k \geq 0$: 
\begin{align} 
\label{eqn_CknFuncDef_v2} 
C_k(n) := \begin{cases} 
     \varepsilon(n), & \text{ if $k = 0$; } \\ 
     \sum\limits_{d|n} \omega(d) C_{k-1}(n/d), & \text{ if $k \geq 1$. } 
     \end{cases} 
\end{align} 
We will illustrate by example the first few cases of these functions for small $k$ after we prove 
the next lemma. 
The sequence of important semi-diagonals of these functions begins as 
\cite[\seqnum{A008480}]{OEIS} 
\[
\{\lambda(n) \cdot C_{\Omega(n)}(n) \}_{n \geq 1} \mapsto \{
     1, -1, -1, 1, -1, 2, -1, -1, 1, 2, -1, -3, -1, 2, 2, 1, -1, -3, -1, \
     -3, 2, 2, -1, 4, 1, 2, \ldots \}. 
\]

\begin{lemma}[An exact formula for $g^{-1}(n)$] 
\label{lemma_AnExactFormulaFor_gInvByMobiusInv_v1} 
For all $n \geq 1$, we have that 
\[
g^{-1}(n) = \sum_{d|n} \mu(n/d) \lambda(d) C_{\Omega(d)}(d). 
\]
\end{lemma}
\begin{proof} 
We first write out the standard recurrence relation for the Dirichlet inverse of 
$\omega+1$ as 
\begin{align*} 
g^{-1}(n) & = - \sum_{\substack{d|n \\ d>1}} (\omega(d) + 1) g^{-1}(n/d) && \implies \\ 
     (g^{-1} \ast 1)(n) & = -(\omega \ast g^{-1})(n). 
\end{align*} 
Now by repeatedly expanding the right-hand-side, and removing corner cases in the nested sums since 
$\omega(1) = 0$ by convention, we find that 
\[
(g^{-1} \ast 1)(n) = (-1)^{\Omega(n)} C_{\Omega(n)}(n) = \lambda(n) C_{\Omega(n)}(n). 
\]
The statement follows by M\"obius inversion applied to each side of the last equation. 
\end{proof}

\begin{example}[Special cases of the functions $C_k(n)$ for small $k$] 
\label{example_SpCase_Ckn} 
We cite the following special cases which should be easy enough to see on paper by 
explicit computation using \eqref{eqn_CknFuncDef_v2}: 
\NBRef{A07-2020-04-26} 
\begin{align*} 
C_0(n) & = \delta_{n,1} \\ 
C_1(n) & = \omega(n) \\ 
C_2(n) & = \sigma_0(n) \times \sum_{p|n} \frac{\nu_p(n)}{\nu_p(n)+1} - \gcd\left(\Omega(n), \omega(n)\right). 
\end{align*} 
We have a recurrence relation between successive $C_k(n)$ values over $k$ of the form 
\begin{equation}
\label{eqn_Ckn_recFormula_v1} 
C_k(n) = \sum_{p|n} \sum_{d\bigr\rvert\frac{n}{p^{\nu_p(n)}}} \sum_{i=1}^{\nu_p(n)} C_{k-1}\left(d \cdot p^i\right). 
\end{equation}
\end{example} 

\begin{summary}[Asymptotics of the $C_k(n)$]
We have the following asymptotic relations relations for the growth of small cases of 
the functions $C_k(n)$: 
\begin{align*} 
\mathbb{E}[C_1(n)] & = \log\log n \\ 
\mathbb{E}[C_2(n)] & = (\log\log n)^3. 
\end{align*} 
The previous limiting asymptotics are computed from the explicit formulas for small $k$ in 
Example \ref{example_SpCase_Ckn} using the average order arguments such that 
$\mathbb{E}[\nu_p(n)] = \log\log n$ and for $p|n$, $\mathbb{E}[p] = \frac{n}{\log n}$. 

Theorem \ref{theorem_Ckn_GeneralAsymptoticsForms} from the introduction is proved next. 
The theorem makes precise what these formulas already 
suggest about the main terms of the growth rates of 
$C_k(n)$ as functions of $k,n$ for limiting cases of $n$ large for fixed $k$. 
Since we will be essentially averaging the inverse functions, $g^{-1}(n)$, via their summatory functions 
over the range $n \leq x$ for $x$ large, we tend not to bound any relevant components to obtaining 
these results but by the 
average order case, which evens out when we sum (i.e., average) and tend to infinity. 
\end{summary} 

\NBRef{A08-2020-04-26} 
\begin{proof}[Proof of Theorem \ref{theorem_Ckn_GeneralAsymptoticsForms}] 
\label{proofOf_theorem_Ckn_GeneralAsymptoticsForms} 
We showed how to compute the formulas for the base cases in the preceeding examples 
discussed above in Example \ref{example_SpCase_Ckn}. 
We can also see that $C_1(n)$ satsfies the formula we must establish when $k := 1$. 
Let's proceed by using induction to prove that our asymptotics hold for all 
$k \geq 1$ using the recurrence formula from 
\eqref{eqn_Ckn_recFormula_v1} 
relating $C_k(n)$ to $C_{k-1}(n)$ whenever $k \geq 2$. 
In particular, suppose that $k \geq 2$ and let the inductive assumption for all $1 \leq m < k$ 
be that 
\[
\mathbb{E}[C_m(n)] = (\log\log n)^{2m-1}. 
\]
Now we have by the recursive formula in \eqref{eqn_Ckn_recFormula_v1} that 
\begin{align} 
\notag 
C_k(n) & = \sum_{p|n} \sum_{d\bigr\rvert\frac{n}{p^{\nu_p(n)}}} \sum_{i=1}^{\nu_p(n)} (\log\log(dp^i))^{2k-3} \\ 
\label{eqn_proof_tag_Ckn_AsymptoticExp_v1} 
     & \sim \sum_{p|n} \sum_{d\bigr\rvert\frac{n}{p^{\nu_p(n)}}} \left[ 
     \int (\log\log(dp^{\alpha}))^{2k-3} d\alpha\right] \Biggr\rvert_{\alpha=\nu_p(n)}. 
\end{align} 
The inner integral in the previous equation can be evaluated using the 
limiting asymptotic expansions for the incomplete gamma function stated in 
Section \ref{subSection_OtherFactsAndResults}. 
In particular, for $p|n$ and $n \geq 2$ large, we let the parameters assume average order values of 
\[
\mathbb{E}[\nu_p(n)] = \log\log n, \mathbb{E}[p] = \frac{n}{\log n}. 
\]
Then we evaluate the integral from above as 
\begin{align*} 
\int (\log\log(dp^{\alpha}))^{2k-3} d\alpha & \sim \alpha \left( 
     \log d + \alpha \cdot \log p\right)^{2k-3} \\ 
     & \sim \alpha \left(\log \alpha + \log\log p + \frac{d}{\alpha \log p}\right)^{2k-3}. 
\end{align*} 
We know that the average order of the number of primes $p|n$ is given by 
$\mathbb{E}[\omega(n)] = \log\log n$, so approximating $p$ as the cited function of $n$ initially 
allows us to take a factor of $\log\log n$ and remove the outer divisor sum in 
\eqref{eqn_proof_tag_Ckn_AsymptoticExp_v1}. So we obtain that \footnote{ 
     Here, we simplify the iterated logarithm expansions as $n \rightarrow \infty$ by writing 
     \begin{align*} 
     \log\log\left(\frac{n}{\log n}\right) & = \log\left[\log n + \log\left(1 + \frac{1}{n\log n}\right)\right] \\ 
          & \sim \log\log n + \frac{1}{n (\log n)^2} \\ 
          & \sim \log\log n. 
     \end{align*} 
}
\begin{align*} 
\mathbb{E}[C_k(n)] & = (\log\log n)^2 \left[
     \log\log\log n + \log\log n + \frac{\pi^2}{12} \frac{n}{\log n} \frac{1}{\left( 
     \frac{n}{\log n}\right)^{\log\log n}} 
     \right]^{2k-3} \\ 
     & \sim (\log\log n)^{2k-1}. 
\end{align*} 
In the previous equation, we have used that the average order of the sum-of-divisors function, $\sigma_1(n)$, 
is given by $\mathbb{E}[\sigma_1(n)] = \frac{\pi^2 \cdot n}{12}$ \cite[\S 27.11]{NISTHB}. 
Thus by mathematical induction, we have proved that the claimed limiting asymptotic average order 
behavior holds for $C_k(n)$ whenever $k \geq 1$ as $n \rightarrow \infty$. 
\end{proof} 

Using Lemma \ref{lemma_AnExactFormulaFor_gInvByMobiusInv_v1} directly is complicated since 
forming the summatory function of the exact $g^{-1}(n)$ that obey this formula leads to 
a nested recurrence relation involving $M(x)$, e.g., 
more in-order sums of consecutive M\"obius function terms that appear yet again in nested form. 
Some suggestive numerical experiments illustrate that this implicit recursive 
dependence of our new formulas for $M(x)$ can be avoided simply by using an inexact, but still 
provably asymptotically sufficient in form expression approximating $g^{-1}(n)$. 
The next corollary provides the specific 
inexact, and asymptotically accurate formula for these inverse functions that we have in mind. 

What Corollary \ref{cor_ComputingInvFuncs_InPractice_DivSumgInvAst1_v1} below 
allows us to do is 
provide a substantially simpler formula and limiting bound on the summatory functions 
$G^{-1}(x)$ of $g^{-1}(n)$. The form of this new formula for $G^{-1}(x)$ is 
established in Corollary \ref{cor_ASemiForm_ForGInvx_v1}, which is subsequently stated and 
easily given a short proof immediately after the next result is established. 
This is an important leap in expressing a workable formula that we can use to bound these 
summatory functions from below when $x$ is large. We require such a bound to prove the result 
rigorously justified in Theorem \ref{theorem_gInv_GeneralAsymptoticsForms}. 

\begin{cor}[A simplification in form towards computing the inverse functions] 
\label{cor_ComputingInvFuncs_InPractice_DivSumgInvAst1_v1} 
For $n \geq 2$ as $n \rightarrow \infty$ we have that 
\[
\mathbb{E}\left[\frac{\sum\limits_{d|n} C_{\Omega(d)}(d)}{|g^{-1}(n)|}\right] \leq 1. 
\]
Thus if we let 
\[
\widetilde{G}^{-1}(x) := \sum_{n \leq x} \lambda(n) \times \sum_{d|n} C_{\Omega(d)}(d), 
\]
denote the summatory function defined by approximating $g^{-1}(n)$ by 
$\lambda(n) \times \sum_{d|n} C_{\Omega(d)}(d)$, we obtain a lower bound in the form of 
$$|G^{-1}(x)| \gg \left\lvert \widetilde{G}^{-1}(x) \right\rvert.$$
\end{cor} 
\begin{proof} 
Let the approximation to the formula for $g^{-1}(n)$ from 
Lemma \ref{lemma_AnExactFormulaFor_gInvByMobiusInv_v1} be denoted by 
\[
S_R(n) := \lambda(n) \times \sum_{d|n} C_{\Omega(d)}(d). 
\]
The sign on the terms $C_{\Omega(d)}(d)$ in the cited approximation to $g^{-1}(n)$ 
given by $S_R(n)$ differs from Lemma \ref{lemma_AnExactFormulaFor_gInvByMobiusInv_v1} when 
\[
\operatorname{sgn}\left(\mu(d) \lambda(n/d)\right) = -\lambda(n) \iff 
     \operatorname{sgn}\left(\mu(d) \lambda\left(\frac{n^2}{d}\right)\right) = -1. 
\]
By a case-by-case analysis of the parity of $(n, d)$, we see that this occurs when one of two 
cases is met: 
\begin{itemize} 
\item[(1)] $n$ is even, $d$ is even, and $\mu(d) = -1$; or 
\item[(2)] $n$ is odd, $d$ is odd, and $\mu(d) = +1$. 
\end{itemize} 
According to the results on the asymptotic densities of the squarefree integers corresponding to 
$\mu(n) = \pm 1$ we cited in the preliminaries from Section \ref{subSection_MertensMxClassical_Intro}, 
we obtain the following statements: 
\begin{itemize} 
\item[(3)] The asymptotic density of the integers which are squarefree and satisfy either (1) or (2) is $\frac{3}{\pi^2}$; 
\item[(4)] The asymptotic density of the integers which are squarefree and satisfy neither (1) nor (2) is $\frac{3}{\pi^2}$. 
\end{itemize} 
Moreover, the asymptotic density of the positive integers that are not squarefree is given by 
$1 - \frac{6}{\pi^2}$. Thus the limiting density of integers such that the sign of the terms in 
$S_R(n)$ matches those in Lemma \ref{lemma_AnExactFormulaFor_gInvByMobiusInv_v1} corresponds to 
$\frac{3}{\pi^2}$. The divisors $d$ of $n$ in the expression for $S_R(n)$ that do not contribute 
any weight to the formula for $g^{-1}(n)$ in 
Lemma \ref{lemma_AnExactFormulaFor_gInvByMobiusInv_v1} have 
asymptotic density $1 - \frac{6}{\pi^2}$ (the density of the non-squarefree integers). 

The average order of the divisor function $d(n)$ is known to satisfy \cite[\S 27.11]{NISTHB} 
\[
\mathbb{E}[d(n)] = \log n + 2\gamma-1 + o(1). 
\]
When we take the difference between $g^{-1}(n)$ from the lemma and the divisor sum $S_R(n)$, 
we have to subtract off the terms for the non-squarefree integers that vanish in the exact result 
due to the M\"obius function, and then subtract off (twice) the terms that have leading coefficient of 
$+1$ where the exact formula weights the term by $-1$. What this argument leads to is an average order 
calculation showing that 
\begin{align} 
\notag 
\mathbb{E}\left[\frac{\sum\limits_{d|n} C_{\Omega(d)}(d)}{|g^{-1}(n)|}\right] & = 
     \mathbb{E}\left\lvert \frac{g^{-1}(n) - \lambda(n) \left(1 - \frac{6}{\pi^2}\right) 
     \mathbb{E}[d(n)] + \lambda(n) \frac{6}{\pi^2} \mathbb{E}[d(n)]}{g^{-1}(n)} 
     \right\rvert \\ 
\notag 
     & = 1 - \mathbb{E}\left[\frac{\log n + 2\gamma-1 + o(1)}{|g^{-1}(n)|}\right] \\ 
\label{eqn_proof_tag_v1_ExpectationFormulagInvnVersisSRn} 
     & = 1 - \frac{\log n + 2\gamma-1 + o(1)}{\mathbb{E}|g^{-1}(n)|}. 
\end{align} 
Now, due to the monotonically increasing nature of $C_k(n)$ in $k \geq 1$ for all large enough $n$, 
and since the terms in the exact expansion of $g^{-1}(n)$ in 
Lemma \ref{lemma_AnExactFormulaFor_gInvByMobiusInv_v1} are signed, 
clearly we obtain from Theorem \ref{theorem_Ckn_GeneralAsymptoticsForms} that 
\[
\mathbb{E}|g^{-1}(n)| \leq \mathbb{E}\left[C_{\Omega(n)}(n)\right] = 
     (\log\log n)^{2\log\log n-1}. 
\]
Since the right-hand-side bound in the previous equation has an asymptotically larger rate of 
growth that $\log n$, from \eqref{eqn_proof_tag_v1_ExpectationFormulagInvnVersisSRn} we have that 
the second term in the difference is $o(1)$ as $n \rightarrow \infty$. 
This implies the first bound we have claimed in this corollary. 
The second conclusion on the bounds satisfied when comparing $G^{-1}(x)$ to the approximate 
summatory function follows along the same lines as the proof of 
Lemma \ref{lemma_lowerBoundsOnLambdaFuncParitySummFuncs} 
given in Section \ref{subsubSection_RoutineProofsNeededForMainBoundOnGInvxFunc}. 
\end{proof} 

\begin{cor} 
\label{cor_ASemiForm_ForGInvx_v1} 
We have that for sufficiently large $x$, as $x \rightarrow \infty$ that 
\begin{align*} 
\left\lvert G^{-1}(x) \right\rvert & \SuccSim 
     \left\lvert 
     B_{\ell} \cdot \widehat{L}_0\left(\log\log x\right) \times \sum_{n \leq \log\log x} 
     \lambda(n) \cdot C_{\Omega(n)}(n) \right\rvert, 
\end{align*} 
for some bounded constant $B_{\ell} > 0$, where the function 
\[
\left\lvert \widehat{L}_0(x) \right\rvert \SuccSim 
     \sqrt{\frac{2}{\pi}} A_0 \cdot x 
     \frac{(\log x)^{2\log 2+ \frac{1}{3 \log 2} - 1}}{ 
     (\log\log x)^{\frac{5}{2} + \log\log x}}, 
\]
and such that $\operatorname{sgn}(\widehat{L}_0(x)) = (-1)^{\floor{\log\log x}}$ 
(as the function is defined inline below), and 
where the exponent $2\log 2+ \frac{1}{3 \log 2} - 1 \approx 0.867193$. 
\end{cor} 
\NBRef{A10-2020.04-26} 
\begin{proof} 
Using Corollary \ref{cor_ComputingInvFuncs_InPractice_DivSumgInvAst1_v1}, we have that 
\begin{align*} 
\left\lvert G^{-1}(x) \right\rvert & \gg 
     \left\lvert \sum_{n \leq x} \lambda(n) \sum_{d|n} C_{\Omega(d)}(d) \right\rvert \\ 
     & = \left\lvert \sum_{d \leq \log\log x} C_{\Omega(d)}(d) \times 
     \sum_{n=1}^{\Floor{x}{d}} \lambda(dn) \right\rvert. 
\end{align*} 
Now we see that by complete additivity of $\Omega(n)$ 
(or corresponding complete multiplicativity of $\lambda(n)$) that 
\begin{align*} 
\sum_{n=1}^{\Floor{x}{d}} \lambda(dn) & = \sum_{n=1}^{\Floor{x}{d}} \lambda(d) \lambda(n) 
     = \lambda(d) \sum_{n \leq \Floor{x}{d}} \lambda(n). 
\end{align*} 
Using the result proved in Section \ref{Section_MVCh7_GzBounds} as 
(see Theorem \ref{theorem_GFs_SymmFuncs_SumsOfRecipOfPowsOfPrimes})
we can establish that 
\begin{align*} 
\left\lvert \sum_{n \leq x} \lambda(n) \right\rvert & \gg 
     \left\lvert \sum_{k \leq \log\log x} (-1)^k \cdot \widehat{\pi}_k(x) \right\rvert 
     =: \left\lvert \widehat{L}_0(x) \right\rvert. 
\end{align*} 
Then since for large enough $x$ and $d \ll x$, 
\[
\log(x/d) \sim \log x, \log\log(x/d) \sim \log\log x, 
\] 
we can obtain the stated result, e.g., so that 
$\left\lvert \widehat{L}_0(\log\log x) \right\rvert \sim 
 \left\lvert \widehat{L}_0(\log\log (x/d)) \right\rvert$ for $d \leq \log\log x$ and 
large $x \rightarrow \infty$. 
The limiting lower bound stated for $\widehat{L}_0(x)$ is computed by symbolic summation 
in \emph{Mathematica} using the new bounds on $\widehat{\pi}_k(x)$ guaranteed by 
Theorem \ref{theorem_GFs_SymmFuncs_SumsOfRecipOfPowsOfPrimes}. 
\end{proof} 

\newpage
\section{Establishing the lower bounds for $M(x)$ by cases along infinite subsequences} 
\label{Section_KeyApplications} 

\subsection{The culmination of what we have done so far} 

As noted before in the previous subsections, we cannot hope to evaluate sums of 
functions weighted by $\lambda(n)$ that define $G^{-1}(x)$ except for on 
average using Abel summation. For this task, 
we need to know the bounds on $\widehat{\pi}_k(x)$ we developed in the 
proof of Corollary \ref{cor_BoundsOnGz_FromMVBook_initial_stmt_v1}. 
A summation by parts argument proves the core component of the next proposition providing an 
integral formula based expression that approximates $M(x)$ closely 
through the form of $G^{-1}(x)$. 

\begin{prop}
\label{prop_Mx_SBP_IntegralFormula} 
For all sufficiently large $x$, we have that 
\begin{align} 
\label{eqn_pf_tag_v2-restated_v2} 
M(x) & \approx G^{-1}(x) - x \cdot \int_1^{x/2} \frac{G^{-1}(t)}{t^2 \cdot \log(x/t)} dt, 
\end{align} 
where $G^{-1}(x) := \sum_{n \leq x} g^{-1}(n)$ is the summatory function of $g^{-1}(n)$. 
\end{prop} 
\begin{proof} 
We know by applying Corollary \ref{cor_Mx_gInvnPixk_formula} that 
\begin{align} 
\notag
M(x) & = \sum_{k=1}^{x} g^{-1}(k) (\pi(x/k)+1) \\ 
\notag
     & = G^{-1}(x) + \sum_{k=1}^{x} g^{-1}(k) \pi(x/k), 
\end{align} 
where we can drop the asymptotically unnecessary floored integer-valued arguments to $\pi(x)$ in place of 
its approximation by $\pi(x) \sim \frac{x}{\log x}$. In fact, since we can always 
bound $$\frac{Ax}{\log x} \leq \pi(x) \leq \frac{Bx}{\log x},$$ for suitably defined 
absolute constants, $A,B > 0$, we are not losing any precision asymptotically by making 
this small leap in approximation from exact summation (by the first formula) to the 
integral formula case approximating $M(x)$ established below. 

What we now require to sum and simplify the right-hand-side summation from the last equation is 
nothing short of an ordinary summation by parts argument. Namely, we obtain that for sufficiently large 
$x \geq 2$ \footnote{
     Since $\pi(1) = 0$, the actual range of summation corresponds to 
     $k \in \left[1, \frac{x}{2}\right]$. 
}
\begin{align*} 
\sum_{k=1}^{x} g^{-1}(k) \pi(x/k) & = G^{-1}(x) \pi(1) - \sum_{k=1}^{x-1} G^{-1}(k) \left[ 
     \pi\left(\frac{x}{k}\right) - \pi\left(\frac{x}{k+1}\right)\right] \\ 
     & = -\sum_{k=1}^{x/2} G^{-1}(k) \left[ 
     \pi\left(\frac{x}{k}\right) - \pi\left(\frac{x}{k+1}\right)\right] \\ 
     & \approx -\sum_{k=1}^{x/2} G^{-1}(k) \left[ 
     \frac{x}{k \cdot \log(x/k)} - \frac{x}{(k+1) \cdot \log(x/k)}\right] \\ 
     & \approx -\sum_{k=1}^{x/2} G^{-1}(k) \frac{x}{k^2 \cdot \log(x/k)}. 
\end{align*} 
Since for $x$ large enough the summand is monotonic as $k$ ranges in order over $k \in [1, x/2]$, and 
since the summands in the last equation are smooth functions of $k$ (and $x$), and also since $G^{-1}(x)$ is 
a summatory function with jumps at the positive integers (signed in magnitude or not), we can approximate 
\[
M(x) \approx G^{-1}(x) - x \cdot \int_1^{x/2} \frac{G^{-1}(t)}{t^2 \cdot \log(x/t)} dt. 
\]
Moreover, since the bounds of integration are finite, we do not need to dwell on the oscillatory nature of the 
factor of $G^{-1}(t)$ in the translation from summation to integral representation. We will 
later only use unsigned lower bound approximations to this function in the next theorems so that 
the signedness of the summatory function term in the integral formula above is even a moot point entirely. 
\end{proof} 

\subsubsection{From the routine: Proofs of a few cut-and-dry lemmas} 
\label{subsubSection_RoutineProofsNeededForMainBoundOnGInvxFunc} 

The results proved next in Lemma \ref{lemma_CLT_and_AbelSummation} and 
Lemma \ref{lemma_lowerBoundsOnLambdaFuncParitySummFuncs} 
are key to completely and carefully rigorously justifying the 
asymptotic bounds obtained below in Theorem \ref{theorem_gInv_GeneralAsymptoticsForms}. 
Thus these two somewhat more routine results are actually 
necessary to prove before we can return to the truly 
interesting matter of the unboundedness of $M(x) \log x / \sqrt{x}$ in the next subsection. 

\begin{lemma} 
\label{lemma_CLT_and_AbelSummation} 
Suppose that $f_k(n)$ is a sequence of arithmetic functions 
such that $f_k(n) > 0$ for all $n \geq 1$, $f_0(n) = \delta_{n,1}$, and 
$f_{\Omega(n)}(n) \SuccSim \widehat{\tau}_{\ell}(n)$ as $n \rightarrow \infty$ where 
$\widehat{\tau}_{\ell}(t)$ is a continuously differentiable function of $t$ for all 
large enough $t \gg 1$ \footnote{ 
     We will require that $\widehat{\tau}_{\ell}(t) \in C^{1}(\mathbb{R})$ when we apply the 
     Abel summation formula in the proof of Theorem \ref{theorem_gInv_GeneralAsymptoticsForms}. 
     At this point, it is technically an unnecessary condition that is 
     vacously satisfied by assumption (by requirement) 
     and will importantly need to hold only when we specialize to the 
     actual functions employed to form our new bounds in the theorem below. 
}.  
We define the $\lambda$-sign-scaled summatory function of $f$ as follows: 
\[
F_{\lambda}(x) := \sum_{\substack{n \leq x \\ \Omega(n) \leq x}} 
     \lambda(n) \cdot f_{\Omega(n)}(n). 
\]
Let 
\[
A_{\Omega}^{(\ell)}(t) := \sum_{k=1}^{\floor{\log\log t}} (-1)^k \widehat{\pi}_k^{(\ell)}(t),  
\]
where $\widehat{\pi}_k(x) \geq \widehat{\pi}_k^{(\ell)}(x) \geq 0$ and 
$\widehat{\pi}_k^{(\ell)}(x)$ is a smooth monotone non-decreasing function of $x$ for all $x$ 
sufficiently large. 
Then we have that there exists a constant $0 < A_{\ell} \leq 2$ such that 
\[
F_{\lambda}(\log\log x) \SuccSim A_{\ell}\left[ 
     A_{\Omega}^{(\ell)}(\log\log x) \widehat{\tau}_{\ell}(\log\log x) - 
     \int_1^{\log\log x} 
     A_{\Omega}^{(\ell)}(t) \widehat{\tau}_{\ell}^{\prime}(t) dt 
     \right]. 
\]
\end{lemma}
\begin{proof} 
We first note that we can form an accurate $C^{1}(\mathbb{R})$ approximation by the smoothness of 
$\widehat{\pi}_k^{(\ell)}(x)$ that allows us to apply the Abel summation formula using the summatory 
function $A_{\Omega}^{(\ell)}(t)$ for $t$ on any connected subinterval of $[1, \infty)$. 
The second stated formula for $F_{\lambda}(\log\log x)$ is valid by Abel summation provided that 
\[
0 \leq \left\lvert \frac{\displaystyle\sum\limits_{\log\log t < k \leq \frac{\log t}{\log 2}} 
     (-1)^k \widehat{\pi}_k(t)}{A_{\Omega}^{(\ell)}(t)}\right\rvert \leq 2, 
     \mathrm{\ as\ } t \rightarrow \infty, 
\]
e.g., the asymptotically dominant terms indicating the parity of 
$\lambda(n)$ are captured up to a constant factor 
by the terms in the range over $k$ summed by 
$A_{\Omega}^{(\ell)}(t)$ for 
sufficiently large $t$ as $t \rightarrow \infty$. 
Using the arguments in Montgomery and Vaughan \cite[\S 7; Thm.\ 7.20]{MV} (see 
Corollary \ref{theorem_MV_Thm7.20}), we can see that 
the assertion above holds in the limit as $t \rightarrow \infty$. 

Thus we again emphasize and conclude that we have captured the 
asymptotically dominant main order terms in our formula as 
$x \rightarrow \infty$ using the definition of $A_{\Omega}(x)$. 
In other words, taking the sum over the summands that defines $A_{\Omega}(x)$ only over the truncated range of 
$k \in [1, \log\log x]$ does not affect the limiting asymptotically 
dominant terms obtained from using this formulation of the summatory function with the 
Abel summation formula -- even when we should technically 
index over all $k \in [1, \log_2(x)]$ to obtain a precise formula for this function. 
\end{proof} 

Observe that we use the superscript and subscript of $(\ell)$ not to denote a formal parameter to 
the functions we define below, but instead to denote that these functions form \emph{lower bound} 
approximations to other forms of the functions without the scripted $(\ell)$. 

\begin{lemma} 
\label{lemma_lowerBoundsOnLambdaFuncParitySummFuncs} 
Suppose that $\widehat{\pi}_k(x) \geq \widehat{\pi}_k^{(\ell)}(x) \geq 0$ 
with $\widehat{\pi}_k^{(\ell)}(x)$ a monotone non-decreasing real-valued function 
for all sufficiently large $x$. 
Let 
\begin{align*} 
A_{\Omega}^{(\ell)}(x) & := \sum_{k \leq \log\log x} (-1)^k \widehat{\pi}_k^{(\ell)}(x) \\ 
A_{\Omega}(x) & := \sum_{k \leq \log\log x} (-1)^k \widehat{\pi}_k(x). 
\end{align*} 
Then for all sufficiently large $x$, we have that 
$$|A_{\Omega}(x)| \gg |A_{\Omega}^{(\ell)}(x)|.$$ 
\end{lemma} 
\begin{proof} 
Given an explicit smooth lower bounding function, $\widehat{\pi}_k^{(\ell)}(x)$, we define the 
similarly smooth and monotone residual terms in approximating $\widehat{\pi}_k(x)$ 
using the following notation: 
\[
\widehat{\pi}_k(x) = \widehat{\pi}_k^{(\ell)}(x) + \widehat{E}_k(x). 
\]
Then we can form the ordinary form (i.e., the exact, non-lower-bound) on the summatory functions as 
\begin{align*} 
|A_{\Omega}(x)| & = \left\lvert \sum_{k \leq \frac{\log\log x}{2}} 
     \left[\widehat{\pi}_{2k}(x) - \widehat{\pi}_{2k-1}(x)\right] \right\rvert \\ 
     & \geq \left\lvert A_{\Omega}^{(\ell)}(x) - \sum_{k \leq \frac{\log\log x}{2}} \widehat{E}_{2k-1}(x) 
     \right\rvert \\ 
     & \geq 
     \left\lvert A_{\Omega}^{(\ell)}(x) \right\rvert - 
     \left\lvert \sum_{k \leq \frac{\log\log x}{2}} \widehat{E}_{2k-1}(x) 
     \right\rvert. 
\end{align*} 
If the latter sum, $$\operatorname{ES}(x) := \sum_{k \leq \frac{\log\log x}{2}} \widehat{E}_{2k-1}(x) \rightarrow \infty,$$ as 
$x \rightarrow \infty$, then we can always find some absolute (by monotonicity) $C_0 > 0$ such that 
$\operatorname{ES}(x) \leq C_0 \cdot A_{\Omega}(x)$. If on the other hand this sum becomes constant as 
$x \rightarrow +\infty$, then we also clearly have another absolute $C_1 > 0$ such that 
$|A_{\Omega}(x)| \geq C_1 \cdot |A_{\Omega}^{(\ell)}(x)|$. 
In either case, the claimed result holds for all large enough $x$. 
\end{proof} 

\subsubsection{A proof of the key bound from below on $G^{-1}(x)$} 

We use the result of 
Corollary \ref{cor_BoundsOnGz_FromMVBook_initial_stmt_v1} 
to prove the following central theorem required to prove 
Corollary \ref{proofOf_cor_ThePipeDreamResult_v1} in Section \ref{subSection_TheCoreResultProof}: 

\begin{theorem}[Asymptotics and bounds for the summatory functions $G^{-1}(x)$] 
\label{theorem_gInv_GeneralAsymptoticsForms}
We define a lower summatory function, $G_{\ell}^{-1}(x)$, 
to provide bounds on the magnitude of $G^{-1}(x)$ such that 
$$|G_{\ell}^{-1}(x)| \ll |G^{-1}(x)|,$$ for all sufficiently large $x \geq 1$. 
We have the next asymptotic approximations for the lower summatory function where 
$C_{\ell,1}$ is the absolute constant defined by 
\[
C_{\ell,1} = 4B_{\ell} \cdot A_0^2  = 
     \frac{B_{\ell} \cdot 128 \cdot 2^{1/8}}{6561 \cdot e^6 \pi \log^6(2)} 
     \exp\left(-\frac{55}{2} \log^2(2)\right) 
     \approx B_{\ell} \cdot 2.76631 \times 10^{-10}, 
\]
for some bounded constant $B_{\ell} > 0$. 
That is, we have on average \footnote{ 
     E.g., within a predictably small interval around each $x$ sufficiently large. 
     This distinction in the statement is necessary since our limiting lower bounds have 
     so far depended on average order estimates of certain sums and arithmetic functions 
     that depend on $\omega(n)$ and $\Omega(n)$, each of which have CLT-like distribution to 
     how closely they are approximated on average when $n$ is large as $n \rightarrow \infty$. 
}
\begin{align*} 
 & \left\lvert G_{\ell}^{-1}\left(x\right) \right\rvert
     \SuccSim 
     C_{\ell,1} \cdot (\log\log x) \frac{(\log\log\log x)^{2\log 2+\frac{1}{3\log 2} - 2}}{ 
     (\log\log\log\log x)^{\frac{5}{2}}} \cdot \frac{\log_{\ast}^5(x)^{2\log 2 + \frac{1}{3\log 2}}}{ 
     \log_{\ast}^{6}(x)^{\frac{5}{2}}}. 
\end{align*} 
The exponent in the previous equation is numerically approximated as 
$2\log 2 + \frac{1}{3\log 2} -2 \approx -0.402633 < 0$. 
\end{theorem} 
\begin{proof}[Initial Sketch: Logarithmic scaling of parameters to the accurate order] 
For the sums given by 
\begin{align*} 
S_{g^{-1}}(x) := \sum_{n \leq x} \lambda(n) \cdot C_{\Omega(n)}(n), 
\end{align*} 
we notice that using the asymptotic bounds (rather than the exact formulas) for the functions 
$C_{\Omega(n)}(n)$ in Theorem \ref{theorem_Ckn_GeneralAsymptoticsForms}, 
we have over-summed on $n$ by quite a bit. 
In particular, following from the intent behind the constructions in the last sections, 
we are really summing only over all $n \leq x$ with $\Omega(n) \leq x$. 
Since $\Omega(n) \leq \floor{\log_2 n}$ maximally, 
many of the terms in the previous equation are actually zero (recall that $C_0(n) = \delta_{n,1}$). 

So we are actually only summing on $n$ up to the average order of 
$\mathbb{E}[\Omega(n)] = \log\log n$ in practice. 
Hence, the sum that we are really interested in bounding is 
bounded below in magnitude by $S_{g^{-1}}(\log\log x)$ as bounded in 
Corollary \ref{theorem_Ckn_GeneralAsymptoticsForms}. After noting this adjustment, 
we can then safely apply the 
asymptotic formulas for the functions $C_k(n)$ from 
Theorem \ref{theorem_Ckn_GeneralAsymptoticsForms} 
that hold once we have verified these important constraints on $k,n,x$. 
\end{proof} 
\NBRef{A10-2020.04-26} 
\begin{proof} 
Recall from our proof of Corollary \ref{cor_BoundsOnGz_FromMVBook_initial_stmt_v1} that 
a lower bound on the variant prime form counting function is given by 
\[
\widehat{\pi}_k(x) \SuccSim \frac{A_0 \cdot x}{\log x \cdot (\log\log x)^4 \cdot (k-1)!} \cdot 
     \left(\frac{4}{3e\log 2}\right)^{k}. 
\]
So we can then form a lower summatory function indicating the parity of all 
$\Omega(n)$ for $n \leq x$ as 
\begin{align} 
\label{proof_thm_GInvFunc_v0} 
\left\lvert A_{\Omega}^{(\ell)}(t) \right\rvert & = 
     \left\lvert \sum_{k \leq \log\log t} (-1)^k \widehat{\pi}_k(x) \right\rvert \\ 
\notag
     & \SuccSim  
     \sqrt{\frac{2}{\pi}} A_0 \cdot (\log\log x) 
     \frac{(\log\log\log x)^{2\log 2+ \frac{1}{3 \log 2} - 1}}{ 
     (\log\log\log\log x)^{\frac{5}{2} + \log\log\log\log x}}, 
\end{align} 
where the actual sign on this function is given by 
$\operatorname{sgn}(A_{\Omega}^{(\ell)}(t)) = (-1)^{\floor{\log\log\log\log x}}$ 
(see Lemma \ref{lemma_lowerBoundsOnLambdaFuncParitySummFuncs}). 

Next, by Corollary \ref{theorem_Ckn_GeneralAsymptoticsForms} 
we recover from the main term approximation to $C_k(n)$, denoted here by 
$\widehat{\tau}_0(t)$, that 
\begin{align*} 
\widehat{\tau}_0^{\prime}(t) & = \frac{d}{dt}\left[ 
     (\log\log t)^{2\log\log t-1} 
     \right] \SuccSim 
     \frac{2 (\log\log t)^{2\log\log t-1} (\log\log\log t)}{t \cdot \log t}. 
\end{align*} 
As in Lemma \ref{lemma_CLT_and_AbelSummation} and Corollary \ref{cor_ASemiForm_ForGInvx_v1}, 
we apply Abel summation to obtain that for some bounded constant $B_{\ell} > 0$ we have 
\begin{equation} 
\label{proof_thm_GInvFunc_v1} 
G_{\ell}^{-1}(x) = B_{\ell} \cdot \widehat{L}_0(\log\log x) \left[
     \widehat{\tau}_0(\log\log x) A_{\Omega}^{(\ell)}(\log\log x) - 
     \widehat{\tau}_0(u_0) A_{\Omega}^{(\ell)}(u_0) - \int_{u_0}^{\log\log x} 
     \widehat{\tau}_0^{\prime}(t) A_{\Omega}^{(\ell)}(t) dt\right]. 
\end{equation} 
The inner integral term on the rightmost side of \eqref{proof_thm_GInvFunc_v1} 
is summed approximately in the form of 
\begin{align} 
\label{eqn_proof_thm_GInvFunc_v3_approx} 
\int_{u_0}^{\log\log x} \widehat{\tau}_0^{\prime}(t) A_{\Omega}^{(\ell)}(t) dt & \sim 
     \sum_{k=u_0+1}^{\frac{1}{2}\log\log\log\log x} \left( 
     I_{\ell}\left(e^{e^{2k+1}}\right) - 
     I_{\ell}\left(e^{e^{2k}}\right) 
     \right) e^{e^{2k}} \\ 
\notag 
     & \approx 
     C_0(u_0) + 
     (-1)^{\Floor{\log\log\log\log x}{2}} \times 
     \int_{\frac{\log\log\log\log x}{2}-1}^{\frac{\log\log\log\log x}{2}} 
     I_{\ell}\left(e^{e^{2k}}\right) 
     e^{e^{2k}} dk. 
\end{align} 
We define the integrand function, 
$I_{\ell}(t) := \widehat{\tau}_0^{\prime}(t) A_{\Omega}^{(\ell)}(t)$, 
from the previous equations with some limiting simplifications for the 
$k \in \left[\frac{\log\log\log\log x}{2}-1, \frac{\log\log\log\log x}{2}\right]$ as 
\begin{align} 
\label{eqn_proof_thm_GInvFunc_v3_approx} 
I_{\ell}\left(e^{e^{2k}}\right) e^{e^{2k}}& \SuccSim 
     \frac{2^{3/2} A_0}{\sqrt{\pi}} \left(\frac{2k}{\sqrt{e}}\right)^{4k} 
     \frac{\log(2k)^{2\log 2 + \frac{1}{3\log 2}}}{\log\log(2k)^{\frac{5}{2} + \log\log(2k)}}. 
\end{align} 
So using the lower bound on the integrand in \eqref{eqn_proof_thm_GInvFunc_v3_approx}, 
we find that \footnote{ 
     We have invoked the simplification that for sufficiently large $x$, 
     \[
     (\log\log\log\log\log x)^2 \PrecSim \exp\left(-(\log\log\log\log\log\log x)^2\right). 
     \]
     %We also simplify the term involved in the resulting bound by writing 
     %\[
     %\left( 
     %     \frac{\log\log x}{\log\log\log\log x} 
     %     \right)^{\log\log\log\log x} = (\log\log\log x)^{\frac{\log\log x}{\log\log\log\log x}} = 
     %     \exp\left(\log\log x\right) = \log x. 
     %\]
} 
\begin{align} 
\label{eqn_proof_thm_GInvFunc_v4_approx} 
B_{\ell} \cdot \widehat{L}_0(\log\log x) & \times \int_{\frac{\log\log\log\log x}{2}-1}^{\frac{\log\log\log\log x}{2}} 
     I_{\ell}\left(e^{e^{2k}}\right) 
     e^{e^{2k}} dk \\ 
\notag 
     & \SuccSim 
     \frac{4 B_{\ell} A_0^2}{\pi} (\log\log x) \frac{(\log\log\log x)^{2\log 2+\frac{1}{3\log 2} - 2}}{ 
     (\log\log\log\log x)^{\frac{5}{2}}} \cdot \frac{\log_{\ast}^5(x)^{2\log 2 + \frac{1}{3\log 2}}}{ 
     \log_{\ast}^{6}(x)^{\frac{5}{2}}}. 
\end{align} 
It is clear from our prior computations of the growth of 
$A_{\Omega}^{(\ell)}(x)$ and $\widehat{\tau}_0(x)$ 
that the asymptotically dominant behavior of the lower bound for 
$|G_{\ell}^{-1}(x)|$ comes from the integral term calculated in the last equation of 
\eqref{eqn_proof_thm_GInvFunc_v4_approx}. 
\end{proof} 

\subsection{Lower bounds on the scaled Mertens function along an infinite subsequence}
\label{subSection_TheCoreResultProof} 

What we will have shown in total concluding the proof of 
Corollary \ref{cor_ThePipeDreamResult_v1} below is a 
logarithmically scaled lesser form of the classically conjectured 
unboundedness property of $M(x)$ in the form of 
\[
\limsup_{x \rightarrow \infty} \frac{|M(x)| \log x}{\sqrt{x}} = +\infty. 
\]
This statement still comprises a better than previously known rate of the minimal asymptotic tendencies of 
$|M(x)| / \sqrt{x}$ towards unboundedness along an infinite subsequence, 
e.g., progress on the classical problem. If refinements of the methods in this proof eventually lead to 
researchers beating down this necessary logarithmic factor in our proof to one, then the classically 
conjectured formulation on the unboundedness of $|M(x)| / \sqrt{x}$
would touch so closely in the direction of an actual predicted 300-year-premature 
affirmative on Riemann's Hypothesis, that we will eventually have to shut up and be happy on the subject. 
This is still a much weaker condition than the RH as stated, and moreover, 
we must emphasize that its construction is much differently 
motivated by the encouraging combinatorial structures we have observed. 

Now we finally address the conclusion of our argument: 

\begin{proof}[Proof of Corollary \ref{cor_ThePipeDreamResult_v1}] 
\label{proofOf_cor_ThePipeDreamResult_v1} 
It suffices to take $u_0 = e^{e^{e^{e}}}$. 
Now, we break up the integral over $t \in [u_0, x/2]$ into two pieces: one that is easily bounded 
from $u_0 \leq t \leq \sqrt{x}$, 
and then another that will conveniently give us our slow-growing tendency towards 
infinity along the subsequence. 
In the next calculations, we assume that $x \mapsto x_y$ is taken along the 
subsequence defined inexplicitly over intervals as stated above.

First, since $\pi(j) = \pi(\sqrt{x})$ for all $\sqrt{x} \leq j < x$, we can take the first chunk 
of the interval of integration and bound it using \eqref{eqn_pf_tag_v2-restated_v2} as 
\begin{align*} 
-\int_{u_0}^{\sqrt{x}} \frac{2\sqrt{x}}{t^2 \log(x)} G_{\ell}^{-1}(t) dt & \SuccSim 
     B_{\ell,2} \times \frac{2}{\log(x)} \cdot \left(\min\limits_{u_0 \leq t \leq \sqrt{x}} 
     G_{\ell}^{-1}(t)\right) = o\left(1\right), 
\end{align*} 
where $B_{\ell,2}$ can be taken as an indefinite, but still some absolute constant with respect to $u_0$. 
The maximum in the previous equation is clearly attained by taking $t := \sqrt{x}$. 

We next have to prove a related bound over the second portion of the interval from 
$\sqrt{x} \leq t \leq x/2$: 
\begin{align*} 
- & \int_{\sqrt{x}}^{x/2} \frac{2 x}{t^2 \log(x)} \cdot G_{\ell}^{-1}(t) dt \SuccSim 
     \frac{2\sqrt{x}}{\log x} \cdot \left(\min_{\sqrt{x} \leq t \leq x/2} G_{\ell}^{-1}(t)\right) \\ 
     & = 2C_{\ell,1} \cdot \sqrt{x} \cdot 
     \frac{(\log\log \sqrt{x})}{\log x} \frac{(\log\log\log \sqrt{x})^{2\log 2+\frac{1}{3\log 2} - 2}}{ 
     (\log\log\log\log \sqrt{x})^{\frac{5}{2}}} \cdot \frac{\log_{\ast}^5(\sqrt{x})^{2\log 2 + \frac{1}{3\log 2}}}{ 
     \log_{\ast}^{6}(\sqrt{x})^{\frac{5}{2}}} + o(1). 
\end{align*} 
Finally, since $G_{\ell}^{-1}(x) = o(\sqrt{x})$, we obtain in total that as 
$x \rightarrow \infty$ along this infinite subsequence: 
\begin{align*} 
|M(x)| & \SuccSim 
     2C_{\ell,1} \cdot \sqrt{x} \cdot 
     \frac{(\log\log \sqrt{x})}{\log x} \frac{(\log\log\log \sqrt{x})^{2\log 2+\frac{1}{3\log 2} - 2}}{ 
     (\log\log\log\log \sqrt{x})^{\frac{5}{2}}} \cdot \frac{\log_{\ast}^5(\sqrt{x})^{2\log 2 + \frac{1}{3\log 2}}}{ 
     \log_{\ast}^{6}(\sqrt{x})^{\frac{5}{2}}}. 
     \qedhere 
\end{align*} 
\end{proof} 

\newpage 
\section{Conclusions} 

\subsection{Summary} 

\begin{itemize} 

\item Using average order bounds, summatory functions, and the $\PrecSim$-type relations for lower bounds. 
\item Somewhat oddly, we did not need substantially improved bounds on $L_0(x) := \sum_{n \leq x} \lambda(n)$ 
      than what is already known in upper bound form to obtain 
      our new bounds on the Mertens function, aka, summatory function of the ``testier'' M\"obius function. 

\end{itemize} 

\subsection{Future research and work that still needs to be done} 

\begin{itemize} 

\item Refinements of these bounds to find the tightest possible lower (limit supremum) bounds, e.g., proofs of an 
      optimal version of Gonek's original conjecture. 
\item Generalizations to weighted Mertens functions of the form $M_{\alpha}(x) := \sum_{n \leq x} \mu(n) n^{-\alpha}$. 
\item Indications of sign changes and exceptionally small, or zero values of $M(x)$. 
\item What our more combinatorial approach to bounding $M(x)$ effectively suggests about necessary, but unproved, 
      zeta zero bounds that have historically formed the basis for arguments bounding $M(x)$ using Mellin inversion. 
\item Evaluate alternate strategies and approaches using different Dirichlet convolution functions 
      besides $g$ and $g^{-1}(n)$ (corresponding to $\pi(x)$) 
      with Theorem \ref{theorem_SummatoryFuncsOfDirCvls}. 

\end{itemize} 

\subsection{Motivating a general technique towards bounding the summatory functions of arbitrary arithmetic $f$} 

\subsection{The general construction using Theorem \ref{theorem_SummatoryFuncsOfDirCvls}} 

\subsubsection{A proposed generalization} 

For each $n \geq 1$, let $A(n) \subseteq \{d: 1 \leq d \leq n, d|n\}$ be a subset of the 
divisors of $n$. We say that a natural number $n \geq 1$ is \emph{$A$-primitive} if $A(n) = \{1, n\}$. 
Under a list of assumptions so that the resulting $A$-convolutions are \emph{regular convolutions}, 
we get a generalized multiplicative M\"obius function \cite[\S 2.2]{HANDBOOKNT-2004}: 
\[
\mu_A(p^{\alpha}) = \begin{cases} 
     1, & \alpha = 0; \\ 
     -1, & p^{\alpha} > 1\text{\ is $A$-primitive; } \\ 
     0, & \text{otherwise.}
     \end{cases} 
\]
We also define the functions $\omega_A(n) := \#\{d | n: \mathrm{d\ is\ an\ A-primitive\ factor\ of\ n}\}$ and 
$\Omega_A(n) := \#\{p^{\alpha} || n: \mathrm{p\ is\ an\ A-primitive\ factor\ of\ n}\}$. Then the characteristic 
function of the set $A := \cup_{n \geq 1} A(n)$ is given by $\chi_A(n) = \Iverson{n \in A}$. By M\"obius inversion, 
we have that $\chi_A = \omega_A \ast_A \mu_A$. Moreover, for the $A$-counting function, $\pi_A(x)$, defined by 
\[
\pi_A(x) := \#\{n \leq x: n \in A\}, 
\]
we can define a corresponding notion of a generalized $A$-Mertens function, $M_A(x) := \sum_{n \leq x} \mu_A(n)$. 
This function then satisfies (by Theorem \ref{theorem_SummatoryFuncsOfDirCvls}) the relation that 
\[
M_A(x) = \sum_{k=1}^{x} (\omega_A + 1)^{-1}(k) \cdot \pi_A(x/k), 
\]
where the inverse function, $(\omega_A + 1)^{-1}(n)$, is defined with respect to $A$-convolution. 
We conjecture, but do not prove here, that 
$\operatorname{sgn}((\omega_A + 1)^{-1}(n)) = \lambda_A(n) =: (-1)^{\Omega_A(n)}$. 

Using formulas similar in construction to \eqref{eqn_pf_tag_v2-restated_v2}, 
we can differentiate to find expressions for $\pi_A(x)$. The significance of this is that provided we can 
prove sufficiently large bounds for $M_A(x)$ along the same lines as we have done for $M(x)$, the 
resulting formula may be able to speak towards the density, or even infinitude in special cases, 
of the set $A$.

\subsection{Working / TODO} 

\begin{itemize} 
           \item[(i)] The average order, $\mathbb{E}[\omega(n)] = \log\log n$, imparts an iterated logarithmic structure 
           to our expansions, which many have conjectured we should see in limiting bounds on $M(x)$, 
           but which are practically elusive in most non-conjectural known formulas I have seen 
           proved rigorously in print. 
           \item[(ii)] The additivity of $\omega(n)$ dictates that the sign of $g^{-1}(n) = (\omega+1)^{-1}(n)$ 
           is $\operatorname{sgn}(g^{-1}(n)) = \lambda(n)$ 
           (see Proposition \ref{prop_SignageDirInvsOfPosBddArithmeticFuncs_v1}). 
           The corresponding weighted summatory functions of 
           $\lambda(n)$ have more established predictable properties, such as known sign biases and upper bounds. 
           These summatory functions are generally speaking more regular and easier to work with than 
           traditional approaches to summing $M(x)$ 
           and its complicating summand terms of the M\"obius function. 
           Note that our proof is essentially much different than what is known about sums of consecutive values of 
           $\mu(n)$ over short intervals, both in interpretation and methodology. 
           \item The method easily generalizes to the $\mu^2(n)$ / $Q(x)$ bounds suggested by the form of the 
           function $\mathcal{G}(z)$ suggested in the exercises section of \cite{MV}. 
           \item Acks: Thanks to Lacey for letting me win the new laptop bet by working on this problem ... 
           \item Need to lookup the proof of the result: RH $\iff$ $M(x) = O(x^{1/2+\epsilon})$ $\forall 0 < \epsilon < 1/2$. 
           This should lend some resolution to how close this bound actually gets us in the direction of the RH. 
           In particular, it ought at least offer semi-significant improvements on what is known about the 
           distribution of non-trivial simple zeros $s = \sigma + \imath t$ with $|\Im(s)| \leq T$ 
           for all $T \rightarrow \infty$ large enough. 
           Perhaps the truth of the limsup bound proved in the key corollary above might also mollify some things as to 
           how deep of a zero free region we can state into the critical strip, for example, a classically limited 
           pentest to start with. 
\end{itemize}


More generally, we have that for $f$ a non-negative additive arithmetic function that vanishes at one, 
$\operatorname{sgn}((f+1)^{-1}) = \lambda(n) = (-1)^{\Omega(n)}$. 
We can state similar properties for the common case of multiplicative $f$ in the 
form of the following result: If $f(n) > 0$ for all $n \geq 1$ and $f$ is multiplicative, then 
$\operatorname{sgn}(f^{-1}(n)) = (-1)^{\omega(n)}$. 


The following observation that is suggestive of the semi-periodicity at play 
with the distinct values of $g^{-1}(n)$ distributed over $n \geq 2$. 

\begin{heuristic}[Symmetry in $g^{-1}(n)$ in the exponents in the prime factorization of $n$] 
Suppose that $n_1, n_2 \geq 2$ are such that their factorizations into distinct primes are 
given by $n_1 = p_1^{\alpha_1} \cdots p_r^{\alpha_r}$ and $n_2 = q_1^{\beta_1} \cdots q_r^{\beta_r}$. 
If $\{\alpha_1, \ldots, \alpha_r\} \equiv \{\beta_1, \ldots, \beta_r\}$ as multisets of prime exponents, 
then $g^{-1}(n_1) = g^{-1}(n_2)$. For example, $g^{-1}$ has the same values on the squarefree integers 
with exactly two, three, and so on prime factors. There does not appear to be an easy, nor subtle 
direct recursion between the distinct $g^{-1}$ values, except through auxiliary function sequences. 
We will settle for an asymptotically accurate main term approximation to $g^{-1}(n)$ for large $n$ as 
$n \rightarrow \infty$ in the average case. 
\end{heuristic} 

\newpage 
\renewcommand{\refname}{References} 
\bibliography{glossaries-bibtex/thesis-references}{}
\bibliographystyle{plain}

\newpage
\setcounter{section}{0} 
\renewcommand{\thesection}{T.\arabic{section}} 

\newpage
\section{Table: Computations with a signed Dirichlet inverse function and its summatory function} 
\label{table_conjecture_Mertens_ginvSeq_approx_values}

\begin{table}[h!]

\centering

\tiny
\begin{equation*}
\boxed{
\begin{array}{|cc|c|ccc|c|c|ccc|c|ccc}
 n & \mathbf{Primes} & & \mathbf{Sqfree} & \mathbf{PPower} & \bar{\mathbb{S}} & & g^{-1}(n) & 
 \lambda(n) \operatorname{sgn}(g^{-1}(n)) & \lambda(n) g^{-1}(n) - \widehat{f}_1(n) & 
 \frac{\sum\limits_{d|n} C_{\Omega(d)}(d)}{|g^{-1}(n)|} & & G^{-1}(n) & G^{-1}_{+}(n) & G^{-1}_{-}(n) \\ \hline 
 1 & 1^1 & \text{--} & \text{Y} & \text{N} & \text{N} & \text{--} & 1 & 1 & 0 & 1.0000000 & \text{--} & 1 & 1 & 0 \\
 2 & 2^1 & \text{--} & \text{Y} & \text{Y} & \text{N} & \text{--} & -2 & 1 & 0 & 1.0000000 & \text{--} & -1 & 1 & -2 \\
 3 & 3^1 & \text{--} & \text{Y} & \text{Y} & \text{N} & \text{--} & -2 & 1 & 0 & 1.0000000 & \text{--} & -3 & 1 & -4 \\
 4 & 2^2 & \text{--} & \text{N} & \text{Y} & \text{N} & \text{--} & 2 & 1 & 0 & 1.5000000 & \text{--} & -1 & 3 & -4 \\
 5 & 5^1 & \text{--} & \text{Y} & \text{Y} & \text{N} & \text{--} & -2 & 1 & 0 & 1.0000000 & \text{--} & -3 & 3 & -6 \\
 6 & 2^1 3^1 & \text{--} & \text{Y} & \text{N} & \text{N} & \text{--} & 5 & 1 & 0 & 1.0000000 & \text{--} & 2 & 8 & -6 \\
 7 & 7^1 & \text{--} & \text{Y} & \text{Y} & \text{N} & \text{--} & -2 & 1 & 0 & 1.0000000 & \text{--} & 0 & 8 & -8 \\
 8 & 2^3 & \text{--} & \text{N} & \text{Y} & \text{N} & \text{--} & -2 & 1 & 0 & 2.0000000 & \text{--} & -2 & 8 & -10 \\
 9 & 3^2 & \text{--} & \text{N} & \text{Y} & \text{N} & \text{--} & 2 & 1 & 0 & 1.5000000 & \text{--} & 0 & 10 & -10 \\
 10 & 2^1 5^1 & \text{--} & \text{Y} & \text{N} & \text{N} & \text{--} & 5 & 1 & 0 & 1.0000000 & \text{--} & 5 & 15 & -10 \\
 11 & 11^1 & \text{--} & \text{Y} & \text{Y} & \text{N} & \text{--} & -2 & 1 & 0 & 1.0000000 & \text{--} & 3 & 15 & -12 \\
 12 & 2^2 3^1 & \text{--} & \text{N} & \text{N} & \text{Y} & \text{--} & -7 & 1 & 2 & 1.2857143 & \text{--} & -4 & 15 & -19 \\
 13 & 13^1 & \text{--} & \text{Y} & \text{Y} & \text{N} & \text{--} & -2 & 1 & 0 & 1.0000000 & \text{--} & -6 & 15 & -21 \\
 14 & 2^1 7^1 & \text{--} & \text{Y} & \text{N} & \text{N} & \text{--} & 5 & 1 & 0 & 1.0000000 & \text{--} & -1 & 20 & -21 \\
 15 & 3^1 5^1 & \text{--} & \text{Y} & \text{N} & \text{N} & \text{--} & 5 & 1 & 0 & 1.0000000 & \text{--} & 4 & 25 & -21 \\
 16 & 2^4 & \text{--} & \text{N} & \text{Y} & \text{N} & \text{--} & 2 & 1 & 0 & 2.5000000 & \text{--} & 6 & 27 & -21 \\
 17 & 17^1 & \text{--} & \text{Y} & \text{Y} & \text{N} & \text{--} & -2 & 1 & 0 & 1.0000000 & \text{--} & 4 & 27 & -23 \\
 18 & 2^1 3^2 & \text{--} & \text{N} & \text{N} & \text{Y} & \text{--} & -7 & 1 & 2 & 1.2857143 & \text{--} & -3 & 27 & -30 \\
 19 & 19^1 & \text{--} & \text{Y} & \text{Y} & \text{N} & \text{--} & -2 & 1 & 0 & 1.0000000 & \text{--} & -5 & 27 & -32 \\
 20 & 2^2 5^1 & \text{--} & \text{N} & \text{N} & \text{Y} & \text{--} & -7 & 1 & 2 & 1.2857143 & \text{--} & -12 & 27 & -39 \\
 21 & 3^1 7^1 & \text{--} & \text{Y} & \text{N} & \text{N} & \text{--} & 5 & 1 & 0 & 1.0000000 & \text{--} & -7 & 32 & -39 \\
 22 & 2^1 11^1 & \text{--} & \text{Y} & \text{N} & \text{N} & \text{--} & 5 & 1 & 0 & 1.0000000 & \text{--} & -2 & 37 & -39 \\
 23 & 23^1 & \text{--} & \text{Y} & \text{Y} & \text{N} & \text{--} & -2 & 1 & 0 & 1.0000000 & \text{--} & -4 & 37 & -41 \\
 24 & 2^3 3^1 & \text{--} & \text{N} & \text{N} & \text{Y} & \text{--} & 9 & 1 & 4 & 1.5555556 & \text{--} & 5 & 46 & -41 \\
 25 & 5^2 & \text{--} & \text{N} & \text{Y} & \text{N} & \text{--} & 2 & 1 & 0 & 1.5000000 & \text{--} & 7 & 48 & -41 \\
 26 & 2^1 13^1 & \text{--} & \text{Y} & \text{N} & \text{N} & \text{--} & 5 & 1 & 0 & 1.0000000 & \text{--} & 12 & 53 & -41 \\
 27 & 3^3 & \text{--} & \text{N} & \text{Y} & \text{N} & \text{--} & -2 & 1 & 0 & 2.0000000 & \text{--} & 10 & 53 & -43 \\
 28 & 2^2 7^1 & \text{--} & \text{N} & \text{N} & \text{Y} & \text{--} & -7 & 1 & 2 & 1.2857143 & \text{--} & 3 & 53 & -50 \\
 29 & 29^1 & \text{--} & \text{Y} & \text{Y} & \text{N} & \text{--} & -2 & 1 & 0 & 1.0000000 & \text{--} & 1 & 53 & -52 \\
 30 & 2^1 3^1 5^1 & \text{--} & \text{Y} & \text{N} & \text{N} & \text{--} & -16 & 1 & 0 & 1.0000000 & \text{--} & -15 & 53 & -68 \\
 31 & 31^1 & \text{--} & \text{Y} & \text{Y} & \text{N} & \text{--} & -2 & 1 & 0 & 1.0000000 & \text{--} & -17 & 53 & -70 \\
 32 & 2^5 & \text{--} & \text{N} & \text{Y} & \text{N} & \text{--} & -2 & 1 & 0 & 3.0000000 & \text{--} & -19 & 53 & -72 \\
 33 & 3^1 11^1 & \text{--} & \text{Y} & \text{N} & \text{N} & \text{--} & 5 & 1 & 0 & 1.0000000 & \text{--} & -14 & 58 & -72 \\
 34 & 2^1 17^1 & \text{--} & \text{Y} & \text{N} & \text{N} & \text{--} & 5 & 1 & 0 & 1.0000000 & \text{--} & -9 & 63 & -72 \\
 35 & 5^1 7^1 & \text{--} & \text{Y} & \text{N} & \text{N} & \text{--} & 5 & 1 & 0 & 1.0000000 & \text{--} & -4 & 68 & -72 \\
 36 & 2^2 3^2 & \text{--} & \text{N} & \text{N} & \text{Y} & \text{--} & 14 & 1 & 9 & 1.3571429 & \text{--} & 10 & 82 & -72 \\
 37 & 37^1 & \text{--} & \text{Y} & \text{Y} & \text{N} & \text{--} & -2 & 1 & 0 & 1.0000000 & \text{--} & 8 & 82 & -74 \\
 38 & 2^1 19^1 & \text{--} & \text{Y} & \text{N} & \text{N} & \text{--} & 5 & 1 & 0 & 1.0000000 & \text{--} & 13 & 87 & -74 \\
 39 & 3^1 13^1 & \text{--} & \text{Y} & \text{N} & \text{N} & \text{--} & 5 & 1 & 0 & 1.0000000 & \text{--} & 18 & 92 & -74 \\
 40 & 2^3 5^1 & \text{--} & \text{N} & \text{N} & \text{Y} & \text{--} & 9 & 1 & 4 & 1.5555556 & \text{--} & 27 & 101 & -74 \\
 41 & 41^1 & \text{--} & \text{Y} & \text{Y} & \text{N} & \text{--} & -2 & 1 & 0 & 1.0000000 & \text{--} & 25 & 101 & -76 \\
 42 & 2^1 3^1 7^1 & \text{--} & \text{Y} & \text{N} & \text{N} & \text{--} & -16 & 1 & 0 & 1.0000000 & \text{--} & 9 & 101 & -92 \\
 43 & 43^1 & \text{--} & \text{Y} & \text{Y} & \text{N} & \text{--} & -2 & 1 & 0 & 1.0000000 & \text{--} & 7 & 101 & -94 \\
 44 & 2^2 11^1 & \text{--} & \text{N} & \text{N} & \text{Y} & \text{--} & -7 & 1 & 2 & 1.2857143 & \text{--} & 0 & 101 & -101 \\
 45 & 3^2 5^1 & \text{--} & \text{N} & \text{N} & \text{Y} & \text{--} & -7 & 1 & 2 & 1.2857143 & \text{--} & -7 & 101 & -108 \\
 46 & 2^1 23^1 & \text{--} & \text{Y} & \text{N} & \text{N} & \text{--} & 5 & 1 & 0 & 1.0000000 & \text{--} & -2 & 106 & -108 \\
 47 & 47^1 & \text{--} & \text{Y} & \text{Y} & \text{N} & \text{--} & -2 & 1 & 0 & 1.0000000 & \text{--} & -4 & 106 & -110 \\
 48 & 2^4 3^1 & \text{--} & \text{N} & \text{N} & \text{Y} & \text{--} & -11 & 1 & 6 & 1.8181818 & \text{--} & -15 & 106 & -121 \\
\end{array}
}
\end{equation*}

\bigskip\hrule\smallskip 

\caption*{\textbf{\rm \bf Table \thesection:} 
          \textbf{Computations of $\mathbf{g^{-1}(n) \equiv (\omega+1)^{-1}(n)}$ 
          for small $\mathbf{1 \leq n \leq 350}$.} \\ 
          The column labeled \texttt{Primes} provides the prime factorization of each $n$ so that the values of 
          $\omega(n)$ and $\Omega(n)$ are easily extracted. The columns labeled, respectively, \texttt{Sqfree}, \texttt{PPower} and 
          $\bar{\mathbb{S}}$ list inclusion of $n$ in the sets of squarefree integers, prime powers, and the set $\bar{\mathbb{S}}$ 
          that denotes the positive integers $n$ which are neither squarefree nor prime powers. The next two columns provide the 
          explicit values of the inverse function $g^{-1}(n)$ and indicate that the sign of this function at $n$ is given by 
          $\lambda(n)$. \\[0.05cm] 
          The next column shows the small-ish magnitude differences between the unsigned 
          magnitude of $g^{-1}(n)$ and the summations $\widehat{f}_1(n) := \sum_{k \geq 0} \binom{\omega(n)}{k} \cdot k!$. 
          The following column in order shows the ratio of $\sum_{d|n} C_{\Omega(d)}(d) / |g^{-1}(n)|$. \\[0.05cm] 
          The last three 
          columns show the summatory function of $g^{-1}(n)$, $G^{-1}(x) := \sum_{n \leq x} g^{-1}(n)$, decomposed into its 
          respective positive and negative summatory function components: $G^{-1}(x) = G^{-1}_{+}(x) + G^{-1}_{-}(x)$ where 
          $G^{-1}_{+}(x) > 0$ and $G^{-1}_{-}(x) < 0$. 
          } 

\end{table}

\newpage
\begin{table}[h!]

\centering

\tiny
\begin{equation*}
\boxed{
\begin{array}{|cc|c|ccc|c|c|ccc|c|ccc}
 n & \mathbf{Primes} & & \mathbf{Sqfree} & \mathbf{PPower} & \bar{\mathbb{S}} & & g^{-1}(n) & 
 \lambda(n) \operatorname{sgn}(g^{-1}(n)) & \lambda(n) g^{-1}(n) - \widehat{f}_1(n) & 
 \frac{\sum\limits_{d|n} C_{\Omega(d)}(d)}{|g^{-1}(n)|} & & G^{-1}(n) & G^{-1}_{+}(n) & G^{-1}_{-}(n) \\ \hline 
 49 & 7^2 & \text{--} & \text{N} & \text{Y} & \text{N} & \text{--} & 2 & 1 & 0 & 1.5000000 & \text{--} & -13 & 108 & -121 \\
 50 & 2^1 5^2 & \text{--} & \text{N} & \text{N} & \text{Y} & \text{--} & -7 & 1 & 2 & 1.2857143 & \text{--} & -20 & 108 & -128 \\
 51 & 3^1 17^1 & \text{--} & \text{Y} & \text{N} & \text{N} & \text{--} & 5 & 1 & 0 & 1.0000000 & \text{--} & -15 & 113 & -128 \\
 52 & 2^2 13^1 & \text{--} & \text{N} & \text{N} & \text{Y} & \text{--} & -7 & 1 & 2 & 1.2857143 & \text{--} & -22 & 113 & -135 \\
 53 & 53^1 & \text{--} & \text{Y} & \text{Y} & \text{N} & \text{--} & -2 & 1 & 0 & 1.0000000 & \text{--} & -24 & 113 & -137 \\
 54 & 2^1 3^3 & \text{--} & \text{N} & \text{N} & \text{Y} & \text{--} & 9 & 1 & 4 & 1.5555556 & \text{--} & -15 & 122 & -137 \\
 55 & 5^1 11^1 & \text{--} & \text{Y} & \text{N} & \text{N} & \text{--} & 5 & 1 & 0 & 1.0000000 & \text{--} & -10 & 127 & -137 \\
 56 & 2^3 7^1 & \text{--} & \text{N} & \text{N} & \text{Y} & \text{--} & 9 & 1 & 4 & 1.5555556 & \text{--} & -1 & 136 & -137 \\
 57 & 3^1 19^1 & \text{--} & \text{Y} & \text{N} & \text{N} & \text{--} & 5 & 1 & 0 & 1.0000000 & \text{--} & 4 & 141 & -137 \\
 58 & 2^1 29^1 & \text{--} & \text{Y} & \text{N} & \text{N} & \text{--} & 5 & 1 & 0 & 1.0000000 & \text{--} & 9 & 146 & -137 \\
 59 & 59^1 & \text{--} & \text{Y} & \text{Y} & \text{N} & \text{--} & -2 & 1 & 0 & 1.0000000 & \text{--} & 7 & 146 & -139 \\
 60 & 2^2 3^1 5^1 & \text{--} & \text{N} & \text{N} & \text{Y} & \text{--} & 30 & 1 & 14 & 1.1666667 & \text{--} & 37 & 176 & -139 \\
 61 & 61^1 & \text{--} & \text{Y} & \text{Y} & \text{N} & \text{--} & -2 & 1 & 0 & 1.0000000 & \text{--} & 35 & 176 & -141 \\
 62 & 2^1 31^1 & \text{--} & \text{Y} & \text{N} & \text{N} & \text{--} & 5 & 1 & 0 & 1.0000000 & \text{--} & 40 & 181 & -141 \\
 63 & 3^2 7^1 & \text{--} & \text{N} & \text{N} & \text{Y} & \text{--} & -7 & 1 & 2 & 1.2857143 & \text{--} & 33 & 181 & -148 \\
 64 & 2^6 & \text{--} & \text{N} & \text{Y} & \text{N} & \text{--} & 2 & 1 & 0 & 3.5000000 & \text{--} & 35 & 183 & -148 \\
 65 & 5^1 13^1 & \text{--} & \text{Y} & \text{N} & \text{N} & \text{--} & 5 & 1 & 0 & 1.0000000 & \text{--} & 40 & 188 & -148 \\
 66 & 2^1 3^1 11^1 & \text{--} & \text{Y} & \text{N} & \text{N} & \text{--} & -16 & 1 & 0 & 1.0000000 & \text{--} & 24 & 188 & -164 \\
 67 & 67^1 & \text{--} & \text{Y} & \text{Y} & \text{N} & \text{--} & -2 & 1 & 0 & 1.0000000 & \text{--} & 22 & 188 & -166 \\
 68 & 2^2 17^1 & \text{--} & \text{N} & \text{N} & \text{Y} & \text{--} & -7 & 1 & 2 & 1.2857143 & \text{--} & 15 & 188 & -173 \\
 69 & 3^1 23^1 & \text{--} & \text{Y} & \text{N} & \text{N} & \text{--} & 5 & 1 & 0 & 1.0000000 & \text{--} & 20 & 193 & -173 \\
 70 & 2^1 5^1 7^1 & \text{--} & \text{Y} & \text{N} & \text{N} & \text{--} & -16 & 1 & 0 & 1.0000000 & \text{--} & 4 & 193 & -189 \\
 71 & 71^1 & \text{--} & \text{Y} & \text{Y} & \text{N} & \text{--} & -2 & 1 & 0 & 1.0000000 & \text{--} & 2 & 193 & -191 \\
 72 & 2^3 3^2 & \text{--} & \text{N} & \text{N} & \text{Y} & \text{--} & -23 & 1 & 18 & 1.4782609 & \text{--} & -21 & 193 & -214 \\
 73 & 73^1 & \text{--} & \text{Y} & \text{Y} & \text{N} & \text{--} & -2 & 1 & 0 & 1.0000000 & \text{--} & -23 & 193 & -216 \\
 74 & 2^1 37^1 & \text{--} & \text{Y} & \text{N} & \text{N} & \text{--} & 5 & 1 & 0 & 1.0000000 & \text{--} & -18 & 198 & -216 \\
 75 & 3^1 5^2 & \text{--} & \text{N} & \text{N} & \text{Y} & \text{--} & -7 & 1 & 2 & 1.2857143 & \text{--} & -25 & 198 & -223 \\
 76 & 2^2 19^1 & \text{--} & \text{N} & \text{N} & \text{Y} & \text{--} & -7 & 1 & 2 & 1.2857143 & \text{--} & -32 & 198 & -230 \\
 77 & 7^1 11^1 & \text{--} & \text{Y} & \text{N} & \text{N} & \text{--} & 5 & 1 & 0 & 1.0000000 & \text{--} & -27 & 203 & -230 \\
 78 & 2^1 3^1 13^1 & \text{--} & \text{Y} & \text{N} & \text{N} & \text{--} & -16 & 1 & 0 & 1.0000000 & \text{--} & -43 & 203 & -246 \\
 79 & 79^1 & \text{--} & \text{Y} & \text{Y} & \text{N} & \text{--} & -2 & 1 & 0 & 1.0000000 & \text{--} & -45 & 203 & -248 \\
 80 & 2^4 5^1 & \text{--} & \text{N} & \text{N} & \text{Y} & \text{--} & -11 & 1 & 6 & 1.8181818 & \text{--} & -56 & 203 & -259 \\
 81 & 3^4 & \text{--} & \text{N} & \text{Y} & \text{N} & \text{--} & 2 & 1 & 0 & 2.5000000 & \text{--} & -54 & 205 & -259 \\
 82 & 2^1 41^1 & \text{--} & \text{Y} & \text{N} & \text{N} & \text{--} & 5 & 1 & 0 & 1.0000000 & \text{--} & -49 & 210 & -259 \\
 83 & 83^1 & \text{--} & \text{Y} & \text{Y} & \text{N} & \text{--} & -2 & 1 & 0 & 1.0000000 & \text{--} & -51 & 210 & -261 \\
 84 & 2^2 3^1 7^1 & \text{--} & \text{N} & \text{N} & \text{Y} & \text{--} & 30 & 1 & 14 & 1.1666667 & \text{--} & -21 & 240 & -261 \\
 85 & 5^1 17^1 & \text{--} & \text{Y} & \text{N} & \text{N} & \text{--} & 5 & 1 & 0 & 1.0000000 & \text{--} & -16 & 245 & -261 \\
 86 & 2^1 43^1 & \text{--} & \text{Y} & \text{N} & \text{N} & \text{--} & 5 & 1 & 0 & 1.0000000 & \text{--} & -11 & 250 & -261 \\
 87 & 3^1 29^1 & \text{--} & \text{Y} & \text{N} & \text{N} & \text{--} & 5 & 1 & 0 & 1.0000000 & \text{--} & -6 & 255 & -261 \\
 88 & 2^3 11^1 & \text{--} & \text{N} & \text{N} & \text{Y} & \text{--} & 9 & 1 & 4 & 1.5555556 & \text{--} & 3 & 264 & -261 \\
 89 & 89^1 & \text{--} & \text{Y} & \text{Y} & \text{N} & \text{--} & -2 & 1 & 0 & 1.0000000 & \text{--} & 1 & 264 & -263 \\
 90 & 2^1 3^2 5^1 & \text{--} & \text{N} & \text{N} & \text{Y} & \text{--} & 30 & 1 & 14 & 1.1666667 & \text{--} & 31 & 294 & -263 \\
 91 & 7^1 13^1 & \text{--} & \text{Y} & \text{N} & \text{N} & \text{--} & 5 & 1 & 0 & 1.0000000 & \text{--} & 36 & 299 & -263 \\
 92 & 2^2 23^1 & \text{--} & \text{N} & \text{N} & \text{Y} & \text{--} & -7 & 1 & 2 & 1.2857143 & \text{--} & 29 & 299 & -270 \\
 93 & 3^1 31^1 & \text{--} & \text{Y} & \text{N} & \text{N} & \text{--} & 5 & 1 & 0 & 1.0000000 & \text{--} & 34 & 304 & -270 \\
 94 & 2^1 47^1 & \text{--} & \text{Y} & \text{N} & \text{N} & \text{--} & 5 & 1 & 0 & 1.0000000 & \text{--} & 39 & 309 & -270 \\
 95 & 5^1 19^1 & \text{--} & \text{Y} & \text{N} & \text{N} & \text{--} & 5 & 1 & 0 & 1.0000000 & \text{--} & 44 & 314 & -270 \\
 96 & 2^5 3^1 & \text{--} & \text{N} & \text{N} & \text{Y} & \text{--} & 13 & 1 & 8 & 2.0769231 & \text{--} & 57 & 327 & -270 \\
 97 & 97^1 & \text{--} & \text{Y} & \text{Y} & \text{N} & \text{--} & -2 & 1 & 0 & 1.0000000 & \text{--} & 55 & 327 & -272 \\
 98 & 2^1 7^2 & \text{--} & \text{N} & \text{N} & \text{Y} & \text{--} & -7 & 1 & 2 & 1.2857143 & \text{--} & 48 & 327 & -279 \\
 99 & 3^2 11^1 & \text{--} & \text{N} & \text{N} & \text{Y} & \text{--} & -7 & 1 & 2 & 1.2857143 & \text{--} & 41 & 327 & -286 \\
 100 & 2^2 5^2 & \text{--} & \text{N} & \text{N} & \text{Y} & \text{--} & 14 & 1 & 9 & 1.3571429 & \text{--} & 55 & 341 & -286 \\
 101 & 101^1 & \text{--} & \text{Y} & \text{Y} & \text{N} & \text{--} & -2 & 1 & 0 & 1.0000000 & \text{--} & 53 & 341 & -288 \\
 102 & 2^1 3^1 17^1 & \text{--} & \text{Y} & \text{N} & \text{N} & \text{--} & -16 & 1 & 0 & 1.0000000 & \text{--} & 37 & 341 & -304 \\
 103 & 103^1 & \text{--} & \text{Y} & \text{Y} & \text{N} & \text{--} & -2 & 1 & 0 & 1.0000000 & \text{--} & 35 & 341 & -306 \\
 104 & 2^3 13^1 & \text{--} & \text{N} & \text{N} & \text{Y} & \text{--} & 9 & 1 & 4 & 1.5555556 & \text{--} & 44 & 350 & -306 \\
 105 & 3^1 5^1 7^1 & \text{--} & \text{Y} & \text{N} & \text{N} & \text{--} & -16 & 1 & 0 & 1.0000000 & \text{--} & 28 & 350 & -322 \\
 106 & 2^1 53^1 & \text{--} & \text{Y} & \text{N} & \text{N} & \text{--} & 5 & 1 & 0 & 1.0000000 & \text{--} & 33 & 355 & -322 \\
 107 & 107^1 & \text{--} & \text{Y} & \text{Y} & \text{N} & \text{--} & -2 & 1 & 0 & 1.0000000 & \text{--} & 31 & 355 & -324 \\
 108 & 2^2 3^3 & \text{--} & \text{N} & \text{N} & \text{Y} & \text{--} & -23 & 1 & 18 & 1.4782609 & \text{--} & 8 & 355 & -347 \\
 109 & 109^1 & \text{--} & \text{Y} & \text{Y} & \text{N} & \text{--} & -2 & 1 & 0 & 1.0000000 & \text{--} & 6 & 355 & -349 \\
 110 & 2^1 5^1 11^1 & \text{--} & \text{Y} & \text{N} & \text{N} & \text{--} & -16 & 1 & 0 & 1.0000000 & \text{--} & -10 & 355 & -365 \\
 111 & 3^1 37^1 & \text{--} & \text{Y} & \text{N} & \text{N} & \text{--} & 5 & 1 & 0 & 1.0000000 & \text{--} & -5 & 360 & -365 \\
 112 & 2^4 7^1 & \text{--} & \text{N} & \text{N} & \text{Y} & \text{--} & -11 & 1 & 6 & 1.8181818 & \text{--} & -16 & 360 & -376 \\
 113 & 113^1 & \text{--} & \text{Y} & \text{Y} & \text{N} & \text{--} & -2 & 1 & 0 & 1.0000000 & \text{--} & -18 & 360 & -378 \\
 114 & 2^1 3^1 19^1 & \text{--} & \text{Y} & \text{N} & \text{N} & \text{--} & -16 & 1 & 0 & 1.0000000 & \text{--} & -34 & 360 & -394 \\
 115 & 5^1 23^1 & \text{--} & \text{Y} & \text{N} & \text{N} & \text{--} & 5 & 1 & 0 & 1.0000000 & \text{--} & -29 & 365 & -394 \\
 116 & 2^2 29^1 & \text{--} & \text{N} & \text{N} & \text{Y} & \text{--} & -7 & 1 & 2 & 1.2857143 & \text{--} & -36 & 365 & -401 \\
 117 & 3^2 13^1 & \text{--} & \text{N} & \text{N} & \text{Y} & \text{--} & -7 & 1 & 2 & 1.2857143 & \text{--} & -43 & 365 & -408 \\
 118 & 2^1 59^1 & \text{--} & \text{Y} & \text{N} & \text{N} & \text{--} & 5 & 1 & 0 & 1.0000000 & \text{--} & -38 & 370 & -408 \\
 119 & 7^1 17^1 & \text{--} & \text{Y} & \text{N} & \text{N} & \text{--} & 5 & 1 & 0 & 1.0000000 & \text{--} & -33 & 375 & -408 \\
 120 & 2^3 3^1 5^1 & \text{--} & \text{N} & \text{N} & \text{Y} & \text{--} & -48 & 1 & 32 & 1.3333333 & \text{--} & -81 & 375 & -456 \\
 121 & 11^2 & \text{--} & \text{N} & \text{Y} & \text{N} & \text{--} & 2 & 1 & 0 & 1.5000000 & \text{--} & -79 & 377 & -456 \\
 122 & 2^1 61^1 & \text{--} & \text{Y} & \text{N} & \text{N} & \text{--} & 5 & 1 & 0 & 1.0000000 & \text{--} & -74 & 382 & -456 \\
 123 & 3^1 41^1 & \text{--} & \text{Y} & \text{N} & \text{N} & \text{--} & 5 & 1 & 0 & 1.0000000 & \text{--} & -69 & 387 & -456 \\
 124 & 2^2 31^1 & \text{--} & \text{N} & \text{N} & \text{Y} & \text{--} & -7 & 1 & 2 & 1.2857143 & \text{--} & -76 & 387 & -463 \\
\end{array}
}
\end{equation*}

\end{table} 


\newpage
\begin{table}[h!]

\centering

\tiny
\begin{equation*}
\boxed{
\begin{array}{|cc|c|ccc|c|c|ccc|c|ccc}
 n & \mathbf{Primes} & & \mathbf{Sqfree} & \mathbf{PPower} & \bar{\mathbb{S}} & & g^{-1}(n) & 
 \lambda(n) \operatorname{sgn}(g^{-1}(n)) & \lambda(n) g^{-1}(n) - \widehat{f}_1(n) & 
 \frac{\sum\limits_{d|n} C_{\Omega(d)}(d)}{|g^{-1}(n)|} & & G^{-1}(n) & G^{-1}_{+}(n) & G^{-1}_{-}(n) \\ \hline 
 125 & 5^3 & \text{--} & \text{N} & \text{Y} & \text{N} & \text{--} & -2 & 1 & 0 & 2.0000000 & \text{--} & -78 & 387 & -465 \\
 126 & 2^1 3^2 7^1 & \text{--} & \text{N} & \text{N} & \text{Y} & \text{--} & 30 & 1 & 14 & 1.1666667 & \text{--} & -48 & 417 & -465 \\
 127 & 127^1 & \text{--} & \text{Y} & \text{Y} & \text{N} & \text{--} & -2 & 1 & 0 & 1.0000000 & \text{--} & -50 & 417 & -467 \\
 128 & 2^7 & \text{--} & \text{N} & \text{Y} & \text{N} & \text{--} & -2 & 1 & 0 & 4.0000000 & \text{--} & -52 & 417 & -469 \\
 129 & 3^1 43^1 & \text{--} & \text{Y} & \text{N} & \text{N} & \text{--} & 5 & 1 & 0 & 1.0000000 & \text{--} & -47 & 422 & -469 \\
 130 & 2^1 5^1 13^1 & \text{--} & \text{Y} & \text{N} & \text{N} & \text{--} & -16 & 1 & 0 & 1.0000000 & \text{--} & -63 & 422 & -485 \\
 131 & 131^1 & \text{--} & \text{Y} & \text{Y} & \text{N} & \text{--} & -2 & 1 & 0 & 1.0000000 & \text{--} & -65 & 422 & -487 \\
 132 & 2^2 3^1 11^1 & \text{--} & \text{N} & \text{N} & \text{Y} & \text{--} & 30 & 1 & 14 & 1.1666667 & \text{--} & -35 & 452 & -487 \\
 133 & 7^1 19^1 & \text{--} & \text{Y} & \text{N} & \text{N} & \text{--} & 5 & 1 & 0 & 1.0000000 & \text{--} & -30 & 457 & -487 \\
 134 & 2^1 67^1 & \text{--} & \text{Y} & \text{N} & \text{N} & \text{--} & 5 & 1 & 0 & 1.0000000 & \text{--} & -25 & 462 & -487 \\
 135 & 3^3 5^1 & \text{--} & \text{N} & \text{N} & \text{Y} & \text{--} & 9 & 1 & 4 & 1.5555556 & \text{--} & -16 & 471 & -487 \\
 136 & 2^3 17^1 & \text{--} & \text{N} & \text{N} & \text{Y} & \text{--} & 9 & 1 & 4 & 1.5555556 & \text{--} & -7 & 480 & -487 \\
 137 & 137^1 & \text{--} & \text{Y} & \text{Y} & \text{N} & \text{--} & -2 & 1 & 0 & 1.0000000 & \text{--} & -9 & 480 & -489 \\
 138 & 2^1 3^1 23^1 & \text{--} & \text{Y} & \text{N} & \text{N} & \text{--} & -16 & 1 & 0 & 1.0000000 & \text{--} & -25 & 480 & -505 \\
 139 & 139^1 & \text{--} & \text{Y} & \text{Y} & \text{N} & \text{--} & -2 & 1 & 0 & 1.0000000 & \text{--} & -27 & 480 & -507 \\
 140 & 2^2 5^1 7^1 & \text{--} & \text{N} & \text{N} & \text{Y} & \text{--} & 30 & 1 & 14 & 1.1666667 & \text{--} & 3 & 510 & -507 \\
 141 & 3^1 47^1 & \text{--} & \text{Y} & \text{N} & \text{N} & \text{--} & 5 & 1 & 0 & 1.0000000 & \text{--} & 8 & 515 & -507 \\
 142 & 2^1 71^1 & \text{--} & \text{Y} & \text{N} & \text{N} & \text{--} & 5 & 1 & 0 & 1.0000000 & \text{--} & 13 & 520 & -507 \\
 143 & 11^1 13^1 & \text{--} & \text{Y} & \text{N} & \text{N} & \text{--} & 5 & 1 & 0 & 1.0000000 & \text{--} & 18 & 525 & -507 \\
 144 & 2^4 3^2 & \text{--} & \text{N} & \text{N} & \text{Y} & \text{--} & 34 & 1 & 29 & 1.6176471 & \text{--} & 52 & 559 & -507 \\
 145 & 5^1 29^1 & \text{--} & \text{Y} & \text{N} & \text{N} & \text{--} & 5 & 1 & 0 & 1.0000000 & \text{--} & 57 & 564 & -507 \\
 146 & 2^1 73^1 & \text{--} & \text{Y} & \text{N} & \text{N} & \text{--} & 5 & 1 & 0 & 1.0000000 & \text{--} & 62 & 569 & -507 \\
 147 & 3^1 7^2 & \text{--} & \text{N} & \text{N} & \text{Y} & \text{--} & -7 & 1 & 2 & 1.2857143 & \text{--} & 55 & 569 & -514 \\
 148 & 2^2 37^1 & \text{--} & \text{N} & \text{N} & \text{Y} & \text{--} & -7 & 1 & 2 & 1.2857143 & \text{--} & 48 & 569 & -521 \\
 149 & 149^1 & \text{--} & \text{Y} & \text{Y} & \text{N} & \text{--} & -2 & 1 & 0 & 1.0000000 & \text{--} & 46 & 569 & -523 \\
 150 & 2^1 3^1 5^2 & \text{--} & \text{N} & \text{N} & \text{Y} & \text{--} & 30 & 1 & 14 & 1.1666667 & \text{--} & 76 & 599 & -523 \\
 151 & 151^1 & \text{--} & \text{Y} & \text{Y} & \text{N} & \text{--} & -2 & 1 & 0 & 1.0000000 & \text{--} & 74 & 599 & -525 \\
 152 & 2^3 19^1 & \text{--} & \text{N} & \text{N} & \text{Y} & \text{--} & 9 & 1 & 4 & 1.5555556 & \text{--} & 83 & 608 & -525 \\
 153 & 3^2 17^1 & \text{--} & \text{N} & \text{N} & \text{Y} & \text{--} & -7 & 1 & 2 & 1.2857143 & \text{--} & 76 & 608 & -532 \\
 154 & 2^1 7^1 11^1 & \text{--} & \text{Y} & \text{N} & \text{N} & \text{--} & -16 & 1 & 0 & 1.0000000 & \text{--} & 60 & 608 & -548 \\
 155 & 5^1 31^1 & \text{--} & \text{Y} & \text{N} & \text{N} & \text{--} & 5 & 1 & 0 & 1.0000000 & \text{--} & 65 & 613 & -548 \\
 156 & 2^2 3^1 13^1 & \text{--} & \text{N} & \text{N} & \text{Y} & \text{--} & 30 & 1 & 14 & 1.1666667 & \text{--} & 95 & 643 & -548 \\
 157 & 157^1 & \text{--} & \text{Y} & \text{Y} & \text{N} & \text{--} & -2 & 1 & 0 & 1.0000000 & \text{--} & 93 & 643 & -550 \\
 158 & 2^1 79^1 & \text{--} & \text{Y} & \text{N} & \text{N} & \text{--} & 5 & 1 & 0 & 1.0000000 & \text{--} & 98 & 648 & -550 \\
 159 & 3^1 53^1 & \text{--} & \text{Y} & \text{N} & \text{N} & \text{--} & 5 & 1 & 0 & 1.0000000 & \text{--} & 103 & 653 & -550 \\
 160 & 2^5 5^1 & \text{--} & \text{N} & \text{N} & \text{Y} & \text{--} & 13 & 1 & 8 & 2.0769231 & \text{--} & 116 & 666 & -550 \\
 161 & 7^1 23^1 & \text{--} & \text{Y} & \text{N} & \text{N} & \text{--} & 5 & 1 & 0 & 1.0000000 & \text{--} & 121 & 671 & -550 \\
 162 & 2^1 3^4 & \text{--} & \text{N} & \text{N} & \text{Y} & \text{--} & -11 & 1 & 6 & 1.8181818 & \text{--} & 110 & 671 & -561 \\
 163 & 163^1 & \text{--} & \text{Y} & \text{Y} & \text{N} & \text{--} & -2 & 1 & 0 & 1.0000000 & \text{--} & 108 & 671 & -563 \\
 164 & 2^2 41^1 & \text{--} & \text{N} & \text{N} & \text{Y} & \text{--} & -7 & 1 & 2 & 1.2857143 & \text{--} & 101 & 671 & -570 \\
 165 & 3^1 5^1 11^1 & \text{--} & \text{Y} & \text{N} & \text{N} & \text{--} & -16 & 1 & 0 & 1.0000000 & \text{--} & 85 & 671 & -586 \\
 166 & 2^1 83^1 & \text{--} & \text{Y} & \text{N} & \text{N} & \text{--} & 5 & 1 & 0 & 1.0000000 & \text{--} & 90 & 676 & -586 \\
 167 & 167^1 & \text{--} & \text{Y} & \text{Y} & \text{N} & \text{--} & -2 & 1 & 0 & 1.0000000 & \text{--} & 88 & 676 & -588 \\
 168 & 2^3 3^1 7^1 & \text{--} & \text{N} & \text{N} & \text{Y} & \text{--} & -48 & 1 & 32 & 1.3333333 & \text{--} & 40 & 676 & -636 \\
 169 & 13^2 & \text{--} & \text{N} & \text{Y} & \text{N} & \text{--} & 2 & 1 & 0 & 1.5000000 & \text{--} & 42 & 678 & -636 \\
 170 & 2^1 5^1 17^1 & \text{--} & \text{Y} & \text{N} & \text{N} & \text{--} & -16 & 1 & 0 & 1.0000000 & \text{--} & 26 & 678 & -652 \\
 171 & 3^2 19^1 & \text{--} & \text{N} & \text{N} & \text{Y} & \text{--} & -7 & 1 & 2 & 1.2857143 & \text{--} & 19 & 678 & -659 \\
 172 & 2^2 43^1 & \text{--} & \text{N} & \text{N} & \text{Y} & \text{--} & -7 & 1 & 2 & 1.2857143 & \text{--} & 12 & 678 & -666 \\
 173 & 173^1 & \text{--} & \text{Y} & \text{Y} & \text{N} & \text{--} & -2 & 1 & 0 & 1.0000000 & \text{--} & 10 & 678 & -668 \\
 174 & 2^1 3^1 29^1 & \text{--} & \text{Y} & \text{N} & \text{N} & \text{--} & -16 & 1 & 0 & 1.0000000 & \text{--} & -6 & 678 & -684 \\
 175 & 5^2 7^1 & \text{--} & \text{N} & \text{N} & \text{Y} & \text{--} & -7 & 1 & 2 & 1.2857143 & \text{--} & -13 & 678 & -691 \\
 176 & 2^4 11^1 & \text{--} & \text{N} & \text{N} & \text{Y} & \text{--} & -11 & 1 & 6 & 1.8181818 & \text{--} & -24 & 678 & -702 \\
 177 & 3^1 59^1 & \text{--} & \text{Y} & \text{N} & \text{N} & \text{--} & 5 & 1 & 0 & 1.0000000 & \text{--} & -19 & 683 & -702 \\
 178 & 2^1 89^1 & \text{--} & \text{Y} & \text{N} & \text{N} & \text{--} & 5 & 1 & 0 & 1.0000000 & \text{--} & -14 & 688 & -702 \\
 179 & 179^1 & \text{--} & \text{Y} & \text{Y} & \text{N} & \text{--} & -2 & 1 & 0 & 1.0000000 & \text{--} & -16 & 688 & -704 \\
 180 & 2^2 3^2 5^1 & \text{--} & \text{N} & \text{N} & \text{Y} & \text{--} & -74 & 1 & 58 & 1.2162162 & \text{--} & -90 & 688 & -778 \\
 181 & 181^1 & \text{--} & \text{Y} & \text{Y} & \text{N} & \text{--} & -2 & 1 & 0 & 1.0000000 & \text{--} & -92 & 688 & -780 \\
 182 & 2^1 7^1 13^1 & \text{--} & \text{Y} & \text{N} & \text{N} & \text{--} & -16 & 1 & 0 & 1.0000000 & \text{--} & -108 & 688 & -796 \\
 183 & 3^1 61^1 & \text{--} & \text{Y} & \text{N} & \text{N} & \text{--} & 5 & 1 & 0 & 1.0000000 & \text{--} & -103 & 693 & -796 \\
 184 & 2^3 23^1 & \text{--} & \text{N} & \text{N} & \text{Y} & \text{--} & 9 & 1 & 4 & 1.5555556 & \text{--} & -94 & 702 & -796 \\
 185 & 5^1 37^1 & \text{--} & \text{Y} & \text{N} & \text{N} & \text{--} & 5 & 1 & 0 & 1.0000000 & \text{--} & -89 & 707 & -796 \\
 186 & 2^1 3^1 31^1 & \text{--} & \text{Y} & \text{N} & \text{N} & \text{--} & -16 & 1 & 0 & 1.0000000 & \text{--} & -105 & 707 & -812 \\
 187 & 11^1 17^1 & \text{--} & \text{Y} & \text{N} & \text{N} & \text{--} & 5 & 1 & 0 & 1.0000000 & \text{--} & -100 & 712 & -812 \\
 188 & 2^2 47^1 & \text{--} & \text{N} & \text{N} & \text{Y} & \text{--} & -7 & 1 & 2 & 1.2857143 & \text{--} & -107 & 712 & -819 \\
 189 & 3^3 7^1 & \text{--} & \text{N} & \text{N} & \text{Y} & \text{--} & 9 & 1 & 4 & 1.5555556 & \text{--} & -98 & 721 & -819 \\
 190 & 2^1 5^1 19^1 & \text{--} & \text{Y} & \text{N} & \text{N} & \text{--} & -16 & 1 & 0 & 1.0000000 & \text{--} & -114 & 721 & -835 \\
 191 & 191^1 & \text{--} & \text{Y} & \text{Y} & \text{N} & \text{--} & -2 & 1 & 0 & 1.0000000 & \text{--} & -116 & 721 & -837 \\
 192 & 2^6 3^1 & \text{--} & \text{N} & \text{N} & \text{Y} & \text{--} & -15 & 1 & 10 & 2.3333333 & \text{--} & -131 & 721 & -852 \\
 193 & 193^1 & \text{--} & \text{Y} & \text{Y} & \text{N} & \text{--} & -2 & 1 & 0 & 1.0000000 & \text{--} & -133 & 721 & -854 \\
 194 & 2^1 97^1 & \text{--} & \text{Y} & \text{N} & \text{N} & \text{--} & 5 & 1 & 0 & 1.0000000 & \text{--} & -128 & 726 & -854 \\
 195 & 3^1 5^1 13^1 & \text{--} & \text{Y} & \text{N} & \text{N} & \text{--} & -16 & 1 & 0 & 1.0000000 & \text{--} & -144 & 726 & -870 \\
 196 & 2^2 7^2 & \text{--} & \text{N} & \text{N} & \text{Y} & \text{--} & 14 & 1 & 9 & 1.3571429 & \text{--} & -130 & 740 & -870 \\
 197 & 197^1 & \text{--} & \text{Y} & \text{Y} & \text{N} & \text{--} & -2 & 1 & 0 & 1.0000000 & \text{--} & -132 & 740 & -872 \\
 198 & 2^1 3^2 11^1 & \text{--} & \text{N} & \text{N} & \text{Y} & \text{--} & 30 & 1 & 14 & 1.1666667 & \text{--} & -102 & 770 & -872 \\
 199 & 199^1 & \text{--} & \text{Y} & \text{Y} & \text{N} & \text{--} & -2 & 1 & 0 & 1.0000000 & \text{--} & -104 & 770 & -874 \\
 200 & 2^3 5^2 & \text{--} & \text{N} & \text{N} & \text{Y} & \text{--} & -23 & 1 & 18 & 1.4782609 & \text{--} & -127 & 770 & -897 \\
\end{array}
}
\end{equation*}

\end{table} 


\newpage
\begin{table}[h!]

\centering

\tiny
\begin{equation*}
\boxed{
\begin{array}{|cc|c|ccc|c|c|ccc|c|ccc}
 n & \mathbf{Primes} & & \mathbf{Sqfree} & \mathbf{PPower} & \bar{\mathbb{S}} & & g^{-1}(n) & 
 \lambda(n) \operatorname{sgn}(g^{-1}(n)) & \lambda(n) g^{-1}(n) - \widehat{f}_1(n) & 
 \frac{\sum\limits_{d|n} C_{\Omega(d)}(d)}{|g^{-1}(n)|} & & G^{-1}(n) & G^{-1}_{+}(n) & G^{-1}_{-}(n) \\ \hline 
 201 & 3^1 67^1 & \text{--} & \text{Y} & \text{N} & \text{N} & \text{--} & 5 & 1 & 0 & 1.0000000 & \text{--} & -122 & 775 & -897 \\
 202 & 2^1 101^1 & \text{--} & \text{Y} & \text{N} & \text{N} & \text{--} & 5 & 1 & 0 & 1.0000000 & \text{--} & -117 & 780 & -897 \\
 203 & 7^1 29^1 & \text{--} & \text{Y} & \text{N} & \text{N} & \text{--} & 5 & 1 & 0 & 1.0000000 & \text{--} & -112 & 785 & -897 \\
 204 & 2^2 3^1 17^1 & \text{--} & \text{N} & \text{N} & \text{Y} & \text{--} & 30 & 1 & 14 & 1.1666667 & \text{--} & -82 & 815 & -897 \\
 205 & 5^1 41^1 & \text{--} & \text{Y} & \text{N} & \text{N} & \text{--} & 5 & 1 & 0 & 1.0000000 & \text{--} & -77 & 820 & -897 \\
 206 & 2^1 103^1 & \text{--} & \text{Y} & \text{N} & \text{N} & \text{--} & 5 & 1 & 0 & 1.0000000 & \text{--} & -72 & 825 & -897 \\
 207 & 3^2 23^1 & \text{--} & \text{N} & \text{N} & \text{Y} & \text{--} & -7 & 1 & 2 & 1.2857143 & \text{--} & -79 & 825 & -904 \\
 208 & 2^4 13^1 & \text{--} & \text{N} & \text{N} & \text{Y} & \text{--} & -11 & 1 & 6 & 1.8181818 & \text{--} & -90 & 825 & -915 \\
 209 & 11^1 19^1 & \text{--} & \text{Y} & \text{N} & \text{N} & \text{--} & 5 & 1 & 0 & 1.0000000 & \text{--} & -85 & 830 & -915 \\
 210 & 2^1 3^1 5^1 7^1 & \text{--} & \text{Y} & \text{N} & \text{N} & \text{--} & 65 & 1 & 0 & 1.0000000 & \text{--} & -20 & 895 & -915 \\
 211 & 211^1 & \text{--} & \text{Y} & \text{Y} & \text{N} & \text{--} & -2 & 1 & 0 & 1.0000000 & \text{--} & -22 & 895 & -917 \\
 212 & 2^2 53^1 & \text{--} & \text{N} & \text{N} & \text{Y} & \text{--} & -7 & 1 & 2 & 1.2857143 & \text{--} & -29 & 895 & -924 \\
 213 & 3^1 71^1 & \text{--} & \text{Y} & \text{N} & \text{N} & \text{--} & 5 & 1 & 0 & 1.0000000 & \text{--} & -24 & 900 & -924 \\
 214 & 2^1 107^1 & \text{--} & \text{Y} & \text{N} & \text{N} & \text{--} & 5 & 1 & 0 & 1.0000000 & \text{--} & -19 & 905 & -924 \\
 215 & 5^1 43^1 & \text{--} & \text{Y} & \text{N} & \text{N} & \text{--} & 5 & 1 & 0 & 1.0000000 & \text{--} & -14 & 910 & -924 \\
 216 & 2^3 3^3 & \text{--} & \text{N} & \text{N} & \text{Y} & \text{--} & 46 & 1 & 41 & 1.5000000 & \text{--} & 32 & 956 & -924 \\
 217 & 7^1 31^1 & \text{--} & \text{Y} & \text{N} & \text{N} & \text{--} & 5 & 1 & 0 & 1.0000000 & \text{--} & 37 & 961 & -924 \\
 218 & 2^1 109^1 & \text{--} & \text{Y} & \text{N} & \text{N} & \text{--} & 5 & 1 & 0 & 1.0000000 & \text{--} & 42 & 966 & -924 \\
 219 & 3^1 73^1 & \text{--} & \text{Y} & \text{N} & \text{N} & \text{--} & 5 & 1 & 0 & 1.0000000 & \text{--} & 47 & 971 & -924 \\
 220 & 2^2 5^1 11^1 & \text{--} & \text{N} & \text{N} & \text{Y} & \text{--} & 30 & 1 & 14 & 1.1666667 & \text{--} & 77 & 1001 & -924 \\
 221 & 13^1 17^1 & \text{--} & \text{Y} & \text{N} & \text{N} & \text{--} & 5 & 1 & 0 & 1.0000000 & \text{--} & 82 & 1006 & -924 \\
 222 & 2^1 3^1 37^1 & \text{--} & \text{Y} & \text{N} & \text{N} & \text{--} & -16 & 1 & 0 & 1.0000000 & \text{--} & 66 & 1006 & -940 \\
 223 & 223^1 & \text{--} & \text{Y} & \text{Y} & \text{N} & \text{--} & -2 & 1 & 0 & 1.0000000 & \text{--} & 64 & 1006 & -942 \\
 224 & 2^5 7^1 & \text{--} & \text{N} & \text{N} & \text{Y} & \text{--} & 13 & 1 & 8 & 2.0769231 & \text{--} & 77 & 1019 & -942 \\
 225 & 3^2 5^2 & \text{--} & \text{N} & \text{N} & \text{Y} & \text{--} & 14 & 1 & 9 & 1.3571429 & \text{--} & 91 & 1033 & -942 \\
 226 & 2^1 113^1 & \text{--} & \text{Y} & \text{N} & \text{N} & \text{--} & 5 & 1 & 0 & 1.0000000 & \text{--} & 96 & 1038 & -942 \\
 227 & 227^1 & \text{--} & \text{Y} & \text{Y} & \text{N} & \text{--} & -2 & 1 & 0 & 1.0000000 & \text{--} & 94 & 1038 & -944 \\
 228 & 2^2 3^1 19^1 & \text{--} & \text{N} & \text{N} & \text{Y} & \text{--} & 30 & 1 & 14 & 1.1666667 & \text{--} & 124 & 1068 & -944 \\
 229 & 229^1 & \text{--} & \text{Y} & \text{Y} & \text{N} & \text{--} & -2 & 1 & 0 & 1.0000000 & \text{--} & 122 & 1068 & -946 \\
 230 & 2^1 5^1 23^1 & \text{--} & \text{Y} & \text{N} & \text{N} & \text{--} & -16 & 1 & 0 & 1.0000000 & \text{--} & 106 & 1068 & -962 \\
 231 & 3^1 7^1 11^1 & \text{--} & \text{Y} & \text{N} & \text{N} & \text{--} & -16 & 1 & 0 & 1.0000000 & \text{--} & 90 & 1068 & -978 \\
 232 & 2^3 29^1 & \text{--} & \text{N} & \text{N} & \text{Y} & \text{--} & 9 & 1 & 4 & 1.5555556 & \text{--} & 99 & 1077 & -978 \\
 233 & 233^1 & \text{--} & \text{Y} & \text{Y} & \text{N} & \text{--} & -2 & 1 & 0 & 1.0000000 & \text{--} & 97 & 1077 & -980 \\
 234 & 2^1 3^2 13^1 & \text{--} & \text{N} & \text{N} & \text{Y} & \text{--} & 30 & 1 & 14 & 1.1666667 & \text{--} & 127 & 1107 & -980 \\
 235 & 5^1 47^1 & \text{--} & \text{Y} & \text{N} & \text{N} & \text{--} & 5 & 1 & 0 & 1.0000000 & \text{--} & 132 & 1112 & -980 \\
 236 & 2^2 59^1 & \text{--} & \text{N} & \text{N} & \text{Y} & \text{--} & -7 & 1 & 2 & 1.2857143 & \text{--} & 125 & 1112 & -987 \\
 237 & 3^1 79^1 & \text{--} & \text{Y} & \text{N} & \text{N} & \text{--} & 5 & 1 & 0 & 1.0000000 & \text{--} & 130 & 1117 & -987 \\
 238 & 2^1 7^1 17^1 & \text{--} & \text{Y} & \text{N} & \text{N} & \text{--} & -16 & 1 & 0 & 1.0000000 & \text{--} & 114 & 1117 & -1003 \\
 239 & 239^1 & \text{--} & \text{Y} & \text{Y} & \text{N} & \text{--} & -2 & 1 & 0 & 1.0000000 & \text{--} & 112 & 1117 & -1005 \\
 240 & 2^4 3^1 5^1 & \text{--} & \text{N} & \text{N} & \text{Y} & \text{--} & 70 & 1 & 54 & 1.5000000 & \text{--} & 182 & 1187 & -1005 \\
 241 & 241^1 & \text{--} & \text{Y} & \text{Y} & \text{N} & \text{--} & -2 & 1 & 0 & 1.0000000 & \text{--} & 180 & 1187 & -1007 \\
 242 & 2^1 11^2 & \text{--} & \text{N} & \text{N} & \text{Y} & \text{--} & -7 & 1 & 2 & 1.2857143 & \text{--} & 173 & 1187 & -1014 \\
 243 & 3^5 & \text{--} & \text{N} & \text{Y} & \text{N} & \text{--} & -2 & 1 & 0 & 3.0000000 & \text{--} & 171 & 1187 & -1016 \\
 244 & 2^2 61^1 & \text{--} & \text{N} & \text{N} & \text{Y} & \text{--} & -7 & 1 & 2 & 1.2857143 & \text{--} & 164 & 1187 & -1023 \\
 245 & 5^1 7^2 & \text{--} & \text{N} & \text{N} & \text{Y} & \text{--} & -7 & 1 & 2 & 1.2857143 & \text{--} & 157 & 1187 & -1030 \\
 246 & 2^1 3^1 41^1 & \text{--} & \text{Y} & \text{N} & \text{N} & \text{--} & -16 & 1 & 0 & 1.0000000 & \text{--} & 141 & 1187 & -1046 \\
 247 & 13^1 19^1 & \text{--} & \text{Y} & \text{N} & \text{N} & \text{--} & 5 & 1 & 0 & 1.0000000 & \text{--} & 146 & 1192 & -1046 \\
 248 & 2^3 31^1 & \text{--} & \text{N} & \text{N} & \text{Y} & \text{--} & 9 & 1 & 4 & 1.5555556 & \text{--} & 155 & 1201 & -1046 \\
 249 & 3^1 83^1 & \text{--} & \text{Y} & \text{N} & \text{N} & \text{--} & 5 & 1 & 0 & 1.0000000 & \text{--} & 160 & 1206 & -1046 \\
 250 & 2^1 5^3 & \text{--} & \text{N} & \text{N} & \text{Y} & \text{--} & 9 & 1 & 4 & 1.5555556 & \text{--} & 169 & 1215 & -1046 \\
 251 & 251^1 & \text{--} & \text{Y} & \text{Y} & \text{N} & \text{--} & -2 & 1 & 0 & 1.0000000 & \text{--} & 167 & 1215 & -1048 \\
 252 & 2^2 3^2 7^1 & \text{--} & \text{N} & \text{N} & \text{Y} & \text{--} & -74 & 1 & 58 & 1.2162162 & \text{--} & 93 & 1215 & -1122 \\
 253 & 11^1 23^1 & \text{--} & \text{Y} & \text{N} & \text{N} & \text{--} & 5 & 1 & 0 & 1.0000000 & \text{--} & 98 & 1220 & -1122 \\
 254 & 2^1 127^1 & \text{--} & \text{Y} & \text{N} & \text{N} & \text{--} & 5 & 1 & 0 & 1.0000000 & \text{--} & 103 & 1225 & -1122 \\
 255 & 3^1 5^1 17^1 & \text{--} & \text{Y} & \text{N} & \text{N} & \text{--} & -16 & 1 & 0 & 1.0000000 & \text{--} & 87 & 1225 & -1138 \\
 256 & 2^8 & \text{--} & \text{N} & \text{Y} & \text{N} & \text{--} & 2 & 1 & 0 & 4.5000000 & \text{--} & 89 & 1227 & -1138 \\
 257 & 257^1 & \text{--} & \text{Y} & \text{Y} & \text{N} & \text{--} & -2 & 1 & 0 & 1.0000000 & \text{--} & 87 & 1227 & -1140 \\
 258 & 2^1 3^1 43^1 & \text{--} & \text{Y} & \text{N} & \text{N} & \text{--} & -16 & 1 & 0 & 1.0000000 & \text{--} & 71 & 1227 & -1156 \\
 259 & 7^1 37^1 & \text{--} & \text{Y} & \text{N} & \text{N} & \text{--} & 5 & 1 & 0 & 1.0000000 & \text{--} & 76 & 1232 & -1156 \\
 260 & 2^2 5^1 13^1 & \text{--} & \text{N} & \text{N} & \text{Y} & \text{--} & 30 & 1 & 14 & 1.1666667 & \text{--} & 106 & 1262 & -1156 \\
 261 & 3^2 29^1 & \text{--} & \text{N} & \text{N} & \text{Y} & \text{--} & -7 & 1 & 2 & 1.2857143 & \text{--} & 99 & 1262 & -1163 \\
 262 & 2^1 131^1 & \text{--} & \text{Y} & \text{N} & \text{N} & \text{--} & 5 & 1 & 0 & 1.0000000 & \text{--} & 104 & 1267 & -1163 \\
 263 & 263^1 & \text{--} & \text{Y} & \text{Y} & \text{N} & \text{--} & -2 & 1 & 0 & 1.0000000 & \text{--} & 102 & 1267 & -1165 \\
 264 & 2^3 3^1 11^1 & \text{--} & \text{N} & \text{N} & \text{Y} & \text{--} & -48 & 1 & 32 & 1.3333333 & \text{--} & 54 & 1267 & -1213 \\
 265 & 5^1 53^1 & \text{--} & \text{Y} & \text{N} & \text{N} & \text{--} & 5 & 1 & 0 & 1.0000000 & \text{--} & 59 & 1272 & -1213 \\
 266 & 2^1 7^1 19^1 & \text{--} & \text{Y} & \text{N} & \text{N} & \text{--} & -16 & 1 & 0 & 1.0000000 & \text{--} & 43 & 1272 & -1229 \\
 267 & 3^1 89^1 & \text{--} & \text{Y} & \text{N} & \text{N} & \text{--} & 5 & 1 & 0 & 1.0000000 & \text{--} & 48 & 1277 & -1229 \\
 268 & 2^2 67^1 & \text{--} & \text{N} & \text{N} & \text{Y} & \text{--} & -7 & 1 & 2 & 1.2857143 & \text{--} & 41 & 1277 & -1236 \\
 269 & 269^1 & \text{--} & \text{Y} & \text{Y} & \text{N} & \text{--} & -2 & 1 & 0 & 1.0000000 & \text{--} & 39 & 1277 & -1238 \\
 270 & 2^1 3^3 5^1 & \text{--} & \text{N} & \text{N} & \text{Y} & \text{--} & -48 & 1 & 32 & 1.3333333 & \text{--} & -9 & 1277 & -1286 \\
 271 & 271^1 & \text{--} & \text{Y} & \text{Y} & \text{N} & \text{--} & -2 & 1 & 0 & 1.0000000 & \text{--} & -11 & 1277 & -1288 \\
 272 & 2^4 17^1 & \text{--} & \text{N} & \text{N} & \text{Y} & \text{--} & -11 & 1 & 6 & 1.8181818 & \text{--} & -22 & 1277 & -1299 \\
 273 & 3^1 7^1 13^1 & \text{--} & \text{Y} & \text{N} & \text{N} & \text{--} & -16 & 1 & 0 & 1.0000000 & \text{--} & -38 & 1277 & -1315 \\
 274 & 2^1 137^1 & \text{--} & \text{Y} & \text{N} & \text{N} & \text{--} & 5 & 1 & 0 & 1.0000000 & \text{--} & -33 & 1282 & -1315 \\
 275 & 5^2 11^1 & \text{--} & \text{N} & \text{N} & \text{Y} & \text{--} & -7 & 1 & 2 & 1.2857143 & \text{--} & -40 & 1282 & -1322 \\
 276 & 2^2 3^1 23^1 & \text{--} & \text{N} & \text{N} & \text{Y} & \text{--} & 30 & 1 & 14 & 1.1666667 & \text{--} & -10 & 1312 & -1322 \\
 277 & 277^1 & \text{--} & \text{Y} & \text{Y} & \text{N} & \text{--} & -2 & 1 & 0 & 1.0000000 & \text{--} & -12 & 1312 & -1324 \\
\end{array}
}
\end{equation*}

\end{table} 

\newpage
\begin{table}[h!]

\centering

\tiny
\begin{equation*}
\boxed{
\begin{array}{|cc|c|ccc|c|c|ccc|c|ccc}
 n & \mathbf{Primes} & & \mathbf{Sqfree} & \mathbf{PPower} & \bar{\mathbb{S}} & & g^{-1}(n) & 
 \lambda(n) \operatorname{sgn}(g^{-1}(n)) & \lambda(n) g^{-1}(n) - \widehat{f}_1(n) & 
 \frac{\sum\limits_{d|n} C_{\Omega(d)}(d)}{|g^{-1}(n)|} & & G^{-1}(n) & G^{-1}_{+}(n) & G^{-1}_{-}(n) \\ \hline 

 278 & 2^1 139^1 & \text{--} & \text{Y} & \text{N} & \text{N} & \text{--} & 5 & 1 & 0 & 1.0000000 & \text{--} & -7 & 1317 & -1324 \\
 279 & 3^2 31^1 & \text{--} & \text{N} & \text{N} & \text{Y} & \text{--} & -7 & 1 & 2 & 1.2857143 & \text{--} & -14 & 1317 & -1331 \\
 280 & 2^3 5^1 7^1 & \text{--} & \text{N} & \text{N} & \text{Y} & \text{--} & -48 & 1 & 32 & 1.3333333 & \text{--} & -62 & 1317 & -1379 \\
 281 & 281^1 & \text{--} & \text{Y} & \text{Y} & \text{N} & \text{--} & -2 & 1 & 0 & 1.0000000 & \text{--} & -64 & 1317 & -1381 \\
 282 & 2^1 3^1 47^1 & \text{--} & \text{Y} & \text{N} & \text{N} & \text{--} & -16 & 1 & 0 & 1.0000000 & \text{--} & -80 & 1317 & -1397 \\
 283 & 283^1 & \text{--} & \text{Y} & \text{Y} & \text{N} & \text{--} & -2 & 1 & 0 & 1.0000000 & \text{--} & -82 & 1317 & -1399 \\
 284 & 2^2 71^1 & \text{--} & \text{N} & \text{N} & \text{Y} & \text{--} & -7 & 1 & 2 & 1.2857143 & \text{--} & -89 & 1317 & -1406 \\
 285 & 3^1 5^1 19^1 & \text{--} & \text{Y} & \text{N} & \text{N} & \text{--} & -16 & 1 & 0 & 1.0000000 & \text{--} & -105 & 1317 & -1422 \\
 286 & 2^1 11^1 13^1 & \text{--} & \text{Y} & \text{N} & \text{N} & \text{--} & -16 & 1 & 0 & 1.0000000 & \text{--} & -121 & 1317 & -1438 \\
 287 & 7^1 41^1 & \text{--} & \text{Y} & \text{N} & \text{N} & \text{--} & 5 & 1 & 0 & 1.0000000 & \text{--} & -116 & 1322 & -1438 \\
 288 & 2^5 3^2 & \text{--} & \text{N} & \text{N} & \text{Y} & \text{--} & -47 & 1 & 42 & 1.7659574 & \text{--} & -163 & 1322 & -1485 \\
 289 & 17^2 & \text{--} & \text{N} & \text{Y} & \text{N} & \text{--} & 2 & 1 & 0 & 1.5000000 & \text{--} & -161 & 1324 & -1485 \\
 290 & 2^1 5^1 29^1 & \text{--} & \text{Y} & \text{N} & \text{N} & \text{--} & -16 & 1 & 0 & 1.0000000 & \text{--} & -177 & 1324 & -1501 \\
 291 & 3^1 97^1 & \text{--} & \text{Y} & \text{N} & \text{N} & \text{--} & 5 & 1 & 0 & 1.0000000 & \text{--} & -172 & 1329 & -1501 \\
 292 & 2^2 73^1 & \text{--} & \text{N} & \text{N} & \text{Y} & \text{--} & -7 & 1 & 2 & 1.2857143 & \text{--} & -179 & 1329 & -1508 \\
 293 & 293^1 & \text{--} & \text{Y} & \text{Y} & \text{N} & \text{--} & -2 & 1 & 0 & 1.0000000 & \text{--} & -181 & 1329 & -1510 \\
 294 & 2^1 3^1 7^2 & \text{--} & \text{N} & \text{N} & \text{Y} & \text{--} & 30 & 1 & 14 & 1.1666667 & \text{--} & -151 & 1359 & -1510 \\
 295 & 5^1 59^1 & \text{--} & \text{Y} & \text{N} & \text{N} & \text{--} & 5 & 1 & 0 & 1.0000000 & \text{--} & -146 & 1364 & -1510 \\
 296 & 2^3 37^1 & \text{--} & \text{N} & \text{N} & \text{Y} & \text{--} & 9 & 1 & 4 & 1.5555556 & \text{--} & -137 & 1373 & -1510 \\
 297 & 3^3 11^1 & \text{--} & \text{N} & \text{N} & \text{Y} & \text{--} & 9 & 1 & 4 & 1.5555556 & \text{--} & -128 & 1382 & -1510 \\
 298 & 2^1 149^1 & \text{--} & \text{Y} & \text{N} & \text{N} & \text{--} & 5 & 1 & 0 & 1.0000000 & \text{--} & -123 & 1387 & -1510 \\
 299 & 13^1 23^1 & \text{--} & \text{Y} & \text{N} & \text{N} & \text{--} & 5 & 1 & 0 & 1.0000000 & \text{--} & -118 & 1392 & -1510 \\
 300 & 2^2 3^1 5^2 & \text{--} & \text{N} & \text{N} & \text{Y} & \text{--} & -74 & 1 & 58 & 1.2162162 & \text{--} & -192 & 1392 & -1584 \\
 301 & 7^1 43^1 & \text{--} & \text{Y} & \text{N} & \text{N} & \text{--} & 5 & 1 & 0 & 1.0000000 & \text{--} & -187 & 1397 & -1584 \\
 302 & 2^1 151^1 & \text{--} & \text{Y} & \text{N} & \text{N} & \text{--} & 5 & 1 & 0 & 1.0000000 & \text{--} & -182 & 1402 & -1584 \\
 303 & 3^1 101^1 & \text{--} & \text{Y} & \text{N} & \text{N} & \text{--} & 5 & 1 & 0 & 1.0000000 & \text{--} & -177 & 1407 & -1584 \\
 304 & 2^4 19^1 & \text{--} & \text{N} & \text{N} & \text{Y} & \text{--} & -11 & 1 & 6 & 1.8181818 & \text{--} & -188 & 1407 & -1595 \\
 305 & 5^1 61^1 & \text{--} & \text{Y} & \text{N} & \text{N} & \text{--} & 5 & 1 & 0 & 1.0000000 & \text{--} & -183 & 1412 & -1595 \\
 306 & 2^1 3^2 17^1 & \text{--} & \text{N} & \text{N} & \text{Y} & \text{--} & 30 & 1 & 14 & 1.1666667 & \text{--} & -153 & 1442 & -1595 \\
 307 & 307^1 & \text{--} & \text{Y} & \text{Y} & \text{N} & \text{--} & -2 & 1 & 0 & 1.0000000 & \text{--} & -155 & 1442 & -1597 \\
 308 & 2^2 7^1 11^1 & \text{--} & \text{N} & \text{N} & \text{Y} & \text{--} & 30 & 1 & 14 & 1.1666667 & \text{--} & -125 & 1472 & -1597 \\
 309 & 3^1 103^1 & \text{--} & \text{Y} & \text{N} & \text{N} & \text{--} & 5 & 1 & 0 & 1.0000000 & \text{--} & -120 & 1477 & -1597 \\
 310 & 2^1 5^1 31^1 & \text{--} & \text{Y} & \text{N} & \text{N} & \text{--} & -16 & 1 & 0 & 1.0000000 & \text{--} & -136 & 1477 & -1613 \\
 311 & 311^1 & \text{--} & \text{Y} & \text{Y} & \text{N} & \text{--} & -2 & 1 & 0 & 1.0000000 & \text{--} & -138 & 1477 & -1615 \\
 312 & 2^3 3^1 13^1 & \text{--} & \text{N} & \text{N} & \text{Y} & \text{--} & -48 & 1 & 32 & 1.3333333 & \text{--} & -186 & 1477 & -1663 \\
 313 & 313^1 & \text{--} & \text{Y} & \text{Y} & \text{N} & \text{--} & -2 & 1 & 0 & 1.0000000 & \text{--} & -188 & 1477 & -1665 \\
 314 & 2^1 157^1 & \text{--} & \text{Y} & \text{N} & \text{N} & \text{--} & 5 & 1 & 0 & 1.0000000 & \text{--} & -183 & 1482 & -1665 \\
 315 & 3^2 5^1 7^1 & \text{--} & \text{N} & \text{N} & \text{Y} & \text{--} & 30 & 1 & 14 & 1.1666667 & \text{--} & -153 & 1512 & -1665 \\
 316 & 2^2 79^1 & \text{--} & \text{N} & \text{N} & \text{Y} & \text{--} & -7 & 1 & 2 & 1.2857143 & \text{--} & -160 & 1512 & -1672 \\
 317 & 317^1 & \text{--} & \text{Y} & \text{Y} & \text{N} & \text{--} & -2 & 1 & 0 & 1.0000000 & \text{--} & -162 & 1512 & -1674 \\
 318 & 2^1 3^1 53^1 & \text{--} & \text{Y} & \text{N} & \text{N} & \text{--} & -16 & 1 & 0 & 1.0000000 & \text{--} & -178 & 1512 & -1690 \\
 319 & 11^1 29^1 & \text{--} & \text{Y} & \text{N} & \text{N} & \text{--} & 5 & 1 & 0 & 1.0000000 & \text{--} & -173 & 1517 & -1690 \\
 320 & 2^6 5^1 & \text{--} & \text{N} & \text{N} & \text{Y} & \text{--} & -15 & 1 & 10 & 2.3333333 & \text{--} & -188 & 1517 & -1705 \\
 321 & 3^1 107^1 & \text{--} & \text{Y} & \text{N} & \text{N} & \text{--} & 5 & 1 & 0 & 1.0000000 & \text{--} & -183 & 1522 & -1705 \\
 322 & 2^1 7^1 23^1 & \text{--} & \text{Y} & \text{N} & \text{N} & \text{--} & -16 & 1 & 0 & 1.0000000 & \text{--} & -199 & 1522 & -1721 \\
 323 & 17^1 19^1 & \text{--} & \text{Y} & \text{N} & \text{N} & \text{--} & 5 & 1 & 0 & 1.0000000 & \text{--} & -194 & 1527 & -1721 \\
 324 & 2^2 3^4 & \text{--} & \text{N} & \text{N} & \text{Y} & \text{--} & 34 & 1 & 29 & 1.6176471 & \text{--} & -160 & 1561 & -1721 \\
 325 & 5^2 13^1 & \text{--} & \text{N} & \text{N} & \text{Y} & \text{--} & -7 & 1 & 2 & 1.2857143 & \text{--} & -167 & 1561 & -1728 \\
 326 & 2^1 163^1 & \text{--} & \text{Y} & \text{N} & \text{N} & \text{--} & 5 & 1 & 0 & 1.0000000 & \text{--} & -162 & 1566 & -1728 \\
 327 & 3^1 109^1 & \text{--} & \text{Y} & \text{N} & \text{N} & \text{--} & 5 & 1 & 0 & 1.0000000 & \text{--} & -157 & 1571 & -1728 \\
 328 & 2^3 41^1 & \text{--} & \text{N} & \text{N} & \text{Y} & \text{--} & 9 & 1 & 4 & 1.5555556 & \text{--} & -148 & 1580 & -1728 \\
 329 & 7^1 47^1 & \text{--} & \text{Y} & \text{N} & \text{N} & \text{--} & 5 & 1 & 0 & 1.0000000 & \text{--} & -143 & 1585 & -1728 \\
 330 & 2^1 3^1 5^1 11^1 & \text{--} & \text{Y} & \text{N} & \text{N} & \text{--} & 65 & 1 & 0 & 1.0000000 & \text{--} & -78 & 1650 & -1728 \\
 331 & 331^1 & \text{--} & \text{Y} & \text{Y} & \text{N} & \text{--} & -2 & 1 & 0 & 1.0000000 & \text{--} & -80 & 1650 & -1730 \\
 332 & 2^2 83^1 & \text{--} & \text{N} & \text{N} & \text{Y} & \text{--} & -7 & 1 & 2 & 1.2857143 & \text{--} & -87 & 1650 & -1737 \\
 333 & 3^2 37^1 & \text{--} & \text{N} & \text{N} & \text{Y} & \text{--} & -7 & 1 & 2 & 1.2857143 & \text{--} & -94 & 1650 & -1744 \\
 334 & 2^1 167^1 & \text{--} & \text{Y} & \text{N} & \text{N} & \text{--} & 5 & 1 & 0 & 1.0000000 & \text{--} & -89 & 1655 & -1744 \\
 335 & 5^1 67^1 & \text{--} & \text{Y} & \text{N} & \text{N} & \text{--} & 5 & 1 & 0 & 1.0000000 & \text{--} & -84 & 1660 & -1744 \\
 336 & 2^4 3^1 7^1 & \text{--} & \text{N} & \text{N} & \text{Y} & \text{--} & 70 & 1 & 54 & 1.5000000 & \text{--} & -14 & 1730 & -1744 \\
 337 & 337^1 & \text{--} & \text{Y} & \text{Y} & \text{N} & \text{--} & -2 & 1 & 0 & 1.0000000 & \text{--} & -16 & 1730 & -1746 \\
 338 & 2^1 13^2 & \text{--} & \text{N} & \text{N} & \text{Y} & \text{--} & -7 & 1 & 2 & 1.2857143 & \text{--} & -23 & 1730 & -1753 \\
 339 & 3^1 113^1 & \text{--} & \text{Y} & \text{N} & \text{N} & \text{--} & 5 & 1 & 0 & 1.0000000 & \text{--} & -18 & 1735 & -1753 \\
 340 & 2^2 5^1 17^1 & \text{--} & \text{N} & \text{N} & \text{Y} & \text{--} & 30 & 1 & 14 & 1.1666667 & \text{--} & 12 & 1765 & -1753 \\
 341 & 11^1 31^1 & \text{--} & \text{Y} & \text{N} & \text{N} & \text{--} & 5 & 1 & 0 & 1.0000000 & \text{--} & 17 & 1770 & -1753 \\
 342 & 2^1 3^2 19^1 & \text{--} & \text{N} & \text{N} & \text{Y} & \text{--} & 30 & 1 & 14 & 1.1666667 & \text{--} & 47 & 1800 & -1753 \\
 343 & 7^3 & \text{--} & \text{N} & \text{Y} & \text{N} & \text{--} & -2 & 1 & 0 & 2.0000000 & \text{--} & 45 & 1800 & -1755 \\
 344 & 2^3 43^1 & \text{--} & \text{N} & \text{N} & \text{Y} & \text{--} & 9 & 1 & 4 & 1.5555556 & \text{--} & 54 & 1809 & -1755 \\
 345 & 3^1 5^1 23^1 & \text{--} & \text{Y} & \text{N} & \text{N} & \text{--} & -16 & 1 & 0 & 1.0000000 & \text{--} & 38 & 1809 & -1771 \\
 346 & 2^1 173^1 & \text{--} & \text{Y} & \text{N} & \text{N} & \text{--} & 5 & 1 & 0 & 1.0000000 & \text{--} & 43 & 1814 & -1771 \\
 347 & 347^1 & \text{--} & \text{Y} & \text{Y} & \text{N} & \text{--} & -2 & 1 & 0 & 1.0000000 & \text{--} & 41 & 1814 & -1773 \\
 348 & 2^2 3^1 29^1 & \text{--} & \text{N} & \text{N} & \text{Y} & \text{--} & 30 & 1 & 14 & 1.1666667 & \text{--} & 71 & 1844 & -1773 \\
 349 & 349^1 & \text{--} & \text{Y} & \text{Y} & \text{N} & \text{--} & -2 & 1 & 0 & 1.0000000 & \text{--} & 69 & 1844 & -1775 \\
 350 & 2^1 5^2 7^1 & \text{--} & \text{N} & \text{N} & \text{Y} & \text{--} & 30 & 1 & 14 & 1.1666667 & \text{--} & 99 & 1874 & -1775 \\
\end{array}
}
\end{equation*}

\end{table} 

%\NBRef{A03-2020-04026}
%\NBRef{A04-2020-04026}

\newpage
\setcounter{section}{0}
\renewcommand{\thesection}{Appendix \Alph{section}}
\renewcommand{\thesubsection}{\Alph{section}.\arabic{subsection}}

\end{document}
