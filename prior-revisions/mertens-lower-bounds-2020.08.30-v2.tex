\documentclass[11pt,reqno,a4letter]{article} 

\usepackage{amsmath,amssymb,amsfonts,amscd}
\usepackage[hidelinks]{hyperref} 
\usepackage{url}
\usepackage[usenames,dvipsnames]{xcolor}
\hypersetup{
    colorlinks,
    linkcolor={green!63!darkgray},
    citecolor={blue!70!white},
    urlcolor={blue!80!white}
}

\usepackage[normalem]{ulem}
\usepackage{graphicx} 
\usepackage{datetime} 
\usepackage{cancel}
\usepackage{subcaption}
\captionsetup{format=hang,labelfont={bf},textfont={small,it}} 
\numberwithin{figure}{section}
\numberwithin{table}{section}

%\usepackage{stmaryrd,tikzsymbols,mathabx,wasysym} 
\usepackage{framed} 
\usepackage{ulem}
\usepackage[T1]{fontenc}
\usepackage{pbsi}


\usepackage{enumitem}
\setlist[itemize]{leftmargin=0.65in}

\usepackage{rotating,adjustbox}

\usepackage{diagbox}
\newcommand{\trianglenk}[2]{$\diagbox{#1}{#2}$}
\newcommand{\trianglenkII}[2]{\diagbox{#1}{#2}}

\let\citep\cite

\newcommand{\undersetbrace}[2]{\underset{\displaystyle{#1}}{\underbrace{#2}}}

\newcommand{\gkpSI}[2]{\ensuremath{\genfrac{\lbrack}{\rbrack}{0pt}{}{#1}{#2}}} 
\newcommand{\gkpSII}[2]{\ensuremath{\genfrac{\lbrace}{\rbrace}{0pt}{}{#1}{#2}}}
\newcommand{\cf}{\textit{cf.\ }} 
\newcommand{\Iverson}[1]{\ensuremath{\left[#1\right]_{\delta}}} 
\newcommand{\floor}[1]{\left\lfloor #1 \right\rfloor} 
\newcommand{\ceiling}[1]{\left\lceil #1 \right\rceil} 
\newcommand{\e}[1]{e\left(#1\right)} 
\newcommand{\seqnum}[1]{\href{http://oeis.org/#1}{\color{ProcessBlue}{\underline{#1}}}}

\usepackage{upgreek,dsfont,amssymb}
\renewcommand{\chi}{\upchi}
\newcommand{\ChiFunc}[1]{\ensuremath{\chi_{\{#1\}}}}
\newcommand{\OneFunc}[1]{\ensuremath{\mathds{1}_{#1}}}

\usepackage{ifthen}
\newcommand{\Hn}[2]{
     \ifthenelse{\equal{#2}{1}}{H_{#1}}{H_{#1}^{\left(#2\right)}}
}

\newcommand{\Floor}[2]{\ensuremath{\left\lfloor \frac{#1}{#2} \right\rfloor}}
\newcommand{\Ceiling}[2]{\ensuremath{\left\lceil \frac{#1}{#2} \right\rceil}}

\DeclareMathOperator{\DGF}{DGF} 
\DeclareMathOperator{\ds}{ds} 
\DeclareMathOperator{\Id}{Id}
\DeclareMathOperator{\fg}{fg}
\DeclareMathOperator{\Div}{div}
\DeclareMathOperator{\rpp}{rpp}
\DeclareMathOperator{\logll}{\ell\ell}

\title{
       \LARGE{
       Lower bounds on the summatory function of the M\"obius function along infinite subsequences 
       } 
}
\author{{\Large Maxie Dion Schmidt} \\ 
        %{\normalsize \href{mailto:maxieds@gmail.com}{maxieds@gmail.com}} \\[0.1cm] 
        {\normalsize Georgia Institute of Technology} \\[0.025cm] 
        {\normalsize School of Mathematics} 
} 

\date{\small\underline{Last Revised:} \today \ @\ \hhmmsstime{} \ -- \ Compiled with \LaTeX2e} 

\usepackage{amsthm} 

\theoremstyle{plain} 
\newtheorem{theorem}{Theorem}
\newtheorem{conjecture}[theorem]{Conjecture}
\newtheorem{claim}[theorem]{Claim}
\newtheorem{prop}[theorem]{Proposition}
\newtheorem{lemma}[theorem]{Lemma}
\newtheorem{cor}[theorem]{Corollary}
\numberwithin{theorem}{section}

\theoremstyle{definition} 
\newtheorem{example}[theorem]{Example}
\newtheorem{remark}[theorem]{Remark}
\newtheorem{definition}[theorem]{Definition}
\newtheorem{notation}[theorem]{Notation}
\newtheorem{question}[theorem]{Question}
\newtheorem{discussion}[theorem]{Discussion}
\newtheorem{facts}[theorem]{Facts}
\newtheorem{summary}[theorem]{Summary}
\newtheorem{heuristic}[theorem]{Heuristic}

\renewcommand{\arraystretch}{1.25} 

\setlength{\textheight}{9in}
\setlength{\topmargin}{-.18in}
\setlength{\textwidth}{7.65in} 
\setlength{\evensidemargin}{-0.25in} 
\setlength{\oddsidemargin}{-0.25in} 
\setlength{\headsep}{8pt} 
%\setlength{\footskip}{10pt} 

\usepackage{geometry}
%\newgeometry{top=0.65in, bottom=18mm, left=15mm, right=15mm, outer=2in, heightrounded, marginparwidth=1.5in, marginparsep=0.15in}
\newgeometry{top=0.65in, bottom=16mm, left=15mm, right=15mm, heightrounded, marginparwidth=0in, marginparsep=0.15in}

\usepackage{fancyhdr}
\pagestyle{empty}
\pagestyle{fancy}
\fancyhead[RO,RE]{Maxie Dion Schmidt -- \today} 
\fancyhead[LO,LE]{}
\fancyheadoffset{0.005\textwidth} 

\setlength{\parindent}{0in}
\setlength{\parskip}{2cm} 

\renewcommand{\thefootnote}{\textbf{\Alph{footnote}}}
\makeatletter
\@addtoreset{footnote}{section}
\makeatother

%\usepackage{marginnote,todonotes}
%\colorlet{NBRefColor}{RoyalBlue!73} 
%\newcommand{\NBRef}[1]{
%     \todo[linecolor=green!85!white,backgroundcolor=orange!50!white,bordercolor=blue!30!black,textcolor=cyan!15!black,shadow,size=\small,fancyline]{
%     \color{NBRefColor}{\textbf{#1}
%     }
%     }
%}
\newcommand{\NBRef}[1]{}  

\newcommand{\SuccSim}[0]{\overset{_{\scriptsize{\blacktriangle}}}{\succsim}} 
\newcommand{\PrecSim}[0]{\overset{_{\scriptsize{\blacktriangle}}}{\precsim}} 
\renewcommand{\SuccSim}[0]{\ensuremath{\gg}} 
\renewcommand{\PrecSim}[0]{\ensuremath{\ll}} 

\renewcommand{\Re}{\operatorname{Re}}
\renewcommand{\Im}{\operatorname{Im}}

\input{glossaries-bibtex/PreambleGlossaries-Mertens}

\usepackage{tikz}
\usetikzlibrary{shapes,arrows}

\usepackage{enumitem} 

\allowdisplaybreaks 

\begin{document} 

\maketitle

\begin{abstract} 
The Mertens function, $M(x) := \sum_{n \leq x} \mu(n)$, is 
defined as the summatory function of the M\"obius function. 
The Mertens conjecture states that $|M(x)| < C \cdot \sqrt{x}$ for some absolute $C > 0$ for all 
$x \geq 1$. 
This classical conjecture has a well-known disproof due to 
Odlyzko and t\'{e} Riele. 
We prove the unboundedness of $|M(x)| / \sqrt{x}$ using new methods by showing that 
$$\limsup_{x \rightarrow \infty} \frac{|M(x)|}{\sqrt{x} \cdot (\log\log x)^{\frac{1}{2}}} > 0.$$ 
The new methods we draw upon connect formulas and recent 
Dirichlet generating function (or DGF) series expansions related to the canonically 
additive functions $\Omega(n)$ and $\omega(n)$. 
The connection between $M(x)$ and the distribution of these core additive functions 
we prove at the start of the article in the form of 
\[
M(x) = \sum_{k=1}^{x} (\omega + 1)^{-1}(k) \left[\pi\left(\left\lfloor \frac{x}{k} \right\rfloor\right) + 1\right],
\]
is an indispensible component to the proof. 
It also leads to regular properties of component sequences in the new formula for $M(x)$ that include 
generalizations of Erd\"os-Kac like theorems satisfied by the distributions of these special auxiliary 
sequences. 

\bigskip 
\noindent
\textbf{Keywords and Phrases:} {\it M\"obius function; Mertens function; 
                                    Dirichlet inverse function; Liouville lambda function; prime omega function; 
                                    prime counting function; Dirichlet generating function; 
                                    Erd\"os-Kac theorem; strongly additive functions. } \\ 
% 11-XX			Number theory
%    11A25  	Arithmetic functions; related numbers; inversion formulas
%    11Y70  	Values of arithmetic functions; tables
%    11-04  	Software, source code, etc. for problems pertaining to number theory
% 11Nxx		Multiplicative number theory
%    11N05  	Distribution of primes
%    11N37  	Asymptotic results on arithmetic functions
%    11N56  	Rate of growth of arithmetic functions
%    11N60  	Distribution functions associated with additive and positive multiplicative functions
%    11N64  	Other results on the distribution of values or the characterization of arithmetic functions
\textbf{Math Subject Classifications (MSC 2010):} {\it 11N37; 11A25; 11N60; 11N64; and 11-04. } 
\end{abstract} 

%\bigskip\hrule\bigskip

\newpage
%\section{Reference on abbreviations, special notation and other conventions} 
\label{Appendix_Glossary_NotationConvs}
     \vskip 0in
     \printglossary[type={symbols},
                    title={Glossary of special notation and conventions},
                    style={glossstyleSymbol},
                    nogroupskip=true]


%\newpage
%\setcounter{tocdepth}{2}
%\renewcommand{\contentsname}{Listing of major sections and topics} 
%\tableofcontents 

\newpage
\section{Introduction} 
\label{subSection_MertensMxClassical_Intro} 

\subsection{Definitions} 

We define the \emph{M\"obius function} to be the signed indicator function 
of the squarefree integers in the form of \cite[\seqnum{A008683}]{OEIS} 
\[
\mu(n) = \begin{cases} 
     1, & \text{if $n = 1$; } \\ 
     (-1)^{\omega(n)}, & \text{if $\omega(n) = \Omega(n)$ and $n \geq 2$; } \\ 
     0, & \text{otherwise.} 
     \end{cases} 
\]
The \emph{Mertens function}, or summatory function of $\mu(n)$, is defined on the 
positive integers as 
\begin{align*} 
M(x) & = \sum_{n \leq x} \mu(n), x \geq 1. 
\end{align*} 
The sequence of slow growing oscillatory values of this 
summatory function begins as follows \cite[\seqnum{A002321}]{OEIS}: 
\[
\{M(x)\}_{x \geq 1} = \{1, 0, -1, -1, -2, -1, -2, -2, -2, -1, -2, -2, -3, -2, 
     -1, -1, -2, -2, -3, -3, -2, -1, -2, \ldots\}. 
\] 
The Mertens function satisfies that $\sum_{n \leq x} M\left(\Floor{x}{n}\right) = 1$, and is related 
to the summatory function $L(x) := \sum_{n \leq x} \lambda(n)$ via the relation 
\cite{LEHMAN-1960} 
\[
L(x) = \sum_{d \leq \sqrt{x}} M\left(\Floor{x}{d^2}\right), x \geq 1. 
\]
Clearly, a positive integer $n \geq 1$ is \emph{squarefree}, or contains no divisors (other than one) which are 
squares, if and only if $\mu^2(n) = 1$. 
A related summatory function which counts the 
number of \emph{squarefree} integers $n \leq x$ satisfies 
\cite[\S 18.6]{HARDYWRIGHT} \cite[\seqnum{A013928}]{OEIS} 
\[ 
Q(x) = \sum_{n \leq x} \mu^2(n) \sim \frac{6x}{\pi^2} + O\left(\sqrt{x}\right). 
\] 
It is known that the asymptotic density of the positively versus negatively 
weighted sets of squarefree numbers characterized by the sign of the 
M\"obius function are in fact equal as $x \rightarrow \infty$: 
\[
\mu_{+}(x) = \frac{\#\{1 \leq n \leq x: \mu(n) = +1\}}{x} \overset{\mathbb{E}}{\sim} 
     \mu_{-}(x) = \frac{\#\{1 \leq n \leq x: \mu(n) = -1\}}{x} 
     \xrightarrow{x \rightarrow \infty} \frac{3}{\pi^2}. 
\]

\subsection{Properties} 

A conventional approach to evaluating the limiting asymptotic 
behavior of $M(x)$ for large $x \rightarrow \infty$ considers an 
inverse Mellin transformation of the reciprocal of the Riemann zeta function. 
In particular, since 
\[
\frac{1}{\zeta(s)} = \prod_{p} \left(1 - \frac{1}{p^s}\right) = 
     s \cdot \int_1^{\infty} \frac{M(x)}{x^{s+1}} dx, \Re(s) > 1, 
\]
we obtain that 
\[
M(x) = \lim_{T \rightarrow \infty}\ \frac{1}{2\pi\imath} \int_{T-\imath\infty}^{T+\imath\infty} 
     \frac{x^s}{s \cdot \zeta(s)} ds. 
\] 
The previous two representations lead us to the 
exact expression of $M(x)$ for any real $x > 0$ 
given by the next theorem. 
\nocite{TITCHMARSH} 

\begin{theorem}[Analytic Formula for $M(x)$, Titchmarsh] 
\label{theorem_MxMellinTransformInvFormula} 
Assuming the Riemann Hypothesis (RH), there exists an infinite sequence 
$\{T_k\}_{k \geq 1}$ satisfying $k \leq T_k \leq k+1$ for each $k$ 
such that for any real $x > 0$ 
\[
M(x) = \lim_{k \rightarrow \infty} 
     \sum_{\substack{\rho: \zeta(\rho) = 0 \\ |\Im(\rho)| < T_k}} 
     \frac{x^{\rho}}{\rho \cdot \zeta^{\prime}(\rho)} - 2 + 
     \sum_{n \geq 1} \frac{(-1)^{n-1}}{n \cdot (2n)! \zeta(2n+1)} 
     \left(\frac{2\pi}{x}\right)^{2n} + 
     \frac{\mu(x)}{2} \Iverson{x \in \mathbb{Z}^{+}}. 
\] 
\end{theorem} 

A historical unconditional bound on the Mertens function due to Walfisz (circa 1963) 
states that there is an absolute constant $C > 0$ such that 
$$M(x) \ll x \cdot \exp\left(-C \cdot \log^{\frac{3}{5}}(x) 
  (\log\log x)^{-\frac{3}{5}}\right).$$ 
Under the assumption of the RH, Soundararajan more recently proved new updated estimates 
bounding $M(x)$ from above for large $x$ in the following forms \cite{SOUND-MERTENS-ANNALS}: 
\begin{align*} 
M(x) & \ll \sqrt{x} \cdot \exp\left((\log x)^{\frac{1}{2}} (\log\log x)^{14}\right), \\ 
M(x) & = O\left(\sqrt{x} \cdot \exp\left( 
     (\log x)^{\frac{1}{2}} (\log\log x)^{\frac{5}{2}+\epsilon}\right)\right),\ 
     \forall \epsilon > 0. 
\end{align*} 

\subsection{Conjectures on boundedness and limiting behavior} 

The RH is equivalent to showing that 
$M(x) = O\left(x^{\frac{1}{2}+\epsilon}\right)$ for any 
$0 < \epsilon < \frac{1}{2}$. 
There is a rich history to the original statement of the \emph{Mertens conjecture} which 
asserts that 
\[ 
|M(x)| < C \cdot \sqrt{x},\ \text{ for some absolute constant $C > 0$. }
\] 
The conjecture was first verified by Mertens for $C = 1$ and all $x < 10000$. 
Since its beginnings in 1897, the Mertens conjecture has been disproven by computation 
of non-trivial simple zeta function zeros with comparitively small imaginary parts in a famous paper by 
Odlyzko and t\'{e} Riele \cite{ODLYZ-TRIELE}. 
Since the truth of the conjecture would have implied the RH, more recent attempts 
at bounding $M(x)$ naturally consider determining the rates at which the function 
$M(x) / \sqrt{x}$ grows with or without bound along infinite 
subsequences, e.g., considering the asymptotics of the function in the limit supremum and 
limit infimum senses. 

We cite that it is only known by computation 
that \cite[\cf \S 4.1]{PRIMEREC} 
\cite[\cf \seqnum{A051400}; \seqnum{A051401}]{OEIS} 
\[
\limsup_{x\rightarrow\infty} \frac{M(x)}{\sqrt{x}} > 1.060\ \qquad (\text{now } \geq 1.826054), 
\] 
and 
\[ 
\liminf_{x\rightarrow\infty} \frac{M(x)}{\sqrt{x}} < -1.009\ \qquad (\text{now } \leq -1.837625). 
\] 
Based on work by Odlyzyko and t\'{e} Riele, it seems probable that 
each of these limits should evaluate to $\pm \infty$, respectively 
\cite{ODLYZ-TRIELE,MREVISITED,ORDER-MERTENSFN,HURST-2017}. 
Extensive computational evidence has produced 
a conjecture due to Gonek that in fact the limiting behavior of 
$M(x)$ satisfies \cite{NG-MERTENS}
$$\limsup_{x \rightarrow \infty} \frac{|M(x)|}{\sqrt{x} \cdot (\log\log\log x)^{\frac{5}{4}}} = O(1).$$ 

\newpage 
\section{A concrete new approach to bounding $M(x)$ from below} 

\subsection{Summatory functions of Dirichlet convolutions of arithmetic functions} 

\begin{theorem}[Summatory functions of Dirichlet convolutions] 
\label{theorem_SummatoryFuncsOfDirCvls} 
Let $f,h: \mathbb{Z}^{+} \rightarrow \mathbb{C}$ be any arithmetic functions such that $f(1) \neq 0$. 
Suppose that $F(x) := \sum_{n \leq x} f(n)$ and $H(x) := \sum_{n \leq x} h(n)$ denote the summatory 
functions of $f$ and $h$, respectively, and that $F^{-1}(x) := \sum_{n \leq x} f^{-1}(n)$ 
denotes the summatory function of the 
Dirichlet inverse of $f$ for any $x \geq 1$. We have the following exact expressions for the 
summatory function of $f \ast h$ for all integers $x \geq 1$: 
\begin{align*} 
\pi_{f \ast h}(x) & := \sum_{n \leq x} \sum_{d|n} f(d) h(n/d) \\ 
     & \phantom{:}= \sum_{d \leq x} f(d) H\left(\Floor{x}{d}\right) \\ 
     & \phantom{:}= \sum_{k=1}^{x} H(k) \left[F\left(\Floor{x}{k}\right) - 
     F\left(\Floor{x}{k+1}\right)\right]. 
\end{align*} 
Moreover, for all $x \geq 1$ 
\begin{align*} 
H(x) & = \sum_{j=1}^{x} \pi_{f \ast h}(j) \left[F^{-1}\left(\Floor{x}{j}\right) - 
     F^{-1}\left(\Floor{x}{j+1}\right)\right] \\ 
     & = \sum_{k=1}^{x} f^{-1}(k) \cdot \pi_{f \ast h}\left(\Floor{x}{k}\right). 
\end{align*} 
\end{theorem} 

\begin{cor}[Convolutions arising from M\"obius inversion] 
\label{cor_CvlGAstMu} 
Suppose that $h$ is an arithmetic function such that 
$h(1) \neq 0$. Define the summatory function of 
the convolution of $h$ with $\mu$ by $\widetilde{H}(x) := \sum_{n \leq x} (h \ast \mu)(n)$. 
Then the Mertens function is expressed by the sum 
\[
M(x) = \sum_{k=1}^{x} \left(\sum_{j=\floor{\frac{x}{k+1}}+1}^{\floor{\frac{x}{k}}} h^{-1}(j)\right) 
     \widetilde{H}(k), \forall x \geq 1. 
\]
\end{cor} 

\begin{cor}[A motivating special case] 
\label{cor_Mx_gInvnPixk_formula} 
We have that for all $x \geq 1$ 
\begin{equation} 
\label{eqn_Mx_gInvnPixk_formula} 
M(x) = \sum_{k=1}^{x} (\omega+1)^{-1}(k) \left[\pi\left(\Floor{x}{k}\right) + 1\right]. 
\end{equation} 
\end{cor} 

\subsection{An exact expression for $M(x)$ in terms of strongly additive functions} 
\label{example_InvertingARecRelForMx_Intro}

Fix the notation for the Dirichlet invertible function $g(n) := \omega(n) + 1$ and define its 
inverse with respect to Dirichlet convolution by $g^{-1}(n) = (\omega+1)^{-1}(n)$. 
We can compute exactly that 
(see Table \ref{table_conjecture_Mertens_ginvSeq_approx_values} starting on page 
\pageref{table_conjecture_Mertens_ginvSeq_approx_values}) 
\[
\{g^{-1}(n)\}_{n \geq 1} = \{1, -2, -2, 2, -2, 5, -2, -2, 2, 5, -2, -7, -2, 5, 5, 2, -2, -7, -2, 
     -7, 5, 5, -2, 9, \ldots \}. 
\] 
There is not a simple meaningful 
direct recursion between the distinct values of $g^{-1}(n)$, except 
through auxiliary function sequences. 
The distribution of distinct sets of prime exponents is still regular since 
$\omega(n)$ and $\Omega(n)$ play a crucial role in the repitition of common values of 
$g^{-1}(n)$. 
The following observation is suggestive of the quasi-periodicity of the distribution of 
distinct values of this inverse function over $n \geq 2$: 

\begin{heuristic}[Symmetry in $g^{-1}(n)$ in the prime factorizations of $n$] 
\label{heuristic_SymmetryIngInvFuncs} 
Suppose that $n_1, n_2 \geq 2$ are such that their factorizations into distinct primes are 
given by $n_1 = p_1^{\alpha_1} \cdots p_r^{\alpha_r}$ and $n_2 = q_1^{\beta_1} \cdots q_r^{\beta_r}$ 
for $ = \omega(n_i) \geq 1$. 
If $\{\alpha_1, \ldots, \alpha_r\} \equiv \{\beta_1, \ldots, \beta_r\}$ as multisets of prime exponents, 
then $g^{-1}(n_1) = g^{-1}(n_2)$. For example, $g^{-1}$ has the same values on the squarefree integers 
with exactly one, two, three, and so on prime factors.  
\end{heuristic} 

\NBRef{A01-2020-04-26}
\begin{conjecture}[Characteristic properties of the inverse sequence] 
\label{lemma_gInv_MxExample} 
We have the following properties characterizing the 
Dirichlet inverse function $g^{-1}(n)$: 
\begin{itemize} 

\item[\textbf{(A)}] For all $n \geq 1$, $\operatorname{sgn}(g^{-1}(n)) = \lambda(n)$; 
\item[\textbf{(B)}] For all squarefree integers $n \geq 1$, we have that 
     \[
     |g^{-1}(n)| = \sum_{m=0}^{\omega(n)} \binom{\omega(n)}{m} \cdot m!; 
     \]
\item[\textbf{(C)}] If $n \geq 2$ and $\Omega(n) = k$, then 
     \[
     2 \leq |g^{-1}(n)| \leq \sum_{j=0}^{k} \binom{k}{j} \cdot j!. 
     \]
\end{itemize} 
\end{conjecture} 

We illustrate the conjecture clearly using the computation of initial values of 
this inverse sequence in 
Table \ref{table_conjecture_Mertens_ginvSeq_approx_values}. 
The signedness property in (A) is proved exactly in 
Proposition \ref{prop_SignageDirInvsOfPosBddArithmeticFuncs_v1}. 
A proof of (B) in fact follows from 
Lemma \ref{lemma_AnExactFormulaFor_gInvByMobiusInv_v1} 
stated on page \pageref{lemma_AnExactFormulaFor_gInvByMobiusInv_v1}. 
The realization that the beautiful and remarkably simple combinatorial form of property (B) 
in Conjecture \ref{lemma_gInv_MxExample} holds for all squarefree $n \geq 1$ 
motivates our pursuit of simpler formulas for the inverse functions $g^{-1}(n)$ 
through sums of auxiliary subsequences of arithmetic functions denoted by $C_k(n)$  
(see Section \ref{Section_InvFunc_PreciseExpsAndAsymptotics}). 
An exact expression for $g^{-1}(n)$ through a key semi-diagonal of these subsequences 
is given by 
\[
g^{-1}(n) = \lambda(n) \times \sum_{d|n} \mu^2\left(\frac{n}{d}\right) C_{\Omega(d)}(d), n \geq 1,  
\]
where the sequence $\lambda(n) C_{\Omega(n)}(n)$ has DGF $(P(s)+1)^{-1}$ for $\Re(s) > 1$. 

In Corollary \ref{cor_ExpectationFormulaAbsgInvn_v2}, we prove that 
\[
\mathbb{E}|g^{-1}(n)| \asymp (\log n)^2 \sqrt{\log\log n}, 
     \mathrm{\ as\ } n \rightarrow \infty. 
\]
The regularity and quasi-periodicity we have alluded to in the remarks above are actually 
quantifiable in so much as $|g^{-1}(n)|$ for $n \leq x$ 
tends to its average order with a non-central normal tendency 
depending on $x$ as $x \rightarrow \infty$. 
In Section \ref{Section_NewFormulasForgInvn}, 
we prove the next variant of an Erd\"os-Kac theorem like analog
for a component sequence $C_{\Omega(n)}(n)$. 
We have the following statement for 
$\mu_x(C) := \log\log x + \hat{a} - \frac{1}{2}\log\log\log x$, $\sigma_x(C) := \sqrt{\mu_x(C)}$, 
$\hat{a}$ an absolute constant, and any $y \in \mathbb{R}$ (see 
Corollary \ref{cor_CLT_VII}): 
\[
\frac{1}{x} \cdot \#\{2 \leq n \leq x: |g^{-1}(n)| - \mathbb{E}|g^{-1}(n)| \leq y\} = 
     \Phi\left(\frac{\frac{\pi^2}{6}y - \mu_x(C)}{\sigma_x(C)}\right) + 
     O\left(\frac{1}{\sqrt{\log\log x}}\right), 
     \mathrm{\ as\ } x \rightarrow \infty. 
\]
We also prove that (see Proposition \ref{prop_Mx_SBP_IntegralFormula}) 
\begin{equation} 
\label{eqn_Mx_gInvnPixk_formula_v2} 
M(x) = G^{-1}(x) + G^{-1}\left(\Floor{x}{2}\right) + 
     \sum_{k=1}^{\frac{x}{2}-1} G^{-1}(k) \left[ 
     \pi\left(\Floor{x}{k}\right) - \pi\left(\Floor{x}{k+1}\right) 
     \right]. 
\end{equation} 
This formula 
implies that we can establish new \emph{lower bounds} on $M(x)$ along large infinite subsequences 
by bounding appropriate estimates of the summatory function $G^{-1}(x)$. 
This take on the regularity of $|g^{-1}(n)|$ is imperative to our argument formally bounding the growth 
$G^{-1}(x)$ from below as $|G^{-1}(x)| \gg (\log x) \sqrt{\log\log x}$ as $x \rightarrow \infty$ 
(see Theorem \ref{theorem_GInvxLowerBoundByGEInvx_v1}). 

\subsection{Uniform asymptotics from certain bivariate counting DGFs} 

\begin{theorem}[Montgomery and Vaughan]
\label{theorem_HatPi_ExtInTermsOfGz} 
Recall that we have defined 
$$\widehat{\pi}_k(x) := \#\{n \leq x: \Omega(n)=k\}.$$ 
For $R < 2$ we have that uniformly for all $1 \leq k \leq R \cdot \log\log x$ 
\[
\widehat{\pi}_k(x) = \mathcal{G}\left(\frac{k-1}{\log\log x}\right) \frac{x}{\log x} 
     \frac{(\log\log x)^{k-1}}{(k-1)!} \left[1 + O_R\left(\frac{k}{(\log\log x)^2}\right)\right], 
\]
where 
\[
\mathcal{G}(z) := \frac{1}{\Gamma(z+1)} \times 
     \prod_p \left(1-\frac{z}{p}\right)^{-1} \left(1-\frac{1}{p}\right)^z, 0 \leq |z| < R. 
\]
\end{theorem} 

The proof of the next result is combinatorially motivated in so much as it interprets 
lower bounds on a key infinite product factor of $\mathcal{G}(z)$ defined in 
Theorem \ref{theorem_HatPi_ExtInTermsOfGz} 
as corresponding to an ordinary generating function of certain 
homogeneous symmetric polynomials involving the primes 
(see \eqref{eqn_pf_tag_hSymmPolysGF} in the proof of 
Theorem \ref{theorem_GFs_SymmFuncs_SumsOfRecipOfPowsOfPrimes}). 
This interpretation allows us to recover the 
following uniform lower bounds on $\widehat{\pi}_k(x)$ as $x \rightarrow \infty$: 

\begin{theorem}[Schmidt, 2020] 
\label{theorem_GFs_SymmFuncs_SumsOfRecipOfPowsOfPrimes} 
\label{cor_BoundsOnGz_FromMVBook_initial_stmt_v1} 
For all sufficiently large $x$ we have uniformly for $1 \leq k \leq \log\log x$ that 
\[
\widehat{\pi}_k(x) \gg 
     \frac{x^{3/4}}{\log x} \cdot 
     \frac{(\log\log x)^{k-1}}{(k-1)!} \times \left[1 + 
     O\left(\frac{k}{(\log\log x)^2}\right)\right]. 
\]
\end{theorem} 

\begin{remark} 
\label{remark_MV_NewDGFApplications} 
We emphasize the recency of the method demonstrated by 
Montgomery and Vaughan in constructing a proof of 
Theorem \ref{theorem_HatPi_ExtInTermsOfGz}. 
To the best of our knowledge, this textbook reference is 
one of the first clear-cut applications documenting something of a hybrid 
DGF-and-OGF approach to enumerating sequences of arithmetic functions 
and their summatory functions. 
This interpretation of certain bivariate DGFs 
offers a window into the best of both generating function series worlds. 
It combines the additivity 
implicit to the coefficients indexed by a formal power series variable formed by 
multiplication of these structures, while coordinating the distinct DGF-best 
property of the multiplicativity of distinct prime powers invoked 
by taking powers of a reciprocal Euler product. 
Another set of proofs constructed based on this type of hybrid power series enabling 
DGF is key in Section \ref{Section_NewFormulasForgInvn} 
when we prove an Erd\"os-Kac theorem like analog 
that holds for the component sequence $C_{\Omega(n)}(n)$, 
crucially related to $g^{-1}(n)$ by the 
results in Section \ref{Section_InvFunc_PreciseExpsAndAsymptotics}. 
\end{remark} 

\subsection{Cracking the classical unboundedness barrier} 

In Section \ref{Section_KeyApplications}, 
we are able to state what forms a bridge between the results 
we carefully prove up to that point the article. 
What we obtain at the conclusion of this last section 
is the next summary theorem that unconditionally 
resolves the classical question of the 
unboundedness of the scaled function Mertens function 
$q(x) := |M(x)| / \sqrt{x}$ in the limit supremum sense. 

\begin{theorem}[Unboundedness of the the Mertens function, $q(x)$] 
\label{cor_ThePipeDreamResult_v1} 
We have that 
\[
\limsup_{x \rightarrow \infty} \frac{|M(x)|}{\sqrt{x}} = +\infty. 
\]
\end{theorem} 

The proof of Theorem \ref{cor_ThePipeDreamResult_v1} is 
the main result we build up to in the article. 
It motivates all of our new constructions behind the additive function based 
sequences we employ to expand $M(x)$ via 
\eqref{eqn_Mx_gInvnPixk_formula} and 
\eqref{eqn_Mx_gInvnPixk_formula_v2}. 
This link relating our new formula for $M(x)$ 
to canonical additive functions and their 
distributions lends a recent distinguishing element to the 
success and characterization of the methods in our proof. 

\subsection{An overview of the core components to the proof} 

We offer the following initial step-by-step summary overview of the core components 
to our proof with the intention of making this new argument easier to parse: 

\begin{itemize} 

\item[\textbf{(1)}] We prove a matrix inversion formula relating the summatory 
           functions of an arithmetic function $f$ and its Dirichlet inverse $f^{-1}$ (for $f(1) \neq 0$). 
           See Theorem \ref{theorem_SummatoryFuncsOfDirCvls} in 
           Section \ref{Section_PrelimProofs_Config}.  
\item[\textbf{(2)}] This crucial step provides us with an exact formula for $M(x)$ in terms of 
           the prime counting function $\pi(x)$, and the 
           Dirichlet inverse of the shifted additive function $g(n) := \omega(n) + 1$. This 
           formula is stated in \eqref{eqn_Mx_gInvnPixk_formula} 
           (see Proposition \ref{prop_Mx_SBP_IntegralFormula}). 
\item[\textbf{(3)}] We tighten bounds from a less classical result proved in 
            \cite[\S 7]{MV} providing uniform asymptotic formulas for lower bounds on the  
           summatory functions, $\widehat{\pi}_k(x)$, large $x \gg e$ and 
           $1 \leq k \leq \log\log x$ 
           (see Theorem \ref{theorem_GFs_SymmFuncs_SumsOfRecipOfPowsOfPrimes}). 
           This allows us to eventually approximate the magnitude of the summatory function 
           \[
           L(x) := \sum_{n \leq x} \lambda(n) \asymp 
                \sum_{k=1}^{\log\log x} (-1)^k \widehat{\pi}_k(x), 
                \mathrm{\ as\ } x \rightarrow \infty, 
           \]
           well from below (see the proof of 
           Theorem \ref{theorem_GInvxLowerBoundByGEInvx_v1}; and 
           Table \ref{table_LAstxSummatoryFuncCompsWithExact_v2} starting on page 
           \pageref{table_LAstxSummatoryFuncCompsWithExact_v2}). 
\item[\textbf{(4)}] In 
           Section \ref{Section_InvFunc_PreciseExpsAndAsymptotics}. 
           we relate $g^{-1}(n)$ to a subsequence of recursively-defined auxiliary functions, $C_k(n)$, 
           that respectively express multiple $k$-convolutions of $\omega(n)$ with itself for 
           $1 \leq k \leq \Omega(n)$ 
           (see Lemma \ref{lemma_AnExactFormulaFor_gInvByMobiusInv_v1} and 
           Lemma \ref{lemma_AbsValueOf_gInvn_FornSquareFree_v1}). 
\item[\textbf{(5)}] In Section \ref{Section_NewFormulasForgInvn}, 
           we prove new expectation formulas for $|g^{-1}(n)|$ and the related component sequence 
           $C_{\Omega(n)}(n)$ by first proving an Erd\"os-Kac like theorem satisfied by $C_{\Omega(n)}(n)$. 
           This allows us to prove asymptotic lower bounds on 
           $|G^{-1}(x)| \gg (\log x) \sqrt{\log\log x}$ when $x$ is large and such that $G^{-1}(x) \neq 0$ in 
           Section \ref{Section_KeyApplications}. 
\item[\textbf{(6)}] When we return to step \textbf{(2)} 
           with our new lower bounds at hand, we are led to a new unconditional proof of the 
           unboundedness of $\frac{|M(x)|}{\sqrt{x}}$ 
           along an exponentially very large increasing infinite subsequence of positive natural numbers 
           (see Section \ref{subSection_TheCoreResultProof}). 
           
\end{itemize} 

\newpage 
\section{Preliminary proofs of new results} 
\label{Section_PrelimProofs_Config} 

\subsection{Establishing the summatory function properties and inversion identities} 

We will offer a proof of Theorem \ref{theorem_SummatoryFuncsOfDirCvls} 
suggested by an intuitive construction through matrix methods. 
Related results on summations of Dirichlet convolutions appear in 
\cite[\S 2.14; \S 3.10; \S 3.12; \cf \S 4.9, p.\ 95]{APOSTOLANUMT}. 

\begin{proof}[Proof of Theorem \ref{theorem_SummatoryFuncsOfDirCvls}] 
\label{proofOf_theorem_SummatoryFuncsOfDirCvls} 
Let $h,g$ be arithmetic functions such that $g(1) \neq 0$. 
Denote the summatory functions of $h$ and $g$, 
respectively, by $H(x) = \sum_{n \leq x} h(n)$ and $G(x) = \sum_{n \leq x} g(n)$. 
We define $\pi_{g \ast h}(x)$ to be the summatory function of the 
Dirichlet convolution of $g$ with $h$. 
We have that the following formulas hold for all $x \geq 1$: 
\begin{align} 
\notag 
\pi_{g \ast h}(x) & := \sum_{n=1}^{x} \sum_{d|n} g(n) h(n/d) = \sum_{d=1}^x g(d) H\left(\floor{\frac{x}{d}}\right) \\ 
\label{eqn_proof_tag_PigAsthx_ExactSummationFormula_exp_v2} 
     & = \sum_{i=1}^x \left[G\left(\floor{\frac{x}{i}}\right) - G\left(\floor{\frac{x}{i+1}}\right)\right] H(i). 
\end{align} 
The first formula above is well known. The second formula is justified directly using 
summation by parts as \cite[\S 2.10(ii)]{NISTHB} 
\begin{align*} 
\pi_{g \ast h}(x) & = \sum_{d=1}^x h(d) G\left(\floor{\frac{x}{d}}\right) \\ 
     & = \sum_{i \leq x} \left(\sum_{j \leq i} h(j)\right) \times 
     \left[G\left(\floor{\frac{x}{i}}\right) - 
     G\left(\floor{\frac{x}{i+1}}\right)\right]. 
\end{align*} 
We next form the invertible matrix of coefficients associated with this linear system defining $H(j)$ for all 
$1 \leq j \leq x$ in \eqref{eqn_proof_tag_PigAsthx_ExactSummationFormula_exp_v2} by setting 
\[
g_{x,j} := G\left(\floor{\frac{x}{j}}\right) - G\left(\floor{\frac{x}{j+1}}\right) \equiv G_{x,j} - G_{x,j+1}, 
\] 
where 
\[
G_{x,j} := G\left(\Floor{x}{j}\right), 1 \leq j \leq x. 
\]
Since $g_{x,x} = G(1) = g(1)$ and $g_{x,j} = 0$ for all $j > x$, 
the matrix we must invert in this problem is lower triangular with a non-zero 
constant on its diagonals, and is hence invertible. 
Moreover, if we let $\hat{G} := (G_{x,j})$, then this matrix is 
expressed by applying an invertible shift operation as 
\[
(g_{x,j}) = \hat{G} (I - U^{T}). 
\]
Here, $U$ is a square matrix with sufficiently large finite dimensions 
whose $(i,j)^{th}$ entries are defined by $(U)_{i,j} = \delta_{i+1,j}$ such that 
\[
\left[(I - U^T)^{-1}\right]_{i,j} = \Iverson{j \leq i}. 
\]
Observe that 
\[
\Floor{x}{j} - \Floor{x-1}{j} = \begin{cases} 
     1, & \text{ if $j|x$; } \\ 
     0, & \text{ otherwise. } 
     \end{cases} 
\] 
The previous property implies that 
\begin{equation} 
\label{eqn_proof_tag_FloorFuncDiffsOfSummatoryFuncs_v2} 
G\left(\floor{\frac{x}{j}}\right) - G\left(\floor{\frac{x-1}{j}}\right) = 
     \begin{cases} 
     g\left(\frac{x}{j}\right), & \text{ if $j | x$; } \\ 
     0, & \text{ otherwise. } 
     \end{cases}
\end{equation} 
We use the last property in \eqref{eqn_proof_tag_FloorFuncDiffsOfSummatoryFuncs_v2} 
to shift the matrix $\hat{G}$, and then invert the result to obtain a matrix involving the 
Dirichlet inverse of $g$ in the following form: 
\begin{align*} 
\left[(I-U^{T}) \hat{G}\right]^{-1} & = \left(g\left(\frac{x}{j}\right) \Iverson{j|x}\right)^{-1} = 
     \left(g^{-1}\left(\frac{x}{j}\right) \Iverson{j|x}\right). 
\end{align*} 
Our target matrix in the inversion problem is defined by 
$$(g_{x,j}) = (I-U^{T}) \left(g\left(\frac{x}{j}\right) \Iverson{j|x}\right) (I-U^{T})^{-1}.$$
We can express its inverse by a similarity transformation conjugated by shift operators as follows: 
\begin{align*} 
(g_{x,j})^{-1} & = (I-U^{T})^{-1} \left(g^{-1}\left(\frac{x}{j}\right) \Iverson{j|x}\right) (I-U^{T}) \\ 
     & = \left(\sum_{k=1}^{\floor{\frac{x}{j}}} g^{-1}(k)\right) (I-U^{T}) \\ 
     & = \left(\sum_{k=1}^{\floor{\frac{x}{j}}} g^{-1}(k) - \sum_{k=1}^{\floor{\frac{x}{j+1}}} g^{-1}(k)\right). 
\end{align*} 
Hence, the summatory function $H(x)$ is given exactly for any $x \geq 1$ 
by a vector product with the inverse matrix from the previous equation in the form of 
\begin{align*} 
H(x) & = \sum_{k=1}^x \left(\sum_{j=\floor{\frac{x}{k+1}}+1}^{\floor{\frac{x}{k}}} g^{-1}(j)\right) 
     \cdot \pi_{g \ast h}(k). 
\end{align*} 
We can prove an inversion formula providing the coefficients of the summatory function 
$G^{-1}(i)$ for $1 \leq i \leq x$ given 
by the last equation by adapting our argument to prove 
\eqref{eqn_proof_tag_PigAsthx_ExactSummationFormula_exp_v2} above. 
This leads to the following identity: 
\[
H(x) = \sum_{k=1}^{x} g^{-1}(x) \cdot \pi_{g \ast h}\left(\Floor{x}{k}\right). 
     \qedhere 
\]
\end{proof} 

\subsection{Proving the characteristic signedness property of $g^{-1}(n)$} 

Let $\chi_{\mathbb{P}}$ denote the characteristic function of the primes, let 
$\varepsilon(n) = \delta_{n,1}$ be the multiplicative identity with respect to Dirichlet convolution, 
and denote by $\omega(n)$ the strongly additive function that counts the number of 
distinct prime factors of $n$. Then we can easily prove using DGFs that 
\begin{equation}
\label{eqn_AntiqueDivisorSumIdent} 
\chi_{\mathbb{P}} + \varepsilon = (\omega + 1) \ast \mu. 
\end{equation} 
When combined with Corollary \ref{cor_CvlGAstMu} 
this convolution identity yields the exact 
formula for $M(x)$ stated in \eqref{eqn_Mx_gInvnPixk_formula} of 
Corollary \ref{cor_Mx_gInvnPixk_formula}. 

\begin{prop}[The signedness property of $g^{-1}(n)$]
\label{prop_SignageDirInvsOfPosBddArithmeticFuncs_v1} 
Let the operator 
$\operatorname{sgn}(h(n)) = \frac{h(n)}{|h(n)| + \Iverson{h(n) = 0}} \in \{0, \pm 1\}$ denote the sign 
of the arithmetic function $h$ at integers $n \geq 1$. 
For the Dirichlet invertible function $g(n) := \omega(n) + 1$, 
we have that $\operatorname{sgn}(g^{-1}(n)) = \lambda(n)$ for all $n \geq 1$. 
\NBRef{A02-2020-04-26}
\end{prop} 
\begin{proof} 
The function $D_f(s) := \sum_{n \geq 1} f(n) n^{-s}$ denotes the 
\emph{Dirichlet generating function} (DGF) of any 
arithmetic function $f(n)$ which is convergent for all $s \in \mathbb{C}$ satisfying 
$\Re(s) > \sigma_f$ for $\sigma_f$ the abcissa of convergence of the series. 
Recall that $D_1(s) = \zeta(s)$, $D_{\mu}(s) = 1 / \zeta(s)$ and $D_{\omega}(s) = P(s) \zeta(s)$ for 
$\Re(s) > 1$. 
Then by \eqref{eqn_AntiqueDivisorSumIdent} and the known property that the DGF of $f^{-1}(n)$ is 
the reciprocal of the DGF of any arithmetic function $f$ such that $f(1) \neq 0$ 
(e.g., this relation between the DGFs of these two functions holds whenever $f^{-1}$ exists), 
we have for all $\Re(s) > 1$ that 
\begin{align} 
\label{eqn_DGF_of_gInvn} 
D_{(\omega+1)^{-1}}(s) = \frac{1}{(P(s)+1) \zeta(s)}. 
\end{align} 
It follows that $(\omega + 1)^{-1}(n) = (h^{-1} \ast \mu)(n)$ when we take 
$h := \chi_{\mathbb{P}} + \varepsilon$. 
We first show that $\operatorname{sgn}(h^{-1}) = \lambda$. 
This observation implies 
that $\operatorname{sgn}(h^{-1} \ast \mu) = \lambda$. The remainder of the proof fills in the 
precise details needed to make our claims and intuition rigorous. 

By the recurrence relation that defines the Dirichlet inverse function of any 
arithmetic function $h$ such that $h(1) = 1$, we have that \cite[\S 2.7]{APOSTOLANUMT} 
\begin{equation} 
\label{eqn_proof_tag_hInvn_ExactRecFormula_v1}
h^{-1}(n) = \begin{cases} 
            1, & n = 1; \\ 
            -\sum\limits_{\substack{d|n \\ d>1}} h(d) h^{-1}(n/d), & n \geq 2. 
            \end{cases} 
\end{equation} 
For $n \geq 2$, the summands in \eqref{eqn_proof_tag_hInvn_ExactRecFormula_v1} 
can be simply indexed over the primes $p|n$ given our definition of $h$ from above. 
This observation yields that we can inductively 
unfold these sums into nested divisor sums provided the depth of the 
expanded divisor sums does not exceed the 
capacity to index summations over the primes dividing $n$. Namely, notice that for $n \geq 2$ 
\begin{align*} 
h^{-1}(n) & = -\sum_{p|n} h^{-1}\left(\frac{n}{p}\right), && \text{\ if\ } \Omega(n) \geq 1 \\ 
     & = \sum_{p_1|n} \sum_{p_2|\frac{n}{p_1}} h^{-1}\left(\frac{n}{p_1p_2}\right), && \text{\ if\ } \Omega(n) \geq 2 \\ 
     & = -\sum_{p_1|n} \sum_{p_2|\frac{n}{p_1}} \sum_{p_3|\frac{n}{p_1p_2}} h^{-1}\left(\frac{n}{p_1p_2p_3}\right), 
     && \text{\ if\ } \Omega(n) \geq 3. 
\end{align*} 
Then by induction with $h^{-1}(1) = h(1) = 1$, we expand these 
nested divisor sums as above to the maximal possible depth as 
\begin{equation} 
\label{eqn_proof_tag_hInvn_ExactNestedSumFormula_v2} 
\lambda(n) \cdot h^{-1}(n) = \sum_{p_1|n} \sum_{p_2|\frac{n}{p_1}} \times \cdots \times 
     \sum_{p_{\Omega(n)}|\frac{n}{p_1p_2 \cdots p_{\Omega(n)-1}}} 1, n \geq 2. 
\end{equation} 
In fact, by a combinatorial argument we recover exactly that 
\begin{equation} 
\label{eqn_proof_tag_hInvn_ExactNestedSumFormula_CombInterpetIdent_v3} 
h^{-1}(n) = \lambda(n) (\Omega(n))! \times \prod_{p^{\alpha} || n} \frac{1}{\alpha!}. 
\end{equation} 
The last two expansions imply that the following property holds for all $n \geq 1$: 
\begin{equation} 
\notag 
\operatorname{sgn}(h^{-1}(n)) = \lambda(n). 
\end{equation} 
Since $\lambda$ is completely multiplicative we have that 
$\lambda\left(\frac{n}{d}\right) \lambda(d) = \lambda(n)$ for all divisors 
$d|n$ when $n \geq 1$. We also know that $\mu(n) = \lambda(n)$ whenever $n$ is squarefree, 
so that we obtain the following result: 
\[
g^{-1}(n) = (h^{-1} \ast \mu)(n) = \lambda(n) \times \sum_{d|n} \mu^2\left(\frac{n}{d}\right) |h^{-1}(n)|, n \geq 1. 
     \qedhere 
\]
\end{proof} 

\subsection{Statements of known limiting asymptotics} 
\label{subSection_OtherFactsAndResults} 

\begin{theorem}[Mertens theorem]
\label{theorem_Mertens_theorem} 
For all $x \geq 2$ we have that 
\[
P_1(x) := \sum_{p \leq x} \frac{1}{p} = \log\log x + B + o(1), 
     \mathrm{\ as\ } x \rightarrow \infty, 
\]
where 
$B \approx 0.2614972128476427837554$ 
is an absolute constant.
\end{theorem} 

\begin{cor}[Product form of Mertens theorem] 
\label{lemma_Gz_productTermV2} 
We have that for all sufficiently large $x \gg 2$ 
\[
\prod_{p \leq x} \left(1 - \frac{1}{p}\right) = \frac{e^{-\gamma}}{\log x}\left( 
     1 + o(1)\right), \mathrm{\ as\ } x \rightarrow \infty. 
\]
Hence, for any real $z$ we obtain that 
\[
\prod_{p \leq x} \left(1 - \frac{1}{p}\right)^{z} \sim 
     \frac{e^{-\gamma z}}{(\log x)^{z}}, \mathrm{\ as\ } x \rightarrow \infty. 
\]
\end{cor} 

Proofs of Theorem \ref{theorem_Mertens_theorem} and 
Corollary \ref{lemma_Gz_productTermV2} are given in 
\cite[\S 22.7; \S 22.8]{HARDYWRIGHT}. 

\begin{facts}[Exponential integrals and the incomplete gamma function] 
\label{facts_ExpIntIncGammaFuncs} 
\begin{subequations}
Two variants of the \emph{exponential integral function} are defined by the 
integral next representations \cite[\S 8.19]{NISTHB} \cite[\S 3.3]{INCGAMMA-BOOK}. 
\begin{align*} 
\operatorname{Ei}(x) & := \int_{-x}^{\infty} \frac{e^{-t}}{t} dt, x \in \mathbb{R} \\ 
E_1(z) & := \int_1^{\infty} \frac{e^{-tz}}{t} dt, \Re(z) \geq 0 
\end{align*} 
These functions are related by $\operatorname{Ei}(-kz) = -E_1(kz)$ for real $k, z > 0$. 
We have the following inequalities providing 
quasi-polynomial upper and lower bounds on $\operatorname{Ei}(\pm x)$ 
for all real $x > 0$: 
\begin{align} 
\label{eqn_facts_ExpIntegralEiInequalities_v12} 
\gamma + \log x - x \leq & \operatorname{Ei}(-x) \leq \gamma + \log x - x + \frac{x^2}{4}, \\ 
\notag 
1 + \gamma + \log x -\frac{3}{4} x \leq & \operatorname{Ei}(x) \phantom{-} \leq 
     1 + \gamma + \log x -\frac{3}{4} x + \frac{11}{36} x^2. 
\end{align}
The (upper) \emph{incomplete gamma function} is defined by \cite[\S 8.4]{NISTHB} 
\[
\Gamma(s, x) = \int_{x}^{\infty} t^{s-1} e^{-t} dt, \Re(s) > 0. 
\]
The following properties of $\Gamma(s, x)$ hold: 
\begin{align} 
\label{eqn_IncompleteGamma_PropA} 
\Gamma(s, x) & = (s-1)! \cdot e^{-x} \times \sum_{k=0}^{s-1} \frac{x^k}{k!}, s \in \mathbb{Z}^{+}, x > 0, \\ 
\label{eqn_IncompleteGamma_PropB} 
\Gamma(s, x) & \sim x^{s-1} \cdot e^{-x}, s > 0, \mathrm{\ as\ } x \rightarrow \infty. 
\end{align}
\end{subequations}
\end{facts} 

\newpage 
\section{Components to the asymptotic analysis of lower bounds for 
         sums of arithmetic functions weighted by $\lambda(n)$} 
\label{Section_MVCh7_GzBounds} 

\subsection{A discussion of the results proved by Montgomery and Vaughan} 
\label{subSection_MVPrereqResultStmts} 

\begin{remark}[Intuition and constructions behind the proof of Theorem \ref{theorem_HatPi_ExtInTermsOfGz}] 
\label{remark_intuitionConstrIn_theorem_HatPi_ExtInTermsOfGz} 
For $|z| < 2$ and $\Re(s) > 1$, let 
\begin{equation} 
\label{eqn_IntuitionMVThm_FszFuncDef_v1} 
F(s, z) := \prod_{p} \left(1 - \frac{z}{p^s}\right)^{-1} \left(1 - \frac{1}{p^s}\right)^{z}, 
\end{equation} 
and define the DGF coefficients, $a_z(n)$ for $n \geq 1$, by the product 
\[
\zeta(s)^{z} \cdot F(s, z) := \sum_{n \geq 1} \frac{a_z(n)}{n^s}, \Re(s) > 1. 
\]
Suppose that $A_z(x) := \sum_{n \leq x} a_z(n)$ for $x \geq 1$. We obtain the next 
generating function like identity in $z$ enumerating the $\widehat{\pi}_k(x)$ for 
$1 \leq k < 2 \log\log x$. 
\begin{equation} 
\label{eqn_remark_MV_AzxCoeffFormlaIntegral_v1} 
A_z(x) = \sum_{n \leq x} z^{\Omega(n)} = \sum_{\substack{0 \leq k \leq \log_2(x)}} \widehat{\pi}_k(x) z^k 
\end{equation} 
Thus for $r < 2$, by Cauchy's integral formula we have 
\[
\widehat{\pi}_k(x) = \frac{1}{2\pi\imath} \int_{|v|=r} \frac{A_v(x)}{v^{k+1}} dv. 
\]
Selecting $r := \frac{k-1}{\log\log x}$ for $1 \leq k < 2 \log\log x$ 
leads to the uniform asymptotic formulas for $\widehat{\pi}_k(x)$ given in 
Theorem \ref{theorem_HatPi_ExtInTermsOfGz}. 
Montgomery and Vaughan then consider individual analysis of the main and error 
terms for $A_z(x)$ to prove that 
\[
\widehat{\pi}_k(x) = \mathcal{G}\left(\frac{k-1}{\log\log x}\right) \frac{x}{\log x} \cdot 
     \frac{(\log\log x)^{k-1}}{(k-1)!} \left[1 + O\left(\frac{k}{(\log\log x)^2}\right)\right]. 
\]
We will require estimates of $A_{-z}(x)$ from below to form summatory functions 
that weight the terms of $\lambda(n)$ in our new formulas derived in the next sections. 
\end{remark} 

\subsection{New uniform asymptotics based on refinements of Theorem \ref{theorem_HatPi_ExtInTermsOfGz}} 
\label{subSection_PartialPrimeProducts_Proofs} 

\begin{prop} 
\label{cor_PartialSumsOfReciprocalsOfPrimePowers} 
For real $s \geq 1$, let 
\[
P_s(x) := \sum_{p \leq x} p^{-s}, x \geq 2. 
\]
When $s := 1$, we have the asymptotic formula from Mertens theorem 
(see Theorem \ref{theorem_Mertens_theorem}). 
For all integers $s \geq 2$ 
there are absolutely defined quasi-polynomial bounding functions 
$\gamma_0(s, x)$ and $\gamma_1(s, x)$ in $s$ and $x$ such that 
\[
\gamma_0(s, x) + o(1) \leq P_s(x) \leq \gamma_1(s, x) + o(1), \mathrm{\ as\ } x \rightarrow \infty. 
\] 
It suffices to define the bounds in the previous equation by the functions 
\begin{align*} 
\gamma_0(s, x) & = s\log\left(\frac{\log x}{\log 2}\right) - 
     s(s-1) \log\left(\frac{x}{2}\right) - 
     \frac{1}{4} s(s-1)^2 \log^2(2), \\ 
\gamma_1(s, x) & = s\log\left(\frac{\log x}{\log 2}\right) - s(s-1) \log\left(\frac{x}{2}\right) + 
     \frac{1}{4} s(s-1)^2 \log^2(x). 
\end{align*}
\end{prop} 
\NBRef{A05-2020-04-26} 
\begin{proof} 
Let $s > 1$ be real-valued. 
By Abel summation with the summatory function 
$A(x) = \pi(x) \sim \frac{x}{\log x}$, and where 
our target smooth function is $f(t) = t^{-s}$ with 
$f^{\prime}(t) = -s \cdot t^{-(s+1)}$, we obtain that 
\begin{align*} 
P_s(x) & \sim \frac{1}{x^s \cdot \log x} + s \times \int_2^{x} \frac{dt}{t^s \log t} \\ 
     & = \operatorname{Ei}(-(s-1) \log x) - \operatorname{Ei}(-(s-1) \log 2) + o(1), 
     \mathrm{\ as\ } x \rightarrow \infty. 
\end{align*} 
Now using the inequalities in Facts \ref{facts_ExpIntIncGammaFuncs}, we obtain that the 
difference of the exponential integral functions in the previous equation 
is respectively bounded below and above by 
\begin{align*} 
\frac{P_s(x)}{s} & \geq \log\left(\frac{\log x}{\log 2}\right) - (s-1) \log\left(\frac{x}{2}\right) - 
     \frac{1}{4} (s-1)^2 \log^2(2) + o(1) \\ 
\frac{P_s(x)}{s} & \leq \log\left(\frac{\log x}{\log 2}\right) - (s-1) \log\left(\frac{x}{2}\right) + 
     \frac{1}{4} (s-1)^2 \log^2(x) + o(1). 
     \qedhere 
\end{align*} 
\end{proof} 

The utility to the quasi-logarithmic bounds tending to 
infinity as $x \rightarrow \infty$ stated in 
Proposition \ref{cor_PartialSumsOfReciprocalsOfPrimePowers} 
will become apparent when we take the exponential of sums of the 
functions $P_j(x)$ for $j \geq 2$ in order to form a lower bound on 
$\mathcal{G}(-z)$ for $z := \frac{k-1}{\log\log x}$ in the 
next subsection. 

\subsubsection{The proof of Theorem \ref{theorem_GFs_SymmFuncs_SumsOfRecipOfPowsOfPrimes}} 

We will first prove the stated form of the lower bound on 
$\mathcal{G}(-z)$ for $z := \frac{k-1}{\log\log x}$. 
Then we will discuss the technical adaptations to Montgomery and Vaughan's proof of 
Theorem \ref{theorem_HatPi_ExtInTermsOfGz} 
below to rigorously prove that the new asymptotic lower bounds on $\widehat{\pi}_k(x)$ that hold uniformly for all 
$1 \leq k \leq \log\log x$. 

\NBRef{A06-2020-04-26} 
\begin{lemma} 
\label{lemma_theorem_HatPi_ExtInTermsOfGz} 
For sufficiently large $x > e$ and $1 \leq k \leq \log\log x$, we have that 
\[
\left\lvert 
     \mathcal{G}\left(\frac{1-k}{\log\log x}\right) \right\rvert 
     \gg x^{-\frac{1}{4}}. 
\]
\end{lemma} 
\begin{proof}
\label{proofOf_theorem_GFs_SymmFuncs_SumsOfRecipOfPowsOfPrimes} 
For $-2 < z < 2$ and integers $x \geq 2$, 
the right-hand-side of the following product is finite: 
\[
\widehat{P}(z, x) := \prod_{p \leq x} \left(1 - \frac{z}{p}\right)^{-1}. 
\]
For fixed $x \geq 2$ let 
\[
\mathbb{P}_x := \left\{n \in \mathbb{Z}^{+}: \mathrm{ all\ prime\ divisors\ } 
     p|n \mathrm{\ satisfy\ } p \leq x\right\}. 
\]
Then we can see that for $x \geq 2$ 
\begin{equation} 
\label{eqn_proof_tag_PHatFiniteTruncProdFactorOfGz_v2} 
\prod_{p \leq x} \left(1 - \frac{z}{p^s}\right)^{-1} = \sum_{n \in \mathbb{P}_x} 
     \frac{z^{\Omega(n)}}{n^s}. 
\end{equation} 
By extending the argument in the proof given in 
\cite[\S 7.4]{MV}, we have that 
\[
A_{-z}(x) := \sum_{n \leq x} \lambda(n) z^{\Omega(n)} = 
     \sum_{0 \leq k \leq \log_2(x)} \widehat{\pi}_k(x) (-z)^k, 
\] 
Let $a_n(z, x)$ be defined as the coefficients of the DGF 
\[
\widehat{P}(z, x) =: \sum_{n \geq 1} \frac{a_n(z, x)}{n^s}. 
\]
We have argued that 
\[
\sum_{n \leq x} a_n(-z, x) = 
     \sum_{k=0}^{\log_2(x)} \widehat{\pi}_k(x) (-z)^k + 
     \sum_{k > \log_2(x)} e_k(x) (-z)^{k}. 
\]
This assertion is correct since the products of all non-negative integral powers of the 
primes $p \leq x$ (counting multiplicity) 
generate the integers $\{1 \leq n \leq x\}$ as a subset. 
Thus we capture all of the relevant terms needed to express 
$(-1)^{k} \cdot \widehat{\pi}_k(x)$ 
via the Cauchy integral formula representation over $A_{-z}(x)$ by 
replacing the corresponding infinite product terms with 
$\widehat{P}(-z, x)$ in the definition of $\mathcal{G}(-z)$. 

Now we argue that 
\[
\mathcal{G}(-z) \gg \prod_{p \leq x} \left(1 + \frac{z}{p}\right)^{-1} 
     \left(1 - \frac{1}{p}\right)^{-z}, 0 \leq z < 1, x \geq 2. 
\]
For $0 \leq z < 1$ and $x \geq 2$, we see that 
\begin{align*} 
\mathcal{G}(-z) & = \exp\left(-\sum_p \left[\log\left(1 + \frac{z}{p}\right) + 
     z \cdot \log\left(1 - \frac{1}{p}\right)\right]\right) \\ 
     & \gg 
     \exp\left(-z \times \sum_{p > x} \left[
     \log\left(1 - \frac{1}{p}\right) + \frac{1}{p}\right] - 
     \sum_{p \leq x} \left[\log\left(1 + \frac{z}{p}\right) + 
     z \cdot \log\left(1 - \frac{1}{p}\right)\right]\right) \\ 
     & \gg_z \widehat{P}(-z, x), \mathrm{\ as\ } x \rightarrow \infty, 
\end{align*} 
where the \emph{Mertens constant} $B$ is defined exactly by the prime sum 
\cite[\S 22.8]{HARDYWRIGHT} 
\[
B := \gamma + \sum_p \left[\log\left(1-\frac{1}{p}\right) + \frac{1}{p}\right]. 
\]
Next, we have for all integers $0 \leq k \leq m < \infty$, and any sequence 
$\{f(n)\}_{n \geq 1}$ with sufficiently bounded partial power sums, that 
\cite[\S 2]{MACDONALD-SYMFUNCS} 
\begin{equation} 
\label{eqn_pf_tag_hSymmPolysGF} 
[z^k] \prod_{1 \leq i \leq m} (1-f(i) z)^{-1} = [z^k] \exp\left(\sum_{j \geq 1} 
     \left(\sum_{i=1}^m f(i)^j\right) \frac{z^j}{j}\right), |z| < 1. 
\end{equation} 
In our case, $f(i)$ denotes the reciprocal of the 
$i^{th}$ prime in the generating function expansion of 
\eqref{eqn_pf_tag_hSymmPolysGF}. 
It follows from Proposition \ref{cor_PartialSumsOfReciprocalsOfPrimePowers} that 
for any real $0 \leq z < 1$ we obtain 
\begin{align} 
\notag 
\log\left[\prod_{p \leq x} \left(1+\frac{z}{p}\right)^{-1}\right] & \geq -(\log\log x + B) z + 
     \sum_{j \geq 0} \left[\log\left(\frac{\log x}{\log 2}\right) - 
     (2j+1) \log\left(\frac{x}{2}\right) - (2j+1)^2 \frac{\log^2 2}{4}\right] z^{2j+2} \\ 
\notag 
     & \phantom{\geq -(\log\log x + B) z\ } - 
     \sum_{j \geq 0} \left[\log\left(\frac{\log x}{\log 2}\right) - 
     (2j+2) \log\left(\frac{x}{2}\right) + (2j+2)^2 \frac{\log^2 x}{4}\right] z^{2j+3} \\ 
\notag 
     & = -(\log\log x + B) z + \sum_{j \geq 0} \left[ 
     \log\left(\frac{\log x}{\log 2}\right) - 
     (j+1) \log\left(\frac{x}{2}\right)\right] (-z)^{j+2} \\ 
\notag 
     & \phantom{= -(B + \log\log x) z\ } - 
     \frac{1}{4} \times \sum_{j \geq 0} \left[ 
     (\log 2)^2 (2j+1)^2 z^{2j+2} + (\log x)^2 (2j+2)^2 z^{2j+3} 
     \right] \\ 
\notag 
     & = -(\log\log x + B) z + 
     \log\left(\frac{\log x}{\log 2}\right) \left[ 
     z - 1 + \frac{1}{z+1}\right] + 
     \log\left(\frac{x}{2}\right)\left[ 
     \frac{2}{1+z} - 1 - \frac{1}{(1+z)^2}\right] \\ 
\notag 
     & \phantom{= -(B + \log\log x) z\ } - 
     (\log x)^2 \times \frac{(z^3+z^5)}{(1-z^2)^3} - 
     (\log 2)^2 \times \frac{(z^2+6z^4+z^6)}{4 (1-z^2)^3} \\ 
\label{eqn_proof_tag_PHatFiniteTruncProdFactorOfGz_v3} 
     & =: \widehat{\mathcal{B}}(x; z). 
\end{align} 
We adjust the uniform bound parameter $R$ so that 
$$z \equiv z(k, x) = \frac{k-1}{\log\log x} \in \left[0, 1\right),$$ 
e.g., whenever $1 \leq k \leq \log\log x$ 
in the notation of Theorem \ref{theorem_HatPi_ExtInTermsOfGz}. 
We have that 
\begin{align*} 
\min_{0 \leq z \leq 1} \left[z - 1 + \frac{1}{z+1}\right] & = 0 \\ 
\min_{0 \leq z \leq 1} \left[\frac{2}{1+z} - 1 - \frac{1}{(1+z)^2}\right] & = -\frac{1}{4}. 
\end{align*} 
Moreover, when we expand out the coefficients of 
$(\log 2)^2$ and $(\log x)^2$ in 
\eqref{eqn_proof_tag_PHatFiniteTruncProdFactorOfGz_v3} 
by partial fractions of $z$, 
we see that all of the terms with an infinitely tending 
singularity as $z \rightarrow 1^{-}$ are positive. 
This means to obtain the lower bound, we can drop these contributions. 
What we are left to minimize is the following terms: 
\begin{align*} 
(\log 2)^2 \times \min_{0 \leq z \leq 1} \left[\frac{1}{4} - \frac{1}{4(1+z)^3} + 
     \frac{5}{8(1+z)^2} - \frac{1}{2(1+z)}\right] & = \frac{13}{108} (\log 2)^2 \\ 
(\log x)^2 \times \min_{0 \leq z \leq 1} \left[\frac{1}{4(1+z)^3} - 
     \frac{5}{8(1+z)^2} + \frac{1}{2(1+z)}\right] & = \frac{7}{54} (\log x)^2. 
\end{align*} 
So we have from \eqref{eqn_proof_tag_PHatFiniteTruncProdFactorOfGz_v3} that 
\begin{align*} 
\widehat{\mathcal{B}}(x; z) & \gg 
     \left(\frac{2}{x}\right)^{\frac{1}{4}} \times \exp\left( 
     \frac{13}{108} (\log 2)^2\right) \times \exp\left( 
     \frac{7}{54} (\log x)^2\right) \gg 
     x^{-\frac{1}{4}}. 
\end{align*} 
In summary, we have arrived at a proof that 
as $x \rightarrow \infty$
\begin{align} 
\label{eqn_proof_tag_simpl_v1} 
\frac{e^{\gamma z}}{(\log x)^{-z}} \times \exp\left(\widehat{\mathcal{B}}(u, x; z)\right) & \gg 
     x^{-\frac{1}{4}}. 
\end{align} 
Finally, to finish our proof of the new lower bound on $\mathcal{G}(-z)$, 
we need only bound the reciprocal factor of $\Gamma(1-z) = -z \cdot \Gamma(-z)$. 
Since $z \equiv z(k, x) = \frac{k-1}{\log\log x}$ for 
$k \in [1, \log\log x]$, or again with $z \in [0, 1)$, 
we obtain for minimal $k$ and all large enough $x \gg 1$ that 
$\Gamma(1-z) = \Gamma(1) = 1$, and for $k$ towards the upper range of 
its interval that 
\[
\Gamma(1-z) \approx \Gamma\left(\frac{1}{\log\log x}\right) = 
     \frac{1}{\log\log x} \times \Gamma\left(1 + \frac{1}{\log\log x}\right) 
     \approx \frac{1}{\log\log x}. 
\]
Therefore, our assertion that the claimed lower bound holds is correct. 
\end{proof} 

%\subsubsection{Technical adjustments in the proof of Theorem \ref{theorem_GFs_SymmFuncs_SumsOfRecipOfPowsOfPrimes}} 
%\label{subsubSection_remark_TechAdjustments_theorem_HatPi_ExtInTermsOfGz_TO_GFs_SymmFuncs_SumsOfRecipOfPowsOfPrimes} 

\begin{proof}[Proof of Theorem \ref{theorem_GFs_SymmFuncs_SumsOfRecipOfPowsOfPrimes}] 
We now discuss the differences between our construction and that in 
the technical proof of Theorem \ref{theorem_HatPi_ExtInTermsOfGz} 
in the reference when we bound $\mathcal{G}(-z)$ from below as in 
the previous lemma. 
The reference proves that for $0 \leq z < 2$ \cite[Thm.\ 7.18]{MV} 
\begin{equation} 
\label{eqn_MV_Azx_formula} 
A_{-z}(x) = -\frac{z F(1, -z)}{\Gamma(1-z)} \cdot x (\log x)^{-(z+1)} + 
     O\left(x (\log x)^{-\Re(z) - 2}\right). 
\end{equation}
Recall that for $r < 2$ we have by Cauchy's integral formula that 
\begin{equation} 
\label{eqn_MV7.61_CIF} 
(-1)^{k} \widehat{\pi}_k(x) = \frac{1}{2\pi\imath} \int_{|v|=r} 
     \frac{A_{-v}(x)}{v^{k+1}} dv. 
\end{equation} 
We first claim that uniformly for large $x$ and $1 \leq k \leq \log\log x$ we have 
\begin{equation} 
\label{eqn_proof_tag_HatPikx_BoundForGmz_v1} 
\widehat{\pi}_k(x) = \mathcal{G}\left(\frac{1-k}{\log\log x}\right) \times 
     \frac{x (\log\log x)^{k-1}}{(\log x) (k-1)!} 
     \left[1 + O\left(\frac{k}{(\log\log x)^2}\right)\right]. 
\end{equation} 
Then since we have proved in Lemma \ref{lemma_theorem_HatPi_ExtInTermsOfGz} that 
\[
\left\lvert \mathcal{G}\left(\frac{1-k}{\log\log x}\right) \right\rvert \gg 
     \frac{1}{x^{\frac{1}{4}}}, 
\]
the result in \eqref{eqn_proof_tag_HatPikx_BoundForGmz_v1} implies our 
stated uniform asymptotic bound. Namely, we obtain that 
\[
\widehat{\pi}_k(x) \gg 
     \frac{x^{\frac{3}{4}}}{\log x} \cdot 
     \frac{(\log\log x)^{k-1}}{(k-1)!} \left[1 + 
     O\left(\frac{k}{(\log\log x)^2}\right)\right]. 
\]
We must provide analogs to the proofs of the 
two separate bounds from the reference corresponding to the error and 
main terms of our estimate according to 
\eqref{eqn_MV_Azx_formula} and \eqref{eqn_MV7.61_CIF}. \\ 
\textit{Step I: Error Term Bound.} 
To prove that the error term bound holds, we estimate the following bounds for 
$r := \frac{k-1}{\log\log x}$ with $r < 1$ whenever $2 \leq k \leq \log\log x$: 
\begin{align} 
\notag 
\left\lvert \frac{1}{2\pi\imath} \int_{|v|=r} 
     \frac{x}{(\log x)^2} \frac{(\log x)^{-\Re(v)}}{v^{k+1}} dv \right\rvert & \ll 
     x (\log x)^{-(r+2)} r^{-(k+1)} 
     \ll \frac{x}{(\log x)^2} \frac{(\log\log x)^{k+1}}{e^{k-1} (k-1)^{k+1}} \\ 
\notag 
     & \ll \frac{x}{(\log x)^2} \frac{(\log\log x)^{k+1}}{e^{2(k-1)} (k-1)! (k-1)^{\frac{3}{2}}} 
     \ll \frac{x}{(\log x)^{2}} \frac{(\log\log x)^{k+1}}{(k-1)!} \\ 
\label{eqn_proof_tag_ErrorTermBounds_v1} 
     & \ll \frac{x}{\log x} \frac{(\log\log x)^{k-4}}{(k-1)!}. 
\end{align} 
By the Cauchy integral formula, we can verify that 
\[
\left\lvert \frac{1}{2\pi\imath} \int_{|v|=r} 
     \frac{x}{(\log x)^2} \frac{(\log x)^{-\Re(v)}}{v^{2}} dv \right\rvert = 
     \frac{x}{(\log x)^2} \cdot (\log\log x)^2 \ll 
     \frac{x}{(\log x) (\log\log x)^2}, 
\]
so that the formula for the error term in \eqref{eqn_proof_tag_ErrorTermBounds_v1} 
also matches when $k := 1$. 

We can calculate that for $0 \leq z < 1$ 
\begin{align*} 
\prod_p \left(1 + \frac{z}{p}\right)^{-1} \left(1 - \frac{1}{p}\right)^{-z} & = 
     \exp\left(-\sum_p \left[\log\left(1 + \frac{z}{p}\right) + z 
     \log\left(1 - \frac{1}{p}\right) \right]\right) \\ 
     & \sim \exp\left(-o(z) \times \sum_p \frac{1}{p^2}\right) \\ 
     & \gg \exp\left(-o(z) \cdot P(2)\right) \gg_z 1. 
\end{align*} 
In other words, we have that 
$\mathcal{G}\left(\frac{1-k}{\log\log x}\right) \gg 1$ whenever $1 \leq k \leq \log\log x$. 
So the error term in \eqref{eqn_proof_tag_ErrorTermBounds_v1} 
is majorized by taking $O\left(\frac{k}{(\log\log x)^3}\right)$ as our 
upper bound. \\ 
\textit{Step II: Main Term Bound.} 
By \eqref{eqn_MV_Azx_formula} the main term 
estimate for \eqref{eqn_MV7.61_CIF} 
is given by $\frac{x}{\log x} \cdot I_x$, where 
\[
I_x := \frac{(-1)^{k-1}}{2\pi\imath} \int_{|v|=r} G(-z) (\log x)^{-v} v^{-k} dv. 
\]
In particular, we can write $I_x = I_{1,x} + I_{2,x}$ where we define 
\begin{align*} 
I_{1,x} & := \frac{G(-r)}{2\pi\imath} \int_{|v|=r} (\log x)^{-v} v^{-k} dv \\ 
    & \phantom{:}= \frac{(-1)^{k-1} G(-r) (\log\log x)^{k-1}}{(k-1)!} \\ 
I_{2,x} & := \frac{1}{2\pi\imath} \int_{|v|=r} (G(-v) - G(-r)) (\log x)^{-v} v^{-k} dv \\ 
    & \phantom{:}= \frac{1}{2\pi\imath} \int_{|v|=r} (G(-v) - G(-r) + G^{\prime}(-r) (v-r)) 
    (\log x)^{-v} v^{-k} dv. 
\end{align*} 
The second integral formula for $I_{2,x}$ results from integration by parts. 

We have by taking a power series expansion of $G^{\prime\prime}(-w)$ about $-r$ and integrating 
the resulting series termwise with respect to $w$ that when $|v| = r$ 
\[
\left\lvert G(-v) - G(-r) + G^{\prime}(-r) (v-r) \right\rvert = 
     \left\lvert \int_{r}^{v} (v-w) G^{\prime\prime}(-w) dw \right\rvert \ll |v-r|^2. 
\] 
Now we parameterize the curve in the contour for $I_{2,x}$ by writing 
$v = re^{2\pi\imath t}$ for $t \in [-1/2, 1/2]$. This leads us to the bounds 
\begin{align*} 
|I_{2,x}| & \ll r^{3-k} \times \int_{-\frac{1}{2}}^{\frac{1}{2}} |e^{2\pi\imath t} - 1|^2 \cdot 
     (\log x)^{r e^{2\pi\imath t}} \cdot e^{2\pi\imath t} dt \\ 
     & \ll r^{3-k} \times \int_{-\frac{1}{2}}^{\frac{1}{2}} \sin^2(\pi t) \cdot 
     e^{(k-1) \cos(2\pi t)} dt. 
\end{align*} 
Whenever $|x| \leq 1$, we know that $|\sin x| \leq |x|$. 
Also, $\cos(2\pi t) \leq 1 - 8t^2$ whenever $|t| \leq \frac{1}{2}$. 
Thus the last bound for $|I_{2,x}|$ becomes 
\begin{align*} 
|I_{2,x}| & \ll r^{3-k} e^{k-1} \times \int_0^{\infty} t^2 \cdot e^{-8(k-1)t^2} dt \\ 
     & \ll \frac{r^{3-k} e^{k-1}}{(k-1)^{3/2}} = \frac{(\log\log x)^{k-3} e^{k-1}}{(k-1)^{k-3/2}} \\ 
     & \ll \frac{k \cdot (\log\log x)^{k-3}}{(k-1)!}. 
\end{align*} 
Thus the contribution from the term $|I_{2,x}|$ can then be absorbed into the error term bound 
in \eqref{eqn_proof_tag_HatPikx_BoundForGmz_v1}. 
\end{proof} 

\subsection{The distribution of exceptional values of $\Omega(n)$} 

The next theorems reproduced from \cite[\S 7.4]{MV} characterize the relative 
scarcity of the distribution of the $\Omega(n)$ for $n \leq x$ such that 
$\Omega(n) > \log\log x$. 

\begin{theorem}[Upper bounds on exceptional values of $\Omega(n)$ for large $n$] 
\label{theorem_MV_Thm7.20-init_stmt} 
Let 
\begin{align*} 
A(x, r) & := \#\left\{n \leq x: \Omega(n) \leq r \cdot \log\log x\right\}, \\ 
B(x, r) & := \#\left\{n \leq x: \Omega(n) \geq r \cdot \log\log x\right\}. 
\end{align*} 
If $0 < r \leq 1$ and $x \geq 2$, then 
\[
A(x, r) \ll x (\log x)^{r-1 - r\log r}, \text{ \ as\ } x \rightarrow \infty. 
\]
If $1 \leq r \leq R < 2$ and $x \geq 2$, then 
\[
B(x, r) \ll_R x \cdot (\log x)^{r-1-r \log r}, \text{ \ as\ } x \rightarrow \infty. 
\]
\end{theorem} 

Theorem \ref{theorem_MV_Thm7.21-init_stmt} is a special case analog to the 
celebrated Erd\"os-Kac theorem typically stated for the 
normally distributed values of the scaled-shifted function $\omega(n)$ over $n \leq x$ as 
$x \rightarrow \infty$ \cite[\cf Thm.\ 7.21]{MV}. 

\begin{theorem}[Exact limiting bounds on exceptional values of $\Omega(n)$ for large $n$] 
\label{theorem_MV_Thm7.21-init_stmt} 
We have that as $x \rightarrow \infty$ 
\[
\#\left\{3 \leq n \leq x: \Omega(n) - \log\log n \leq 0\right\} = 
     \frac{x}{2} + O\left(\frac{x}{\sqrt{\log\log x}}\right). 
\]
\end{theorem} 

The key interpretation we need to take away from the statements 
of Theorem \ref{theorem_MV_Thm7.20-init_stmt} and 
Theorem \ref{theorem_MV_Thm7.21-init_stmt} 
is the result proved in the next corollary. 
The role of the parameter $R$ involved in stating the previous theorem 
is a critical bound as the scalar factor in the upper bound on $k \leq \log\log x$ in 
Theorem \ref{theorem_HatPi_ExtInTermsOfGz} up to which our uniform bounds given by 
Theorem \ref{theorem_GFs_SymmFuncs_SumsOfRecipOfPowsOfPrimes} hold. 
In contrast, for $n \geq 2$ we can actually 
have contributions from values distributed throughout the range $1 \leq \Omega(n) \leq \log_2(n)$ 
infinitely often. 
It is then crucial that we can show that the main term in the asymptotic formulas we obtain 
for the summatory function over $\widehat{\pi}_k(x)$ 
is captured by summing only over the truncated range of 
$k \in [1, \log\log x]$ where the uniform bounds 
guaranteed by Theorem \ref{theorem_HatPi_ExtInTermsOfGz} and 
Theorem \ref{theorem_GFs_SymmFuncs_SumsOfRecipOfPowsOfPrimes} hold. 

\begin{cor} 
\label{theorem_MV_Thm7.20} 
Using the notation for $A(x, r)$ and $B(x, r)$ from 
Theorem \ref{theorem_MV_Thm7.20-init_stmt}, 
we have that for $x \geq 2$ and $\delta > 0$, 
\[
\frac{B(x, 1+\delta)}{A(x, 1)} = o_{\delta}(1), 
     \mathrm{\ as\ } x \rightarrow \infty. 
\]
\end{cor} 
\begin{proof} 
To show that the asymptotic bound is correct, we compute using 
Theorem \ref{theorem_MV_Thm7.20-init_stmt} and 
Theorem \ref{theorem_MV_Thm7.21-init_stmt} that 
\begin{align*} 
\frac{B(x, 1+\delta)}{A(x, 1)} & \ll 
     \frac{x \cdot (\log x)^{\delta - (1+\delta)\log(1+\delta)}}{ 
     O(1) + \frac{x}{2} + 
     O\left(\frac{x}{\sqrt{\log\log x}}\right)} 
     \sim 
     o_{\delta}(1),  
\end{align*} 
as $x \rightarrow \infty$. 
\end{proof} 

\begin{remark}[Applications and key consequences] 
Since $\mathbb{E}[\Omega(n)] = \log\log n + B$, with $0 < B < 1$ the 
absolute constant from Mertens theorem, 
when we denote the range of $k > \log\log x$ as holding in the form of 
$k > (1 + \delta) \log\log x$ for $\delta > 0$ at large $x$, we can assume that 
$\delta \rightarrow 0^{+}$ as $x \rightarrow \infty$ when we apply 
Corollary \ref{theorem_MV_Thm7.20} in practice. 
In particular, this type of bound holds since $k > \log\log x$ implies that 
\[
\floor{\log\log x} + 1 \geq (1 + \delta) \log\log x \quad\implies\quad 
     \delta \leq \frac{1 + \left\{\log\log x\right\}}{\log\log x} = o(1), 
     \mathrm{\ as\ } x \rightarrow \infty. 
\] 
The key consequence is that the ratio 
\[
\left\lvert \frac{\sum\limits_{k > \log\log x} (-1)^k \widehat{\pi}_k(x)}{ 
     \sum\limits_{k \leq \log\log x} (-1)^k \widehat{\pi}_k(x)} \right\rvert 
     \ll 
     \frac{\sqrt{\log\log x} \cdot B(x, 1+\delta)}{x} = o_{\delta}(1), 
\] 
is bounded above by at most a small constant for any $\delta > 0$ when $x$ is large. 
The second term in the last bound is obtained by summing over the uniform estimates 
guaranteed by Theorem \ref{theorem_HatPi_ExtInTermsOfGz} and 
applying \eqref{eqn_IncompleteGamma_PropB} to the 
resulting expression involving the incomplete gamma function. 
\end{remark} 

\newpage
\section{Auxiliary sequences to express the Dirichlet inverse function, $g^{-1}(n)$} 
\label{Section_InvFunc_PreciseExpsAndAsymptotics} 

The pages of tabular data given as Table \ref{table_conjecture_Mertens_ginvSeq_approx_values} 
in the appendix section (refer to 
page \pageref{table_conjecture_Mertens_ginvSeq_approx_values}) are intended to 
provide clear insight into why we eventually arrived at the approximations to 
$g^{-1}(n)$ proved in this section. The table provides illustrative 
numerical data by examining the approximate behavior 
at hand for the cases of $1 \leq n \leq 500$ with \emph{Mathematica} 
\cite{SCHMIDT-MERTENS-COMPUTATIONS}. 

\subsection{Definitions and basic properties of component function sequences} 

We define the following auxiliary coefficient sequence for integers $n \geq 1$ and $k \geq 0$: 
\begin{align} 
\label{eqn_CknFuncDef_v2} 
C_k(n) := \begin{cases} 
     \varepsilon(n), & \text{ if $k = 0$; } \\ 
     \sum\limits_{d|n} \omega(d) C_{k-1}(n/d), & \text{ if $k \geq 1$. } 
     \end{cases} 
\end{align} 

By recursively expanding the definition of $C_k(n)$ 
at any fixed $n \geq 2$, we see that 
we can form a chain of at most $\Omega(n)$ iterated (or nested) divisor sums by 
unfolding the definition of \eqref{eqn_CknFuncDef_v2} inductively. 
By the same argument, we see that at fixed $n$, the function 
$C_k(n)$ is seen to be non-zero only for positive integers 
$k \leq \Omega(n)$ whenever $n \geq 2$. 
A sequence of relevant signed semi-diagonals of the functions $C_k(n)$ begins as follows 
\cite[\seqnum{A008480}]{OEIS}: 
\[
\{\lambda(n) \cdot C_{\Omega(n)}(n) \}_{n \geq 1} \mapsto \{
     1, -1, -1, 1, -1, 2, -1, -1, 1, 2, -1, -3, -1, 2, 2, 1, -1, -3, -1, \
     -3, 2, 2, -1, 4, 1, 2, \ldots \}. 
\]
We can see that $C_{\Omega(n)}(n) \leq (\Omega(n))!$ for all $n \geq 1$. In fact, 
$h^{-1}(n) \equiv \lambda(n) C_{\Omega(n)}(n)$ is the same function given by 
the formula in \eqref{eqn_proof_tag_hInvn_ExactNestedSumFormula_CombInterpetIdent_v3} from 
Proposition \ref{prop_SignageDirInvsOfPosBddArithmeticFuncs_v1}. 
This sequence of semi-diagonals of 
\eqref{eqn_CknFuncDef_v2} 
is precisely related to $g^{-1}(n)$ in the next subsection. 
In Section \ref{Section_NewFormulasForgInvn} 
we prove exact probabilistic distributions for the values of 
$C_{\Omega(n)}(n)$. 

\subsection{Relating the auxiliary functions $C_{\Omega(n)}(n)$ to formulas approximating $g^{-1}(n)$} 
\label{subSection_Relating_CknFuncs_to_gInvn} 

\begin{lemma}[An exact initial formula for $g^{-1}(n)$] 
\label{lemma_AnExactFormulaFor_gInvByMobiusInv_v1} 
For all $n \geq 1$, we have that 
\[
g^{-1}(n) = \sum_{d|n} \mu\left(\frac{n}{d}\right) \lambda(d) C_{\Omega(d)}(d). 
\]
\end{lemma}
\begin{proof} 
We first write out the standard recurrence relation for the Dirichlet inverse as 
\begin{align} 
\label{eqn_proof_tag_gInvCvlOne_EQ_omegaCvlgInvCvl_v1} 
g^{-1}(n) & = - \sum_{\substack{d|n \\ d>1}} (\omega(d) + 1) g^{-1}(n/d) 
     \quad\implies\quad 
     (g^{-1} \ast 1)(n) = -(\omega \ast g^{-1})(n). 
\end{align} 
We argue that for $1 \leq m \leq \Omega(n)$, we can inductively expand the 
implication on the right-hand-side of \eqref{eqn_proof_tag_gInvCvlOne_EQ_omegaCvlgInvCvl_v1} 
in the form of $(g^{-1} \ast 1)(n) = F_m(n)$ where 
$F_m(n) := (-1)^{m} \cdot (C_m(-) \ast g^{-1})(n)$, or so that 
\[
F_m(n) = - 
     \begin{cases} 
     \sum\limits_{\substack{d|n \\ d > 1}} F_{m-1}(d) \times \sum\limits_{\substack{r|\frac{n}{d} \\ r > 1}} 
     \omega(r) g^{-1}\left(\frac{n}{dr}\right), & m \geq 2, \\ 
     (\omega \ast g^{-1})(n), & m = 1. 
     \end{cases} 
\]
By repeatedly expanding the right-hand-side of the previous equation, 
we find that for $m := \Omega(n)$ (i.e., with the expansions at a 
maximal depth in the previous equation) 
\begin{equation} 
\label{eqn_proof_tag_gInvCvlOne_EQ_omegaCvlgInvCvl_v2} 
(g^{-1} \ast 1)(n) = (-1)^{\Omega(n)} C_{\Omega(n)}(n) = \lambda(n) C_{\Omega(n)}(n). 
\end{equation} 
The formula then follows from \eqref{eqn_proof_tag_gInvCvlOne_EQ_omegaCvlgInvCvl_v2} 
by M\"obius inversion applied to each side of the last equation. 
\end{proof} 

\begin{cor} 
\label{cor_AnExactFormulaFor_gInvByMobiusInv_nSqFree_v2} 
For all squarefree integers $n \geq 1$, we have that 
\begin{equation} 
\label{eqn_gInvnSqFreeN_exactDivSum_Formula} 
g^{-1}(n) = \lambda(n) \times \sum_{d|n} C_{\Omega(d)}(d). 
\end{equation} 
\end{cor} 
\begin{proof} 
Since $g^{-1}(1) = 1$, clearly the claim is true for $n = 1$. Suppose that $n \geq 2$ and that 
$n$ is squarefree. Then $n = p_1p_2 \cdots p_{\omega(n)}$ where $p_i$ is prime for all 
$1 \leq i \leq \omega(n)$. Since all divisors of any squarefree $n$ are necessarily also squarefree, 
we can transform the exact divisor sum guaranteed for all $n$ in 
Lemma \ref{lemma_AnExactFormulaFor_gInvByMobiusInv_v1} into a sum that partitions the divisors 
according to the number of distinct prime factors as follows: 
\begin{align*} 
g^{-1}(n) & = \sum_{i=0}^{\omega(n)} \sum_{\substack{d|n \\ \omega(d)=i}} (-1)^{\omega(n) - i} (-1)^{i} \cdot 
     C_{\Omega(d)}(d) \\ 
     & = \lambda(n) \times \sum_{i=0}^{\omega(n)} \sum_{\substack{d|n \\ \omega(d)=i}} C_{\Omega(d)}(d) \\ 
     & = \lambda(n) \times \sum_{d|n} C_{\Omega(d)}(d). 
\end{align*} 
The signed contributions in the first of the previous equations is 
justified by noting that $\lambda(n) = \mu(n) = (-1)^{\omega(n)}$ 
whenever $n$ is squarefree, and that for $d \geq 1$
squarefree we have the correspondence 
$\omega(d) = k$ $\implies$ $\Omega(d) = k$. 
\end{proof} 

Since $C_{\Omega(n)}(n) = |h^{-1}(n)|$ using the notation defined in the the proof of 
Proposition \ref{prop_SignageDirInvsOfPosBddArithmeticFuncs_v1}, we can see that 
$C_{\Omega(n)}(n) = (\omega(n))!$ for squarefree $n \geq 1$. 
A proof of part (C) of Conjecture \ref{lemma_gInv_MxExample} 
follows as an immediate consequence. 

\begin{lemma} 
\label{lemma_AbsValueOf_gInvn_FornSquareFree_v1} 
For all positive integers $n \geq 1$, we have that 
\begin{equation} 
\label{eqn_AbsValueOf_gInvn_FornSquareFree_v1} 
|g^{-1}(n)| = \sum_{d|n} \mu^2\left(\frac{n}{d}\right) C_{\Omega(d)}(d). 
\end{equation} 
\end{lemma} 
\begin{proof} 
By applying 
Lemma \ref{lemma_AnExactFormulaFor_gInvByMobiusInv_v1}, 
Proposition \ref{prop_SignageDirInvsOfPosBddArithmeticFuncs_v1} and the 
complete multiplicativity of $\lambda(n)$, 
we easily obtain the stated result. 
In particular, since $\mu(n)$ is non-zero only at squarefree integers and 
at any squarefree $d \geq 1$ we have $\mu(d) = (-1)^{\omega(d)} = \lambda(d)$, 
Lemma \ref{lemma_AnExactFormulaFor_gInvByMobiusInv_v1} implies 
\begin{align*} 
|g^{-1}(n)| & = \lambda(n) \times \sum_{d|n} \mu\left(\frac{n}{d}\right) \lambda(d) C_{\Omega(d)}(d) \\ 
     & = \sum_{d|n} \mu^2\left(\frac{n}{d}\right) \lambda\left(\frac{n}{d}\right) 
     \lambda(nd) C_{\Omega(d)}(d) \\ 
     & = \lambda(n^2) \times \sum_{d|n} \mu^2\left(\frac{n}{d}\right) C_{\Omega(d)}(d). 
\end{align*} 
In the last equation, we see that 
that $\lambda(n^2) = +1$ for all $n \geq 1$ since the number of distinct 
prime factors (counting multiplicity) of any square integer is even. 
\end{proof} 

Combined with the signedness property of $g^{-1}(n)$ guaranteed by 
Proposition \ref{prop_SignageDirInvsOfPosBddArithmeticFuncs_v1}, 
Lemma \ref{lemma_AbsValueOf_gInvn_FornSquareFree_v1} shows that its summatory 
function is expressed as 
\[
G^{-1}(x) = \sum_{d \leq x} \lambda(d) C_{\Omega(d)}(d) M\left(\Floor{x}{d}\right). 
\]
Additionally, since \eqref{eqn_AntiqueDivisorSumIdent} implies that 
$$\lambda(d) C_{\Omega(d)}(d) = (g^{-1} \ast 1)(d) = (\chi_{\mathbb{P}} + \varepsilon)^{-1}(d),$$ 
where $\chi_{\mathbb{P}}$ denotes the characteristic function of the primes, we also clearly 
recover by inversion that 
\[
M(x) = G^{-1}(x) + \sum_{p \leq x} G^{-1}\left(\Floor{x}{p}\right), x \geq 1. 
\]

\subsection{A connection to the distribution of the primes} 

The combinatorial complexity of $g^{-1}(n)$ is deeply tied to the distribution of the primes 
$p \leq n$ as $n \rightarrow \infty$. 
The magnitudes and dispersion of the primes $p \leq x$ certainly restricts the 
repeating of these distinct sequence values. 
Nonetheless, we can see that the following 
is still clear about the relation of the weight functions $|g^{-1}(n)|$ to the 
distribution of the primes: 
The value of $|g^{-1}(n)|$ is entirely dependent on the pattern of the \emph{exponents} 
(viewed as multisets) of the distinct prime factors of $n \geq 2$ 
(\cf Heuristic \ref{heuristic_SymmetryIngInvFuncs}). 
The relation of the repitition of the distinct values 
of $|g^{-1}(n)|$ in forming bounds on $G^{-1}(x)$ makes another clear tie to 
$M(x)$ through Proposition \ref{prop_Mx_SBP_IntegralFormula}. 

\begin{example}[Combinatorial significance to the distribution of $g^{-1}(n)$] 
We have a natural extremal behavior with respect to distinct values of $\Omega(n)$ 
corresponding to squarefree integers and prime powers. Namely, if for $k \geq 1$ we define the 
infinite sets $M_k$ and $m_k$ to correspond to the maximal (minimal) sets of 
positive integers such that 
\begin{align*} 
M_k & := \left\{n \geq 2: |g^{-1}(n)| = \underset{{\substack{j \geq 2 \\ \Omega(j) = k}}}{\operatorname{sup}} 
     |g^{-1}(j)|\right\} \subseteq \mathbb{Z}^{+}, \\  
m_k & := \left\{n \geq 2: |g^{-1}(n)| = \underset{{\substack{j \geq 2 \\ \Omega(j) = k}}}{\operatorname{inf}} 
     |g^{-1}(j)|\right\} \subseteq \mathbb{Z}^{+}, 
\end{align*} 
then any element of $M_k$ is squarefree and any element of $m_k$ is a prime power. 
In particular, we have that for any $N_k \in M_k$ and $n_k \in m_k$
\[
N_k = \sum_{j=0}^{k} \binom{k}{j} \cdot j!, \quad \mathrm{\ and\ } \quad n_k = 2 \cdot (-1)^{k}. 
\]
The formula for the function $h^{-1}(n) = (g^{-1} \ast 1)(n)$ defined in the proof of 
Proposition \ref{prop_SignageDirInvsOfPosBddArithmeticFuncs_v1} implies that we can express 
an exact formula for $g^{-1}(n)$ in terms of symmetric polynomials in the 
exponents of the prime factorization of $n$. 
Namely, for $n \geq 2$ and $0 \leq k \leq \omega(n)$ let 
\[
\widehat{e}_k(n) := [z^k] \prod_{p|n} (1 + z \cdot \nu_p(n)) = [z^k] \prod_{p^{\alpha} || n} (1 + \alpha z). 
\]
Then we have essentially shown using 
\eqref{eqn_proof_tag_hInvn_ExactNestedSumFormula_CombInterpetIdent_v3} and 
\eqref{eqn_AbsValueOf_gInvn_FornSquareFree_v1} that we can expand formulas for 
these inverse functions in the following form: 
\[
g^{-1}(n) = h^{-1}(n) \times \sum_{k=0}^{\omega(n)} \binom{\Omega(n)}{k}^{-1} 
     \frac{\widehat{e}_k(n)}{k!}, n \geq 2. 
\]
The combinatorial formula for 
$h^{-1}(n) = \lambda(n) \cdot (\Omega(n))! \times \prod_{p^{\alpha} || n} (\alpha !)^{-1}$ 
we derived in the proof of the key signedness proposition in 
Section \ref{Section_PrelimProofs_Config} 
suggests further patterns and more regularity in the contributions of the distinct weighted 
terms for $G^{-1}(x)$. 
Our interpretations leading to the proof of the bounds on $|G^{-1}(x)|$ from below via 
Theorem \ref{theorem_GInvxLowerBoundByGEInvx_v1} 
is less combinatorially motivated. 
\end{example} 

\newpage
\section{The precise limiting distributions of 
         $C_{\Omega(n)}(n)$ and $|g^{-1}(n)|$} 
\label{Section_NewFormulasForgInvn} 

We have remarked already in the introduction that the relation of the component 
functions, $g^{-1}(n)$ and $C_{\Omega(n)}(n)$, to the canonical additive functions 
$\omega(n)$ and $\Omega(n)$ leads to the regular properties of these functions 
witnessed (\'{a} priori) in Table \ref{table_conjecture_Mertens_ginvSeq_approx_values}. 
In particular, each of $\omega(n)$ and $\Omega(n)$ satisfies 
an Erd\"os-Kac theorem that shows that the density of a shifted and scaled variant of each 
of the sets of these function values for $n \leq x$ can be expressed through a 
limiting normal distribution as $x \rightarrow \infty$ 
\cite{ERDOS-KAC-REF,BILLINGSLY-CLT-PRIMEDIVFUNC,RENYI-TURAN}. 
In the remainder of this section we establish more analytically motivated proofs of 
related properties of these key sequences used to express $G^{-1}(x)$, 
again in the spirit of Montgomery and Vaughan's reference manual 
(\cf Remark \ref{remark_MV_NewDGFApplications}). 

\begin{prop} 
\label{prop_HatAzx_ModSummatoryFuncExps_RelatedToCkn} 
Let the function $\widehat{F}(s, z)$ is defined for $\Re(s) \geq 2$ and $|z| < |P(s)|^{-1}$ 
in terms of the prime zeta function by 
\[
\widehat{F}(s, z) := \frac{1}{1-P(s) z} 
     \times \prod_p \left(1 - \frac{1}{p^s}\right)^{z}. 
\]
For $|z| < P(2)^{-1}$, let the summatory function of the coefficients of the 
DGF expansion of $\widehat{F}(s, z)$ be defined as follows: 
\[
\widehat{A}_z(x) := \sum_{n \leq x} (-1)^{\omega(n)} 
     C_{\Omega(n)}(n) z^{\Omega(n)}. 
\]
We have that for all sufficiently large $x$ 
\[
\widehat{A}_z(x) = \frac{x}{\Gamma(z)} \cdot \widehat{F}(2, z) \cdot (\log x)^{z-1} + 
     O_{z}\left(x \cdot (\log x)^{\Re(z) - 2}\right), |z| < P(2)^{-1}. 
\]
\end{prop} 
\begin{proof} 
We can see by adapting the notation from the proof of 
Proposition \ref{prop_SignageDirInvsOfPosBddArithmeticFuncs_v1} 
that for $n \geq 2$ 
\[
C_{\Omega(n)}(n) = (\Omega(n))! \times \prod_{p^{\alpha}||n} \frac{1}{\alpha!}. 
\]
We can generate scaled forms of these terms through the Dirichlet series identity 
\begin{align*} 
\sum_{n \geq 1} \frac{C_{\Omega(n)}(n)}{(\Omega(n))!} \cdot 
     \frac{(-1)^{\omega(n)} z^{\Omega(n)}}{n^s} & = \prod_p \left(1 + \sum_{r \geq 1} 
     \frac{z^{\Omega(p^r)}}{r! \cdot p^{rs}}\right)^{-1} 
     = \exp\left(z \cdot P(s)\right), \Re(s) \geq 2, z \in \mathbb{C}. 
\end{align*} 
By computing a Laplace transform on the right-hand-side of the above, we obtain 
\begin{align*} 
\sum_{n \geq 1} \frac{C_{\Omega(n)}(n) \cdot (-1)^{\omega(n)} z^{\Omega(n)}}{n^s} & = 
     \int_0^{\infty} e^{-t} \exp\left(tz \cdot P(s)\right) dt = \frac{1}{1 - P(s) z}, 
     \Re(s) \geq 2, |z| < |P(s)|^{-1}. 
\end{align*} 
It follows that 
\[
\sum_{n \geq 1} \frac{\lambda_{\ast}(n) C_{\Omega(n)}(n) z^{\Omega(n)}}{n^s} = 
     \zeta(s)^z \times \widehat{F}(s, z), \Re(s) \geq 2, |z| < |P(s)|^{-1}. 
\]
Since $\widehat{F}(s, z)$ is convergent as an analytic function of $s$ for all $\Re(s) > 1$ 
whenever $|z| < |P(s)|^{-1}$, 
if $b_z(n)$ are the coefficients in the DGF expansion of $\widehat{F}(s, z)$, then 
\[
\left\lvert \sum_{n \geq 1} \frac{b_z(n) (\log n)^{2R+1}}{n^s} \right\rvert < +\infty, 
\]
is uniformly bounded for $|z| \leq R$. This fact follows by repeated termwise differentiation 
with respect to $s$. 

We must adapt the details to the case where the next proof method arises in the first 
application instance from the reference \cite[\S 7.4; Thm.\ 7.18]{MV}. 
Let the function $d_z(n)$ be generated as the coefficients of the DGF 
$\zeta(s)^{z}$ for $\Re(s) > 1$, with corresponding 
summatory function $D_z(x) := \sum_{n \leq x} d_z(n)$. 
The theorem in \cite[Thm.\ 7.17; \S 7.4]{MV} implies that for any $z \in \mathbb{C}$ and $x \geq 2$ 
\[
D_z(x) = \frac{x (\log x)^{z-1}}{\Gamma(z)} + O\left(x \cdot (\log x)^{\Re(z)-2}\right). 
\]
Taking the notation from the reference, we set 
$b_z(n) \equiv (-1)^{\omega(n)} C_{\Omega(n)}(n) z^{\Omega(n)}$, set the convolution 
$a_z(n) := \sum_{d|n} b_z(d) d_z(n/d)$, and define its summatory function 
$A_z(x) := \sum_{n \leq x} a_z(n)$. 
Then we have that 
\begin{align} 
\notag 
A_z(x) & = \sum_{m \leq x/2} b_z(m) D_z(x/m) + \sum_{x/2 < m \leq x} b_z(m) \\ 
\label{eqn_proof_tag_Azx_FullTermsFormulaSum_v1} 
     & = \frac{x}{\Gamma(z)} \times \sum_{m \leq x/2} 
     \frac{b_z(m)}{m^2} \times m \cdot \log\left(\frac{x}{m}\right)^{z-1} + 
     O\left(\sum_{m \leq x} \frac{x \cdot |b_z(m)|}{m^2} \times m \cdot 
     \log\left(\frac{2x}{m}\right)^{\Re(z) - 2}\right). 
\end{align} 
We can sum the coefficients for $u > e$ large as 
\begin{align} 
\label{eqn_proof_tag_FsEQ2z_justification_formula} 
\sum_{m \leq u} \frac{b_z(m)}{m} & = (\widehat{F}(2, z) + O(u^{-2})) u - \int_1^{u} 
     (\widehat{F}(2, z) + O(t^{-2})) dt 
     = \widehat{F}(2, z) + O(1 + u^{-1}). 
\end{align} 
Suppose that $|z| \leq R < P(2)^{-1}$. 
The error term in \eqref{eqn_proof_tag_Azx_FullTermsFormulaSum_v1} satisfies 
\begin{align*} 
\sum_{m \leq x} \frac{x \cdot |b_z(m)|}{m^2} \times m \cdot 
     \log\left(\frac{2x}{m}\right)^{\Re(z) - 2} & \ll 
     x (\log x)^{\Re(z) - 2} \times \sum_{m \leq \sqrt{x}} \frac{|b_z(m)|}{m} \\ 
     & \phantom{\ll x\ } + 
     x (\log x)^{-(R+2)} \times \sum_{m > \sqrt{x}} \frac{|b_z(m)|}{m} (\log m)^{2R} \\ 
     & \ll x (\log x)^{\Re(z) - 2} \cdot \widehat{F}(2, z) 
     = O_z\left(x \cdot (\log x)^{\Re(z) - 2}\right), |z| \leq R. 
\end{align*} 
In the main term estimate for $A_z(x)$ from 
\eqref{eqn_proof_tag_Azx_FullTermsFormulaSum_v1}, when $m \leq \sqrt{x}$ we have 
\[
\log\left(\frac{x}{m}\right)^{z-1} = (\log x)^{z-1} + 
     O\left((\log m) (\log x)^{\Re(z) - 2}\right). 
\]
The total sum over the interval $m \leq x/2$ corresponds to bounding the following 
sum components when we take $|z| \leq R$: 
\begin{align*} 
\sum_{m \leq x/2} b_z(m) D_z(x/m) & = \frac{x}{\Gamma(z)} (\log x)^{z-1} \times 
     \sum_{m \leq x/2} \frac{b_z(m)}{m} \\ 
     & \phantom{=\quad\ } + 
     O_z\left(x (\log x)^{\Re(z)-2} \times \sum_{m \leq \sqrt{x}} \frac{|b_z(m)|}{m} + 
     x (\log x)^{R-1} \times \sum_{m > \sqrt{x}} \frac{|b_z(m)|}{m}\right) \\ 
     & = \frac{x}{\Gamma(z)} (\log x)^{z-1} \widehat{F}(2, z) + O_z\left( 
     x (\log x)^{\Re(z)-2} \times \sum_{m \geq 1} \frac{b_z(m) (\log m)^{2R+1}}{m^2} 
     \right) \\ 
     & = \frac{x}{\Gamma(z)} (\log x)^{z-1} \widehat{F}(2, z) + O_z\left( 
     x (\log x)^{\Re(z)-2}\right). 
     \qedhere  
\end{align*} 
\end{proof} 

\begin{theorem} 
\label{theorem_CnkSpCasesScaledSummatoryFuncs} 
We have uniformly for $1 \leq k < \log\log x$ 
that as $x \rightarrow \infty$ 
\[
\widehat{C}_k(x) := 
     \sum_{\substack{n \leq x \\ \Omega(n) = k}} (-1)^{\omega(n)} 
     C_k(n) \asymp 
     -\frac{x}{\log x} \cdot \frac{(\log\log x - \log\zeta(2))^{k-1}}{(k-1)!} \left[1 + 
     O\left(\frac{k}{(\log\log x)^2}\right)\right]. 
\]
\end{theorem} 
\begin{proof} 
The proof is a related adaptation of the method of Montgomery and Vaughan we cited in 
Remark \ref{remark_intuitionConstrIn_theorem_HatPi_ExtInTermsOfGz} 
to prove our variant of 
Theorem \ref{theorem_GFs_SymmFuncs_SumsOfRecipOfPowsOfPrimes}. 
We begin by bounding a contour integral over the error term for fixed large $x$ when 
$r := \frac{k-1}{\log\log x}$ with $r < 2$: 
\begin{align*} 
\left\lvert \int_{|v|=r} \frac{x \cdot (\log x)^{-(\Re(v) + 2)}}{v^{k+1}} dv \right\rvert & \ll 
     x (\log x)^{-(r+2)} r^{-(k+1)} \ll \frac{x}{(\log x)^2} \cdot 
     \frac{(\log\log x)^{k+1}}{(k-1)^{k+1}} \cdot \frac{1}{e^{k-1}} \\ 
     & \ll \frac{x}{(\log x)^2} \cdot \frac{(\log\log x)^{k+1}}{(k-1)^{3/2}} \cdot 
     \frac{1}{e^{2k} (k-1)!} \\ 
     & \ll \frac{x}{(\log x)^2} \cdot \frac{(\log\log x)^{k-1}}{(k-1)!} \ll 
     \frac{x}{\log x} \cdot \frac{k \cdot (\log\log x)^{k-5}}{(k-1)!}. 
\end{align*} 
We must find an asymptotically accurate main term approximation to the coefficients 
of the following contour integral for $r \in [0, z_{\max}]$ where $z_{\max} < P(2)^{-1}$: 
\begin{align} 
\label{eqn_WideTildeArx_CountourIntDef_v1} 
\widetilde{A}_r(x) := 
     -\int_{|v|=r} \frac{x \cdot (\log x)^{-v} \zeta(2)^{v}}{(\log x) \Gamma(1+v) \cdot 
     v^{k} (1 + P(2) v)} dv. 
\end{align} 
Finding an exact formula for the derivatives of the function that is implicit to the 
Cauchy integral formula (CIF) for \eqref{eqn_WideTildeArx_CountourIntDef_v1} 
is complicated significantly by the need to differentiate $\Gamma(1+v)^{-1}$ 
up to any integer order $k$ in the formula. 
We can show that provided a restriction to 
$1 \leq r < 1$, we can approximate the contour integral in 
\eqref{eqn_WideTildeArx_CountourIntDef_v1} where 
the resulting main term is accurate up to a bounded constant factor. 
This procedure removes the 
gamma function term in the denominator of the integrand by essentially applying 
a mean value theorem type analog for smoothly 
parameterized contours. The logic used to justify this 
type of simplification of form argument is discussed next. 

We observe that for $r := 1$, the function $|\Gamma(1+re^{2\pi\imath t})|$ has a 
singularity (pole) when $t := \frac{1}{2}$. Thus we restrict the range of $|v| = r$ 
so that $0 \leq r < 1$ to necessarily avoid this problematic value of $t$ when 
we parameterize $v = r e^{2\pi\imath t}$ by a real-line integral over $t \in [0, 1]$. 
We can compute the finite extremal values of this function as 
\begin{align*} 
\min\limits_{\substack{0 \leq r < 1 \\ 0 \leq t \leq 1}} |\Gamma(1+re^{2\pi\imath t})| & = 
     |\Gamma(1+re^{2\pi\imath t})| \Biggr\rvert_{(r,t) \approx (1, 0.740592)} \approx 
     0.520089 \\ 
\max\limits_{\substack{0 \leq r < 1 \\ 0 \leq t \leq 1}} |\Gamma(1+re^{2\pi\imath t})| & = 
     |\Gamma(1+re^{2\pi\imath t})| \Biggr\rvert_{(r,t) \approx (1, 0.999887)} \approx 1. 
\end{align*} 
This shows that 
\begin{align} 
\label{eqn_WideTildeArx_CountourIntDef_v2} 
\widetilde{A}_r(x) \asymp 
     -\int_{|v|=r} \frac{x \cdot (\log x)^{-v} \zeta(2)^{v}}{(\log x) \cdot 
     v^{k} (1 + P(2) v)} dv, 
\end{align} 
where as $x \rightarrow \infty$ 
\[
\frac{\widetilde{A}_r(x)}{-\int_{|v|=r} \frac{x (\log x)^{-v} \zeta(2)^{v}}{(\log x) \cdot 
     v^{k} (1 + P(2) v)} dv} \in [1, 1.92275]. 
\] 
By induction we can compute the remaining coefficients 
$[z^k] \Gamma(1+z) \times \widehat{A}_z(x)$ with respect to 
$x$ for fixed $k \leq \log\log x$ using the CIF. 
Namely, it is not difficult to see that for any integer $m \geq 0$, 
we have the $m^{th}$ partial derivative of the integrand with respect to $z$ 
has the following limiting expansion by applying 
\eqref{eqn_IncompleteGamma_PropB}: 
\begin{align*} 
\frac{1}{m!} \times \frac{\partial^{(m)}}{{\partial v}^{(m)}}\left[ 
     \frac{(\log x)^{-v} \zeta(2)^{v}}{1 + P(2) v}\right] \Biggr\rvert_{v=0} & = 
     \sum_{j=0}^{m} \frac{(-1)^{m} P(2)^{j} (\log\log x - \log\zeta(2))^{m-j}}{(m-j)!} \\ 
     & = 
     \frac{(-P(2))^{m} (\log x)^{\frac{1}{P(2)}} \zeta(2)^{-\frac{1}{P(2)}}}{m!} \times 
     \Gamma\left(m+1, \frac{\log\log x - \log\zeta(2)}{P(2)}\right) \\ 
     & \sim \frac{(-1)^m (\log\log x -\log\zeta(2))^{m}}{m!}. 
\end{align*} 
Now by parameterizing the countour around $|z| = r := \frac{k-1}{\log\log x} < 1$ we 
deduce that the the main term of our approximation corresponds to 
\begin{align*} 
-\int_{|z|=r} \frac{x \cdot (\log x)^{-z} \zeta(2)^{z}}{(\log x) z^{k} (1 + P(2) z)} dz & \asymp 
     -\frac{x}{\log x} \cdot \frac{(-1)^{k-1} (\log\log x - \log\zeta(2))^{k-1}}{(k-1)!}. 
     \qedhere 
\end{align*} 
\end{proof} 

An exact DGF expression for 
$\lambda(n) C_{\Omega(n)}(n)$ is in fact very much complicated by the need to estimate the asymptotics 
of the coefficients of the more difficult right-hand-side product forms of 
\begin{align*} 
\sum_{n \geq 1} \frac{\lambda(n) C_{\Omega(n)}(n) z^{\Omega(n)}}{(\Omega(n))! \cdot n^s} & = 
     \prod_p \left(2 - \exp\left(-z \cdot p^{-s}\right)\right)^{-1}, 
     \Re(s) > 1, |z| < \log 2. 
\end{align*} 
It is unclear how to exactly, and effectively, bound the 
coefficients of powers of $z$ in the DGF expansion defined by the last equation. 
We use an alternate intermediate method in 
Corollary \ref{cor_SummatoryFuncsOfUnsignedSeqs_v2} 
to obtain the asymptotics for the 
summatory functions on which we require average case bounds. 

\begin{cor} 
\label{cor_SummatoryFuncsOfUnsignedSeqs_v2} 
We have that for large $x \geq 2$ that uniformly for $1 \leq k \leq \log\log x$ 
\begin{align*} 
\sum_{\substack{n \leq x \\ \Omega(n) = k}} C_{\Omega(n)}(n) & \asymp 
     2\sqrt{2\pi} \cdot x \times 
     \frac{(\log\log x)^{k+\frac{1}{2}}}{(2k+1)(k-1)!}. 
\end{align*} 
\end{cor} 
\begin{proof} 
We have an integral formula involving the non-sign-weighted 
sequence that results by 
applying ordinary Abel summation (and integrating by parts) in the form of 
\begin{align} 
\label{eqn_AbelSummationIBPReverseFormula_stmt_v1} 
\sum_{n \leq x} \lambda_{\ast}(n) h(n) & = \left(\sum_{n \leq x} \lambda_{\ast}(n)\right) h(x) - 
     \int_{1}^{x} \left(\sum_{n \leq t} \lambda_{\ast}(n)\right) h^{\prime}(t) dt \\ 
\notag 
     & \left\{\begin{array}{ll} 
     u_t = L_{\ast}(t) & v_t^{\prime} = h^{\prime}(t) dt \\ 
     u_t^{\prime} = L_{\ast}^{\prime}(t) dt & v_t = h(t) 
     \end{array} 
     \right\} \\ 
     & \asymp 
     \int_1^{x} \frac{d}{dt}\left[\sum_{n \leq t} \lambda_{\ast}(n)\right] h(t) dt. 
\end{align} 
Let the signed left-hand-side summatory function for our function 
corresponding to \eqref{eqn_AbelSummationIBPReverseFormula_stmt_v1} be defined by 
\begin{align*} 
\widehat{C}_{k,\ast}(x) & := \sum_{\substack{n \leq x \\ \Omega(n)=k}} 
     (-1)^{\omega(n)} C_{\Omega(n)}(n) \\ 
     & \phantom{:} \asymp  
     -\frac{x}{\log x} \cdot \frac{(\log\log x - \log\zeta(2))^{k-1}}{(k-1)!} \left[ 
     1 + O\left(\frac{1}{\log\log x}\right)\right] \\ 
     & \phantom{:} \asymp 
     -\frac{x}{\log x} \cdot \frac{(\log\log x)^{k-1}}{(k-1)!} \left[ 
     1 + O\left(\frac{1}{\log\log x}\right)\right]
\end{align*} 
where the second equation above follows from the proof of 
Theorem \ref{theorem_CnkSpCasesScaledSummatoryFuncs}. 

We handle transforming our previous results for the partial sums over the unsigned sequence 
$C_{\Omega(n)}(n)$ such that $\Omega(n) = k$. 
The argument is based on approximating the smooth summatory function of 
$\lambda_{\ast}(n) := (-1)^{\omega(n)}$ using the following 
uniform approximation of $\pi_k(x)$ when $1 \leq k \leq \log\log x$ as 
$x \rightarrow \infty$: 
\[
\pi_k(x) \asymp \frac{x}{\log x} \frac{(\log\log x)^{k-1}}{(k-1)!} (1 + o(1)). 
\]
In particular, we have that 
(compare to Table \ref{table_LAstxSummatoryFuncCompsWithExact_v2} starting on page 
\pageref{table_LAstxSummatoryFuncCompsWithExact_v2}) 
\begin{align*} 
L_{\ast}(t) & := \left\lvert \sum_{n \leq t} (-1)^{\omega(n)} \right\rvert = 
     \left\lvert \sum_{k=1}^{\log\log x} (-1)^{k} \pi_k(x) \right\rvert \sim 
     \frac{t}{\sqrt{2\pi} \sqrt{\log\log t}}, \mathrm{\ as\ } t \rightarrow \infty. 
\end{align*} 
The main term for the reciprocal of the derivative of this summatory function is given by 
\[
\frac{1}{L_{\ast}^{\prime}(t)} \asymp \sqrt{2\pi} \cdot (\log\log t)^{\frac{1}{2}}. 
\]
After applying the formula from \eqref{eqn_AbelSummationIBPReverseFormula_stmt_v1},  
we deduce that the unsigned summatory function variant satisfies 
\begin{align*} 
\widehat{C}_{k,\ast}(x) & = \int_1^{x} L_{\ast}^{\prime}(t) C_{\Omega(t)}(t) dt \qquad \implies 
C_{\Omega(x)}(x) \asymp \frac{\widehat{C}_{k,\ast}^{\prime}(x)}{L_{\ast}^{\prime}(x)} \qquad \implies \\ 
C_{\Omega(x)}(x) & \asymp \sqrt{2\pi} \cdot \frac{(\log\log x)^{\frac{1}{2}}}{\log x} \cdot  
     \left[\frac{(\log\log x)^{k-1}}{(k-1)!} \left(1 - 
     \frac{1}{\log x}\right) + \frac{(\log\log x)^{k-2}}{(\log x) (k-2)!}\right] \\ 
     & \asymp \sqrt{2\pi} \cdot \frac{(\log\log x)^{k-\frac{1}{2}}}{(\log x) (k-1)!} 
     =: \widehat{C}_{k,\ast\ast}(x). 
\end{align*} 
So applying to the ordinary Abel summation formula, and integrating by parts, we obtain that 
the main term for this function is given by 
\begin{align*} 
\sum_{\substack{n \leq x \\ \Omega(n)=k}} C_{\Omega(n)}(n) & \asymp 
     \int \widehat{C}_{k,\ast\ast}(x)dx \\ 
     & \asymp 2\sqrt{2\pi} \cdot x \times 
     \frac{(\log\log x)^{k+\frac{1}{2}}}{(2k+1)(k-1)!}. 
     \qedhere 
\end{align*} 
\end{proof}

\begin{lemma} 
\label{lemma_HatCAstxSum_ExactFormulaWithError_v1} 
We have that as $x \rightarrow \infty$ 
\[
\mathbb{E}\left[C_{\Omega(n)}(n)\right] \asymp 2\sqrt{2\pi} \cdot (\log x) \sqrt{\log\log x}. 
\] 
\end{lemma} 
\begin{proof} 
We first compute the absolute value of the following 
summatory function by applying 
Corollary \ref{cor_SummatoryFuncsOfUnsignedSeqs_v2}:
\begin{align} 
\label{eqn_proof_tag_PartialSumsOver_HatCkx_v3} 
\sum_{k=1}^{\log\log x} \sum_{\substack{n \leq x \\ \Omega(n) = k}} C_{\Omega(n)}(n) & \asymp 
     2\sqrt{2\pi} \cdot x \cdot (\log x) \sqrt{\log\log x}. 
\end{align} 
We claim that 
\begin{equation} 
\label{eqn_proof_tag_PartialSumsOver_HatCkx_v1} 
\sum_{n \leq x} C_{\Omega(n)}(n) = 
     \sum_{k=1}^{\log_2(x)} \sum_{\substack{n \leq x \\ \Omega(n) = k}} C_{\Omega(n)}(n) \asymp 
     \sum_{k=1}^{\log\log x} \sum_{\substack{n \leq x \\ \Omega(n) = k}} C_{\Omega(n)}(n). 
\end{equation} 
To prove \eqref{eqn_proof_tag_PartialSumsOver_HatCkx_v1}, it suffices to show that 
\begin{equation} 
\label{eqn_proof_tag_PartialSumsOver_HatCkx_EquivCond_v2} 
\frac{\sum\limits_{\log\log x < k \leq \log_2(x)} \sum\limits_{\substack{n \leq x \\ \Omega(n) = k}} C_{\Omega(n)}(n)}{ 
     \sum\limits_{k=1}^{\log\log x} \sum\limits_{\substack{n \leq x \\ \Omega(n) = k}} C_{\Omega(n)}(n)} = o(1), 
     \mathrm{\ as\ } x \rightarrow \infty. 
\end{equation} 
Next, define the following component sums for large $x$ and $0 < \varepsilon < 1$ so that 
$(\log\log x)^{\frac{\varepsilon \log\log x}{\log\log\log x}} = o(\log x)$: 
\begin{align*} 
S_{2,\varepsilon}(x) & := \sum_{\log\log x < k \leq \log x} \sum_{\substack{n \leq x \\ \Omega(n) = k}} C_{\Omega(n)}(n). 
\end{align*} 
Then 
\[
\sum_{k=1}^{\log_2(x)} \sum_{\substack{n \leq x \\ \Omega(n) = k}} C_{\Omega(n)}(n) 
     \gg S_{2,\varepsilon}(x),  
\]
with equality as $\varepsilon \rightarrow 1$ when the upper bound of summation tends to $\log x$. 
To show that \eqref{eqn_proof_tag_PartialSumsOver_HatCkx_EquivCond_v2} holds, 
observe that whenever $\Omega(n) = k$, we have that $C_{\Omega(n)}(n) \leq k!$. 
We can then bound the sum defined above using 
Theorem \ref{theorem_GFs_SymmFuncs_SumsOfRecipOfPowsOfPrimes} and 
Theorem \ref{theorem_MV_Thm7.20-init_stmt} for large $x \rightarrow \infty$ as 
\begin{align*} 
S_{2,\varepsilon}(x) & \leq 
     \sum_{\log\log x \leq k \leq \log x} \sum_{\substack{n \leq x \\ \Omega(n)=k}} C_{\Omega(n)}(n) 
     \ll \sum_{k=\log\log x}^{(\log\log x)^{\frac{\varepsilon \log\log x}{\log\log\log x}}} 
      \frac{\widehat{\pi}_k(x)}{x} \cdot k! \\ 
     & \ll \sum_{k=\log\log x}^{(\log\log x)^{\frac{\varepsilon \log\log x}{\log\log\log x}}} 
     (\log x)^{\frac{k}{\log\log x} - 1 - \frac{k}{\log\log x} \left( 
     \log k - \log\log\log x\right)} \cdot \left(\frac{k}{e}\right)^{k} \sqrt{2\pi k} \\ 
     & \ll \sum_{k=\log\log x}^{\frac{\varepsilon \log\log x}{\log\log\log x}} 
     (\log x)^{\frac{2k \cdot \log\log\log x}{\log\log x} - 1} \sqrt{k} \\ 
     & \ll \frac{1}{(\log x)} \times \int_{\log\log x}^{ 
     \frac{\varepsilon \log\log x}{\log\log\log x}} (\log\log x)^{2t} \sqrt{t} \cdot dt \\ 
     & \ll \frac{1}{(\log x)} \sqrt{\frac{\varepsilon \cdot \log\log x}{\log\log\log x}} 
     (\log\log x)^{\frac{2\varepsilon \cdot \log\log x}{\log\log\log x}} = o(x), 
\end{align*} 
where we have a simplification by noticing that 
$\lim_{x \rightarrow \infty} (\log x)^{\frac{1}{\log\log x}} = e$. 
So by \eqref{eqn_proof_tag_PartialSumsOver_HatCkx_v3} this 
form of the ratio in \eqref{eqn_proof_tag_PartialSumsOver_HatCkx_EquivCond_v2} clearly tends to zero. 
If we have a contribution from the terms as $\varepsilon \rightarrow 1$, 
e.g., if $x$ is a power of two, then $C_{\Omega(x)}(x) = 1$ by the formula in 
\eqref{eqn_proof_tag_hInvn_ExactNestedSumFormula_CombInterpetIdent_v3}, so that 
the contribution from this upper-most indexed term is negligible: 
\[
x=2^k \implies \Omega(x) = k \implies C_{\Omega(x)}(x) = \frac{(\Omega(x))!}{k!} = 1. 
\]
The formula for the expectation claimed in the statement of this lemma above then 
follows from \eqref{eqn_proof_tag_PartialSumsOver_HatCkx_v3} by scaling by 
$\frac{1}{x}$ and dropping the asymptotically lesser error terms in the bound. 
\end{proof} 

\begin{cor}[Expectation formulas] 
\label{cor_ExpectationFormulaAbsgInvn_v2} 
We have that as $n \rightarrow \infty$ 
\begin{align*} 
\mathbb{E}|g^{-1}(n)| & \asymp (\log n)^2 \sqrt{\log\log n}. 
\end{align*} 
\end{cor} 
\begin{proof} 
We use the formula from Lemma \ref{lemma_HatCAstxSum_ExactFormulaWithError_v1} 
to find $\mathbb{E}[C_{\Omega(n)}(n)]$ up to a small bounded multiplicative 
constant factor as $n \rightarrow \infty$.
This implies that for large $x$ 
\begin{align*} 
\int \frac{\mathbb{E}[C_{\Omega(t)}(t)]}{t} dt & \asymp 
     \sqrt{2\pi} \cdot (\log t)^2 \sqrt{\log\log t} - 
     \frac{\pi}{2} \operatorname{erfi}\left(\sqrt{2\log\log t}\right) \\ 
     & \asymp \sqrt{2\pi} \cdot (\log t)^2 \sqrt{\log\log t}. 
\end{align*} 
Recall from the introduction that the summatory function of the 
squarefree integers is given by 
\[
Q(x) := \sum_{n \leq x} \mu^2(n) = \frac{6x}{\pi^2} + O(\sqrt{x}). 
\]
Therefore summing over \eqref{eqn_AbsValueOf_gInvn_FornSquareFree_v1} 
we find that  
\begin{align} 
\notag 
\mathbb{E}|g^{-1}(n)| & = \frac{1}{n} \times \sum_{d \leq n} 
     C_{\Omega(d)}(d) Q\left(\Floor{n}{d}\right) \\ 
\notag 
     & \sim \sum_{d \leq n} C_{\Omega(d)}(d) \left[\frac{6}{d \cdot \pi^2} + O\left(\frac{1}{\sqrt{dn}}\right) 
     \right] \\ 
\notag 
     & = \frac{6}{\pi^2} \left[\mathbb{E}[C_{\Omega(n)}(n)] + \sum_{d<n} 
     \frac{\mathbb{E}[C_{\Omega(d)}(d)]}{d}\right] + 
     O\left(\frac{1}{\sqrt{n}} \times \int_0^{n} t^{-1/2} dt\right) \\ 
\label{eqn_proof_tag_ExpAbsgInvn_ExpandedByExpsOfCkn_v1} 
     & = \frac{6}{\pi^2}\left[\mathbb{E}[C_{\Omega(n)}(n)] + 
     \sum_{d<n} \frac{\mathbb{E}[C_{\Omega(d)}(d)]}{d}\right] + O(1) \\ 
\notag 
     & \asymp \frac{6 \sqrt{2}}{\pi^{\frac{3}{2}}} (\log n)^2 \sqrt{\log\log n}. 
     \qedhere 
\end{align} 
\end{proof} 

\begin{theorem} 
\label{theorem_CLT_VI} 
Let the mean and variance analogs be denoted by 
\[
\mu_x(C) := \log\log x + \hat{a} - \frac{1}{2} \cdot \log\log\log x, 
     \qquad \mathrm{\ and\ } \qquad 
     \sigma_x(C) := \sqrt{\mu_x(C)}, 
\]
where the absolute constant 
$\hat{a} := \log\left(\frac{1}{\sqrt{2\pi}}\right) \approx -0.918939$. 
Set $Y > 0$ and suppose that $z \in [-Y, Y]$. Then we have 
uniformly for all $-Y \leq z \leq Y$ that 
\[
\frac{1}{x} \cdot \#\left\{2 \leq n \leq x: 
     \frac{C_{\Omega(n)}(n) - \mu_x(C)}{\sigma_x(C)} \leq z\right\} = 
     \Phi(z) + O\left(\frac{1}{\sqrt{\log\log x}}\right), 
     \mathrm{\ as\ } x \rightarrow \infty. 
\] 
\end{theorem} 
\begin{proof} 
For large $x$ and $n \leq x$, define the following auxiliary variables: 
\[
\alpha_n := \frac{C_{\Omega(n)}(n) - \mu_n(C)}{\sigma_n(C)}, \quad\mathrm{and}\quad 
     \beta_{n,x} := \frac{C_{\Omega(n)}(n) - \mu_x(C)}{\sigma_x(C)}. 
\] 
Let the corresponding densities (whose limiting distributions we must verify) 
be defined by the functions 
\[
\Phi_1(x, z) := \frac{1}{x} \cdot \#\{n \leq x: \alpha_n \leq z\}, 
\]
and 
\[
\Phi_2(x, z) := \frac{1}{x} \cdot \#\{n \leq x: \beta_{n,x} \leq z\}. 
\] 
We first argue that it suffices to consider the distribution of $\Phi_2(x, z)$ as 
$x \rightarrow \infty$ in place of $\Phi_1(x, z)$ to obtain our desired result. 
The difference of the two auxiliary variables is neglibible as 
$x \rightarrow \infty$ for $(n,x)$ taken over the ranges that contribute the non-trivial 
weight to the main term of each density function. In particular, we have for 
$\sqrt{x} \leq n \leq x$ and $C_{\Omega(n)}(n) \leq 2 \cdot \mu_x(C)$ that the 
following is true: 
\[
|\alpha_n - \beta_{n,x}| \ll \frac{1}{\sigma_x(C)} \xrightarrow{x \rightarrow \infty} 0. 
\]
Thus we can replace $\alpha_n$ by $\beta_{n,x}$ and estimate the limiting 
densities corresponding to the alternate terms. 
The rest of our argument follows the method in the proof of the related theorem in 
\cite[Thm.\ 7.21; \S 7.4]{MV} closely. Readers familiar with the methods in the 
reference will see many parallels to that construction. 

We use the formula proved in Corollary \ref{cor_SummatoryFuncsOfUnsignedSeqs_v2} 
to estimate the densities claimed within the ranges bounded by 
$z$ as $x \rightarrow \infty$. 
Let $k \geq 1$ be a natural number defined by $k := t + \mu_x(C)$. 
We write the small parameter $\delta_{t,x} := \frac{t}{\mu_x(C)}$. 
When $|t| \leq \frac{1}{2} \mu_x(C)$, we have by Stirling's formula that 
\begin{align*} 
2\sqrt{2\pi} \cdot x \times 
     \frac{(\log\log x)^{k+\frac{1}{2}}}{(2k+1)(k-1)!} & \sim 
     \frac{e^{\hat{a} + t} 
     (\log\log x)^{\mu_x(C)(1+\delta_{t,x})}}{ 
     \sigma_x(C) \cdot \mu_x(C)^{\mu_x(C) (1 + \delta_{t,x})}
     (1 + \delta_{t,x})^{\mu_x(C) (1 + \delta_{t,x}) + \frac{3}{2}}} \\ 
     & \sim \frac{e^{t}}{\sqrt{2\pi} \cdot \sigma_x(C)} (1 + \delta_{t,x})^{-\left( 
     \mu_x(C) (1 + \delta_{t,x}) + \frac{3}{2}\right)}, 
\end{align*} 
since $\frac{\mu_x(C)}{\log\log x} = 1 + o(1)$ as $x \rightarrow \infty$. 

We have the uniform estimate 
$\log(1+\delta_{t,x}) = \delta_{t,x} - \frac{\delta_{t,x}^2}{2} + O(|\delta_{t,x}|^3)$ whenever 
$|\delta_{t,x}| \leq \frac{1}{2}$. Then we can expand the factor involving $\delta_{t,x}$ 
in the previous equation as follows: 
\begin{align*} 
(1+\delta_{t,x})^{-\mu_x(C) (1+\delta_{t,x}) - \frac{1}{2}} & = 
     \exp\left(\left(\frac{1}{2}+\mu_x(C) (1+\delta_{t,x})\right) \times 
     \left(-\delta_{t,x} + \frac{\delta_{t,x}^2}{2} + O(|\delta_{t,x}|^3)\right)\right) \\ 
     & = \exp\left(-t - \frac{3t+t^2}{2\mu_x(C)} + \frac{3t^2}{4\mu_x(C)^2} + 
     O\left(\frac{|t|^3}{\mu_x(C)^2}\right)\right). 
\end{align*} 
For both $|t| \leq \mu_x(C)^{1/2}$ and 
$\mu_x(C)^{1/2} < |t| \leq \mu_x(C)^{2/3}$, 
we see that 
\[
\frac{t}{\mu_x(C)} \ll \frac{1}{\sqrt{\mu_x(C)}} + \frac{|t|^3}{\mu_x(C)^2}. 
\]
Similarly, for $|t| \leq 1$ and $|t| > 1$, we see that both 
\[
\frac{t^2}{\mu_x(C)^2} \ll \frac{1}{\sqrt{\mu_x(C)}} + 
     \frac{|t|^3}{\mu_x(C)^2}. 
\] 
Let the corresponding error terms in $(x, t)$ be denoted by 
\[
\widetilde{E}(x, t) := O\left(\frac{1}{\sigma_x(C)}\right) + 
     O\left(\frac{|t|^3}{\mu_x(C)^2}\right). 
\]
Combining these estimates with the previous computations, we can deduce that 
uniformly for $|t| \leq \mu_x(C)^{2/3}$ 
\begin{align*} 
2\sqrt{2\pi} \cdot x \times 
     \frac{(\log\log x)^{k+\frac{1}{2}}}{(2k+1)(k-1)!} & \sim 
     \frac{1}{\sqrt{2\pi} \cdot \sigma_x(C)} 
     \cdot \exp\left(-\frac{t^2}{2\sigma_x(C)^2}\right) \times 
     \left[1 + \widetilde{E}(x, t)\right]. 
\end{align*} 
By the same argument utilized in the proof of 
Lemma \ref{lemma_HatCAstxSum_ExactFormulaWithError_v1}, we see that 
the contributions of these summatory functions for 
$k \leq \mu_x(C) - \mu_x(C)^{2/3}$ is negligible. 
We also require that $k \leq \log\log x$ as we have worked out in 
Theorem \ref{theorem_CnkSpCasesScaledSummatoryFuncs}. So we sum over a 
corresponding range of 
\[
\mu_x(C) -\mu_x(C)^{2/3} \leq k \leq R_{z,x} \cdot \mu_x(C) + z \cdot \sigma_x(C), 
\] 
for $R_{z,x} := 1 - \frac{z}{\sigma_x(C)}$ to approximate the 
stated normalized densities. 
Then finally as $x \rightarrow \infty$, the 
three terms that result (one main term, two error terms, respectively) 
can be considered to each correspond to a Riemann sum for an associated integral. 
\end{proof} 

\begin{cor} 
\label{cor_CLT_VII} 
Let $Y > 0$. 
Uniformly for all $-Y \leq y \leq Y$ 
we have that 
\begin{align*} 
\frac{1}{x} \cdot \#\left\{2 \leq n \leq x:|g^{-1}(n)| - 
     \frac{6}{\pi^2} \mathbb{E}|g^{-1}(n)| \leq y\right\} & = 
     \Phi\left(\frac{\frac{\pi^2}{6} y - \mu_x(C)}{\sigma_x(C)}\right) + 
     O\left(\frac{1}{\sqrt{\log\log x}}\right), 
     \mathrm{\ as\ } x \rightarrow \infty. 
\end{align*} 
\end{cor} 
\begin{proof} 
We claim that 
\begin{align*} 
|g^{-1}(n)| - \frac{6}{\pi^2} \mathbb{E}|g^{-1}(n)| & \sim \frac{6}{\pi^2} C_{\Omega(n)}(n). 
\end{align*} 
From \eqref{eqn_proof_tag_ExpAbsgInvn_ExpandedByExpsOfCkn_v1}
we obtain that 
\begin{align*} 
\frac{1}{x} \times \sum_{n \leq x} |g^{-1}(n)| & = 
     \frac{6}{\pi^2} \left[\mathbb{E}[C_{\Omega(x)}(x)] + \sum_{d<x} 
     \frac{\mathbb{E}[C_{\Omega(d)}(d)]}{d}\right] + O(1). 
\end{align*} 
Let the \emph{backwards difference operator} with respect to $x$ 
be defined for $x \geq 2$ and any arithmetic function $f$ as 
$\Delta_x(f(x)) := f(x) - f(x-1)$. Then from the proof of 
Corollary \ref{cor_ExpectationFormulaAbsgInvn_v2}, 
we see that for large $n$ 
\begin{align*} 
|g^{-1}(n)| & = \Delta_n(n \cdot \mathbb{E}|g^{-1}(n)|) 
     \sim \Delta_n\left(\sum_{d \leq n} \frac{6}{\pi^2} \cdot C_{\Omega(d)}(d) \cdot \frac{n}{d}\right) \\ 
     & = \frac{6}{\pi^2}\left[C_{\Omega(n)}(n) + \sum_{d < n} C_{\Omega(d)}(d) \frac{n}{d} - 
     \sum_{d<n} C_{\Omega(d)}(d) \frac{(n-1)}{d}\right] \\ 
     & = \frac{6}{\pi^2} C_{\Omega(n)}(n) + \frac{6}{\pi^2} \mathbb{E}|g^{-1}(n-1)|, 
     \mathrm{\ as\ } n \rightarrow \infty. 
\end{align*} 
The result finally follows from Theorem \ref{theorem_CLT_VI}. 
\end{proof} 

\newpage 
\section{Lower bounds for $M(x)$ along infinite subsequences} 
\label{Section_KeyApplications} 

\subsection{Establishing initial lower bounds on the summatory function $G^{-1}(x)$} 
\label{Section_ProofOfValidityOfAverageOrderLowerBounds} 

\begin{lemma} 
\label{lemma_ProbsOfAbsgInvnDist_v2} 
If $x$ is sufficiently large and we pick any integer $n \in [2, x]$ uniformly at random, then 
each of the following statements holds: 
\begin{align*} 
\tag{A}
\mathbb{P}\left(|g^{-1}(n)| - \frac{6}{\pi^2} \mathbb{E}|g^{-1}(n)| \leq 0\right) & = o(1) \\ 
\tag{B} 
\mathbb{P}\left(|g^{-1}(n)| - \frac{6}{\pi^2} \mathbb{E}|g^{-1}(n)| \leq \frac{6}{\pi^2} \mu_x(C)\right) & = 
     \frac{1}{2} + o(1). 
\end{align*} 
Moreover, for any positive real $\delta > 0$ we have that 
\begin{align*} 
\tag{C} 
\mathbb{P}\left(|g^{-1}(n)| - \frac{6}{\pi^2} \mathbb{E}|g^{-1}(n)| \leq \frac{6}{\pi^2} \mu_x(C)^{1 + \delta}\right) & = 
     1 + o_{\delta}(1), 
     \mathrm{\ as\ } x \rightarrow \infty. 
\end{align*} 
\end{lemma} 
\begin{proof} 
Each of these results is a consequence of Corollary \ref{cor_CLT_VII}. 
Let the densities $\gamma_z(x)$ be defined for $z \in \mathbb{R}$ and 
large $x > e$ as follows: 
\[
\gamma_z(x) := \frac{1}{x} \cdot \#\{2 \leq n \leq x: |g^{-1}(n)| - \mathbb{E}|g^{-1}(n)| \leq z\}. 
\]
To prove (A), observe that by Corollary \ref{cor_CLT_VII} for $z := 0$ we have that 
\[
\gamma_0(x) = \Phi\left(-\sigma_x(C)\right) + o(1), \mathrm{\ as\ } x \rightarrow \infty. 
\]
We can see that $\sigma_x(C) \xrightarrow{x \rightarrow \infty} +\infty$ where for $z \geq 0$ we have the 
reflection identity $\Phi(z) = 1 - \Phi(-z)$. Combined we have by an asymptotic approximation 
to the error function expanded by 
\begin{align*} 
\Phi(z) & = \frac{1}{2}\left(1 + \operatorname{erf}\left(\frac{z}{\sqrt{2}}\right)\right) \\ 
     & = 1 - \frac{2e^{-z^2/2}}{\sqrt{2\pi}}\left[ 
     z^{-1} - z^{-3} + 3z^{-5} - 15z^{-7} + \cdots 
     \right], \mathrm{\ as\ } |z| \rightarrow \infty, 
\end{align*} 
that 
\[
\gamma_0(x) = \Phi\left(-\sigma_x(C)\right) \asymp \frac{1}{\sigma_x(C) \exp(\mu_x(C)/2)} = o(1). 
\]
To prove (B), observe that setting $z_1 := \frac{6}{\pi^2} \mu_x(C)$ yields 
\[
\gamma_{z_1}(x) = \Phi(0) + o(1) = \frac{1}{2} + o(1), \mathrm{\ as\ } x \rightarrow \infty. 
\]
To prove (C), we require that 
$\mu_x(C)^{\frac{1}{2} + \delta} - \sigma_x(C) \xrightarrow{x \rightarrow \infty} +\infty$. 
Since this happens as $x \rightarrow \infty$ for any fixed $\delta > 0$, we have that 
with $z(\delta) := \frac{6}{\pi^2} \mu_x(C)^{1 + \delta}$ 
\begin{align*} 
\gamma_{z(\delta)} & = \Phi\left(\mu_x(C)^{\frac{1}{2} + \delta} - \sigma_x(C)\right) + o(1) \\ 
     & \sim 1 - \frac{1}{\sqrt{2\pi}} \cdot \frac{1}{\left(\mu_x(C)^{\frac{1}{2} + \delta} - \sigma_x(C)\right)} \times 
     \exp\left(-\frac{\mu_x(C)}{4} \cdot \left(\mu_x(C)^{\delta} - 1\right)^2\right) \\ 
     & = 1 + o_{\delta}(1), \mathrm{\ as\ } x \rightarrow \infty. 
     \qedhere 
\end{align*} 
\end{proof} 

\begin{remark}[Interpretations for constructing bounds on $G^{-1}(x)$] 
A consequence of (A) and (C) in Lemma \ref{lemma_ProbsOfAbsgInvnDist_v2} 
is that for any fixed $\delta > 0$ and $n \in \mathcal{S}_1(\delta)$ 
taken within a set of asymptotic density one we have that 
\begin{equation} 
\label{eqn_AbsgInvn_ProbEffectiveBounds_v3} 
\frac{6}{\pi^2}\mathbb{E}|g^{-1}(n)| \leq |g^{-1}(n)| \leq \frac{6}{\pi^2}\mathbb{E}|g^{-1}(n)| + \frac{6}{\pi^2} 
     \mu_x(C)^{\frac{1}{2} + \delta}. 
\end{equation} 
Thus when we integrate over a sufficiently spaced set of disjoint consecutive intervals 
containing large enough integer values, 
we can assume that an asymptotic lower bound on the 
contribution of $|g^{-1}(n)|$ is given by its average 
order, and an upper bound is given by the upper limit above for any fixed $\delta > 0$. 
In particular, observe that by 
Corollary \ref{cor_CLT_VII} and 
Corollary \ref{cor_ExpectationFormulaAbsgInvn_v2} 
we can see that 
\begin{align*} 
\frac{\pi^2}{6 \cdot \sigma_x(C)} \times \int_{-\infty}^{\infty} z \cdot 
     \Phi^{\prime}\left(\frac{\frac{\pi^2}{6} z - \mu_x(C)}{\sigma_x(C)}\right) dz & = 
     \frac{6}{\pi^2} \cdot \sigma_x(C) = o\left(\mathbb{E}|g^{-1}(x)|\right). 
\end{align*} 
Emphasizing the point above, 
we can thus again interpret the previous calculation as implying that for $n$ on a large 
interval, the contribution from $|g^{-1}(n)|$ can be approximated above and below 
accurately as in the bounds from \eqref{eqn_AbsgInvn_ProbEffectiveBounds_v3}. 
\end{remark} 

\begin{theorem} 
\label{theorem_GInvxLowerBoundByGEInvx_v1} 
For all sufficiently large integers $x$, whenever $G^{-1}(x) \neq 0$ we have that 
\[
|G^{-1}(x)| \gg (\log x) \sqrt{\log\log x}, \mathrm{\ as\ } x \rightarrow \infty. 
\]
\end{theorem} 
\begin{proof} 
We will require a couple of observations to sum $G^{-1}(x)$ 
in absolute value and bound it from below. 
We will use a lower bound approximating the summatory function of $\lambda(n)$ for $n \leq t$ and 
$t$ large by summing over the uniform asymptotic lower bounds proved in 
Theorem \ref{theorem_GFs_SymmFuncs_SumsOfRecipOfPowsOfPrimes}. 
To be careful about the expected sign of this summatory function, 
we first appeal to the original uniform approximations to the 
functions $\widehat{\pi}_k(x)$ 
given by Theorem \ref{theorem_HatPi_ExtInTermsOfGz}. 
As noted in \cite[\S 7.4]{MV}, the function $\mathcal{G}(z)$ satisfies 
\[
\mathcal{G}\left(\frac{k-1}{\log\log x}\right) = O(1), 1 \leq k \leq \log\log x, 
\]
so that uniformly for $1 \leq k \leq \log\log x$ we can write 
\[
\widehat{\pi}_k(x) \asymp \frac{x}{\log x} \cdot \frac{(\log\log x)^{k-1}}{(k-1)!} \left[ 
     1 + O\left(\frac{1}{\log\log x}\right)\right]. 
\]
By Corollary \ref{theorem_MV_Thm7.20}, the 
following summatory function represents the asymptotic main term 
in the summation $L(x) := \sum_{n \leq x} \lambda(n)$ as $x \rightarrow \infty$ 
(see Table \ref{table_LAstxSummatoryFuncCompsWithExact_v2} on page 
\pageref{table_LAstxSummatoryFuncCompsWithExact_v2}): 
\begin{align*} 
\widehat{L}_2(x) & = \sum_{k=1}^{\log\log x} (-1)^{k} \widehat{\pi}_k(x) 
     = - \frac{x}{(\log x)^2} \times \Gamma(\log\log x, -\log\log x) 
     \sim \frac{(-1)^{1+\ceiling{\log\log x}} \cdot x}{\sqrt{2\pi} \sqrt{\log\log x}}
\end{align*} 
So we expect the sign of our summatory function approximation to be approximately 
given by $(-1)^{1+\ceiling{\log\log x}}$ for sufficiently large $x$. 

We now find a lower bound on the unsigned magnitude of these summatory functions. 
In particular, using Theorem \ref{theorem_GFs_SymmFuncs_SumsOfRecipOfPowsOfPrimes}, 
we have that $\widehat{\pi}_k(x) \gg \widehat{\pi}_k^{(\ell)}(x)$ where 
\[
\widehat{\pi}_k^{(\ell)}(x) := \frac{x^{\frac{3}{4}}}{\log x} \cdot 
     \frac{(\log\log x)^{k-1}}{(k-1)!} \left[1 + 
     O\left(\frac{k}{(\log\log x)^2}\right)\right].
\]
Thus we define our lower bound by 
\[
\widehat{L}_0(x) := \left\lvert \sum_{k=1}^{\log\log x} (-1)^{k} \widehat{\pi}_k^{(\ell)}(x) \right\rvert 
     \asymp \frac{x^{\frac{3}{4}}}{\sqrt{\log\log x}}, 
\]
where limiting asymptotics for the incomplete gamma function expression for the inner summation dominates the 
main term in the previous equation. 
The derivative of this summatory function satisfies 
\[
\widehat{L}_0^{\prime}(x) \asymp \frac{1}{x^{\frac{1}{4}} \cdot \sqrt{\log\log x}}. 
\]
We observe that we can break the interval $t \in (e, x]$ into disjoint subintervals 
according to which we have the expected sign contributions from the 
summatory function $\widehat{L}_0(x)$. Namely, we expect that 
for $1 \leq k \leq \frac{\log\log x}{2}$ we expect that 
(compare to Table \ref{table_LAstxSummatoryFuncCompsWithExact_v2}) 
\begin{align*} 
\operatorname{sgn}\left(\widehat{L}_0(x)\right) = -1 & \mathrm{\ on\ } 
     \left[e^{e^{2k}}, e^{e^{2k+1}}\right) \\ 
\operatorname{sgn}\left(\widehat{L}_0(x)\right) = +1 & \mathrm{\ on\ } 
     \left[e^{e^{2k+1}}, e^{e^{2k+2}}\right). 
\end{align*} 
Moreover, since the derivative $\widehat{L}_0^{\prime}(x)$ is monotone decreasing in $x$, 
we can construct our lower bounds by placing the input points to this function 
in the Abel summation formula from 
\eqref{eqn_AbelSummationIBPReverseFormula_stmt_v1} 
over these signed intervals at the extremal endpoints depending on the leading sign terms. 
As we have argued in Lemma \ref{lemma_ProbsOfAbsgInvnDist_v2} and 
observed in the preceeding remark, we have the bounds in 
\eqref{eqn_AbsgInvn_ProbEffectiveBounds_v3} upon which 
we can similarly construct the lower bound on $|G^{-1}(x)|$ based on the 
sign term of the subinterval (as above) and the extremal points within the interval. 
The idea used to conclude this proof below is to underestimate the resulting 
integral formula beyond all reasonable doubt. 

Let $I_k := \left[e^{e^{k}}, e^{e^{k+1}}\right)$. 
By Lemma \ref{lemma_ProbsOfAbsgInvnDist_v2} and its consequence stated in 
\eqref{eqn_AbsgInvn_ProbEffectiveBounds_v3}, we know that for sufficiently large $k$ 
for approximately one half of the integers $n$ in the interval $I_k$ satisfy 
$|g^{-1}(n)| \leq k$. Thus by applying the inequalities from 
\eqref{eqn_facts_ExpIntegralEiInequalities_v12} in the last step below 
we have the following asymptotic lower bounds: 
\begin{align*} 
\notag 
|G^{-1}(x)| & \gg \left\lvert \int_2^x \widehat{L}_0^{\prime}(t) |g^{-1}(t)| dt \right\rvert \\ 
     & \gg \sum_{k=1}^{\frac{\log\log x}{2}} 
     (-1)^{k} \widehat{L}_0^{\prime}\left(e^{e^{k}}\right) 
     \left\lvert g^{-1}\left(e^{e^{k}}\right) \right\rvert \\ 
     & \gg \left\lvert \sum_{k=\frac{\log\log x}{4}}^{\frac{\log\log x}{2}} \left[ 
     \widehat{L}_0^{\prime}\left(e^{e^{2k}}\right) \left\lvert g^{-1}\left(e^{e^{2k}}\right) \right\rvert - 
      \widehat{L}_0^{\prime}\left(e^{e^{2k-1}}\right) \left\lvert g^{-1}\left(e^{e^{2k-1}}\right) \right\rvert
     \right] 
     \right\rvert \\ 
     & \gg \left\lvert \sum_{k=\frac{\log\log x}{4}}^{\frac{\log\log x}{2}} \left[ 
     \sqrt{2k} \exp\left(-\frac{1}{4} e^{2k}\right) - 
     \sqrt{2k-1} \exp\left(-\frac{1}{4} e^{2k-1}\right) 
     \right] 
     \right\rvert \\ 
     & \gg \left\lvert \int_{k=\frac{\log\log x}{4}}^{\frac{\log\log x}{2}} 
     \sqrt{t} \cdot \exp\left(-\frac{1}{4} e^{2t}\right) 
     \right\rvert \\ 
     & \gg \left\lvert \sqrt{\log\log x} \cdot \operatorname{Ei}\left(-\frac{\log x}{4}\right) 
     \right\rvert \\ 
     & \gg (\log x) \cdot \sqrt{\log\log x}. 
     \qedhere 
\end{align*} 
\end{proof} 

\subsection{Proof of the unboundedness of the scaled Mertens function}
\label{subSection_TheCoreResultProof} 

\begin{prop} 
\label{prop_Mx_SBP_IntegralFormula} 
For all sufficiently large $x$, we have that 
\begin{align} 
\label{eqn_pf_tag_v2-restated_v2} 
M(x) & = G^{-1}(x) + G^{-1}\left(\Floor{x}{2}\right) + 
     \sum_{k=1}^{\frac{x}{2}-1} G^{-1}(k) \left[ 
     \pi\left(\Floor{x}{k}\right) - \pi\left(\Floor{x}{k+1}\right) 
     \right]. 
\end{align} 
\end{prop} 
\begin{proof} 
We know by applying Corollary \ref{cor_Mx_gInvnPixk_formula} that 
\begin{align} 
\notag
M(x) & = \sum_{k=1}^{x} g^{-1}(k) \left[\pi\left(\Floor{x}{k}\right)+1\right] \\ 
\notag 
     & = G^{-1}(x) + \sum_{k=1}^{x/2} g^{-1}(k) \pi\left(\Floor{x}{k}\right) \\ 
\notag 
     & = G^{-1}(x) + G^{-1}\left(\Floor{x}{2}\right) + 
     \sum_{k=1}^{x/2-1} G^{-1}(k) \left[ 
     \pi\left(\Floor{x}{k}\right) - \pi\left(\Floor{x}{k+1}\right) 
     \right], 
\end{align} 
where the upper bound on the sum is truncated in the second equation 
by the fact that $\pi(1) = 0$. 
\end{proof} 

\begin{lemma}
\label{lemma_PrimePix_ErrorBoundDiffs_SimplifyingConditions_v1} 
For sufficiently large $x$, $k \in \left[\sqrt{x}, \frac{x}{2}\right]$ and 
integers $m \geq 0$, we have that 
\begin{equation} 
\tag{A} 
\frac{x}{k \cdot \log^m\left(\frac{x}{k}\right)} - 
     \frac{x}{(k+1) \cdot \log^m\left(\frac{x}{k+1}\right)}
     \gg \frac{x}{(\log x)^m \cdot k(k+1)}, 
\end{equation} 
and 
\begin{equation} 
\tag{B} 
\sum_{k=\sqrt{x}}^{\frac{x}{2}} \frac{x}{k(k+1)} = 
     \sum_{k=\sqrt{x}}^{\frac{x}{2}} \frac{x}{k^2} + O(1). 
\end{equation} 
\end{lemma} 
\begin{proof} 
The proof of (A) is obvious since for $k_0 \in \left[\sqrt{x}, \frac{x}{2}\right]$ we have that 
\[
\log(2) (1 + o(1)) \leq \log\left(\frac{x}{k_0}\right) \leq \log(x). 
\]
To prove (B), notice that 
\[
\frac{x}{k(k+1)} - \frac{x}{k^2} = -\frac{x}{k^2(k+1)}. 
\]
Then we see that 
\[
\left\lvert \int_{\sqrt{x}}^{\frac{x}{2}} \frac{x}{t^2(t+1)} dt \right\rvert \leq 
     \left\lvert \int_{\sqrt{x}}^{\frac{x}{2}} \frac{x}{t^3} dt \right\rvert \asymp 1. 
     \qedhere 
\]
\end{proof} 

We finally address the main conclusion of our arguments given so far with the 
following proof: 

\begin{proof}[Proof of Theorem \ref{cor_ThePipeDreamResult_v1}] 
\label{proofOf_cor_ThePipeDreamResult_v1} 
Define the infinite increasing subsequence, 
$\{x_{0,y}\}_{y \geq Y_0}$, by $x_{0,y} := e^{2e^{2y+1}}$ for the sequence indices $y$ 
starting at some sufficiently 
large finite integer $Y_0$. 
We can verify that for sufficiently large $y \rightarrow \infty$, this infinitely 
tending subsequence is well defined as $x_{0,y+1} > x_{0,y}$, and also importantly 
$\log\log(x_{0,y+1}) > \log\log(x_{0,y})$ whenever $y \geq Y_0$ 
(see concluding argument below). 
Given a fixed large infinitely tending $y$, we have some (at least one) point 
$\widehat{x}_0(y) \in\mathbb{X}_y$ defined such that 
$|G^{-1}(t)|$ is minimal and non-vanishing on the interval 
$\mathbb{X}_y := \left[\sqrt{x_{0,y+1}}, \frac{x_{0,y+1}}{2}\right)$ 
in the form of 
\[
\left\lvert G^{-1}(\widehat{x}_0(y)) \right\rvert := 
     \min_{\substack{\sqrt{x_{0,y+1}} \leq t < \frac{x_{0,y+1}}{2} \\ G^{-1}(t) \neq 0}} 
     |G^{-1}(t)|. 
\] 
In the last step, we observe that $G^{-1}(x) \neq 0$ for $x$ on a set of 
asymptotic density \emph{at least} bounded below by $\frac{1}{2}$, so that our 
claim is accurate as the summand's lower bound on this interval 
does not trivially vanish at large $y$. This happens since the sequence 
$g^{-1}(n)$ is non-zero for all $n \geq 1$, so that if we do encounter a zero of the 
summatory function at $x$, we find a non-zero summatory function value at $x+1$. 
Let the shorthand notation $|G_{\min}^{-1}(x_y)| := |G^{-1}(\hat{x}_0(y))|$. 

We need to bound the prime counting function differences in the formula given by 
Proposition \ref{prop_Mx_SBP_IntegralFormula}. 
We will require the following known bounds on the prime counting 
function due to Rosser and Schoenfeld for large $x \gg 59$ 
\cite[Thm.\ 1]{ROSSER-SCHOENFELD-1962}: 
\begin{equation} 
\label{eqn_RosserSchoenfeld_PrimePixBounds_v2} 
\frac{x}{\log x}\left(1 + \frac{1}{2\log x}\right) \leq \pi(x) \leq 
     \frac{x}{\log x}\left(1 + \frac{3}{2 \log x}\right). 
\end{equation} 
Let the component function $U_M(y)$ be defined for all large $y$ as 
\[
U_M(y) := \sum_{k=1}^{\sqrt{\hat{x}_{0,y+1}}} |G^{-1}(k)| \left[ 
     \pi\left(\frac{\hat{x}_{0,y+1}}{k}\right) - 
     \pi\left(\frac{\hat{x}_{0,y+1}}{k+1}\right)
     \right]. 
\]
Combined with Lemma \ref{lemma_PrimePix_ErrorBoundDiffs_SimplifyingConditions_v1}, 
these estimates on $\pi(x)$ lead to the following approximations that hold on the 
increasing sequences taken within the subintervals defined by $\widehat{x}_0(y)$: 
\begin{align*} 
|U_M(y)| & \gg \sum_{k=1}^{\frac{\hat{x}_{0,y+1}}{2}-1} |G^{-1}(k)| \Biggl[ 
     \frac{\hat{x}_{0,y+1}}{k \cdot \log\left(\frac{\hat{x}_{0,y+1}}{k}\right)} \left(1 + 
     \frac{1}{2 \cdot \log\left(\frac{\hat{x}_{0,y+1}}{k}\right)}\right) \\ 
     & \phantom{\gg \sum_{k=1}^{\frac{\hat{x}_{0,y+1}}{2}-1} |G^{-1}(k)| \Biggl[\ } - 
     \frac{\hat{x}_{0,y+1}}{(k+1) \cdot \log\left(\frac{\hat{x}_{0,y+1}}{k+1}\right)} \left(1 + 
     \frac{3}{2 \cdot \log\left(\frac{\hat{x}_{0,y+1}}{k+1}\right)}\right)
     \Biggr] \\ 
     & \gg 
     \sum_{k=\sqrt{\hat{x}_{0,y+1}}}^{\frac{\hat{x}_{0,y+1}}{2}-1} \frac{\hat{x}_{0,y+1} \cdot 
     |G_{\min}^{-1}(x_y)|}{k^2} \left[ 
     \frac{1}{\log(\hat{x}_{0,y+1})} + \frac{1}{2 \log^2(\hat{x}_{0,y+1})}\right] \\ 
     & \gg \hat{x}_{0,y+1} \times |G_{\min}^{-1}(x_y)| \left(\frac{1}{\log(\hat{x}_{0,y+1})} + 
     \frac{1}{2 \log^2(\hat{x}_{0,y+1})}\right) \times 
     \left\lvert \int_{\sqrt{\hat{x}_{0,y+1}}}^{\frac{\hat{x}_{0,y+1}}{2}} 
     \frac{dt}{t^2} \right\rvert \\ 
     & \gg \sqrt{\hat{x}_{0,y+1}} \times 
     \frac{|G_{\min}^{-1}(x_y)|}{\log(\hat{x}_{0,y+1})} 
     + o(1), \mathrm{\ as\ } y \rightarrow \infty. 
\end{align*} 
We clearly see from Theorem \ref{theorem_GInvxLowerBoundByGEInvx_v1} and 
Proposition \ref{prop_Mx_SBP_IntegralFormula} that 
\begin{align} 
\notag 
\frac{|M(\hat{x}_{0,y+1})|}{\sqrt{\hat{x}_{0,y+1}}} & \gg \frac{1}{\sqrt{\hat{x}_{0,y+1}}} \times 
     \left\lvert \left\lvert 
     G^{-1}(\hat{x}_{0,y+1}) + G^{-1}\left(\frac{\hat{x}_{0,y+1}}{2}\right) \right\rvert + 
     \left\lvert U_M(y) \right\rvert \right\rvert \\ 
\notag 
     & \gg \frac{1}{\sqrt{\hat{x}_{0,y+1}}} \times \left\lvert U_M(y) \right\rvert \\ 
\label{eqn_MxGInvxLowerBound_stmt_v3} 
     & \gg \log\log\left(\sqrt{\hat{x}_{0,y+1}}\right)^{\frac{1}{2}}. 
\end{align} 
There is a small, but nonetheless insightful point in question 
to explain about a 
technicality in stating \eqref{eqn_MxGInvxLowerBound_stmt_v3}. 
Namely, we are not asserting that 
$|M(x)| / \sqrt{x}$ grows unbounded along the precise subsequence of 
$x \mapsto \hat{x}_{0,y+1}$ itself as $y \rightarrow \infty$. 
Rather, we are asserting that the unboundedness of this function 
can be witnessed along some subsequence whose points are taken within a 
large interval window of 
$x \in \mathbb{X}_y$ as 
$y \rightarrow \infty$. 
We choose to state the lower bound given on the right-hand-side of 
\eqref{eqn_MxGInvxLowerBound_stmt_v3} using the 
lower bound on $|G^{-1}(x)|$ we proved in 
Theorem \ref{theorem_GInvxLowerBoundByGEInvx_v1} 
minimally with $\hat{x}_0(y) \geq \sqrt{\hat{x}_{0,y+1}}$ on the interval 
for all $y \geq Y_0$. 
It is also necessary that $\log\log(x_{0,y+1}) > \log\log(x_{0,y})$ 
for all sufficiently large $y$ so that we indeed to obtain an increasing infinite 
subsequence along which to show the unboundedness of 
\eqref{eqn_MxGInvxLowerBound_stmt_v3}. 
\end{proof} 

\newpage 
\renewcommand{\refname}{References} 
\bibliography{glossaries-bibtex/thesis-references}{}
\bibliographystyle{plain}

\newpage
\setcounter{section}{0} 
\renewcommand{\thesection}{T.\arabic{section}} 

\section{Table: The Dirichlet inverse function $g^{-1}(n)$ and the 
         distribution of its summatory function} 
\label{table_conjecture_Mertens_ginvSeq_approx_values}

\begin{table}[ht!]

\centering

\tiny
\begin{equation*}
\boxed{
\begin{array}{cc|cc|ccc|cc|ccc}
 n & \mathbf{Primes} & \mathbf{Sqfree} & \mathbf{PPower} & g^{-1}(n) & 
 \lambda(n) g^{-1}(n) - \widehat{f}_1(n) & 
 \frac{\sum_{d|n} C_{\Omega(d)}(d)}{|g^{-1}(n)|} & 
 \mathcal{L}_{+}(n) & \mathcal{L}_{-}(n) & 
 G^{-1}(n) & G^{-1}_{+}(n) & G^{-1}_{-}(n) \\ \hline 
1 & 1^1 & \text{Y} & \text{N} & 1 & 0 & 1.0000000 & 1.000000 & 0.000000 & 1 & 1 & 0 \\
 2 & 2^1 & \text{Y} & \text{Y} & -2 & 0 & 1.0000000 & 0.500000 & 0.500000 & -1 & 1 & -2 \\
 3 & 3^1 & \text{Y} & \text{Y} & -2 & 0 & 1.0000000 & 0.333333 & 0.666667 & -3 & 1 & -4 \\
 4 & 2^2 & \text{N} & \text{Y} & 2 & 0 & 1.5000000 & 0.500000 & 0.500000 & -1 & 3 & -4 \\
 5 & 5^1 & \text{Y} & \text{Y} & -2 & 0 & 1.0000000 & 0.400000 & 0.600000 & -3 & 3 & -6 \\
 6 & 2^1 3^1 & \text{Y} & \text{N} & 5 & 0 & 1.0000000 & 0.500000 & 0.500000 & 2 & 8 & -6 \\
 7 & 7^1 & \text{Y} & \text{Y} & -2 & 0 & 1.0000000 & 0.428571 & 0.571429 & 0 & 8 & -8 \\
 8 & 2^3 & \text{N} & \text{Y} & -2 & 0 & 2.0000000 & 0.375000 & 0.625000 & -2 & 8 & -10 \\
 9 & 3^2 & \text{N} & \text{Y} & 2 & 0 & 1.5000000 & 0.444444 & 0.555556 & 0 & 10 & -10 \\
 10 & 2^1 5^1 & \text{Y} & \text{N} & 5 & 0 & 1.0000000 & 0.500000 & 0.500000 & 5 & 15 & -10 \\
 11 & 11^1 & \text{Y} & \text{Y} & -2 & 0 & 1.0000000 & 0.454545 & 0.545455 & 3 & 15 & -12 \\
 12 & 2^2 3^1 & \text{N} & \text{N} & -7 & 2 & 1.2857143 & 0.416667 & 0.583333 & -4 & 15 & -19 \\
 13 & 13^1 & \text{Y} & \text{Y} & -2 & 0 & 1.0000000 & 0.384615 & 0.615385 & -6 & 15 & -21 \\
 14 & 2^1 7^1 & \text{Y} & \text{N} & 5 & 0 & 1.0000000 & 0.428571 & 0.571429 & -1 & 20 & -21 \\
 15 & 3^1 5^1 & \text{Y} & \text{N} & 5 & 0 & 1.0000000 & 0.466667 & 0.533333 & 4 & 25 & -21 \\
 16 & 2^4 & \text{N} & \text{Y} & 2 & 0 & 2.5000000 & 0.500000 & 0.500000 & 6 & 27 & -21 \\
 17 & 17^1 & \text{Y} & \text{Y} & -2 & 0 & 1.0000000 & 0.470588 & 0.529412 & 4 & 27 & -23 \\
 18 & 2^1 3^2 & \text{N} & \text{N} & -7 & 2 & 1.2857143 & 0.444444 & 0.555556 & -3 & 27 & -30 \\
 19 & 19^1 & \text{Y} & \text{Y} & -2 & 0 & 1.0000000 & 0.421053 & 0.578947 & -5 & 27 & -32 \\
 20 & 2^2 5^1 & \text{N} & \text{N} & -7 & 2 & 1.2857143 & 0.400000 & 0.600000 & -12 & 27 & -39 \\
 21 & 3^1 7^1 & \text{Y} & \text{N} & 5 & 0 & 1.0000000 & 0.428571 & 0.571429 & -7 & 32 & -39 \\
 22 & 2^1 11^1 & \text{Y} & \text{N} & 5 & 0 & 1.0000000 & 0.454545 & 0.545455 & -2 & 37 & -39 \\
 23 & 23^1 & \text{Y} & \text{Y} & -2 & 0 & 1.0000000 & 0.434783 & 0.565217 & -4 & 37 & -41 \\
 24 & 2^3 3^1 & \text{N} & \text{N} & 9 & 4 & 1.5555556 & 0.458333 & 0.541667 & 5 & 46 & -41 \\
 25 & 5^2 & \text{N} & \text{Y} & 2 & 0 & 1.5000000 & 0.480000 & 0.520000 & 7 & 48 & -41 \\
 26 & 2^1 13^1 & \text{Y} & \text{N} & 5 & 0 & 1.0000000 & 0.500000 & 0.500000 & 12 & 53 & -41 \\
 27 & 3^3 & \text{N} & \text{Y} & -2 & 0 & 2.0000000 & 0.481481 & 0.518519 & 10 & 53 & -43 \\
 28 & 2^2 7^1 & \text{N} & \text{N} & -7 & 2 & 1.2857143 & 0.464286 & 0.535714 & 3 & 53 & -50 \\
 29 & 29^1 & \text{Y} & \text{Y} & -2 & 0 & 1.0000000 & 0.448276 & 0.551724 & 1 & 53 & -52 \\
 30 & 2^1 3^1 5^1 & \text{Y} & \text{N} & -16 & 0 & 1.0000000 & 0.433333 & 0.566667 & -15 & 53 & -68 \\
 31 & 31^1 & \text{Y} & \text{Y} & -2 & 0 & 1.0000000 & 0.419355 & 0.580645 & -17 & 53 & -70 \\
 32 & 2^5 & \text{N} & \text{Y} & -2 & 0 & 3.0000000 & 0.406250 & 0.593750 & -19 & 53 & -72 \\
 33 & 3^1 11^1 & \text{Y} & \text{N} & 5 & 0 & 1.0000000 & 0.424242 & 0.575758 & -14 & 58 & -72 \\
 34 & 2^1 17^1 & \text{Y} & \text{N} & 5 & 0 & 1.0000000 & 0.441176 & 0.558824 & -9 & 63 & -72 \\
 35 & 5^1 7^1 & \text{Y} & \text{N} & 5 & 0 & 1.0000000 & 0.457143 & 0.542857 & -4 & 68 & -72 \\
 36 & 2^2 3^2 & \text{N} & \text{N} & 14 & 9 & 1.3571429 & 0.472222 & 0.527778 & 10 & 82 & -72 \\
 37 & 37^1 & \text{Y} & \text{Y} & -2 & 0 & 1.0000000 & 0.459459 & 0.540541 & 8 & 82 & -74 \\
 38 & 2^1 19^1 & \text{Y} & \text{N} & 5 & 0 & 1.0000000 & 0.473684 & 0.526316 & 13 & 87 & -74 \\
 39 & 3^1 13^1 & \text{Y} & \text{N} & 5 & 0 & 1.0000000 & 0.487179 & 0.512821 & 18 & 92 & -74 \\
 40 & 2^3 5^1 & \text{N} & \text{N} & 9 & 4 & 1.5555556 & 0.500000 & 0.500000 & 27 & 101 & -74 \\
 41 & 41^1 & \text{Y} & \text{Y} & -2 & 0 & 1.0000000 & 0.487805 & 0.512195 & 25 & 101 & -76 \\
 42 & 2^1 3^1 7^1 & \text{Y} & \text{N} & -16 & 0 & 1.0000000 & 0.476190 & 0.523810 & 9 & 101 & -92 \\
 43 & 43^1 & \text{Y} & \text{Y} & -2 & 0 & 1.0000000 & 0.465116 & 0.534884 & 7 & 101 & -94 \\
 44 & 2^2 11^1 & \text{N} & \text{N} & -7 & 2 & 1.2857143 & 0.454545 & 0.545455 & 0 & 101 & -101 \\
 45 & 3^2 5^1 & \text{N} & \text{N} & -7 & 2 & 1.2857143 & 0.444444 & 0.555556 & -7 & 101 & -108 \\
 46 & 2^1 23^1 & \text{Y} & \text{N} & 5 & 0 & 1.0000000 & 0.456522 & 0.543478 & -2 & 106 & -108 \\
 47 & 47^1 & \text{Y} & \text{Y} & -2 & 0 & 1.0000000 & 0.446809 & 0.553191 & -4 & 106 & -110 \\
 48 & 2^4 3^1 & \text{N} & \text{N} & -11 & 6 & 1.8181818 & 0.437500 & 0.562500 & -15 & 106 & -121 \\ 
\end{array}
}
\end{equation*}

\bigskip\hrule\smallskip 

\captionsetup{singlelinecheck=off} 
\caption*{\textbf{\rm \bf Table \thesection:} 
          \textbf{Computations with $\mathbf{g^{-1}(n) \equiv (\omega+1)^{-1}(n)}$ 
          for $\mathbf{1 \leq n \leq 500}$.} 
          \begin{itemize}[noitemsep,topsep=0pt,leftmargin=0.23in] 
          \item[$\blacktriangleright$] 
          The column labeled \texttt{Primes} provides the prime factorization of each $n$ so that the values of 
          $\omega(n)$ and $\Omega(n)$ are easily extracted. 
          The columns labeled \texttt{Sqfree} and \texttt{PPower}, respectively, 
          list inclusion of $n$ in the sets of squarefree integers and the prime powers. 
          \item[$\blacktriangleright$] 
          The next three columns provide the 
          explicit values of the inverse function $g^{-1}(n)$ and compare its explicit value with other estimates. 
          We define the function $\widehat{f}_1(n) := \sum_{k=0}^{\omega(n)} \binom{\omega(n)}{k} \cdot k!$. 
          \item[$\blacktriangleright$] 
          The last several columns indicate properties of the summatory function of $g^{-1}(n)$. 
          The notation for the densities of the sign weight of $g^{-1}(n)$ is defined as 
          $\mathcal{L}_{\pm}(x) := \frac{1}{n} \cdot \#\left\{n \leq x: \lambda(n) = \pm 1\right\}$. 
          The last three 
          columns then show the explicit components to the signed summatory function, 
          $G^{-1}(x) := \sum_{n \leq x} g^{-1}(n)$, decomposed into its 
          respective positive and negative magnitude sum contributions: $G^{-1}(x) = G^{-1}_{+}(x) + G^{-1}_{-}(x)$ where 
          $G^{-1}_{+}(x) > 0$ and $G^{-1}_{-}(x) < 0$ for all $x \geq 1$. 
          \end{itemize} 
          } 
\clearpage 

\end{table}

\newpage
\begin{table}[ht]

\centering

\tiny
\begin{equation*}
\boxed{
\begin{array}{cc|cc|ccc|cc|ccc}
 n & \mathbf{Primes} & \mathbf{Sqfree} & \mathbf{PPower} & g^{-1}(n) & 
 \lambda(n) g^{-1}(n) - \widehat{f}_1(n) & 
 \frac{\sum_{d|n} C_{\Omega(d)}(d)}{|g^{-1}(n)|} & 
 \mathcal{L}_{+}(n) & \mathcal{L}_{-}(n) & 
 G^{-1}(n) & G^{-1}_{+}(n) & G^{-1}_{-}(n) \\ \hline 
 49 & 7^2 & \text{N} & \text{Y} & 2 & 0 & 1.5000000 & 0.448980 & 0.551020 & -13 & 108 & -121 \\
 50 & 2^1 5^2 & \text{N} & \text{N} & -7 & 2 & 1.2857143 & 0.440000 & 0.560000 & -20 & 108 & -128 \\
 51 & 3^1 17^1 & \text{Y} & \text{N} & 5 & 0 & 1.0000000 & 0.450980 & 0.549020 & -15 & 113 & -128 \\
 52 & 2^2 13^1 & \text{N} & \text{N} & -7 & 2 & 1.2857143 & 0.442308 & 0.557692 & -22 & 113 & -135 \\
 53 & 53^1 & \text{Y} & \text{Y} & -2 & 0 & 1.0000000 & 0.433962 & 0.566038 & -24 & 113 & -137 \\
 54 & 2^1 3^3 & \text{N} & \text{N} & 9 & 4 & 1.5555556 & 0.444444 & 0.555556 & -15 & 122 & -137 \\
 55 & 5^1 11^1 & \text{Y} & \text{N} & 5 & 0 & 1.0000000 & 0.454545 & 0.545455 & -10 & 127 & -137 \\
 56 & 2^3 7^1 & \text{N} & \text{N} & 9 & 4 & 1.5555556 & 0.464286 & 0.535714 & -1 & 136 & -137 \\
 57 & 3^1 19^1 & \text{Y} & \text{N} & 5 & 0 & 1.0000000 & 0.473684 & 0.526316 & 4 & 141 & -137 \\
 58 & 2^1 29^1 & \text{Y} & \text{N} & 5 & 0 & 1.0000000 & 0.482759 & 0.517241 & 9 & 146 & -137 \\
 59 & 59^1 & \text{Y} & \text{Y} & -2 & 0 & 1.0000000 & 0.474576 & 0.525424 & 7 & 146 & -139 \\
 60 & 2^2 3^1 5^1 & \text{N} & \text{N} & 30 & 14 & 1.1666667 & 0.483333 & 0.516667 & 37 & 176 & -139 \\
 61 & 61^1 & \text{Y} & \text{Y} & -2 & 0 & 1.0000000 & 0.475410 & 0.524590 & 35 & 176 & -141 \\
 62 & 2^1 31^1 & \text{Y} & \text{N} & 5 & 0 & 1.0000000 & 0.483871 & 0.516129 & 40 & 181 & -141 \\
 63 & 3^2 7^1 & \text{N} & \text{N} & -7 & 2 & 1.2857143 & 0.476190 & 0.523810 & 33 & 181 & -148 \\
 64 & 2^6 & \text{N} & \text{Y} & 2 & 0 & 3.5000000 & 0.484375 & 0.515625 & 35 & 183 & -148 \\
 65 & 5^1 13^1 & \text{Y} & \text{N} & 5 & 0 & 1.0000000 & 0.492308 & 0.507692 & 40 & 188 & -148 \\
 66 & 2^1 3^1 11^1 & \text{Y} & \text{N} & -16 & 0 & 1.0000000 & 0.484848 & 0.515152 & 24 & 188 & -164 \\
 67 & 67^1 & \text{Y} & \text{Y} & -2 & 0 & 1.0000000 & 0.477612 & 0.522388 & 22 & 188 & -166 \\
 68 & 2^2 17^1 & \text{N} & \text{N} & -7 & 2 & 1.2857143 & 0.470588 & 0.529412 & 15 & 188 & -173 \\
 69 & 3^1 23^1 & \text{Y} & \text{N} & 5 & 0 & 1.0000000 & 0.478261 & 0.521739 & 20 & 193 & -173 \\
 70 & 2^1 5^1 7^1 & \text{Y} & \text{N} & -16 & 0 & 1.0000000 & 0.471429 & 0.528571 & 4 & 193 & -189 \\
 71 & 71^1 & \text{Y} & \text{Y} & -2 & 0 & 1.0000000 & 0.464789 & 0.535211 & 2 & 193 & -191 \\
 72 & 2^3 3^2 & \text{N} & \text{N} & -23 & 18 & 1.4782609 & 0.458333 & 0.541667 & -21 & 193 & -214 \\
 73 & 73^1 & \text{Y} & \text{Y} & -2 & 0 & 1.0000000 & 0.452055 & 0.547945 & -23 & 193 & -216 \\
 74 & 2^1 37^1 & \text{Y} & \text{N} & 5 & 0 & 1.0000000 & 0.459459 & 0.540541 & -18 & 198 & -216 \\
 75 & 3^1 5^2 & \text{N} & \text{N} & -7 & 2 & 1.2857143 & 0.453333 & 0.546667 & -25 & 198 & -223 \\
 76 & 2^2 19^1 & \text{N} & \text{N} & -7 & 2 & 1.2857143 & 0.447368 & 0.552632 & -32 & 198 & -230 \\
 77 & 7^1 11^1 & \text{Y} & \text{N} & 5 & 0 & 1.0000000 & 0.454545 & 0.545455 & -27 & 203 & -230 \\
 78 & 2^1 3^1 13^1 & \text{Y} & \text{N} & -16 & 0 & 1.0000000 & 0.448718 & 0.551282 & -43 & 203 & -246 \\
 79 & 79^1 & \text{Y} & \text{Y} & -2 & 0 & 1.0000000 & 0.443038 & 0.556962 & -45 & 203 & -248 \\
 80 & 2^4 5^1 & \text{N} & \text{N} & -11 & 6 & 1.8181818 & 0.437500 & 0.562500 & -56 & 203 & -259 \\
 81 & 3^4 & \text{N} & \text{Y} & 2 & 0 & 2.5000000 & 0.444444 & 0.555556 & -54 & 205 & -259 \\
 82 & 2^1 41^1 & \text{Y} & \text{N} & 5 & 0 & 1.0000000 & 0.451220 & 0.548780 & -49 & 210 & -259 \\
 83 & 83^1 & \text{Y} & \text{Y} & -2 & 0 & 1.0000000 & 0.445783 & 0.554217 & -51 & 210 & -261 \\
 84 & 2^2 3^1 7^1 & \text{N} & \text{N} & 30 & 14 & 1.1666667 & 0.452381 & 0.547619 & -21 & 240 & -261 \\
 85 & 5^1 17^1 & \text{Y} & \text{N} & 5 & 0 & 1.0000000 & 0.458824 & 0.541176 & -16 & 245 & -261 \\
 86 & 2^1 43^1 & \text{Y} & \text{N} & 5 & 0 & 1.0000000 & 0.465116 & 0.534884 & -11 & 250 & -261 \\
 87 & 3^1 29^1 & \text{Y} & \text{N} & 5 & 0 & 1.0000000 & 0.471264 & 0.528736 & -6 & 255 & -261 \\
 88 & 2^3 11^1 & \text{N} & \text{N} & 9 & 4 & 1.5555556 & 0.477273 & 0.522727 & 3 & 264 & -261 \\
 89 & 89^1 & \text{Y} & \text{Y} & -2 & 0 & 1.0000000 & 0.471910 & 0.528090 & 1 & 264 & -263 \\
 90 & 2^1 3^2 5^1 & \text{N} & \text{N} & 30 & 14 & 1.1666667 & 0.477778 & 0.522222 & 31 & 294 & -263 \\
 91 & 7^1 13^1 & \text{Y} & \text{N} & 5 & 0 & 1.0000000 & 0.483516 & 0.516484 & 36 & 299 & -263 \\
 92 & 2^2 23^1 & \text{N} & \text{N} & -7 & 2 & 1.2857143 & 0.478261 & 0.521739 & 29 & 299 & -270 \\
 93 & 3^1 31^1 & \text{Y} & \text{N} & 5 & 0 & 1.0000000 & 0.483871 & 0.516129 & 34 & 304 & -270 \\
 94 & 2^1 47^1 & \text{Y} & \text{N} & 5 & 0 & 1.0000000 & 0.489362 & 0.510638 & 39 & 309 & -270 \\
 95 & 5^1 19^1 & \text{Y} & \text{N} & 5 & 0 & 1.0000000 & 0.494737 & 0.505263 & 44 & 314 & -270 \\
 96 & 2^5 3^1 & \text{N} & \text{N} & 13 & 8 & 2.0769231 & 0.500000 & 0.500000 & 57 & 327 & -270 \\
 97 & 97^1 & \text{Y} & \text{Y} & -2 & 0 & 1.0000000 & 0.494845 & 0.505155 & 55 & 327 & -272 \\
 98 & 2^1 7^2 & \text{N} & \text{N} & -7 & 2 & 1.2857143 & 0.489796 & 0.510204 & 48 & 327 & -279 \\
 99 & 3^2 11^1 & \text{N} & \text{N} & -7 & 2 & 1.2857143 & 0.484848 & 0.515152 & 41 & 327 & -286 \\
 100 & 2^2 5^2 & \text{N} & \text{N} & 14 & 9 & 1.3571429 & 0.490000 & 0.510000 & 55 & 341 & -286 \\
 101 & 101^1 & \text{Y} & \text{Y} & -2 & 0 & 1.0000000 & 0.485149 & 0.514851 & 53 & 341 & -288 \\
 102 & 2^1 3^1 17^1 & \text{Y} & \text{N} & -16 & 0 & 1.0000000 & 0.480392 & 0.519608 & 37 & 341 & -304 \\
 103 & 103^1 & \text{Y} & \text{Y} & -2 & 0 & 1.0000000 & 0.475728 & 0.524272 & 35 & 341 & -306 \\
 104 & 2^3 13^1 & \text{N} & \text{N} & 9 & 4 & 1.5555556 & 0.480769 & 0.519231 & 44 & 350 & -306 \\
 105 & 3^1 5^1 7^1 & \text{Y} & \text{N} & -16 & 0 & 1.0000000 & 0.476190 & 0.523810 & 28 & 350 & -322 \\
 106 & 2^1 53^1 & \text{Y} & \text{N} & 5 & 0 & 1.0000000 & 0.481132 & 0.518868 & 33 & 355 & -322 \\
 107 & 107^1 & \text{Y} & \text{Y} & -2 & 0 & 1.0000000 & 0.476636 & 0.523364 & 31 & 355 & -324 \\
 108 & 2^2 3^3 & \text{N} & \text{N} & -23 & 18 & 1.4782609 & 0.472222 & 0.527778 & 8 & 355 & -347 \\
 109 & 109^1 & \text{Y} & \text{Y} & -2 & 0 & 1.0000000 & 0.467890 & 0.532110 & 6 & 355 & -349 \\
 110 & 2^1 5^1 11^1 & \text{Y} & \text{N} & -16 & 0 & 1.0000000 & 0.463636 & 0.536364 & -10 & 355 & -365 \\
 111 & 3^1 37^1 & \text{Y} & \text{N} & 5 & 0 & 1.0000000 & 0.468468 & 0.531532 & -5 & 360 & -365 \\
 112 & 2^4 7^1 & \text{N} & \text{N} & -11 & 6 & 1.8181818 & 0.464286 & 0.535714 & -16 & 360 & -376 \\
 113 & 113^1 & \text{Y} & \text{Y} & -2 & 0 & 1.0000000 & 0.460177 & 0.539823 & -18 & 360 & -378 \\
 114 & 2^1 3^1 19^1 & \text{Y} & \text{N} & -16 & 0 & 1.0000000 & 0.456140 & 0.543860 & -34 & 360 & -394 \\
 115 & 5^1 23^1 & \text{Y} & \text{N} & 5 & 0 & 1.0000000 & 0.460870 & 0.539130 & -29 & 365 & -394 \\
 116 & 2^2 29^1 & \text{N} & \text{N} & -7 & 2 & 1.2857143 & 0.456897 & 0.543103 & -36 & 365 & -401 \\
 117 & 3^2 13^1 & \text{N} & \text{N} & -7 & 2 & 1.2857143 & 0.452991 & 0.547009 & -43 & 365 & -408 \\
 118 & 2^1 59^1 & \text{Y} & \text{N} & 5 & 0 & 1.0000000 & 0.457627 & 0.542373 & -38 & 370 & -408 \\
 119 & 7^1 17^1 & \text{Y} & \text{N} & 5 & 0 & 1.0000000 & 0.462185 & 0.537815 & -33 & 375 & -408 \\
 120 & 2^3 3^1 5^1 & \text{N} & \text{N} & -48 & 32 & 1.3333333 & 0.458333 & 0.541667 & -81 & 375 & -456 \\
 121 & 11^2 & \text{N} & \text{Y} & 2 & 0 & 1.5000000 & 0.462810 & 0.537190 & -79 & 377 & -456 \\
 122 & 2^1 61^1 & \text{Y} & \text{N} & 5 & 0 & 1.0000000 & 0.467213 & 0.532787 & -74 & 382 & -456 \\
 123 & 3^1 41^1 & \text{Y} & \text{N} & 5 & 0 & 1.0000000 & 0.471545 & 0.528455 & -69 & 387 & -456 \\
 124 & 2^2 31^1 & \text{N} & \text{N} & -7 & 2 & 1.2857143 & 0.467742 & 0.532258 & -76 & 387 & -463 \\ 
\end{array}
}
\end{equation*}
\clearpage 

\end{table} 


\newpage
\begin{table}[ht]

\centering

\tiny
\begin{equation*}
\boxed{
\begin{array}{cc|cc|ccc|cc|ccc}
 n & \mathbf{Primes} & \mathbf{Sqfree} & \mathbf{PPower} & g^{-1}(n) & 
 \lambda(n) g^{-1}(n) - \widehat{f}_1(n) & 
 \frac{\sum_{d|n} C_{\Omega(d)}(d)}{|g^{-1}(n)|} & 
 \mathcal{L}_{+}(n) & \mathcal{L}_{-}(n) & 
 G^{-1}(n) & G^{-1}_{+}(n) & G^{-1}_{-}(n) \\ \hline 
 125 & 5^3 & \text{N} & \text{Y} & -2 & 0 & 2.0000000 & 0.464000 & 0.536000 & -78 & 387 & -465 \\
 126 & 2^1 3^2 7^1 & \text{N} & \text{N} & 30 & 14 & 1.1666667 & 0.468254 & 0.531746 & -48 & 417 & -465 \\
 127 & 127^1 & \text{Y} & \text{Y} & -2 & 0 & 1.0000000 & 0.464567 & 0.535433 & -50 & 417 & -467 \\
 128 & 2^7 & \text{N} & \text{Y} & -2 & 0 & 4.0000000 & 0.460938 & 0.539062 & -52 & 417 & -469 \\
 129 & 3^1 43^1 & \text{Y} & \text{N} & 5 & 0 & 1.0000000 & 0.465116 & 0.534884 & -47 & 422 & -469 \\
 130 & 2^1 5^1 13^1 & \text{Y} & \text{N} & -16 & 0 & 1.0000000 & 0.461538 & 0.538462 & -63 & 422 & -485 \\
 131 & 131^1 & \text{Y} & \text{Y} & -2 & 0 & 1.0000000 & 0.458015 & 0.541985 & -65 & 422 & -487 \\
 132 & 2^2 3^1 11^1 & \text{N} & \text{N} & 30 & 14 & 1.1666667 & 0.462121 & 0.537879 & -35 & 452 & -487 \\
 133 & 7^1 19^1 & \text{Y} & \text{N} & 5 & 0 & 1.0000000 & 0.466165 & 0.533835 & -30 & 457 & -487 \\
 134 & 2^1 67^1 & \text{Y} & \text{N} & 5 & 0 & 1.0000000 & 0.470149 & 0.529851 & -25 & 462 & -487 \\
 135 & 3^3 5^1 & \text{N} & \text{N} & 9 & 4 & 1.5555556 & 0.474074 & 0.525926 & -16 & 471 & -487 \\
 136 & 2^3 17^1 & \text{N} & \text{N} & 9 & 4 & 1.5555556 & 0.477941 & 0.522059 & -7 & 480 & -487 \\
 137 & 137^1 & \text{Y} & \text{Y} & -2 & 0 & 1.0000000 & 0.474453 & 0.525547 & -9 & 480 & -489 \\
 138 & 2^1 3^1 23^1 & \text{Y} & \text{N} & -16 & 0 & 1.0000000 & 0.471014 & 0.528986 & -25 & 480 & -505 \\
 139 & 139^1 & \text{Y} & \text{Y} & -2 & 0 & 1.0000000 & 0.467626 & 0.532374 & -27 & 480 & -507 \\
 140 & 2^2 5^1 7^1 & \text{N} & \text{N} & 30 & 14 & 1.1666667 & 0.471429 & 0.528571 & 3 & 510 & -507 \\
 141 & 3^1 47^1 & \text{Y} & \text{N} & 5 & 0 & 1.0000000 & 0.475177 & 0.524823 & 8 & 515 & -507 \\
 142 & 2^1 71^1 & \text{Y} & \text{N} & 5 & 0 & 1.0000000 & 0.478873 & 0.521127 & 13 & 520 & -507 \\
 143 & 11^1 13^1 & \text{Y} & \text{N} & 5 & 0 & 1.0000000 & 0.482517 & 0.517483 & 18 & 525 & -507 \\
 144 & 2^4 3^2 & \text{N} & \text{N} & 34 & 29 & 1.6176471 & 0.486111 & 0.513889 & 52 & 559 & -507 \\
 145 & 5^1 29^1 & \text{Y} & \text{N} & 5 & 0 & 1.0000000 & 0.489655 & 0.510345 & 57 & 564 & -507 \\
 146 & 2^1 73^1 & \text{Y} & \text{N} & 5 & 0 & 1.0000000 & 0.493151 & 0.506849 & 62 & 569 & -507 \\
 147 & 3^1 7^2 & \text{N} & \text{N} & -7 & 2 & 1.2857143 & 0.489796 & 0.510204 & 55 & 569 & -514 \\
 148 & 2^2 37^1 & \text{N} & \text{N} & -7 & 2 & 1.2857143 & 0.486486 & 0.513514 & 48 & 569 & -521 \\
 149 & 149^1 & \text{Y} & \text{Y} & -2 & 0 & 1.0000000 & 0.483221 & 0.516779 & 46 & 569 & -523 \\
 150 & 2^1 3^1 5^2 & \text{N} & \text{N} & 30 & 14 & 1.1666667 & 0.486667 & 0.513333 & 76 & 599 & -523 \\
 151 & 151^1 & \text{Y} & \text{Y} & -2 & 0 & 1.0000000 & 0.483444 & 0.516556 & 74 & 599 & -525 \\
 152 & 2^3 19^1 & \text{N} & \text{N} & 9 & 4 & 1.5555556 & 0.486842 & 0.513158 & 83 & 608 & -525 \\
 153 & 3^2 17^1 & \text{N} & \text{N} & -7 & 2 & 1.2857143 & 0.483660 & 0.516340 & 76 & 608 & -532 \\
 154 & 2^1 7^1 11^1 & \text{Y} & \text{N} & -16 & 0 & 1.0000000 & 0.480519 & 0.519481 & 60 & 608 & -548 \\
 155 & 5^1 31^1 & \text{Y} & \text{N} & 5 & 0 & 1.0000000 & 0.483871 & 0.516129 & 65 & 613 & -548 \\
 156 & 2^2 3^1 13^1 & \text{N} & \text{N} & 30 & 14 & 1.1666667 & 0.487179 & 0.512821 & 95 & 643 & -548 \\
 157 & 157^1 & \text{Y} & \text{Y} & -2 & 0 & 1.0000000 & 0.484076 & 0.515924 & 93 & 643 & -550 \\
 158 & 2^1 79^1 & \text{Y} & \text{N} & 5 & 0 & 1.0000000 & 0.487342 & 0.512658 & 98 & 648 & -550 \\
 159 & 3^1 53^1 & \text{Y} & \text{N} & 5 & 0 & 1.0000000 & 0.490566 & 0.509434 & 103 & 653 & -550 \\
 160 & 2^5 5^1 & \text{N} & \text{N} & 13 & 8 & 2.0769231 & 0.493750 & 0.506250 & 116 & 666 & -550 \\
 161 & 7^1 23^1 & \text{Y} & \text{N} & 5 & 0 & 1.0000000 & 0.496894 & 0.503106 & 121 & 671 & -550 \\
 162 & 2^1 3^4 & \text{N} & \text{N} & -11 & 6 & 1.8181818 & 0.493827 & 0.506173 & 110 & 671 & -561 \\
 163 & 163^1 & \text{Y} & \text{Y} & -2 & 0 & 1.0000000 & 0.490798 & 0.509202 & 108 & 671 & -563 \\
 164 & 2^2 41^1 & \text{N} & \text{N} & -7 & 2 & 1.2857143 & 0.487805 & 0.512195 & 101 & 671 & -570 \\
 165 & 3^1 5^1 11^1 & \text{Y} & \text{N} & -16 & 0 & 1.0000000 & 0.484848 & 0.515152 & 85 & 671 & -586 \\
 166 & 2^1 83^1 & \text{Y} & \text{N} & 5 & 0 & 1.0000000 & 0.487952 & 0.512048 & 90 & 676 & -586 \\
 167 & 167^1 & \text{Y} & \text{Y} & -2 & 0 & 1.0000000 & 0.485030 & 0.514970 & 88 & 676 & -588 \\
 168 & 2^3 3^1 7^1 & \text{N} & \text{N} & -48 & 32 & 1.3333333 & 0.482143 & 0.517857 & 40 & 676 & -636 \\
 169 & 13^2 & \text{N} & \text{Y} & 2 & 0 & 1.5000000 & 0.485207 & 0.514793 & 42 & 678 & -636 \\
 170 & 2^1 5^1 17^1 & \text{Y} & \text{N} & -16 & 0 & 1.0000000 & 0.482353 & 0.517647 & 26 & 678 & -652 \\
 171 & 3^2 19^1 & \text{N} & \text{N} & -7 & 2 & 1.2857143 & 0.479532 & 0.520468 & 19 & 678 & -659 \\
 172 & 2^2 43^1 & \text{N} & \text{N} & -7 & 2 & 1.2857143 & 0.476744 & 0.523256 & 12 & 678 & -666 \\
 173 & 173^1 & \text{Y} & \text{Y} & -2 & 0 & 1.0000000 & 0.473988 & 0.526012 & 10 & 678 & -668 \\
 174 & 2^1 3^1 29^1 & \text{Y} & \text{N} & -16 & 0 & 1.0000000 & 0.471264 & 0.528736 & -6 & 678 & -684 \\
 175 & 5^2 7^1 & \text{N} & \text{N} & -7 & 2 & 1.2857143 & 0.468571 & 0.531429 & -13 & 678 & -691 \\
 176 & 2^4 11^1 & \text{N} & \text{N} & -11 & 6 & 1.8181818 & 0.465909 & 0.534091 & -24 & 678 & -702 \\
 177 & 3^1 59^1 & \text{Y} & \text{N} & 5 & 0 & 1.0000000 & 0.468927 & 0.531073 & -19 & 683 & -702 \\
 178 & 2^1 89^1 & \text{Y} & \text{N} & 5 & 0 & 1.0000000 & 0.471910 & 0.528090 & -14 & 688 & -702 \\
 179 & 179^1 & \text{Y} & \text{Y} & -2 & 0 & 1.0000000 & 0.469274 & 0.530726 & -16 & 688 & -704 \\
 180 & 2^2 3^2 5^1 & \text{N} & \text{N} & -74 & 58 & 1.2162162 & 0.466667 & 0.533333 & -90 & 688 & -778 \\
 181 & 181^1 & \text{Y} & \text{Y} & -2 & 0 & 1.0000000 & 0.464088 & 0.535912 & -92 & 688 & -780 \\
 182 & 2^1 7^1 13^1 & \text{Y} & \text{N} & -16 & 0 & 1.0000000 & 0.461538 & 0.538462 & -108 & 688 & -796 \\
 183 & 3^1 61^1 & \text{Y} & \text{N} & 5 & 0 & 1.0000000 & 0.464481 & 0.535519 & -103 & 693 & -796 \\
 184 & 2^3 23^1 & \text{N} & \text{N} & 9 & 4 & 1.5555556 & 0.467391 & 0.532609 & -94 & 702 & -796 \\
 185 & 5^1 37^1 & \text{Y} & \text{N} & 5 & 0 & 1.0000000 & 0.470270 & 0.529730 & -89 & 707 & -796 \\
 186 & 2^1 3^1 31^1 & \text{Y} & \text{N} & -16 & 0 & 1.0000000 & 0.467742 & 0.532258 & -105 & 707 & -812 \\
 187 & 11^1 17^1 & \text{Y} & \text{N} & 5 & 0 & 1.0000000 & 0.470588 & 0.529412 & -100 & 712 & -812 \\
 188 & 2^2 47^1 & \text{N} & \text{N} & -7 & 2 & 1.2857143 & 0.468085 & 0.531915 & -107 & 712 & -819 \\
 189 & 3^3 7^1 & \text{N} & \text{N} & 9 & 4 & 1.5555556 & 0.470899 & 0.529101 & -98 & 721 & -819 \\
 190 & 2^1 5^1 19^1 & \text{Y} & \text{N} & -16 & 0 & 1.0000000 & 0.468421 & 0.531579 & -114 & 721 & -835 \\
 191 & 191^1 & \text{Y} & \text{Y} & -2 & 0 & 1.0000000 & 0.465969 & 0.534031 & -116 & 721 & -837 \\
 192 & 2^6 3^1 & \text{N} & \text{N} & -15 & 10 & 2.3333333 & 0.463542 & 0.536458 & -131 & 721 & -852 \\
 193 & 193^1 & \text{Y} & \text{Y} & -2 & 0 & 1.0000000 & 0.461140 & 0.538860 & -133 & 721 & -854 \\
 194 & 2^1 97^1 & \text{Y} & \text{N} & 5 & 0 & 1.0000000 & 0.463918 & 0.536082 & -128 & 726 & -854 \\
 195 & 3^1 5^1 13^1 & \text{Y} & \text{N} & -16 & 0 & 1.0000000 & 0.461538 & 0.538462 & -144 & 726 & -870 \\
 196 & 2^2 7^2 & \text{N} & \text{N} & 14 & 9 & 1.3571429 & 0.464286 & 0.535714 & -130 & 740 & -870 \\
 197 & 197^1 & \text{Y} & \text{Y} & -2 & 0 & 1.0000000 & 0.461929 & 0.538071 & -132 & 740 & -872 \\
 198 & 2^1 3^2 11^1 & \text{N} & \text{N} & 30 & 14 & 1.1666667 & 0.464646 & 0.535354 & -102 & 770 & -872 \\
 199 & 199^1 & \text{Y} & \text{Y} & -2 & 0 & 1.0000000 & 0.462312 & 0.537688 & -104 & 770 & -874 \\
 200 & 2^3 5^2 & \text{N} & \text{N} & -23 & 18 & 1.4782609 & 0.460000 & 0.540000 & -127 & 770 & -897 \\ 
\end{array}
}
\end{equation*}
\clearpage 

\end{table} 

\newpage
\begin{table}[ht]

\centering

\tiny
\begin{equation*}
\boxed{
\begin{array}{cc|cc|ccc|cc|ccc}
 n & \mathbf{Primes} & \mathbf{Sqfree} & \mathbf{PPower} & g^{-1}(n) & 
 \lambda(n) g^{-1}(n) - \widehat{f}_1(n) & 
 \frac{\sum_{d|n} C_{\Omega(d)}(d)}{|g^{-1}(n)|} & 
 \mathcal{L}_{+}(n) & \mathcal{L}_{-}(n) & 
 G^{-1}(n) & G^{-1}_{+}(n) & G^{-1}_{-}(n) \\ \hline 
 201 & 3^1 67^1 & \text{Y} & \text{N} & 5 & 0 & 1.0000000 & 0.462687 & 0.537313 & -122 & 775 & -897 \\
 202 & 2^1 101^1 & \text{Y} & \text{N} & 5 & 0 & 1.0000000 & 0.465347 & 0.534653 & -117 & 780 & -897 \\
 203 & 7^1 29^1 & \text{Y} & \text{N} & 5 & 0 & 1.0000000 & 0.467980 & 0.532020 & -112 & 785 & -897 \\
 204 & 2^2 3^1 17^1 & \text{N} & \text{N} & 30 & 14 & 1.1666667 & 0.470588 & 0.529412 & -82 & 815 & -897 \\
 205 & 5^1 41^1 & \text{Y} & \text{N} & 5 & 0 & 1.0000000 & 0.473171 & 0.526829 & -77 & 820 & -897 \\
 206 & 2^1 103^1 & \text{Y} & \text{N} & 5 & 0 & 1.0000000 & 0.475728 & 0.524272 & -72 & 825 & -897 \\
 207 & 3^2 23^1 & \text{N} & \text{N} & -7 & 2 & 1.2857143 & 0.473430 & 0.526570 & -79 & 825 & -904 \\
 208 & 2^4 13^1 & \text{N} & \text{N} & -11 & 6 & 1.8181818 & 0.471154 & 0.528846 & -90 & 825 & -915 \\
 209 & 11^1 19^1 & \text{Y} & \text{N} & 5 & 0 & 1.0000000 & 0.473684 & 0.526316 & -85 & 830 & -915 \\
 210 & 2^1 3^1 5^1 7^1 & \text{Y} & \text{N} & 65 & 0 & 1.0000000 & 0.476190 & 0.523810 & -20 & 895 & -915 \\
 211 & 211^1 & \text{Y} & \text{Y} & -2 & 0 & 1.0000000 & 0.473934 & 0.526066 & -22 & 895 & -917 \\
 212 & 2^2 53^1 & \text{N} & \text{N} & -7 & 2 & 1.2857143 & 0.471698 & 0.528302 & -29 & 895 & -924 \\
 213 & 3^1 71^1 & \text{Y} & \text{N} & 5 & 0 & 1.0000000 & 0.474178 & 0.525822 & -24 & 900 & -924 \\
 214 & 2^1 107^1 & \text{Y} & \text{N} & 5 & 0 & 1.0000000 & 0.476636 & 0.523364 & -19 & 905 & -924 \\
 215 & 5^1 43^1 & \text{Y} & \text{N} & 5 & 0 & 1.0000000 & 0.479070 & 0.520930 & -14 & 910 & -924 \\
 216 & 2^3 3^3 & \text{N} & \text{N} & 46 & 41 & 1.5000000 & 0.481481 & 0.518519 & 32 & 956 & -924 \\
 217 & 7^1 31^1 & \text{Y} & \text{N} & 5 & 0 & 1.0000000 & 0.483871 & 0.516129 & 37 & 961 & -924 \\
 218 & 2^1 109^1 & \text{Y} & \text{N} & 5 & 0 & 1.0000000 & 0.486239 & 0.513761 & 42 & 966 & -924 \\
 219 & 3^1 73^1 & \text{Y} & \text{N} & 5 & 0 & 1.0000000 & 0.488584 & 0.511416 & 47 & 971 & -924 \\
 220 & 2^2 5^1 11^1 & \text{N} & \text{N} & 30 & 14 & 1.1666667 & 0.490909 & 0.509091 & 77 & 1001 & -924 \\
 221 & 13^1 17^1 & \text{Y} & \text{N} & 5 & 0 & 1.0000000 & 0.493213 & 0.506787 & 82 & 1006 & -924 \\
 222 & 2^1 3^1 37^1 & \text{Y} & \text{N} & -16 & 0 & 1.0000000 & 0.490991 & 0.509009 & 66 & 1006 & -940 \\
 223 & 223^1 & \text{Y} & \text{Y} & -2 & 0 & 1.0000000 & 0.488789 & 0.511211 & 64 & 1006 & -942 \\
 224 & 2^5 7^1 & \text{N} & \text{N} & 13 & 8 & 2.0769231 & 0.491071 & 0.508929 & 77 & 1019 & -942 \\
 225 & 3^2 5^2 & \text{N} & \text{N} & 14 & 9 & 1.3571429 & 0.493333 & 0.506667 & 91 & 1033 & -942 \\
 226 & 2^1 113^1 & \text{Y} & \text{N} & 5 & 0 & 1.0000000 & 0.495575 & 0.504425 & 96 & 1038 & -942 \\
 227 & 227^1 & \text{Y} & \text{Y} & -2 & 0 & 1.0000000 & 0.493392 & 0.506608 & 94 & 1038 & -944 \\
 228 & 2^2 3^1 19^1 & \text{N} & \text{N} & 30 & 14 & 1.1666667 & 0.495614 & 0.504386 & 124 & 1068 & -944 \\
 229 & 229^1 & \text{Y} & \text{Y} & -2 & 0 & 1.0000000 & 0.493450 & 0.506550 & 122 & 1068 & -946 \\
 230 & 2^1 5^1 23^1 & \text{Y} & \text{N} & -16 & 0 & 1.0000000 & 0.491304 & 0.508696 & 106 & 1068 & -962 \\
 231 & 3^1 7^1 11^1 & \text{Y} & \text{N} & -16 & 0 & 1.0000000 & 0.489177 & 0.510823 & 90 & 1068 & -978 \\
 232 & 2^3 29^1 & \text{N} & \text{N} & 9 & 4 & 1.5555556 & 0.491379 & 0.508621 & 99 & 1077 & -978 \\
 233 & 233^1 & \text{Y} & \text{Y} & -2 & 0 & 1.0000000 & 0.489270 & 0.510730 & 97 & 1077 & -980 \\
 234 & 2^1 3^2 13^1 & \text{N} & \text{N} & 30 & 14 & 1.1666667 & 0.491453 & 0.508547 & 127 & 1107 & -980 \\
 235 & 5^1 47^1 & \text{Y} & \text{N} & 5 & 0 & 1.0000000 & 0.493617 & 0.506383 & 132 & 1112 & -980 \\
 236 & 2^2 59^1 & \text{N} & \text{N} & -7 & 2 & 1.2857143 & 0.491525 & 0.508475 & 125 & 1112 & -987 \\
 237 & 3^1 79^1 & \text{Y} & \text{N} & 5 & 0 & 1.0000000 & 0.493671 & 0.506329 & 130 & 1117 & -987 \\
 238 & 2^1 7^1 17^1 & \text{Y} & \text{N} & -16 & 0 & 1.0000000 & 0.491597 & 0.508403 & 114 & 1117 & -1003 \\
 239 & 239^1 & \text{Y} & \text{Y} & -2 & 0 & 1.0000000 & 0.489540 & 0.510460 & 112 & 1117 & -1005 \\
 240 & 2^4 3^1 5^1 & \text{N} & \text{N} & 70 & 54 & 1.5000000 & 0.491667 & 0.508333 & 182 & 1187 & -1005 \\
 241 & 241^1 & \text{Y} & \text{Y} & -2 & 0 & 1.0000000 & 0.489627 & 0.510373 & 180 & 1187 & -1007 \\
 242 & 2^1 11^2 & \text{N} & \text{N} & -7 & 2 & 1.2857143 & 0.487603 & 0.512397 & 173 & 1187 & -1014 \\
 243 & 3^5 & \text{N} & \text{Y} & -2 & 0 & 3.0000000 & 0.485597 & 0.514403 & 171 & 1187 & -1016 \\
 244 & 2^2 61^1 & \text{N} & \text{N} & -7 & 2 & 1.2857143 & 0.483607 & 0.516393 & 164 & 1187 & -1023 \\
 245 & 5^1 7^2 & \text{N} & \text{N} & -7 & 2 & 1.2857143 & 0.481633 & 0.518367 & 157 & 1187 & -1030 \\
 246 & 2^1 3^1 41^1 & \text{Y} & \text{N} & -16 & 0 & 1.0000000 & 0.479675 & 0.520325 & 141 & 1187 & -1046 \\
 247 & 13^1 19^1 & \text{Y} & \text{N} & 5 & 0 & 1.0000000 & 0.481781 & 0.518219 & 146 & 1192 & -1046 \\
 248 & 2^3 31^1 & \text{N} & \text{N} & 9 & 4 & 1.5555556 & 0.483871 & 0.516129 & 155 & 1201 & -1046 \\
 249 & 3^1 83^1 & \text{Y} & \text{N} & 5 & 0 & 1.0000000 & 0.485944 & 0.514056 & 160 & 1206 & -1046 \\
 250 & 2^1 5^3 & \text{N} & \text{N} & 9 & 4 & 1.5555556 & 0.488000 & 0.512000 & 169 & 1215 & -1046 \\
 251 & 251^1 & \text{Y} & \text{Y} & -2 & 0 & 1.0000000 & 0.486056 & 0.513944 & 167 & 1215 & -1048 \\
 252 & 2^2 3^2 7^1 & \text{N} & \text{N} & -74 & 58 & 1.2162162 & 0.484127 & 0.515873 & 93 & 1215 & -1122 \\
 253 & 11^1 23^1 & \text{Y} & \text{N} & 5 & 0 & 1.0000000 & 0.486166 & 0.513834 & 98 & 1220 & -1122 \\
 254 & 2^1 127^1 & \text{Y} & \text{N} & 5 & 0 & 1.0000000 & 0.488189 & 0.511811 & 103 & 1225 & -1122 \\
 255 & 3^1 5^1 17^1 & \text{Y} & \text{N} & -16 & 0 & 1.0000000 & 0.486275 & 0.513725 & 87 & 1225 & -1138 \\
 256 & 2^8 & \text{N} & \text{Y} & 2 & 0 & 4.5000000 & 0.488281 & 0.511719 & 89 & 1227 & -1138 \\
 257 & 257^1 & \text{Y} & \text{Y} & -2 & 0 & 1.0000000 & 0.486381 & 0.513619 & 87 & 1227 & -1140 \\
 258 & 2^1 3^1 43^1 & \text{Y} & \text{N} & -16 & 0 & 1.0000000 & 0.484496 & 0.515504 & 71 & 1227 & -1156 \\
 259 & 7^1 37^1 & \text{Y} & \text{N} & 5 & 0 & 1.0000000 & 0.486486 & 0.513514 & 76 & 1232 & -1156 \\
 260 & 2^2 5^1 13^1 & \text{N} & \text{N} & 30 & 14 & 1.1666667 & 0.488462 & 0.511538 & 106 & 1262 & -1156 \\
 261 & 3^2 29^1 & \text{N} & \text{N} & -7 & 2 & 1.2857143 & 0.486590 & 0.513410 & 99 & 1262 & -1163 \\
 262 & 2^1 131^1 & \text{Y} & \text{N} & 5 & 0 & 1.0000000 & 0.488550 & 0.511450 & 104 & 1267 & -1163 \\
 263 & 263^1 & \text{Y} & \text{Y} & -2 & 0 & 1.0000000 & 0.486692 & 0.513308 & 102 & 1267 & -1165 \\
 264 & 2^3 3^1 11^1 & \text{N} & \text{N} & -48 & 32 & 1.3333333 & 0.484848 & 0.515152 & 54 & 1267 & -1213 \\
 265 & 5^1 53^1 & \text{Y} & \text{N} & 5 & 0 & 1.0000000 & 0.486792 & 0.513208 & 59 & 1272 & -1213 \\
 266 & 2^1 7^1 19^1 & \text{Y} & \text{N} & -16 & 0 & 1.0000000 & 0.484962 & 0.515038 & 43 & 1272 & -1229 \\
 267 & 3^1 89^1 & \text{Y} & \text{N} & 5 & 0 & 1.0000000 & 0.486891 & 0.513109 & 48 & 1277 & -1229 \\
 268 & 2^2 67^1 & \text{N} & \text{N} & -7 & 2 & 1.2857143 & 0.485075 & 0.514925 & 41 & 1277 & -1236 \\
 269 & 269^1 & \text{Y} & \text{Y} & -2 & 0 & 1.0000000 & 0.483271 & 0.516729 & 39 & 1277 & -1238 \\
 270 & 2^1 3^3 5^1 & \text{N} & \text{N} & -48 & 32 & 1.3333333 & 0.481481 & 0.518519 & -9 & 1277 & -1286 \\
 271 & 271^1 & \text{Y} & \text{Y} & -2 & 0 & 1.0000000 & 0.479705 & 0.520295 & -11 & 1277 & -1288 \\
 272 & 2^4 17^1 & \text{N} & \text{N} & -11 & 6 & 1.8181818 & 0.477941 & 0.522059 & -22 & 1277 & -1299 \\
 273 & 3^1 7^1 13^1 & \text{Y} & \text{N} & -16 & 0 & 1.0000000 & 0.476190 & 0.523810 & -38 & 1277 & -1315 \\
 274 & 2^1 137^1 & \text{Y} & \text{N} & 5 & 0 & 1.0000000 & 0.478102 & 0.521898 & -33 & 1282 & -1315 \\
 275 & 5^2 11^1 & \text{N} & \text{N} & -7 & 2 & 1.2857143 & 0.476364 & 0.523636 & -40 & 1282 & -1322 \\
 276 & 2^2 3^1 23^1 & \text{N} & \text{N} & 30 & 14 & 1.1666667 & 0.478261 & 0.521739 & -10 & 1312 & -1322 \\
 277 & 277^1 & \text{Y} & \text{Y} & -2 & 0 & 1.0000000 & 0.476534 & 0.523466 & -12 & 1312 & -1324 \\ 
\end{array}
}
\end{equation*}
\clearpage 

\end{table} 

\newpage
\begin{table}[ht]

\centering

\tiny
\begin{equation*}
\boxed{
\begin{array}{cc|cc|ccc|cc|ccc}
 n & \mathbf{Primes} & \mathbf{Sqfree} & \mathbf{PPower} & g^{-1}(n) & 
 \lambda(n) g^{-1}(n) - \widehat{f}_1(n) & 
 \frac{\sum_{d|n} C_{\Omega(d)}(d)}{|g^{-1}(n)|} & 
 \mathcal{L}_{+}(n) & \mathcal{L}_{-}(n) & 
 G^{-1}(n) & G^{-1}_{+}(n) & G^{-1}_{-}(n) \\ \hline 
 278 & 2^1 139^1 & \text{Y} & \text{N} & 5 & 0 & 1.0000000 & 0.478417 & 0.521583 & -7 & 1317 & -1324 \\
 279 & 3^2 31^1 & \text{N} & \text{N} & -7 & 2 & 1.2857143 & 0.476703 & 0.523297 & -14 & 1317 & -1331 \\
 280 & 2^3 5^1 7^1 & \text{N} & \text{N} & -48 & 32 & 1.3333333 & 0.475000 & 0.525000 & -62 & 1317 & -1379 \\
 281 & 281^1 & \text{Y} & \text{Y} & -2 & 0 & 1.0000000 & 0.473310 & 0.526690 & -64 & 1317 & -1381 \\
 282 & 2^1 3^1 47^1 & \text{Y} & \text{N} & -16 & 0 & 1.0000000 & 0.471631 & 0.528369 & -80 & 1317 & -1397 \\
 283 & 283^1 & \text{Y} & \text{Y} & -2 & 0 & 1.0000000 & 0.469965 & 0.530035 & -82 & 1317 & -1399 \\
 284 & 2^2 71^1 & \text{N} & \text{N} & -7 & 2 & 1.2857143 & 0.468310 & 0.531690 & -89 & 1317 & -1406 \\
 285 & 3^1 5^1 19^1 & \text{Y} & \text{N} & -16 & 0 & 1.0000000 & 0.466667 & 0.533333 & -105 & 1317 & -1422 \\
 286 & 2^1 11^1 13^1 & \text{Y} & \text{N} & -16 & 0 & 1.0000000 & 0.465035 & 0.534965 & -121 & 1317 & -1438 \\
 287 & 7^1 41^1 & \text{Y} & \text{N} & 5 & 0 & 1.0000000 & 0.466899 & 0.533101 & -116 & 1322 & -1438 \\
 288 & 2^5 3^2 & \text{N} & \text{N} & -47 & 42 & 1.7659574 & 0.465278 & 0.534722 & -163 & 1322 & -1485 \\
 289 & 17^2 & \text{N} & \text{Y} & 2 & 0 & 1.5000000 & 0.467128 & 0.532872 & -161 & 1324 & -1485 \\
 290 & 2^1 5^1 29^1 & \text{Y} & \text{N} & -16 & 0 & 1.0000000 & 0.465517 & 0.534483 & -177 & 1324 & -1501 \\
 291 & 3^1 97^1 & \text{Y} & \text{N} & 5 & 0 & 1.0000000 & 0.467354 & 0.532646 & -172 & 1329 & -1501 \\
 292 & 2^2 73^1 & \text{N} & \text{N} & -7 & 2 & 1.2857143 & 0.465753 & 0.534247 & -179 & 1329 & -1508 \\
 293 & 293^1 & \text{Y} & \text{Y} & -2 & 0 & 1.0000000 & 0.464164 & 0.535836 & -181 & 1329 & -1510 \\
 294 & 2^1 3^1 7^2 & \text{N} & \text{N} & 30 & 14 & 1.1666667 & 0.465986 & 0.534014 & -151 & 1359 & -1510 \\
 295 & 5^1 59^1 & \text{Y} & \text{N} & 5 & 0 & 1.0000000 & 0.467797 & 0.532203 & -146 & 1364 & -1510 \\
 296 & 2^3 37^1 & \text{N} & \text{N} & 9 & 4 & 1.5555556 & 0.469595 & 0.530405 & -137 & 1373 & -1510 \\
 297 & 3^3 11^1 & \text{N} & \text{N} & 9 & 4 & 1.5555556 & 0.471380 & 0.528620 & -128 & 1382 & -1510 \\
 298 & 2^1 149^1 & \text{Y} & \text{N} & 5 & 0 & 1.0000000 & 0.473154 & 0.526846 & -123 & 1387 & -1510 \\
 299 & 13^1 23^1 & \text{Y} & \text{N} & 5 & 0 & 1.0000000 & 0.474916 & 0.525084 & -118 & 1392 & -1510 \\
 300 & 2^2 3^1 5^2 & \text{N} & \text{N} & -74 & 58 & 1.2162162 & 0.473333 & 0.526667 & -192 & 1392 & -1584 \\
 301 & 7^1 43^1 & \text{Y} & \text{N} & 5 & 0 & 1.0000000 & 0.475083 & 0.524917 & -187 & 1397 & -1584 \\
 302 & 2^1 151^1 & \text{Y} & \text{N} & 5 & 0 & 1.0000000 & 0.476821 & 0.523179 & -182 & 1402 & -1584 \\
 303 & 3^1 101^1 & \text{Y} & \text{N} & 5 & 0 & 1.0000000 & 0.478548 & 0.521452 & -177 & 1407 & -1584 \\
 304 & 2^4 19^1 & \text{N} & \text{N} & -11 & 6 & 1.8181818 & 0.476974 & 0.523026 & -188 & 1407 & -1595 \\
 305 & 5^1 61^1 & \text{Y} & \text{N} & 5 & 0 & 1.0000000 & 0.478689 & 0.521311 & -183 & 1412 & -1595 \\
 306 & 2^1 3^2 17^1 & \text{N} & \text{N} & 30 & 14 & 1.1666667 & 0.480392 & 0.519608 & -153 & 1442 & -1595 \\
 307 & 307^1 & \text{Y} & \text{Y} & -2 & 0 & 1.0000000 & 0.478827 & 0.521173 & -155 & 1442 & -1597 \\
 308 & 2^2 7^1 11^1 & \text{N} & \text{N} & 30 & 14 & 1.1666667 & 0.480519 & 0.519481 & -125 & 1472 & -1597 \\
 309 & 3^1 103^1 & \text{Y} & \text{N} & 5 & 0 & 1.0000000 & 0.482201 & 0.517799 & -120 & 1477 & -1597 \\
 310 & 2^1 5^1 31^1 & \text{Y} & \text{N} & -16 & 0 & 1.0000000 & 0.480645 & 0.519355 & -136 & 1477 & -1613 \\
 311 & 311^1 & \text{Y} & \text{Y} & -2 & 0 & 1.0000000 & 0.479100 & 0.520900 & -138 & 1477 & -1615 \\
 312 & 2^3 3^1 13^1 & \text{N} & \text{N} & -48 & 32 & 1.3333333 & 0.477564 & 0.522436 & -186 & 1477 & -1663 \\
 313 & 313^1 & \text{Y} & \text{Y} & -2 & 0 & 1.0000000 & 0.476038 & 0.523962 & -188 & 1477 & -1665 \\
 314 & 2^1 157^1 & \text{Y} & \text{N} & 5 & 0 & 1.0000000 & 0.477707 & 0.522293 & -183 & 1482 & -1665 \\
 315 & 3^2 5^1 7^1 & \text{N} & \text{N} & 30 & 14 & 1.1666667 & 0.479365 & 0.520635 & -153 & 1512 & -1665 \\
 316 & 2^2 79^1 & \text{N} & \text{N} & -7 & 2 & 1.2857143 & 0.477848 & 0.522152 & -160 & 1512 & -1672 \\
 317 & 317^1 & \text{Y} & \text{Y} & -2 & 0 & 1.0000000 & 0.476341 & 0.523659 & -162 & 1512 & -1674 \\
 318 & 2^1 3^1 53^1 & \text{Y} & \text{N} & -16 & 0 & 1.0000000 & 0.474843 & 0.525157 & -178 & 1512 & -1690 \\
 319 & 11^1 29^1 & \text{Y} & \text{N} & 5 & 0 & 1.0000000 & 0.476489 & 0.523511 & -173 & 1517 & -1690 \\
 320 & 2^6 5^1 & \text{N} & \text{N} & -15 & 10 & 2.3333333 & 0.475000 & 0.525000 & -188 & 1517 & -1705 \\
 321 & 3^1 107^1 & \text{Y} & \text{N} & 5 & 0 & 1.0000000 & 0.476636 & 0.523364 & -183 & 1522 & -1705 \\
 322 & 2^1 7^1 23^1 & \text{Y} & \text{N} & -16 & 0 & 1.0000000 & 0.475155 & 0.524845 & -199 & 1522 & -1721 \\
 323 & 17^1 19^1 & \text{Y} & \text{N} & 5 & 0 & 1.0000000 & 0.476780 & 0.523220 & -194 & 1527 & -1721 \\
 324 & 2^2 3^4 & \text{N} & \text{N} & 34 & 29 & 1.6176471 & 0.478395 & 0.521605 & -160 & 1561 & -1721 \\
 325 & 5^2 13^1 & \text{N} & \text{N} & -7 & 2 & 1.2857143 & 0.476923 & 0.523077 & -167 & 1561 & -1728 \\
 326 & 2^1 163^1 & \text{Y} & \text{N} & 5 & 0 & 1.0000000 & 0.478528 & 0.521472 & -162 & 1566 & -1728 \\
 327 & 3^1 109^1 & \text{Y} & \text{N} & 5 & 0 & 1.0000000 & 0.480122 & 0.519878 & -157 & 1571 & -1728 \\
 328 & 2^3 41^1 & \text{N} & \text{N} & 9 & 4 & 1.5555556 & 0.481707 & 0.518293 & -148 & 1580 & -1728 \\
 329 & 7^1 47^1 & \text{Y} & \text{N} & 5 & 0 & 1.0000000 & 0.483283 & 0.516717 & -143 & 1585 & -1728 \\
 330 & 2^1 3^1 5^1 11^1 & \text{Y} & \text{N} & 65 & 0 & 1.0000000 & 0.484848 & 0.515152 & -78 & 1650 & -1728 \\
 331 & 331^1 & \text{Y} & \text{Y} & -2 & 0 & 1.0000000 & 0.483384 & 0.516616 & -80 & 1650 & -1730 \\
 332 & 2^2 83^1 & \text{N} & \text{N} & -7 & 2 & 1.2857143 & 0.481928 & 0.518072 & -87 & 1650 & -1737 \\
 333 & 3^2 37^1 & \text{N} & \text{N} & -7 & 2 & 1.2857143 & 0.480480 & 0.519520 & -94 & 1650 & -1744 \\
 334 & 2^1 167^1 & \text{Y} & \text{N} & 5 & 0 & 1.0000000 & 0.482036 & 0.517964 & -89 & 1655 & -1744 \\
 335 & 5^1 67^1 & \text{Y} & \text{N} & 5 & 0 & 1.0000000 & 0.483582 & 0.516418 & -84 & 1660 & -1744 \\
 336 & 2^4 3^1 7^1 & \text{N} & \text{N} & 70 & 54 & 1.5000000 & 0.485119 & 0.514881 & -14 & 1730 & -1744 \\
 337 & 337^1 & \text{Y} & \text{Y} & -2 & 0 & 1.0000000 & 0.483680 & 0.516320 & -16 & 1730 & -1746 \\
 338 & 2^1 13^2 & \text{N} & \text{N} & -7 & 2 & 1.2857143 & 0.482249 & 0.517751 & -23 & 1730 & -1753 \\
 339 & 3^1 113^1 & \text{Y} & \text{N} & 5 & 0 & 1.0000000 & 0.483776 & 0.516224 & -18 & 1735 & -1753 \\
 340 & 2^2 5^1 17^1 & \text{N} & \text{N} & 30 & 14 & 1.1666667 & 0.485294 & 0.514706 & 12 & 1765 & -1753 \\
 341 & 11^1 31^1 & \text{Y} & \text{N} & 5 & 0 & 1.0000000 & 0.486804 & 0.513196 & 17 & 1770 & -1753 \\
 342 & 2^1 3^2 19^1 & \text{N} & \text{N} & 30 & 14 & 1.1666667 & 0.488304 & 0.511696 & 47 & 1800 & -1753 \\
 343 & 7^3 & \text{N} & \text{Y} & -2 & 0 & 2.0000000 & 0.486880 & 0.513120 & 45 & 1800 & -1755 \\
 344 & 2^3 43^1 & \text{N} & \text{N} & 9 & 4 & 1.5555556 & 0.488372 & 0.511628 & 54 & 1809 & -1755 \\
 345 & 3^1 5^1 23^1 & \text{Y} & \text{N} & -16 & 0 & 1.0000000 & 0.486957 & 0.513043 & 38 & 1809 & -1771 \\
 346 & 2^1 173^1 & \text{Y} & \text{N} & 5 & 0 & 1.0000000 & 0.488439 & 0.511561 & 43 & 1814 & -1771 \\
 347 & 347^1 & \text{Y} & \text{Y} & -2 & 0 & 1.0000000 & 0.487032 & 0.512968 & 41 & 1814 & -1773 \\
 348 & 2^2 3^1 29^1 & \text{N} & \text{N} & 30 & 14 & 1.1666667 & 0.488506 & 0.511494 & 71 & 1844 & -1773 \\
 349 & 349^1 & \text{Y} & \text{Y} & -2 & 0 & 1.0000000 & 0.487106 & 0.512894 & 69 & 1844 & -1775 \\
 350 & 2^1 5^2 7^1 & \text{N} & \text{N} & 30 & 14 & 1.1666667 & 0.488571 & 0.511429 & 99 & 1874 & -1775 \\ 
\end{array}
}
\end{equation*}
\clearpage 

\end{table} 

\newpage
\begin{table}[ht]

\centering
\tiny
\begin{equation*}
\boxed{
\begin{array}{cc|cc|ccc|cc|ccc}
 n & \mathbf{Primes} & \mathbf{Sqfree} & \mathbf{PPower} & g^{-1}(n) & 
 \lambda(n) g^{-1}(n) - \widehat{f}_1(n) & 
 \frac{\sum_{d|n} C_{\Omega(d)}(d)}{|g^{-1}(n)|} & 
 \mathcal{L}_{+}(n) & \mathcal{L}_{-}(n) & 
 G^{-1}(n) & G^{-1}_{+}(n) & G^{-1}_{-}(n) \\ \hline 
 351 & 3^3 13^1 & \text{N} & \text{N} & 9 & 4 & 1.5555556 & 0.490028 & 0.509972 & 108 & 1883 & -1775 \\
 352 & 2^5 11^1 & \text{N} & \text{N} & 13 & 8 & 2.0769231 & 0.491477 & 0.508523 & 121 & 1896 & -1775 \\
 353 & 353^1 & \text{Y} & \text{Y} & -2 & 0 & 1.0000000 & 0.490085 & 0.509915 & 119 & 1896 & -1777 \\
 354 & 2^1 3^1 59^1 & \text{Y} & \text{N} & -16 & 0 & 1.0000000 & 0.488701 & 0.511299 & 103 & 1896 & -1793 \\
 355 & 5^1 71^1 & \text{Y} & \text{N} & 5 & 0 & 1.0000000 & 0.490141 & 0.509859 & 108 & 1901 & -1793 \\
 356 & 2^2 89^1 & \text{N} & \text{N} & -7 & 2 & 1.2857143 & 0.488764 & 0.511236 & 101 & 1901 & -1800 \\
 357 & 3^1 7^1 17^1 & \text{Y} & \text{N} & -16 & 0 & 1.0000000 & 0.487395 & 0.512605 & 85 & 1901 & -1816 \\
 358 & 2^1 179^1 & \text{Y} & \text{N} & 5 & 0 & 1.0000000 & 0.488827 & 0.511173 & 90 & 1906 & -1816 \\
 359 & 359^1 & \text{Y} & \text{Y} & -2 & 0 & 1.0000000 & 0.487465 & 0.512535 & 88 & 1906 & -1818 \\
 360 & 2^3 3^2 5^1 & \text{N} & \text{N} & 145 & 129 & 1.3034483 & 0.488889 & 0.511111 & 233 & 2051 & -1818 \\
 361 & 19^2 & \text{N} & \text{Y} & 2 & 0 & 1.5000000 & 0.490305 & 0.509695 & 235 & 2053 & -1818 \\
 362 & 2^1 181^1 & \text{Y} & \text{N} & 5 & 0 & 1.0000000 & 0.491713 & 0.508287 & 240 & 2058 & -1818 \\
 363 & 3^1 11^2 & \text{N} & \text{N} & -7 & 2 & 1.2857143 & 0.490358 & 0.509642 & 233 & 2058 & -1825 \\
 364 & 2^2 7^1 13^1 & \text{N} & \text{N} & 30 & 14 & 1.1666667 & 0.491758 & 0.508242 & 263 & 2088 & -1825 \\
 365 & 5^1 73^1 & \text{Y} & \text{N} & 5 & 0 & 1.0000000 & 0.493151 & 0.506849 & 268 & 2093 & -1825 \\
 366 & 2^1 3^1 61^1 & \text{Y} & \text{N} & -16 & 0 & 1.0000000 & 0.491803 & 0.508197 & 252 & 2093 & -1841 \\
 367 & 367^1 & \text{Y} & \text{Y} & -2 & 0 & 1.0000000 & 0.490463 & 0.509537 & 250 & 2093 & -1843 \\
 368 & 2^4 23^1 & \text{N} & \text{N} & -11 & 6 & 1.8181818 & 0.489130 & 0.510870 & 239 & 2093 & -1854 \\
 369 & 3^2 41^1 & \text{N} & \text{N} & -7 & 2 & 1.2857143 & 0.487805 & 0.512195 & 232 & 2093 & -1861 \\
 370 & 2^1 5^1 37^1 & \text{Y} & \text{N} & -16 & 0 & 1.0000000 & 0.486486 & 0.513514 & 216 & 2093 & -1877 \\
 371 & 7^1 53^1 & \text{Y} & \text{N} & 5 & 0 & 1.0000000 & 0.487871 & 0.512129 & 221 & 2098 & -1877 \\
 372 & 2^2 3^1 31^1 & \text{N} & \text{N} & 30 & 14 & 1.1666667 & 0.489247 & 0.510753 & 251 & 2128 & -1877 \\
 373 & 373^1 & \text{Y} & \text{Y} & -2 & 0 & 1.0000000 & 0.487936 & 0.512064 & 249 & 2128 & -1879 \\
 374 & 2^1 11^1 17^1 & \text{Y} & \text{N} & -16 & 0 & 1.0000000 & 0.486631 & 0.513369 & 233 & 2128 & -1895 \\
 375 & 3^1 5^3 & \text{N} & \text{N} & 9 & 4 & 1.5555556 & 0.488000 & 0.512000 & 242 & 2137 & -1895 \\
 376 & 2^3 47^1 & \text{N} & \text{N} & 9 & 4 & 1.5555556 & 0.489362 & 0.510638 & 251 & 2146 & -1895 \\
 377 & 13^1 29^1 & \text{Y} & \text{N} & 5 & 0 & 1.0000000 & 0.490716 & 0.509284 & 256 & 2151 & -1895 \\
 378 & 2^1 3^3 7^1 & \text{N} & \text{N} & -48 & 32 & 1.3333333 & 0.489418 & 0.510582 & 208 & 2151 & -1943 \\
 379 & 379^1 & \text{Y} & \text{Y} & -2 & 0 & 1.0000000 & 0.488127 & 0.511873 & 206 & 2151 & -1945 \\
 380 & 2^2 5^1 19^1 & \text{N} & \text{N} & 30 & 14 & 1.1666667 & 0.489474 & 0.510526 & 236 & 2181 & -1945 \\
 381 & 3^1 127^1 & \text{Y} & \text{N} & 5 & 0 & 1.0000000 & 0.490814 & 0.509186 & 241 & 2186 & -1945 \\
 382 & 2^1 191^1 & \text{Y} & \text{N} & 5 & 0 & 1.0000000 & 0.492147 & 0.507853 & 246 & 2191 & -1945 \\
 383 & 383^1 & \text{Y} & \text{Y} & -2 & 0 & 1.0000000 & 0.490862 & 0.509138 & 244 & 2191 & -1947 \\
 384 & 2^7 3^1 & \text{N} & \text{N} & 17 & 12 & 2.5882353 & 0.492188 & 0.507812 & 261 & 2208 & -1947 \\
 385 & 5^1 7^1 11^1 & \text{Y} & \text{N} & -16 & 0 & 1.0000000 & 0.490909 & 0.509091 & 245 & 2208 & -1963 \\
 386 & 2^1 193^1 & \text{Y} & \text{N} & 5 & 0 & 1.0000000 & 0.492228 & 0.507772 & 250 & 2213 & -1963 \\
 387 & 3^2 43^1 & \text{N} & \text{N} & -7 & 2 & 1.2857143 & 0.490956 & 0.509044 & 243 & 2213 & -1970 \\
 388 & 2^2 97^1 & \text{N} & \text{N} & -7 & 2 & 1.2857143 & 0.489691 & 0.510309 & 236 & 2213 & -1977 \\
 389 & 389^1 & \text{Y} & \text{Y} & -2 & 0 & 1.0000000 & 0.488432 & 0.511568 & 234 & 2213 & -1979 \\
 390 & 2^1 3^1 5^1 13^1 & \text{Y} & \text{N} & 65 & 0 & 1.0000000 & 0.489744 & 0.510256 & 299 & 2278 & -1979 \\
 391 & 17^1 23^1 & \text{Y} & \text{N} & 5 & 0 & 1.0000000 & 0.491049 & 0.508951 & 304 & 2283 & -1979 \\
 392 & 2^3 7^2 & \text{N} & \text{N} & -23 & 18 & 1.4782609 & 0.489796 & 0.510204 & 281 & 2283 & -2002 \\
 393 & 3^1 131^1 & \text{Y} & \text{N} & 5 & 0 & 1.0000000 & 0.491094 & 0.508906 & 286 & 2288 & -2002 \\
 394 & 2^1 197^1 & \text{Y} & \text{N} & 5 & 0 & 1.0000000 & 0.492386 & 0.507614 & 291 & 2293 & -2002 \\
 395 & 5^1 79^1 & \text{Y} & \text{N} & 5 & 0 & 1.0000000 & 0.493671 & 0.506329 & 296 & 2298 & -2002 \\
 396 & 2^2 3^2 11^1 & \text{N} & \text{N} & -74 & 58 & 1.2162162 & 0.492424 & 0.507576 & 222 & 2298 & -2076 \\
 397 & 397^1 & \text{Y} & \text{Y} & -2 & 0 & 1.0000000 & 0.491184 & 0.508816 & 220 & 2298 & -2078 \\
 398 & 2^1 199^1 & \text{Y} & \text{N} & 5 & 0 & 1.0000000 & 0.492462 & 0.507538 & 225 & 2303 & -2078 \\
 399 & 3^1 7^1 19^1 & \text{Y} & \text{N} & -16 & 0 & 1.0000000 & 0.491228 & 0.508772 & 209 & 2303 & -2094 \\
 400 & 2^4 5^2 & \text{N} & \text{N} & 34 & 29 & 1.6176471 & 0.492500 & 0.507500 & 243 & 2337 & -2094 \\
 401 & 401^1 & \text{Y} & \text{Y} & -2 & 0 & 1.0000000 & 0.491272 & 0.508728 & 241 & 2337 & -2096 \\
 402 & 2^1 3^1 67^1 & \text{Y} & \text{N} & -16 & 0 & 1.0000000 & 0.490050 & 0.509950 & 225 & 2337 & -2112 \\
 403 & 13^1 31^1 & \text{Y} & \text{N} & 5 & 0 & 1.0000000 & 0.491315 & 0.508685 & 230 & 2342 & -2112 \\
 404 & 2^2 101^1 & \text{N} & \text{N} & -7 & 2 & 1.2857143 & 0.490099 & 0.509901 & 223 & 2342 & -2119 \\
 405 & 3^4 5^1 & \text{N} & \text{N} & -11 & 6 & 1.8181818 & 0.488889 & 0.511111 & 212 & 2342 & -2130 \\
 406 & 2^1 7^1 29^1 & \text{Y} & \text{N} & -16 & 0 & 1.0000000 & 0.487685 & 0.512315 & 196 & 2342 & -2146 \\
 407 & 11^1 37^1 & \text{Y} & \text{N} & 5 & 0 & 1.0000000 & 0.488943 & 0.511057 & 201 & 2347 & -2146 \\
 408 & 2^3 3^1 17^1 & \text{N} & \text{N} & -48 & 32 & 1.3333333 & 0.487745 & 0.512255 & 153 & 2347 & -2194 \\
 409 & 409^1 & \text{Y} & \text{Y} & -2 & 0 & 1.0000000 & 0.486553 & 0.513447 & 151 & 2347 & -2196 \\
 410 & 2^1 5^1 41^1 & \text{Y} & \text{N} & -16 & 0 & 1.0000000 & 0.485366 & 0.514634 & 135 & 2347 & -2212 \\
 411 & 3^1 137^1 & \text{Y} & \text{N} & 5 & 0 & 1.0000000 & 0.486618 & 0.513382 & 140 & 2352 & -2212 \\
 412 & 2^2 103^1 & \text{N} & \text{N} & -7 & 2 & 1.2857143 & 0.485437 & 0.514563 & 133 & 2352 & -2219 \\
 413 & 7^1 59^1 & \text{Y} & \text{N} & 5 & 0 & 1.0000000 & 0.486683 & 0.513317 & 138 & 2357 & -2219 \\
 414 & 2^1 3^2 23^1 & \text{N} & \text{N} & 30 & 14 & 1.1666667 & 0.487923 & 0.512077 & 168 & 2387 & -2219 \\
 415 & 5^1 83^1 & \text{Y} & \text{N} & 5 & 0 & 1.0000000 & 0.489157 & 0.510843 & 173 & 2392 & -2219 \\
 416 & 2^5 13^1 & \text{N} & \text{N} & 13 & 8 & 2.0769231 & 0.490385 & 0.509615 & 186 & 2405 & -2219 \\
 417 & 3^1 139^1 & \text{Y} & \text{N} & 5 & 0 & 1.0000000 & 0.491607 & 0.508393 & 191 & 2410 & -2219 \\
 418 & 2^1 11^1 19^1 & \text{Y} & \text{N} & -16 & 0 & 1.0000000 & 0.490431 & 0.509569 & 175 & 2410 & -2235 \\
 419 & 419^1 & \text{Y} & \text{Y} & -2 & 0 & 1.0000000 & 0.489260 & 0.510740 & 173 & 2410 & -2237 \\
 420 & 2^2 3^1 5^1 7^1 & \text{N} & \text{N} & -155 & 90 & 1.1032258 & 0.488095 & 0.511905 & 18 & 2410 & -2392 \\
 421 & 421^1 & \text{Y} & \text{Y} & -2 & 0 & 1.0000000 & 0.486936 & 0.513064 & 16 & 2410 & -2394 \\
 422 & 2^1 211^1 & \text{Y} & \text{N} & 5 & 0 & 1.0000000 & 0.488152 & 0.511848 & 21 & 2415 & -2394 \\
 423 & 3^2 47^1 & \text{N} & \text{N} & -7 & 2 & 1.2857143 & 0.486998 & 0.513002 & 14 & 2415 & -2401 \\
 424 & 2^3 53^1 & \text{N} & \text{N} & 9 & 4 & 1.5555556 & 0.488208 & 0.511792 & 23 & 2424 & -2401 \\
 425 & 5^2 17^1 & \text{N} & \text{N} & -7 & 2 & 1.2857143 & 0.487059 & 0.512941 & 16 & 2424 & -2408 \\ 
\end{array}
}
\end{equation*}
\clearpage 

\end{table} 

\newpage

\begin{table}[ht]
\label{table_conjecture_Mertens_ginvSeq_approx_values_LastPage} 

\centering
\tiny
\begin{equation*}
\boxed{
\begin{array}{cc|cc|ccc|cc|ccc}
 n & \mathbf{Primes} & \mathbf{Sqfree} & \mathbf{PPower} & g^{-1}(n) & 
 \lambda(n) g^{-1}(n) - \widehat{f}_1(n) & 
 \frac{\sum_{d|n} C_{\Omega(d)}(d)}{|g^{-1}(n)|} & 
 \mathcal{L}_{+}(n) & \mathcal{L}_{-}(n) & 
 G^{-1}(n) & G^{-1}_{+}(n) & G^{-1}_{-}(n) \\ \hline 
 426 & 2^1 3^1 71^1 & \text{Y} & \text{N} & -16 & 0 & 1.0000000 & 0.485915 & 0.514085 & 0 & 2424 & -2424 \\
 427 & 7^1 61^1 & \text{Y} & \text{N} & 5 & 0 & 1.0000000 & 0.487119 & 0.512881 & 5 & 2429 & -2424 \\
 428 & 2^2 107^1 & \text{N} & \text{N} & -7 & 2 & 1.2857143 & 0.485981 & 0.514019 & -2 & 2429 & -2431 \\
 429 & 3^1 11^1 13^1 & \text{Y} & \text{N} & -16 & 0 & 1.0000000 & 0.484848 & 0.515152 & -18 & 2429 & -2447 \\
 430 & 2^1 5^1 43^1 & \text{Y} & \text{N} & -16 & 0 & 1.0000000 & 0.483721 & 0.516279 & -34 & 2429 & -2463 \\
 431 & 431^1 & \text{Y} & \text{Y} & -2 & 0 & 1.0000000 & 0.482599 & 0.517401 & -36 & 2429 & -2465 \\
 432 & 2^4 3^3 & \text{N} & \text{N} & -80 & 75 & 1.5625000 & 0.481481 & 0.518519 & -116 & 2429 & -2545 \\
 433 & 433^1 & \text{Y} & \text{Y} & -2 & 0 & 1.0000000 & 0.480370 & 0.519630 & -118 & 2429 & -2547 \\
 434 & 2^1 7^1 31^1 & \text{Y} & \text{N} & -16 & 0 & 1.0000000 & 0.479263 & 0.520737 & -134 & 2429 & -2563 \\
 435 & 3^1 5^1 29^1 & \text{Y} & \text{N} & -16 & 0 & 1.0000000 & 0.478161 & 0.521839 & -150 & 2429 & -2579 \\
 436 & 2^2 109^1 & \text{N} & \text{N} & -7 & 2 & 1.2857143 & 0.477064 & 0.522936 & -157 & 2429 & -2586 \\
 437 & 19^1 23^1 & \text{Y} & \text{N} & 5 & 0 & 1.0000000 & 0.478261 & 0.521739 & -152 & 2434 & -2586 \\
 438 & 2^1 3^1 73^1 & \text{Y} & \text{N} & -16 & 0 & 1.0000000 & 0.477169 & 0.522831 & -168 & 2434 & -2602 \\
 439 & 439^1 & \text{Y} & \text{Y} & -2 & 0 & 1.0000000 & 0.476082 & 0.523918 & -170 & 2434 & -2604 \\
 440 & 2^3 5^1 11^1 & \text{N} & \text{N} & -48 & 32 & 1.3333333 & 0.475000 & 0.525000 & -218 & 2434 & -2652 \\
 441 & 3^2 7^2 & \text{N} & \text{N} & 14 & 9 & 1.3571429 & 0.476190 & 0.523810 & -204 & 2448 & -2652 \\
 442 & 2^1 13^1 17^1 & \text{Y} & \text{N} & -16 & 0 & 1.0000000 & 0.475113 & 0.524887 & -220 & 2448 & -2668 \\
 443 & 443^1 & \text{Y} & \text{Y} & -2 & 0 & 1.0000000 & 0.474041 & 0.525959 & -222 & 2448 & -2670 \\
 444 & 2^2 3^1 37^1 & \text{N} & \text{N} & 30 & 14 & 1.1666667 & 0.475225 & 0.524775 & -192 & 2478 & -2670 \\
 445 & 5^1 89^1 & \text{Y} & \text{N} & 5 & 0 & 1.0000000 & 0.476404 & 0.523596 & -187 & 2483 & -2670 \\
 446 & 2^1 223^1 & \text{Y} & \text{N} & 5 & 0 & 1.0000000 & 0.477578 & 0.522422 & -182 & 2488 & -2670 \\
 447 & 3^1 149^1 & \text{Y} & \text{N} & 5 & 0 & 1.0000000 & 0.478747 & 0.521253 & -177 & 2493 & -2670 \\
 448 & 2^6 7^1 & \text{N} & \text{N} & -15 & 10 & 2.3333333 & 0.477679 & 0.522321 & -192 & 2493 & -2685 \\
 449 & 449^1 & \text{Y} & \text{Y} & -2 & 0 & 1.0000000 & 0.476615 & 0.523385 & -194 & 2493 & -2687 \\
 450 & 2^1 3^2 5^2 & \text{N} & \text{N} & -74 & 58 & 1.2162162 & 0.475556 & 0.524444 & -268 & 2493 & -2761 \\
 451 & 11^1 41^1 & \text{Y} & \text{N} & 5 & 0 & 1.0000000 & 0.476718 & 0.523282 & -263 & 2498 & -2761 \\
 452 & 2^2 113^1 & \text{N} & \text{N} & -7 & 2 & 1.2857143 & 0.475664 & 0.524336 & -270 & 2498 & -2768 \\
 453 & 3^1 151^1 & \text{Y} & \text{N} & 5 & 0 & 1.0000000 & 0.476821 & 0.523179 & -265 & 2503 & -2768 \\
 454 & 2^1 227^1 & \text{Y} & \text{N} & 5 & 0 & 1.0000000 & 0.477974 & 0.522026 & -260 & 2508 & -2768 \\
 455 & 5^1 7^1 13^1 & \text{Y} & \text{N} & -16 & 0 & 1.0000000 & 0.476923 & 0.523077 & -276 & 2508 & -2784 \\
 456 & 2^3 3^1 19^1 & \text{N} & \text{N} & -48 & 32 & 1.3333333 & 0.475877 & 0.524123 & -324 & 2508 & -2832 \\
 457 & 457^1 & \text{Y} & \text{Y} & -2 & 0 & 1.0000000 & 0.474836 & 0.525164 & -326 & 2508 & -2834 \\
 458 & 2^1 229^1 & \text{Y} & \text{N} & 5 & 0 & 1.0000000 & 0.475983 & 0.524017 & -321 & 2513 & -2834 \\
 459 & 3^3 17^1 & \text{N} & \text{N} & 9 & 4 & 1.5555556 & 0.477124 & 0.522876 & -312 & 2522 & -2834 \\
 460 & 2^2 5^1 23^1 & \text{N} & \text{N} & 30 & 14 & 1.1666667 & 0.478261 & 0.521739 & -282 & 2552 & -2834 \\
 461 & 461^1 & \text{Y} & \text{Y} & -2 & 0 & 1.0000000 & 0.477223 & 0.522777 & -284 & 2552 & -2836 \\
 462 & 2^1 3^1 7^1 11^1 & \text{Y} & \text{N} & 65 & 0 & 1.0000000 & 0.478355 & 0.521645 & -219 & 2617 & -2836 \\
 463 & 463^1 & \text{Y} & \text{Y} & -2 & 0 & 1.0000000 & 0.477322 & 0.522678 & -221 & 2617 & -2838 \\
 464 & 2^4 29^1 & \text{N} & \text{N} & -11 & 6 & 1.8181818 & 0.476293 & 0.523707 & -232 & 2617 & -2849 \\
 465 & 3^1 5^1 31^1 & \text{Y} & \text{N} & -16 & 0 & 1.0000000 & 0.475269 & 0.524731 & -248 & 2617 & -2865 \\
 466 & 2^1 233^1 & \text{Y} & \text{N} & 5 & 0 & 1.0000000 & 0.476395 & 0.523605 & -243 & 2622 & -2865 \\
 467 & 467^1 & \text{Y} & \text{Y} & -2 & 0 & 1.0000000 & 0.475375 & 0.524625 & -245 & 2622 & -2867 \\
 468 & 2^2 3^2 13^1 & \text{N} & \text{N} & -74 & 58 & 1.2162162 & 0.474359 & 0.525641 & -319 & 2622 & -2941 \\
 469 & 7^1 67^1 & \text{Y} & \text{N} & 5 & 0 & 1.0000000 & 0.475480 & 0.524520 & -314 & 2627 & -2941 \\
 470 & 2^1 5^1 47^1 & \text{Y} & \text{N} & -16 & 0 & 1.0000000 & 0.474468 & 0.525532 & -330 & 2627 & -2957 \\
 471 & 3^1 157^1 & \text{Y} & \text{N} & 5 & 0 & 1.0000000 & 0.475584 & 0.524416 & -325 & 2632 & -2957 \\
 472 & 2^3 59^1 & \text{N} & \text{N} & 9 & 4 & 1.5555556 & 0.476695 & 0.523305 & -316 & 2641 & -2957 \\
 473 & 11^1 43^1 & \text{Y} & \text{N} & 5 & 0 & 1.0000000 & 0.477801 & 0.522199 & -311 & 2646 & -2957 \\
 474 & 2^1 3^1 79^1 & \text{Y} & \text{N} & -16 & 0 & 1.0000000 & 0.476793 & 0.523207 & -327 & 2646 & -2973 \\
 475 & 5^2 19^1 & \text{N} & \text{N} & -7 & 2 & 1.2857143 & 0.475789 & 0.524211 & -334 & 2646 & -2980 \\
 476 & 2^2 7^1 17^1 & \text{N} & \text{N} & 30 & 14 & 1.1666667 & 0.476891 & 0.523109 & -304 & 2676 & -2980 \\
 477 & 3^2 53^1 & \text{N} & \text{N} & -7 & 2 & 1.2857143 & 0.475891 & 0.524109 & -311 & 2676 & -2987 \\
 478 & 2^1 239^1 & \text{Y} & \text{N} & 5 & 0 & 1.0000000 & 0.476987 & 0.523013 & -306 & 2681 & -2987 \\
 479 & 479^1 & \text{Y} & \text{Y} & -2 & 0 & 1.0000000 & 0.475992 & 0.524008 & -308 & 2681 & -2989 \\
 480 & 2^5 3^1 5^1 & \text{N} & \text{N} & -96 & 80 & 1.6666667 & 0.475000 & 0.525000 & -404 & 2681 & -3085 \\
 481 & 13^1 37^1 & \text{Y} & \text{N} & 5 & 0 & 1.0000000 & 0.476091 & 0.523909 & -399 & 2686 & -3085 \\
 482 & 2^1 241^1 & \text{Y} & \text{N} & 5 & 0 & 1.0000000 & 0.477178 & 0.522822 & -394 & 2691 & -3085 \\
 483 & 3^1 7^1 23^1 & \text{Y} & \text{N} & -16 & 0 & 1.0000000 & 0.476190 & 0.523810 & -410 & 2691 & -3101 \\
 484 & 2^2 11^2 & \text{N} & \text{N} & 14 & 9 & 1.3571429 & 0.477273 & 0.522727 & -396 & 2705 & -3101 \\
 485 & 5^1 97^1 & \text{Y} & \text{N} & 5 & 0 & 1.0000000 & 0.478351 & 0.521649 & -391 & 2710 & -3101 \\
 486 & 2^1 3^5 & \text{N} & \text{N} & 13 & 8 & 2.0769231 & 0.479424 & 0.520576 & -378 & 2723 & -3101 \\
 487 & 487^1 & \text{Y} & \text{Y} & -2 & 0 & 1.0000000 & 0.478439 & 0.521561 & -380 & 2723 & -3103 \\
 488 & 2^3 61^1 & \text{N} & \text{N} & 9 & 4 & 1.5555556 & 0.479508 & 0.520492 & -371 & 2732 & -3103 \\
 489 & 3^1 163^1 & \text{Y} & \text{N} & 5 & 0 & 1.0000000 & 0.480573 & 0.519427 & -366 & 2737 & -3103 \\
 490 & 2^1 5^1 7^2 & \text{N} & \text{N} & 30 & 14 & 1.1666667 & 0.481633 & 0.518367 & -336 & 2767 & -3103 \\
 491 & 491^1 & \text{Y} & \text{Y} & -2 & 0 & 1.0000000 & 0.480652 & 0.519348 & -338 & 2767 & -3105 \\
 492 & 2^2 3^1 41^1 & \text{N} & \text{N} & 30 & 14 & 1.1666667 & 0.481707 & 0.518293 & -308 & 2797 & -3105 \\
 493 & 17^1 29^1 & \text{Y} & \text{N} & 5 & 0 & 1.0000000 & 0.482759 & 0.517241 & -303 & 2802 & -3105 \\
 494 & 2^1 13^1 19^1 & \text{Y} & \text{N} & -16 & 0 & 1.0000000 & 0.481781 & 0.518219 & -319 & 2802 & -3121 \\
 495 & 3^2 5^1 11^1 & \text{N} & \text{N} & 30 & 14 & 1.1666667 & 0.482828 & 0.517172 & -289 & 2832 & -3121 \\
 496 & 2^4 31^1 & \text{N} & \text{N} & -11 & 6 & 1.8181818 & 0.481855 & 0.518145 & -300 & 2832 & -3132 \\
 497 & 7^1 71^1 & \text{Y} & \text{N} & 5 & 0 & 1.0000000 & 0.482897 & 0.517103 & -295 & 2837 & -3132 \\
 498 & 2^1 3^1 83^1 & \text{Y} & \text{N} & -16 & 0 & 1.0000000 & 0.481928 & 0.518072 & -311 & 2837 & -3148 \\
 499 & 499^1 & \text{Y} & \text{Y} & -2 & 0 & 1.0000000 & 0.480962 & 0.519038 & -313 & 2837 & -3150 \\
 500 & 2^2 5^3 & \text{N} & \text{N} & -23 & 18 & 1.4782609 & 0.480000 & 0.520000 & -336 & 2837 & -3173 \\  
\end{array}
}
\end{equation*}

\end{table} 

\clearpage 

\newpage
\section{Table: Approximations of the summatory functions of $\lambda(n)$ and $\lambda_{\ast}(n)$} 
\label{table_LAstxSummatoryFuncCompsWithExact_v2} 

\begin{table}[ht!] 

\centering
\tiny 
\begin{equation*} 
\boxed{
\begin{array}{ccccc|ccc||ccccc|ccc} 
x & L(x) & R_{\pm}(x) & 
    \frac{L(x)}{L_{\approx,1}(x)} & \frac{L(x)}{L_{\approx,2}(x)} & 
    x & L_{\ast}(x) & \frac{L_{\ast}(x)}{L_{\approx}^{\ast}(x)} & 
x & L(x) & R_{\pm}(x) & 
    \frac{L(x)}{L_{\approx,1}(x)} & \frac{L(x)}{L_{\approx,2}(x)} & 
    x & L_{\ast}(x) & \frac{L_{\ast}(x)}{L_{\approx}^{\ast}(x)} \\ \hline 
100000 & -401 & 1 & 0.0320 & -1.28 & 100000 & -720 & -0.0282 & 100045 & -389 & 1 & 0.0310 & -1.24 & 100045 & -711 & -0.0278  \\
100001 & -400 & 1 & 0.0319 & -1.28 & 100001 & -719 & -0.0282 & 100046 & -388 & 1 & 0.0309 & -1.24 & 100046 & -710 & -0.0278  \\
100002 & -398 & 1 & 0.0318 & -1.27 & 100002 & -718 & -0.0281 & 100047 & -387 & 1 & 0.0308 & -1.24 & 100047 & -709 & -0.0278  \\
100003 & -399 & 1 & 0.0318 & -1.28 & 100003 & -719 & -0.0282 & 100048 & -395 & 1 & 0.0315 & -1.26 & 100048 & -710 & -0.0278  \\
100004 & -398 & 1 & 0.0318 & -1.27 & 100004 & -720 & -0.0282 & 100049 & -396 & 1 & 0.0316 & -1.27 & 100049 & -711 & -0.0278  \\
100005 & -397 & 1 & 0.0317 & -1.27 & 100005 & -719 & -0.0282 & 100050 & -392 & 1 & 0.0312 & -1.25 & 100050 & -712 & -0.0279  \\
100006 & -398 & 1 & 0.0318 & -1.27 & 100006 & -720 & -0.0282 & 100051 & -391 & 1 & 0.0312 & -1.25 & 100051 & -711 & -0.0278  \\
100007 & -397 & 1 & 0.0317 & -1.27 & 100007 & -719 & -0.0282 & 100052 & -392 & 1 & 0.0312 & -1.25 & 100052 & -710 & -0.0278  \\
100008 & -403 & 1 & 0.0322 & -1.29 & 100008 & -720 & -0.0282 & 100053 & -394 & 1 & 0.0314 & -1.26 & 100053 & -709 & -0.0278  \\
100009 & -400 & 1 & 0.0319 & -1.28 & 100009 & -721 & -0.0283 & 100054 & -395 & 1 & 0.0315 & -1.26 & 100054 & -710 & -0.0278  \\
100010 & -399 & 1 & 0.0318 & -1.28 & 100010 & -720 & -0.0282 & 100055 & -394 & 1 & 0.0314 & -1.26 & 100055 & -709 & -0.0278  \\
100011 & -398 & 1 & 0.0317 & -1.27 & 100011 & -719 & -0.0282 & 100056 & -393 & 1 & 0.0313 & -1.26 & 100056 & -708 & -0.0277  \\
100012 & -397 & 1 & 0.0317 & -1.27 & 100012 & -720 & -0.0282 & 100057 & -394 & 1 & 0.0314 & -1.26 & 100057 & -709 & -0.0278  \\
100013 & -396 & 1 & 0.0316 & -1.27 & 100013 & -719 & -0.0282 & 100058 & -391 & 1 & 0.0312 & -1.25 & 100058 & -710 & -0.0278  \\
100014 & -395 & 1 & 0.0315 & -1.26 & 100014 & -718 & -0.0281 & 100059 & -390 & 1 & 0.0311 & -1.25 & 100059 & -709 & -0.0278  \\
100015 & -396 & 1 & 0.0316 & -1.27 & 100015 & -719 & -0.0282 & 100060 & -388 & 1 & 0.0309 & -1.24 & 100060 & -710 & -0.0278  \\
100016 & -396 & 1 & 0.0316 & -1.27 & 100016 & -718 & -0.0281 & 100061 & -389 & 1 & 0.0310 & -1.24 & 100061 & -711 & -0.0278  \\
100017 & -398 & 1 & 0.0317 & -1.27 & 100017 & -717 & -0.0281 & 100062 & -388 & 1 & 0.0309 & -1.24 & 100062 & -710 & -0.0278  \\
100018 & -399 & 1 & 0.0318 & -1.28 & 100018 & -718 & -0.0281 & 100063 & -387 & 1 & 0.0308 & -1.24 & 100063 & -709 & -0.0278  \\
100019 & -400 & 1 & 0.0319 & -1.28 & 100019 & -719 & -0.0282 & 100064 & -389 & 1 & 0.0310 & -1.24 & 100064 & -710 & -0.0278  \\
100020 & -405 & 1 & 0.0323 & -1.30 & 100020 & -718 & -0.0281 & 100065 & -388 & 1 & 0.0309 & -1.24 & 100065 & -709 & -0.0278  \\
100021 & -404 & 1 & 0.0322 & -1.29 & 100021 & -717 & -0.0281 & 100066 & -387 & 1 & 0.0308 & -1.24 & 100066 & -708 & -0.0277  \\
100022 & -405 & 1 & 0.0323 & -1.30 & 100022 & -718 & -0.0281 & 100067 & -389 & 1 & 0.0310 & -1.24 & 100067 & -707 & -0.0277  \\
100023 & -404 & 1 & 0.0322 & -1.29 & 100023 & -717 & -0.0281 & 100068 & -391 & 1 & 0.0312 & -1.25 & 100068 & -706 & -0.0276  \\
100024 & -403 & 1 & 0.0321 & -1.29 & 100024 & -716 & -0.0280 & 100069 & -392 & 1 & 0.0312 & -1.25 & 100069 & -707 & -0.0277  \\
100025 & -406 & 1 & 0.0324 & -1.30 & 100025 & -715 & -0.0280 & 100070 & -393 & 1 & 0.0313 & -1.26 & 100070 & -708 & -0.0277  \\
100026 & -404 & 1 & 0.0322 & -1.29 & 100026 & -716 & -0.0280 & 100071 & -395 & 1 & 0.0315 & -1.26 & 100071 & -707 & -0.0277  \\
100027 & -403 & 1 & 0.0321 & -1.29 & 100027 & -715 & -0.0280 & 100072 & -394 & 1 & 0.0314 & -1.26 & 100072 & -708 & -0.0277  \\
100028 & -402 & 1 & 0.0321 & -1.29 & 100028 & -716 & -0.0280 & 100073 & -395 & 1 & 0.0315 & -1.26 & 100073 & -709 & -0.0277  \\
100029 & -401 & 1 & 0.0320 & -1.28 & 100029 & -715 & -0.0280 & 100074 & -394 & 1 & 0.0314 & -1.26 & 100074 & -708 & -0.0277  \\
100030 & -400 & 1 & 0.0319 & -1.28 & 100030 & -714 & -0.0280 & 100075 & -397 & 1 & 0.0316 & -1.27 & 100075 & -707 & -0.0277  \\
100031 & -399 & 1 & 0.0318 & -1.28 & 100031 & -713 & -0.0279 & 100076 & -396 & 1 & 0.0316 & -1.27 & 100076 & -708 & -0.0277  \\
100032 & -394 & 1 & 0.0314 & -1.26 & 100032 & -714 & -0.0280 & 100077 & -395 & 1 & 0.0315 & -1.26 & 100077 & -707 & -0.0277  \\
100033 & -393 & 1 & 0.0313 & -1.26 & 100033 & -713 & -0.0279 & 100078 & -396 & 1 & 0.0316 & -1.27 & 100078 & -708 & -0.0277  \\
100034 & -394 & 1 & 0.0314 & -1.26 & 100034 & -714 & -0.0280 & 100079 & -394 & 1 & 0.0314 & -1.26 & 100079 & -709 & -0.0277  \\
100035 & -397 & 1 & 0.0317 & -1.27 & 100035 & -713 & -0.0279 & 100080 & -384 & 1 & 0.0306 & -1.23 & 100080 & -708 & -0.0277  \\
100036 & -396 & 1 & 0.0316 & -1.27 & 100036 & -714 & -0.0280 & 100081 & -383 & 1 & 0.0305 & -1.22 & 100081 & -707 & -0.0277  \\
100037 & -397 & 1 & 0.0317 & -1.27 & 100037 & -715 & -0.0280 & 100082 & -384 & 1 & 0.0306 & -1.23 & 100082 & -708 & -0.0277  \\
100038 & -398 & 1 & 0.0317 & -1.27 & 100038 & -716 & -0.0280 & 100083 & -385 & 1 & 0.0307 & -1.23 & 100083 & -709 & -0.0277  \\
100039 & -397 & 1 & 0.0317 & -1.27 & 100039 & -715 & -0.0280 & 100084 & -384 & 1 & 0.0306 & -1.23 & 100084 & -710 & -0.0278  \\
100040 & -396 & 1 & 0.0316 & -1.27 & 100040 & -714 & -0.0280 & 100085 & -385 & 1 & 0.0307 & -1.23 & 100085 & -711 & -0.0278  \\
100041 & -395 & 1 & 0.0315 & -1.26 & 100041 & -713 & -0.0279 & 100086 & -383 & 1 & 0.0305 & -1.22 & 100086 & -710 & -0.0278  \\
100042 & -394 & 1 & 0.0314 & -1.26 & 100042 & -712 & -0.0279 & 100087 & -382 & 1 & 0.0305 & -1.22 & 100087 & -709 & -0.0277  \\
100043 & -395 & 1 & 0.0315 & -1.26 & 100043 & -713 & -0.0279 & 100088 & -381 & 1 & 0.0304 & -1.22 & 100088 & -708 & -0.0277  \\
100044 & -390 & 1 & 0.0311 & -1.25 & 100044 & -712 & -0.0279 & 100089 & -383 & 1 & 0.0305 & -1.22 & 100089 & -709 & -0.0277  \\
\end{array}
}
\end{equation*} 

\bigskip\hrule\smallskip 

\captionsetup{singlelinecheck=off} 
\caption*{\textbf{\rm \bf Table \thesection:} 
          \textbf{Approximations to the summatory functions of $\lambda(n)$ and $\lambda_{\ast}(n)$. } 
          \begin{itemize}[noitemsep,topsep=0pt,leftmargin=0.23in] 
          \item[$\blacktriangleright$] 
          We define the exact summatory functions over these sequences by 
          $L(x) := \sum_{n \leq x} \lambda(n)$ and $L_{\ast}(n) := \sum_{n \leq x} \lambda_{\ast}(n)$. 
          \item[$\blacktriangleright$] Let the expected sign ratio function be defined by 
          $R_{\pm}(x) := \frac{\operatorname{sgn}(L(x))}{(-1)^{\ceiling{\log\log x}}}$. 
          \item[$\blacktriangleright$] 
          We compare the ratios of the following two functions with $L(x)$: 
          $L_{\approx,1}(x) := \sum_{k=1}^{\log\log x} \frac{x}{\log x} \cdot \frac{(-\log\log x)^{k-1}}{(k-1)!}$ and 
          $L_{\approx,2}(x) := \frac{x^{3/4}}{(\log x) \sqrt{\log\log x}}$. 
          \item[$\blacktriangleright$] Finally, we compare the approximations (very accurate) to 
          $L_{\ast}(x)$ by the summatory function $\sum_{k \leq x} (-1)^{\omega(n)}$ using the 
          approximation $L_{\approx}^{\ast}(x) := \frac{x}{\sqrt{2\pi} \sqrt{\log\log x}}$. 
          \end{itemize} 
          We are expecting to see and verify numerically that for sufficiently large $x$ the following properties: 
          \begin{itemize}[noitemsep,topsep=0pt,leftmargin=0.23in] 
          \item[$\blacktriangleright$] Almost always we have that $R_{\pm}(x) = -1$. 
          \item[$\blacktriangleright$] The ratio $\frac{L(x)}{L_{\approx,1}(x)}$ should be bounded by a constant approximately 
                                       equal to one, and 
                                       the ratio $\frac{L(x)}{L_{\approx,2}(x)}$ should be at least one. 
          \item[$\blacktriangleright$] The ratio $\frac{L_{\ast}(x)}{L_{\approx}^{\ast}(x)}$ tends towards an absolute constant.   
          \end{itemize} 
          The summatory functions $L(x)$ and $L_{\ast}(x)$ are numerically intensive to compute 
          directly using standard packages for large $x$. 
          We have written a software package in \cite{SCHMIDT-MERTENS-COMPUTATIONS} in 
          \texttt{Python3} for use with the \texttt{SageMath} platform that employs known 
          algorithms for more efficiently computing these functions. 
          } 

\end{table}
\clearpage 

\newpage
\begin{table}[ht!] 

\centering
\tiny 
\begin{equation*} 
\boxed{
\begin{array}{ccccc|ccc||ccccc|ccc} 
x & L(x) & R_{\pm}(x) & 
    \frac{L(x)}{L_{\approx,1}(x)} & \frac{L(x)}{L_{\approx,2}(x)} & 
    x & L_{\ast}(x) & \frac{L_{\ast}(x)}{L_{\approx}^{\ast}(x)} & 
x & L(x) & R_{\pm}(x) & 
    \frac{L(x)}{L_{\approx,1}(x)} & \frac{L(x)}{L_{\approx,2}(x)} & 
    x & L_{\ast}(x) & \frac{L_{\ast}(x)}{L_{\approx}^{\ast}(x)} \\ \hline 
100090 & -384 & 1 & 0.0306 & -1.23 & 100090 & -710 & -0.0278 & 100165 & -370 & 1 & 0.0295 & -1.18 & 100165 & -707 & -0.0277  \\
100091 & -383 & 1 & 0.0305 & -1.22 & 100091 & -709 & -0.0277 & 100166 & -369 & 1 & 0.0294 & -1.18 & 100166 & -706 & -0.0276  \\
100092 & -385 & 1 & 0.0307 & -1.23 & 100092 & -708 & -0.0277 & 100167 & -370 & 1 & 0.0295 & -1.18 & 100167 & -707 & -0.0277  \\
100093 & -386 & 1 & 0.0308 & -1.23 & 100093 & -709 & -0.0277 & 100168 & -371 & 1 & 0.0296 & -1.19 & 100168 & -708 & -0.0277  \\
100094 & -385 & 1 & 0.0307 & -1.23 & 100094 & -708 & -0.0277 & 100169 & -372 & 1 & 0.0296 & -1.19 & 100169 & -709 & -0.0277  \\
100095 & -386 & 1 & 0.0308 & -1.23 & 100095 & -709 & -0.0277 & 100170 & -383 & 1 & 0.0305 & -1.22 & 100170 & -710 & -0.0278  \\
100096 & -382 & 1 & 0.0305 & -1.22 & 100096 & -710 & -0.0278 & 100171 & -382 & 1 & 0.0304 & -1.22 & 100171 & -709 & -0.0277  \\
100097 & -381 & 1 & 0.0304 & -1.22 & 100097 & -709 & -0.0277 & 100172 & -386 & 1 & 0.0307 & -1.23 & 100172 & -710 & -0.0278  \\
100098 & -383 & 1 & 0.0305 & -1.22 & 100098 & -708 & -0.0277 & 100173 & -385 & 1 & 0.0307 & -1.23 & 100173 & -709 & -0.0277  \\
100099 & -382 & 1 & 0.0305 & -1.22 & 100099 & -707 & -0.0277 & 100174 & -384 & 1 & 0.0306 & -1.23 & 100174 & -708 & -0.0277  \\
100100 & -389 & 1 & 0.0310 & -1.24 & 100100 & -708 & -0.0277 & 100175 & -387 & 1 & 0.0308 & -1.24 & 100175 & -707 & -0.0276  \\
100101 & -390 & 1 & 0.0311 & -1.25 & 100101 & -709 & -0.0277 & 100176 & -384 & 1 & 0.0306 & -1.23 & 100176 & -708 & -0.0277  \\
100102 & -389 & 1 & 0.0310 & -1.24 & 100102 & -708 & -0.0277 & 100177 & -385 & 1 & 0.0307 & -1.23 & 100177 & -709 & -0.0277  \\
100103 & -390 & 1 & 0.0311 & -1.25 & 100103 & -709 & -0.0277 & 100178 & -386 & 1 & 0.0307 & -1.23 & 100178 & -710 & -0.0278  \\
100104 & -388 & 1 & 0.0309 & -1.24 & 100104 & -708 & -0.0277 & 100179 & -388 & 1 & 0.0309 & -1.24 & 100179 & -709 & -0.0277  \\
100105 & -387 & 1 & 0.0308 & -1.24 & 100105 & -707 & -0.0277 & 100180 & -386 & 1 & 0.0307 & -1.23 & 100180 & -710 & -0.0278  \\
100106 & -386 & 1 & 0.0308 & -1.23 & 100106 & -706 & -0.0276 & 100181 & -387 & 1 & 0.0308 & -1.24 & 100181 & -711 & -0.0278  \\
100107 & -393 & 1 & 0.0313 & -1.26 & 100107 & -707 & -0.0277 & 100182 & -386 & 1 & 0.0307 & -1.23 & 100182 & -710 & -0.0278  \\
100108 & -392 & 1 & 0.0312 & -1.25 & 100108 & -708 & -0.0277 & 100183 & -387 & 1 & 0.0308 & -1.24 & 100183 & -711 & -0.0278  \\
100109 & -393 & 1 & 0.0313 & -1.26 & 100109 & -709 & -0.0277 & 100184 & -386 & 1 & 0.0307 & -1.23 & 100184 & -712 & -0.0278  \\
100110 & -390 & 1 & 0.0311 & -1.25 & 100110 & -710 & -0.0278 & 100185 & -387 & 1 & 0.0308 & -1.24 & 100185 & -713 & -0.0279  \\
100111 & -391 & 1 & 0.0312 & -1.25 & 100111 & -711 & -0.0278 & 100186 & -386 & 1 & 0.0307 & -1.23 & 100186 & -712 & -0.0278  \\
100112 & -393 & 1 & 0.0313 & -1.26 & 100112 & -710 & -0.0278 & 100187 & -385 & 1 & 0.0307 & -1.23 & 100187 & -711 & -0.0278  \\
100113 & -392 & 1 & 0.0312 & -1.25 & 100113 & -709 & -0.0277 & 100188 & -397 & 1 & 0.0316 & -1.27 & 100188 & -710 & -0.0278  \\
100114 & -393 & 1 & 0.0313 & -1.26 & 100114 & -710 & -0.0278 & 100189 & -398 & 1 & 0.0317 & -1.27 & 100189 & -711 & -0.0278  \\
100115 & -392 & 1 & 0.0312 & -1.25 & 100115 & -709 & -0.0277 & 100190 & -397 & 1 & 0.0316 & -1.27 & 100190 & -710 & -0.0278  \\
100116 & -385 & 1 & 0.0307 & -1.23 & 100116 & -710 & -0.0278 & 100191 & -396 & 1 & 0.0315 & -1.27 & 100191 & -709 & -0.0277  \\
100117 & -384 & 1 & 0.0306 & -1.23 & 100117 & -709 & -0.0277 & 100192 & -393 & 1 & 0.0313 & -1.26 & 100192 & -710 & -0.0278  \\
100118 & -385 & 1 & 0.0307 & -1.23 & 100118 & -710 & -0.0278 & 100193 & -394 & 1 & 0.0314 & -1.26 & 100193 & -711 & -0.0278  \\
100119 & -386 & 1 & 0.0308 & -1.23 & 100119 & -711 & -0.0278 & 100194 & -395 & 1 & 0.0314 & -1.26 & 100194 & -712 & -0.0278  \\
100120 & -388 & 1 & 0.0309 & -1.24 & 100120 & -712 & -0.0279 & 100195 & -396 & 1 & 0.0315 & -1.27 & 100195 & -713 & -0.0279  \\
100121 & -387 & 1 & 0.0308 & -1.24 & 100121 & -711 & -0.0278 & 100196 & -395 & 1 & 0.0314 & -1.26 & 100196 & -714 & -0.0279  \\
100122 & -388 & 1 & 0.0309 & -1.24 & 100122 & -712 & -0.0279 & 100197 & -398 & 1 & 0.0317 & -1.27 & 100197 & -713 & -0.0279  \\
100123 & -387 & 1 & 0.0308 & -1.24 & 100123 & -711 & -0.0278 & 100198 & -397 & 1 & 0.0316 & -1.27 & 100198 & -712 & -0.0278  \\
100124 & -388 & 1 & 0.0309 & -1.24 & 100124 & -710 & -0.0278 & 100199 & -396 & 1 & 0.0315 & -1.27 & 100199 & -711 & -0.0278  \\
100125 & -383 & 1 & 0.0305 & -1.22 & 100125 & -711 & -0.0278 & 100200 & -401 & 1 & 0.0319 & -1.28 & 100200 & -710 & -0.0278  \\
100126 & -384 & 1 & 0.0306 & -1.23 & 100126 & -712 & -0.0279 & 100201 & -400 & 1 & 0.0318 & -1.28 & 100201 & -709 & -0.0277  \\
100127 & -383 & 1 & 0.0305 & -1.22 & 100127 & -711 & -0.0278 & 100202 & -399 & 1 & 0.0318 & -1.27 & 100202 & -708 & -0.0277  \\
100128 & -381 & 1 & 0.0304 & -1.22 & 100128 & -710 & -0.0278 & 100203 & -400 & 1 & 0.0318 & -1.28 & 100203 & -709 & -0.0277  \\
100129 & -382 & 1 & 0.0304 & -1.22 & 100129 & -711 & -0.0278 & 100204 & -401 & 1 & 0.0319 & -1.28 & 100204 & -708 & -0.0277  \\
100130 & -383 & 1 & 0.0305 & -1.22 & 100130 & -712 & -0.0279 & 100205 & -398 & 1 & 0.0317 & -1.27 & 100205 & -709 & -0.0277  \\
100131 & -382 & 1 & 0.0304 & -1.22 & 100131 & -711 & -0.0278 & 100206 & -398 & 1 & 0.0317 & -1.27 & 100206 & -708 & -0.0277  \\
100132 & -383 & 1 & 0.0305 & -1.22 & 100132 & -710 & -0.0278 & 100207 & -399 & 1 & 0.0318 & -1.27 & 100207 & -709 & -0.0277  \\
100133 & -382 & 1 & 0.0304 & -1.22 & 100133 & -709 & -0.0277 & 100208 & -401 & 1 & 0.0319 & -1.28 & 100208 & -708 & -0.0277  \\
100134 & -380 & 1 & 0.0303 & -1.21 & 100134 & -710 & -0.0278 & 100209 & -400 & 1 & 0.0318 & -1.28 & 100209 & -707 & -0.0276  \\
100135 & -381 & 1 & 0.0304 & -1.22 & 100135 & -711 & -0.0278 & 100210 & -399 & 1 & 0.0318 & -1.27 & 100210 & -706 & -0.0276  \\
100136 & -380 & 1 & 0.0303 & -1.21 & 100136 & -710 & -0.0278 & 100211 & -398 & 1 & 0.0317 & -1.27 & 100211 & -705 & -0.0276  \\
100137 & -381 & 1 & 0.0304 & -1.22 & 100137 & -711 & -0.0278 & 100212 & -401 & 1 & 0.0319 & -1.28 & 100212 & -704 & -0.0275  \\
100138 & -380 & 1 & 0.0303 & -1.21 & 100138 & -710 & -0.0278 & 100213 & -402 & 1 & 0.0320 & -1.28 & 100213 & -705 & -0.0276  \\
100139 & -379 & 1 & 0.0302 & -1.21 & 100139 & -709 & -0.0277 & 100214 & -403 & 1 & 0.0321 & -1.29 & 100214 & -706 & -0.0276  \\
100140 & -384 & 1 & 0.0306 & -1.23 & 100140 & -708 & -0.0277 & 100215 & -405 & 1 & 0.0322 & -1.29 & 100215 & -705 & -0.0276  \\
100141 & -383 & 1 & 0.0305 & -1.22 & 100141 & -707 & -0.0277 & 100216 & -404 & 1 & 0.0322 & -1.29 & 100216 & -704 & -0.0275  \\
100142 & -382 & 1 & 0.0304 & -1.22 & 100142 & -706 & -0.0276 & 100217 & -408 & 1 & 0.0325 & -1.30 & 100217 & -703 & -0.0275  \\
100143 & -380 & 1 & 0.0303 & -1.21 & 100143 & -705 & -0.0276 & 100218 & -409 & 1 & 0.0326 & -1.31 & 100218 & -704 & -0.0275  \\
100144 & -378 & 1 & 0.0301 & -1.21 & 100144 & -706 & -0.0276 & 100219 & -410 & 1 & 0.0326 & -1.31 & 100219 & -705 & -0.0276  \\
100145 & -377 & 1 & 0.0300 & -1.21 & 100145 & -705 & -0.0276 & 100220 & -408 & 1 & 0.0325 & -1.30 & 100220 & -706 & -0.0276  \\
100146 & -378 & 1 & 0.0301 & -1.21 & 100146 & -706 & -0.0276 & 100221 & -409 & 1 & 0.0326 & -1.31 & 100221 & -707 & -0.0276  \\
100147 & -379 & 1 & 0.0302 & -1.21 & 100147 & -707 & -0.0277 & 100222 & -408 & 1 & 0.0325 & -1.30 & 100222 & -706 & -0.0276  \\
100148 & -380 & 1 & 0.0303 & -1.21 & 100148 & -706 & -0.0276 & 100223 & -409 & 1 & 0.0326 & -1.31 & 100223 & -707 & -0.0276  \\
100149 & -379 & 1 & 0.0302 & -1.21 & 100149 & -705 & -0.0276 & 100224 & -422 & 1 & 0.0336 & -1.35 & 100224 & -708 & -0.0277  \\
100150 & -376 & 1 & 0.0300 & -1.20 & 100150 & -706 & -0.0276 & 100225 & -419 & 1 & 0.0334 & -1.34 & 100225 & -709 & -0.0277  \\
100151 & -377 & 1 & 0.0300 & -1.21 & 100151 & -707 & -0.0277 & 100226 & -420 & 1 & 0.0334 & -1.34 & 100226 & -710 & -0.0277  \\
100152 & -381 & 1 & 0.0304 & -1.22 & 100152 & -706 & -0.0276 & 100227 & -419 & 1 & 0.0334 & -1.34 & 100227 & -709 & -0.0277  \\
100153 & -382 & 1 & 0.0304 & -1.22 & 100153 & -707 & -0.0277 & 100228 & -420 & 1 & 0.0334 & -1.34 & 100228 & -708 & -0.0277  \\
100154 & -381 & 1 & 0.0304 & -1.22 & 100154 & -706 & -0.0276 & 100229 & -419 & 1 & 0.0334 & -1.34 & 100229 & -707 & -0.0276  \\
100155 & -380 & 1 & 0.0303 & -1.21 & 100155 & -705 & -0.0276 & 100230 & -422 & 1 & 0.0336 & -1.35 & 100230 & -708 & -0.0277  \\
100156 & -375 & 1 & 0.0299 & -1.20 & 100156 & -706 & -0.0276 & 100231 & -421 & 1 & 0.0335 & -1.35 & 100231 & -707 & -0.0276  \\
100157 & -374 & 1 & 0.0298 & -1.20 & 100157 & -705 & -0.0276 & 100232 & -420 & 1 & 0.0334 & -1.34 & 100232 & -706 & -0.0276  \\
100158 & -375 & 1 & 0.0299 & -1.20 & 100158 & -706 & -0.0276 & 100233 & -422 & 1 & 0.0336 & -1.35 & 100233 & -705 & -0.0276  \\
100159 & -374 & 1 & 0.0298 & -1.20 & 100159 & -705 & -0.0276 & 100234 & -423 & 1 & 0.0337 & -1.35 & 100234 & -706 & -0.0276  \\
100160 & -369 & 1 & 0.0294 & -1.18 & 100160 & -706 & -0.0276 & 100235 & -422 & 1 & 0.0336 & -1.35 & 100235 & -705 & -0.0276  \\
100161 & -367 & 1 & 0.0292 & -1.17 & 100161 & -707 & -0.0277 & 100236 & -420 & 1 & 0.0334 & -1.34 & 100236 & -706 & -0.0276  \\
100162 & -368 & 1 & 0.0293 & -1.18 & 100162 & -708 & -0.0277 & 100237 & -421 & 1 & 0.0335 & -1.35 & 100237 & -707 & -0.0276  \\
100163 & -369 & 1 & 0.0294 & -1.18 & 100163 & -709 & -0.0277 & 100238 & -420 & 1 & 0.0334 & -1.34 & 100238 & -706 & -0.0276  \\
100164 & -371 & 1 & 0.0296 & -1.19 & 100164 & -708 & -0.0277 & 100239 & -419 & 1 & 0.0334 & -1.34 & 100239 & -705 & -0.0275  \\
\end{array}
}
\end{equation*} 

\end{table}
\clearpage 


\newpage
\begin{table}[ht!] 

\centering
\tiny 
\begin{equation*} 
\boxed{
\begin{array}{ccccc|ccc||ccccc|ccc} 
x & L(x) & R_{\pm}(x) & 
    \frac{L(x)}{L_{\approx,1}(x)} & \frac{L(x)}{L_{\approx,2}(x)} & 
    x & L_{\ast}(x) & \frac{L_{\ast}(x)}{L_{\approx}^{\ast}(x)} & 
x & L(x) & R_{\pm}(x) & 
    \frac{L(x)}{L_{\approx,1}(x)} & \frac{L(x)}{L_{\approx,2}(x)} & 
    x & L_{\ast}(x) & \frac{L_{\ast}(x)}{L_{\approx}^{\ast}(x)} \\ \hline 
100240 & -420 & 1 & 0.0334 & -1.34 & 100240 & -704 & -0.0275 & 100315 & -410 & 1 & 0.0326 & -1.31 & 100315 & -699 & -0.0273  \\
100241 & -419 & 1 & 0.0333 & -1.34 & 100241 & -703 & -0.0275 & 100316 & -409 & 1 & 0.0325 & -1.31 & 100316 & -700 & -0.0273  \\
100242 & -417 & 1 & 0.0332 & -1.33 & 100242 & -704 & -0.0275 & 100317 & -408 & 1 & 0.0324 & -1.30 & 100317 & -699 & -0.0273  \\
100243 & -418 & 1 & 0.0333 & -1.34 & 100243 & -705 & -0.0275 & 100318 & -407 & 1 & 0.0324 & -1.30 & 100318 & -698 & -0.0273  \\
100244 & -417 & 1 & 0.0332 & -1.33 & 100244 & -706 & -0.0276 & 100319 & -406 & 1 & 0.0323 & -1.30 & 100319 & -697 & -0.0272  \\
100245 & -416 & 1 & 0.0331 & -1.33 & 100245 & -705 & -0.0275 & 100320 & -415 & 1 & 0.0330 & -1.32 & 100320 & -698 & -0.0273  \\
100246 & -415 & 1 & 0.0330 & -1.33 & 100246 & -704 & -0.0275 & 100321 & -414 & 1 & 0.0329 & -1.32 & 100321 & -697 & -0.0272  \\
100247 & -414 & 1 & 0.0329 & -1.32 & 100247 & -703 & -0.0275 & 100322 & -415 & 1 & 0.0330 & -1.32 & 100322 & -698 & -0.0273  \\
100248 & -416 & 1 & 0.0331 & -1.33 & 100248 & -704 & -0.0275 & 100323 & -413 & 1 & 0.0328 & -1.32 & 100323 & -699 & -0.0273  \\
100249 & -415 & 1 & 0.0330 & -1.33 & 100249 & -703 & -0.0275 & 100324 & -412 & 1 & 0.0328 & -1.32 & 100324 & -700 & -0.0273  \\
100250 & -418 & 1 & 0.0333 & -1.34 & 100250 & -704 & -0.0275 & 100325 & -415 & 1 & 0.0330 & -1.32 & 100325 & -699 & -0.0273  \\
100251 & -420 & 1 & 0.0334 & -1.34 & 100251 & -705 & -0.0275 & 100326 & -414 & 1 & 0.0329 & -1.32 & 100326 & -698 & -0.0273  \\
100252 & -419 & 1 & 0.0333 & -1.34 & 100252 & -706 & -0.0276 & 100327 & -413 & 1 & 0.0328 & -1.32 & 100327 & -697 & -0.0272  \\
100253 & -418 & 1 & 0.0333 & -1.34 & 100253 & -705 & -0.0275 & 100328 & -412 & 1 & 0.0328 & -1.32 & 100328 & -696 & -0.0272  \\
100254 & -417 & 1 & 0.0332 & -1.33 & 100254 & -706 & -0.0276 & 100329 & -413 & 1 & 0.0328 & -1.32 & 100329 & -697 & -0.0272  \\
100255 & -416 & 1 & 0.0331 & -1.33 & 100255 & -705 & -0.0275 & 100330 & -412 & 1 & 0.0328 & -1.32 & 100330 & -696 & -0.0272  \\
100256 & -418 & 1 & 0.0333 & -1.34 & 100256 & -706 & -0.0276 & 100331 & -413 & 1 & 0.0328 & -1.32 & 100331 & -697 & -0.0272  \\
100257 & -419 & 1 & 0.0333 & -1.34 & 100257 & -707 & -0.0276 & 100332 & -409 & 1 & 0.0325 & -1.31 & 100332 & -698 & -0.0273  \\
100258 & -418 & 1 & 0.0333 & -1.34 & 100258 & -706 & -0.0276 & 100333 & -410 & 1 & 0.0326 & -1.31 & 100333 & -699 & -0.0273  \\
100259 & -417 & 1 & 0.0332 & -1.33 & 100259 & -705 & -0.0275 & 100334 & -409 & 1 & 0.0325 & -1.31 & 100334 & -698 & -0.0273  \\
100260 & -411 & 1 & 0.0327 & -1.31 & 100260 & -704 & -0.0275 & 100335 & -410 & 1 & 0.0326 & -1.31 & 100335 & -699 & -0.0273  \\
100261 & -410 & 1 & 0.0326 & -1.31 & 100261 & -703 & -0.0275 & 100336 & -412 & 1 & 0.0328 & -1.32 & 100336 & -698 & -0.0273  \\
100262 & -409 & 1 & 0.0325 & -1.31 & 100262 & -702 & -0.0274 & 100337 & -411 & 1 & 0.0327 & -1.31 & 100337 & -697 & -0.0272  \\
100263 & -410 & 1 & 0.0326 & -1.31 & 100263 & -703 & -0.0275 & 100338 & -409 & 1 & 0.0325 & -1.31 & 100338 & -696 & -0.0272  \\
100264 & -411 & 1 & 0.0327 & -1.31 & 100264 & -704 & -0.0275 & 100339 & -408 & 1 & 0.0324 & -1.30 & 100339 & -695 & -0.0271  \\
100265 & -412 & 1 & 0.0328 & -1.32 & 100265 & -705 & -0.0275 & 100340 & -410 & 1 & 0.0326 & -1.31 & 100340 & -694 & -0.0271  \\
100266 & -411 & 1 & 0.0327 & -1.31 & 100266 & -704 & -0.0275 & 100341 & -412 & 1 & 0.0328 & -1.32 & 100341 & -693 & -0.0271  \\
100267 & -412 & 1 & 0.0328 & -1.32 & 100267 & -705 & -0.0275 & 100342 & -413 & 1 & 0.0328 & -1.32 & 100342 & -694 & -0.0271  \\
100268 & -411 & 1 & 0.0327 & -1.31 & 100268 & -706 & -0.0276 & 100343 & -414 & 1 & 0.0329 & -1.32 & 100343 & -695 & -0.0271  \\
100269 & -409 & 1 & 0.0325 & -1.31 & 100269 & -707 & -0.0276 & 100344 & -412 & 1 & 0.0328 & -1.32 & 100344 & -694 & -0.0271  \\
100270 & -408 & 1 & 0.0325 & -1.30 & 100270 & -706 & -0.0276 & 100345 & -411 & 1 & 0.0327 & -1.31 & 100345 & -693 & -0.0271  \\
100271 & -409 & 1 & 0.0325 & -1.31 & 100271 & -707 & -0.0276 & 100346 & -412 & 1 & 0.0328 & -1.32 & 100346 & -694 & -0.0271  \\
100272 & -406 & 1 & 0.0323 & -1.30 & 100272 & -708 & -0.0277 & 100347 & -411 & 1 & 0.0327 & -1.31 & 100347 & -693 & -0.0271  \\
100273 & -405 & 1 & 0.0322 & -1.29 & 100273 & -707 & -0.0276 & 100348 & -412 & 1 & 0.0328 & -1.32 & 100348 & -692 & -0.0270  \\
100274 & -406 & 1 & 0.0323 & -1.30 & 100274 & -708 & -0.0277 & 100349 & -411 & 1 & 0.0327 & -1.31 & 100349 & -691 & -0.0270  \\
100275 & -409 & 1 & 0.0325 & -1.31 & 100275 & -707 & -0.0276 & 100350 & -405 & 1 & 0.0322 & -1.29 & 100350 & -690 & -0.0269  \\
100276 & -407 & 1 & 0.0324 & -1.30 & 100276 & -706 & -0.0276 & 100351 & -404 & 1 & 0.0321 & -1.29 & 100351 & -689 & -0.0269  \\
100277 & -406 & 1 & 0.0323 & -1.30 & 100277 & -705 & -0.0275 & 100352 & -422 & 1 & 0.0336 & -1.35 & 100352 & -688 & -0.0269  \\
100278 & -403 & 1 & 0.0321 & -1.29 & 100278 & -706 & -0.0276 & 100353 & -423 & 1 & 0.0336 & -1.35 & 100353 & -689 & -0.0269  \\
100279 & -404 & 1 & 0.0322 & -1.29 & 100279 & -707 & -0.0276 & 100354 & -422 & 1 & 0.0336 & -1.35 & 100354 & -688 & -0.0269  \\
100280 & -402 & 1 & 0.0320 & -1.28 & 100280 & -706 & -0.0276 & 100355 & -421 & 1 & 0.0335 & -1.34 & 100355 & -687 & -0.0268  \\
100281 & -401 & 1 & 0.0319 & -1.28 & 100281 & -705 & -0.0275 & 100356 & -419 & 1 & 0.0333 & -1.34 & 100356 & -688 & -0.0269  \\
100282 & -405 & 1 & 0.0322 & -1.29 & 100282 & -706 & -0.0276 & 100357 & -420 & 1 & 0.0334 & -1.34 & 100357 & -689 & -0.0269  \\
100283 & -407 & 1 & 0.0324 & -1.30 & 100283 & -705 & -0.0275 & 100358 & -416 & 1 & 0.0331 & -1.33 & 100358 & -690 & -0.0269  \\
100284 & -409 & 1 & 0.0325 & -1.31 & 100284 & -704 & -0.0275 & 100359 & -419 & 1 & 0.0333 & -1.34 & 100359 & -691 & -0.0270  \\
100285 & -410 & 1 & 0.0326 & -1.31 & 100285 & -705 & -0.0275 & 100360 & -417 & 1 & 0.0331 & -1.33 & 100360 & -690 & -0.0269  \\
100286 & -411 & 1 & 0.0327 & -1.31 & 100286 & -706 & -0.0276 & 100361 & -418 & 1 & 0.0332 & -1.33 & 100361 & -691 & -0.0270  \\
100287 & -409 & 1 & 0.0325 & -1.31 & 100287 & -707 & -0.0276 & 100362 & -417 & 1 & 0.0331 & -1.33 & 100362 & -690 & -0.0269  \\
100288 & -413 & 1 & 0.0329 & -1.32 & 100288 & -706 & -0.0276 & 100363 & -418 & 1 & 0.0332 & -1.33 & 100363 & -691 & -0.0270  \\
100289 & -412 & 1 & 0.0328 & -1.32 & 100289 & -705 & -0.0275 & 100364 & -417 & 1 & 0.0331 & -1.33 & 100364 & -692 & -0.0270  \\
100290 & -409 & 1 & 0.0325 & -1.31 & 100290 & -704 & -0.0275 & 100365 & -418 & 1 & 0.0332 & -1.33 & 100365 & -693 & -0.0270  \\
100291 & -410 & 1 & 0.0326 & -1.31 & 100291 & -705 & -0.0275 & 100366 & -417 & 1 & 0.0331 & -1.33 & 100366 & -692 & -0.0270  \\
100292 & -411 & 1 & 0.0327 & -1.31 & 100292 & -704 & -0.0275 & 100367 & -416 & 1 & 0.0331 & -1.33 & 100367 & -691 & -0.0270  \\
100293 & -412 & 1 & 0.0328 & -1.32 & 100293 & -705 & -0.0275 & 100368 & -408 & 1 & 0.0324 & -1.30 & 100368 & -690 & -0.0269  \\
100294 & -411 & 1 & 0.0327 & -1.31 & 100294 & -704 & -0.0275 & 100369 & -407 & 1 & 0.0323 & -1.30 & 100369 & -689 & -0.0269  \\
100295 & -412 & 1 & 0.0328 & -1.32 & 100295 & -705 & -0.0275 & 100370 & -408 & 1 & 0.0324 & -1.30 & 100370 & -690 & -0.0269  \\
100296 & -417 & 1 & 0.0332 & -1.33 & 100296 & -704 & -0.0275 & 100371 & -407 & 1 & 0.0323 & -1.30 & 100371 & -689 & -0.0269  \\
100297 & -418 & 1 & 0.0332 & -1.33 & 100297 & -705 & -0.0275 & 100372 & -406 & 1 & 0.0323 & -1.30 & 100372 & -690 & -0.0269  \\
100298 & -417 & 1 & 0.0332 & -1.33 & 100298 & -704 & -0.0275 & 100373 & -407 & 1 & 0.0323 & -1.30 & 100373 & -691 & -0.0270  \\
100299 & -418 & 1 & 0.0332 & -1.33 & 100299 & -705 & -0.0275 & 100374 & -408 & 1 & 0.0324 & -1.30 & 100374 & -692 & -0.0270  \\
100300 & -414 & 1 & 0.0329 & -1.32 & 100300 & -704 & -0.0275 & 100375 & -411 & 1 & 0.0327 & -1.31 & 100375 & -693 & -0.0270  \\
100301 & -413 & 1 & 0.0329 & -1.32 & 100301 & -703 & -0.0275 & 100376 & -410 & 1 & 0.0326 & -1.31 & 100376 & -692 & -0.0270  \\
100302 & -412 & 1 & 0.0328 & -1.32 & 100302 & -702 & -0.0274 & 100377 & -408 & 1 & 0.0324 & -1.30 & 100377 & -693 & -0.0270  \\
100303 & -409 & 1 & 0.0325 & -1.31 & 100303 & -703 & -0.0275 & 100378 & -409 & 1 & 0.0325 & -1.31 & 100378 & -694 & -0.0271  \\
100304 & -411 & 1 & 0.0327 & -1.31 & 100304 & -702 & -0.0274 & 100379 & -410 & 1 & 0.0326 & -1.31 & 100379 & -695 & -0.0271  \\
100305 & -413 & 1 & 0.0329 & -1.32 & 100305 & -703 & -0.0275 & 100380 & -402 & 1 & 0.0320 & -1.28 & 100380 & -696 & -0.0272  \\
100306 & -412 & 1 & 0.0328 & -1.32 & 100306 & -702 & -0.0274 & 100381 & -401 & 1 & 0.0319 & -1.28 & 100381 & -695 & -0.0271  \\
100307 & -411 & 1 & 0.0327 & -1.31 & 100307 & -701 & -0.0274 & 100382 & -402 & 1 & 0.0320 & -1.28 & 100382 & -696 & -0.0272  \\
100308 & -411 & 1 & 0.0327 & -1.31 & 100308 & -700 & -0.0273 & 100383 & -401 & 1 & 0.0319 & -1.28 & 100383 & -695 & -0.0271  \\
100309 & -413 & 1 & 0.0329 & -1.32 & 100309 & -699 & -0.0273 & 100384 & -399 & 1 & 0.0317 & -1.27 & 100384 & -694 & -0.0271  \\
100310 & -412 & 1 & 0.0328 & -1.32 & 100310 & -698 & -0.0273 & 100385 & -400 & 1 & 0.0318 & -1.28 & 100385 & -695 & -0.0271  \\
100311 & -413 & 1 & 0.0329 & -1.32 & 100311 & -699 & -0.0273 & 100386 & -404 & 1 & 0.0321 & -1.29 & 100386 & -694 & -0.0271  \\
100312 & -412 & 1 & 0.0328 & -1.32 & 100312 & -698 & -0.0273 & 100387 & -403 & 1 & 0.0320 & -1.29 & 100387 & -693 & -0.0270  \\
100313 & -413 & 1 & 0.0329 & -1.32 & 100313 & -699 & -0.0273 & 100388 & -404 & 1 & 0.0321 & -1.29 & 100388 & -692 & -0.0270  \\
100314 & -411 & 1 & 0.0327 & -1.31 & 100314 & -700 & -0.0273 & 100389 & -405 & 1 & 0.0322 & -1.29 & 100389 & -693 & -0.0270  \\
\end{array}
}
\end{equation*} 

\end{table}
\clearpage 


\newpage
\begin{table}[ht!] 

\centering
\tiny 
\begin{equation*} 
\boxed{
\begin{array}{ccccc|ccc||ccccc|ccc} 
x & L(x) & R_{\pm}(x) & 
    \frac{L(x)}{L_{\approx,1}(x)} & \frac{L(x)}{L_{\approx,2}(x)} & 
    x & L_{\ast}(x) & \frac{L_{\ast}(x)}{L_{\approx}^{\ast}(x)} & 
x & L(x) & R_{\pm}(x) & 
    \frac{L(x)}{L_{\approx,1}(x)} & \frac{L(x)}{L_{\approx,2}(x)} & 
    x & L_{\ast}(x) & \frac{L_{\ast}(x)}{L_{\approx}^{\ast}(x)} \\ \hline 
100390 & -406 & 1 & 0.0323 & -1.30 & 100390 & -694 & -0.0271 & 100465 & -400 & 1 & 0.0318 & -1.28 & 100465 & -685 & -0.0267  \\
100391 & -407 & 1 & 0.0323 & -1.30 & 100391 & -695 & -0.0271 & 100466 & -401 & 1 & 0.0318 & -1.28 & 100466 & -686 & -0.0268  \\
100392 & -405 & 1 & 0.0322 & -1.29 & 100392 & -694 & -0.0271 & 100467 & -413 & 1 & 0.0328 & -1.32 & 100467 & -685 & -0.0267  \\
100393 & -406 & 1 & 0.0323 & -1.30 & 100393 & -695 & -0.0271 & 100468 & -414 & 1 & 0.0329 & -1.32 & 100468 & -684 & -0.0267  \\
100394 & -405 & 1 & 0.0322 & -1.29 & 100394 & -694 & -0.0271 & 100469 & -415 & 1 & 0.0329 & -1.32 & 100469 & -685 & -0.0267  \\
100395 & -407 & 1 & 0.0323 & -1.30 & 100395 & -693 & -0.0270 & 100470 & -418 & 1 & 0.0332 & -1.33 & 100470 & -686 & -0.0268  \\
100396 & -406 & 1 & 0.0323 & -1.30 & 100396 & -694 & -0.0271 & 100471 & -419 & 1 & 0.0332 & -1.34 & 100471 & -687 & -0.0268  \\
100397 & -405 & 1 & 0.0322 & -1.29 & 100397 & -693 & -0.0270 & 100472 & -420 & 1 & 0.0333 & -1.34 & 100472 & -688 & -0.0268  \\
100398 & -404 & 1 & 0.0321 & -1.29 & 100398 & -692 & -0.0270 & 100473 & -421 & 1 & 0.0334 & -1.34 & 100473 & -689 & -0.0269  \\
100399 & -403 & 1 & 0.0320 & -1.29 & 100399 & -691 & -0.0270 & 100474 & -422 & 1 & 0.0335 & -1.35 & 100474 & -690 & -0.0269  \\
100400 & -412 & 1 & 0.0327 & -1.31 & 100400 & -692 & -0.0270 & 100475 & -425 & 1 & 0.0337 & -1.36 & 100475 & -689 & -0.0269  \\
100401 & -409 & 1 & 0.0325 & -1.31 & 100401 & -693 & -0.0270 & 100476 & -429 & 1 & 0.0341 & -1.37 & 100476 & -690 & -0.0269  \\
100402 & -410 & 1 & 0.0326 & -1.31 & 100402 & -694 & -0.0271 & 100477 & -430 & 1 & 0.0341 & -1.37 & 100477 & -691 & -0.0270  \\
100403 & -411 & 1 & 0.0327 & -1.31 & 100403 & -695 & -0.0271 & 100478 & -431 & 1 & 0.0342 & -1.38 & 100478 & -692 & -0.0270  \\
100404 & -415 & 1 & 0.0330 & -1.32 & 100404 & -696 & -0.0271 & 100479 & -430 & 1 & 0.0341 & -1.37 & 100479 & -691 & -0.0270  \\
100405 & -416 & 1 & 0.0331 & -1.33 & 100405 & -697 & -0.0272 & 100480 & -435 & 1 & 0.0345 & -1.39 & 100480 & -692 & -0.0270  \\
100406 & -417 & 1 & 0.0331 & -1.33 & 100406 & -698 & -0.0272 & 100481 & -434 & 1 & 0.0345 & -1.39 & 100481 & -691 & -0.0269  \\
100407 & -416 & 1 & 0.0331 & -1.33 & 100407 & -697 & -0.0272 & 100482 & -435 & 1 & 0.0345 & -1.39 & 100482 & -692 & -0.0270  \\
100408 & -417 & 1 & 0.0331 & -1.33 & 100408 & -696 & -0.0271 & 100483 & -436 & 1 & 0.0346 & -1.39 & 100483 & -693 & -0.0270  \\
100409 & -418 & 1 & 0.0332 & -1.33 & 100409 & -697 & -0.0272 & 100484 & -437 & 1 & 0.0347 & -1.39 & 100484 & -692 & -0.0270  \\
100410 & -415 & 1 & 0.0330 & -1.32 & 100410 & -696 & -0.0271 & 100485 & -435 & 1 & 0.0345 & -1.39 & 100485 & -693 & -0.0270  \\
100411 & -416 & 1 & 0.0331 & -1.33 & 100411 & -697 & -0.0272 & 100486 & -436 & 1 & 0.0346 & -1.39 & 100486 & -694 & -0.0271  \\
100412 & -415 & 1 & 0.0330 & -1.32 & 100412 & -698 & -0.0272 & 100487 & -437 & 1 & 0.0347 & -1.39 & 100487 & -695 & -0.0271  \\
100413 & -413 & 1 & 0.0328 & -1.32 & 100413 & -697 & -0.0272 & 100488 & -435 & 1 & 0.0345 & -1.39 & 100488 & -694 & -0.0271  \\
100414 & -412 & 1 & 0.0327 & -1.31 & 100414 & -696 & -0.0271 & 100489 & -428 & 1 & 0.0340 & -1.37 & 100489 & -695 & -0.0271  \\
100415 & -411 & 1 & 0.0327 & -1.31 & 100415 & -695 & -0.0271 & 100490 & -427 & 1 & 0.0339 & -1.36 & 100490 & -694 & -0.0271  \\
100416 & -406 & 1 & 0.0323 & -1.30 & 100416 & -696 & -0.0271 & 100491 & -426 & 1 & 0.0338 & -1.36 & 100491 & -693 & -0.0270  \\
100417 & -407 & 1 & 0.0323 & -1.30 & 100417 & -697 & -0.0272 & 100492 & -427 & 1 & 0.0339 & -1.36 & 100492 & -692 & -0.0270  \\
100418 & -406 & 1 & 0.0323 & -1.30 & 100418 & -696 & -0.0271 & 100493 & -428 & 1 & 0.0340 & -1.37 & 100493 & -693 & -0.0270  \\
100419 & -405 & 1 & 0.0322 & -1.29 & 100419 & -695 & -0.0271 & 100494 & -430 & 1 & 0.0341 & -1.37 & 100494 & -694 & -0.0271  \\
100420 & -403 & 1 & 0.0320 & -1.29 & 100420 & -696 & -0.0271 & 100495 & -431 & 1 & 0.0342 & -1.38 & 100495 & -695 & -0.0271  \\
100421 & -402 & 1 & 0.0320 & -1.28 & 100421 & -695 & -0.0271 & 100496 & -429 & 1 & 0.0340 & -1.37 & 100496 & -696 & -0.0271  \\
100422 & -405 & 1 & 0.0322 & -1.29 & 100422 & -694 & -0.0271 & 100497 & -430 & 1 & 0.0341 & -1.37 & 100497 & -697 & -0.0272  \\
100423 & -404 & 1 & 0.0321 & -1.29 & 100423 & -693 & -0.0270 & 100498 & -431 & 1 & 0.0342 & -1.38 & 100498 & -698 & -0.0272  \\
100424 & -403 & 1 & 0.0320 & -1.29 & 100424 & -692 & -0.0270 & 100499 & -428 & 1 & 0.0340 & -1.37 & 100499 & -697 & -0.0272  \\
100425 & -406 & 1 & 0.0323 & -1.30 & 100425 & -691 & -0.0270 & 100500 & -435 & 1 & 0.0345 & -1.39 & 100500 & -696 & -0.0271  \\
100426 & -407 & 1 & 0.0323 & -1.30 & 100426 & -692 & -0.0270 & 100501 & -436 & 1 & 0.0346 & -1.39 & 100501 & -697 & -0.0272  \\
100427 & -406 & 1 & 0.0323 & -1.30 & 100427 & -691 & -0.0270 & 100502 & -437 & 1 & 0.0347 & -1.39 & 100502 & -698 & -0.0272  \\
100428 & -404 & 1 & 0.0321 & -1.29 & 100428 & -692 & -0.0270 & 100503 & -435 & 1 & 0.0345 & -1.39 & 100503 & -699 & -0.0273  \\
100429 & -403 & 1 & 0.0320 & -1.29 & 100429 & -691 & -0.0270 & 100504 & -436 & 1 & 0.0346 & -1.39 & 100504 & -700 & -0.0273  \\
100430 & -405 & 1 & 0.0322 & -1.29 & 100430 & -690 & -0.0269 & 100505 & -435 & 1 & 0.0345 & -1.39 & 100505 & -699 & -0.0273  \\
100431 & -407 & 1 & 0.0323 & -1.30 & 100431 & -689 & -0.0269 & 100506 & -433 & 1 & 0.0344 & -1.38 & 100506 & -698 & -0.0272  \\
100432 & -409 & 1 & 0.0325 & -1.31 & 100432 & -688 & -0.0268 & 100507 & -432 & 1 & 0.0343 & -1.38 & 100507 & -697 & -0.0272  \\
100433 & -408 & 1 & 0.0324 & -1.30 & 100433 & -687 & -0.0268 & 100508 & -433 & 1 & 0.0344 & -1.38 & 100508 & -696 & -0.0271  \\
100434 & -407 & 1 & 0.0323 & -1.30 & 100434 & -686 & -0.0268 & 100509 & -432 & 1 & 0.0343 & -1.38 & 100509 & -695 & -0.0271  \\
100435 & -408 & 1 & 0.0324 & -1.30 & 100435 & -687 & -0.0268 & 100510 & -436 & 1 & 0.0346 & -1.39 & 100510 & -694 & -0.0271  \\
100436 & -409 & 1 & 0.0325 & -1.31 & 100436 & -686 & -0.0268 & 100511 & -437 & 1 & 0.0347 & -1.39 & 100511 & -695 & -0.0271  \\
100437 & -408 & 1 & 0.0324 & -1.30 & 100437 & -685 & -0.0267 & 100512 & -428 & 1 & 0.0340 & -1.37 & 100512 & -696 & -0.0271  \\
100438 & -409 & 1 & 0.0325 & -1.30 & 100438 & -686 & -0.0268 & 100513 & -429 & 1 & 0.0340 & -1.37 & 100513 & -697 & -0.0272  \\
100439 & -408 & 1 & 0.0324 & -1.30 & 100439 & -685 & -0.0267 & 100514 & -430 & 1 & 0.0341 & -1.37 & 100514 & -698 & -0.0272  \\
100440 & -415 & 1 & 0.0330 & -1.32 & 100440 & -684 & -0.0267 & 100515 & -431 & 1 & 0.0342 & -1.38 & 100515 & -699 & -0.0273  \\
100441 & -416 & 1 & 0.0330 & -1.33 & 100441 & -685 & -0.0267 & 100516 & -430 & 1 & 0.0341 & -1.37 & 100516 & -700 & -0.0273  \\
100442 & -415 & 1 & 0.0330 & -1.32 & 100442 & -684 & -0.0267 & 100517 & -431 & 1 & 0.0342 & -1.38 & 100517 & -701 & -0.0273  \\
100443 & -416 & 1 & 0.0330 & -1.33 & 100443 & -685 & -0.0267 & 100518 & -430 & 1 & 0.0341 & -1.37 & 100518 & -700 & -0.0273  \\
100444 & -417 & 1 & 0.0331 & -1.33 & 100444 & -684 & -0.0267 & 100519 & -431 & 1 & 0.0342 & -1.38 & 100519 & -701 & -0.0273  \\
100445 & -416 & 1 & 0.0330 & -1.33 & 100445 & -683 & -0.0266 & 100520 & -431 & 1 & 0.0342 & -1.38 & 100520 & -700 & -0.0273  \\
100446 & -417 & 1 & 0.0331 & -1.33 & 100446 & -684 & -0.0267 & 100521 & -428 & 1 & 0.0340 & -1.37 & 100521 & -701 & -0.0273  \\
100447 & -418 & 1 & 0.0332 & -1.33 & 100447 & -685 & -0.0267 & 100522 & -427 & 1 & 0.0339 & -1.36 & 100522 & -700 & -0.0273  \\
100448 & -420 & 1 & 0.0334 & -1.34 & 100448 & -686 & -0.0268 & 100523 & -428 & 1 & 0.0340 & -1.37 & 100523 & -701 & -0.0273  \\
100449 & -422 & 1 & 0.0335 & -1.35 & 100449 & -685 & -0.0267 & 100524 & -426 & 1 & 0.0338 & -1.36 & 100524 & -702 & -0.0274  \\
100450 & -411 & 1 & 0.0326 & -1.31 & 100450 & -684 & -0.0267 & 100525 & -429 & 1 & 0.0340 & -1.37 & 100525 & -701 & -0.0273  \\
100451 & -410 & 1 & 0.0326 & -1.31 & 100451 & -683 & -0.0266 & 100526 & -428 & 1 & 0.0340 & -1.37 & 100526 & -700 & -0.0273  \\
100452 & -411 & 1 & 0.0326 & -1.31 & 100452 & -682 & -0.0266 & 100527 & -429 & 1 & 0.0340 & -1.37 & 100527 & -701 & -0.0273  \\
100453 & -412 & 1 & 0.0327 & -1.31 & 100453 & -683 & -0.0266 & 100528 & -427 & 1 & 0.0339 & -1.36 & 100528 & -702 & -0.0274  \\
100454 & -411 & 1 & 0.0326 & -1.31 & 100454 & -682 & -0.0266 & 100529 & -426 & 1 & 0.0338 & -1.36 & 100529 & -701 & -0.0273  \\
100455 & -410 & 1 & 0.0326 & -1.31 & 100455 & -681 & -0.0266 & 100530 & -429 & 1 & 0.0340 & -1.37 & 100530 & -700 & -0.0273  \\
100456 & -411 & 1 & 0.0326 & -1.31 & 100456 & -682 & -0.0266 & 100531 & -428 & 1 & 0.0340 & -1.37 & 100531 & -699 & -0.0272  \\
100457 & -412 & 1 & 0.0327 & -1.31 & 100457 & -683 & -0.0266 & 100532 & -427 & 1 & 0.0339 & -1.36 & 100532 & -700 & -0.0273  \\
100458 & -410 & 1 & 0.0326 & -1.31 & 100458 & -684 & -0.0267 & 100533 & -426 & 1 & 0.0338 & -1.36 & 100533 & -699 & -0.0272  \\
100459 & -411 & 1 & 0.0326 & -1.31 & 100459 & -685 & -0.0267 & 100534 & -425 & 1 & 0.0337 & -1.36 & 100534 & -698 & -0.0272  \\
100460 & -409 & 1 & 0.0325 & -1.30 & 100460 & -686 & -0.0267 & 100535 & -424 & 1 & 0.0336 & -1.35 & 100535 & -697 & -0.0272  \\
100461 & -408 & 1 & 0.0324 & -1.30 & 100461 & -685 & -0.0267 & 100536 & -422 & 1 & 0.0335 & -1.35 & 100536 & -696 & -0.0271  \\
100462 & -407 & 1 & 0.0323 & -1.30 & 100462 & -684 & -0.0267 & 100537 & -423 & 1 & 0.0336 & -1.35 & 100537 & -697 & -0.0272  \\
100463 & -406 & 1 & 0.0322 & -1.30 & 100463 & -683 & -0.0266 & 100538 & -424 & 1 & 0.0336 & -1.35 & 100538 & -698 & -0.0272  \\
100464 & -399 & 1 & 0.0317 & -1.27 & 100464 & -684 & -0.0267 & 100539 & -426 & 1 & 0.0338 & -1.36 & 100539 & -697 & -0.0272  \\
\end{array}
}
\end{equation*} 

\end{table}
\clearpage 

\newpage
\begin{table}[ht!] 

\centering
\tiny 
\begin{equation*} 
\boxed{
\begin{array}{ccccc|ccc||ccccc|ccc} 
x & L(x) & R_{\pm}(x) & 
    \frac{L(x)}{L_{\approx,1}(x)} & \frac{L(x)}{L_{\approx,2}(x)} & 
    x & L_{\ast}(x) & \frac{L_{\ast}(x)}{L_{\approx}^{\ast}(x)} & 
x & L(x) & R_{\pm}(x) & 
    \frac{L(x)}{L_{\approx,1}(x)} & \frac{L(x)}{L_{\approx,2}(x)} & 
    x & L_{\ast}(x) & \frac{L_{\ast}(x)}{L_{\approx}^{\ast}(x)} \\ \hline 
100540 & -428 & 1 & 0.0340 & -1.37 & 100540 & -696 & -0.0271 & 100615 & -419 & 1 & 0.0332 & -1.34 & 100615 & -687 & -0.0268  \\
100541 & -429 & 1 & 0.0340 & -1.37 & 100541 & -697 & -0.0272 & 100616 & -418 & 1 & 0.0331 & -1.33 & 100616 & -686 & -0.0267  \\
100542 & -428 & 1 & 0.0340 & -1.37 & 100542 & -696 & -0.0271 & 100617 & -419 & 1 & 0.0332 & -1.34 & 100617 & -687 & -0.0268  \\
100543 & -427 & 1 & 0.0339 & -1.36 & 100543 & -695 & -0.0271 & 100618 & -420 & 1 & 0.0333 & -1.34 & 100618 & -688 & -0.0268  \\
100544 & -431 & 1 & 0.0342 & -1.37 & 100544 & -694 & -0.0271 & 100619 & -419 & 1 & 0.0332 & -1.34 & 100619 & -687 & -0.0268  \\
100545 & -432 & 1 & 0.0343 & -1.38 & 100545 & -695 & -0.0271 & 100620 & -429 & 1 & 0.0340 & -1.37 & 100620 & -688 & -0.0268  \\
100546 & -431 & 1 & 0.0342 & -1.37 & 100546 & -694 & -0.0271 & 100621 & -430 & 1 & 0.0341 & -1.37 & 100621 & -689 & -0.0268  \\
100547 & -432 & 1 & 0.0343 & -1.38 & 100547 & -695 & -0.0271 & 100622 & -429 & 1 & 0.0340 & -1.37 & 100622 & -688 & -0.0268  \\
100548 & -420 & 1 & 0.0333 & -1.34 & 100548 & -694 & -0.0271 & 100623 & -430 & 1 & 0.0341 & -1.37 & 100623 & -689 & -0.0268  \\
100549 & -421 & 1 & 0.0334 & -1.34 & 100549 & -695 & -0.0271 & 100624 & -428 & 1 & 0.0339 & -1.36 & 100624 & -690 & -0.0269  \\
100550 & -418 & 1 & 0.0332 & -1.33 & 100550 & -696 & -0.0271 & 100625 & -425 & 1 & 0.0337 & -1.35 & 100625 & -691 & -0.0269  \\
100551 & -416 & 1 & 0.0330 & -1.33 & 100551 & -697 & -0.0272 & 100626 & -424 & 1 & 0.0336 & -1.35 & 100626 & -690 & -0.0269  \\
100552 & -415 & 1 & 0.0329 & -1.32 & 100552 & -696 & -0.0271 & 100627 & -423 & 1 & 0.0335 & -1.35 & 100627 & -689 & -0.0268  \\
100553 & -414 & 1 & 0.0329 & -1.32 & 100553 & -695 & -0.0271 & 100628 & -422 & 1 & 0.0334 & -1.35 & 100628 & -690 & -0.0269  \\
100554 & -415 & 1 & 0.0329 & -1.32 & 100554 & -696 & -0.0271 & 100629 & -420 & 1 & 0.0333 & -1.34 & 100629 & -689 & -0.0268  \\
100555 & -419 & 1 & 0.0332 & -1.34 & 100555 & -695 & -0.0271 & 100630 & -419 & 1 & 0.0332 & -1.34 & 100630 & -688 & -0.0268  \\
100556 & -418 & 1 & 0.0331 & -1.33 & 100556 & -696 & -0.0271 & 100631 & -418 & 1 & 0.0331 & -1.33 & 100631 & -687 & -0.0267  \\
100557 & -420 & 1 & 0.0333 & -1.34 & 100557 & -695 & -0.0271 & 100632 & -417 & 1 & 0.0331 & -1.33 & 100632 & -686 & -0.0267  \\
100558 & -421 & 1 & 0.0334 & -1.34 & 100558 & -696 & -0.0271 & 100633 & -416 & 1 & 0.0330 & -1.33 & 100633 & -685 & -0.0267  \\
100559 & -422 & 1 & 0.0335 & -1.35 & 100559 & -697 & -0.0272 & 100634 & -417 & 1 & 0.0331 & -1.33 & 100634 & -686 & -0.0267  \\
100560 & -427 & 1 & 0.0339 & -1.36 & 100560 & -696 & -0.0271 & 100635 & -418 & 1 & 0.0331 & -1.33 & 100635 & -687 & -0.0267  \\
100561 & -426 & 1 & 0.0338 & -1.36 & 100561 & -695 & -0.0271 & 100636 & -417 & 1 & 0.0331 & -1.33 & 100636 & -688 & -0.0268  \\
100562 & -425 & 1 & 0.0337 & -1.36 & 100562 & -694 & -0.0270 & 100637 & -416 & 1 & 0.0330 & -1.33 & 100637 & -687 & -0.0267  \\
100563 & -424 & 1 & 0.0336 & -1.35 & 100563 & -693 & -0.0270 & 100638 & -414 & 1 & 0.0328 & -1.32 & 100638 & -688 & -0.0268  \\
100564 & -423 & 1 & 0.0335 & -1.35 & 100564 & -694 & -0.0270 & 100639 & -415 & 1 & 0.0329 & -1.32 & 100639 & -689 & -0.0268  \\
100565 & -422 & 1 & 0.0335 & -1.35 & 100565 & -693 & -0.0270 & 100640 & -412 & 1 & 0.0327 & -1.31 & 100640 & -688 & -0.0268  \\
100566 & -424 & 1 & 0.0336 & -1.35 & 100566 & -692 & -0.0270 & 100641 & -411 & 1 & 0.0326 & -1.31 & 100641 & -687 & -0.0267  \\
100567 & -425 & 1 & 0.0337 & -1.36 & 100567 & -693 & -0.0270 & 100642 & -410 & 1 & 0.0325 & -1.31 & 100642 & -686 & -0.0267  \\
100568 & -426 & 1 & 0.0338 & -1.36 & 100568 & -694 & -0.0270 & 100643 & -409 & 1 & 0.0324 & -1.30 & 100643 & -685 & -0.0267  \\
100569 & -427 & 1 & 0.0339 & -1.36 & 100569 & -695 & -0.0271 & 100644 & -407 & 1 & 0.0323 & -1.30 & 100644 & -686 & -0.0267  \\
100570 & -426 & 1 & 0.0338 & -1.36 & 100570 & -694 & -0.0270 & 100645 & -406 & 1 & 0.0322 & -1.29 & 100645 & -685 & -0.0267  \\
100571 & -425 & 1 & 0.0337 & -1.36 & 100571 & -693 & -0.0270 & 100646 & -412 & 1 & 0.0327 & -1.31 & 100646 & -684 & -0.0266  \\
100572 & -418 & 1 & 0.0331 & -1.33 & 100572 & -692 & -0.0270 & 100647 & -410 & 1 & 0.0325 & -1.31 & 100647 & -685 & -0.0267  \\
100573 & -419 & 1 & 0.0332 & -1.34 & 100573 & -693 & -0.0270 & 100648 & -411 & 1 & 0.0326 & -1.31 & 100648 & -686 & -0.0267  \\
100574 & -418 & 1 & 0.0331 & -1.33 & 100574 & -692 & -0.0270 & 100649 & -412 & 1 & 0.0327 & -1.31 & 100649 & -687 & -0.0267  \\
100575 & -413 & 1 & 0.0327 & -1.32 & 100575 & -693 & -0.0270 & 100650 & -407 & 1 & 0.0323 & -1.30 & 100650 & -688 & -0.0268  \\
100576 & -413 & 1 & 0.0327 & -1.32 & 100576 & -694 & -0.0270 & 100651 & -406 & 1 & 0.0322 & -1.29 & 100651 & -687 & -0.0267  \\
100577 & -412 & 1 & 0.0327 & -1.31 & 100577 & -693 & -0.0270 & 100652 & -407 & 1 & 0.0323 & -1.30 & 100652 & -686 & -0.0267  \\
100578 & -413 & 1 & 0.0327 & -1.32 & 100578 & -694 & -0.0270 & 100653 & -408 & 1 & 0.0323 & -1.30 & 100653 & -687 & -0.0267  \\
100579 & -412 & 1 & 0.0327 & -1.31 & 100579 & -693 & -0.0270 & 100654 & -409 & 1 & 0.0324 & -1.30 & 100654 & -688 & -0.0268  \\
100580 & -414 & 1 & 0.0328 & -1.32 & 100580 & -692 & -0.0270 & 100655 & -410 & 1 & 0.0325 & -1.31 & 100655 & -689 & -0.0268  \\
100581 & -415 & 1 & 0.0329 & -1.32 & 100581 & -693 & -0.0270 & 100656 & -401 & 1 & 0.0318 & -1.28 & 100656 & -690 & -0.0269  \\
100582 & -414 & 1 & 0.0328 & -1.32 & 100582 & -692 & -0.0270 & 100657 & -402 & 1 & 0.0319 & -1.28 & 100657 & -691 & -0.0269  \\
100583 & -413 & 1 & 0.0327 & -1.32 & 100583 & -691 & -0.0269 & 100658 & -401 & 1 & 0.0318 & -1.28 & 100658 & -690 & -0.0269  \\
100584 & -418 & 1 & 0.0331 & -1.33 & 100584 & -690 & -0.0269 & 100659 & -400 & 1 & 0.0317 & -1.27 & 100659 & -689 & -0.0268  \\
100585 & -417 & 1 & 0.0331 & -1.33 & 100585 & -689 & -0.0268 & 100660 & -402 & 1 & 0.0319 & -1.28 & 100660 & -688 & -0.0268  \\
100586 & -418 & 1 & 0.0331 & -1.33 & 100586 & -690 & -0.0269 & 100661 & -401 & 1 & 0.0318 & -1.28 & 100661 & -687 & -0.0267  \\
100587 & -417 & 1 & 0.0331 & -1.33 & 100587 & -689 & -0.0268 & 100662 & -400 & 1 & 0.0317 & -1.27 & 100662 & -686 & -0.0267  \\
100588 & -418 & 1 & 0.0331 & -1.33 & 100588 & -688 & -0.0268 & 100663 & -399 & 1 & 0.0316 & -1.27 & 100663 & -685 & -0.0267  \\
100589 & -419 & 1 & 0.0332 & -1.34 & 100589 & -689 & -0.0268 & 100664 & -398 & 1 & 0.0315 & -1.27 & 100664 & -684 & -0.0266  \\
100590 & -421 & 1 & 0.0334 & -1.34 & 100590 & -690 & -0.0269 & 100665 & -396 & 1 & 0.0314 & -1.26 & 100665 & -685 & -0.0267  \\
100591 & -422 & 1 & 0.0334 & -1.35 & 100591 & -691 & -0.0269 & 100666 & -395 & 1 & 0.0313 & -1.26 & 100666 & -684 & -0.0266  \\
100592 & -424 & 1 & 0.0336 & -1.35 & 100592 & -690 & -0.0269 & 100667 & -396 & 1 & 0.0314 & -1.26 & 100667 & -685 & -0.0267  \\
100593 & -426 & 1 & 0.0338 & -1.36 & 100593 & -689 & -0.0268 & 100668 & -394 & 1 & 0.0312 & -1.26 & 100668 & -686 & -0.0267  \\
100594 & -425 & 1 & 0.0337 & -1.36 & 100594 & -688 & -0.0268 & 100669 & -395 & 1 & 0.0313 & -1.26 & 100669 & -687 & -0.0267  \\
100595 & -424 & 1 & 0.0336 & -1.35 & 100595 & -687 & -0.0268 & 100670 & -396 & 1 & 0.0314 & -1.26 & 100670 & -688 & -0.0268  \\
100596 & -426 & 1 & 0.0338 & -1.36 & 100596 & -686 & -0.0267 & 100671 & -397 & 1 & 0.0315 & -1.27 & 100671 & -689 & -0.0268  \\
100597 & -429 & 1 & 0.0340 & -1.37 & 100597 & -685 & -0.0267 & 100672 & -410 & 1 & 0.0325 & -1.31 & 100672 & -690 & -0.0268  \\
100598 & -430 & 1 & 0.0341 & -1.37 & 100598 & -686 & -0.0267 & 100673 & -411 & 1 & 0.0325 & -1.31 & 100673 & -691 & -0.0269  \\
100599 & -429 & 1 & 0.0340 & -1.37 & 100599 & -685 & -0.0267 & 100674 & -409 & 1 & 0.0324 & -1.30 & 100674 & -692 & -0.0269  \\
100600 & -425 & 1 & 0.0337 & -1.35 & 100600 & -686 & -0.0267 & 100675 & -412 & 1 & 0.0327 & -1.31 & 100675 & -691 & -0.0269  \\
100601 & -424 & 1 & 0.0336 & -1.35 & 100601 & -685 & -0.0267 & 100676 & -413 & 1 & 0.0327 & -1.32 & 100676 & -690 & -0.0268  \\
100602 & -427 & 1 & 0.0338 & -1.36 & 100602 & -686 & -0.0267 & 100677 & -414 & 1 & 0.0328 & -1.32 & 100677 & -691 & -0.0269  \\
100603 & -426 & 1 & 0.0338 & -1.36 & 100603 & -685 & -0.0267 & 100678 & -415 & 1 & 0.0329 & -1.32 & 100678 & -692 & -0.0269  \\
100604 & -425 & 1 & 0.0337 & -1.35 & 100604 & -686 & -0.0267 & 100679 & -414 & 1 & 0.0328 & -1.32 & 100679 & -691 & -0.0269  \\
100605 & -424 & 1 & 0.0336 & -1.35 & 100605 & -685 & -0.0267 & 100680 & -409 & 1 & 0.0324 & -1.30 & 100680 & -690 & -0.0268  \\
100606 & -423 & 1 & 0.0335 & -1.35 & 100606 & -684 & -0.0267 & 100681 & -410 & 1 & 0.0325 & -1.31 & 100681 & -691 & -0.0269  \\
100607 & -424 & 1 & 0.0336 & -1.35 & 100607 & -685 & -0.0267 & 100682 & -409 & 1 & 0.0324 & -1.30 & 100682 & -690 & -0.0268  \\
100608 & -419 & 1 & 0.0332 & -1.34 & 100608 & -686 & -0.0267 & 100683 & -406 & 1 & 0.0322 & -1.29 & 100683 & -691 & -0.0269  \\
100609 & -420 & 1 & 0.0333 & -1.34 & 100609 & -687 & -0.0268 & 100684 & -407 & 1 & 0.0322 & -1.30 & 100684 & -690 & -0.0268  \\
100610 & -421 & 1 & 0.0334 & -1.34 & 100610 & -688 & -0.0268 & 100685 & -408 & 1 & 0.0323 & -1.30 & 100685 & -691 & -0.0269  \\
100611 & -419 & 1 & 0.0332 & -1.34 & 100611 & -689 & -0.0268 & 100686 & -407 & 1 & 0.0322 & -1.30 & 100686 & -690 & -0.0268  \\
100612 & -420 & 1 & 0.0333 & -1.34 & 100612 & -688 & -0.0268 & 100687 & -406 & 1 & 0.0322 & -1.29 & 100687 & -689 & -0.0268  \\
100613 & -421 & 1 & 0.0334 & -1.34 & 100613 & -689 & -0.0268 & 100688 & -406 & 1 & 0.0322 & -1.29 & 100688 & -688 & -0.0268  \\
100614 & -420 & 1 & 0.0333 & -1.34 & 100614 & -688 & -0.0268 & 100689 & -405 & 1 & 0.0321 & -1.29 & 100689 & -687 & -0.0267  \\
\end{array}
}
\end{equation*} 

\end{table}
\clearpage 

\newpage
\begin{table}[ht!] 

\centering
\tiny 
\begin{equation*} 
\boxed{
\begin{array}{ccccc|ccc||ccccc|ccc} 
x & L(x) & R_{\pm}(x) & 
    \frac{L(x)}{L_{\approx,1}(x)} & \frac{L(x)}{L_{\approx,2}(x)} & 
    x & L_{\ast}(x) & \frac{L_{\ast}(x)}{L_{\approx}^{\ast}(x)} & 
x & L(x) & R_{\pm}(x) & 
    \frac{L(x)}{L_{\approx,1}(x)} & \frac{L(x)}{L_{\approx,2}(x)} & 
    x & L_{\ast}(x) & \frac{L_{\ast}(x)}{L_{\approx}^{\ast}(x)} \\ \hline 
100690 & -406 & 1 & 0.0322 & -1.29 & 100690 & -688 & -0.0268 & 100765 & -390 & 1 & 0.0309 & -1.24 & 100765 & -683 & -0.0266  \\
100691 & -405 & 1 & 0.0321 & -1.29 & 100691 & -687 & -0.0267 & 100766 & -389 & 1 & 0.0308 & -1.24 & 100766 & -682 & -0.0265  \\
100692 & -409 & 1 & 0.0324 & -1.30 & 100692 & -688 & -0.0268 & 100767 & -388 & 1 & 0.0307 & -1.24 & 100767 & -681 & -0.0265  \\
100693 & -410 & 1 & 0.0325 & -1.31 & 100693 & -689 & -0.0268 & 100768 & -390 & 1 & 0.0309 & -1.24 & 100768 & -682 & -0.0265  \\
100694 & -407 & 1 & 0.0322 & -1.30 & 100694 & -688 & -0.0268 & 100769 & -391 & 1 & 0.0309 & -1.24 & 100769 & -683 & -0.0266  \\
100695 & -410 & 1 & 0.0325 & -1.31 & 100695 & -687 & -0.0267 & 100770 & -388 & 1 & 0.0307 & -1.24 & 100770 & -682 & -0.0265  \\
100696 & -411 & 1 & 0.0325 & -1.31 & 100696 & -688 & -0.0268 & 100771 & -387 & 1 & 0.0306 & -1.23 & 100771 & -681 & -0.0265  \\
100697 & -410 & 1 & 0.0325 & -1.31 & 100697 & -687 & -0.0267 & 100772 & -388 & 1 & 0.0307 & -1.24 & 100772 & -680 & -0.0264  \\
100698 & -409 & 1 & 0.0324 & -1.30 & 100698 & -686 & -0.0267 & 100773 & -390 & 1 & 0.0309 & -1.24 & 100773 & -679 & -0.0264  \\
100699 & -410 & 1 & 0.0325 & -1.31 & 100699 & -687 & -0.0267 & 100774 & -389 & 1 & 0.0308 & -1.24 & 100774 & -678 & -0.0264  \\
100700 & -409 & 1 & 0.0324 & -1.30 & 100700 & -686 & -0.0267 & 100775 & -386 & 1 & 0.0305 & -1.23 & 100775 & -679 & -0.0264  \\
100701 & -407 & 1 & 0.0322 & -1.30 & 100701 & -687 & -0.0267 & 100776 & -389 & 1 & 0.0308 & -1.24 & 100776 & -680 & -0.0264  \\
100702 & -408 & 1 & 0.0323 & -1.30 & 100702 & -688 & -0.0268 & 100777 & -388 & 1 & 0.0307 & -1.24 & 100777 & -679 & -0.0264  \\
100703 & -409 & 1 & 0.0324 & -1.30 & 100703 & -689 & -0.0268 & 100778 & -389 & 1 & 0.0308 & -1.24 & 100778 & -680 & -0.0264  \\
100704 & -412 & 1 & 0.0326 & -1.31 & 100704 & -690 & -0.0268 & 100779 & -390 & 1 & 0.0309 & -1.24 & 100779 & -681 & -0.0265  \\
100705 & -413 & 1 & 0.0327 & -1.32 & 100705 & -691 & -0.0269 & 100780 & -388 & 1 & 0.0307 & -1.24 & 100780 & -682 & -0.0265  \\
100706 & -414 & 1 & 0.0328 & -1.32 & 100706 & -692 & -0.0269 & 100781 & -387 & 1 & 0.0306 & -1.23 & 100781 & -681 & -0.0265  \\
100707 & -413 & 1 & 0.0327 & -1.32 & 100707 & -691 & -0.0269 & 100782 & -389 & 1 & 0.0308 & -1.24 & 100782 & -680 & -0.0264  \\
100708 & -412 & 1 & 0.0326 & -1.31 & 100708 & -692 & -0.0269 & 100783 & -388 & 1 & 0.0307 & -1.24 & 100783 & -679 & -0.0264  \\
100709 & -411 & 1 & 0.0325 & -1.31 & 100709 & -691 & -0.0269 & 100784 & -390 & 1 & 0.0309 & -1.24 & 100784 & -678 & -0.0264  \\
100710 & -408 & 1 & 0.0323 & -1.30 & 100710 & -690 & -0.0268 & 100785 & -391 & 1 & 0.0309 & -1.24 & 100785 & -679 & -0.0264  \\
100711 & -409 & 1 & 0.0324 & -1.30 & 100711 & -691 & -0.0269 & 100786 & -390 & 1 & 0.0309 & -1.24 & 100786 & -678 & -0.0264  \\
100712 & -408 & 1 & 0.0323 & -1.30 & 100712 & -690 & -0.0268 & 100787 & -391 & 1 & 0.0309 & -1.24 & 100787 & -679 & -0.0264  \\
100713 & -409 & 1 & 0.0324 & -1.30 & 100713 & -691 & -0.0269 & 100788 & -393 & 1 & 0.0311 & -1.25 & 100788 & -678 & -0.0264  \\
100714 & -410 & 1 & 0.0325 & -1.31 & 100714 & -692 & -0.0269 & 100789 & -392 & 1 & 0.0310 & -1.25 & 100789 & -677 & -0.0263  \\
100715 & -409 & 1 & 0.0324 & -1.30 & 100715 & -691 & -0.0269 & 100790 & -393 & 1 & 0.0311 & -1.25 & 100790 & -678 & -0.0264  \\
100716 & -410 & 1 & 0.0325 & -1.31 & 100716 & -692 & -0.0269 & 100791 & -391 & 1 & 0.0309 & -1.24 & 100791 & -677 & -0.0263  \\
100717 & -411 & 1 & 0.0325 & -1.31 & 100717 & -693 & -0.0270 & 100792 & -392 & 1 & 0.0310 & -1.25 & 100792 & -678 & -0.0264  \\
100718 & -410 & 1 & 0.0325 & -1.31 & 100718 & -692 & -0.0269 & 100793 & -402 & 1 & 0.0318 & -1.28 & 100793 & -679 & -0.0264  \\
100719 & -419 & 1 & 0.0332 & -1.33 & 100719 & -693 & -0.0270 & 100794 & -401 & 1 & 0.0317 & -1.28 & 100794 & -678 & -0.0264  \\
100720 & -416 & 1 & 0.0329 & -1.33 & 100720 & -694 & -0.0270 & 100795 & -402 & 1 & 0.0318 & -1.28 & 100795 & -679 & -0.0264  \\
100721 & -415 & 1 & 0.0329 & -1.32 & 100721 & -693 & -0.0270 & 100796 & -401 & 1 & 0.0317 & -1.28 & 100796 & -680 & -0.0264  \\
100722 & -416 & 1 & 0.0329 & -1.33 & 100722 & -694 & -0.0270 & 100797 & -400 & 1 & 0.0317 & -1.27 & 100797 & -679 & -0.0264  \\
100723 & -415 & 1 & 0.0329 & -1.32 & 100723 & -693 & -0.0270 & 100798 & -401 & 1 & 0.0317 & -1.28 & 100798 & -680 & -0.0264  \\
100724 & -418 & 1 & 0.0331 & -1.33 & 100724 & -694 & -0.0270 & 100799 & -402 & 1 & 0.0318 & -1.28 & 100799 & -681 & -0.0265  \\
100725 & -421 & 1 & 0.0333 & -1.34 & 100725 & -693 & -0.0270 & 100800 & -434 & 1 & 0.0343 & -1.38 & 100800 & -680 & -0.0264  \\
100726 & -420 & 1 & 0.0332 & -1.34 & 100726 & -692 & -0.0269 & 100801 & -435 & 1 & 0.0344 & -1.39 & 100801 & -681 & -0.0265  \\
100727 & -419 & 1 & 0.0332 & -1.33 & 100727 & -691 & -0.0269 & 100802 & -436 & 1 & 0.0345 & -1.39 & 100802 & -682 & -0.0265  \\
100728 & -413 & 1 & 0.0327 & -1.32 & 100728 & -692 & -0.0269 & 100803 & -435 & 1 & 0.0344 & -1.39 & 100803 & -681 & -0.0265  \\
100729 & -412 & 1 & 0.0326 & -1.31 & 100729 & -691 & -0.0269 & 100804 & -436 & 1 & 0.0345 & -1.39 & 100804 & -680 & -0.0264  \\
100730 & -411 & 1 & 0.0325 & -1.31 & 100730 & -690 & -0.0268 & 100805 & -435 & 1 & 0.0344 & -1.39 & 100805 & -679 & -0.0264  \\
100731 & -410 & 1 & 0.0325 & -1.31 & 100731 & -689 & -0.0268 & 100806 & -440 & 1 & 0.0348 & -1.40 & 100806 & -678 & -0.0264  \\
100732 & -411 & 1 & 0.0325 & -1.31 & 100732 & -688 & -0.0268 & 100807 & -439 & 1 & 0.0347 & -1.40 & 100807 & -677 & -0.0263  \\
100733 & -412 & 1 & 0.0326 & -1.31 & 100733 & -689 & -0.0268 & 100808 & -438 & 1 & 0.0347 & -1.39 & 100808 & -676 & -0.0263  \\
100734 & -411 & 1 & 0.0325 & -1.31 & 100734 & -688 & -0.0268 & 100809 & -436 & 1 & 0.0345 & -1.39 & 100809 & -677 & -0.0263  \\
100735 & -410 & 1 & 0.0325 & -1.31 & 100735 & -687 & -0.0267 & 100810 & -435 & 1 & 0.0344 & -1.38 & 100810 & -676 & -0.0263  \\
100736 & -406 & 1 & 0.0322 & -1.29 & 100736 & -686 & -0.0267 & 100811 & -436 & 1 & 0.0345 & -1.39 & 100811 & -677 & -0.0263  \\
100737 & -404 & 1 & 0.0320 & -1.29 & 100737 & -685 & -0.0267 & 100812 & -438 & 1 & 0.0347 & -1.39 & 100812 & -676 & -0.0263  \\
100738 & -403 & 1 & 0.0319 & -1.28 & 100738 & -684 & -0.0266 & 100813 & -437 & 1 & 0.0346 & -1.39 & 100813 & -675 & -0.0262  \\
100739 & -402 & 1 & 0.0318 & -1.28 & 100739 & -683 & -0.0266 & 100814 & -436 & 1 & 0.0345 & -1.39 & 100814 & -674 & -0.0262  \\
100740 & -397 & 1 & 0.0314 & -1.26 & 100740 & -684 & -0.0266 & 100815 & -437 & 1 & 0.0346 & -1.39 & 100815 & -675 & -0.0262  \\
100741 & -398 & 1 & 0.0315 & -1.27 & 100741 & -685 & -0.0267 & 100816 & -439 & 1 & 0.0347 & -1.40 & 100816 & -674 & -0.0262  \\
100742 & -399 & 1 & 0.0316 & -1.27 & 100742 & -686 & -0.0267 & 100817 & -438 & 1 & 0.0347 & -1.39 & 100817 & -673 & -0.0262  \\
100743 & -398 & 1 & 0.0315 & -1.27 & 100743 & -685 & -0.0267 & 100818 & -440 & 1 & 0.0348 & -1.40 & 100818 & -674 & -0.0262  \\
100744 & -395 & 1 & 0.0313 & -1.26 & 100744 & -686 & -0.0267 & 100819 & -439 & 1 & 0.0347 & -1.40 & 100819 & -673 & -0.0262  \\
100745 & -394 & 1 & 0.0312 & -1.25 & 100745 & -685 & -0.0267 & 100820 & -447 & 1 & 0.0354 & -1.42 & 100820 & -674 & -0.0262  \\
100746 & -396 & 1 & 0.0314 & -1.26 & 100746 & -684 & -0.0266 & 100821 & -448 & 1 & 0.0354 & -1.43 & 100821 & -675 & -0.0262  \\
100747 & -397 & 1 & 0.0314 & -1.26 & 100747 & -685 & -0.0267 & 100822 & -447 & 1 & 0.0354 & -1.42 & 100822 & -674 & -0.0262  \\
100748 & -396 & 1 & 0.0314 & -1.26 & 100748 & -686 & -0.0267 & 100823 & -448 & 1 & 0.0354 & -1.43 & 100823 & -675 & -0.0262  \\
100749 & -395 & 1 & 0.0313 & -1.26 & 100749 & -685 & -0.0267 & 100824 & -450 & 1 & 0.0356 & -1.43 & 100824 & -676 & -0.0263  \\
100750 & -392 & 1 & 0.0310 & -1.25 & 100750 & -684 & -0.0266 & 100825 & -447 & 1 & 0.0354 & -1.42 & 100825 & -677 & -0.0263  \\
100751 & -393 & 1 & 0.0311 & -1.25 & 100751 & -685 & -0.0267 & 100826 & -448 & 1 & 0.0354 & -1.43 & 100826 & -678 & -0.0264  \\
100752 & -390 & 1 & 0.0309 & -1.24 & 100752 & -686 & -0.0267 & 100827 & -446 & 1 & 0.0353 & -1.42 & 100827 & -679 & -0.0264  \\
100753 & -389 & 1 & 0.0308 & -1.24 & 100753 & -685 & -0.0267 & 100828 & -450 & 1 & 0.0356 & -1.43 & 100828 & -678 & -0.0263  \\
100754 & -388 & 1 & 0.0307 & -1.24 & 100754 & -684 & -0.0266 & 100829 & -451 & 1 & 0.0357 & -1.44 & 100829 & -679 & -0.0264  \\
100755 & -386 & 1 & 0.0305 & -1.23 & 100755 & -685 & -0.0266 & 100830 & -448 & 1 & 0.0354 & -1.43 & 100830 & -678 & -0.0263  \\
100756 & -387 & 1 & 0.0306 & -1.23 & 100756 & -684 & -0.0266 & 100831 & -447 & 1 & 0.0354 & -1.42 & 100831 & -677 & -0.0263  \\
100757 & -386 & 1 & 0.0305 & -1.23 & 100757 & -683 & -0.0266 & 100832 & -449 & 1 & 0.0355 & -1.43 & 100832 & -678 & -0.0263  \\
100758 & -384 & 1 & 0.0304 & -1.22 & 100758 & -682 & -0.0265 & 100833 & -448 & 1 & 0.0354 & -1.43 & 100833 & -677 & -0.0263  \\
100759 & -383 & 1 & 0.0303 & -1.22 & 100759 & -681 & -0.0265 & 100834 & -447 & 1 & 0.0354 & -1.42 & 100834 & -676 & -0.0263  \\
100760 & -381 & 1 & 0.0302 & -1.21 & 100760 & -680 & -0.0264 & 100835 & -446 & 1 & 0.0353 & -1.42 & 100835 & -675 & -0.0262  \\
100761 & -380 & 1 & 0.0301 & -1.21 & 100761 & -679 & -0.0264 & 100836 & -450 & 1 & 0.0356 & -1.43 & 100836 & -676 & -0.0263  \\
100762 & -381 & 1 & 0.0302 & -1.21 & 100762 & -680 & -0.0264 & 100837 & -451 & 1 & 0.0357 & -1.44 & 100837 & -677 & -0.0263  \\
100763 & -382 & 1 & 0.0302 & -1.22 & 100763 & -681 & -0.0265 & 100838 & -452 & 1 & 0.0358 & -1.44 & 100838 & -678 & -0.0263  \\
100764 & -389 & 1 & 0.0308 & -1.24 & 100764 & -682 & -0.0265 & 100839 & -451 & 1 & 0.0357 & -1.44 & 100839 & -677 & -0.0263  \\
\end{array}
}
\end{equation*} 

\end{table}
\clearpage 

\newpage
\begin{table}[ht!] 

\centering
\tiny 
\begin{equation*} 
\boxed{
\begin{array}{ccccc|ccc||ccccc|ccc} 
x & L(x) & R_{\pm}(x) & 
    \frac{L(x)}{L_{\approx,1}(x)} & \frac{L(x)}{L_{\approx,2}(x)} & 
    x & L_{\ast}(x) & \frac{L_{\ast}(x)}{L_{\approx}^{\ast}(x)} & 
x & L(x) & R_{\pm}(x) & 
    \frac{L(x)}{L_{\approx,1}(x)} & \frac{L(x)}{L_{\approx,2}(x)} & 
    x & L_{\ast}(x) & \frac{L_{\ast}(x)}{L_{\approx}^{\ast}(x)} \\ \hline 
100840 & -453 & 1 & 0.0358 & -1.44 & 100840 & -678 & -0.0263 & 100915 & -463 & 1 & 0.0366 & -1.47 & 100915 & -679 & -0.0264  \\
100841 & -452 & 1 & 0.0358 & -1.44 & 100841 & -677 & -0.0263 & 100916 & -464 & 1 & 0.0367 & -1.48 & 100916 & -678 & -0.0263  \\
100842 & -459 & 1 & 0.0363 & -1.46 & 100842 & -678 & -0.0263 & 100917 & -466 & 1 & 0.0368 & -1.48 & 100917 & -677 & -0.0263  \\
100843 & -458 & 1 & 0.0362 & -1.46 & 100843 & -677 & -0.0263 & 100918 & -465 & 1 & 0.0368 & -1.48 & 100918 & -676 & -0.0262  \\
100844 & -457 & 1 & 0.0361 & -1.45 & 100844 & -678 & -0.0263 & 100919 & -466 & 1 & 0.0368 & -1.48 & 100919 & -677 & -0.0263  \\
100845 & -460 & 1 & 0.0364 & -1.46 & 100845 & -679 & -0.0264 & 100920 & -474 & 1 & 0.0375 & -1.51 & 100920 & -676 & -0.0262  \\
100846 & -459 & 1 & 0.0363 & -1.46 & 100846 & -678 & -0.0263 & 100921 & -473 & 1 & 0.0374 & -1.51 & 100921 & -675 & -0.0262  \\
100847 & -460 & 1 & 0.0364 & -1.46 & 100847 & -679 & -0.0264 & 100922 & -472 & 1 & 0.0373 & -1.50 & 100922 & -674 & -0.0262  \\
100848 & -462 & 1 & 0.0365 & -1.47 & 100848 & -678 & -0.0263 & 100923 & -471 & 1 & 0.0372 & -1.50 & 100923 & -673 & -0.0261  \\
100849 & -461 & 1 & 0.0365 & -1.47 & 100849 & -677 & -0.0263 & 100924 & -470 & 1 & 0.0372 & -1.50 & 100924 & -674 & -0.0262  \\
100850 & -458 & 1 & 0.0362 & -1.46 & 100850 & -678 & -0.0263 & 100925 & -467 & 1 & 0.0369 & -1.49 & 100925 & -675 & -0.0262  \\
100851 & -457 & 1 & 0.0362 & -1.45 & 100851 & -677 & -0.0263 & 100926 & -471 & 1 & 0.0372 & -1.50 & 100926 & -674 & -0.0262  \\
100852 & -456 & 1 & 0.0361 & -1.45 & 100852 & -678 & -0.0263 & 100927 & -472 & 1 & 0.0373 & -1.50 & 100927 & -675 & -0.0262  \\
100853 & -457 & 1 & 0.0362 & -1.45 & 100853 & -679 & -0.0264 & 100928 & -468 & 1 & 0.0370 & -1.49 & 100928 & -676 & -0.0262  \\
100854 & -459 & 1 & 0.0363 & -1.46 & 100854 & -678 & -0.0263 & 100929 & -469 & 1 & 0.0371 & -1.49 & 100929 & -677 & -0.0263  \\
100855 & -460 & 1 & 0.0364 & -1.46 & 100855 & -679 & -0.0264 & 100930 & -470 & 1 & 0.0371 & -1.50 & 100930 & -678 & -0.0263  \\
100856 & -459 & 1 & 0.0363 & -1.46 & 100856 & -680 & -0.0264 & 100931 & -471 & 1 & 0.0372 & -1.50 & 100931 & -679 & -0.0264  \\
100857 & -458 & 1 & 0.0362 & -1.46 & 100857 & -679 & -0.0264 & 100932 & -471 & 1 & 0.0372 & -1.50 & 100932 & -678 & -0.0263  \\
100858 & -459 & 1 & 0.0363 & -1.46 & 100858 & -680 & -0.0264 & 100933 & -470 & 1 & 0.0371 & -1.50 & 100933 & -677 & -0.0263  \\
100859 & -460 & 1 & 0.0364 & -1.46 & 100859 & -681 & -0.0265 & 100934 & -471 & 1 & 0.0372 & -1.50 & 100934 & -678 & -0.0263  \\
100860 & -448 & 1 & 0.0354 & -1.43 & 100860 & -680 & -0.0264 & 100935 & -469 & 1 & 0.0371 & -1.49 & 100935 & -679 & -0.0264  \\
100861 & -450 & 1 & 0.0356 & -1.43 & 100861 & -679 & -0.0264 & 100936 & -468 & 1 & 0.0370 & -1.49 & 100936 & -678 & -0.0263  \\
100862 & -449 & 1 & 0.0355 & -1.43 & 100862 & -678 & -0.0263 & 100937 & -469 & 1 & 0.0371 & -1.49 & 100937 & -679 & -0.0264  \\
100863 & -447 & 1 & 0.0354 & -1.42 & 100863 & -679 & -0.0264 & 100938 & -470 & 1 & 0.0371 & -1.50 & 100938 & -680 & -0.0264  \\
100864 & -443 & 1 & 0.0350 & -1.41 & 100864 & -678 & -0.0263 & 100939 & -469 & 1 & 0.0371 & -1.49 & 100939 & -679 & -0.0264  \\
100865 & -442 & 1 & 0.0350 & -1.41 & 100865 & -677 & -0.0263 & 100940 & -463 & 1 & 0.0366 & -1.47 & 100940 & -678 & -0.0263  \\
100866 & -443 & 1 & 0.0350 & -1.41 & 100866 & -678 & -0.0263 & 100941 & -462 & 1 & 0.0365 & -1.47 & 100941 & -677 & -0.0263  \\
100867 & -442 & 1 & 0.0350 & -1.41 & 100867 & -677 & -0.0263 & 100942 & -463 & 1 & 0.0366 & -1.47 & 100942 & -678 & -0.0263  \\
100868 & -441 & 1 & 0.0349 & -1.40 & 100868 & -678 & -0.0263 & 100943 & -464 & 1 & 0.0367 & -1.48 & 100943 & -679 & -0.0264  \\
100869 & -440 & 1 & 0.0348 & -1.40 & 100869 & -677 & -0.0263 & 100944 & -473 & 1 & 0.0374 & -1.51 & 100944 & -680 & -0.0264  \\
100870 & -441 & 1 & 0.0349 & -1.40 & 100870 & -678 & -0.0263 & 100945 & -474 & 1 & 0.0375 & -1.51 & 100945 & -681 & -0.0264  \\
100871 & -440 & 1 & 0.0348 & -1.40 & 100871 & -677 & -0.0263 & 100946 & -475 & 1 & 0.0375 & -1.51 & 100946 & -682 & -0.0265  \\
100872 & -446 & 1 & 0.0353 & -1.42 & 100872 & -678 & -0.0263 & 100947 & -476 & 1 & 0.0376 & -1.51 & 100947 & -683 & -0.0265  \\
100873 & -445 & 1 & 0.0352 & -1.42 & 100873 & -677 & -0.0263 & 100948 & -477 & 1 & 0.0377 & -1.52 & 100948 & -682 & -0.0265  \\
100874 & -446 & 1 & 0.0353 & -1.42 & 100874 & -678 & -0.0263 & 100949 & -482 & 1 & 0.0381 & -1.53 & 100949 & -681 & -0.0264  \\
100875 & -449 & 1 & 0.0355 & -1.43 & 100875 & -679 & -0.0264 & 100950 & -487 & 1 & 0.0385 & -1.55 & 100950 & -680 & -0.0264  \\
100876 & -450 & 1 & 0.0356 & -1.43 & 100876 & -678 & -0.0263 & 100951 & -486 & 1 & 0.0384 & -1.55 & 100951 & -679 & -0.0264  \\
100877 & -449 & 1 & 0.0355 & -1.43 & 100877 & -677 & -0.0263 & 100952 & -485 & 1 & 0.0383 & -1.54 & 100952 & -678 & -0.0263  \\
100878 & -450 & 1 & 0.0356 & -1.43 & 100878 & -678 & -0.0263 & 100953 & -483 & 1 & 0.0382 & -1.54 & 100953 & -677 & -0.0263  \\
100879 & -449 & 1 & 0.0355 & -1.43 & 100879 & -677 & -0.0263 & 100954 & -484 & 1 & 0.0383 & -1.54 & 100954 & -678 & -0.0263  \\
100880 & -452 & 1 & 0.0358 & -1.44 & 100880 & -676 & -0.0263 & 100955 & -485 & 1 & 0.0383 & -1.54 & 100955 & -679 & -0.0264  \\
100881 & -450 & 1 & 0.0356 & -1.43 & 100881 & -677 & -0.0263 & 100956 & -487 & 1 & 0.0385 & -1.55 & 100956 & -678 & -0.0263  \\
100882 & -449 & 1 & 0.0355 & -1.43 & 100882 & -676 & -0.0263 & 100957 & -488 & 1 & 0.0386 & -1.55 & 100957 & -679 & -0.0264  \\
100883 & -448 & 1 & 0.0354 & -1.43 & 100883 & -675 & -0.0262 & 100958 & -487 & 1 & 0.0385 & -1.55 & 100958 & -678 & -0.0263  \\
100884 & -451 & 1 & 0.0357 & -1.44 & 100884 & -674 & -0.0262 & 100959 & -488 & 1 & 0.0386 & -1.55 & 100959 & -679 & -0.0264  \\
100885 & -450 & 1 & 0.0356 & -1.43 & 100885 & -673 & -0.0261 & 100960 & -491 & 1 & 0.0388 & -1.56 & 100960 & -680 & -0.0264  \\
100886 & -451 & 1 & 0.0357 & -1.44 & 100886 & -674 & -0.0262 & 100961 & -490 & 1 & 0.0387 & -1.56 & 100961 & -679 & -0.0264  \\
100887 & -450 & 1 & 0.0356 & -1.43 & 100887 & -673 & -0.0261 & 100962 & -492 & 1 & 0.0389 & -1.56 & 100962 & -678 & -0.0263  \\
100888 & -449 & 1 & 0.0355 & -1.43 & 100888 & -672 & -0.0261 & 100963 & -491 & 1 & 0.0388 & -1.56 & 100963 & -677 & -0.0263  \\
100889 & -448 & 1 & 0.0354 & -1.43 & 100889 & -671 & -0.0261 & 100964 & -490 & 1 & 0.0387 & -1.56 & 100964 & -678 & -0.0263  \\
100890 & -447 & 1 & 0.0354 & -1.42 & 100890 & -672 & -0.0261 & 100965 & -489 & 1 & 0.0386 & -1.56 & 100965 & -677 & -0.0263  \\
100891 & -444 & 1 & 0.0351 & -1.41 & 100891 & -673 & -0.0261 & 100966 & -490 & 1 & 0.0387 & -1.56 & 100966 & -678 & -0.0263  \\
100892 & -443 & 1 & 0.0350 & -1.41 & 100892 & -674 & -0.0262 & 100967 & -489 & 1 & 0.0386 & -1.56 & 100967 & -677 & -0.0263  \\
100893 & -439 & 1 & 0.0347 & -1.40 & 100893 & -675 & -0.0262 & 100968 & -488 & 1 & 0.0386 & -1.55 & 100968 & -676 & -0.0262  \\
100894 & -440 & 1 & 0.0348 & -1.40 & 100894 & -676 & -0.0263 & 100969 & -489 & 1 & 0.0386 & -1.56 & 100969 & -677 & -0.0263  \\
100895 & -441 & 1 & 0.0349 & -1.40 & 100895 & -677 & -0.0263 & 100970 & -488 & 1 & 0.0386 & -1.55 & 100970 & -676 & -0.0262  \\
100896 & -444 & 1 & 0.0351 & -1.41 & 100896 & -678 & -0.0263 & 100971 & -486 & 1 & 0.0384 & -1.55 & 100971 & -677 & -0.0263  \\
100897 & -443 & 1 & 0.0350 & -1.41 & 100897 & -677 & -0.0263 & 100972 & -487 & 1 & 0.0385 & -1.55 & 100972 & -676 & -0.0262  \\
100898 & -444 & 1 & 0.0351 & -1.41 & 100898 & -678 & -0.0263 & 100973 & -486 & 1 & 0.0384 & -1.55 & 100973 & -675 & -0.0262  \\
100899 & -446 & 1 & 0.0353 & -1.42 & 100899 & -679 & -0.0264 & 100974 & -487 & 1 & 0.0385 & -1.55 & 100974 & -676 & -0.0262  \\
100900 & -450 & 1 & 0.0356 & -1.43 & 100900 & -680 & -0.0264 & 100975 & -484 & 1 & 0.0382 & -1.54 & 100975 & -677 & -0.0263  \\
100901 & -451 & 1 & 0.0357 & -1.44 & 100901 & -681 & -0.0265 & 100976 & -486 & 1 & 0.0384 & -1.55 & 100976 & -676 & -0.0262  \\
100902 & -450 & 1 & 0.0356 & -1.43 & 100902 & -680 & -0.0264 & 100977 & -487 & 1 & 0.0385 & -1.55 & 100977 & -677 & -0.0263  \\
100903 & -449 & 1 & 0.0355 & -1.43 & 100903 & -679 & -0.0264 & 100978 & -488 & 1 & 0.0386 & -1.55 & 100978 & -678 & -0.0263  \\
100904 & -448 & 1 & 0.0354 & -1.43 & 100904 & -678 & -0.0263 & 100979 & -487 & 1 & 0.0385 & -1.55 & 100979 & -677 & -0.0263  \\
100905 & -454 & 1 & 0.0359 & -1.44 & 100905 & -677 & -0.0263 & 100980 & -495 & 1 & 0.0391 & -1.57 & 100980 & -678 & -0.0263  \\
100906 & -455 & 1 & 0.0360 & -1.45 & 100906 & -678 & -0.0263 & 100981 & -496 & 1 & 0.0392 & -1.58 & 100981 & -679 & -0.0263  \\
100907 & -456 & 1 & 0.0361 & -1.45 & 100907 & -679 & -0.0264 & 100982 & -497 & 1 & 0.0393 & -1.58 & 100982 & -680 & -0.0264  \\
100908 & -460 & 1 & 0.0364 & -1.46 & 100908 & -680 & -0.0264 & 100983 & -498 & 1 & 0.0394 & -1.58 & 100983 & -681 & -0.0264  \\
100909 & -461 & 1 & 0.0365 & -1.47 & 100909 & -681 & -0.0264 & 100984 & -499 & 1 & 0.0394 & -1.59 & 100984 & -682 & -0.0265  \\
100910 & -462 & 1 & 0.0365 & -1.47 & 100910 & -682 & -0.0265 & 100985 & -500 & 1 & 0.0395 & -1.59 & 100985 & -683 & -0.0265  \\
100911 & -461 & 1 & 0.0365 & -1.47 & 100911 & -681 & -0.0264 & 100986 & -501 & 1 & 0.0396 & -1.59 & 100986 & -684 & -0.0265  \\
100912 & -461 & 1 & 0.0365 & -1.47 & 100912 & -680 & -0.0264 & 100987 & -502 & 1 & 0.0397 & -1.60 & 100987 & -685 & -0.0266  \\
100913 & -462 & 1 & 0.0365 & -1.47 & 100913 & -681 & -0.0264 & 100988 & -503 & 1 & 0.0397 & -1.60 & 100988 & -684 & -0.0265  \\
100914 & -464 & 1 & 0.0367 & -1.48 & 100914 & -680 & -0.0264 & 100989 & -510 & 1 & 0.0403 & -1.62 & 100989 & -685 & -0.0266  \\
\end{array}
}
\end{equation*} 

\end{table}
\clearpage 

\end{document}
