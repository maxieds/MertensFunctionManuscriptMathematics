\documentclass[11pt,reqno,a4letter]{article} 

\usepackage{amsthm,amsfonts,amscd,amsmath}
\usepackage[hidelinks]{hyperref} 
\usepackage{url}
\usepackage[usenames,dvipsnames]{xcolor}
\hypersetup{
    colorlinks,
    linkcolor={red!70!white},
    citecolor={blue!70!white},
    urlcolor={blue!80!white}
}

\usepackage[normalem]{ulem}
\usepackage{graphicx} 
\usepackage{datetime} 
\usepackage{cancel}
\usepackage{subcaption}
\captionsetup{format=hang,labelfont={bf},textfont={small,it}} 
\numberwithin{figure}{section}
\numberwithin{table}{section}

\usepackage{stmaryrd,tikzsymbols} 
\usepackage{framed} 
\usepackage{ulem}
\usepackage[T1]{fontenc}
\usepackage{pbsi}


\usepackage{enumitem}
\setlist[itemize]{leftmargin=0.65in}

\usepackage{rotating,adjustbox}

\usepackage{diagbox}
\newcommand{\trianglenk}[2]{$\diagbox{#1}{#2}$}
\newcommand{\trianglenkII}[2]{\diagbox{#1}{#2}}

\let\citep\cite

\newcommand{\undersetbrace}[2]{\underset{\displaystyle{#1}}{\underbrace{#2}}}

\newcommand{\gkpSI}[2]{\ensuremath{\genfrac{\lbrack}{\rbrack}{0pt}{}{#1}{#2}}} 
\newcommand{\gkpSII}[2]{\ensuremath{\genfrac{\lbrace}{\rbrace}{0pt}{}{#1}{#2}}}
\newcommand{\cf}{\textit{cf.\ }} 
\newcommand{\Iverson}[1]{\ensuremath{\left[#1\right]_{\delta}}} 
\newcommand{\floor}[1]{\left\lfloor #1 \right\rfloor} 
\newcommand{\ceiling}[1]{\left\lceil #1 \right\rceil} 
\newcommand{\e}[1]{e\left(#1\right)} 
\newcommand{\seqnum}[1]{\href{http://oeis.org/#1}{\color{ProcessBlue}{\underline{#1}}}}

\usepackage{upgreek,dsfont,amssymb}
\renewcommand{\chi}{\upchi}
\newcommand{\ChiFunc}[1]{\ensuremath{\chi_{\{#1\}}}}
\newcommand{\OneFunc}[1]{\ensuremath{\mathds{1}_{#1}}}

\usepackage{ifthen}
\newcommand{\Hn}[2]{
     \ifthenelse{\equal{#2}{1}}{H_{#1}}{H_{#1}^{\left(#2\right)}}
}

\newcommand{\Floor}[2]{\ensuremath{\left\lfloor \frac{#1}{#2} \right\rfloor}}
\newcommand{\Ceiling}[2]{\ensuremath{\left\lceil \frac{#1}{#2} \right\rceil}}

\DeclareMathOperator{\DGF}{DGF} 
\DeclareMathOperator{\ds}{ds} 
\DeclareMathOperator{\Id}{Id}
\DeclareMathOperator{\fg}{fg}
\DeclareMathOperator{\Div}{div}
\DeclareMathOperator{\rpp}{rpp}
\DeclareMathOperator{\logll}{\ell\ell}

\usepackage{fancyhdr}
\pagestyle{empty}
\pagestyle{fancy}
\fancyhead[RO,RE]{Maxie Dion Schmidt -- \today} 
\fancyhead[LO,LE]{}

%\usepackage{pifont}
%\newcommand{\checkmark}[0]{\ding{51}} 

\title{
       \LARGE{
       Lower bounds on the Mertens function $M(x)$ for $x \gg 2.3315 \times 10^{1656520}$ 
       } 
       %\\ 
       %\large{\it New unique lower bounds on $M(x) / \sqrt{x}$ along an asymptotically 
       %   huge infinite subsequence of reals} 
}
\author{{\large Maxie Dion Schmidt} \\ 
        {\normalsize \href{mailto:maxieds@gmail.com}{maxieds@gmail.com}} \\[0.1cm] 
        {\small Georgia Institute of Technology} 
} 

\date{\small\underline{Last Revised:} \today\ \ -- \ \ Compiled with \LaTeX2e} 

\theoremstyle{plain} 
\newtheorem{theorem}{Theorem}
\newtheorem{conjecture}[theorem]{Conjecture}
\newtheorem{claim}[theorem]{Claim}
\newtheorem{prop}[theorem]{Proposition}
\newtheorem{lemma}[theorem]{Lemma}
\newtheorem{cor}[theorem]{Corollary}
\numberwithin{theorem}{section}

\theoremstyle{definition} 
\newtheorem{example}[theorem]{Example}
\newtheorem{remark}[theorem]{Remark}
\newtheorem{definition}[theorem]{Definition}
\newtheorem{notation}[theorem]{Notation}
\newtheorem{question}[theorem]{Question}
\newtheorem{discussion}[theorem]{Discussion}
\newtheorem{facts}[theorem]{Facts}
\newtheorem{summary}[theorem]{Summary}
\newtheorem{heuristic}[theorem]{Heuristic}

\renewcommand{\arraystretch}{1.25} 

\setlength{\textheight}{9in}
\setlength{\topmargin}{-.1in}
\setlength{\textwidth}{7in} 
\setlength{\evensidemargin}{-0.25in} 
\setlength{\oddsidemargin}{-0.25in} 

\usepackage[top=0.65in, bottom=18mm, left=15mm, right=15mm]{geometry}

\setlength{\parindent}{0in}
\setlength{\parskip}{2cm} 

\input{glossaries-bibtex/PreambleGlossaries-Mertens}

\allowdisplaybreaks 

\begin{document} 

\maketitle

\begin{abstract} 
The Mertens function, $M(x) = \sum_{n \leq x} \mu(n)$, is classically 
defined to be the summatory function of the M\"obius function 
$\mu(n)$. In some sense, the M\'obius function can be viewed as a 
signed indicator function of the squarefree integers which have 
asymptotic density of $6 / \pi^2 \approx 0.607927$ and a corresponding 
well-known asymptotic average order formula. The signed terms in the 
sums in the definition of the Mertens function introduce complications 
in the form of semi-randomness and cancellation inherent to the 
distribution of the M\"obius function over the natural numbers. The 
Mertens conjecture which states that $|M(x)| < C \cdot \sqrt{x}$ for all 
$x \geq 1$ has a well-known disproof due to Odlyzko et. al. 
It is widely believed that $M(x) / \sqrt{x}$ is an unbounded function 
which changes sign infinitely often and exhibits a negative bias 
over all natural numbers $x \geq 1$. 

We focus on obtaining new lower bounds 
for $M(x)$ by methods that generalize to handle other related cases of 
special number theoretic summatory functions. 
The key to our proofs calls upon a known result from the standardized 
summatory function enumeration by Dirichlet generating functions (DGFs) 
found in Chapter 7 of Montogomery and Vaughan. There is also a distinct flavor of combinatorial analysis 
peppered in with the standard methods from analytic number theory which distinguishes 
our methods. 
%We make surprising claims about a classical conjecture on the boundedness of 
%$|M(x)| / \sqrt{x}$ along infinite subsequences of reals (\emph{TBA}). 

\bigskip %\hrule\bigskip
\noindent
\textbf{Keywords and Phrases:} {\it M\"obius function sums; Mertens function; summatory function; 
                                    arithmetic functions; 
                                    Dirichlet inverse; Liouville lambda function; prime omega functions; 
                                    prime counting functions; Dirichlet series and DGFs; 
                                    asymptotic lower bounds; Mertens conjecture; 
                                    asymptotic methods from the Montgomery and Vaughan textbook. } \\ 
% 11-XX			Number theory
%    11A25  	Arithmetic functions; related numbers; inversion formulas
%    11Y70  	Values of arithmetic functions; tables
%    11-04  	Software, source code, etc. for problems pertaining to number theory
% 11Nxx		Multiplicative number theory
%    11N05  	Distribution of primes
%    11N37  	Asymptotic results on arithmetic functions
%    11N56  	Rate of growth of arithmetic functions
%    11N60  	Distribution functions associated with additive and positive multiplicative functions
%    11N64  	Other results on the distribution of values or the characterization of arithmetic functions
\textbf{Primary Math Subject Classifications (2010):} {\it 11N37; 11A25; 11N60; 11N64; and 11-04. } 
\end{abstract} 

\bigskip\hrule\bigskip
\noindent
Note that this draft has been checked and semi-arguably verified as (mostly, e.g., almost surely) 
error free by the following distinguished full professors of mathematics 
(in addition to the author): 
\begin{itemize} 
     \item[\checkmark] TODO -- \textit{Affiliation} 
     \item[\checkmark] TODO -- \textit{Affiliation} 
\end{itemize} 

\newpage
\section{Reference on common abbreviations, special notation and other conventions} 
\label{Appendix_Glossary_NotationConvs}

     \vskip -0.35in
     \printglossary[type={symbols},title={},style={glossstyleSymbol},nogroupskip=true]

\newpage
\section{Crib notes: Core components to the proof} 
We offer a brief overview of the critical components to our proof outlined in the introduction, 
and then piece-by-piece in the next sections of the article:
\begin{itemize} 

\item[(1)] We prove an apparently yet undiscovered matrix inversion formula relating the summatory 
           functions of an arithmetic function $f$ and its Dirichlet inverse $f^{-1}$ (for $f(1) \neq 0$). 
\item[(2)] This crucial step provides us with an exact formula for $M(x)$ in terms of $\pi(x)$ and the 
           Dirichlet inverse of the shifted additive function $\omega(n)+1$. 
           \begin{itemize} 
           \item[(i)] The average order, $\omega(n) \sim \log\log n$, imparts an iterated logarithmic structure 
           to our expansions, which many have conjectured we should see in limiting bounds on $M(x)$, 
           but which are practically elusive in most non-conjectural known formulas I have seen. 
           \item[(ii)] The additivity of $\omega(n)$ dictates that the sign of $g^{-1}(n) = (\omega+1)^{-1}(n)$ 
           is $\operatorname{sgn}(g^{-1}(n)) = \lambda(n)$. The corresponding weighted summatory functions of 
           $\lambda(n)$ have more established predictable properties, such as known sign biases and upper bounds. 
           These summatory functions are generally speaking more regular and easier to work with that $M(x)$ 
           and its summand of the M\"obius function. 
           \end{itemize}
\item[(3)] We tighten a result from \cite[\S 7]{MV} providing summatory functions that indicate the parity of 
           $\lambda(n)$ using elementary arguments and combinatorially motivated expansions of Dirichlet series. 
           Our motivations are different than in the reference for exploiting its unique properties. 
           Namely, we are not after a CLT-like statement for the functions $\Omega(n)$ and $\omega(n)$. 
           Rather, we seek to sum $\sum_{n \leq x} \lambda(n) f(n)$ for general non-negative arithmetic 
           functions $f$. 
\item[(4)] We then turn to the asymptotics if the quasi-periodic $g^{-1}(n)$, estimating their limiting asymptotics 
           for large $n$. We eventually use these estimates to prove a substantially unique new lower bound formula 
           for the summatory function $G^{-1}(x) := \sum_{n \leq x} g^{-1}(n)$. 
\item[(5)] When we return to (2) with our new lower bounds, and bootstrap, we recover ``magic'' in the form of 
           showing the unboundedness of $\frac{|M(x)|}{\sqrt{x}}$ along a very large increasing infinite subsequence 
           of positive natural numbers. 
\item[(6)] We remark that while this technique and approach to the classical problem at hand is certainly new, 
           it is not just novel, and its discovery will invariably lead to similar applications given careful 
           study of limsup-type bounds on the summatory functions of other special signed arithmetic function 
           sequences. 
           
           Note that in these cases, if $f$ is multiplicative and $f(n) > 0$ for all $n \geq 1$, then 
           $\operatorname{sgn}(f^{-1}(n)) = (-1)^{\omega(n)}$. This signedness tends to complicate, but still 
           closely parallel our argument involving the parity of $\lambda(n) = (-1)^{\Omega(n)}$ for the 
           Mertens function case. 
           
\end{itemize} 

\newpage
\section{Introduction} 

\subsection{The Mertens function -- definition, properties, known results and conjectures} 
\label{subSection_MertensMxClassical_Intro} 

Suppose that $n \geq 1$ is a natural number with factorization into 
distinct primes given by 
$n = p_1^{\alpha_1} p_2{\alpha_2} \cdots p_k^{\alpha_k}$. 
We define the \emph{M\"oebius function} to be the signed indicator function 
of the squarefree integers: 
\[
\mu(n) = \begin{cases} 
     1, & \text{if $n = 1$; } \\ 
     (-1)^k, & \text{if $\alpha_i = 1$, $\forall 1 \leq i \leq k$; } \\ 
     0, & \text{otherwise.} 
     \end{cases} 
\]
There are many known variants and special properties of the M\"oebius function 
and its generalizations \cite[\cf \S 2]{HANDBOOKNT-2004}, however, for our 
purposes we seek to explore the properties and asymptotics of weighted 
summatory functions over $\mu(n)$. 
The Mertens summatory function, or \emph{Mertens function}, is defined as 
\cite[\seqnum{A002321}]{OEIS} 
\begin{align*} 
M(x) & = \sum_{n \leq x} \mu(n),\ x \geq 1, \\ 
     & \longmapsto \{1, 0, -1, -1, -2, -1, -2, -2, -2, -1, -2, -2, -3, -2, 
     -1, -1, -2, -2, -3, -3, -2, -1, -2, -2\}
\end{align*} 
A related function which counts the 
number of \emph{squarefree} integers than $x$ sums the average order of the M\"obius function as 
\cite[\seqnum{A013928}]{OEIS} 
\[ 
Q(n) = \sum_{n \leq x} |\mu(n)| \sim \frac{6x}{\pi^2} + O\left(\sqrt{x}\right). 
\] 
It is known that the asymptotic density of the positively versus negatively 
weighted sets of squarefree numbers are in fact equal as $x \rightarrow \infty$: 
\[
\mu_{+}(x) = \frac{\#\{1 \leq n \leq x: \mu(n) = +1\}}{Q(x)} = 
     \mu_{-}(x) = \frac{\#\{1 \leq n \leq x: \mu(n) = -1\}}{Q(x)} 
     \xrightarrow[n \rightarrow \infty]{} \frac{3}{\pi^2}. 
\]
While this limiting law suggests an even bias for the Mertens function, 
in practice $M(x)$ has a noted negative bias in its values, and the actual 
local oscillations between the approximate densities of the sets 
$\mu_{\pm}(x)$ lend an unpredictable nature to the function and its 
characteistic oscillatory sawtooth shaped plot -- even over asymptotically 
large and variable intervals.

\subsubsection{Properties} 

The well-known approach to evaluating the behavior of $M(x)$ for large 
$x \rightarrow \infty$ results from a formulation of this summatory 
function as a predictable exact sum involving $x$ and the non-trivial 
zeros of the Riemann zeta function for all real $x > 0$. 
This formula is easily expressed via an inverse Mellin transformation 
over the reciprocal zeta function. In particular, 
we notice that since by Perron's formula we have 
\[
\frac{1}{\zeta(s)} = \int_1^{\infty} \frac{s \cdot M(x)}{x^{s+1}} dx, 
\]
we then obtain that 
\[
M(x) = \frac{1}{2\pi\imath} \int_{T-\imath\infty}^{T+\imath\infty} 
     \frac{x^s}{s \cdot \zeta(s)} ds. 
\] 
This representation along with the standard Euler product 
representation for the reciprocal zeta function leads us to the 
exact expression for $M(x)$ when $x > 0$ given by the next theorem. 

\begin{theorem}[Analytic Formula for $M(x)$] 
\label{theorem_MxMellinTransformInvFormula} 
If the RH is true, then there exists an infinite sequence 
$\{T_k\}_{k \geq 1}$ satisfying $k \leq T_k \leq k+1$ for each $k$ 
such that for any $x \in \mathbb{R}_{>0}$ 
\[
M(x) = \lim_{k \rightarrow \infty} 
     \sum_{\substack{\rho: \zeta(\rho) = 0 \\ |\Im(\rho)| < T_k}} 
     \frac{x^{\rho}}{\rho \cdot \zeta^{\prime}(\rho)} - 2 + 
     \sum_{n \geq 1} \frac{(-1)^{n-1}}{n \cdot (2n)! \zeta(2n+1)} 
     \left(\frac{2\pi}{x}\right)^{2n} + 
     \frac{\mu(x)}{2} \Iverson{x \in \mathbb{Z}^{+}}. 
\] 
\end{theorem} 

An unconditional bound on the Mertens function due to Walfisz 
\cite{see Ivic} states that there is an absolute constant $C > 0$ such that 
$$M(x) \ll x \exp\left(-C \cdot \log^{3/5}(x) 
  (\log\log x)^{-3/5}\right).$$ 
Under the assumption of the RH, Soundararajan proved new updated estimates 
bounding $M(x)$ for large $x$ in 2009 of the following forms: 
\begin{align*} 
M(x) & \ll \sqrt{x} \exp\left(\log^{1/2}(x) (\log\log x)^{14}\right), \\ 
M(x) & = O\left(\sqrt{x} \exp\left( 
     \log^{1/2}(x) (\log\log x)^{5/2+\epsilon}\right)\right),\ 
     \forall \epsilon > 0. 
\end{align*} 
Other explicit bounds due to the article by Kotnik include the following 
simpler estimates for the Mertens function when $x$ is sufficiently 
large: 
\begin{align*} 
|M(x)| & < \frac{x}{4345},\ \forall x > 2160535, \\ 
|M(x)| & < \frac{0.58782 \cdot x}{\log^{11/9}(x)},\ \forall x > 685. 
\end{align*} 

\subsubsection{Conjectures} 

The Riemann Hypothesis (RH) is equivalent to showing that 
$M(x) = O\left(x^{1/2+\varepsilon}\right)$ for any 
$0 < \varepsilon < \frac{1}{2}$. 
It is still unresolved whether 
\[ 
\limsup_{x\rightarrow\infty} |M(x)| / \sqrt{x} = \infty, 
\] 
although computational evidence suggests that this is a likely conjecture 
\cite{ORDER-MERTENSFN,HURST-2017}. 
There is a rich history to the original statement of the \emph{Mertens conjecture} which 
states that 
\[ 
|M(x)| < c \cdot x^{1/2},\ \text{ some constant $c > 0$, }
\] 
which was first verified by Mertens for $c = 1$ and $x < 10000$, 
although since its beginnings in 1897 has since been disproved by computation by 
Odlyzko and t\'{e} Riele in the early 1980's. 

There are a number of other interesting unsolved and at 
least somewhat accessible open problems 
related to the asymptotic behavior of $M(x)$ at large $x$. 
It is believed that the sign of $M(x)$ changes infinitely often. 
That is to say that it is widely believed that $M(x)$ is 
ocsillatory and exhibits a negative bias insomuch as 
$M(x) < 0$ more frequently than $M(x) > 0$ over all 
$x \in \mathbb{N}$. 
One of the most famous still unanswered questions about the Mertens 
function concerns whether $|M(x)| / \sqrt{x}$ is unbounded on the 
natural numbers. In particular, the precise statement of this 
problem is to produce an affirmative answer whether 
$\limsup_{x \rightarrow \infty} |M(x)| / \sqrt{x} = +\infty$, or 
equivalently whether there is an infinite sequence of natural numbers 
$\{x_1, x_2, x_3, \ldots\}$ such that $M(x_i) x_i^{-1/2}$ grows without 
bound along this subsequence. 

Extensive computational evidence has produced 
a conjecture due to Gonek that in fact the limiting behavior of 
$M(x)$ satisfies 
that $$\limsup_{x \rightarrow \infty} \frac{|M(x)|}{\sqrt{x} 
(\log\log x)^{5/4}},$$ 
corresponds to some bounded constant. 
A probabilistic proof along these 
lines has been given by Ng in 2008. To date an exact rigorous 
proof that $M(x) / \sqrt{x}$ is unbounded still remains elusive. 
We cite that prior to this point it is known that \cite[\cf \S 4.1]{PRIMEREC} 
\[
\limsup_{x\rightarrow\infty} \frac{M(x)}{\sqrt{x}} > 1.060\ \qquad (\text{now } 1.826054), 
\] 
and 
\[ 
\liminf_{x\rightarrow\infty} \frac{M(x)}{\sqrt{x}} < -1.009\ \qquad (\text{now } -1.837625), 
\] 
although based on work by Odlyzyko and te Riele it seems probable that 
each of these limits should be $\pm \infty$, respectively 
\cite{ODLYZ-TRIELE,MREVISITED,ORDER-MERTENSFN,HURST-2017}. 
It is also known that $M(x) = \Omega_{\pm}(\sqrt{x})$ and 
$M(x) / \sqrt{x} = \Omega_{\pm}(1)$. 

\subsection{A new approach to bounding $M(x)$ from below} 

\subsubsection{Summing series over Dirichlet convolutions} 

\begin{theorem}[Summatory functions of Dirichlet convolutions] 
\label{theorem_SummatoryFuncsOfDirCvls} 
Let $f,g: \mathbb{Z}^{+} \rightarrow \mathbb{C}$ be any arithmetic functions such that $f(1) \neq 0$. 
Suppose that $F(x) := \sum_{n \leq x} f(n)$ and $G(x) := \sum_{n \leq x} g(n)$ denote the summatory 
functions of $f,g$, respectively, and that $F^{-1}(x)$ denotes the summatory function of the 
Dirichlet inverse $f^{-1}(n)$ of $f$, i.e., the unique arithmetic function such that 
$f \ast f^{-1} = \varepsilon$ where $\varepsilon(n) = \delta_{n,1}$ is the multiplicative identity 
with respect to Dirichlet convolution. Then we have the following equivalent expressions for the 
summatory function of $f \ast g$ for integers $x \geq 1$: 
\begin{align*} 
\pi_{f \ast g}(x) & = \sum_{n \leq x} \sum_{d|n} f(d) g(n/d) \\ 
     & = \sum_{d \leq x} f(d) G\left(\Floor{x}{d}\right) \\ 
     & = \sum_{k=1}^{x} G(k) \left[F\left(\Floor{x}{k}\right) - 
     F\left(\Floor{x}{k+1}\right)\right]. 
\end{align*} 
Moreover, we can invert the linear system determining the coefficients of $G(k)$ for $1 \leq k \leq x$ 
naturally to express $G(x)$ as a linear combination of the original left-hand-side summatory function as:
\begin{align*} 
G(x) & = \sum_{j=1}^{x} \pi_{f \ast g}(j) \left[F^{-1}\left(\Floor{x}{j}\right) - 
     F^{-1}\left(\Floor{x}{j+1}\right)\right] \\ 
     & = \sum_{n=1}^{x} f^{-1}(n) \pi_{f \ast g}\left(\Floor{x}{n}\right). 
\end{align*} 
\end{theorem} 

\begin{cor}[Convolutions Arising From M\"obius Inversion] 
\label{cor_CvlGAstMu} 
Suppose that $g$ is an arithmetic function with $g(1) \neq 0$. Define the summatory function of 
the convolution of $g$ with $\mu$ by $\widetilde{G}(x) := \sum_{n \leq x} (g \ast \mu)(n)$. 
Then the Mertens function equals 
\[
M(x) = \sum_{k=1}^{x} \left(\sum_{j=\floor{\frac{x}{k+1}}+1}^{\floor{\frac{x}{k}}} g^{-1}(j)\right) 
     \widetilde{G}(k), \forall x \geq 1. 
\]
\end{cor} 

\subsubsection{A motivating special case} 
\label{example_InvertingARecRelForMx_Intro}
 
Using $\chi_{\mathbb{P}} + \varepsilon = (\omega + 1) \ast \mu$, 
where $\chi_{\mathbb{P}}$ is the characteristic 
function of the primes, we have that $\widetilde{G}(x) = \pi(x) + 1$ in 
Corollary \ref{cor_CvlGAstMu}. 
In particular, the corollary implies that 
\begin{equation} 
\label{eqn_Mx_gInvnPixk_formula} 
M(x) = \sum_{k=1}^{x} (\omega+1)^{-1}(k) \left[\pi\left(\Floor{x}{k}\right) + 1\right]. 
\end{equation} 
We can compute the first few terms for the
Dirichlet inverse sequence of 
$g(n) := \omega(n) + 1$ numerically for the first few sequence values as 
\[
\{g^{-1}(n)\}_{n \geq 1} = \{1, -2, -2, 2, -2, 5, -2, -2, 2, 5, -2, -7, -2, 5, 5, 2, -2, -7, -2, 
     -7, 5, 5, -2, 9, \ldots \}. 
\] 
The sign of these terms is given by $\lambda(n) = (-1)^{\Omega(n)}$ 
(see Proposition \ref{prop_SignageDirInvsOfPosBddArithmeticFuncs_v1}). 
Note that since the DGF of $\omega(n)$ is given by 
$D_{\omega}(s) = P(s) \zeta(s)$ where $P(s)$ is the \emph{prime zeta function}, we do have a 
Dirichlet series for the inverse functions to invert coefficient-wise using more classical 
contour integral methods, e.g., using \cite[\S 11]{APOSTOLANUMT} 
\[
f(n) = \lim_{T \rightarrow \infty} \frac{1}{2T} \int_{-T}^{T} 
     \frac{n^{\sigma+\imath t}}{\zeta(\sigma+\imath t)(P(\sigma+\imath t) + 1)}, \sigma > 1. 
\]
Fr\"oberg has previously done some preliminary investigation as to the properties of the 
inversion to find the coefficients of $(1+P(s))^{-1}$ \cite{FROBERG-1968}. 

We will instead take a more combinatorial tack to investigating bounds on this inverse function 
sequence in the coming sections. 
Consider the following motivating conjecture: 

\begin{conjecture}
\label{lemma_gInv_MxExample} 
Suppose that $n \geq 1$ is a squarefree integer. We have the following properties characterizing the 
Dirichlet inverse function $g^{-1}(n) = (\omega+1)^{-1}(n)$ over these integers: 
\begin{itemize} 

\item[(A)] $g^{-1}(1) = 1$, which follows immediately by computation; 
\item[(B)] $\operatorname{sgn}(g^{-1}(n)) = \mu(n) \equiv \lambda(n)$; 
\item[(C)] If $w(n) = k$, we can write the inverse function at $k$ as 
     \[
     g^{-1}(n) = \sum_{m=0}^{k} \binom{k}{m} \cdot m!. 
     \]
\end{itemize} 
We illustrate parts (B)--(C) of this conjecture clearly using 
Table \ref{table_conjecture_Mertens_ginvSeq_approx_values} given on 
page \pageref{table_conjecture_Mertens_ginvSeq_approx_values} of the appendix section. 
\end{conjecture} 

\textit{Why exactly is the Dirichlet inverse function, $g^{-1}(n)$, difficult to evaluate? } 
There are several apparent reasons for this. The first is that the Dirichlet inverse function not only 
depends on the prime factorization of $n$ in the typical way, involving weighted sums of $\Omega(n)$ terms of the 
function $\omega(n) + 1$, but also in the additive nature of how we build up and assemble these terms in an essentially 
non-multiplicative, but instead very additive, way.  
Note that for distinct primes $a,b$ and positive integers $m,n \geq 1$, the 
(incomplete) additivity of $\omega(n)$ implies that $\omega(a^m b^n) = \omega(a) + \omega(b)$ with  
$\omega(1) = 0$. Secondly, the extra additive factor of $+1$ (that was added to make the function Dirichlet invertible) 
also does not depend on $n$ in the corresponding expansions of the Dirichlet inverse terms. 
A table of the first several explicit values of $(f+1)^{-1}(n)$ for $f(1) = 0$ and $f$ additive
are given in Table \ref{table_DirInvFuncExps_fp1_fAdditive} on page 
\pageref{table_DirInvFuncExps_fp1_fAdditive}. 
Note that the additivity of $f$ in forming the Dirichlet inverse of $(f+1)^{-1}(n)$ significantly influences the 
sign of the inverse function, given by $\lambda(n)$. 

The realization that the beautiful, and simplistic, e.g., not terribly complicated considering the 
subject matter, form of property (C) 
in Conjecture \ref{lemma_gInv_MxExample} holds for all squarefree $n \geq 1$ 
motivates our pursuit of formulas for the inverse functions $g^{-1}(n)$ based on the configuration of the 
exponents in the prime factorization of any $n \geq 2$. 
In Section \ref{Section_InvFunc_PreciseExpsAndAsymptotics} we consider expansions of these inverse functions 
recursively, starting from a few first exact cases of an auxillary function, $C_k(n)$, 
that depends on the precise exponents in the prime factorization of $n$. 
We then prove limiting asymptotics for these functions and assemble the main terms in the expansion of 
$g^{-1}(n)$ using artifacts from combinatorial analysis. 
Combined with the DGF-based generating function for certain summatory functions indicating the parity of 
$\Omega(n)$ introduced in the next subsection of this introduction, 
this take on the identity in \eqref{eqn_Mx_gInvnPixk_formula} provides us with a powerful new method to 
bound $M(x)$ from below. 
We will sketch the key results and formulation to the construction we actually use to prove the 
new lower bounds on $M(x)$ next. 

From this point on, we fix the Dirichlet invertible function $g(n) := \omega(n) + 1$ and denote its 
inverse with respect to Dirichlet convolution by $g^{-1}(n) = (\omega+1)^{-1}(n)$. 
For natural numbers $n \geq 1, k \geq 0$, let 
\begin{align*} 
C_k(n) := \begin{cases} 
     \varepsilon(n) = \delta_{n,1}, & \text{ if $k = 0$; } \\ 
     \sum\limits_{d|n} \omega(d) C_{k-1}(n/d), & \text{ if $k \geq 1$. } 
     \end{cases} 
\end{align*} 
By M\"obius inversion (see Lemma \ref{lemma_AnExactFormulaFor_gInvByMobiusInv_v1}), 
we have that 
\[
(g^{-1} \ast 1)(n) = \lambda(n) \cdot C_{\Omega(n)}(n), \forall n \geq 1. 
\]
We have limiting asymptotics on these functions given by the following theorem: 

\begin{theorem}[Asymptotics for the functions $C_k(n)$] 
\label{theorem_Ckn_GeneralAsymptoticsForms} 
Let $\mathds{1}_{\ast_m}(n)$ denote the $m$-fold Dirichlet convolution of one with itself at $n$. 
The function $\sigma_0 \ast 1_{\ast_m}$ is multiplicative with values at prime powers 
given by 
\[
(\sigma_0 \ast \mathds{1}_{\ast_m})(p^{\alpha}) = \binom{\alpha+m+1}{m+1}. 
\]
We have the following asymptotic bases cases for the functions $C_k(n)$: 
\begin{align*} 
C_1(n) & \sim \log\log n \\ 
C_2(n) & \sim \frac{\sigma_0(n) n}{\log n} + O(\log\log n) \\ 
C_3(n) & \sim -\frac{(\sigma_0 \ast 1)(n) n^2}{\log n} + 
     O\left(n \cdot \log\log n\right). 
\end{align*} 
For all $k \geq 4$, we obtain that the dominant asymptotic term and the error bound terms for 
$C_k(n)$ are given by 
\[
C_k(n) \sim (\sigma_0 \ast \mathds{1}_{\ast_{k-2}})(n) \times \frac{(-1)^{k} n^{k-1}}{(\log n)^{k-1} (k-1)!} + 
     O_k\left(\frac{n^{k-2}}{(k-2)!} \cdot \frac{(\log\log n)^{k-2}}{(\log n)^{k-2}}\right), 
     \mathrm{\ as\ }n \rightarrow \infty. 
\]
\end{theorem} 

Then we can prove (see Corollary \ref{cor_ASemiForm_ForGInvx_v1}) that 
\[
g^{-1}(n) \sim \lambda(n) \times \sum_{d|n} C_{\Omega(d)}(d). 
\]
This in turn implies that 
\[
G^{-1}(x) \succsim \sum_{n \leq x} \lambda(n) \cdot C_{\Omega(n)}(n) \times 
     \sum_{d=1}^{\Floor{x}{n}} \lambda(d). 
\]
Now we require the bounds suggested in the next section to work at summing the 
summatory functions, $G^{-1}(x)$, for large $x$ as $x \rightarrow \infty$. 

\subsubsection{DGFs from Mongomery and Vaughan} 

Our inspiration for the new bounds found in the last sections of this article allows us to sum 
non-negative arithmetic functions weighted by the Liouville lambda function, 
$\lambda(n) = (-1)^{\Omega(n)}$. In particular, it uses a hybrid generating function and DGF method 
under which we are able to recover ``good enough'' asymptotics about the summatory functions that 
encapsulate the parity of $\lambda(n)$: 
\[
\widehat{\pi}_k(x) := \#\{n \leq x: \Omega(n) = k\}, k \geq 1. 
\] 
The precise statement of the theorem that we transform for these new bounds is re-stated as follows: 

\begin{theorem}[Montgomery and Vaughan, \S 7.4]
\label{theorem_HatPi_ExtInTermsOfGz} 
Let $\widehat{\pi}_k(x) := \#\{n \leq x: \Omega(n)=k\}$. For $R < 2$ we have that 
\[
\widehat{\pi}_k(x) = \mathcal{G}\left(\frac{k-1}{\log\log x}\right) \frac{x}{\log x} 
     \frac{(\log\log x)^{k-1}}{(k-1)!} \left(1 + O_R\left(\frac{k}{(\log\log x)^2}\right)\right),  
\]
uniformly for $1 \leq k \leq R \log\log x$ where 
\[
\mathcal{G}(z) := \frac{F(1, z)}{\Gamma(z+1)} = \frac{1}{\Gamma(z+1)} \times 
     \prod_p \left(1-\frac{z}{p}\right)^{-1} \left(1-\frac{1}{p}\right)^z. 
\]
\end{theorem} 

The precise formulations of the inverse function asymptotics 
proved in Section \ref{Section_InvFunc_PreciseExpsAndAsymptotics} depend on being able to form 
significant lower bounds on the summatory functions of an always positive arithmetic function 
weighted by $\lambda(n)$. 
The next theorem, proved in Section \ref{Section_MVCh7_GzBounds}, 
is the crux of the starting point for our new asymptotic lower bounds. 

\begin{theorem}[Generating functions of symmetric functions] 
\label{theorem_GFs_SymmFuncs_SumsOfRecipOfPowsOfPrimes} 
We obtain upper and lower bounds of the form
\begin{align*} 
\alpha_0(z, x) & \leq \prod_{p \leq x} \left(1-\frac{z}{p}\right)^{-1} \leq \alpha_1(z, x), 
\end{align*} 
where it suffices to take 
\begin{align*}
\alpha_0(z, x) & = \frac{\exp\left(\frac{55}{4} \log^2 2\right)}{\log^3 2} (\log x)^3 
     \left(\frac{e^{B} \log^2 x}{\log 2}\right)^{z} \\ 
\alpha_1(z, x) & = \exp\left(\frac{11}{3} \log^2 x\right) \left(e^{B} \log 2\right)^{z}. 
\end{align*}  
\end{theorem} 

The argument providing new lower bounds for $G(z)$ is completed by the 
proof given in Corollary \ref{cor_BoundsOnGz_FromMVBook_initial_stmt_v1}. 
This leads to a structure involving the 
incomplete gamma function inherited from Theorem \ref{theorem_HatPi_ExtInTermsOfGz}. 
In Lemma \ref{lemma_CLT_and_AbelSummation}, we justify that this construction, 
which holds uniformly for $k \leq \frac{3}{2} \log\log x$ (taking $R := \frac{3}{2}$), 
allows us to asymptotically enumerate the main terms 
in the expansions of $\widehat{\pi}_k(x)$ when we sum over just $k$ in this range 
(as opposed to $k \leq \frac{\log x}{\log 2}$). 

\newpage 
\section{Preliminary proofs and configuration} 
\label{Section_PrelimProofs_Config} 

\subsection{Establishing the summatory function inversion identities} 

Given the interpretation of the summatory functions over an arbitrary Dirichlet convolution 
(and the vast number of such identities for special number theoretic functions -- \cf 
\cite{CATALOG-INTDIRSERIES,CATALOG-LAMBERTSERIES}), it is not surprising that this formulation of the first theorem 
may well provide many fruitful applications, indeed. In addition to those cited in the 
compendia of the catalog reference, we have notable identities of the form: 
$(f \ast 1)(n) = [q^n] \sum_{m \geq 1} f(m) q^m / (1-q^m)$, 
$\sigma_k = \operatorname{Id}_k \ast 1$, $\operatorname{Id}_1 = \phi \ast \sigma_0$, 
$\chi_{\operatorname{sq}} = \lambda \ast 1$ (see sections below), 
$\operatorname{Id}_k = J_k \ast 1$, $\log = \Lambda \ast 1$, and of course 
$2^{\omega} = \mu^2 \ast 1$. 
The result in Theorem \ref{theorem_SummatoryFuncsOfDirCvls} is 
natural and displays a quite beautiful form of symmetry between the 
initial matrix terms, $$t_{x,j}(f) = \sum_{k=\floor{\frac{x}{j+1}}+1}^{\floor{\frac{x}{j}}} f(k),$$ and the 
corresponding inverse matrix, $$t_{x,j}^{-1}(f) = \sum_{k=\floor{\frac{x}{j+1}}+1}^{\floor{\frac{x}{j}}} f^{-1}(k),$$ 
as expressed by the duality of $f$ and its Dirichlet inverse function $f^{-1}$. Since the recurrence relations for the 
summatory functions $G(x)$ arise naturally in applications where we have established bounds on sums of 
Dirichlet convolutions of arithmetic functions, we will go ahead and prove it here before moving along to the 
motivating examples of the use of this theorem. 

\begin{proof}[Proof of Theorem \ref{theorem_SummatoryFuncsOfDirCvls}]
Let $h,g$ be arithmetic functions where $g(1) \neq 1$ has a Dirichlet inverse. Denote the summatory functions of $h$ and $g$, 
respectively, by $H(x) = \sum_{n \leq x} h(n)$ and $G(x) = \sum_{n \leq x} g(n)$. 
We define $S_{g,h}(x)$ to be the summatory function of the Dirichlet convolution of $g$ with $h$: $g \ast h$. 
Then we can easily see that the following expansions hold: 
\begin{align*} 
S_{g,h}(x) & := \sum_{n=1}^{x} \sum_{d|n} g(n) h(n/d) = \sum_{d=1}^x g(d) H\left(\floor{\frac{x}{d}}\right) \\ 
     & = \sum_{i=1}^x \left[G\left(\floor{\frac{x}{i}}\right) - G\left(\floor{\frac{x}{i+1}}\right)\right] H(i). 
\end{align*} 
Thus we have an implicit statement of a recurrence relation for the summatory function $H$, weighted by $g$ and $G$, 
whose non-homogeneous term is the summatory function, $S_{g,h}(x)$, of the Dirichlet convolutions $g \ast h$. 
We form the matrix of coefficients associated with this system for $H(x)$, and proceed to invert it to express an 
exact solution for this function over all $x \geq 1$. Let the ordinary (initial, non-inverse) matrix entries be denoted by 
\[
g_{x,j} := G\left(\floor{\frac{x}{j}}\right) - G\left(\floor{\frac{x}{j+1}}\right) \equiv G_{x,j} - G_{x,j+1}. 
\]
Then the matrix we must invert in this problem is lower triangular, with ones on its diagonals -- and hence is invertible. 
Moreover, if we let $\hat{G} := (G_{x,j})$, then this matrix is 
expressable by an invertible shift operation as 
\[
(g_{x,j}) = \hat{G} (I - U^{T}); \qquad U = (\delta_{i,j+1}), (I - U^T)^{-1} = (\Iverson{j \leq i}). 
\]
Here, $U$ is the $N \times N$ matrix whose $(i,j)^{th}$ entries are defined by 
$(U)_{i,j} = \delta_{i+1,j}$. 

It is a useful fact that if we take successive differences of floor functions, we get non-zero behavior at divisors: 
\[
G\left(\floor{\frac{x}{j}}\right) - G\left(\floor{\frac{x-1}{j}}\right) = 
     \begin{cases} 
     g\left(\frac{x}{j}\right), & \text{ if $j | x$; } \\ 
     0, & \text{ otherwise. } 
     \end{cases}
\]
We use this property to shift the matrix $\hat{G}$, and then invert the result, to obtain a matrix involving the 
Dirichlet inverse of $g$: 
\begin{align*} 
\left[(I-U^{T}) \hat{G}\right]^{-1} & = \left(g\left(\frac{x}{j}\right) \Iverson{j|x}\right)^{-1} = 
     \left(g^{-1}\left(\frac{x}{j}\right) \Iverson{j|x}\right). 
\end{align*} 
Now we can express the inverse of the target matrix $(g_{x,j})$ in terms of these Dirichlet inverse functions 
as follows: 
\begin{align*} 
(g_{x,j}) & = (I-U^{T})^{-1} \left(g\left(\frac{x}{j}\right) \Iverson{j|x}\right) (I-U^{T}) \\ 
(g_{x,j})^{-1} & = (I-U^{T})^{-1} \left(g^{-1}\left(\frac{x}{j}\right) \Iverson{j|x}\right) (I-U^{T}) \\ 
     & = \left(\sum_{k=1}^{\floor{\frac{x}{j}}} g^{-1}(k)\right) (I-U^{T}) \\ 
     & = \left(\sum_{k=1}^{\floor{\frac{x}{j}}} g^{-1}(k) - \sum_{k=1}^{\floor{\frac{x}{j+1}}} g^{-1}(k)\right). 
\end{align*} 
Thus the summatory function $H$ is exactly expressed by the inverse vector product of the form 
\begin{align*} 
H(x) & = \sum_{k=1}^x g_{x,k}^{-1} \cdot S_{g,h}(k) \\ 
     & = \sum_{k=1}^x \left(\sum_{j=\floor{\frac{x}{k+1}}+1}^{\floor{\frac{x}{k}}} g^{-1}(j)\right) \cdot S_{g,h}(k). 
     \qedhere
\end{align*} 
\end{proof} 

\subsection{Proving the crucial property from the conjecture over the squarefree integers} 

\begin{prop}[The characteristic function of the primes] 
\label{prop_AntiqueDivisorSumIdent} 
Let $\chi_{\mathbb{P}}$ denote the characteristic function of the primes, 
$\varepsilon(n) = \delta_{n,1}$ be the identity with respect to Dirichlet convolution, 
and denote by $\omega(n)$ the additive function that counts the number of 
distinct prime factors of $n$. 
Then 
$$\chi_{\mathbb{P}} + \varepsilon = (\omega + 1) \ast \mu.$$ 
The summatory function of the LHS is $\widetilde{G}(x) = \pi(x)+1$. 
The corresponding characteristic function for the prime powers is similarly given by 
$\chi_{\operatorname{PP}} = \Omega \ast \mu$. 
\end{prop}
\begin{proof} 
The core is to prove that for all $n \geq 1$, 
$\chi_{\mathbb{P}}(n) = (\mu \ast \omega)(n)$ -- our claim. 
We notice that the Mellin transform of $\pi(x)$ -- the summatory function of 
$\chi_{\mathbb{P}}(n)$ -- at $-s$ is given by 
\begin{align*} 
s \cdot \int_1^{\infty} \frac{\pi(x)}{x^{s+1}} dx & = \sum_{n \geq 1} \frac{\nabla[\pi](n-1)}{n^s} \\ 
     & = \sum_{n \geq 1} \frac{\chi_{\mathbb{P}}(n)}{n^s} = P(s). 
\end{align*} 
This is typical construction which more generally relates the Mellin transform $\mathcal{M}[S_f](-s)$ to the 
DGF of the sequence $f(n)$ as cited, for example, in \cite[\S 11]{APOSTOLANUMT}. Now we consider the 
DGF of the right-hand-side function, $f(n) := (\mu \ast \omega)(n)$, as 
\[
D_f(s) = \frac{1}{\zeta(s)} \times \sum_{n \geq 1} \frac{\omega(n)}{n^s} = P(s).  
\]
Thus for any $\Re(s) > 1$, the DGFs of each side of the 
claimed equation coincide. So by uniqueness of Dirichlet series, we see that in fact the claim 
holds. To obtain the full result, we add to each side of this equation a term of 
$\varepsilon(n) \equiv (\mu \ast 1)(n)$, and then factor the resulting convolution identity. 
\end{proof} 

\begin{prop}[The sign of $g^{-1}(n)$]
\label{prop_SignageDirInvsOfPosBddArithmeticFuncs_v1} 
For all $n \geq 1$, $\operatorname{sgn}(g^{-1}(n)) = \lambda(n)$. 
\end{prop} 
\begin{proof} 
Let $D_f(s) := \sum_{n \geq 1} f(n) n^{-s}$ denote the Dirichlet generating function (DGF) of $f(n)$. 
Then we have that 
\begin{align*} 
D_{(\omega+1)^{-1}}(s) = \frac{D_{\lambda}(s)}{(P(s)+1) \zeta(2s)}. 
\end{align*} 
Let $h^{-1}(n) := (\omega \ast \mu + \varepsilon)^{-1}(n) = [n^{-s}](P(s) + 1)^{-1}$. 
Then we have that 
\begin{align*} 
(h^{-1} \ast 1)(n) & = - \sum_{p_1|n} h^{-1}\left(\frac{n}{p_1}\right) 
     = \lambda(n) \times \sum_{p_1|n} \sum_{p_2|\frac{n}{p_1}} \cdots \sum_{p_{\Omega(n)} | 
     \frac{n}{p_1p_2\cdots p_{\Omega(n)-1}}} 1 \\ 
     & = \begin{cases} 
     \lambda(n) \times (\Omega(n) - 1)!, & n \geq 2; \\ 
     \lambda(n), & n=1. 
     \end{cases} 
\end{align*} 
So by M\"obius inversion 
\begin{align*} 
h^{-1}(n) & = \lambda(n) \left[\sum_{\substack{d|n \\ d<n}} \lambda(d) \mu(d) (\Omega(n/d)-1)! + 1\right] 
     = \lambda(n) \left[\sum_{\substack{d|n \\ d<n}} \mu^2(d) (\Omega(n/d)-1)! + 1\right]. 
\end{align*} 
Then we finally have that 
\begin{align*} 
(\omega+1)^{-1}(n) & = \lambda(n) \times \sum_{d|n} \lambda(d) 
     \left[\sum_{\substack{r|\frac{n}{d} \\ r < \frac{n}{d}}}  \mu^2(r) (\Omega\left(\frac{n}{dr}\right)-1)!+ 1\right] 
     \chi_{\operatorname{sq}}(d) \mu(\sqrt{d}), 
\end{align*} 
where $\chi_{\operatorname{sq}}$ is the characteristic function of the squares. 
In either case of $\lambda(n) = \pm 1$, there are positive constants $C_{1,n},C_{2,n} > 0$ such that 
\[
\lambda(n) C_{1,n} \times \sum_{d^2|n} \lambda(d^2) \mu(d) \leq g^{-1}(n) \leq 
     \lambda(n) C_{1,n} \times \sum_{d^2|n} \lambda(d^2) \mu(d), 
\]
where $\sum_{d^2|n} \lambda(d^2) \mu(d) = \sum_{d^2|n} \mu^2(n) > 0$. 
This proves the result. 
\end{proof} 

\subsection{Other facts and listings of results we will need in our proofs} 
\label{subSection_OtherFactsAndResults} 

\begin{theorem}[Mertens theorem]
\label{theorem_Mertens_theorem}  
\[
P_1(x) := \sum_{p \leq x} \frac{1}{p} = \log\log x + B + O\left(e^{-(\log x)^{\frac{1}{14}}}\right), 
\]
where $B \approx 0.2614972128476427837554$ is an absolute constant.
\end{theorem} 

\begin{cor}
\label{lemma_Gz_productTermV2} 
We have that for sufficiently large $x \gg 1$ 
\[
\prod_{p \leq x} \left(1 - \frac{1}{p}\right) = \frac{e^{-B}}{\log x}\left[ 
     1 - \frac{(\log x)^{1/14}}{B} + o\left((\log x)^{1/14}\right)\right]. 
\]
Hence, for $1 < |z| < R < 2$ we obtain that 
\[
\prod_{p \leq x} \left(1 - \frac{1}{p}\right)^{z} = \frac{e^{-Bz}}{(\log x)^{z}} \left[ 
     1 - \frac{z}{B} (\log x)^{\frac{1}{14}} + o_z\left(z^2 \cdot (\log x)^{\frac{1}{14}}\right)\right]. 
\]
\end{cor} 
\begin{proof} 
By taking logarithms and using Mertens theorem above, we obtain that 
\begin{align*} 
\log \prod_{p \leq x} \left(1-\frac{1}{p}\right) & = \sum_{p \leq x} \log\left(1-\frac{1}{p}\right) \\ 
     & \approx -\log\log x - B +O\left(e^{-(\log x)^{1/14}}\right). 
\end{align*} 
Hence, the first formula follows by expanding out an alternating series for the exponential 
function. The second formula follows for $z \notin \mathbb{Z}$ by an application of the 
generalized binomial series given by 
\[
\log \prod_{p \leq x} \left(1-\frac{1}{p}\right)^{z} \approx \frac{e^{-Bz}}{(\log x)^{z}} \times 
     \sum_{r \geq 0} \binom{z}{r} \frac{(-1)^r}{B^r} (\log x)^{\frac{r}{14}}, 
\]
where for $1 < |z| < 2$, we obtain the next result stated above with 
$\binom{z}{1} = z$ and $\binom{z}{2} = z(z-1) / 2$. 
\end{proof} 

\begin{facts}[Exponential Integrals and Incomplete Gamma Functions] 
\label{facts_ExpIntIncGammaFuncs} 
\begin{subequations}
The following two variants of the \emph{exponential integral function} are defined by 
\begin{align*} 
\operatorname{Ei}(x) & := \int_{-x}^{\infty} \frac{e^{-t}}{t} dt, \\ 
E_1(z) & := \int_1^{\infty} \frac{e^{-tz}}{t} dt, \Re(z) \geq 0, 
\end{align*} 
where $\operatorname{Ei}(-kz) = -E_1(kz)$. We have the following inequalities providing 
quasi-polynomial upper and lower bounds on $E_1(z)$: 
\begin{equation}
1-\frac{3}{4} z \leq E_1(z) - \gamma - \log z \leq 1-\frac{3}{4} z + \frac{11}{36} z^2. 
\end{equation}
A related function is the (upper) \emph{incomplete gamma function} defined by 
\[
\Gamma(s, x) = \int_{x}^{\infty} t^{s-1} e^{-t} dt, \Re(s) > 0. 
\]
We have the following properties of $\Gamma(s, x)$: 
\begin{align} 
\Gamma(s, x) & = (s-1)! \cdot e^{-x} \times \sum_{k=0}^{s-1} \frac{x^k}{k!}, s \in \mathbb{Z}^{+}, \\ 
\Gamma(s+1, 1) & = e^{-1} \Floor{s!}{e}, s \in \mathbb{Z}^{+}, \\ 
\Gamma(s, x) & \sim x^{s-1} \cdot e^{-x}, |x| \rightarrow +\infty. 
\end{align}
\end{subequations}
\end{facts} 

\newpage 
\section{Summing arithmetic functions weighted by the function $\lambda(n) := (-1)^{\Omega(n)}$} 
\label{Section_MVCh7_GzBounds} 
        
\subsection{Discussion: The enumerative DGF result in Theorem \ref{theorem_HatPi_ExtInTermsOfGz} from 
            Montgomery and Vaughan} 

In the reference we have defined $F(s, z)$ for $\Re(s) > 1$ such that the 
Dirichlet series coefficients, $a_z(n)$, are defined by 
\[
\zeta(s)^z F(s, z) := \sum_{n \geq 1} \frac{a_z(n)}{n^s}, \Re(s) > 1. 
\]
For the function 
\[
F(s, z) := \prod_p \left(1 - \frac{z}{p^s}\right) \left(1-\frac{1}{p^s}\right)^z, 
\]
we obtain in the notation above that $a_z(n) \equiv z^{\Omega(n)}$, and that the summatory 
function satisfies 
\[
A_z(x) := \sum_{n \leq x} z^{\Omega(n)} = \sum_{k \geq 0} \widehat{\pi}_k(x) z^k. 
\]
Hence, by the Cauchy integral formula, for $r < 2$ we get that 
\[
\widehat{\pi}_k(x) = \frac{1}{2\pi\imath} \oint_{|z|=r} \frac{A_z(x)}{z^{k+1}} dz, 
\]
from which we obtain the stated formula in the theorem. 

What this enumeratively-flavored result of Montgomery and Vaughan allows us to do is get a 
``good enough'' lower bound on sums of positive and asymptotically bounded arithmetic functions 
weighted by the Lioville lambda function, $\lambda(n) = (-1)^{\Omega(n)}$. 
For comparison, we have known, more classical bounds due to Erd\"os (or earlier) that state for 
\[
\pi_k(x) := \#\{n \leq x: \omega(n) = k\}, 
\]
we have tightly that \cite{ERDOS-PRIMEK-FUNC,MV} 
\[
\pi_k(x) = (1 + o(1)) \cdot \frac{x}{\log x} \frac{(\log\log x)^{k-1}}{(k-1)!}. 
\] 

We seek to approximate the right-hand-side of $G(z)$ by only taking the products of the primes 
$p \leq x$, e.g., $p \in \left\{2,3,5,\ldots,x\right\}$. 
We will require some fairly elementary estimates of products of primes, Mertens theorem on the 
rate of divergence of the sum of the reciprocals of the primes, and on some generating function 
techniques involving elementary symmetric functions. 
The statements in Section \ref{subSection_OtherFactsAndResults} provide the basis for proving 
most of the lemmas we require. 

\subsection{The key new results utilizing Theorem \ref{theorem_HatPi_ExtInTermsOfGz}} 

\begin{cor} 
\label{cor_PartialSumsOfReciprocalsOfPrimePowers} 
For real $s \geq 1$, let 
\[
P_s(x) := \sum_{p \leq x} p^{-s}, x \gg 1. 
\]
When $s := 1$, we have the known bound in Mertens theorem. For $s > 1$, we obtain that 
\[
P_s(x) \approx E_1((s-1) \log 2) - E_1((s-1) \log x) + o(1). 
\]
It follows that 
\[
\gamma_0(s, x) + o(1) \leq P_s(x) \leq \gamma_1(s, x) + o(1), 
\]
where it suffices to take 
\begin{align*}
\gamma_0(z, x) & = -s\log\left(\frac{\log x}{\log 2}\right) - \frac{3}{4}s(s-1) \log(x/2) - 
     \frac{11}{36} s(s-1)^2 \log^2(2) \\ 
\gamma_1(z, x) & = \phantom{-} s\log\left(\frac{\log x}{\log 2}\right) - \frac{3}{4}s(s-1) \log(x/2) + 
     \frac{11}{36} s(s-1)^2 \log^2(x). 
\end{align*}
\end{cor} 
\begin{proof} 
Let $s > 1$ be real-valued. 
By Abel summation where our summatory function is given by $A(x) = \pi(x) \sim \frac{x}{\log x}$ and 
our function $f(t) = t^{-s}$ so that $f^{\prime}(t) = -s \cdot t^{-(s+1)}$, we obtain that 
\begin{align*} 
P_s(x) & = \frac{1}{x^s \cdot \log x} + s \cdot \int_2^{x} \frac{dt}{t^s \log t} \\ 
     & = E_1((s-1) \log x) - E_1((s-1) \log 2) + o(1), |x| \rightarrow \infty. 
\end{align*} 
Now using the inequalities in Facts \ref{facts_ExpIntIncGammaFuncs}, we obtain that the 
difference of the exponential integral functions is bounded above and below by 
\begin{align*} 
\frac{P_s(x)}{s} & \geq -\log\left(\frac{\log x}{\log 2}\right) - \frac{3}{4}(s-1) \log(x/2) - 
     \frac{11}{36} (s-1)^2 \log^2(2) \\ 
\frac{P_s(x)}{s} & \leq \phantom{-} \log\left(\frac{\log x}{\log 2}\right) - \frac{3}{4}(s-1) \log(x/2) + 
     \frac{11}{36} (s-1)^2 \log^2(x). 
\end{align*} 
This completes the proof of the bounds cited above in the statement of this lemma. 
\end{proof} 

\begin{proof}[Proof of Theorem \ref{theorem_GFs_SymmFuncs_SumsOfRecipOfPowsOfPrimes}] 
We have that for all integers $0 \leq k \leq m$
\[
[z^k] \prod_{1 \leq i \leq m} (1-f(i) z)^{-1} = [z^k] \exp\left(\sum_{j \geq 1} 
     \left(\sum_{i=1}^m f(i)^j\right) \frac{z^j}{j}\right). 
\]
In our case we have that $f(i)$ denotes the $i^{th}$ prime. Hence, summing over all $p \leq x$ 
in place of $0 \leq k \leq m$ in the previous formula applied in tandem with 
Corollary \ref{cor_PartialSumsOfReciprocalsOfPrimePowers}, we obtain that the logarithm of the 
generating function series we are after corresponds to 
\begin{align*} 
\log\left[\prod_{p \leq x} \left(1-\frac{z}{p}\right)^{-1}\right] & = (B + \log\log x) z + 
     \sum_{j \geq 2} \left[a(x) + b(x)(j-1) + c(x) (j-1)^2\right] z^j \\ 
     & = (B + \log\log x) z - a(x) \left(1 + \frac{1}{z-1} + z\right) + b(x) \left( 
     1 + \frac{2}{z-1} + \frac{1}{(z-1)^2}\right) \\ 
     & \phantom{= (B + \log\log x) z\ } - 
     c(x) \left( 
     1 + \frac{4}{z-1} + \frac{5}{(z-1)^2} + \frac{2}{(z-1)^3}\right). 
\end{align*} 
In the previous equations, the upper and lower bounds formed by the functions $(a,b,c)$ are 
given by 
\begin{align*} 
(a_{\ell}, b_{\ell}, c_{\ell}) & := \left(-\log\left(\frac{\log x}{\log 2}\right), 
     \frac{3}{4} \log\left(\frac{x}{2}\right), - \frac{11}{36} \log^2 2\right) \\ 
(a_u, b_u, c_u) & := \left(\log\left(\frac{\log x}{\log 2}\right), 
     -\frac{3}{4} \log\left(\frac{x}{2}\right), \frac{11}{36} \log^2 x\right). 
\end{align*} 
Now we make a prudent decision to set the uniform bound parameter to a middle ground value of 
$1 < R < 2$ as $R := \frac{3}{2}$ so that 
$$z \equiv z(k, x) = \frac{k-1}{\log\log x} \in [0, R),$$ for $x \gg 1$ very large. 
Thus $(z-1)^{-m} \in [(-1)^m, 2^m]$ for integers $m \geq 1$, and we can then form the upper and 
lower bounds from above. What we get out of these formulas is stated as in the theorem bounds. 
\end{proof} 

\begin{cor}[Bounds on $\mathcal{G}(z)$ from MV] 
\label{cor_BoundsOnGz_FromMVBook_initial_stmt_v1} 
We have that for the function $\mathcal{G}(z) := F(1, z) / \Gamma(z+1)$ from Montgomery and Vaughan, there 
is a constant $A_0$ and functions of $x$ only, $B_0(x), C_0(x)$, so that 
\[
A_0 \cdot B_0(x) \cdot C_0(x)^{z} \left(1 - \frac{z}{B} (\log x)^{\frac{1}{14}}\right) \leq \mathcal{G}(z). 
\]
It suffices to take 
\begin{align*} 
A_0 & = \frac{\exp\left(\frac{55}{4} \log^2 2\right)}{\log^3(2) \cdot \Gamma(5/2)} 
     \approx 1670.84511225 \\ 
B_0(x) & = \log^3 x \\ 
C_0(x) & = \frac{\log x}{\log 2}. 
\end{align*} 
\end{cor}
\begin{proof} 
This result is a consequence of applying both 
Corollary \ref{lemma_Gz_productTermV2} and 
Theorem \ref{theorem_GFs_SymmFuncs_SumsOfRecipOfPowsOfPrimes} to the definition of $G(z)$. 
In particular, we obtain bounds of the following form from the theorem: 
\[
\frac{A_0 \cdot B_0(x) \cdot C_0(x)^{z}}{\Gamma(z+1)} \leq 
     \frac{G(z)}{\prod_p \left(1-\frac{1}{p}\right)^{z}}. 
\]
Since $z \equiv z(k, x) = \frac{k-1}{\log\log x}$ and $k \in [1, R\log\log x]$, we obtain that 
for small $k$ and $x \gg 1$ large $\Gamma(z+1) \approx 1$, and for $k$ towards the upper bound of 
its interval that $\Gamma(z+1) \approx \Gamma(5/2) = \frac{3}{4} \sqrt{\pi}$ 
(recall that we set $R := 3/2$ in the 
preceeding proof of Theorem \ref{theorem_GFs_SymmFuncs_SumsOfRecipOfPowsOfPrimes}). 
Thus when we expand out the formula given by the corollary in conjunction with these bounds on the 
gamma function, we obtain the claimed results. 
\end{proof} 

\begin{lemma} 
\label{lemma_CLT_and_AbelSummation} 
Suppose that $f_k(n)$ is a sequence of arithmetic functions 
such that $f_k(n) > 0$ for all $n \geq 1$, $f_0(n) = \delta_{n,1}$, and 
$f_{\Omega(n)}(n) \succsim \widehat{\tau}_{\ell}(n)$ as $n \rightarrow \infty$ where 
$\widehat{\tau}_{\ell}(t)$ is a continuously differentiable function of $t$ for all 
large enough $t \gg 1$.  
We define the $\lambda$-sign-scaled summatory function of $f$ as follows: 
\[
F_{\lambda}(x) := \sum_{\substack{n \leq x \\ \Omega(n) \leq x}} 
     \lambda(n) \cdot f_{\Omega(n)}(n). 
\]
Let 
\[
A_{\Omega}^{(\ell)}(t) := \sum_{k=1}^{\floor{\frac{3}{2} \log\log t}} (-1)^k \widehat{\pi}_k(t). 
\]
Then we have that 
\[
F_{\lambda}(\log\log x) \succsim A_{\Omega}^{(\ell)}(x) \widehat{\tau}_{\ell}(\log\log x) - 
     \int_1^{\log\log x} 
     A_{\Omega}^{(\ell)}(t) \widehat{\tau}_{\ell}^{\prime}(t) dt. 
\]
\end{lemma}
\begin{proof} 
The formula for $F_{\lambda}(x)$ is valid by Abel summation provided that 
\[
\left\lvert \frac{\displaystyle\sum\limits_{\frac{3}{2} \log\log t < k \leq \frac{\log t}{\log 2}} 
     (-1)^k \widehat{\pi}_k(t)}{A_{\Omega}^{(\ell)}(t)}\right\rvert = o(1), 
\]
e.g., the asymptotically dominant terms indicating the parity of 
$\lambda(n)$ are encompassed by the terms summed by $A_{\Omega}^{(\ell)}(t)$ for 
sufficiently large $t$ as $t \rightarrow \infty$. 
Using the arguments in Montgomery and Vaughan \cite[\S 7; Thm.\ 7.21]{MV}, we can see that 
uniformly in $x$ 
\begin{align} 
\label{eqn_ProbCLT_for_Omegan_cited_result} 
\#\left\{n \leq x: \frac{\Omega(n) - \frac{3}{2}\log\log n}{\sqrt{\log\log n}} > 0\right\} & \sim
     x \left(1-\Phi(\sqrt{\log\log x})\right) \\ 
\notag 
     & = x \cdot \Phi(-\sqrt{\log\log x}) \xrightarrow{x \rightarrow \infty} 0, 
\end{align} 
where $\Phi(z)$ is the CDF of a standard normal random variable. 
Thus we have captured the asymptotically dominant main order terms in our formula as 
$x \rightarrow \infty$. 
\end{proof} 

\newpage
\section{Precisely enumerating and bounding the 
         Dirichlet inverse functions, $g^{-1}(n) := (\omega+1)^{-1}(n)$} 
\label{Section_InvFunc_PreciseExpsAndAsymptotics} 

\subsection{Developing an improved conjecture: 
            Proving precise bounds on the inverse functions} 

Conjecture \ref{lemma_gInv_MxExample} is not the most accurate fomulation of the limiting behavior of the 
Dirichlet inverse functions $g^{-1}(n)$ that we can see and prove. 
We need to come up with better bounds to plug back into the asymptotic analysis we obtain in the 
next sections. It turns out that these results are related to symmetric functions of the exponents in the 
prime factorizations of each $n \leq x$. The idea is that by having information about $g^{-1}(n)$ 
in terms of its prime factorization exponents for $n \leq x$, we should be able to extrapolate 
what we need which is information about the average behavior of the summatory functions, $G^{-1}(x)$, 
from the proofs above. 
Moreover, we notice the following observation that is suggestive of the semi-periodicity at play 
with the distinct values of $g^{-1}(n)$ distributed over $n \geq 2$.

\begin{heuristic}[Symmetry in $g^{-1}(n)$ in the exponents in the prime factorization of $n$] 
Suppose that $n_1, n_2 \geq 2$ are such that their factorizations into distinct primes are 
given by $n_1 = p_1^{\alpha_1} \cdots p_r^{\alpha_r}$ and $n_2 = q_1^{\beta_1} \cdots q_r^{\beta_r}$. 
If $\{\alpha_1, \ldots, \alpha_r\} \equiv \{\beta_1, \ldots, \beta_r\}$ as multisets of prime exponents, 
then $g^{-1}(n_1) = g^{-1}(n_2)$. For example, $g^{-1}$ has the same values on the squarefree integers 
with exactly two, three, and so on prime factors. There does not appear to be an easy, nor subtle 
direct recursion between the distinct $g^{-1}$ values, except through auxiliary function sequences. 
We will settle for an asymptotically accurate main term approximation to $g^{-1}(n)$ for large $n$ as 
$n \rightarrow \infty$ in the average case. 
\end{heuristic} 

With all of this in mind, we define the following sequence for integers $n \geq 1, k \geq 0$: 
\begin{align} 
C_k(n) := \begin{cases} 
     \varepsilon(n), & \text{ if $k = 0$; } \\ 
     \sum\limits_{d|n} \omega(d) C_{k-1}(n/d), & \text{ if $k \geq 1$. } 
     \end{cases} 
\end{align} 
We will illustrate by example the first few cases of these functions for small $k$ after we prove 
the next lemma. 
The sequence of important semi-diagonals of these functions begins as 
\cite[\seqnum{A008480}]{OEIS} 
\[
\{\lambda(n) \cdot C_{\Omega(n)}(n) \}_{n \geq 1} \mapsto \{
     1, -1, -1, 1, -1, 2, -1, -1, 1, 2, -1, -3, -1, 2, 2, 1, -1, -3, -1, \
     -3, 2, 2, -1, 4, 1, 2, \ldots \}. 
\]

\begin{lemma}[An exact formula for $g^{-1}(n)$] 
\label{lemma_AnExactFormulaFor_gInvByMobiusInv_v1} 
For all $n \geq 1$, we have that 
\[
g^{-1}(n) = \sum_{d|n} \mu(n/d) \lambda(d) C_{\Omega(d)}(d). 
\]
\end{lemma}
\begin{proof} 
We first write out the standard recurrence relation for the Dirichlet inverse of 
$\omega+1$ as 
\begin{align*} 
g^{-1}(n) & = - \sum_{\substack{d|n \\ d>1}} (\omega(d) + 1) f^{-1}(n/d) && \implies \\ 
     (g^{-1} \ast 1)(n) & = -(\omega \ast g^{-1})(n). 
\end{align*} 
Now by repeatedly expanding the right-hand-side, and removing corner cases in the nested sums since 
$\omega(1) = 0$ by convention, we find that 
\[
(g^{-1} \ast 1)(n) = (-1)^{\Omega(n)} C_{\Omega(n)}(n) = \lambda(n) C_{\Omega(n)}(n). 
\]
The statement follows by M\"obius inversion applied to each side of the last equation. 
\end{proof}

\begin{example}[Special cases of the functions $C_k(n)$ for small $k$]
We cite the following special cases which should be easy enough to see on paper: 
\begin{align*} 
C_0(n) & = \delta_{n,1} \\ 
C_1(n) & = \omega(n) \\ 
C_2(n) & = \sigma_0(n) \times \sum_{p|n} \frac{\nu_p(n)}{\nu_p(n)+1} - \gcd\left(\Omega(n), \omega(n)\right). 
\end{align*} 
We also can see a recurrence relation between successive $C_k(n)$ values over $k$ of the form 
\begin{equation}
\label{eqn_Ckn_recFormula_v1} 
C_k(n) = \sum_{p|n} \sum_{d\rvert\frac{n}{p^{\nu_p(n)}}} \sum_{i=1}^{\nu_p(n)} C_{k-1}\left(d \cdot p^i\right). 
\end{equation}
Thus we can work out further cases of the $C_k(n)$ for a while until we are able to understand the 
general trends of its asymptotic behaviors. 
In particular, we can compute the main term of $C_3(n)$ as follows where we use the notation that 
$p,q$ are prime indices: 
\begin{align*} 
C_3(n) & \sim \sum_{p|n} \sum_{d\rvert\frac{n}{p^{\nu_p(n)}}} \sum_{i=1}^{\nu_p(n)} \sum_{q|dp^i} 
     \frac{\nu_q(dp^i)}{\nu_q(dp^i)+1} \sigma_0(d) (i+1) \\ 
     & = \sum_{p|n} \sum_{d\rvert\frac{n}{p^{\nu_p(n)}}} \sum_{i=1}^{\nu_p(n)} \left[ 
     \sum_{q|d} \frac{\nu_q(d)}{\nu_q(d)+1} \sigma_0(d) (i+1) + \sum_{j=1}^{i} 
     \frac{j}{(j+1)} \sigma_0(d) (i+1)
     \right] \\ 
     & = \sum_{p|n} \sum_{d\rvert\frac{n}{p^{\nu_p(n)}}} \sum_{q|d} \sigma_0(d) \left[ 
     \frac{\nu_p(n)(\nu_p(n)+3)}{2} \frac{\nu_q(d)}{\nu_q(d)+1} + 
     \frac{1}{12}(\nu_p(n)+1)(\nu_p(n)+2)\left(4\nu_p(n)+9-6 H_{\nu_p(n)+2}^{(1)}\right) 
     \right]. 
\end{align*} 
We will break the two key component sums into separate calculations. First, we compute that\footnote{ 
     Here, the arithmetic function $\sigma_0 \ast 1$ is multiplicative. Its value at prime powers can be 
     computed as 
     \[
     (\sigma_0 \ast 1)(p^{\alpha}) = \sum_{i=0}^{\alpha} (i+1) = \frac{(\alpha+1)(\alpha+2)}{2}, 
     \]
     where $\sigma_0(p^{\beta}) = \beta + 1$. 
}
\begin{align*} 
C_{3,1}(n) & = \sum_{p|n} \sum_{d\rvert\frac{n}{p^{\nu_p(n)}}} 
     \frac{\nu_p(n)(\nu_p(n)+3)}{2} \times \sum_{q|d} \frac{\nu_q(d)}{\nu_q(d)+1} \sigma_0(d) \\ 
     & = \sum_{\substack{p,q|n \\ p \neq q}} \sum_{d\rvert\frac{n}{p^{\nu_p(n)}q^{\nu_q(n)}}} 
     \frac{\nu_p(n)(\nu_p(n)+3)}{2} \times \sum_{i=1}^{\nu_q(n)} \frac{\nu_q(dq^i)}{\nu_q(dq^i)+1} 
     \sigma_0(dq^i) \\ 
     & = \sum_{\substack{p,q|n \\ p \neq q}} \sum_{d\rvert\frac{n}{p^{\nu_p(n)}q^{\nu_q(n)}}} 
     \frac{\nu_p(n)(\nu_p(n)+3)\nu_q(n)(\nu_q(n)+3)}{4}\sigma_0(d) \\ 
     & = (\sigma_0 \ast 1)(n) \times \sum_{\substack{p,q|n \\ p \neq q}} 
     \frac{\nu_p(n)(\nu_p(n)+3)\nu_q(n)(\nu_q(n)+3)}{(\nu_p(n)+1)(\nu_p(n)+2)(\nu_q(n)+1)(\nu_q(n)+2)}. 
\end{align*} 
Next, we have that 
\begin{align*} 
C_{3,2}(n) & = \sum_{p|n} \sum_{d\rvert\frac{n}{p^{\nu_p(n)}}} \sum_{q|d} 
     \frac{1}{12}(\nu_p(n)+1)(\nu_p(n)+2)\left(4\nu_p(n)+9-6 H_{\nu_p(n)+2}^{(1)}\right) \sigma_0(d) \\ 
     & = \sum_{\substack{p,q|n \\ p \neq q}} \sum_{d\rvert\frac{n}{p^{\nu_p(n)}q^{\nu_q(n)}}} 
     \sum_{i=1}^{\nu_q(n)} 
     \frac{1}{12}(\nu_p(n)+1)(\nu_p(n)+2)\left(4\nu_p(n)+9-6 H_{\nu_p(n)+2}^{(1)}\right) \sigma_0(d) (i+1) \\ 
     & = (\sigma_0 \ast 1)(n) \times \sum_{\substack{p,q|n \\ p \neq q}} 
     \frac{1}{6}\frac{\nu_q(n) (\nu_q(n) + 3)}{ 
     (\nu_q(n)+1)(\nu_q(n)+2)} \left(4\nu_p(n)+9-6 H_{\nu_p(n)+2}^{(1)}\right). 
\end{align*} 
Now to roughly bound the error term, e.g., the GCD of prime omega functions from the exact formula for $C_3(n)$, 
we observe that the divisor function has average order of the form: 
\[
d(n) \sim \log n + (2\gamma-1) + O\left(\frac{1}{\sqrt{n}}\right). 
\] 
Then using that $\omega(n), \Omega(n) \sim \log\log n$ (except in rare cases when 
$n$ is primorial, a power of $2$, etc.), as discussed in the next remarks, we bound the 
error as 
\begin{align*} 
C_{3,3}(n) & = -\sum_{p|n} \sum_{d\rvert\frac{n}{p^{\nu_p(n)}}} \sum_{i=1}^{\nu_p(n)} 
     \gcd\left(\Omega(d) + i, \omega(d) + 1\right) \\ 
     & = \sum_{p|n} \frac{\nu_p(n)}{\nu_p(n) + 1} O\left(\sigma_0(n) \cdot \log\log n\right) \\ 
     & = O\left(\pi(n) \cdot \log n \cdot \log\log n\right) \\ 
     & = O\left(n \cdot \log\log n\right). 
\end{align*} 
In total, we obtain that 
\begin{align} 
\label{eqn_ExactFormula_Ck3n} 
C_3(n) & = (\sigma_0 \ast 1)(n) \times \sum_{\substack{p,q|n \\ p \neq q}} 
     \frac{1}{6}\frac{\nu_q(n) (\nu_q(n) + 3)}{ 
     (\nu_q(n)+1)(\nu_q(n)+2)} \left(4\nu_p(n)+9-6 H_{\nu_p(n)+2}^{(1)}\right)) \\ 
\notag 
     & \phantom{= \quad\ } + 
     \sigma_0(n) \times \sum_{\substack{p,q|n \\ q \neq p}} 
     \frac{2^{\nu_q(n)} \nu_p(n) (\nu_p(n) + 3)}{4 (\nu_p(n) + 1) (\nu_q(n) + 1)} \\ 
\notag
     & \phantom{= \quad\ } + 
     O\left(n \cdot \log\log n\right). 
\end{align} 
For the next cases, we would use similar techniques. The key is to compute enough small cases that we can see 
the dominant asymptotic terms in these expansions. We will expand more on this below. 
\end{example}

\begin{remark}[Recursive growth of the functions $C_k(n)$ in the average case]
We note that the average order of $\Omega(n) \sim \log\log n$, so that for large $x \gg 1$ tending to 
infinity, we can expect (on average) that for $p|n$, $1 \leq \nu_p(n)$ (for large $p|x, p \sim \frac{x}{\log x}$) 
and $\nu_p(n) \approx \log\log n$. However, if $x$ is primorial, we can have 
$\Omega(x) \sim \frac{\log x}{\log\log x}$. There is, however, a duality with the size of $\Omega(x)$ and the 
rate of growth of the $\nu_p(x)$ exponents. That is to say that on average, 
even though $\nu_p(x) \sim \log\log n$ for most $p|x$, if $\Omega(x) = m \approx O(1)$ is small, then 
\[
\nu_p(x) \approx \log_{\sqrt[m]{\frac{x}{\log x}}}(x) = \frac{m \log x}{\log\left(\frac{x}{\log x}\right)}. 
\]
Since we will be essentially averaging the inverse functions, $g^{-1}(n)$, via their summatory functions 
over the range $n \leq x$ for $x$ large, we tend not to worry about bounding anything but by the 
average case, which wins when we sum (i.e., average) and tend to infinity. 
Given these observations, we can use the function $C_3(n)$ we just painstakingly computed exactly 
as an asymptotic benchmark to build further approximations. In particular, the dominant order terms in 
$C_3(n)$ are given by 
\[
C_3(n) \sim \frac{(\sigma_0 \ast 1)(n) n^2}{\log^2 n} - 
     \frac{(\sigma_0 \ast 1)(n) n^2}{\log n} + 
     O\left(n \cdot \log\log n\right). 
\]
We will leave the terms involving the divisor function $\sigma_0(n)$ and convolutions 
involving it unevaluated because of how much their growth can fluctuate depending on prime 
factorizations for now. 
\end{remark} 

\begin{summary}[Asymptotics of the $C_k(n)$]
We have the following asymptotic relations relations for the growth of small cases of 
the functions $C_k(n)$: 
\begin{align*} 
C_1(n) & \sim \log\log n \\ 
C_2(n) & \sim \frac{\sigma_0(n) n}{\log n} + O(\log\log n) \\ 
C_3(n) & \sim \frac{(\sigma_0 \ast 1)(n) n^2}{\log^2 n} - 
     \frac{(\sigma_0 \ast 1)(n) n^2}{\log n} + 
     O\left(n \cdot \log\log n\right). 
\end{align*} 
Theorem \ref{theorem_Ckn_GeneralAsymptoticsForms} proved in the next section  
makes precise what these formulas suggest about the growth rates of 
$C_k(n)$. 
\end{summary} 

\begin{proof}[Proof of Theorem \ref{theorem_Ckn_GeneralAsymptoticsForms}] 
We showed how to compute the formulas for the base cases in the preceeding examples 
discussed above. We can also see that $C_3(n)$ satsfies the target formula specification. 
Let's proceed by using induction with the recurrence formula from 
\eqref{eqn_Ckn_recFormula_v1} 
relating $C_k(n)$ to $C_{k-1}(n)$ for all $k \geq 1$. The strategy is to 
precisely evaluate the sums recursively, and drop the messy troublesome lower order 
terms that contribute to the nuances of the full formulas. What results is 
precise for sufficiently large $n \gg 1$ as $n \rightarrow \infty$. 
We will compute the main term formula first, then complete the proof 
by bounding the easier error term calculations. 

Suppose that $k \geq 4$. By the recurrence formula for $C_k(n)$, we have that 
\begin{align*} 
C_k(n) & \sum_{p|n} \sum_{d|np^{-\nu_p(n)}} \sum_{i=1}^{\nu_p(n)} -\frac{(dp^i)^{k-1}}{(\log(dp^i))^{k-1}} 
     \binom{i+k-1}{k-1} (\sigma_0 \ast \mathds{1}_{\ast_{k-2}})(d). 
\end{align*} 
Now to handle the inner sum, we bound by setting $\alpha \equiv \nu_p(n)$ and 
invoking \emph{Mathematica} in the form of 
\begin{align*} 
\operatorname{IC}_k(n) & = \sum_{i=1}^{\alpha} -\frac{(dp^i)^{k-1}}{(\log(dp^i))^{k-1}} 
     \binom{i+k-1}{k-1} \\ 
     & \approx \int -\frac{(dp^{\alpha})^{k-1}}{(\log(dp^{\alpha}))^{k-1}} 
     \binom{\alpha+k-1}{k-1} \\ 
     & \sim \frac{1}{(k-1)! \log^k p} \left( 
     \operatorname{Ei}((k-2) \log(dp^{\alpha})) \left[
     \log^{k-1}(d) - (k-1)! \log^{k-1}(p)\right] 
     \right) \\ 
     & \phantom{\approx\qquad\ } - 
     \frac{1}{(k-2) (k-1)! \log^k p} \left( 
     \log^{k-2}(d) + \alpha^{k-2} \log^{k-2}(p) 
     \right). 
\end{align*} 
We now simplify somewhat again by setting
$$p \mapsto \left(\frac{n}{e}\right)^{\frac{1}{\log\log n}}, \alpha \mapsto \log\log n, 
  \log p \mapsto \frac{\log n}{\log\log n}.$$ 
Also, since $p \gg_{n} d$, we obtain the dominant asymptotic growth terms of 
\begin{align*} 
\operatorname{IC}_k(n) & \sim \frac{\alpha^{k-2}}{(k-2) (k-1)! \log^2 p} \\ 
     & \approx \frac{(\log\log n)^k}{(k-2) (k-1)! \log^2 n}. 
\end{align*} 
Now, as we did in the previous example work, we handle the sums by pulling out a factor of the inner 
divisor sum depending only on $n$ (and $k$): 
\begin{align*} 
C_k(n) & = \sum_{p|n} (\sigma_0 \ast \mathds{1}_{\ast_{k-1}})(n) 
     \binom{p^{\nu_p(n)} + k}{k}^{-1} \times \operatorname{IC}_k(n) \\ 
     & = (\sigma_0 \ast \mathds{1}_{\ast_{k-1}})(n) 
     \binom{p^{\nu_p(n)} + k}{k}^{-1} \cdot \pi(n) \times \operatorname{IC}_k(n)
\end{align*} 
Combining with the remaining terms we get by induction, we have proved our target bound holds 
for $C_k(n)$. 

To bound the error terms, again suppose inductively that $k \geq 4$. We compute the 
big-O bounds as follows letting $\alpha \equiv \nu_p(n)$: 
\begin{align*} 
\operatorname{ET}_k(n) & = 
     \sum_{i=1}^{\nu_p(n)} n^{k-2} \cdot \frac{(\log\log n)^{k-2}}{(\log n)^{k-2}} \\ 
     & \approx 
     \int (dp^{\alpha})^{k-2} \log\log(dp^{\alpha}) d\alpha \\ 
     & = -\frac{\operatorname{Ei}((k-2) \log(dp^{\alpha}))}{(k-2) \log p} + 
     \frac{d^{k-2} p^{(k-2)\alpha}}{(k-2) \log p} \log(dp^{\alpha}) \\ 
     & \sim \frac{d^{k-2} p^{(k-2)\alpha}}{(k-2) \log p} \log(dp^{\alpha}). 
\end{align*} 
In the last expansion, we have dropped the exponential integral terms since they provide at most 
polynomial powers of the logarithm of their inputs. 

To evaluate the outer divisor sum from the recurrence relation for $C_k(n)$, we will require the 
following bound providing an average order on the \emph{generalied sum-of-divisors functions}, 
$\sigma_{\alpha}(n) := \sum_{d|n} d^{\alpha}$. In particular, we have that for integers $\alpha \geq 2$ 
\cite[\S 27.11]{NISTHB}: 
\[
\sigma_{\alpha}(n) \sim \frac{\zeta(\alpha+1)}{\alpha+1} x^{\alpha} + O(x^{\alpha-1}). 
\]
Approximating the number of terms in the prime divisor sum by $\pi(x) \sim \frac{x}{\log x}$, 
we thus obtain 
\begin{align*} 
\operatorname{ET}_k(n) & \approx \frac{(\log\log n)^{k-1} e^{k-2}}{(k-1)(k-2)} 
     x^{(k-2)\left(1-\frac{1}{\log\log x}\right)+1+\log\log x} \zeta(k-1). 
\end{align*} 
So up to what is effectively constant in $k$, and dropping lower order terms for a slightly 
suboptimal, but still sufficient for our purposes, error bound formula, 
we have completed the proof by induction. 
\end{proof} 

\begin{cor}[Computing the inverse functions] 
\label{cor_ComputingInvFuncs_InPractice_DivSumgInvAst1_v1} 
In contrast to the complicated formulation given by 
Lemma \ref{lemma_AnExactFormulaFor_gInvByMobiusInv_v1}, we have that 
\[
g^{-1}(n) \sim \lambda(n) \times \sum_{d|n} C_{\Omega(d)}(d). 
\]
This is to say that for all $n \geq 2$ 
\[
\left\lvert 1 - \frac{\lambda(n) \times \sum_{d|n} C_{\Omega(d)}(d)}{g^{-1}(n)} \right\rvert = 
     o\left(\sum_{d|n} C_{\Omega(d)}(d)\right). 
\]
Moreover, we can bound the error terms as 
\[
\left\lvert \frac{\lambda(n) \times \sum_{d|n} C_{\Omega(d)}(d)}{g^{-1}(n)} \right\rvert = 
     O\left(\frac{(\log\log n)^2}{\log n} \cdot 
     \frac{\Gamma(\log\log n)}{ 
     n^{\log\log n} \cdot (\log n)^{\log\log n}}\right) 
     \xrightarrow{n \rightarrow \infty} 0. 
\]
\end{cor} 
\begin{proof} 
Using Lemma \ref{lemma_AnExactFormulaFor_gInvByMobiusInv_v1}, it suffices to show that 
the squarefree divisors $d|n$ such that $\operatorname{sgn}(\mu(d) \lambda(n/d)) = -1$ 
have an order of magnitude less adbundancy than the corresponding cases of positive sign on 
the terms in the divisor sum from the lemma. Let $n$ have $m_1$ prime factors $p_1$ such that 
$v_{p_1}(n) = 1$, $m_2$ such that $v_{p_2}(n) = 2$, and the remaining 
$m_3 := \Omega(n) - m_1 - 2m_2$ prime factors of higher-order exponentation. 
We have a few cases to consider after re-writing the sum from the lemma in the following form: 
\[
g^{-1}(n) = \lambda(n) C_{\Omega(n)}(n) + \sum_{i=1}^{\omega(n)} \left\{
     \sum_{\substack{d|n \\ \omega(d) = \Omega(d) = i \\ \#\{p|d:\nu_p(d) = 1\} = k_1 \\ 
     \#\{p|d:\nu_p(d) = 2\} = k_2 \\ \#\{p|d:\nu_p(d) \geq 3\} = k_3}} 
     \mu(d) \lambda(n/d) C_{\Omega(n/d)}(n/d) \right\}. 
\]
We obtain the following cases of the squarefree divisors contributing to the signage on the 
terms in the above sum: 
\begin{itemize} 
\item The sign of $\mu(d)$ is $(-1)^{i} = (-1)^{k_1+k_2+k_3}$; 
\item If $m_3 < \#\{p|n: \nu_p(n) \geq 3\}$, then $\lambda(n/d) = 1$ (since $\mu(n/d) = 0$); 
\item Given $(k_1, k_2, k_3)$ as above, since $\lambda(n) = (-1)^{\Omega(n)}$, we have that 
      $\mu(d) \cdot \lambda(n/d) = (-1)^{i-k_1-k_2} \lambda(n)$. 
\end{itemize} 
Thus we define the following sums, parameterized in the $(m_1,m_2,m_3; n)$, which corresponds to a 
change in expected parity transitioning from the M\"obius inversion sum from 
Lemma \ref{lemma_AnExactFormulaFor_gInvByMobiusInv_v1} to the 
sum approximating $g^{-1}(n)$ defined at the start of this result: 
\begin{align*} 
\widetilde{S}_{\operatorname{odd}}(m_1, m_2, m_3; n) & := 
     \sum_{i=1}^{\omega(n)/2} \sum_{k_1=0}^{\Floor{i}{2}} \sum_{k_2=0}^{\Floor{i}{2}-k_1} \left[
     \binom{m_1}{2k_1+1} \binom{2m_2}{2k_2+1} + \binom{m_1}{2k_1} \binom{2m_2}{2k_2}
     \right] \Iverson{i-k_1-k_2 = k_3 \equiv m_3} \\ 
\widetilde{S}_{\operatorname{even}}(m_1, m_2, m_3; n) & := 
     \sum_{i=1}^{\omega(n)/2} \sum_{k_1=0}^{\Floor{i}{2}} \sum_{k_2=0}^{\Floor{i}{2}-k_1} \left[
     \binom{m_1}{2k_1} \binom{2m_2}{2k_2+1} + \binom{m_1}{2k_1+1} \binom{2m_2}{2k_2}
     \right] \Iverson{i-k_1-k_2 = k_3 \equiv m_3}. 
\end{align*} 
\textit{Part I (Lower bounds on the inner sums of the count functions). } 
We claim that 
\begin{align}
\label{eqn_SoddSeven_lower_bounds_v1} 
\widetilde{S}_{\operatorname{odd}}(m_1, m_2, m_3; n) & \gg 
     \binom{m_1}{i+1} + \binom{m_1}{\frac{i}{2}} \binom{2m_2-1}{\frac{i}{2}+1} \\ 
\notag 
\widetilde{S}_{\operatorname{even}}(m_1, m_2, m_3; n) & \gg 
     \binom{m_1}{i+1} + \binom{m_1}{\frac{i}{2}-1} \binom{2m_2}{\frac{i}{2}+1}. 
\end{align} 
To prove \eqref{eqn_SoddSeven_lower_bounds_v1} we have to provide a straightforward bound that 
represents the maximums of the terms in $m_1,m_2$. In particular, observe that for 
\begin{align*} 
\widetilde{S}_{\operatorname{odd}}(m_1, m_2; u) & = 
     \sum_{k_1=0}^{u} \sum_{k_2=0}^{u-k_1} \left[\binom{m_1}{2k_1} \binom{2m_2}{2k_2+1} + 
     \binom{m_1}{2k_1+1} \binom{2m_2}{2k_2}\right] \\ 
\widetilde{S}_{\operatorname{even}}(m_1, m_2; u) & = 
     \sum_{k_1=0}^{u} \sum_{k_2=0}^{u-k_1} \left[\binom{m_1}{2k_1+1} \binom{2m_2}{2k_2+1} + 
     \binom{m_1}{2k_1} \binom{2m_2}{2k_2}\right], 
\end{align*} 
we have that 
\begin{align*} 
\widetilde{S}_{\operatorname{odd}}(m_1, m_2; u) & \succsim \binom{m_1}{2u+1} + 
     \max_{1 \leq k_1 \leq u} \binom{m_1}{2k_1+1} \binom{2m_2}{2u+1-2k_1} \\ 
     & = \binom{m_1}{2u+1} + \binom{m_1}{2k_1+1} \binom{2m_2}{2u+1-2k_1} \Biggr\rvert_{k_1=\frac{u}{2}} \\ 
     & =  \binom{m_1}{2u+1} + \binom{m_1}{u+1} \binom{2m_2}{u+1} \\ 
\widetilde{S}_{\operatorname{even}}(m_1, m_2; u) & \succsim \binom{m_1}{2u+1} + 
     \max_{1 \leq k_1 \leq u} \binom{m_1}{2k_1} \binom{2m_2}{2u+1-2k_1} \\ 
     & = \binom{m_1}{2u+1} + \binom{m_1}{u-1} \binom{2m_2}{u+1}.  
\end{align*} 
The lower bounds in \eqref{eqn_SoddSeven_lower_bounds_v1} then follow by setting 
$u \equiv \Floor{i}{2}$. \\ 
\textit{Part II (Bounding $m_1,m_2,m_3$ and effective $(i, k_1, k_2)$ contributing to the count). } 
We thus have to determine the asymptotic growth rate of 
$\widetilde{S}_{\operatorname{odd}}(m_1, m_2, m_3; n) + \widetilde{S}_{\operatorname{even}}(m_1, m_2, m_3; n)$, 
and show that it is of comparatively small order. First, we bound the count of non-zero $m_3$ for 
$n \leq x$ from below. 
For the cases where we expect differences in 
signage, it's the last Iverson convention term that kills the order of growth, e.g., we expect differences 
when the parameter $m_3$ is larger than the usual configuration. 
We know that 
\[
\pi_k(x) \sim \frac{x}{\log x} \frac{(\log\log x)^{k-1}}{(k-1)!}. 
\]
Using the formula for $\pi_k(x)$, we can count the average orders of $m_1,m_2$ as 
\begin{align*}
N_{m_1}(x) & \approx \frac{1}{x} \#\{n \leq x: \omega(n) = 1\} \sim \frac{\log\log x}{\log x} \\ 
N_{m_2}(x) & \approx \frac{1}{x} \#\{n \leq x: \omega(n) = 2\} \sim \frac{(\log\log x)^2}{\log x}. 
\end{align*} 
Additionally, in Corollary \ref{cor_BoundsOnGz_FromMVBook_initial_stmt_v1} on 
page \pageref{cor_BoundsOnGz_FromMVBook_initial_stmt_v1} 
we will prove a lower bound on $\widehat{\pi}_k(x)$. We use this result immediately 
below without proof. 

When we have parameters with respect to some $n \geq 1$ 
such that $m_3 > 0$, it must be the case that 
\[
\Omega(n) - \omega(n) > \begin{cases} 
     0, & \text{ if $\omega(n) \geq 2$; } \\ 
     1, & \text{ if $\omega(n) = 1$. } 
     \end{cases}
\]
To count the number of cases $n \leq x$ where this happens, we form the sums 
\begin{align*} 
N_{m_3}(x) & \gg \pi_1(x) \times \sum_{k=3}^{\frac{3}{2} \log\log x} \widehat{\pi}_k(x) + 
     \sum_{k=2}^{\frac{3}{2} \log\log x} \sum_{j=k+1}^{\frac{3}{2} \log\log x} 
     \pi_k(x) \widehat{\pi}_j(x) \\ 
     & \succsim \frac{Ae}{B} \frac{x}{\log^{\frac{13}{14}}(x)} + 
     \sum_{k=2}^{\log\log x} \pi_k(x) \left[ 
     \frac{Ae}{B} \log^{\frac{1}{14}}(x) 
     \right] \\ 
     & \succsim \frac{Ae}{B} \frac{x}{\log^{\frac{13}{14}}(x)} + 
     \frac{Ae \sqrt{2}}{2 \sqrt{\pi} B} \frac{x}{\log^{\frac{13}{14}}(x) \sqrt{\log\log x}}. 
\end{align*} 
Now in practice, we are not summing up $n \leq x$, but rather $n \leq \log\log x$. So the 
above function evaluates to 
\[
N_{m_3}(\log\log x) \gg \frac{\log\log x}{(\log\log\log x)^{13/14}} \gg 
     \frac{\log\log x}{\log\log\log x}. 
\]
Next, we go about solving the subproblem of finding when $i-k_1-k_2 = m_3$. First, we find a lower 
solution index on $i$ using asymptotics for the \emph{Lambert $W$-function}, 
$W_0(x) = \log x - \log\log x + o(1)$: 
\begin{align*} 
\frac{i}{2} = \frac{\log\log x}{\log\log\log x} & \iff \log\log x \succsim \frac{i}{2}\left( 
     \log i + \log\log i\right) \\ 
     & \iff \frac{i}{2} \sim \frac{\log\log x}{\log\log\log x}. 
\end{align*} 
Now since $2 \leq k_1+k_2 \leq i/2$, when $x$ is large, we actually obtain a number of 
soultions on the order of 
\[
\frac{\log\log x}{2} - \frac{\log\log x}{\log\log\log x} = \frac{\log\log x}{2} (1+o(1)). 
\]
\textit{Part III (Putting it all together). } 
Using the binomial coefficient inequality 
\[
\binom{n}{k} \geq \frac{n^k}{k^k}, 
\]
we can work out carefully on paper using \eqref{eqn_SoddSeven_lower_bounds_v1} that 
\begin{align*} 
\widetilde{S}_{\operatorname{odd}}(m_1, m_2, m_3; n) & \succsim 
     \frac{\log\log x}{2} \left(\frac{\log\log\log x}{2 \log x}\right)^{\frac{2\log\log x}{\log\log\log x}+1} 
     \left[1 + \frac{(\log\log\log x)^2}{\log^2 x} 
     (4 \log\log x \cdot \log\log\log x)^{\frac{\log\log x}{2 \log\log\log x} + 1}\right] \\ 
\widetilde{S}_{\operatorname{odd}}(m_1, m_2, m_3; n) & \succsim 
     \frac{\log\log x}{2} \left(\frac{\log\log\log x}{2 \log x}\right)^{\frac{2\log\log x}{\log\log\log x}+1} 
     \left[1 + \left(\frac{\log x}{2 \log\log\log x}\right) 
     (8 \log\log x)^{\frac{\log\log x}{\log\log\log x} + 1}\right]. 
\end{align*} 
\textit{Part IV (Obtaining the rate at which the ratio goes to zero). } 
Let 
\[
S_{\operatorname{diff}}(m_1, m_2, m_3; n) := 
     S_{\operatorname{even}}(m_1, m_2, m_3; n) + S_{\operatorname{odd}}(m_1, m_2, m_3; n). 
\]
Then we have that 
\begin{align*} 
\left\lvert \frac{\lambda(x) \times \sum_{d|x} C_{\Omega(d)}(d)}{g^{-1}(x)} \right\rvert & = 
     \frac{\sum\limits_{\substack{d|x \\ d \leq \log\log x}} C_{\Omega(d)}(d)}{(\log\log d)}{|g^{-1}(x)|} = \\ 
     & = O\left(\frac{S_{\operatorname{diff}}\left(\frac{\log\log x}{\log x}, \frac{(\log\log x)^2}{\log x}, 
     \frac{(\log\log x)^2}{\log\log\log x}; x\right)}{C_{\Omega(x)}(x)}\right) \\ 
     & = O\left(\frac{\log x \cdot \log\log x}{\log\log\log x \cdot C_{\Omega(x)}(x)} 
     \left(\frac{\sqrt{2 \log\log x} \cdot \log\log\log x}{\log x} 
     \right)^{\frac{2\log\log x}{\log\log\log x}}\right) \\ 
     & = O\left(\frac{(\log\log x)^2 \cdot \log x}{\log\log\log x \cdot C_{\Omega(x)}(x)}\right). 
\end{align*} 
We borrow from Corollary \ref{cor_Asymptotics_KeyCases_Of_Ckn_v1} 
proved below in this section to get a lower bound on $C_{\Omega(x)}(x)$. 
This result implies the stated bound, which tends to zero as $x \rightarrow \infty$. 
Thus the divisor sum in the corollary statement accuarately approximates the main term and 
sign of $g^{-1}(n)$ as $n \rightarrow \infty$. 
\end{proof} 

\begin{cor} 
\label{cor_ASemiForm_ForGInvx_v1} 
We have that for sufficiently large $x$, as $x \rightarrow \infty$ that 
\begin{align*} 
G^{-1}(x) & \succsim \widehat{L}_0(\log\log x) \times \sum_{n \leq \log\log x} 
     \lambda(n) \cdot C_{\Omega(n)}(n), 
\end{align*} 
where the function 
\[
\widehat{L}_0(\log\log x) := (-1)^{\floor{\frac{3}{2} \log\log\log\log x} + 1} \left\{ 
     \sqrt{\frac{3}{\pi}} \frac{A (2e+3)}{4B \log^{\frac{3}{2}}(2)}\right\} \cdot 
     \frac{(\log\log\log x)^{\frac{43}{14} + \frac{3}{2 \log 2} - \frac{3}{2 \log 3}}}{ 
     \sqrt{\log\log\log\log x}}, 
\]
with the exponent $\frac{43}{14} + \frac{3}{2 \log 2} - \frac{3}{2 \log 3} \approx 3.87011$. 
\end{cor} 
\begin{proof} 
Using Corollary \ref{cor_ComputingInvFuncs_InPractice_DivSumgInvAst1_v1}, we have that 
\begin{align*} 
G^{-1}(x) & \approx \sum_{n \leq x} \lambda(n) \cdot (g^{-1} \ast 1)(n) \\ 
     & = \sum_{d \leq \log\log x} C_{\Omega(d)}(d) \times \sum_{n=1}^{\Floor{x}{d}} \lambda(dn). 
\end{align*} 
Now we see that by complete additivity (multiplicativity) of $\Omega$ ($\lambda$) that 
\begin{align*} 
\sum_{n=1}^{\Floor{x}{d}} \lambda(dn) & = \sum_{n=1}^{\Floor{x}{d}} \lambda(d) \lambda(n) 
     = \lambda(d) \sum_{n \leq \Floor{x}{d}} \lambda(n). 
\end{align*} 
Borrowing a result from the next sections 
(proved in Section \ref{Section_MVCh7_GzBounds}), 
we can establish that 
\begin{align*} 
\sum_{n \leq x} \lambda(n) & \gg \sum_{n \leq \frac{3}{2} \log\log x} (-1)^k \cdot \widehat{\pi}_k(x) \\ 
     & \succsim (-1)^{\floor{\frac{3}{2} \log\log x} + 1} \left( 
     \sqrt{\frac{3}{\pi}} \frac{A (2e+3)}{4B \log^{\frac{3}{2}}(2)}\right) \cdot 
     \frac{(\log x)^{\frac{43}{14} + \frac{3}{2 \log 2} - \frac{3}{2 \log 3}}}{ 
     \sqrt{\log\log x}} \\ 
     & =: \widehat{L}_0(x). 
\end{align*} 
Then since for large enough $x$ and $d \leq x$, 
\[
\log(x/d) \sim \log x, \log\log(x/d) \sim \log\log x, 
\] 
we can obtain the stated result, e.g., so that 
$\widehat{L}_0(x) \sim \widehat{L}_0(x/d)$ for large $x \rightarrow \infty$. 
\end{proof} 

The previous corollary is employed to prove the exact lower bounds on 
$G^{-1}(x)$ given in Theorem \ref{theorem_gInv_GeneralAsymptoticsForms} in the next section. 
The parity of $\floor{2\log\log\log\log x}$ determines subsequences of real $x \gg 1$ 
along which we break these bounds into cases. 
The next result provides complete asymptotic upper and lower bound 
information on the functions $C_k(n)$ when $k \equiv \Omega(n)$. 

\begin{cor}[Asymptotics for very special case of the functions $C_k(n)$] 
\label{cor_Asymptotics_KeyCases_Of_Ckn_v1} 
For $k \gg 1$ sufficiently large, we have that 
\[
C_{\Omega(n)}(n) \sim (\sigma_0 \ast \mathds{1}_{\ast_{\log\log n-2}})(n) \times \lambda(n) 
     \frac{n^{\log\log n -1}}{(\log n)^{\log\log -1} \Gamma(\log\log n)}. 
\]
Moreover, by considering the average orders of the function $\nu_p(n)$ for $p$ large and 
tending to infinity, we have bounds on the aysmptotic behavior of these functions 
of the form 
\[
\lambda(n)\widehat{\tau}_0(n) \precsim C_{\Omega(n)}(n) \precsim \lambda(n)\widehat{\tau}_1(n). 
\]
It suffices to take the functions 
\begin{align*} 
\widehat{\tau}_0(n) & := \frac{1}{\log 2} \cdot \frac{\log n}{(\log n)^{\log\log n}} \cdot 
     \frac{n^{\log\log n-1}}{\Gamma(\log\log n)} \\ 
\widehat{\tau}_1(n) & := \frac{1}{2e \log 2} \cdot \frac{(\log n)^2}{(\log n)^{\log\log n}} \cdot 
     \frac{n^{\log\log n}}{\Gamma(\log\log n)}. 
\end{align*} 
\end{cor} 
\begin{proof} 
The first stated formula follows from 
Theorem \ref{theorem_Ckn_GeneralAsymptoticsForms} by setting 
$k := \Omega(n) \sim \log\log n$ and simplifying. We evaluate the Dirichlet convolution 
functions and approximate as follows: 
\begin{align*} 
(\sigma_0 \ast \mathds{1}_{\log\log n-2})(n) & = \sum_{p|n} \binom{\nu_p(n) + \log\log n-1}{\log\log n-1} \\ 
     & \geq \sum_{p|n} \frac{(\nu_p(n) + \log\log n-1)^{\log\log n-1}}{(\log\log n)^{\log\log n-1}} \\ 
     & \sim \frac{n}{\log 2} \\ 
(\sigma_0 \ast \mathds{1}_{\log\log n-2})(n) & \leq \left(
     \frac{(\nu_p(n) + \log\log n-1)e}{\log\log n-1} 
     \right)^{\log\log n -1} \\ 
     & \sim (2e)^{\log\log n-1} \\ 
     & = \frac{n \cdot \log n}{2e \log 2}.    
\end{align*} 
The upper and lower bounds are obtained from the next well known binomial coefficient approximations 
using Stirling's formula. 
\[
\frac{n^k}{k^k} \leq \binom{n}{k} \leq \frac{n^k}{k!} < \left(\frac{ne}{k}\right)^k 
     \qedhere 
\]
\end{proof} 

Now that we have accurate asymptotic bounds on $|g^{-1}(n)|$ as $n \rightarrow \infty$, we must form the 
summatory functions $G^{-1}(x)$ of $g^{-1}$ whose terms vary widely when including the parity of 
$\Omega(n)$ (sign of $\lambda(n)$). The natural mechanism for this is to employ Abel summation. 
However, we do not yet have a sufficient grasp on the summatory functions, 
$A_{\Omega}(x)$, that indicate the sign shifts of $\lambda(n)$ for $n \leq x$. 
To effectively bound these functions for large $x$, we will require asymptotic lower bounds 
on $\widehat{\pi}_k(x)$ for $k \geq 1$ and $k$ bounded high enough above (with respect to $x$) so that 
the resulting functions $A_{\Omega}(x)$ are asymptotically accurate. 

\newpage
\section{Key applications: Establishing lower bounds for $M(x)$ by cases along infinite subsequences} 
\label{Section_KeyApplications} 

\subsection{The culmination of what we have done so far} 

As noted before in the previous subsections, we cannot hope to evaluate
functions weighted by $\lambda(n)$ except for on 
average using Abel summation. For this task, 
we need to know the bounds on $\widehat{\pi}_k(x)$ we developed in the 
proof of Corollary \ref{cor_BoundsOnGz_FromMVBook_initial_stmt_v1}. 
A summation by parts argument shows that 
\begin{align} 
\notag
M(x) & = \sum_{k=1}^{x} g^{-1}(k) (\pi(x/k)+1) \\ 
\label{eqn_pf_tag_v2-restated_v2} 
     & \approx G^{-1}(x) - \sum_{k=1}^{x/2} G^{-1}(k) \cdot \frac{x}{k^2 \log(x/k)}. 
\end{align} 
Thus it suffices for us to compute the effective \emph{average order} of $g^{-1}(n)$ 
by summing its summatory function, $G^{-1}(n)$, including absorbing the parity of the 
$\lambda(n)$ terms into the parity of the $\Omega(n)$ terms we are summing over. 
The result in Lemma \ref{lemma_CLT_and_AbelSummation} is key to justifying the 
asymptotics obtained next in Theorem \ref{theorem_gInv_GeneralAsymptoticsForms}. 

To simplify notation, for integers $m \geq 1$, let the \emph{iterated logarithm function} 
(not to be confused with powers of $\log x$) be defined for $x > 0$ by 
\[
\log_{\ast}^{m}(x) := \begin{cases} 
     x, & \text{ if $m = 0$; } \\ 
     \log x, & \text{ if $m = 1$; } \\ 
     \log\left(\log_{\ast}^{m-1}(x)\right), & \text{ if $m \geq 2$. } 
     \end{cases}
\]
So $\log_{\ast}^2(x) = \log\log x$, $\log_{\ast}^3(x) = \log\log\log x$, 
$\log_{\ast}^4(x) = \log\log\log\log x$, $\log_{\ast}^5(x) = \log\log\log\log\log x$, and so on. 
This notation will come in handy to abbreviate the dominant asymptotic terms we find next in 
Theorem \ref{theorem_gInv_GeneralAsymptoticsForms}. 

We use the result of 
Corollary \ref{cor_Asymptotics_KeyCases_Of_Ckn_v1} and 
Corollary \ref{cor_BoundsOnGz_FromMVBook_initial_stmt_v1} 
to prove the following central theorem: 

\begin{theorem}[Asymptotics and bounds for the summatory functions $G^{-1}(x)$] 
\label{theorem_gInv_GeneralAsymptoticsForms}
We define the lower summatory function, $G_u^{-1}(x)$, 
to provide bounds on the magnitude of $G^{-1}(x)$: 
$$|G_{\ell}^{-1}(x)| \ll |G^{-1}(x)|,$$ for all sufficiently large $x \gg 1$. 
We have the following asymptotic approximations for the lower summatory function where 
$C_{\ell,1}, C_{\ell,2}$ are absolute constants defined by 
\[
C_{\ell,1} = \frac{3}{16} \sqrt{\frac{3}{2}} \frac{A_0^2 (2e+3)^2}{\pi e B^2 (\log 2)^3}, 
     C_{\ell,2} = \frac{27 A_0^2 (2e+3)^3}{128 \pi^{3/2} B^2 (\log 2)^3}, 
\]
and $\widehat{L}_0(x)$ is the multiplier function from 
Corollary \ref{cor_ASemiForm_ForGInvx_v1}: 
\begin{align*} 
 & \left\lvert G_{\ell}^{-1}\left(x\right) \right\rvert
     \succsim \\ 
     & \phantom{\succsim\ } 
     \Biggl\lvert 
     (-1)^{\floor{\frac{3}{2} \log_{\ast}^4(x)}} C_{\ell,1} \cdot (\log x)^{\frac{11}{7}} 
     (\log\log x)^{\frac{71}{14} + \frac{3}{2\log 2} - \frac{3}{2\log 3} - \log_{\ast}^4(x)} 
     \log_{\ast}^3(x)^{1 + \frac{3}{2} \log\log x + \log_{\ast}^4(x)} 
     \log_{\ast}^4(x)^{\log_{\ast}^4(x) - \frac{1}{2}} \\ 
     & \phantom{\succsim\Biggl\lvert\ } - 
     (-1)^{\floor{2\log_{\ast}^4(x)}} C_{\ell,2} \cdot 
     \frac{\log_{\ast}^3(x)^{\frac{9}{2} + \frac{25}{6} \log 2 + \frac{3}{2 \log 2} 
     - \frac{4}{3} \log 3 - \frac{3}{2 \log 3}}}{\sqrt{\log_{\ast}^4(x)}} 
     \log_{\ast}^5(x)^{\frac{11}{7} + \frac{3}{2} \log_{\ast}^7(x)} 
     \Biggr\rvert. 
\end{align*} 
\end{theorem} 
\begin{proof}[Proof Sketch: Logarithmic scaling to the accurate order of the inverse functions] 
For the sums given by 
\begin{align*} 
S_{g^{-1}}(x) := \sum_{n \leq x} \lambda(n) \cdot C_{\Omega(n)}(n), 
\end{align*} 
we notice that using the asymptotic bounds (rather than the exact formulas) for the functions 
$C_{\Omega(n)}(n)$, we have over-summed by quite a bit. 
In particular, following from the intent behind the constructions in the last sections, 
we are really summing only over all $n \leq x$ with $\Omega(n) \leq x$. 
Since $\Omega(n) \leq \floor{\log_2 n} = \Floor{\log n}{\log 2}$, 
many of the terms in the previous equation are actually zero (recall that $C_0(n) = \delta_{n,1}$). 
So we are actually only going to sum up to the average order of 
$\Omega(n) \sim \log\log n$ in practice, or to the slightly larger bound if the leading sign term on 
$G_{\ell}^{-1}(x)$ is negative. 
Hence, the sum (in general) that we are really interested in bounding is 
bounded below in magnitude by $S_{g^{-1}}(\log\log x)$ or 
$S_{g^{-1}}(\log_2(x))$, where we can now safely apply the 
asymptotic formulas for the $C_k(n)$ functions from 
Corollary \ref{cor_Asymptotics_KeyCases_Of_Ckn_v1} 
that hold once we have verified these constraints. 
\end{proof} 
\begin{proof}[Proof (Lower Bounds)] 
Recall from our proof of Corollary \ref{cor_BoundsOnGz_FromMVBook_initial_stmt_v1} that 
a lower bound on the function $\widehat{\pi}_k(x)$ is given by $G\left(\frac{k-1}{\log\log x}\right)$ 
where the function $G(z)$ is bounded below by 
\[
G(z) \gg A_0 x \frac{(\log\log x)^{k-1}}{(k-1)!} \left(\frac{\log x}{\log 2}\right)^{z} \log^2 x \left( 
     1 - \frac{z}{B} \log^{\frac{1}{14}}(x)\right). 
\]
Thus we can form a lower summatory function indicating the parity of all 
$\Omega(n)$ for $n \leq x$ as 
\begin{align} 
\notag 
A_{\Omega}^{(\ell)}(t) & = \sum_{k \leq \frac{3}{2}\log\log t} (-1)^k G\left(\frac{k-1}{\log\log t}\right) \\ 
\label{proof_thm_GInvFunc_v0} 
     & \sim (-1)^{1+\Floor{3\log\log t}{2}} \cdot 
     \frac{3A_0}{4eB \log^{\frac{3}{2}}(2) \Gamma\left(1 + \frac{3}{2}\log\log t\right)} \left( 
     (2e+3) \log^{\frac{1}{14}}(t) - 2B\right) 
     \log^{\frac{3}{2}}(t) (\log\log t)^{\frac{3}{2}\log\log t}. 
\end{align} 
Next, as in Lemma \ref{lemma_CLT_and_AbelSummation}, we apply Abel summation to obtain that  
\begin{equation} 
\label{proof_thm_GInvFunc_v1} 
G_{\ell}^{-1}(x) = \widehat{\tau}_0(\log\log x) A_{\Omega}^{(\ell)}(x) - 
     \widehat{\tau}_0(u_0) A_{\Omega}^{(\ell)}(u_0) - \int_{u_0}^{\log\log x} 
     \widehat{\tau}_0^{\prime}(t) A_{\Omega}^{(\ell)}(t) dt, 
\end{equation} 
where we define the integrand function, 
$I_{\ell}(t) := \widehat{\tau}_0^{\prime}(t) A_{\Omega}^{(\ell)}(t)$, 
with some limiting simplifications as 
\begin{align*} 
I_{\ell}\left(e^{e^{\frac{4k}{3}}}\right) e^{e^{\frac{4k}{3}}} & = 
     \frac{4A_0 4^{2 k-1} 9^{-k} k^{2k} \left((3+2 e) e^{2k/21}-2 B\right) ]
     \exp \left(-\frac{16 k^2}{9}+2k+e^{4k/3} 
     \left(\frac{4k}{3}-1\right)-1\right)}{3B \log ^{\frac{5}{2}}(2)} \times \\ 
     & \phantom{=\ } \times 
     \left(4 e^{4 k/3} k-8k-3 \log k-3 \gamma +6+3 
     \log 3-6 \log 2\right).
\end{align*} 
The integration term in \eqref{proof_thm_GInvFunc_v1} is summed approximately as follows: 
\begin{align*} 
\int_{u_0-1}^{\log\log x} \widehat{\tau}_0^{\prime}(t) A_{\Omega}^{(\ell)}(t) dt & \sim 
     \sum_{k=u_0+1}^{\frac{1}{2}\log\log\log\log x} \left( 
     \frac{I_{\ell}\left(e^{e^{\frac{4k+2}{3}}}\right)}{(2k)! \left(\frac{4k}{3}\right)!} - 
     \frac{I_{\ell}\left(e^{e^{\frac{4k}{3}}}\right)}{(2k)! \left(\frac{4k}{3}\right)!}
     \right) e^{e^{\frac{4k}{3}}} \\ 
     & \approx 
     C_0(u_0) + 
     (-1)^{\Floor{\log\log\log\log x}{2}} \times 
     \int_{\frac{\log\log\log\log x}{2}-\frac{1}{2}}^{\frac{\log\log\log\log x}{2}} 
     \frac{I_{\ell}\left(e^{e^{\frac{4k}{3}}}\right)}{(2k)! \left(\frac{4k}{3}\right)!} 
     e^{e^{\frac{4k}{3}}} dk. 
\end{align*} 
The differences on the upper and lower bounds on each integral in the last equation 
is small, and in particular $\frac{1}{2} \lll \log\log x$. 
So we can use a small perturbation of $+1$ in the power terms of $I_{\ell}(t)$ combined with 
an appeal to the binomial series, the expansion of binomial coefficients by the Stirling numbers 
of the first kind, and the following exact indefinite integral for 
$x,z \in \mathbb{R}$ moving forward: 
\begin{align*} 
\int t^p e^{ct} dt & = \frac{(-1)^p}{c^{p+1}} \Gamma(p+1, -ct) \sim 
     \frac{e^{ct} t^p}{c}. 
\end{align*} 
Define the following function of $t$ and note the change of variable $t \mapsto \frac{k-1}{2}$: 
\begin{align*} 
I_{\ell}\left(e^{e^{\frac{4k}{3}}}\right) e^{e^{\frac{4k}{3}}} & = 
     (1+k)^{2k} \exp\left(-\frac{16k^2}{9} \left(\frac{4k}{3}-1\right) e^{\frac{4k}{3}}\right) 
     e^{2k-1} \widehat{f}(t_0). 
\end{align*} 
So we take one reciprocal factor in the next integrand, and set the remaining powers of $t^p$ to be 
$t_0^p$ for $t_0$ a bound of integration which results in a lower bound on our target integrand from 
Abel summation. 

From this perspective, we obtain using the exponential generating functions for the 
Stirling numbers of the first kind that \cite[\S 7.4]{GKP}\footnote{ 
     Namely, that for natural numbers $j \geq 0$ 
     \[
     \sum_{k \geq 0} \gkpSI{k}{j} \frac{z^k}{k!} = \frac{(-1)^j}{j!} \operatorname{Log}(1-z)^j. 
     \]
} 
\begin{align*}
\widehat{T}_{\ell}(t_0; t) & = \int \widehat{I}_{\ell}(t) dt \\ 
     & \gg \sum_{m \geq 0} \sum_{n \geq 0} \sum_{q \geq 0} \sum_{j \geq 0} \sum_{r \geq 0} 
     \frac{(-1)^{m+q+j+r}}{m! n! q! j!} \left(\frac{4}{3}\right)^{2m+n} \gkpSI{j}{r} 
     \left\{\int 
     t^{2m+n+j+r} \exp\left(\left(2 + \frac{4}{3}(n+q)\right) t\right) dt 
     \right\} \frac{\widehat{f}(t_0)}{e} \\ 
     & \succsim 
     -\frac{3 \widehat{f}(t_0)}{4e} e^{2t} e^{-\frac{16 k^2}{9}} \left( 
     \gamma + \frac{e^{te^{\frac{4t}{3}}}}{t e^{\frac{4t}{3}}} + \frac{4t}{3}
     \right) \left(
     \gamma + \frac{e^{te^{\frac{4t}{3}}}}{k e^{\frac{4t}{3}}} - \frac{4t}{3} 
     \right) t^{2t} 
\end{align*} 
In the previous equation, we have used that $(n+q+12)^{-1} \succsim \frac{1}{nq}$ and that 
for large $x \ggg 1$ tending to infinity 
\[
\sum_{m \geq 1} \frac{(-x)^m}{m \cdot m!} = -\left(\gamma + \Gamma(0, x) + \log x\right) \sim 
     -\left(\gamma + \frac{e^{-x}}{x} + \log x\right). 
\]
Now we can define the coefficient functions, which as multipliers above 
would have otherwise complicated our integrals, 
in the form of $\widehat{f}(t_0) = \operatorname{cf}_{+}(t_0) - \operatorname{cf}_{-}(t_0)$ as 
\begin{align*}
\operatorname{cf}_{+}(t) & := \left(\frac{16}{9}\right)^t \left(2 B (8 t+3 \gamma +
     6 \log 2)+6 B \log t+12 e^{10 t/7} t+8 e^{\frac{10 t}{7}+1} t+6 e^{\frac{2 t}{21}+1}
     (2+\log 3)+9 e^{2 t/21} (2+\log 3)\right)\\ 
\operatorname{cf}_{-}(t) & := \left(\frac{16}{9}\right)^t \left(2 B \left(4 e^{4 t/3} t+6+3 
     \log 3\right)+(3+2 e) e^{2 t/21} (8 t+3 \gamma +6 \log 2)+3 (3+2 e) e^{2 t/21}
     \log t\right). 
\end{align*} 
Let 
\[
\widehat{h}(t) := 3 \cdot 4^{-t-1} \left(\frac{3}{4}\right)^{\frac{4t}{3}} \frac{\sqrt{3}}{ 
     16\pi t^{\frac{10t}{3}+1}}. 
\]
Applying Stirling's formula again when $x$ is large, we have that 
\begin{align} 
\label{proof_thm_GInvFunc_v2}  
\widehat{R}_{\ell}(x) & = (-1)^{\Floor{\log\log\log\log x}{2}} \times 
     \int_{\frac{\log\log\log\log x}{2}-\frac{1}{2}}^{\frac{\log\log\log\log x}{2}} 
     \frac{I_{\ell}\left(e^{e^{\frac{4k}{3}}}\right)}{(2k)! \left(\frac{4k}{3}\right)!} 
     e^{e^{\frac{4k}{3}}} dk \\ 
\notag 
     & \succsim (-1)^{\Floor{x}{2}} \times \widehat{h}\left(\frac{\log\log\log\log x}{2}\right) \Biggl[ \\ 
\notag 
     & \phantom{\succsim\ - } 
     \widehat{T}_{\ell}\left(\frac{\log\log\log\log x}{2}; \frac{\log\log\log\log x}{2}\right) \left( 
     \operatorname{cf}_{+}\left(\frac{\log\log\log\log x-1}{2}\right) - 
     \operatorname{cf}_{-}\left(\frac{\log\log\log\log x}{2}\right)
     \right) \\ 
\notag 
     & \phantom{\succsim\ } - 
     \widehat{T}_{\ell}\left(\frac{\log\log\log\log x-1}{2}; \frac{\log\log\log\log x-1}{2}\right) \left( 
     \operatorname{cf}_{+}\left(\frac{\log\log\log\log x}{2}\right) - 
     \operatorname{cf}_{-}\left(\frac{\log\log\log\log x-1}{2}\right)
     \right) 
     \Biggr]. 
\end{align} 
Since for real $0 < s < 1$ such that $s \rightarrow 0$, we have that $\log(1+s) \sim s$ and 
$(1+s)^{-1} \sim 1-s$, we can approximate the differences implied by the 
last estimate using that for $t$ large tending to infinity we have 
\[
     \log_{\ast}^m\left(t - \frac{1}{2}\right) \sim \log_{\ast}^{m}(t) - 
     \frac{1}{2 \log^{m-1} t}, m \geq 1. 
\] 
Then applying these simplifications to 
\eqref{proof_thm_GInvFunc_v2} above and removing lower-order terms that 
do not contribute to the dominant asymptotic terms, we find that 
\begin{align} 
\label{proof_thm_GInvFunc_v3} 
\int_{u_0}^{\log\log x} & \widehat{\tau}_0^{\prime}(t) A_{\Omega}^{(\ell)}(t) \\ 
\notag 
     & \succsim 
      C_0(u_0) + \frac{(-1)^{\Floor{\log\log\log\log x}{2}} \cdot 
      9 \sqrt{3} A_0 (2e+3)^2}{32 B\pi e \log^{3/2}(2)} 
      (\log\log\log x)^{\frac{10}{7} + \frac{25}{6}\log 2 - \frac{4}{3} \log 3 - 
      \frac{5}{2} \log_{\ast}^5(x)} 
      \log_{\ast}^5(x)^{\frac{11}{7} + \frac{3}{2} \log_{\ast}^7(x)}. 
\end{align} 
Finally, using Stirling's formula for very large $x$ and 
\eqref{proof_thm_GInvFunc_v0}, we can see that 
\begin{align*} 
\widehat{\tau}_0(x) & \sim \frac{\log^2 x \cdot \log\log x}{\sqrt{2\pi} \cdot x} \left( 
     \frac{x}{\log x \cdot \log\log x}\right)^{\log\log x} \\ 
A_{\Omega}^{(\ell)}(x) & \sim 
     \frac{3A_0(2e+3)}{4eB \log^{\frac{3}{2}}(2)} \log^{\frac{11}{7}}(x) (\log\log x)^{\frac{3}{2}\log\log x}. 
\end{align*} 
So we have that the first terms in \eqref{proof_thm_GInvFunc_v1} 
are given by 
\begin{align*} 
\widehat{\tau}_0(\log\log x) A_{\Omega}^{(\ell)}(x) & \succsim 
     \frac{3A_0(2e+3)}{4 \sqrt{2\pi} \cdot eB \log^{\frac{3}{2}}(2)} \cdot 
     \log^{\frac{11}{7}}(x) \frac{\log\log\log x \cdot 
     (\log\log x)^{1+\frac{3\log\log x}{2}+\log\log\log\log x}}{ 
     (\log\log\log x \cdot \log\log\log\log x)^{\log\log\log\log x-1}}. 
%-\widehat{\tau}_0\left(\frac{\log x}{\log 2}\right) A_{\Omega}^{(\ell)}(x) & \succsim 
%     \frac{3A(2e+3)}{4 \sqrt{2\pi} \cdot eB \log^{\frac{3}{2}}(2)} \cdot 
%     \frac{(\log x)^{\frac{4}{7} + \log\log x - \log\log 2} \cdot \log\log x}{ 
%     (\log 2 \cdot \log\log x \cdot \log\log\log x)^{\log\log x - \log\log 2 - 1}}. 
\end{align*} 
These last formulas imply the forms of the stated bounds when we drop the lower-order 
constant term and multiply through by the bounds for the function 
$\widehat{L}_0(\log\log x)$ 
proved in Corollary \ref{cor_ASemiForm_ForGInvx_v1}. 
\end{proof} 

\subsection{Lower bounds on the scaled Mertens function along an infinite subsequence}

\begin{cor}[Bounds for the classically scaled Mertens function] 
\label{cor_ThePipeDreamResult_v1} 
Let $u_0 := e^{e^{e^{e}}}$ and define the infinite increasing subsequence, 
$\{x_n\}_{n \geq 1}$, by $x_n := e^{e^{e^{e^{6n}}}}$. 
We have that along the increasing subsequence $x_y$ for large 
$y \geq \max\left(\ceiling{e^{e^{e^{e}}}}, u_0+1\right), y \gg 3 \times 10^{1656520}$:  
\begin{align*} 
\frac{|M(x_{y})|}{\sqrt{x_{y}}} & \succsim 
     \frac{48 C_{\ell,1} x_y^{1/4}}{17} (\log x_y^{15/32})^{\frac{3}{2} 
      \log_{\ast}^4(x_y^{15/32}) - \frac{10}{7}} 
      (\log\log x_y^{15/32})^{\frac{71}{14} + \frac{3}{2 \log 2} - \frac{3}{2 \log 3}} 
      (\log\log\log x_y^{15/32}) \times \\ 
      & \phantom{\succsim\ } \times 
      \sqrt{\log\log\log\log x_y^{15/32}}  + o(1),  
\end{align*} 
as $y \rightarrow \infty$. 
\end{cor} 
\begin{proof}[Proof of the Asymptotic Lower Bound] 
It suffices to take $u_0 = e^{e^{e^{e}}}$, 
and a sufficient requirement on $x$ is that the parity of 
$\floor{\log\log\log\log x} \equiv 0 \pmod{2}$ using the formula for 
$-G_{\ell}^{-1}(t)$ proved in Theorem \ref{theorem_gInv_GeneralAsymptoticsForms}. 
Since on $x/2 \geq t \gg u_0$, we have that 
\[
\frac{d}{dt}\left[G_{\ell}^{-1}(t)\right] \succsim 
     \frac{3 C_{\ell,1}}{t} (\log t)^{\frac{3}{2} \log_{\ast}^4(t) - \frac{3}{7}} 
     (\log\log t)^{\frac{71}{14} + \frac{3}{2 \log 2} - \frac{3}{2 \log 3}} 
     (\log\log\log t) \sqrt{\log\log\log\log t}. 
\]
The input to the derivative operator in the last equation is justified by establishing the limit 
\[
\lim_{x \rightarrow \infty} (\log\log x)^{\frac{(\log\log\log\log x)^2}{\log\log x}} = 1. 
\]
We can then write \eqref{eqn_pf_tag_v2-restated_v2} in the following form using Abel summation: 
\begin{align} 
\label{eqn_pf_tag_v2-restated_v3} 
|M(x)| & \succsim \Biggl\lvert G_{\ell}^{-1}(x) + G_{\ell}^{-1}(u_0) A_{\Omega}^{(\ell)}(u_0) - 
      G_{\ell}^{-1}(x) A_{\Omega}^{(\ell)}(x) \\ 
\notag 
      & \phantom{\succsim G_{ell}^{-1}(x)\ } + 
      \left\{\int_{u_0}^{x^{1/6}} + \int_{x^{1/6}}^{x^{15/32}} + \int_{x^{15/32}}^{\sqrt{x}} + 
      \int_{\sqrt{x}}^{x/2}\right\} 
      \frac{x}{t^2 \log(x/u_0)} \frac{d}{dt}\left[G_{\ell}^{-1}(t)\right] dt 
      \Biggr\rvert \\ 
\notag 
      & \succsim \Biggl\lvert G_{\ell}^{-1}(x) - 
      G_{\ell}^{-1}(x) A_{\Omega}^{(\ell)}(x) + o(\sqrt{x}) \\ 
\notag 
      & \phantom{\succsim \Biggl\lvert\ } + 
      \frac{96 C_{\ell,1} x}{17} (\log x^{15/32})^{\frac{3}{2} 
      \log_{\ast}^4(x^{15/32}) - \frac{10}{7}} 
      (\log\log x^{15/32})^{\frac{71}{14} + \frac{3}{2 \log 2} - \frac{3}{2 \log 3}} 
      (\log\log\log x^{15/32}) \times \\ 
\notag 
      & \phantom{\succsim \Biggl\lvert\ } \times 
      \sqrt{\log\log\log\log x^{15/32}} 
      \int_{x^{15/32}}^{\sqrt{x}} \frac{dt}{t^3}  
      \Biggr\rvert 
\end{align} 
Now when we scale $M(x)$ by a reciprocal of $\sqrt{x}$ and let 
$x \rightarrow \infty$ along this infinite subsequence, we obtain that 
\begin{align*} 
\frac{|M(x)|}{\sqrt{x}} & \succsim 
     \frac{48 C_{\ell,1} x^{1/4}}{17} (\log x^{15/32})^{\frac{3}{2} 
      \log_{\ast}^4(x^{15/32}) - \frac{10}{7}} 
      (\log\log x^{15/32})^{\frac{71}{14} + \frac{3}{2 \log 2} - \frac{3}{2 \log 3}} 
      (\log\log\log x^{15/32}) \times \\ 
      & \phantom{\succsim\ } \times 
      \sqrt{\log\log\log\log x^{15/32}}  + o(1), 
\end{align*} 
where the constant power 
$\frac{71}{14} + \frac{3}{2\log 2} - \frac{3}{2\log 3} \approx 5.87011$. 
Notice that the above expression tends to $+\infty$ as $x \rightarrow \infty$, however, only 
extremly slowly and along a subsequence of asymptotically very large $x$. 
\end{proof} 

\subsection{Remarks} 

\begin{remark}[Tightness of the lower bounds] 
One remaining question for scaling $|M(x)| / f(x)$ is exactly how tight can the function 
$f \in \mathcal{F}$ be made so that 
(for $\mathcal{F}$ some reasonable function space with bases in polynomials of $x$ and 
powers of iterated logarithms) 
\[
f(x) := 
     \operatorname{argmax}\limits_{h \in \mathcal{F}} 
     \left\{\limsup_{x \rightarrow \infty} \frac{|M(x)|}{h(x)} = C_h \right\},  
\] 
for some absolute constants $C_h > 0$? 
What we have proved is that we can take 
\[
f(x) =  \sqrt{x} \cdot TODO, 
\]
to obtain the above where the limiting constant is 
$$C_f \mapsto \frac{9}{17} \sqrt{\frac{3}{2}} \frac{A_0^2 (2e+3)^2}{\pi e B^2 (\log 2)^3}.$$ 
But is this the tightest possible $f$ provably? 
%(\textit{Note that Gonek's famous conjecture states that we ought have 
%$f(x) := \sqrt{x} \cdot (\log\log x)^{5/4}$. 
%We apparently are able to asymptotically win here by a logarithmic landslide 
%over this estimate, but at a cost of our witness infinite subsequence of positive 
%reals being gigantic in asymptotic order, and hence numerically infeasible to 
%compute $M(x)$ along with modern methods $\dSmiley$.}) 

There is also, of course, 
the possibility of tightening the bound from above using the upper bounds proved in 
Theorem \ref{theorem_GFs_SymmFuncs_SumsOfRecipOfPowsOfPrimes} 
along an infinite subsequence tending to infinity. 
We do not approach this problem here 
due to length constraints and that our lower bounds seem to have done much better than was 
previously known before about the (un)boundedness of the scaled Mertens function -- our initial 
so-called ``pipe dream'', or ''impossible'', result.  
\end{remark} 

\begin{remark}[Computational limitations on numerically verifying the new lower bounds]
\end{remark} 

\newpage 
\section{Conclusions (TODO)} 

\newpage 
\renewcommand{\refname}{References} 
\bibliography{glossaries-bibtex/thesis-references}{}
\bibliographystyle{plain}

\newpage
\renewcommand{\thesubsection}{T.\arabic{subsection}}
\subsection{Table: Computations with a highly signed Dirichlet inverse function} 
\label{table_conjecture_Mertens_ginvSeq_approx_values}

\begin{table}[h!]

\centering

\tiny
\begin{equation*}
\boxed{
\begin{array}{|cc|c|ccc|c|c|ccc|c|ccc}
 n & \mathbf{Primes} & & \mathbf{Sqfree} & \mathbf{PPower} & \bar{\mathbb{S}} & & g^{-1}(n) & 
 \lambda(n) \operatorname{sgn}(g^{-1}(n)) & \lambda(n) g^{-1}(n) - \widehat{f}_1(n) & 
 \lambda(n) g^{-1}(n) - \widehat{f}_2(n) & & G^{-1}(n) & G^{-1}_{+}(n) & G^{-1}_{-}(n) \\ \hline 
 1 & 1^1 & \text{--} & \text{Y} & \text{N} & \text{N} & \text{--} & 1 & 1 & 0 & 0 & \text{--} & 1 & 1 & 0 \\
 2 & 2^1 & \text{--} & \text{Y} & \text{Y} & \text{N} & \text{--} & -2 & 1 & 0 & 0 & \text{--} & -1 & 1 & -2 \\
 3 & 3^1 & \text{--} & \text{Y} & \text{Y} & \text{N} & \text{--} & -2 & 1 & 0 & 0 & \text{--} & -3 & 1 & -4 \\
 4 & 2^2 & \text{--} & \text{N} & \text{Y} & \text{N} & \text{--} & 2 & 1 & 0 & -1 & \text{--} & -1 & 3 & -4 \\
 5 & 5^1 & \text{--} & \text{Y} & \text{Y} & \text{N} & \text{--} & -2 & 1 & 0 & 0 & \text{--} & -3 & 3 & -6 \\
 6 & 2^1 3^1 & \text{--} & \text{Y} & \text{N} & \text{N} & \text{--} & 5 & 1 & 0 & -1 & \text{--} & 2 & 8 & -6 \\
 7 & 7^1 & \text{--} & \text{Y} & \text{Y} & \text{N} & \text{--} & -2 & 1 & 0 & 0 & \text{--} & 0 & 8 & -8 \\
 8 & 2^3 & \text{--} & \text{N} & \text{Y} & \text{N} & \text{--} & -2 & 1 & 0 & -2 & \text{--} & -2 & 8 & -10 \\
 9 & 3^2 & \text{--} & \text{N} & \text{Y} & \text{N} & \text{--} & 2 & 1 & 0 & -1 & \text{--} & 0 & 10 & -10 \\
 10 & 2^1 5^1 & \text{--} & \text{Y} & \text{N} & \text{N} & \text{--} & 5 & 1 & 0 & -1 & \text{--} & 5 & 15 & -10 \\
 11 & 11^1 & \text{--} & \text{Y} & \text{Y} & \text{N} & \text{--} & -2 & 1 & 0 & 0 & \text{--} & 3 & 15 & -12 \\
 12 & 2^2 3^1 & \text{--} & \text{N} & \text{N} & \text{Y} & \text{--} & -7 & 1 & 2 & -2 & \text{--} & -4 & 15 & -19 \\
 13 & 13^1 & \text{--} & \text{Y} & \text{Y} & \text{N} & \text{--} & -2 & 1 & 0 & 0 & \text{--} & -6 & 15 & -21 \\
 14 & 2^1 7^1 & \text{--} & \text{Y} & \text{N} & \text{N} & \text{--} & 5 & 1 & 0 & -1 & \text{--} & -1 & 20 & -21 \\
 15 & 3^1 5^1 & \text{--} & \text{Y} & \text{N} & \text{N} & \text{--} & 5 & 1 & 0 & -1 & \text{--} & 4 & 25 & -21 \\
 16 & 2^4 & \text{--} & \text{N} & \text{Y} & \text{N} & \text{--} & 2 & 1 & 0 & -3 & \text{--} & 6 & 27 & -21 \\
 17 & 17^1 & \text{--} & \text{Y} & \text{Y} & \text{N} & \text{--} & -2 & 1 & 0 & 0 & \text{--} & 4 & 27 & -23 \\
 18 & 2^1 3^2 & \text{--} & \text{N} & \text{N} & \text{Y} & \text{--} & -7 & 1 & 2 & -2 & \text{--} & -3 & 27 & -30 \\
 19 & 19^1 & \text{--} & \text{Y} & \text{Y} & \text{N} & \text{--} & -2 & 1 & 0 & 0 & \text{--} & -5 & 27 & -32 \\
 20 & 2^2 5^1 & \text{--} & \text{N} & \text{N} & \text{Y} & \text{--} & -7 & 1 & 2 & -2 & \text{--} & -12 & 27 & -39 \\
 21 & 3^1 7^1 & \text{--} & \text{Y} & \text{N} & \text{N} & \text{--} & 5 & 1 & 0 & -1 & \text{--} & -7 & 32 & -39 \\
 22 & 2^1 11^1 & \text{--} & \text{Y} & \text{N} & \text{N} & \text{--} & 5 & 1 & 0 & -1 & \text{--} & -2 & 37 & -39 \\
 23 & 23^1 & \text{--} & \text{Y} & \text{Y} & \text{N} & \text{--} & -2 & 1 & 0 & 0 & \text{--} & -4 & 37 & -41 \\
 24 & 2^3 3^1 & \text{--} & \text{N} & \text{N} & \text{Y} & \text{--} & 9 & 1 & 4 & -3 & \text{--} & 5 & 46 & -41 \\
 25 & 5^2 & \text{--} & \text{N} & \text{Y} & \text{N} & \text{--} & 2 & 1 & 0 & -1 & \text{--} & 7 & 48 & -41 \\
 26 & 2^1 13^1 & \text{--} & \text{Y} & \text{N} & \text{N} & \text{--} & 5 & 1 & 0 & -1 & \text{--} & 12 & 53 & -41 \\
 27 & 3^3 & \text{--} & \text{N} & \text{Y} & \text{N} & \text{--} & -2 & 1 & 0 & -2 & \text{--} & 10 & 53 & -43 \\
 28 & 2^2 7^1 & \text{--} & \text{N} & \text{N} & \text{Y} & \text{--} & -7 & 1 & 2 & -2 & \text{--} & 3 & 53 & -50 \\
 29 & 29^1 & \text{--} & \text{Y} & \text{Y} & \text{N} & \text{--} & -2 & 1 & 0 & 0 & \text{--} & 1 & 53 & -52 \\
 30 & 2^1 3^1 5^1 & \text{--} & \text{Y} & \text{N} & \text{N} & \text{--} & -16 & 1 & 0 & -4 & \text{--} & -15 & 53 & -68 \\
 31 & 31^1 & \text{--} & \text{Y} & \text{Y} & \text{N} & \text{--} & -2 & 1 & 0 & 0 & \text{--} & -17 & 53 & -70 \\
 32 & 2^5 & \text{--} & \text{N} & \text{Y} & \text{N} & \text{--} & -2 & 1 & 0 & -4 & \text{--} & -19 & 53 & -72 \\
 33 & 3^1 11^1 & \text{--} & \text{Y} & \text{N} & \text{N} & \text{--} & 5 & 1 & 0 & -1 & \text{--} & -14 & 58 & -72 \\
 34 & 2^1 17^1 & \text{--} & \text{Y} & \text{N} & \text{N} & \text{--} & 5 & 1 & 0 & -1 & \text{--} & -9 & 63 & -72 \\
 35 & 5^1 7^1 & \text{--} & \text{Y} & \text{N} & \text{N} & \text{--} & 5 & 1 & 0 & -1 & \text{--} & -4 & 68 & -72 \\
 36 & 2^2 3^2 & \text{--} & \text{N} & \text{N} & \text{Y} & \text{--} & 14 & 1 & 9 & 1 & \text{--} & 10 & 82 & -72 \\
 37 & 37^1 & \text{--} & \text{Y} & \text{Y} & \text{N} & \text{--} & -2 & 1 & 0 & 0 & \text{--} & 8 & 82 & -74 \\
 38 & 2^1 19^1 & \text{--} & \text{Y} & \text{N} & \text{N} & \text{--} & 5 & 1 & 0 & -1 & \text{--} & 13 & 87 & -74 \\
 39 & 3^1 13^1 & \text{--} & \text{Y} & \text{N} & \text{N} & \text{--} & 5 & 1 & 0 & -1 & \text{--} & 18 & 92 & -74 \\
 40 & 2^3 5^1 & \text{--} & \text{N} & \text{N} & \text{Y} & \text{--} & 9 & 1 & 4 & -3 & \text{--} & 27 & 101 & -74 \\
 41 & 41^1 & \text{--} & \text{Y} & \text{Y} & \text{N} & \text{--} & -2 & 1 & 0 & 0 & \text{--} & 25 & 101 & -76 \\
 42 & 2^1 3^1 7^1 & \text{--} & \text{Y} & \text{N} & \text{N} & \text{--} & -16 & 1 & 0 & -4 & \text{--} & 9 & 101 & -92 \\
 43 & 43^1 & \text{--} & \text{Y} & \text{Y} & \text{N} & \text{--} & -2 & 1 & 0 & 0 & \text{--} & 7 & 101 & -94 \\
 44 & 2^2 11^1 & \text{--} & \text{N} & \text{N} & \text{Y} & \text{--} & -7 & 1 & 2 & -2 & \text{--} & 0 & 101 & -101 \\
 45 & 3^2 5^1 & \text{--} & \text{N} & \text{N} & \text{Y} & \text{--} & -7 & 1 & 2 & -2 & \text{--} & -7 & 101 & -108 \\
 46 & 2^1 23^1 & \text{--} & \text{Y} & \text{N} & \text{N} & \text{--} & 5 & 1 & 0 & -1 & \text{--} & -2 & 106 & -108 \\
 47 & 47^1 & \text{--} & \text{Y} & \text{Y} & \text{N} & \text{--} & -2 & 1 & 0 & 0 & \text{--} & -4 & 106 & -110 \\
 48 & 2^4 3^1 & \text{--} & \text{N} & \text{N} & \text{Y} & \text{--} & -11 & 1 & 6 & -4 & \text{--} & -15 & 106 & -121 \\
% 49 & 7^2 & \text{--} & \text{N} & \text{Y} & \text{N} & \text{--} & 2 & 1 & 0 & -1 & \text{--} & -13 & 108 & -121 \\
% 50 & 2^1 5^2 & \text{--} & \text{N} & \text{N} & \text{Y} & \text{--} & -7 & 1 & 2 & -2 & \text{--} & -20 & 108 & -128 \\
% 51 & 3^1 17^1 & \text{--} & \text{Y} & \text{N} & \text{N} & \text{--} & 5 & 1 & 0 & -1 & \text{--} & -15 & 113 & -128 \\
% 52 & 2^2 13^1 & \text{--} & \text{N} & \text{N} & \text{Y} & \text{--} & -7 & 1 & 2 & -2 & \text{--} & -22 & 113 & -135 \\
% 53 & 53^1 & \text{--} & \text{Y} & \text{Y} & \text{N} & \text{--} & -2 & 1 & 0 & 0 & \text{--} & -24 & 113 & -137 \\
% 54 & 2^1 3^3 & \text{--} & \text{N} & \text{N} & \text{Y} & \text{--} & 9 & 1 & 4 & -3 & \text{--} & -15 & 122 & -137 \\
% 55 & 5^1 11^1 & \text{--} & \text{Y} & \text{N} & \text{N} & \text{--} & 5 & 1 & 0 & -1 & \text{--} & -10 & 127 & -137 \\
% 56 & 2^3 7^1 & \text{--} & \text{N} & \text{N} & \text{Y} & \text{--} & 9 & 1 & 4 & -3 & \text{--} & -1 & 136 & -137 \\
% 57 & 3^1 19^1 & \text{--} & \text{Y} & \text{N} & \text{N} & \text{--} & 5 & 1 & 0 & -1 & \text{--} & 4 & 141 & -137 \\
% 58 & 2^1 29^1 & \text{--} & \text{Y} & \text{N} & \text{N} & \text{--} & 5 & 1 & 0 & -1 & \text{--} & 9 & 146 & -137 \\
% 59 & 59^1 & \text{--} & \text{Y} & \text{Y} & \text{N} & \text{--} & -2 & 1 & 0 & 0 & \text{--} & 7 & 146 & -139 \\
% 60 & 2^2 3^1 5^1 & \text{--} & \text{N} & \text{N} & \text{Y} & \text{--} & 30 & 1 & 14 & 0 & \text{--} & 37 & 176 & -139 \\
% 61 & 61^1 & \text{--} & \text{Y} & \text{Y} & \text{N} & \text{--} & -2 & 1 & 0 & 0 & \text{--} & 35 & 176 & -141 \\
% 62 & 2^1 31^1 & \text{--} & \text{Y} & \text{N} & \text{N} & \text{--} & 5 & 1 & 0 & -1 & \text{--} & 40 & 181 & -141 \\
% 63 & 3^2 7^1 & \text{--} & \text{N} & \text{N} & \text{Y} & \text{--} & -7 & 1 & 2 & -2 & \text{--} & 33 & 181 & -148 \\
% 64 & 2^6 & \text{--} & \text{N} & \text{Y} & \text{N} & \text{--} & 2 & 1 & 0 & -5 & \text{--} & 35 & 183 & -148 \\
% 65 & 5^1 13^1 & \text{--} & \text{Y} & \text{N} & \text{N} & \text{--} & 5 & 1 & 0 & -1 & \text{--} & 40 & 188 & -148 \\
% 66 & 2^1 3^1 11^1 & \text{--} & \text{Y} & \text{N} & \text{N} & \text{--} & -16 & 1 & 0 & -4 & \text{--} & 24 & 188 & -164 \\
% 67 & 67^1 & \text{--} & \text{Y} & \text{Y} & \text{N} & \text{--} & -2 & 1 & 0 & 0 & \text{--} & 22 & 188 & -166 \\
% 68 & 2^2 17^1 & \text{--} & \text{N} & \text{N} & \text{Y} & \text{--} & -7 & 1 & 2 & -2 & \text{--} & 15 & 188 & -173 \\
% 69 & 3^1 23^1 & \text{--} & \text{Y} & \text{N} & \text{N} & \text{--} & 5 & 1 & 0 & -1 & \text{--} & 20 & 193 & -173 \\
% 70 & 2^1 5^1 7^1 & \text{--} & \text{Y} & \text{N} & \text{N} & \text{--} & -16 & 1 & 0 & -4 & \text{--} & 4 & 193 & -189 \\
% 71 & 71^1 & \text{--} & \text{Y} & \text{Y} & \text{N} & \text{--} & -2 & 1 & 0 & 0 & \text{--} & 2 & 193 & -191 \\
% 72 & 2^3 3^2 & \text{--} & \text{N} & \text{N} & \text{Y} & \text{--} & -23 & 1 & 18 & 6 & \text{--} & -21 & 193 & -214 \\
\end{array}
}
\end{equation*}

\bigskip\hrule\smallskip 

\caption*{\textbf{\rm \bf Table \thesubsection:} 
          \textbf{Computations of the first several cases of $\mathbf{g^{-1}(n) \equiv (\omega+1)^{-1}(n)}$ 
          for $\mathbf{1 \leq n \leq 56}$.} \\ 
          The column labeled \texttt{Primes} provides the prime factorization of each $n$ so that the values of 
          $\omega(n)$ and $\Omega(n)$ are easily extracted. The columns labeled, respectively, \texttt{Sqfree}, \texttt{PPower} and 
          $\bar{\mathbb{S}}$ list inclusion of $n$ in the sets of squarefree integers, prime powers, and the set $\bar{\mathbb{S}}$ 
          that denotes the positive integers $n$ which are neither squarefree nor prime powers. The next two columns provide the 
          explicit values of the inverse function $g^{-1}(n)$ and indicate that the sign of this function at $n$ is given by 
          $\lambda(n) = (-1)^{\Omega(n)}$. \\[0.15cm] 
          Then the next two columns show the small-ish magnitude differences between the unsigned 
          magnitude of $g^{-1}(n)$ and the summations $\widehat{f}_1(n) := \sum_{k \geq 0} \binom{\omega(n)}{k} \cdot k!$ and 
          $\widehat{f}_2(n) := \sum_{k \geq 0} \binom{\omega(n)}{k} \cdot \#\{d|n: \omega(d) = k\}$. Finally, the last three 
          columns show the summatory function of $g^{-1}(n)$, $G^{-1}(x) := \sum_{n \leq x} g^{-1}(n)$, deconvolved into its 
          respective positive and negative components: $G^{-1}_{+}(x) := \sum_{n \leq x} g^{-1}(n) \Iverson{g^{-1}(n) > 0}$ and 
          $G^{-1}_{-}(x) := \sum_{n \leq x} g^{-1}(n) \Iverson{g^{-1}(n) < 0}$. } 

\end{table}

\newpage 
\subsection{Table: Dirichlet inverse functions of $(f+1)(n)$ for $f$ additive} 
\label{table_DirInvFuncExps_fp1_fAdditive}

\begin{table}[h!]

\centering

\tiny
\begin{equation*}
\boxed{
\begin{array}{|c|c|l|} \hline 
n & \lambda(n) & (f+1)^{-1}(n) \\ \hline 
 1 & 1 & 1 \\
 2 & -1 & -f(2)-1 \\
 3 & -1 & -f(3)-1 \\
 4 & 1 & f(2)^2+2 f(2)-f(4) \\
 5 & -1 & -f(5)-1 \\
 6 & 1 & 2 f(3) f(2)+f(2)+f(3)+1 \\
 7 & -1 & -f(7)-1 \\
 8 & -1 & -f(2)^3-3 f(2)^2+2 f(4) f(2)-f(2)+2 f(4)-f(8) \\
 9 & 1 & f(3)^2+2 f(3)-f(9) \\
 10 & 1 & 2 f(5) f(2)+f(2)+f(5)+1 \\
 11 & -1 & -f(11)-1 \\
 12 & -1 & -3 f(3) f(2)^2-f(2)^2-4 f(3) f(2)-2 f(2)+2 f(3) f(4)+f(4) \\
 13 & -1 & -f(13)-1 \\
 14 & 1 & 2 f(7) f(2)+f(2)+f(7)+1 \\
 15 & 1 & 2 f(5) f(3)+f(3)+f(5)+1 \\
 16 & 1 & f(2)^4+4 f(2)^3-3 f(4) f(2)^2+3 f(2)^2-6 f(4) f(2)+2 f(8) f(2)+f(4)^2-f(4)+2 f(8)-f(16) \\
 17 & -1 & -f(17)-1 \\
 18 & -1 & -3 f(2) f(3)^2-f(3)^2-4 f(2) f(3)-2 f(3)+2 f(2) f(9)+f(9) \\
 19 & -1 & -f(19)-1 \\
 20 & -1 & -3 f(5) f(2)^2-f(2)^2-4 f(5) f(2)-2 f(2)+f(4)+2 f(4) f(5) \\
 21 & 1 & 2 f(7) f(3)+f(3)+f(7)+1 \\
 22 & 1 & 2 f(11) f(2)+f(2)+f(11)+1 \\
 23 & -1 & -f(23)-1 \\
 24 & 1 & 4 f(3) f(2)^3+f(2)^3+9 f(3) f(2)^2+3 f(2)^2+2 f(3) f(2)-6 f(3) f(4) f(2)-2 f(4) f(2)+f(2)-4 f(3) f(4)-2 f(4)+2 f(3) f(8)+f(8) \\
 25 & 1 & f(5)^2+2 f(5)-f(25) \\
 26 & 1 & 2 f(13) f(2)+f(2)+f(13)+1 \\
 27 & -1 & -f(3)^3-3 f(3)^2+2 f(9) f(3)-f(3)+2 f(9)-f(27) \\
 28 & -1 & -3 f(7) f(2)^2-f(2)^2-4 f(7) f(2)-2 f(2)+f(4)+2 f(4) f(7) \\
 29 & -1 & -f(29)-1 \\
 30 & -1 & -2 f(3) f(2)-6 f(3) f(5) f(2)-2 f(5) f(2)-f(2)-f(3)-2 f(3) f(5)-f(5)-1 \\
 31 & -1 & -f(31)-1 \\ 
   \hline 
\end{array}
}
\end{equation*} 

\bigskip\hrule\smallskip 

\caption*{\textbf{\rm \bf Table \thesubsection:} 
          \textbf{Dirichlet inverse functions of additive arithmetic functions.} 
          The table provides a list of the Dirichlet inverse functions of $(f+1)(n)$ for 
          $f$ additive such that $f(1) = 0$. } 

\end{table}


\newpage
\setcounter{section}{0}
\renewcommand{\thesection}{Appendix \Alph{section}}
\renewcommand{\thesubsection}{\Alph{section}.\arabic{subsection}}

\end{document}
